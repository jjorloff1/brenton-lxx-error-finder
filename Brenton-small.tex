\input{preamble.tex}
% Override color setting
\definecolor{bookheadingcolor}{HTML}{000000} % 000000 - Pure black for B&W printing

\title{Η ΠΑΛΑΙΑ ΔΙΑΘΗΚΗ}
\author{Ἡ μετάφρασις τῶν Ἑβδομήκοντα}
\date{}

\begin{document}
\begin{spacing}{1.1}
\maketitle

\cleardoublepage
\begin{titlepage}
  \begin{center}
    \textcolor{bookheadingcolor}{\Huge Preface}\par
  \end{center}
  \vspace{2em}
  
  This project was undertaken in love and respect for the Bible, with a desire to have an accessible and
  beautiful Greek Bible available in print to anyone who would like one. While there are many great Greek
  New Testaments available in print, the Septuagint has been less accessible, particularly in a format that
  is both compact and minimalist. Most of the Septuagints available in print are quite large. Additionally,
  there are almost no complete Greek Bibles available to purchase for a reasonable price. I have undertaken
  this project for those out there who, like me, want to take a physical Greek Bible along 
  with them where ever they want to go.

  When I started this project, I went hunting for open and public domain editions of the Septuagint and the NT
  that I could use as the texts for this Bible. While there are several great options out there, I settled on
  the Brenton Septuagint and the OpenGNT new testament. The reason for choosing Brenton's Septuagint was simple:
  I found a great open source project that had already digitized the text and prepared it for print: 
  https://github.com/mrgreekgeek/Brenton-LXX-Latex-print-project/. Starting with this baseline, I was able to
  style the text in a way that I liked. Then I had to find and prepare a NT Text.

  For the NT I chose the Open GNT (https://opengnt.com/) which was prepared by Eliran Wong and released under
  the Creative Commons Attribution 4.0 International License (CC BY 4.0). This project was created "to offer a 
  FREE NA-equivalent text of Greek New Testament, compiled from open-resources" and provided access to the text
  in a format that I could adapt to my needs.

  As for formatting, I was inspired by some of the beautiful minimalist reader Bibles available in English. As
  much as possible, I wanted to keep the text front and center, eliminating distractions and unnecessary elements.
  I have tried to mitigate the distraction from things like section headings, spacing between chapters, and even
  chapter numbers to some degree. I ultimately decided to leave verse numbers in place, because I think navigating
  the Old Testament may have been more difficult without them; however, I tried to minimize their visual impact.
  My goal is to facilitate a novel-like reading experience, free of distractions.

  The source code that I have used to extract, process, and format the texts used for this Bible is available 
  free of charge at https://github.com/jjorloff1/lxx-nt-greek-bible-builder.

  I hope that this Greek Bible will serve you well as you study and meditate on the Scriptures.
  Glory to God!

  \vfill
  \begin{flushright}
    {\large\textit{Jesse Orloff}\par}
    {\large www.jesseorloff.com\par}
    {\large August 2025\par}
  \end{flushright}  
\end{titlepage}

\cleardoublepage
\pagestyle{empty}
\begingroup
\centering
{\huge \textcolor{bookheadingcolor}{Table of Contents} \par}
\endgroup

\begin{multicols}{2}
\makeatletter
\renewcommand{\tableofcontents}{\@starttoc{toc}}
\makeatother
\tableofcontents
\end{multicols}
\pagestyle{fancy}


\def\book{ΓΕΝΕΣΙΣ}
\biblebook{ΓΕΝΕΣΙΣ}


\lettrine[lines=2, loversize=0.2, nindent=0em, findent=.25em]{\textcolor{bookheadingcolor}{Ἐ}}{Ν} ἀρχῇ ἐποίησεν ὁ Θεὸς τὸν οὐρανὸν καὶ τὴν γῆν.
\vs{2}Ἡ δὲ γῆ ἦν ἀόρατος καὶ ἀκατασκεύαστος, καὶ σκότος ἐπάνω τῆς ἀβύσσου· καὶ πνεῦμα Θεοῦ ἐπεφέρετο ἐπάνω τοῦ ὕδατος.
\vs{3}Καὶ εἶπεν ὁ Θεὸς, γενηθήτω φῶς· καὶ ἐγένετο φῶς.
\vs{4}Καὶ εἶδεν ὁ Θεὸς τὸ φῶς, ὅτι καλόν· καὶ διεχώρισεν ὁ Θεὸς ἀνὰ μέσον τοῦ φωτὸς, καὶ ἀνὰ μέσον τοῦ σκότους.
\vs{5}Καὶ ἐκάλεσεν ὁ Θεὸς τὸ φῶς ἡμέραν, καὶ τὸ σκότος ἐκάλεσε νύκτα. Καὶ ἐγένετο ἑσπέρα, καὶ ἐγένετο πρωῒ, ἡμέρα μία.

\vs{6}Καὶ εἶπεν ὁ Θεὸς, γενηθήτω στερέωμα ἐν μέσῳ τοῦ ὕδατος· καὶ ἔστω διαχωρίζον ἀνὰ μέσον ὕδατος καὶ ὕδατος· καὶ ἐγένετο οὕτως.
\vs{7}Καὶ ἐποίησεν ὁ Θεὸς τὸ στερέωμα· καὶ διεχώρισεν ὁ Θεὸς ἀνὰ μέσον τοῦ ὕδατος, ὃ ἦν ὑποκάτω τοῦ στερεώματος, καὶ ἀνὰ μέσον τοῦ ὕδατος, τοῦ ἐπάνω τοῦ στερεώματος.
\vs{8}Καὶ ἐκάλεσεν ὁ Θεὸς τὸ στερέωμα οὐρανόν· καὶ εἶδεν ὁ Θεὸς ὅτι καλόν· καὶ ἐγένετο ἑσπέρα, καὶ ἐγένετο πρωῒ, ἡμέρα δευτέρα.

\vs{9}Καὶ εἶπεν ὁ Θεὸς, συναχθήτω τὸ ὕδωρ τὸ ὑποκάτω τοῦ οὐρανοῦ εἰς συναγωγὴν μίαν, καὶ ὀφθήτω ἡ ξηρά· καὶ ἐγένετο οὕτως· καὶ συνήχθη τὸ ὕδωρ τὸ ὑποκάτω τοῦ οὐρανοῦ εἰς τὰς συναγωγὰς αὐτῶν, καὶ ὤφθη ἡ ξηρά.
\vs{10}Καὶ ἐκάλεσεν ὁ Θεὸς τὴν ξηρὰν, γῆν· καὶ τὰ συστήματα τῶν ὑδάτων ἐκάλεσε θαλάσσας· καὶ εἶδεν ὁ Θεὸς ὅτι καλόν.
\vs{11}Καὶ εἶπεν ὁ Θεὸς, βλαστησάτω ἡ γῆ βοτάνην χόρτου, σπεῖρον σπέρμα κατὰ γένος καὶ καθʼ ὁμοιότητα, καὶ ξύλον κάρπιμον ποιοῦν καρπὸν, οὗ τὸ σπέρμα αὐτοῦ ἐν αὐτῷ κατὰ γένος ἐπὶ τῆς γῆς· καὶ ἐγένετο οὕτως.
\vs{12}Καὶ ἐξήνεγκεν ἡ γῆ βοτάνην χόρτου, σπεῖρον σπέρμα κατὰ γένος καὶ καθʼ ὁμοιότητα, καὶ ξύλον κάρπιμον ποιοῦν καρπὸν, οὗ τὸ σπέρμα αὐτοῦ ἐν αὐτῷ κατὰ γένος ἐπὶ τῆς γῆς· καὶ εἶδεν ὁ Θεὸς ὅτι καλόν.
\vs{13}Καὶ ἐγένετο ἑσπέρα, καὶ ἐγένετο πρωῒ, ἡμέρα τρίτη.

\vs{14}Καὶ εἶπεν ὁ Θεὸς, γενηθήτωσαν φωστῆρες ἐν τῷ στερεώματι τοῦ οὐρανοῦ εἰς φαῦσιν ἐπὶ τῆς γῆς, τοῦ διαχωρίζειν ἀνὰ μέσον τῆς ἡμέρας καὶ ἀνὰ μέσον τῆς νυκτός· καὶ ἔστωσαν εἰς σημεῖα, καὶ εἰς καιροὺς, καὶ εἰς ἡμέρας, καὶ εἰς ἐνιαυτούς.
\vs{15}Καὶ ἔστωσαν εἰς φαῦσιν ἐν τῷ στερεώματι τοῦ οὐρανοῦ, ὥστε φαίνειν ἐπὶ τῆς γῆς· καὶ ἐγένετο οὕτως.
\vs{16}Καὶ ἐποίησεν ὁ Θεὸς τοὺς δύο φωστῆρας τοὺς μεγάλους· τὸν φωστῆρα τὸν μέγαν εἰς ἀρχὰς τῆς ἡμέρας, καὶ τὸν φωστῆρα τὸν ἐλάσσω εἰς ἀρχὰς τῆς νυκτὸς, καὶ τοὺς ἀστέρας.
\vs{17}Καὶ ἔθετο αὐτοὺς ὁ Θεὸς ἐν τῷ στερεώματι τοῦ οὐρανοῦ, ὥστε φαίνειν ἐπὶ τῆς γῆς,
\vs{18}καὶ ἄρχειν τῆς ἡμέρας καὶ τῆς νυκτὸς, καὶ διαχωρίζειν ἀνὰ μέσον τοῦ φωτὸς, καὶ ἀνὰ μέσον τοῦ σκότους· καὶ εἶδεν ὁ Θεὸς ὅτι καλόν.
\vs{19}Καὶ ἐγένετο ἑσπέρα καὶ ἐγένετο πρωῒ, ἡμέρα τετάρτη.

\vs{20}Καὶ εἶπεν ὁ Θεὸς, ἐξαγαγέτω τὰ ὕδατα ἑρπετὰ ψυχῶν ζωσῶν, καὶ πετεινὰ πετόμενα ἐπὶ τῆς γῆς κατὰ τὸ στερέωμα τοῦ οὐρανοῦ· καὶ ἐγένετο οὕτως.
\vs{21}Καὶ ἐποίησεν ὁ Θεὸς τὰ κήτη τὰ μεγάλα, καὶ πᾶσαν ψυχὴν ζώων ἑρπετῶν, ἃ ἐξήγαγε τὰ ὕδατα κατὰ γένη αὐτῶν, καὶ πᾶν πετεινὸν πτερωτὸν κατὰ γένος· καὶ εἶδεν ὁ Θεὸς ὅτι καλά.
\vs{22}Καὶ εὐλόγησεν αὐτὰ ὁ Θεὸς, λέγων, αὐξάνεσθε καὶ πληθύνεσθε, καὶ πληρώσατε τὰ ὕδατα ἐν ταῖς θαλάσσαις, καὶ τὰ πετεινὰ πληθυνέσθωσαν ἐπὶ τῆς γῆς.
\vs{23}Καὶ ἐγένετο ἑσπέρα, καὶ ἐγένετο πρωῒ, ἡμέρα πέμπτη.

\vs{24}Καὶ εἶπεν ὁ Θεὸς, ἐξαγαγέτω ἡ γῆ ψυχὴν ζῶσαν κατὰ γένος, τετράποδα, καὶ ἑρπετὰ, καὶ θηρία τῆς γῆς κατὰ γένος· καὶ ἐγένετο οὕτως.
\vs{25}Καὶ ἐποίησεν ὁ Θεὸς τὰ θηρία τῆς γῆς κατὰ γένος, καὶ τὰ κτήνη κατὰ γένος αὐτῶν, καὶ πάντα τὰ ἑρπετὰ τῆς γῆς κατὰ γένος· καὶ εἶδεν ὁ Θεὸς ὅτι καλά.

\vs{26}Καὶ εἶπεν ὁ Θεός, Ποιήσωμεν ἄνθρωπον κατʼ εἰκόνα ἡμετέραν καὶ καθʼ ὁμοίωσιν· καὶ ἀρχέτωσαν τῶν ἰχθύων τῆς θαλάσσης, καὶ τῶν πετεινῶν τοῦ οὐρανοῦ, καὶ τῶν κτηνῶν, καὶ πάσης τῆς γῆς, καὶ πάντων τῶν ἑρπετῶν τῶν ἑρπόντων ἐπὶ τῆς γῆς.
\vs{27}Καὶ ἐποιήσεν ὁ Θεὸς τὸν ἄνθρωπον· κατʼ εἰκόνα Θεοῦ ἐποίησεν αὐτόν· ἄρσεν καὶ θῆλυ ἐποίησεν αὐτούς.
\vs{28}Καὶ εὐλόγησεν αὐτοὺς ὁ Θεὸς, λέγων, αὐξάνεσθε καὶ πληθύνεσθε, καὶ πληρώσατε τὴν γῆν, καὶ κατακυριεύσατε αὐτῆς· καὶ ἄρχετε τῶν ἰχθύων τῆς θαλάσσης, καὶ τῶν πετεινῶν τοῦ οὐρανοῦ, καὶ πάντων τῶν κτηνῶν, καὶ πάσης τῆς γῆς, καὶ πάντων τῶν ἑρπετῶν τῶν ἑρπόντων ἐπὶ τῆς γῆς.
\vs{29}Καὶ εἶπεν ὁ Θεός, Ἰδοὺ δέδωκα ὑμῖν πάντα χόρτον σπόριμον σπεῖρον σπέρμα, ὅ ἐστιν ἐπάνω πάσης τῆς γῆς· καὶ πᾶν ξύλον, ὃ ἔχει ἐν ἑαυτῷ καρπὸν σπέρματος σπορίμου, ὑμῖν ἔσται εἰς βρῶσιν,
\vs{30}καὶ πᾶσι τοῖς θηρίοις τῆς γῆς, καὶ πᾶσι τοῖς πετεινοῖς τοῦ οὐρανοῦ, καὶ παντὶ ἑρπετῷ ἕρποντι ἐπὶ τῆς γῆς, ὃ ἔχει ἐν ἑαυτῷ ψυχὴν ζωῆς, καὶ πάντα χόρτον χλωρὸν εἰς βρῶσιν· καὶ ἐγένετο οὕτως.
\vs{31}Καὶ εἶδεν ὁ Θεὸς τὰ πάντα, ὅσα ἐποίησε, καὶ ἰδοὺ καλὰ λίαν· καὶ ἐγένετο ἑσπέρα, καὶ ἐγένετο πρωῒ, ἡμέρα ἕκτη.

\ch{2}Καὶ συνετελέσθησαν ὁ οὐρανὸς καὶ ἡ γῆ, καὶ πᾶς ὁ κόσμος αὐτῶν.

\vs{2}Καὶ συνετέλεσεν ὁ Θεὸς ἐν τῇ ἡμέρᾳ τῇ ἕκτῃ τὰ ἔργα αὐτοῦ, ἃ ἐποίησε· καὶ κατέπαυσε τῇ ἡμέρᾳ τῇ ἑβδόμῃ ἀπὸ πάντων τῶν ἔργων αὐτοῦ, ὧν ἐποίησε.
\vs{3}Καὶ εὐλόγησεν ὁ Θεὸς τὴν ἡμέραν τὴν ἑβδόμην, καὶ ἡγίασεν αὐτήν, ὅτι ἐν αὐτῇ κατέπαυσεν ἀπὸ πάντων τῶν ἔργων αὐτοῦ, ὧν ἤρξατο ὁ Θεὸς ποιῆσαι.

\vs{4}Αὕτη ἡ βίβλος γενέσεως οὐρανοῦ καὶ γῆς, ὅτε ἐγένετο, ᾗ ἡμέρᾳ ἐποίησε Κύριος ὁ Θεὸς τὸν οὐρανὸν καὶ τὴν γῆν,
\vs{5}καὶ πᾶν χλωρὸν ἀγροῦ πρὸ τοῦ γενέσθαι ἐπὶ τῆς γῆς, καὶ πάντα χόρτον ἀγροῦ πρὸ τοῦ ἀνατεῖλαι· οὐ γὰρ ἔβρεξεν ὁ Θεὸς ἐπὶ τὴν γῆν, καὶ ἄνθρωπος οὐκ ἦν ἐργάζεσθαι αὐτήν.
\vs{6}Πηγὴ δὲ ἀνέβαινεν ἐκ τῆς γῆς, καὶ ἐπότιζε πᾶν τὸ πρόσωπον τῆς γῆς.
\vs{7}Καὶ ἔπλασεν ὁ Θεὸς τὸν ἄνθρωπον, χοῦν ἀπὸ τῆς γῆς· καὶ ἐνεφύσησεν εἰς τὸ πρόσωπον αὐτοῦ πνοὴν ζωῆς, καὶ ἐγένετο ὁ ἄνθρωπος εἰς ψυχὴν ζῶσαν.

\vs{8}Καὶ ἐφύτευσεν ὁ Θεὸς παράδεισον ἐν Ἐδὲμ κατὰ ἀνατολάς· καὶ ἔθετο ἐκεῖ τὸν ἄνθρωπον, ὃν ἔπλασε.
\vs{9}Καὶ ἐξανέτειλεν ὁ Θεὸς ἔτι ἐκ τῆς γῆς πᾶν ξύλον ὡραῖον εἰς ὅρασιν, καὶ καλὸν εἰς βρῶσιν, καὶ τὸ ξύλον τῆς ζωῆς ἐν μέσῳ τοῦ παραδείσου, καὶ τὸ ξύλον τοῦ εἰδέναι γνωστὸν καλοῦ καὶ πονηροῦ.
\vs{10}Ποταμὸς δὲ ἐκπορεύεται ἐξ Ἐδὲμ ποτίζειν τὸν παράδεισον· ἐκεῖθεν ἀφορίζεται εἰς τέσσαρας ἀρχάς.
\vs{11}Ὄνομα τῷ ἑνὶ, Φισῶν· οὗτος ὁ κυκλῶν πᾶσαν τὴν γῆν Εὐιλάτ· ἐκεῖ οὗ ἐστι τὸ χρυσίον.
\vs{12}Τὸ δὲ χρυσίον τῆς γῆς ἐκείνης καλόν· καὶ ἐκεῖ ἐστιν ὁ ἄνθραξ, καὶ ὁ λίθος ὁ πράσινος.
\vs{13}Καὶ ὄνομα τῷ ποταμῷ τῷ δευτέρῳ, Γεῶν· οὗτος ὁ κυκλῶν πᾶσαν τὴν γὴν Αἰθιοπίας.
\vs{14}Καὶ ὁ ποταμὸς ὁ τρίτος, Τίγρις· οὗτος ὁ προπορευόμενος κατέναντι Ἀσσυρίων· ὁ δὲ ποταμὸς ὁ τέταρτος, Εὐφράτης.
\vs{15}Καὶ ἔλαβε Κύριος ὁ Θεὸς τὸν ἄνθρωπον ὃν ἔπλασε, καὶ ἔθετο αὐτὸν ἐν τῷ παραδείσῳ τῆς τρυφῆς, ἐργάζεσθαι αὐτὸν καὶ φυλάσσειν.
\vs{16}Καὶ ἐνετείλατο Κύριος ὁ Θεὸς τῷ Ἀδὰμ, λέγων, ἀπὸ παντὸς ξύλου τοῦ ἐν τῷ παραδείσῳ βρώσει φαγῇ.
\vs{17}Ἀπὸ δὲ τοῦ ξύλου τοῦ γινώσκειν καλὸν καὶ πονηρὸν, οὐ φάγεσθε ἀπʼ αὐτοῦ· ᾗ δʼ ἂν ἡμέρᾳ φάγητε ἀπʼ αὐτοῦ, θανάτῳ ἀποθανεῖσθε.

\vs{18}Καὶ εἶπε Κύριος ὁ Θεὸς, οὐ καλὸν εἶναι τὸν ἄνθρωπον μόνον· ποιήσωμεν αὐτῷ βοηθὸν κατʼ αὐτόν.
\vs{19}Καὶ ἔπλασεν ὁ Θεὸς ἔτι ἐκ τῆς γῆς πάντα τὰ θηρία τοῦ ἀγροῦ, καὶ πάντα τὰ πετεινὰ τοῦ οὐρανοῦ· καὶ ἤγαγεν αὐτὰ πρὸς τὸν Ἀδὰμ, ἰδεῖν τί καλέσει αὐτά· καὶ πᾶν ὃ ἐὰν ἐκάλεσεν αὐτὸ Ἀδὰμ ψυχὴν ζῶσαν, τοῦτο ὄνομα αὐτῷ.
\vs{20}Καὶ ἐκάλεσεν Ἀδὰμ ὀνόματα πᾶσι τοῖς κτήνεσι, καὶ πᾶσι τοῖς πετεινοῖς τοῦ οὐρανοῦ, καὶ πᾶσι τοῖς θηρίοις τοῦ ἀγροῦ· τῷ δὲ Ἀδὰμ οὐχ εὑρέθη βοηθὸς ὅμοιος αὐτῷ.
\vs{21}Καὶ ἐπέβαλεν ὁ Θεὸς ἔκστασιν ἐπὶ τὸν Ἀδὰμ, καὶ ὕπνωσε· καὶ ἔλαβε μίαν τῶν πλευρῶν αὐτοῦ, καὶ ἀνεπλήρωσε σάρκα ἀντʼ αὐτῆς.
\vs{22}Καὶ ᾠκοδόμησεν ὁ Θεὸς τὴν πλευρὰν, ἣν ἔλαβεν ἀπὸ τοῦ Ἀδὰμ εἰς γυναῖκα· καὶ ἤγαγεν αὐτὴν πρὸς τὸν Ἀδάμ.
\vs{23}Καὶ εἶπεν Ἀδάμ· τοῦτο νῦν ὀστοῦν ἐκ τῶν ὀστέων μου, καὶ σὰρξ ἐκ τῆς σαρκός μου· αὕτη κληθήσεται γυνὴ, ὅτι ἐκ τοῦ ἀνδρὸς αὐτῆς ἐλήφθη.
\vs{24}Ἕνεκεν τούτου καταλείψει ἄνθρωπος τὸν πατέρα αὐτοῦ καὶ τὴν μητέρα, καὶ προσκολληθήσεται πρὸς τὴν γυναῖκα αὐτοῦ· καὶ ἔσονται οἱ δύο εἰς σάρκα μίαν.
\vs{25}Καὶ ἦσαν οἱ δύο γυμνοὶ, ὅ, τε Ἀδὰμ καὶ ἡ γυνὴ αὐτοῦ, καὶ οὐκ ᾐσχύνοντο.

\ch{3}
Ὁ δὲ ὄφις ἦν φρονιμώτατος πάντων τῶν θηρίων τῶν ἐπὶ τῆς γῆς, ὧν ἐποίησε Κύριος ὁ Θεός· καὶ εἶπεν ὁ ὄφις τῇ γυναικὶ, τί ὅτι εἶπεν ὁ Θεός, οὐ μὴ φάγητε ἀπὸ παντὸς ξύλου τοῦ παραδείσου;
\vs{2}Καὶ εἶπεν ἡ γυνὴ τῷ ὄφει, ἀπὸ καρποῦ τοῦ ξύλου τοῦ παραδείσου φαγούμεθα·
\vs{3}Ἀπὸ δὲ τοῦ καρποῦ τοῦ ξύλου, ὅ ἐστιν ἐν μέσῳ τοῦ παραδείσου, εἶπεν ὁ Θεός, οὐ φάγεσθε ἀπʼ αὐτοῦ, οὐδὲ μὴ ἅψησθε αὐτοῦ, ἵνα μὴ ἀποθάνητε.
\vs{4}Καὶ εἶπεν ὁ ὄφις τῇ γυναικί· οὐ θανάτῳ ἀποθανεῖσθε·
\vs{5}Ἤδει γὰρ ὁ Θεὸς, ὅτι ᾗ ἂν ἡμέρᾳ φάγητε ἀπʼ αὐτοῦ, διανοιχθήσονται ὑμῶν οἱ ὀφθαλμοί, καὶ ἔσεσθε ὡς θεοί, γινώσκοντες καλὸν καὶ πονηρόν.
\vs{6}Καὶ εἶδεν ἡ γυνὴ, ὅτι καλὸν τὸ ξύλον εἰς βρῶσιν, καὶ ὅτι ἀρεστὸν τοῖς ὀφθαλμοῖς ἰδεῖν, καὶ ὡραῖόν ἐστι τοῦ κατανοῆσαι· καὶ λαβοῦσα ἀπὸ τοῦ καρποῦ αὐτοῦ, ἔφαγε· καὶ ἔδωκε καὶ τῷ ἀνδρὶ αὐτῆς μετʼ αὐτῆς, καὶ ἔφαγον.
\vs{7}Καὶ διηνοίχθησαν οἱ ὀφθαλμοὶ τῶν δύο, καὶ ἔγνωσαν ὅτι γυμνοὶ ἦσαν· καὶ ἔῤῥαψαν φύλλα συκῆς, καὶ ἐποίησαν ἑαυτοῖς περιζώματα.
\vs{8}Καὶ ἤκουσαν τὴς φωνὴς Κυρίου τοῦ Θεοῦ περιπατοῦντος ἐν τῷ παραδείσῳ τὸ δειλινόν· καὶ ἐκρύβησαν ὅ, τε Ἀδὰμ καὶ ἡ γυνὴ αὐτοῦ ἀπὸ προσώπου Κυρίου τοῦ Θεοῦ ἐν μέσῳ τοῦ ξύλου τοῦ παραδείσου.
\vs{9}Καὶ ἐκάλεσεν Κύριος ὁ Θεὸς τὸν Ἀδὰμ, καὶ εἶπεν αὐτῷ· Ἀδὰμ ποῦ εἶ;
\vs{10}Καὶ εἶπεν αὐτῷ· τὴς φωνῆς σου ἤκουσα περιπατοῦντος ἐν τῷ παραδείσῳ, καὶ ἐφοβήθην ὅτι γυμνός εἰμι, καὶ ἐκρύβην.
\vs{11}Καὶ εἶπεν αὐτῷ ὁ Θεὸς, τὶς ἀνήγγειλέ σοι ὅτι γυμνὸς εἶ, εἰ μὴ ἀπὸ τοῦ ξύλου, οὗ ἐνετειλάμην σοι τούτου μόνου μὴ φαγεῖν, ἀπʼ αὐτοῦ ἔφαγες;
\vs{12}Καὶ εἶπεν ὁ Ἀδάμ· ἡ γυνή, ἣν ἔδωκας μετʼ ἐμοῦ, αὕτη μοι ἔδωκεν ἀπὸ τοῦ ξύλου, καὶ ἔφαγον.
\vs{13}Καὶ εἶπε Κύριος ὁ Θεὸς τῇ γυναικί· τί τοῦτο ἐποιήσας; καὶ εἶπεν ἡ γυνὴ, ὁ ὄφις ἠπάτησέ με, καὶ ἔφαγον.

\vs{14}Καὶ εἶπε Κύριος ὁ Θεὸς τῷ ὄφει· ὅτι ἐποίησας τοῦτο, ἐπικατάρατος σὺ ἀπὸ πάντων τῶν κτηνῶν, καὶ ἀπὸ πάντων τῶν θηρίων τῶν ἐπὶ τῆς γῆς· ἐπὶ τῷ στήθει σου καὶ τῇ κοιλίᾳ πορεύσῃ, καὶ γῆν φαγῃ πάσας τὰς ἡμέρας τῆς ζωῆς σου.
\vs{15}Καὶ ἔχθραν θήσω ἀνὰ μέσον σοῦ καὶ ἀνὰ μέσον τῆς γυναικὸς, καὶ ἀνὰ μέσον τοῦ σπέρματός σου, καὶ ἀνὰ μέσον τοῦ σπέρματος αὐτῆς· αὐτός σοῦ τηρήσει κεφαλὴν, καὶ σὺ τηρήσεις αὐτοῦ πτέρναν.
\vs{16}Καὶ τῇ γυναικὶ εἶπε· πληθύνων πληθυνῶ τὰς λύπας σου, καὶ τὸν στεναγμόν σου· ἐν λύπαις τέξῃ τέκνα, καὶ πρὸς τὸν ἄνδρα σου ἡ ἀποστροφή σου· καὶ αὐτός σου κυριεύσει.
\vs{17}Τῷ δὲ Ἀδὰμ εἶπεν· ὅτι ἤκουσας τῆς φωνῆς τῆς γυναικός σου, καὶ ἔφαγες ἀπὸ τοῦ ξύλου, οὗ ἐνετειλάμην σοι τούτου μόνου μὴ φαγεῖν, ἀπʼ αὐτοῦ ἔφαγες, ἐπικατάρατος ἡ γῆ ἐν τοῖς ἔργοις σου· ἐν λύπαις φάγῃ αὐτὴν πάσας τὰς ἡμέρας τῆς ζωῆς σου.
\vs{18}Ἀκάνθας καὶ τριβόλους ἀνατελεῖ σοι, καὶ φαγῇ τὸν χόρτον τοῦ ἀγροῦ.
\vs{19}Ἐν ἱδρῶτι τοῦ προσώπου σου φαγῃ τὸν ἄρτον σου, ἕως τοῦ ἀποστρέψαι σε εἰς τὴν γῆν ἐξ ἧς ἐλήμφθης· ὅτι γῆ εἶ, καὶ εἰς γῆν ἀπελεύσῃ.
\vs{20}Καὶ ἐκάλεσεν Ἀδὰμ τὸ ὄνομα τῆς γυναικὸς αὐτοῦ Ζωή, ὅτι μήτηρ πάντων τῶν ζώντων.
\vs{21}Καὶ ἐποίησε Κύριος ὁ Θεὸς τῷ Ἀδὰμ, καὶ τῇ γυναικὶ αὐτοῦ χιτῶνας δερματίνους, καὶ ἐνέδυσεν αὐτούς.

\vs{22}Καὶ εἶπεν ὁ Θεός, ἰδοὺ Ἀδὰμ γέγονεν ὡς εἷς ἐξ ἡμῶν, τοῦ γινώσκειν καλὸν καὶ πονηρόν· καὶ νῦν μή ποτε ἐκτείνῃ τὴν χεῖρα αὐτοῦ, καὶ λάβῃ τοῦ ξύλου τῆς ζωῆς καὶ φάγῃ, καὶ ζήσεται εἰς τὸν αἰῶνα.
\vs{23}Καὶ ἐξαπέστειλεν αὐτὸν Κύριος ὁ Θεὸς ἐκ τοῦ παραδείσου τῆς τρυφῆς, ἐργάζεσθαι τὴν γῆν ἐξ ἧς ἐλήμφθη.
\vs{24}Καὶ ἐξέβαλεν τὸν Ἀδὰμ, καὶ κατῴκισεν αὐτὸν ἀπέναντι τοῦ παραδείσου τῆς τρυφῆς· καὶ ἔταξε τὰ χερουβὶμ· καὶ τὴν φλογίνην ῥομφαίαν τὴν στρεφομένην, φυλάσσειν τὴν ὁδὸν τοῦ ξύλου τῆς ζωῆς.

\ch{4}
Ἀδὰμ δὲ ἔγνω Εὔαν τὴν γυναῖκα αὐτοῦ, καὶ συλλαβοῦσα ἔτεκε τὸν Κάϊν· καὶ εἶπεν, ἐκτησάμην ἄνθρωπον διὰ τοῦ Θεοῦ.
\vs{2}Καὶ προσέθηκε τεκεῖν τὸν ἀδελφὸν αὐτοῦ τὸν Ἄβελ· καὶ ἐγένετο Ἄβελ ποιμὴν προβάτων, Κάϊν δὲ ἦν ἐργαζόμενος τὴν γῆν.
\vs{3}Καὶ ἐγένετο μεθʼ ἡμέρας ἤνεγκε Κάϊν ἀπὸ τῶν καρπῶν τῆς γῆς θυσίαν τῷ Κυρίῳ·
\vs{4}Καὶ Ἄβελ ἤνεγκε καὶ αὐτὸς ἀπὸ τῶν πρωτοτόκων τῶν προβάτων αὐτοῦ, καὶ ἀπὸ τῶν στεάτων αὐτῶν· καὶ ἐπεῖδεν ὁ Θεὸς ἐπὶ Ἄβελ, καὶ ἐπὶ τοῖς δώροις αὐτοῦ.
\vs{5}Ἐπὶ δὲ Κάϊν, καὶ ἐπὶ ταῖς θυσίαις αὐτοῦ, οὐ προσέσχε· καὶ ἐλυπήθη Κάϊν λίαν, καὶ συνέπεσε τῷ προσώπῳ αὐτοῦ.
\vs{6}Καὶ εἶπε Κύριος ὁ Θεὸς τῷ Κάϊν, ἵνα τί περίλυπος ἐγένου, καὶ ἵνα τί συνέπεσε τὸ πρόσωπόν σου;
\vs{7}Οὐκ ἐὰν ὀρθῶς προσενέγκῃς, ὀρθῶς δὲ μὴ διέλῃς, ἥμαρτες; ἡσυχασον· πρός σὲ ἡ ἀποστροφὴ αὐτοῦ, καὶ σὺ ἄρξεις αὐτοῦ.

\vs{8}Καὶ εἶπεν Κάϊν πρὸς Ἄβελ τὸν ἀδελφὸν αὐτοῦ, διέλθωμεν εἰς τὸ πεδίον· καὶ ἐγένετο ἐν τῷ εἶναι αὐτοὺς ἐν τῷ πεδίῳ, ἀνέστη Κάϊν ἐπὶ Ἄβελ τὸν ἀδελφὸν αὐτοῦ, καὶ ἀπέκτεινεν αὐτόν.
\vs{9}Καὶ εἶπε Κύπιος ὁ Θεὸς πρὸς Κάϊν· ποῦ ἔστιν Ἄβελ ὁ ἀδελφός σου; καὶ εἶπεν, οὐ γινώσκω· μὴ φύλαξ τοῦ ἀδελφοῦ μου εἰμὶ ἐγώ;
\vs{10}Καὶ εἶπε Κύριος, τί πεποίηκας; φωνὴ αἵματος τοῦ ἀδελφοῦ σου βοᾷ πρός με ἐκ τῆς γῆς.
\vs{11}Καὶ νῦν ἐπικατάρατος σὺ ἀπὸ τῆς γῆς, ἣ ἔχανε τὸ στόμα αὐτῆς δέξασθαι τὸ αἷμα τοῦ ἀδελφοῦ σου ἐκ τῆς χειρός σου.
\vs{12}Ὅτε ἐργᾷ τὴν γῆν, καὶ οὐ προσθήσει τὴν ἰσχὺν αὐτῆς δοῦναί σοι· στένων καὶ τρέμων ἐσῃ ἐπὶ τῆς γῆς.
\vs{13}Καὶ εἶπε Κάϊν πρὸς Κύριον τὸν Θεὸν, μείζων ἡ αἰτία μου τοῦ ἀφεθῆναί με.
\vs{14}Εἰ ἐκβάλλεις με σήμερον ἀπὸ προσώπου τῆς γῆς, καὶ ἀπὸ τοῦ προσώπου σου κρυβήσομαι, καὶ ἔσομαι στένων καὶ τρέμων ἐπὶ τῆς γῆς, καὶ ἔσται πᾶς ὁ εὑρίσκων με, ἀποκτενεῖ με.
\vs{15}Καὶ εἴπεν αὐτῷ Κύριος ὁ Θεὸς, οὐχ οὕτω· πᾶς ὁ ἀποκτείνας Κάϊν, ἑπτὰ ἐκδικούμενα παραλύσει. Καὶ ἔθετο Κύριος ὁ Θεὸς σημεῖον τῷ Κάϊν, τοῦ μὴ ἀνελεῖν αὐτὸν πάντα τὸν εὑρίσκοντα αὐτόν.
\vs{16}Ἐξῆλθεν δὲ Κάϊν ἀπὸ προσώπου τοῦ Θεοῦ, καὶ ᾤκησεν ἐν γῇ Ναὶδ κατέναντι Ἐδέμ.

\vs{17}Καὶ ἔγνω Κάϊν τὴν γυναῖκα αὐτοῦ· καὶ συλλαβοῦσα ἔτεκε τὸν Ἐνώχ. Καὶ ἦν οἰκοδομῶν πόλιν· καὶ ἐπῳνόμασε τὴν πόλιν ἐπὶ τῷ ὀνόματι τοῦ υἱοῦ αὐτοῦ, Ἐνώχ.
\vs{18}Ἐγενήθη δὲ τῷ Ἐνὼχ Γαϊδάδ· καὶ Γαϊδὰδ ἐγέννησε τὸν Μαλελεὴλ· καὶ Μαλελεὴλ ἐγέννησε τὸν Μαθουσάλα· καὶ Μαθουσάλα ἐγέννησε τὸν Λάμεχ.

\vs{19}Καὶ ἔλαβεν ἑαυτῷ Λάμεχ δύο γυναῖκας· ὄνομα τῇ μιᾷ, Ἀδά· καὶ ὄνομα τῇ δευτέρᾳ, Σελλά.
\vs{20}Καὶ ἔτεκεν Ἀδὰ τὸν Ἰωβήλ· οὗτος ἦν πατὴρ οἰκούντων ἐν σκηναῖς κτηνοτρόφων.
\vs{21}Καὶ ὄνομα τῷ ἀδελφῷ αὐτοῦ, Ἰουβάλ· οὗτος ἦν ὁ καταδείξας ψαλτήριον καὶ κιθάραν.
\vs{22}Σελλὰ δὲ καὶ αὐτὴ ἔτεκε τὸν Θόβελ· καὶ ἦν σφυροκόπος χαλκεὺς χαλκοῦ καὶ σιδήρου. ἀδελφὴ δὲ Θόβελ, Νοεμά.
\vs{23}Εἶπε δὲ Λάμεχ ταῖς ἑαυτοῦ γυναιξίν, Ἀδὰ καὶ Σελλὰ, ἀκούσατέ μου τῆς φωνῆς, γυναῖκες Λάμεχ, ἐνωτίσασθέ μου τοὺς λόγους· ὅτι ἄνδρα ἀπέκτεινα εἰς τραῦμα ἐμοὶ, καὶ νεανίσκον εἰς μώλωπα ἐμοί.
\vs{24}Ὅτι ἑπτάκις ἐκδεδίκηται ἐκ Κάϊν· ἐκ δὲ Λάμεχ, ἑβδομηκοντάκις ἑπτά.

\vs{25}Ἔγνω δὲ Ἀδὰμ Εὔαν τὴν γυναῖκα αὐτοῦ· καὶ συλλαβοῦσα ἔτεκεν υἱόν· καὶ ἐπωνόμασε τὸ ὄνομα αὐτοῦ Σὴθ, λέγουσα, ἐξανέστησε γάρ μοι ὁ Θεὸς σπέρμα ἕτερον ἀντὶ Ἄβελ, ὃν ἀπέκτεινε Κάϊν.
\vs{26}Καὶ τῷ Σὴθ ἐγένετο υἱός· ἐπωνόμασε δὲ τὸ ὄνομα αὐτοῦ, Ἑνώς· οὗτος ἤλπισεν ἐπικαλεῖσθαι τὸ ὄνομα Κυρίου τοῦ Θεοῦ.

\ch{5}
Αὕτη ἡ βίβλος γενέσεως ἀνθρώπων· ᾗ ἡμέρᾳ ἐποίησεν ὁ Θεὸς τὸν Ἀδὰμ, κατʼ εἰκόνα Θεοῦ ἐποίησεν αὐτόν·
\vs{2}Ἄρσεν καὶ θῆλυ ἐποίησεν αὐτούς· καὶ εὐλόγησεν αὐτούς· καὶ ἐπωνόμασε τὸ ὄνομα αὐτοῦ Ἀδὰμ, ᾗ ἡμέρᾳ ἐποίησεν αὐτούς.
\vs{3}Ἔζησεν δὲ Ἀδὰμ τριάκοντα καὶ διακόσια ἔτη, καὶ ἐγέννησε κατὰ τὴν ἰδέαν αὐτοῦ, καὶ κατὰ τὴν εἰκόνα αὐτοῦ, καὶ ἐπωνόμασε τὸ ὄνομα αὐτοῦ, Σήθ.
\vs{4}Ἐγένοντο δὲ αἱ ἡμέραι Ἀδὰμ, ἃς ἔζησε μετὰ τὸ γεννῆσαι αὐτὸν τὸν Σὴθ, ἔτη ἑπτακόσια· καὶ ἐγέννησεν υἱοὺς καὶ θυγατέρας.
\vs{5}Καὶ ἐγένοντο πᾶσαι αἱ ἡμέραι Ἀδὰμ, ἃς ἔζησε, τριάκοντα καὶ ἐννακόσια ἔτη· καὶ ἀπέθανεν.
\vs{6}Ἔζησε δὲ Σὴθ πέντε καὶ διακόσια ἔτη· καὶ ἐγέννησε τὸν Ἐνώς.
\vs{7}Καὶ ἔζησε Σὴθ μετὰ τὸ γεννῆσαι αὐτὸν τὸν Ἐνὼς, ἑπτὰ ἔτη καὶ ἑπτακόσια· καὶ ἐγέννησεν υἱοὺς καὶ θυγατέρας.
\vs{8}Καὶ ἐγένοντο πᾶσαι αἱ ἡμέραι Σὴθ δώδεκα καὶ ἐννακόσια ἔτη· καὶ ἀπέθανε.
\vs{9}Καὶ ἔζησεν Ἐνὼς ἔτη ἑκατὸν ἐννεήκοντα· καὶ ἐγέννησε τὸν Καϊνᾶν.
\vs{10}Καὶ ἔζησεν Ἐνὼς μετὰ τὸ γεννῆσαι αὐτὸν τὸν Καϊνᾶν, πεντεκαίδεκα ἔτη καὶ ἑπτκόσια· καὶ ἐγέννησεν υἱοὺς καὶ θυγατέρας.
\vs{11}Καὶ ἐγένοντο πᾶσαι αἱ ἡμέραι Ἐνὼς πέντε ἔτη καὶ ἐννακόσια· καὶ ἀπέθανε.
\vs{12}Καὶ ἔζησεν Καϊνᾶν ἑβδομήκοντα καὶ ἑκατὸν ἔτη· καὶ ἐγέννησε τὸν Μαλελεήλ.
\vs{13}Καὶ ἔζησε Καϊνᾶν μετὰ τὸ γεννῆσαι αὐτὸν τὸν Μαλελεὴλ, τεσσεράκοντα καὶ ἑπτακόσια ἔτη· καὶ ἐγέννησεν υἱοὺς καὶ θυγατέρας.
\vs{14}Καὶ ἐγένοντο πᾶσαι αἱ ἡμέραι Καϊνᾶν δέκα ἔτη καὶ ἐννακόσια· καὶ ἀπέθανε.

\vs{15}Καὶ ἔζησε Μαλελεὴλ πέντε καὶ ἑξήκοντα καὶ ἑκατὸν ἔτη· καὶ ἐγέννησε τὸν Ἰάρεδ.
\vs{16}Καὶ ἔζησε Μαλελεὴλ μετὰ τὸ γεννῆσαι αὐτὸν τὸν Ἰάρεδ, ἔτη τριάκοντα καὶ ἑπτακόσια· καὶ ἐγέννησεν υἱοὺς καὶ θυγατέρας.
\vs{17}Καὶ ἐγένοντο πᾶσαι αἱ ἡμέραι Μαλελεὴλ, ἔτη πέντε καὶ ἐννενήκοντα καὶ ὀκτακόσια· καὶ ἀπέθανε.
\vs{18}Καὶ ἔζησεν Ἰάρεδ δύο καὶ ἑξήκοντα ἔτη καὶ ἑκατὸν· καὶ ἐγέννησε τὸν Ἐνώχ.
\vs{19}Καὶ ἔζησεν Ἰάρεδ μετὰ τὸ γεννῆσαι αὐτὸν τὸν Ἐνὼχ, ὀκτακόσια ἔτη· καὶ ἐγέννησεν υἱοὺς καὶ θυγατέρας.
\vs{20}Καὶ ἐγένοντο πᾶσαι αἱ ἡμέραι Ἰάρεδ, δύο καὶ ἑξήκοντα καὶ ἐννακόσια ἔτη· καὶ ἀπέθανε.
\vs{21}Καὶ ἔζησεν Ἐνὼχ πέντε καὶ ἑξήκοντα καὶ ἑκατὸν ἔτη· καὶ ἐγέννησε τὸν Μαθουσάλα.
\vs{22}Εὐηρέστησε δὲ Ἐνὼχ τῷ Θεῷ μετὰ τὸ γεννῆσαι αὐτὸν τὸν Μαθουσάλα, διακόσια ἔτη· καὶ ἐγέννησεν υἱοὺς καὶ θυγατέρας.
\vs{23}Καὶ ἐγένοντο πᾶσαι αἱ ἡμέραι Ἐνὼχ, πέντε καὶ ἑξήκοντα καὶ τριακόσια ἔτη.
\vs{24}Καὶ εὐηρέστησεν Ἐνὼχ τῷ Θεῷ· καὶ οὐχ εὑρίσκετο, ὅτι μετέθηκεν αὐτὸν ὁ Θεός.
\vs{25}Καὶ ἔζησε Μαθουσάλα ἑπτὰ ἔτη καὶ ἑξήκοντα καὶ ἑκατόν· καὶ ἐγέννησε τὸν Λάμεχ.
\vs{26}Καὶ ἔζησε Μαθουσάλα μετὰ τὸ γεννῆσαι αὐτὸν τὸν Λάμεχ, δύο καὶ ὀκτακόσια ἔτη· καὶ ἐγέννησεν υἱοὺς καὶ θυγατέρας.
\vs{27}Καὶ ἐγένοντο πᾶσαι αἱ ἡμέραι Μαθουσάλα ἃς ἔζησεν, ἐννέα καὶ ἑξήκοντα καὶ ἐννακόσια ἔτη· καὶ ἀπέθανε.
\vs{28}Καὶ ἔζησε Λάμεχ ὀκτὼ καὶ ὀγδοήκοντα καὶ ἑκατὸν ἔτη· καὶ ἐγέννησεν υἱόν.
\vs{29}Καὶ ἐπωνόμασε τὸ ὄνομα αὐτοῦ Νῶε, λέγων, οὗτος διαναπαύσει ἡμᾶς ἀπὸ τῶν ἔργων ἡμῶν, καὶ ἀπὸ τῶν λυπῶν τῶν χειρῶν ἡμῶν, καὶ ἀπὸ τῆς γῆς, ἧς κατηράσατο Κύριος ὁ Θεός.
\vs{30}Καὶ ἔζησε Λάμεχ μετὰ τὸ γεννῆσαι αὐτὸν τὸν Νῶε, πεντακόσια καὶ ἑξήκοντα καὶ πέντε ἔτη· καὶ ἐγέννησεν υἱοὺς καὶ θυγατέρας.
\vs{31}Καὶ ἐγένοντο πᾶσαι αἱ ἡμέραι Λάμεχ, ἑπτακόσια καὶ πεντήκοντα τρία ἔτη· καὶ ἀπέθανε.
\vs{32}Καὶ ἦν Νῶε ἐτῶν πεντακοσίων· καὶ ἐγέννησε τρεῖς υἱοὺς, τὸν Σὴμ, τὸν Χὰμ, τὸν Ἰάφεθ.

\ch{6}
Καὶ ἐγένετο ἡνίκα ἤρξαντο οἱ ἄνθρωποι πολλοὶ γίνεσθαι ἐπὶ τῆς γῆς, καὶ θυγατέρες ἐγεννήθησαν αὐτοῖς.
\vs{2}Ἰδόντες δὲ υἱοὶ τοῦ Θεοῦ τὰς θυγατέρας τῶν ἀνθρώπων, ὅτι καλαί εἰσιν, ἔλαβον ἑαυτοῖς γυναῖκας ἀπὸ πασῶν, ὧν ἐξελέξαντο.
\vs{3}Καὶ εἶπε Κύριος ὁ Θεὸς, οὐ μὴ καταμείνῃ τὸ πνεῦμά μου ἐν τοῖς ἀνθρώποις τούτοις εἰς τὸν αἰῶνα, διὰ τὸ εἶναι αὐτοὺς σάρκας· ἔσονται δὲ αἱ ἡμέραι αὐτῶν, ἑκατὸν εἴκοσιν ἔτη.
\vs{4}Οἱ δὲ γίγαντες ἦσαν ἐπὶ τῆς γῆς ἐν ταῖς ἡμέραις ἐκείναις, καὶ μετʼ ἐκεῖνο, ὡς ἂν εἰσεπορεύοντο οἱ υἱοὶ τοῦ Θεοῦ πρὸς τὰς θυγατέρας τῶν ἀνθρώπων, καὶ ἐγεννῶσαν αὐτοῖς· ἐκεῖνοι ἦσαν οἱ γίγαντες οἱ ἀπʼ αἰῶνος, οἱ ἄνθρωποι οἱ ὀνομαστοί.

\vs{5}Ἰδὼν δὲ Κύριος ὁ Θεὸς, ὅτι ἐπληθύνθησαν αἱ κακίαι τῶν ἀνθρώπων ἐπὶ τῆς γῆς, καὶ πᾶς τις διανοεῖται ἐν τῇ καρδίᾳ αὐτοῦ ἐπιμελῶς ἐπὶ τὰ πονηρὰ πάσας τὰς ἡμέρας·
\vs{6}Καὶ ἐνεθυμήθη ὁ Θεὸς, ὅτι ἐποίησε τὸν ἄνθρωπον ἐπὶ τῆς γῆς, καὶ διενοήθη.
\vs{7}Καὶ εἶπεν ὁ Θεὸς, ἀπαλείψω τὸν ἄνθρωπον, ὃν ἐποίησα, ἀπὸ προσώπου τῆς γῆς, ἀπὸ ἀνθρώπου ἕως κτήνους, καὶ ἀπὸ ἑρπετῶν ἕως πετεινῶν τοῦ οὐρανοῦ· ὅτι ἐνεθυμήθην, ὅτι ἐποίησα αὐτούς.

\vs{8}Νῶε δὲ εὗρε χάριν ἐναντίον Κυρίου τοῦ Θεοῦ.
\vs{9}Αὗται δὲ αἱ γενέσεις Νῶε. Νῶε ἄνθρωπος δίκαιος, τέλειος ὢν ἐν τῇ γενεᾷ αὐτοῦ, τῷ Θεῷ εὐηρέστησε Νῶε.
\vs{10}Ἐγέννησε δὲ Νῶε τρεῖς υἱοὺς, τὸν Σὴμ, τὸν Χὰμ, τὸν Ἰάφεθ.
\vs{11}Ἐφθάρη δὲ ἡ γῆ ἐναντίον τοῦ Θεοῦ, καὶ ἐπλήσθη ἡ γῆ ἀδικίας.
\vs{12}Καὶ εἶδε Κύριος ὁ Θεὸς τὴν γῆν, καὶ ἦν κατεφθαρμένη· ὅτι κατέφθειρε πᾶσα σὰρξ τὴν ὁδὸν αὐτοῦ ἐπὶ τῆς γῆς.
\vs{13}Καὶ εἶπε Κύριος ὁ Θεὸς τῷ Νῶε, καιρὸς παντὸς ἀνθρώπου ἥκει ἐναντίον μου, ὅτι ἐπλήσθη ἡ γῆ ἀδικίας ἀπʼ αὐτῶν· καὶ ἰδοὺ ἐγὼ καταφθείρω αὐτοὺς καὶ τὴν γῆν.

\vs{14}Ποίησον οὖν σεαυτῷ κιβωτὸν ἐκ ξύλων τετραγώνων· νοσσιὰς ποιήσεις τὴν κιβωτόν· καὶ ἀσφαλτώσεις αὐτὴν ἔσωθεν καὶ ἔξωθεν τῇ ἀσφάλτῳ.
\vs{15}Καὶ οὕτω ποιήσεις τὴν κιβωτόν· τριακοσίων πήχεων τὸ μῆκος τῆς κιβωτοῦ, καὶ πεντήκοντα πήχεων τὸ πλάτος, καὶ τριάκοντα πήχεων τὸ ὕψος αὐτῆς.
\vs{16}Ἐπισυνάγων ποιήσεις τὴν κιβωτὸν, καὶ εἰς πῆχυν συντελέσεις αὐτὴν ἄνωθεν· τὴν δὲ θύραν τῆς κιβωτοῦ ποιήσεις ἐκ πλαγίων· κατάγαια διώροφα καὶ τριώροφα ποιήσεις αὐτήν.
\vs{17}Ἐγὼ δὲ ἰδοὺ ἐπάγω τὸν κατακλυσμὸν, ὕδωρ ἐπὶ τὴν γῆν, καταφθεῖραι πᾶσαν σάρκα, ἐν ᾗ ἐστι πνεῦμα ζωῆς ὑποκάτω τοῦ οὐρανοῦ· καὶ ὅσα ἂν ᾖ ἐπὶ τῆς γῆς, τελευτήσει.

\vs{18}Καὶ στήσω τὴν διαθήκην μου μετά σου· εἰσελεύσῃ δὲ εἰς τὴν κιβωτὸν σὺ, καὶ οἱ υἱοί σου, καὶ ἡ γυνή σου, καὶ αἱ γυναῖκες τῶν υἱῶν σου μετά σου.
\vs{19}Καὶ ἀπὸ πάντων τῶν κτηνῶν, καὶ ἀπὸ πάντων τῶν ἑρπετῶν, καὶ ἀπὸ πάντων τῶν θηρίων, καὶ ἀπὸ πάσης σαρκὸς δύο δύο ἀπὸ πάντων εἰσάξεις εἰς τὴν κιβωτὸν, ἵνα τρέφῃς μετὰ σεαυτοῦ· ἄρσεν καὶ θῆλυ ἔσονται.
\vs{20}Ἀπὸ πάντων τῶν ὀρνέων τῶν πετεινῶν κατὰ γένος, καὶ ἀπὸ πάντων τῶν κτηνῶν κατὰ γένος, καὶ ἀπὸ πάντων τῶν ἑρπετῶν τῶν ἑρπόντων ἐπὶ τῆς γῆς κατὰ γένος αὐτῶν, δύο δύο ἀπὸ πάντων εἰσελεύσονται πρὸς σὲ τρέφεσθαι μετά σου, ἄρσεν καὶ θῆλυ.
\vs{21}Σὺ δὲ λήψῃ σεαυτῷ ἀπὸ πάντων τῶν βρωμάτων ἃ ἔδεσθε, καὶ συνάξεις πρὸς σεαυτὸν, καὶ ἔσται σοι καὶ ἐκείνοις φαγεῖν.
\vs{22}Καὶ ἐποίησε Νῶε πάντα ὅσα ἐνετείλατο αὐτῷ Κύριος ὁ Θεὸς, οὕτως ἐποίησε.

\ch{7}
Καὶ εἶπε Κύριος ὁ Θεὸς πρὸς Νῶε, εἴσελθε σὺ καὶ πᾶς ὁ οἶκός σου εἰς τὴν κιβωτὸν, ὅτι σὲ εἶδον δίκαιον ἐναντίον μου ἐν τῇ γενεᾷ ταύτῃ.
\vs{2}Ἀπὸ δὲ τῶν κτηνῶν τῶν καθαρῶν εἰσάγαγε πρὸς σὲ ἑπτὰ ἑπτὰ ἄρσεν καὶ θῆλυ, ἀπὸ δὲ τῶν κτηνῶν τῶν μὴ καθαρῶν δύο δύο ἄρσεν καὶ θῆλυ.
\vs{3}Καὶ ἀπὸ τῶν πετεινῶν τοῦ οὐρανοῦ τῶν καθαρῶν ἑπτὰ ἑπτὰ ἄρσεν καὶ θῆλυ, καὶ ἀπὸ πάντων τῶν πετεινῶν τῶν μὴ καθαρῶν δύο δύο ἄρσεν καὶ θῆλυ, διαθρέψαι σπέρμα ἐπὶ πᾶσαν τὴν γῆν.
\vs{4}Ἔτι γὰρ ἡμερῶν ἑπτὰ ἐγὼ ἐπάγω ὑετὸν ἐπὶ τὴν γῆν, τεσσαράκοντα ἡμέρας καὶ τεσσαράκοντα νύκτας· καὶ ἐξαλείψω πᾶν τὸ ἀνάστημα, ὃ ἐποίησα ἀπὸ προσώπου πάσης τῆς γῆς.
\vs{5}Καὶ ἐποίησε Νῶε πάντα, ὅσα ἐνετείλατο αὐτῷ Κύριος ὁ Θεός.
\vs{6}Νῶε δὲ ἦν ἐτῶν ἑξακοσίων, καὶ ὁ κατακλυσμὸς τοῦ ὕδατος ἐγένετο ἐπὶ τῆς γῆς.
\vs{7}Εἰσῆλθε δὲ Νῶε καὶ οἱ υἱοὶ αὐτοῦ, καὶ ἡ γυνὴ αὐτοῦ, καὶ αἱ γυναῖκες τῶν υἱῶν αὐτοῦ μετʼ αὐτοῦ εἰς τὴν κιβωτὸν, διὰ τὸ ὕδωρ τοῦ κατακλυσμοῦ.
\vs{8}Καὶ ἀπὸ τῶν πετεινῶν τῶν καθαρῶν, καὶ ἀπὸ τῶν πετεινῶν τῶν μὴ καθαρῶν, καὶ ἀπὸ τῶν κτηνῶν τῶν καθαρῶν, καὶ ἀπὸ τῶν κτηνῶν τῶν μὴ καθαρῶν, καὶ ἀπὸ πάντων τῶν ἑρπόντων ἐπὶ τῆς γῆς,
\vs{9}δύο δύο εἰσῆλθον πρὸς Νῶε εἰς τὴν κιβωτὸν ἄρσεν καὶ θῆλυ, καθὰ ἐνετείλατο ὁ Θεὸς τῷ Νῶε.
\vs{10}Καὶ ἐγένετο μετὰ τὰς ἑπτὰ ἡμέρας, καὶ τὸ ὕδωρ τοῦ κατακλυσμοῦ ἐγένετο ἐπὶ τῆς γῆς.
\vs{11}Ἐν τῷ ἑξακοσιοστῷ ἔτει ἐν τῇ ζωῇ τοῦ Νῶε, τοῦ δευτέρου μηνὸς, ἑβδόμῃ καὶ εἰκάδι τοῦ μηνὸς, τῇ ἡμέρᾳ ταύτῃ ἐῤῥάγησαν πᾶσαι αἱ πηγαὶ τῆς ἀβύσσου, καὶ οἱ καταῤῥάκται τοῦ οὐρανοῦ ἠνεῴχθησαν.
\vs{12}Καὶ ἐγένετο ὁ ὑετὸς ἐπὶ τῆς γῆς τεσσαράκοντα ἡμέρας καὶ τεσσαράκοντα νύκτας.
\vs{13}Ἐν τῇ ἡμέρᾳ ταύτῃ εἰσῆλθε Νῶε, Σὴμ, Χὰμ, Ἰάφεθ, οἱ υἱοὶ Νῶε, καὶ ἡ γυνὴ Νῶε, καὶ αἱ τρεῖς γυναῖκες τῶν υἱῶν αὐτοῦ μετʼ αὐτοῦ, εἰς τὴν κιβωτόν.
\vs{14}Καὶ πάντα τὰ θηρία κατὰ γένος, καὶ πάντα τὰ κτήνη κατὰ γένος, καὶ πᾶν ἑρπετὸν κινούμενον ἐπὶ τῆς γῆς κατὰ γένος, καὶ πᾶν ὄρνεον πετεινὸν κατὰ γένος αὐτοῦ,
\vs{15}εἰσῆλθον πρὸς Νῶε εἰς τὴν κιβωτὸν, δύο δύο ἄρσεν καὶ θῆλυ ἀπὸ πάσης σαρκὸς, ἐν ᾧ ἐστι πνεῦμα ζωῆς.
\vs{16}Καὶ τὰ εἰσπορευόμενα ἄρσεν καὶ θῆλυ ἀπὸ πάσης σαρκὸς εἰσῆλθε, καθὰ ἐνετείλατο ὁ Θεὸς τῷ Νῶε· καὶ ἔκλεισε Κύριος ὁ Θεὸς τὴν κιβωτὸν ἔξωθεν αὐτοῦ.

\vs{17}Καὶ ἐγένετο ὁ κατακλυσμὸς τεσσαράκοντα ἡμέρας καὶ τεσσαράκοντα νύκτας ἐπὶ τῆς γῆς· καὶ ἐπεπληθύνθη τὸ ὕδωρ· καὶ ἐπῇρε τὴν κιβωτὸν, καὶ ὑψώθη ἀπὸ τῆς γῆς.
\vs{18}Καὶ ἐπεκράτει τὸ ὕδωρ, καὶ ἐπληθύνετο σφόδρα ἐπὶ τῆς γῆς· καὶ ἐπεφέρετο ἡ κιβωτὸς ἐπάνω τοῦ ὕδατος.
\vs{19}Τὸ δὲ ὕδωρ ἐπεκράτει σφόδρα σφόδρα ἐπὶ τῆς γῆς· καὶ ἐκάλυψε πάντα τὰ ὄρη τὰ ὑψηλὰ, ἃ ἦν ὑποκάτω τοῦ οὐρανοῦ.
\vs{20}Πεντεκαίδεκα πήχεις ὑπεράνω ὑψώθη τὸ ὕδωρ· καὶ ἐπεκάλυψε πάντα τὰ ὄρη τὰ ὑψηλά.
\vs{21}Καὶ ἀπέθανε πᾶσα σὰρξ κινουμένη ἐπὶ τῆς γῆς τῶν πετεινῶν, καὶ τῶν κτηνῶν, καὶ τῶν θηρίων· καὶ πᾶν ἑρπετὸν κινούμενον ἐπὶ τῆς γῆς, καὶ πᾶς ἄνθρωπος.
\vs{22}Καὶ πάντα ὅσα ἔχει πνοὴν ζωῆς, καὶ πᾶν ὃ ἦν ἐπὶ τῆς ξηρᾶς, ἀπέθανε.
\vs{23}Καὶ ἐξήλειψε πᾶν τὸ ἀνάστημα, ὃ ἦν ἐπὶ προσώπου τῆς γῆς, ἀπὸ ἀνθρώπου ἕως κτήνους, καὶ ἑρπετῶν, καὶ τῶν πετεινῶν τοῦ οὐρανοῦ· καὶ ἐξηλείφησαν ἀπὸ τῆς γῆς· καὶ κατελείφθη μόνος Νῶε, καὶ οἱ μετʼ αὐτοῦ ἐν τῇ κιβωτῷ.
\vs{24}Καὶ ὑψώθη τὸ ὕδωρ ἐπὶ τῆς γῆς ἡμέρας ἑκατὸν πεντήκοντα.

\ch{8}
Καὶ ἀνεμνήσθη ὁ Θεὸς τοῦ Νῶε, καὶ πάντων τῶν θηρίων, καὶ πάντων τῶν κτηνῶν, καὶ πάντων τῶν πετεινῶν, καὶ πάντων τῶν ἑρπετῶν τῶν ἑρπόντων, ὅσα ἦν μετʼ αὐτοῦ ἐν τῇ κιβωτῷ· καὶ ἐπήγαγεν ὁ Θεὸς πνεῦμα ἐπὶ τὴν γῆν, καὶ ἐκόπασε τὸ ὕδωρ.
\vs{2}Καὶ ἐπεκαλύφθησαν αἱ πηγαὶ τῆς ἀβύσσου, καὶ οἱ καταῤῥάκται τοῦ οὐρανοῦ, καὶ συνεσχέθη ὁ ὑετὸς ἀπὸ τοῦ οὐρανοῦ.
\vs{3}Καὶ ἐνεδίδου τὸ ὕδωρ πορευόμενον ἀπὸ τῆς γῆς· καὶ ἠλαττονοῦτο τὸ ὕδωρ μετὰ πεντήκοντα καὶ ἑκατὸν ἡμέρας.
\vs{4}Καὶ ἐκάθισεν ἡ κιβωτὸς ἐν μηνὶ τῷ ἑβδόμῳ, ἑβδόμῃ καὶ εἰκάδι τοῦ μηνὸς, ἐπὶ τὰ ὄρη τὰ Ἀραράτ.
\vs{5}Τὸ δὲ ὕδωρ ἠλαττονοῦτο ἕως τοῦ δεκάτου μηνός. Καὶ ἐν τῷ δεκάτῳ μηνὶ, τῇ πρώτῃ τοῦ μηνὸς, ὤφθησαν αἱ κεφαλαὶ τῶν ὀρέων.
\vs{6}Καὶ ἐγένετο μετὰ τεσσαράκοντα ἡμέρας ἠνέῳξε Νῶε τὴν θυρίδα τῆς κιβωτοῦ, ἣν ἐποίησε.
\vs{7}Καὶ ἀπέστειλε τὸν κόρακα· καὶ ἐξελθὼν, οὐκ ἀνέστρεψεν ἕως τοῦ ξηρανθῆναι τὸ ὕδωρ ἀπὸ τῆς γῆς.
\vs{8}Καὶ ἀπέστειλε τὴν περιστερὰν ὀπίσω αὐτοῦ, ἰδεῖν εἰ κεκόπακε τὸ ὕδωρ ἀπὸ τῆς γῆς.
\vs{9}Καὶ οὐχ εὑροῦσα ἡ περιστερὰ ἀνάπαυσιν τοῖς ποσὶν αὐτῆς, ἀνέστρεψε πρὸς αὐτὸν εἰς τὴν κιβωτὸν, ὅτι ὕδωρ ἦν ἐπὶ πᾶν τὸ πρόσωπον τῆς γῆς· καὶ ἐκτείνας τὴν χεῖρα ἔλαβεν αὐτὴν, καὶ εἰσήγαγεν αὐτὴν πρὸς ἑαυτὸν εἰς τὴν κιβωτόν.
\vs{10}Καὶ ἐπισχὼν ἔτι ἡμέρας ἑπτὰ ἑτέρας, πάλιν ἐξαπέστειλε τὴν περιστερὰν ἐκ τῆς κιβωτοῦ.
\vs{11}Καὶ ἀνέστρεψε πρὸς αὐτὸν ἡ περιστερὰ τὸ πρὸς ἑσπέραν· καὶ εἶχε φύλλον ἐλαίας κάρφος ἐν τῷ στόματι αὐτῆς· καὶ ἔγνω Νῶε, ὅτι κεκόπακε τὸ ὕδωρ ἀπὸ τῆς γῆς.
\vs{12}Καὶ ἐπισχὼν ἔτι ἡμέρας ἑπτὰ ἑτέρας, πάλιν ἐξαπέστειλε τὴν περιστερὰν, καὶ οὐ προσέθετο τοῦ ἐπιστρέψαι πρὸς αὐτὸν ἔτι.
\vs{13}Καὶ ἐγένετο ἐν τῷ ἑνὶ καὶ ἑξακοσιοστῷ ἔτει ἐν τῇ ζωῇ τοῦ Νῶε, τοῦ πρώτου μηνὸς, μιᾷ τοῦ μηνὸς, ἐξέλιπε τὸ ὕδωρ ἀπὸ τῆς γῆς. Καὶ ἀπεκάλυψε Νῶε τὴν στέγην τῆς κιβωτοῦ, ἣν ἐποίησε· καὶ εἶδεν ὅτι ἐξέλιπε τὸ ὕδωρ ἀπὸ προσώπου τῆς γῆς.
\vs{14}Ἐν δὲ τῷ δευτέρῳ μηνὶ ἐξηράνθη ἡ γῆ, ἑβδόμῃ καὶ εἰκάδι τοῦ μηνός.

\vs{15}Καὶ εἶπε Κύριος ὁ Θεὸς πρὸς Νῶε, λέγων,
\vs{16}Ἔξελθε ἐκ τῆς κιβωτοῦ σὺ, καὶ ἡ γυνή σου, καὶ οἱ υἱοί σου, καὶ αἱ γυναῖκες τῶν υἱῶν σου μετὰ σοῦ,
\vs{17}Καὶ πάντα τὰ θηρία ὅσα ἐστὶ μετὰ σοῦ, καὶ πᾶσα σὰρξ ἀπὸ πετεινῶν ἕως κτηνῶν, καὶ πᾶν ἑρπετὸν κινούμενον ἐπὶ τῆς γῆς, ἐξάγαγε μετὰ σεαυτοῦ. καὶ αὐξάνεσθε καὶ πληθύνεσθε ἐπὶ τῆς γῆς.
\vs{18}Καὶ ἐξῆλθε Νῶε, καὶ ἡ γυνὴ αὐτοῦ, καὶ οἱ υἱοὶ αὐτοῦ, καὶ αἱ γυναῖκες τῶν υἱῶν αὐτοῦ μετʼ αὐτοῦ·
\vs{19}Καὶ πάντα τὰ θηρία, καὶ πάντα τὰ κτήνη, καὶ πᾶν πετεινὸν, καὶ πᾶν ἑρπετὸν κινούμενον ἐπὶ τῆς γῆς κατὰ γένος αὐτῶν, ἐξήλθοσαν ἐκ τῆς κιβωτοῦ.

\vs{20}Καὶ ᾠκοδόμησε Νῶε θυσιαστήριον τῷ Κυρίῳ· καὶ ἔλαβεν ἀπὸ πάντων τῶν κτηνῶν τῶν καθαρῶν, καὶ ἀπὸ πάντων τῶν πετεινῶν τῶν καθαρῶν, καὶ ἀνήνεγκεν εἰς ὁλοκάρπωσιν ἐπὶ τὸ θυσιαστήριον.
\vs{21}Καὶ ὠσφράνθη Κύριος ὁ Θεὸς ὀσμὴν εὐωδίας. Καὶ εἶπε Κύριος ὁ Θεὸς διανοηθείς, οὐ προσθήσω ἔτι καταράσασθαι τὴν γῆν διὰ τὰ ἔργα τῶν ἀνθρώπων· ὅτι ἔγκειται ἡ διάνοια τοῦ ἀνθρώπου ἐπιμελῶς ἐπὶ τὰ πονηρὰ ἐκ νεότητος αὐτοῦ· οὐ προσθήσω οὖν ἔτι πατάξαι πᾶσαν σάρκα ζῶσαν, καθὼς ἐποίησα.
\vs{22}Πάσας τὰς ἡμέρας τῆς γῆς, σπέρμα καὶ θερισμὸς, ψύχος καὶ καῦμα, θέρος καὶ ἔαρ, ἡμέραν καὶ νύκτα, οὐ καταπαύσουσι.

\ch{9}
Καὶ εὐλόγησεν ὁ Θεὸς τὸν Νῶε, καὶ τοὺς υἱοὺς αὐτοῦ· καὶ εἶπεν αὐτοῖς· αὐξάνεσθε καὶ πληθύνεσθε, καὶ πληρώσατε τὴν γῆν, καὶ κατακυριεύσατε αὐτῆς.
\vs{2}Καὶ ὁ τρόμος, καὶ ὁ φόβος ὑμῶν, ἔσται ἐπὶ πᾶσι τοῖς θηρίοις τῆς γῆς, ἐπὶ πάντα τὰ πετεινὰ τοῦ οὐρανοῦ, καὶ ἐπὶ πάντα τὰ κινούμενα ἐπὶ τῆς γῆς, καὶ ἐπὶ πάντας τοὺς ἰχθύας τῆς θαλάσσης· ὑπὸ χεῖρας ὑμῖν δέδωκα.
\vs{3}Καὶ πᾶν ἑρπετὸν, ὅ ἐστι ζῶν, ὑμῖν ἔσται εἰς βρῶσιν· ὡς λάχανα χόρτου δέδωκα ὑμῖν τὰ πάντα.
\vs{4}Πλὴν κρέας ἐν αἵματι ψυχῆς οὐ φάγεσθε.
\vs{5}Καὶ γὰρ τὸ ὑμέτερον αἷμα τῶν ψυχῶν ὑμῶν ἐκ χειρὸς πάντων τῶν θηρίων ἐκζητήσω αὐτό· καὶ ἐκ χειρὸς ἀνθρώπου ἀδελφοῦ ἐκζητήσω τὴν ψυχὴν τοῦ ἀνθρώπου.
\vs{6}Ὁ ἐκχέων αἷμα ἀνθρώπου, ἀντὶ τοῦ αἵματος αὐτοῦ ἐκχυθήσεται, ὅτι ἐν εἰκόνι Θεοῦ ἐποίησα τὸν ἄνθρωπον.
\vs{7}Ὑμεῖς δὲ αὐξάνεσθε, καὶ πληθύνεσθε, καὶ πληρώσατε τὴν γῆν, καὶ κατακυριεύσατε αὐτῆς.

\vs{8}Καὶ εἶπεν ὁ Θεὸς τῷ Νῶε καὶ τοῖς υἱοῖς αὐτοῦ, μετʼ αὐτοῦ λέγων,
\vs{9}καὶ ἰδοὺ ἐγὼ ἀνίστημι τὴν διαθήκην μου ὑμῖν, καὶ τῷ σπέρματι ὑμῶν μεθʼ ὑμᾶς,
\vs{10}καὶ πάσῃ ψυχῇ ζώσῃ μεθʼ ὑμῶν, ἀπὸ ὀρνέων, καὶ ἀπὸ κτηνῶν· καὶ πᾶσι τοῖς θηρίοις τῆς γῆς, ὅσα ἐστὶ μεθʼ ὑμῶν ἀπὸ πάντων τῶν ἐξελθόντων ἐκ τῆς κιβωτοῦ.
\vs{11}Καὶ στήσω τὴν διαθήκην μου πρὸς ὑμᾶς· καὶ οὐκ ἀποθανεῖται πᾶσα σὰρξ ἔτι ἀπὸ τοῦ ὕδατος τοῦ κατακλυσμοῦ· καὶ οὐκ ἔτι ἔσται κατακλυσμὸς ὕδατος, καταφθεῖραι πᾶσαν τὴν γῆν.
\vs{12}Καὶ εἶπε Κύριος ὁ Θεὸς πρὸς Νῶε· τοῦτο τὸ σημεῖον τῆς διαθήκης, ὃ ἐγὼ δίδωμι ἀνὰ μέσον ἐμοῦ καὶ ὑμῶν, καὶ ἀνὰ μέσον πάσης ψυχῆς ζώσης, ἥ ἐστι μεθʼ ὑμῶν εἰς γενεὰς αἰωνίους.
\vs{13}Τὸ τόξον μου τίθημι ἐν τῇ νεφέλῃ, καὶ ἔσται εἰς σημεῖον διαθήκης ἀνὰ μέσον ἐμοῦ καὶ τῆς γῆς.
\vs{14}Καὶ ἔσται ἐν τῷ συννεφεῖν με νεφέλας ἐπὶ τὴν γῆν, ὀφθήσεται τὸ τόξον ἐν τῇ νεφέλῃ.
\vs{15}Καὶ μνησθήσομαι τῆς διαθήκης μου, ἥ ἐστιν ἀνὰ μέσον ἐμοῦ καὶ ὑμῶν, καὶ ἀνὰ μέσον πάσης ψυχῆς ζώσης ἐν πάσῃ σαρκί· καὶ οὐκ ἔσται ἔτι τὸ ὕδωρ εἰς κατακλυσμὸν, ὥστε ἐξαλεῖψαι πᾶσαν σάρκα.
\vs{16}Καὶ ἔσται τὸ τόξον μου ἐν τῇ νεφέλῃ· καὶ ὄψομαι τοῦ μνησθῆναι διαθήκην αἰώνιον ἀνὰ μέσον ἐμοῦ καὶ τῆς γῆς, καὶ ἀνὰ μέσον ψυχῆς ζώσης ἐν πάσῃ σαρκὶ, ἥ ἐστιν ἐπὶ τῆς γῆς.
\vs{17}Καὶ εἶπεν ὁ Θεὸς τῷ Νῶε, τοῦτο τὸ σημεῖον τῆς διαθήκης, ἧς διεθέμην ἀνὰ μέσον ἐμοῦ, καὶ ἀνὰ μέσον πάσης σαρκὸς, ἥ ἐστιν ἐπὶ τῆς γῆς.

\vs{18}Ἦσαν δὲ οἱ υἱοὶ Νῶε, οἱ ἐξελθόντες ἐκ τῆς κιβωτοῦ, Σὴμ, Χὰμ, Ἰάφεθ. Χὰμ δὲ ἦν πατὴρ Χαναάν.
\vs{19}Τρεῖς οὗτοί εἰσιν υἱοὶ Νῶε· ἀπὸ τούτων διεσπάρησαν ἐπὶ πᾶσαν τὴν γῆν.
\vs{20}Καὶ ἤρξατο Νῶε ἄνθρωπος γεωργὸς γῆς, καὶ ἐφύτευσεν ἀμπελῶνα.
\vs{21}Καὶ ἔπιεν ἐκ τοῦ οἴνου, καὶ ἐμεθύσθη, καὶ ἐγυμνώθη ἐν τῷ οἴκῳ αὐτοῦ.
\vs{22}Καὶ εἶδε Χὰμ ὁ πατὴρ Χαναὰν τὴν γύμνωσιν τοῦ πατρὸς αὐτοῦ, καὶ ἐξελθὼν ἀνήγγειλε τοῖς δυσὶν ἀδελφοῖς αὐτοῦ ἔξω.
\vs{23}Καὶ λαβόντες Σὴμ καὶ Ἰάφεθ τὸ ἱμάτιον, ἐπέθεντο ἐπὶ τὰ δύο νῶτα αὐτῶν, καὶ ἐπορεύθησαν ὀπισθοφανῶς, καὶ συνεκάλυψαν τὴν γύμνωσιν τοῦ πατρὸς αὐτῶν· καὶ τὸ πρόσωπον αὐτῶν ὀπισθοφανῶς, καὶ τὴν γύμνωσιν τοῦ πατρὸς αὐτῶν οὐκ εἶδον.
\vs{24}Ἐξένηψε δὲ Νῶε ἀπὸ τοῦ οἴνου, καὶ ἔγνω ὅσα ἐποίησεν αὐτῷ ὁ υἱὸς αὐτοῦ ὁ νεώτερος.
\vs{25}Καὶ εἶπεν, ἐπικατάρατος Χαναὰν παῖς· οἰκέτης ἔσται τοῖς ἀδελφοῖς αὐτοῦ.
\vs{26}Καὶ εἶπεν, εὐλογητὸς Κύριος ὁ Θεὸς τοῦ Σήμ· καὶ ἔσται Χαναὰν παῖς οἰκέτης αὐτοῦ.
\vs{27}Πλατύναι ὁ Θεὸς τῷ Ἰάφεθ, καὶ κατοικησάτω ἐν τοῖς οἴκοις τοῦ Σήμ· καὶ γενηθήτω Χαναὰν παῖς αὐτοῦ.

\vs{28}Ἔζησε δὲ Νῶε μετὰ τὸν κατακλυσμὸν ἔτη τριακόσια πεντήκοντα.
\vs{29}Καὶ ἐγένοντο πᾶσαι αἱ ἡμέραι Νῶε ἐννακόσια πεντήκοντα ἔτη· καὶ ἀπέθανεν.

\ch{10}
Αὗται δὲ αἱ γενέσεις τῶν υἱῶν Νῶε, Σὴμ, Χὰμ, Ἰάφεθ· καὶ ἐγεννήθησαν αὐτοῖς υἱοὶ μετὰ τὸν κατακλυσμόν.

\vs{2}Υἱοὶ Ἰάφεθ, Γαμὲρ, καὶ Μαγὼγ, καὶ Μαδοὶ, καὶ Ἰωύαν, καὶ Ἐλισὰ, καὶ Θοβὲλ, καὶ Μοσὸχ, καὶ Θείρας.
\vs{3}Καὶ υἱοὶ Γαμὲρ, Ἀσχανὰζ, καὶ Ῥιφὰθ, καὶ Θοργαμά.
\vs{4}Καὶ υἱοὶ Ἰωύαν, Ἐλισὰ, καὶ Θάρσεις, Κήτιοι, Ῥόδὶοι.
\vs{5}Ἐκ τούτων ἀφωρίσθησαν νῆσοι τῶν ἐθνῶν ἐν τῇ γῇ αὐτῶν· ἕκαστος κατὰ γλῶσσαν ἐν ταῖς φυλαῖς αὐτῶν, καὶ ἐν τοῖς ἔθνεσιν αὐτῶν.

\vs{6}Υἱοὶ δὲ Χὰμ, Χοὺς, καὶ Μεσραῒν, Φοὺδ, καὶ Χαναάν.
\vs{7}Υἱοὶ δὲ Χοὺς, Σαβὰ, καὶ Εὐϊλὰ, καὶ Σαβαθὰ, καὶ Ῥεγμὰ, καὶ Σαβαθακά· υἱοὶ δὲ Ῥεγμὰ, Σαβὰ, καὶ Δαδάν.
\vs{8}Χοὺς δὲ ἐγέννησε τὸν Νεβρώδ· οὗτος ἤρξατο εἶναι γίγας ἐπὶ τῆς γῆς.
\vs{9}Οὗτος ἦν γίγας κυνηγὸς ἐναντίον Κυρίου τοῦ Θεοῦ· διὰ τοῦτο ἐροῦσιν, ὡς Νεβρὼδ γίγας κυνηγὸς ἐναντίον Κυρίου.
\vs{10}Καὶ ἐγένετο ἀρχὴ τῆς βασιλείας αὐτοῦ Βαβυλὼν, καὶ Ὀρὲχ, καὶ Ἀρχὰδ, καὶ Χαλάννη, ἐν τῇ γῇ Σεναάρ.
\vs{11}Ἐκ τῆς γῆς ἐκείνης ἐξῆλθεν Ἀσσούρ· καὶ ᾠκοδόμησε τὴν Νινευῒ, καὶ τὴν Ῥοωβὼθ πόλιν, καὶ τὴν Χαλὰχ,
\vs{12}καὶ τὴν Δασὴ ἀνὰ μέσον Νινευῒ, καὶ ἀνὰ μέσον Χαλάχ· αὕτη ἡ πόλις μεγάλη.
\vs{13}Καὶ Μεσραῒν ἐγέννησε τοὺς Λουδιεὶμ, καὶ τοὺς Νεφθαλεὶμ, καὶ τοὺς Ἐνεμετιεὶμ, καὶ τοὺς Λαβιεὶμ,
\vs{14}καὶ τοὺς Πατροσωνιεὶμ, καὶ τοὺς Χασμωνιεὶμ, ὅθεν ἐξῆλθε Φυλιστιεὶμ, καὶ τοὺς Γαφθοριείμ.
\vs{15}Χαναὰν δὲ ἐγέννησε τὸν Σιδῶνα πρωτότοκον αὐτοῦ, καὶ τὸν Χετταῖον,
\vs{16}καὶ τὸν Ἰεβουσαῖον, καὶ τὸν Ἀμοῤῥαῖον, καὶ τὸν Γεργεσαῖον,
\vs{17}καὶ τὸν Εὐαῖον, καὶ τὸν Ἀρουκαῖον, καὶ τὸν Ἀσενναῖον,
\vs{18}καὶ τον Ἀράδιον, καὶ τὸν Σαμαραῖον, καὶ τὸν Ἀμαθί. Καὶ μετὰ τοῦτο διεσπάρησαν αἱ φυλαὶ τῶν Χαναναίων.
\vs{19}Καὶ ἐγένετο τὰ ὅρια τῶν Χαναναίων ἀπὸ Σιδῶνος ἕως ἐλθεῖν εἰς Γεραρὰ καὶ Γαζὰν, ἕως ἐλθεῖν ἕως Σοδόμων καὶ Γομόῤῥας, Ἀδαμὰ καὶ Σεβωῒμ ἕως Δασά.
\vs{20}Οὗτοι υἱοὶ Χὰμ, ἐν ταῖς φυλαῖς αὐτῶν, κατὰ γλώσσας αὐτῶν, ἐν ταῖς χώραις αὐτῶν, καὶ ἐν τοῖς ἔθνεσιν αὐτῶν.

\vs{21}Καὶ τῷ Σὴμ ἐγεννήθη καὶ αὐτῷ πατρὶ πάντων τῶν υἱῶν Ἕβερ, ἀδελφῷ Ἰάφεθ τοῦ μείζονος.
\vs{22}Υἱοὶ Σὴμ, Ἐλὰμ, καὶ Ἀσσοὺρ, καὶ Ἀρφαξὰδ, καὶ Λοὺδ, καὶ Ἀρὰμ, καὶ Καϊνᾶν.
\vs{23}Καὶ υἱοὶ Ἀρὰμ, Οὒζ, καὶ Οὒλ, καὶ Γατὲρ, καὶ Μοσόχ.
\vs{24}Καὶ Ἀρφαξὰδ ἐγέννησε τὸν Καϊνᾶν, καὶ Καϊνᾶν ἐγέννησε τὸν Σαλά· Σαλὰ δὲ ἐγέννησε τὸν Ἕβερ.
\vs{25}Καὶ τῷ Ἕβερ ἐγεννήθησαν δύο υἱοί· ὄνομα τῷ ἑνὶ, Φαλὲγ, ὅτι ἐν ταῖς ἡμέραις αὐτοῦ διεμερίσθη ἡ γῆ· καὶ ὄνομα τῷ ἀδελφῷ αὐτοῦ Ἰεκτάν.
\vs{26}Ἰεκτὰν δὲ ἐγέννησε τὸν Ἐλμωδὰδ, καὶ Σαλὲθ, καὶ τὸν Σαρμὼθ, καὶ Ἰαρὰχ,
\vs{27}καὶ Ὁδοῤῥὰ, καὶ Αἰβὴλ, καὶ Δεκλὰ, καὶ Εὐὰλ,
\vs{28}καὶ Ἀβιμαὲλ, καὶ Σαβὰ,
\vs{29}καὶ Οὐφεὶρ, καὶ Εὑεϊλὰ, καὶ Ἰωβάβ· πάντες οὗτοι υἱοὶ Ἰεκτάν.
\vs{30}Καὶ ἐγένετο ἡ κατοίκησις αὐτῶν, ἀπὸ Μασσῆ ἕως ἐλθεῖν εἰς Σαφηρὰ ὄρος ἀνατολῶν.
\vs{31}Οὗτοι υἱοὶ Σὴμ, ἐν ταῖς φυλαῖς αὐτῶν, κατὰ γλώσσας αὐτῶν, ἐν ταῖς χώραις αὐτῶν, καὶ ἐν τοῖς ἔθνεσιν αὐτῶν.
\vs{32}Αὗται αἱ φυλαὶ υἱῶν Νῶε κατὰ γενέσεις αὐτῶν, κατὰ ἔθνη αὐτῶν· ἀπὸ τούτων διεσπάρησαν νῆσοι τῶν ἐθνῶν ἐπὶ τῆς γῆς μετὰ τὸν κατακλυσμόν.

\ch{11}
Καὶ ἦν πᾶσα ἡ γῆ χεῖλος ἓν, καὶ φωνὴ μία πᾶσι.
\vs{2}Καὶ ἐγένετο ἐν τῷ κινῆσαι αὐτοὺς ἀπὸ ἀνατολῶν, εὗρον πεδίον ἐν γῇ Σεναὰρ, καὶ κατῴκησαν ἐκεῖ.
\vs{3}Καὶ εἶπεν ἄνθρωπος τῷ πλησίον αὐτοῦ, δεῦτε πλινθεύσωμεν πλίνθους, καὶ ὀπτήσωμεν αὐτὰς πυρί· καὶ ἐγένετο αὐτοῖς ἡ πλίνθος εἰς λίθον, καὶ ἄσφαλτος ἦν αὐτοῖς ὁ πηλός.
\vs{4}Καὶ εἶπαν, δεῦτε οἰκοδομήσωμεν ἑαυτοῖς πόλιν καὶ πύργον, οὗ ἔσται ἡ κεφαλὴ ἕως τοῦ οὐρανοῦ, καὶ ποιήσωμεν ἑαυτοῖς ὄνομα, πρὸ τοῦ διασπαρῆναι ἡμᾶς ἐπὶ προσώπου πάσης τῆς γῆς.
\vs{5}Καὶ κατέβη Κύριος ἰδεῖν τὴν πόλιν καὶ τὸν πύργον, ὃν ᾠκοδόμησαν οἱ υἱοὶ τῶν ἀνθρώπων.
\vs{6}Καὶ εἶπε Κύριος, ἰδοὺ γένος ἓν, καὶ χεῖλος ἓν πάντων, καὶ τοῦτο ἤρξαντο ποιῆσαι, καὶ νῦν οὐκ ἐκλείψει ἀπʼ αὐτῶν πάντα ὅσα ἂν ἐπιθῶνται ποιεῖν.
\vs{7}Δεῦτε, καὶ καταβάντες συγχέωμεν αὐτῶν ἐκεῖ τὴν γλῶσσαν, ἵνα μὴ ἀκούσωσιν ἕκαστος τὴν φωνὴν τοῦ πλησίον.
\vs{8}Καὶ διέσπειρεν αὐτοὺς Κύριος ἐκεῖθεν ἐπὶ πρόσωπον πάσης τῆς γῆς· καὶ ἐπαύσαντο οἰκοδομοῦντες τῆν πόλιν καὶ τὸν πύργον.
\vs{9}Διὰ τοῦτο ἐκλήθη τὸ ὄνομα αὐτῆς, Σύγχυσις, ὅτι ἐκεῖ συνέχεε Κύριος τὰ χείλη πάσης τῆς γῆς, καὶ ἐκεῖθεν διέσπειρεν αὐτοὺς Κύριος ἐπὶ πρόσωπον πάσης τῆς γῆς.

\vs{10}Καὶ αὗται αἱ γενέσεις Σήμ· καὶ ἦν Σὴμ υἱὸς ἑκατὸν ἐτῶν, ὅτε ἐγέννησε τὸν Ἀρφαξὰδ, δευτέρου ἔτους μετὰ τὸν κατακλυσμόν.
\vs{11}Καὶ ἔζησε Σὴμ, μετὰ τὸ γεννῆσαι αὐτὸν τὸν Ἀρφαξὰδ, ἔτη πεντακόσια, καὶ ἐγέννησεν υἱοὺς καὶ θυγατέρας, καὶ ἀπέθανε.
\vs{12}Καὶ ἔζησεν Ἀρφαξὰδ ἑκατὸν τριακονταπέντε ἔτη, καὶ ἐγέννησε τὸν Καϊνᾶν.
\vs{13}Καὶ ἔζησεν Ἀρφαξὰδ, μετὰ τὸ γεννῆσαι αὐτὸν τὸν Καϊνᾶν, ἔτη τετρακόσια, καὶ ἐγέννησεν υἱοὺς καὶ θυγατέρας, καὶ ἀπέθανε. Καὶ ἔζησε Καϊνᾶν ἑκατὸν καὶ τριάκοντα ἔτη, καὶ ἐγέννησε τὸν Σαλά· καὶ ἔξησε Καϊνᾶν, μετὰ τὸ γεννῆσαι αὐτὸν τὸν Σαλὰ, ἔτη τριακόσια τριάκοντα, καὶ ἐγέννησεν υἱοὺς καὶ θυγατέρας, καὶ ἀπέθανε.
\vs{14}Καὶ ἔζησε Σαλὰ ἑκατὸν τριάκοντα ἔτη, καὶ ἐγέννησε τὸν Ἕβερ.
\vs{15}Καὶ ἔζησε Σαλὰ μετὰ τὸ γεννῆσαι αὐτὸν τὸν Ἕβερ, τριακόσια τριάκοντα ἔτη, καὶ ἐγέννησεν υἱοὺς καὶ θυγατέρας· καὶ ἀπέθανε.
\vs{16}Καὶ ἔζησεν Ἕβερ ἑκατὸν τριάκοντα τέσσαρα ἔτη, καὶ ἐγέννησε τὸν Φαλέγ.
\vs{17}Καὶ ἔξησεν Ἕβερ, μετὰ τὸ γεννῆσαι αὐτὸν τὸν Φαλὲγ, ἔτη διακόσια ἑβδομήκοντα, καὶ ἐγέννησεν υἱοὺς καὶ θυγατέρας, καὶ ἀπέθανε.
\vs{18}Καὶ ἔζησε Φαλὲγ τριάκοντα καὶ ἑκατὸν ἔτη, καὶ ἐγέννησε τὸν Ῥαγαῦ.
\vs{19}Καὶ ἔζησε Φαλὲγ, μετὰ τὸ γεννῆσαι αὐτὸν τὸν Ῥαγαῦ, ἐννέα καὶ διακόσια ἔτη, καὶ ἐγέννησεν υἱοὺς και θυγατέρας, καὶ ἀπέθανε.
\vs{20}Καὶ ἔζησε Ῥαγαὺ ἑκατὸν τριάκοντα καὶ δύο ἔτη, καὶ ἐγέννησε τὸν Σερούχ.
\vs{21}Καὶ ἔζησε Ῥαγαῦ, μετὰ τὸ γεννῆσαι αὐτὸν τὸν Σεροὺχ, διακόσια ἑπτὰ ἔτη, καὶ ἐγέννησεν υἱοὺς καὶ θυγατέρας, καὶ ἀπέθανε.
\vs{22}Καὶ ἔζησε Σεροὺχ ἑκατὸν τριάκοντα ἔτη, καὶ ἐγέννησε τὸν Ναχώρ.
\vs{23}Καὶ ἔζησε Σεροὺχ, μετὰ τὸ γεννῆσαι αὐτὸν τὸν Ναχὼρ, ἔτη διακόσια, καὶ ἐγέννησεν υἱοὺς καὶ θυγατέρας, καὶ ἀπέθανε.
\vs{24}Καὶ ἔζησε Ναχὼρ ἔτη ἑκατὸν ἑβδομηκονταεννέα, καὶ ἐγέννησε τὸν Θάῤῥα.
\vs{25}Καὶ ἔζησε Ναχὼρ, μετὰ τὸ γεννῆσαι αὐτὸν τὸν Θάῤῥα, ἔτη ἑκατὸν εἰκοσιπὲντε, καὶ ἐγεννησεν υἱοὺς καὶ θυγατέρας, καὶ ἀπέθανε.
\vs{26}Καὶ ἔζησε Θάῤῥα ἑβδομήκοντα ἔτη, καὶ ἐγέννησε τὸν Ἄβραμ, καὶ τὸν Ναχὼρ, καὶ τὸν Ἀῤῥάν.

\vs{27}Αὗται δὲ αἱ γενέσεις Θάῤῥα· Θάῤῥα ἐγέννησε τὸν Ἅβραμ, καὶ τὸν Ναχὼρ, καὶ τὸν Ἀῤῥάν· καὶ Ἀῤῥὰν ἐγέννησε τὸν Λώτ.
\vs{28}Καὶ ἀπέθανεν Ἀῤῥὰν ἐνώπιον Θάῤῥα τοῦ πατρὸς αὐτοῦ ἐν τῇ γῇ ᾗ ἐγενήθη, ἐν τῇ χώρᾳ τῶν Χαλδαίων.
\vs{29}Καὶ ἔλαβον Ἅβραμ καὶ Ναχὼρ ἑαυτοῖς γυναῖκας· ὄνομα τῇ γυναικὶ Ἅβραμ, Σάρα, καὶ ὄνομα τῇ γυναικὶ Ναχὼρ, Μελχά, θυγάτηρ Ἀῤῥάν· καὶ πατὴρ Μελχὰ, καὶ πατὴρ Ἰεσχά.
\vs{30}Καὶ ἦν Σάρα στεῖρα, καὶ οὐκ ἐτεκνοποίει.
\vs{31}Καὶ ἔλαβε Θάῤῥα τὸν Ἅβραμ υἱὸν αὐτοῦ, καὶ τὸν Λὼτ υἱὸν Ἀῤῥάν, υἱὸν τοῦ υἱοῦ αὐτοῦ, καὶ τὴν Σάραν τὴν νύμφην αὐτοῦ, γυναῖκα Ἅβραμ τοῦ υἱοῦ αὐτοῦ, καὶ ἐξήγαγεν αὐτοὺς ἐκ τῆς χώρας τῶν Χαλδαίων, πορευθῆναι εἰς γῆν Χαναάν· καὶ ἦλθον ἕως Χαῤῥὰν, καὶ κατῴκησεν ἐκεῖ.
\vs{32}Καὶ ἐγένοντο πᾶσαι αἱ ἡμέραι Θάῤῥα ἐν γῇ Χαῤῥὰν, διακόσια πέντε ἔτη· καὶ ἀπέθανε Θάῤῥα ἐν Χαῤῥάν.

\ch{12}
Καὶ εἶπε Κύριος τῷ Ἅβραμ, ἔξελθε ἐκ τῆς γῆς σου, καὶ ἐκ τῆς συγγενείας σου, καὶ ἐκ τοῦ οἴκου τοῦ πατρός σου, καὶ δεῦρο εἰς τὴν γῆν, ἣν ἄν σοι δείξω.
\vs{2}Καὶ ποιήσω σε εἰς ἔθνος μέγα, καὶ εὐλογήσω σε, καὶ μεγαλυνῶ τὸ ὄνομά σου, καὶ ἔσῃ εὐλογημένος.
\vs{3}Καὶ εὐλογήσω τοὺς εὐλογοῦντάς σε, καὶ τοὺς καταρωμένους σε καταράσομαι, καὶ ἐνευλογηθήσονται ἐν σοὶ πᾶσαι αἱ φυλαὶ τῆς γῆς.
\vs{4}Καὶ ἐπορεύθη Ἅβραμ, καθάπερ ἐλάλησεν αὐτῷ Κύριος, καὶ ᾤχετο μετʼ αὐτοῦ Λώτ· Ἅβραμ δὲ ἦν ἐτῶν ἑβδομηκονταπέντε, ὅτε ἐξῆλθεν ἐκ Χαῤῥάν.
\vs{5}Καὶ ἔλαβεν Ἅβραμ Σάραν τὴν γυναῖκα αὐτοῦ, καὶ τὸν Λὼτ υἱὸν τοῦ ἀδελφοῦ αὐτοῦ, καὶ πάντα τὰ ὑπάρχοντα αὐτῶν ὅσα ἐκτήσαντο, καὶ πᾶσαν ψυχὴν ἣν ἐκτήσαντο, ἐκ Χαῤῥάν, καὶ ἐξήλθοσαν πορευθῆναι εἰς γῆν Χανάαν.
\vs{6}Καὶ διώδευσεν Ἅβραμ τὴν γῆν εἰς τὸ μῆκος αὐτῆς ἕως τοῦ τόπου Συχέμ, ἐπὶ τὴν δρῦν τὴν ὑψηλήν· οἱ δὲ Χαναναῖοι τότε κατῴκουν τὴν γῆν.
\vs{7}Καὶ ὤφθη Κύριος τῷ Ἅβραμ, καὶ εἶπεν αὐτῷ, τῷ σπέρματί σου δώσω τὴν γῆν ταύτην· καὶ ᾠκοδόμησεν ἐκεῖ Ἅβραμ θυσιαστήριον Κυρίῳ τῷ ὀφθέντι αὐτῷ.
\vs{8}Καὶ ἀπέστη ἐκεῖθεν εἰς τὸ ὄρος κατὰ ἀνατολὰς Βαιθήλ· καὶ ἔστησεν ἐκεῖ τὴν σκηνὴν αὐτοῦ ἐν Βαιθὴλ κατὰ θάλασσαν, καὶ Ἀγγαὶ κατὰ ἀνατολάς· καὶ ᾠκοδόμησεν ἐκεῖ θυσιαστήριον τῷ Κυρίῳ, καὶ ἐπεκαλέσατο ἐπὶ τῷ ὀνόματι Κυρίου.
\vs{9}Καὶ ἀπῇρεν Ἅβραμ, καὶ πορευθεὶς ἐστρατοπέδευσεν ἐν τῇ ἐρήμῳ.

\vs{10}Καὶ ἐγένετο λιμὸς ἐπὶ τῆς γῆς· καὶ κατέβη Ἅβραμ εἰς Αἴγυπτον παροικῆσαι ἐκεῖ, ὅτι ἐνίσχυσεν ὁ λιμὸς ἐπὶ τῆς γῆς.
\vs{11}Ἐγένετο δὲ ἡνίκα ἤγγισεν Ἅβραμ εἰσελθεῖν εἰς Αἴγυπτον, εἶπεν Ἅβραμ Σάρα τῇ γυναικὶ, γινώσκω ἐγὼ, ὅτι γυνὴ εὐπρόσωπος εἶ.
\vs{12}Ἔσται οὖν ὡς ἂν ἴδωσί σε οἱ Αἰγύπτιοι, ἐροῦσιν ὅτι γυνὴ αὐτοῦ ἐστιν αὐτὴ, καὶ ἀποκτενοῦσί με, σὲ δὲ περιποιήσονται.
\vs{13}Εἶπον οὖν, ὅτι ἀδελφὴ αὐτοῦ εἰμι, ὅπως ἄν εὖ μοι γένηται διὰ σὲ, καὶ ζήσεται ἡ ψυχή μου ἕνεκέν σου.
\vs{14}Ἐγένετο δὲ, ἡνίκα εἰσῆλθεν Ἅβραμ εἰς Αἴγυπτον, ἰδόντες οἱ Αἰγύπτιοι τὴν γυναῖκα αὐτοῦ, ὅτι καλὴ ἦν σφόδρα.
\vs{15}Καὶ ἴδον αὐτὴν οἱ ἄρχοντες Φαραὼ, καὶ ἐπῄνεσαν αὐτὴν πρὸς Φαραὼ, καὶ εἰσήγαγον αὐτὴν εἰς τὸν οἶκον Φαραώ.
\vs{16}Καὶ τῷ Ἅβραμ εὖ ἐχρήσαντο διʼ αὐτήν· καὶ ἐγένοντο αὐτῷ πρόβατα, καὶ μόσχοι, καὶ ὄνοι, καὶ παῖδες, καὶ παιδίσκαι, καὶ ἡμίονοι, καὶ κάμηλοι.
\vs{17}Καὶ ἤτασεν ὁ Θεὸς τὸν Φαραὼ ἐτασμοῖς μεγάλοις καὶ πονηροῖς, καὶ τὸν οἶκον αὐτοῦ, περὶ Σάρας τῆς γυναικὸς Ἅβραμ.
\vs{18}Καλέσας δὲ Φαραὼ τὸν Ἅβραμ, εἶπεν, τί τοῦτο ἐποίησάς μοι, ὅτι οὐκ ἀπήγγειλάς μοι, ὅτι γυνή σου ἐστίν;
\vs{19}Ἱνατί εἶπας ὅτι ἀδελφή μου ἐστίν; καὶ ἔλαβον αὐτὴν ἐμαυτῷ γυναῖκα· καὶ νῦν ἰδοὺ ἡ γυνή σου ἔναντί σου, λαβὼν ἀπότρεχε.
\vs{20}Καὶ ἐνετείλατο Φαραὼ ἀνδράσι περὶ Ἅβραμ συμπροπέμψαι αὐτὸν, καὶ τὴν γυναῖκα αὐτοῦ, καὶ πάντα ὅσα ἦν αὐτῷ.

\ch{13}
Ἀνέβη δὲ Ἅβραμ ἐξ Αἰγύπτου αὐτὸς, καὶ ἡ γυνὴ αὐτοῦ, καὶ πάντα τὰ αὐτοῦ, καὶ Λὼτ μετʼ αὐτοῦ, εἰς τὴν ἔρημον.
\vs{2}Ἅβραμ δὲ ἦν πλούσιος σφόδρα κτήνεσι, καὶ ἀργυρίῳ, καὶ χρυσίῳ.
\vs{3}Καὶ ἐπορεύθη ὅθεν ἦλθεν εἰς τὴν ἔρημον ἕως Βαιθὴλ, ἕως τοῦ τόπου οὗ ἦν ἡ σκηνὴ αὐτοῦ τὸ πρότερον, ἀνὰ μέσον Βαιθὴλ καὶ ἀνὰ μέσον Ἀγγαί,
\vs{4}εἰς τὸν τόπον τοῦ θυσιαστηρίου, οὗ ἐποίησεν ἐκεῖ τὴν ἀρχὴν, καὶ ἐπεκαλέσατο ἐκεῖ Ἅβραμ τὸ ὄνομα τοῦ Κυρίου.
\vs{5}Καὶ Λὼτ τῷ συμπορευομένῳ μετὰ Ἅβραμ ἦν πρόβατα, καὶ βόες, καὶ σκηναί.
\vs{6}Καὶ οὐκ ἐχώρει αὐτοὺς ἡ γῆ κατοικεῖν ἅμα, ὅτι ἦν τὰ ὑπάρχοντα αὐτῶν πολλά· καὶ οὐκ ἐχώρει αὐτοὺδ ἡ γῆ κατοικεῖν ἅμα.
\vs{7}Καὶ ἐγενετο μάχη ἀνὰ μέσον τῶν ποιμένων τῶν κτηνῶν τοῦ Ἅβραμ, καὶ ἀνὰ μέσον τῶν ποιμένων τῶν κτηνῶν τοῦ Λώτ· οἱ δὲ Χαναναῖοι καὶ οἱ Φερεζαῖοι τότε κατῴκουν τὴν γῆν.
\vs{8}Εἶπε δὲ Ἅβραμ τῷ Λὼτ, μὴ ἔστω μάχη ἀνὰ μέσον ἐμοῦ καὶ σοῦ, καὶ ἀνὰ μέσον τῶν ποιμένων μου καὶ ἀνὰ μέσον τῶν ποιμένων σοῦ, ὅτι ἄνθρωποι ἀδελφοὶ ἐσμὲν ἡμεῖς.
\vs{9}Οὐκ ἰδοὺ πᾶσα ἡ γῆ ἐναντίον σου ἐστί; διαχωρίσθητι ἀπʼ ἐμοῦ· εἰ σὺ εἰς ἀριστερὰ, ἐγὼ εἰς δεξιά· εἰ δὲ σὺ εἰς δεξιὰ, ἐγὼ εἰς ἀριστερά.
\vs{10}Καὶ ἐπάρας Λὼτ τοὺς ὀφθαλμοὺς αὐτοῦ, ἐπεῖδε πᾶσαν τὴν περίχωρον τοῦ Ἰορδάνου, ὅτι πᾶσα ἦν ποτιζομένη, πρὸ τοῦ καταστρέψαι τὸν Θεὸν Σόδομα καὶ Γόμοῤῥα, ὡς ὁ παράδεισος τοῦ Θεοῦ, καὶ ὡς ἡ γῆ Αἰγύπτου, ἕως ἐλθεῖν εἰς Ζόγορα.
\vs{11}Καὶ ἐξελέξατο ἑαυτῷ Λὼτ πᾶσαν τὴν περίχωρον τοῦ Ἰορδάνου· καὶ ἀπῇρε Λὼτ ἀπὸ ἀνατολῶν· καὶ διεχωρίσθησαν ἕκαστος ἀπὸ τοῦ ἀδελφοῦ αὐτοῦ.
\vs{12}Ἅβραμ δὲ κατῴκησεν ἐν γῇ Χαναάν· Λὼτ δὲ κατῴκησεν ἐν πόλει τῶν περιχώρων, καὶ ἐσκήνωσεν ἐν Σοδόμοις.
\vs{13}Οἱ δὲ ἄνθρωποι οἱ ἐν Σοδόμοις πονηροὶ καὶ ἁμαρτωλοὶ ἐναντίον τοῦ Θεοῦ σφόδρα.
\vs{14}Ὁ δὲ Θεὸς εἶπε τῷ Ἅβραμ μετὰ τὸ διαχωρισθῆναι τὸν Λὼτ ἀπʼ αὐτοῦ, ἀνάβλεψον τοῖς ὀφθαλμοῖς σου, καὶ ἴδε ἀπὸ τοῦ τόπου οὗ νῦν σὺ εἶ πρὸς βοῤῥὰν καὶ λίβα καὶ ἀνατολὰς καὶ θάλασσαν·
\vs{15}ὅτι πᾶσαν τὴν γῆν, ἣν σὺ ὁρᾷς, σοὶ δώσω αὐτὴν καὶ τῷ σπέρματί σου ἕως αἰῶνος.
\vs{16}Καὶ ποιήσω τὸ σπέρμα σου, ὡς τὴν ἄμμον τῆς γῆς· εἰ δύναταί τις ἐξαριθμῆσαι τὴν ἄμμον τῆς γῆς, καὶ τὸ σπέρμα σου ἐξαριθμηθήσεται.
\vs{17}Ἀναστὰς διόδευσον τὴν γῆν εἴς τε τὸ μῆκος αὐτῆς καὶ εἰς τὸ πλάτος· ὅτι σοι δώσω αὐτὴν καὶ τῷ σπέρματί σου εἰς τὸν αἰῶνα.
\vs{18}Καὶ ἀποσκηνώσας Ἅβραμ, ἐλθὼν κατῴκησε παρὰ τὴν δρῦν τὴν Μαμβρῆ, ἣ ἦν ἐν Χεβρὼμ, καὶ ᾠκοδόμησεν ἐκεῖ θυσιαστήριον τῷ Κυρίῳ.

\ch{14}
Ἐγένετο δὲ ἐν τῇ βασιλείᾳ τῇ Ἀμαρφὰλ βασιλέως Σενναὰρ, καὶ Ἀριὼχ βασιλέως Ἑλλασὰρ, Χοδολλογομὸρ βασιλεὺς Ἐλὰμ, καὶ Θαργὰλ βασιλεὺς ἐθνῶν,
\vs{2}ἐποίησαν πόλεμον μετὰ Βαλλὰ βασιλέως Σοδόμων, καὶ μετὰ Βαρσὰ βασιλέως Γομόῤῥας, καὶ μετὰ Σενναὰρ βασιλέως Ἀδαμὰ, καὶ μετὰ Συμοβὸρ βασιλέως Σεβωεὶμ, καὶ βασιλέως Βαλάκ· αὕτη ἐστὶ Σηγώρ.
\vs{3}Πάντες οὗτοι συνεφώνησαν ἐπὶ τὴν φάραγγα τὴν ἁλυκήν· αὕτη ἡ θάλασσα τῶν ἁλῶν.
\vs{4}Δώδεκα ἔτη αὐτοὶ ἐδούλευσαν τῷ Χοδολλογομόρ· τῷ δὲ τρισκαιδεκάτῳ ἔτει ἀπέστησαν.
\vs{5}Ἐν δὲ τῷ τεσσαρεσκαιδεκάτῳ ἔτει ἦλθε Χοδολλογομὸρ καὶ οἱ βασιλεῖς μετʼ αὐτοῦ, καὶ κατέκοψαν τοὺς γίγαντας τοὺς ἐν Ἀσταρὼθ, καὶ Καρναῒν, καὶ ἔθνη ἰσχυρὰ ἅμα αὐτοῖς, καὶ τοὺς Ὀμμαίους τοὺς ἐν Σαυῇ τῇ πόλει.
\vs{6}Καὶ τοὺς Χοῤῥαίους τοὺς ἐν τοῖς ὄρεσι Σηεὶρ, ἕως τῆς τερεβίνθου τῆς Φαρὰν, ἥ ἐστιν ἐν τῇ ἐρήμῳ.
\vs{7}Καὶ ἀναστρέψαντες ἦλθον ἐπὶ τὴν πηγὴν τῆς κρίσεως· αὕτη ἐστὶ Κάδης· καὶ κατέκοψαν πάντας τοὺς ἄρχοντας Ἀμαλὴκ, καὶ τοὺς Ἀμοῤῥαίους τοὺς κατοικοῦντας ἐν ʼΑσασονθαμὰρ
\vs{8}Ἐξῆλθε δὲ βασιλεὺς Σοδόμων, καὶ βασιλεὺς Γομόῤῥας, καὶ βασιλεὺς Ἀδαμὰ, καὶ βασιλεὺς Σεβωεὶμ, καὶ βασιλεὺς Βαλάκ· αὕτη ἐστὶ Σηγώρ· καὶ παρετάξαντο αὐτοῖς εἰς πόλεμον ἐν τῇ κοιλάδι, τῇ ἁλυκῇ,
\vs{9}πρὸς Χοδολλογομὸρ βασιλέα Ἐλὰμ, καὶ Θαπγὰλ βασιλέα ἐθνῶν, καὶ Ἀμαρφὰλ βασιλέα Σενναὰρ, καὶ Ἀριὼχ βασιλέα Ἑλλασὰρ, οἱ τέσσαρες βασιλεῖς πρὸς τοὺς πέντε.
\vs{10}Ἡ δὲ κοιλὰς ἡ ἁλυκὴ, φρέατα ἀσφάλτου· ἔφυγε δὲ βασιλεὺς Σοδόμων καὶ βασιλεὺς Γομόῤῥας, καὶ ἐνέπεσαν ἐκεῖ· οἱ δὲ καταλειφθέντες εἰς τὴν ὀρεινὴν ἔφυγον.
\vs{11}Ἔλαβον δὲ τὴν ἵππον πᾶσαν τὴν Σοδόμων καὶ Γομόῤῥας, καὶ πάντα τὰ βρώματα αὐτῶν, καὶ ἀπῆλθον.
\vs{12}Ἔλαβον δὲ καὶ τὸν Λὼτ τὸν υἱὸν τοῦ ἀδελφοῦ Ἅβραμ, καὶ τὴν ἀποσκευὴν αὐτοῦ, καὶ ἀπῴχοντο· ἦν γὰρ κατοικῶν ἐν Σοδόμοις.

\vs{13}Παραγενόμενος δὲ τῶν ἀνασωθέντων τις ἀπήγγειλεν Ἅβραμ τῷ περάτῃ· αὐτὸς δὲ κατῴκει παρὰ τῇ δρυῒ τῇ Μαμβρῇ Ἀμοῤῥαίου τοῦ ἀδελφοῦ Ἐσχὼλ, καὶ τοῦ ἀδελφοῦ Αὐνὰν, οἳ ἦσαν συνωμόται τοῦ Ἅβραμ.
\vs{14}Ἀκούσας δὲ Ἅβραμ ὅτι ᾐχμαλώτευται Λὼτ ὁ ἀδελφοῦς αὐτοῦ, ἠρίθμησε τοὺς ἰδίους οἰκογενεῖς αὐτοῦ τριακοσίους δέκα καὶ ὀκτώ· καὶ κατεδίωξεν ὀπίσω αὐτῶν ἕως Δάν.
\vs{15}Καὶ ἐπέπεσεν ἐπʼ αὐτοὺς τὴν νύκτα αὐτὸς, καὶ οἱ παῖδες αὐτοῦ, καὶ ἐπάταξεν αὐτοὺς, καὶ κατεδίωξεν αὐτοὺς ἕως Χοβὰ, ἥ ἐστιν ἐν ἀριστερᾷ Δαμασκοῦ.
\vs{16}Καὶ ἀπέστρεψε πᾶσαν τὴν ἵππον Σοδόμων· καὶ Λὼτ τὸν ἀδελφιδοῦν αὐτοῦ ἀπέστρεψε, καὶ πάντα τὰ ὑπάρχοντα αὐτοῦ, καὶ τὰς γυναῖκας, καὶ τὸν λαόν.
\vs{17}Ἐξῆλθε δὲ βασιλεὺς Σοδόμων εἰς συνάντησιν αὐτῷ, μετὰ τὸ ὑποστρέψαι αὐτὸν ἀπὸ τῆς κοπῆς τοῦ Χοδολλογομὸρ, καὶ τῶν βασιλέων τῶν μετʼ αὐτοῦ εἰς τὴν κοιλάδα τοῦ Σαβύ· τοῦτο ἦν τὸ πεδίον τῶν βασιλέων.

\vs{18}Καὶ Μελχισεδὲκ βασιλεὺς Σαλὴμ ἐξήνεγκεν ἄρτους καὶ οἶνον· ἦν δὲ ἱερεὺς τοῦ Θεοῦ τοῦ ὑψίστου.
\vs{19}Καὶ εὐλόγησε τὸν Ἅβραμ, καὶ εἶπεν, εὐλογημένος Ἅβραμ τῷ Θεῷ τῷ ὑψίστῳ, ὃς ἔκτισε τὸν οὐρανὸν καὶ τὴν γῆν.
\vs{20}Καὶ εὐλογητὸς ὁ Θεὸς ὁ ὕψιστος, ὃς παρέδωκε τοὺς ἐχθρούς σου ὑποχειρίους σοι· καὶ ἔδωκεν αὐτῷ Ἅβραμ δεκάτην ἀπὸ πάντων.
\vs{21}Εἶπε δὲ βασιλεὺς Σοδόμων πρὸς Ἅβραμ, δός μοι τοὺς ἄνδρας, τὴν δὲ ἵππον λάβε σεαυτῷ.
\vs{22}Εἶπε δὲ Ἅβραμ πρὸς τὸν βασιλέα Σοδόμων, ἐκτενῶ τὴν χεῖρά μου πρὸς Κύπιον τὸν Θεὸν τὸν ὕψιστον, ὃς ἔκτισε τὸν οὐρανὸν καὶ τὴν γῆν,
\vs{23}εἰ ἀπὸ σπαρτίου ἕως σφυρωτῆρος ὑποδήματος λήψομαι ἀπὸ πάντων τῶν σῶν, ἵνα μὴ εἴπῃς, ὅτι ἐγὼ ἐπλούτισα τὸν Ἅβραμ.
\vs{24}Πλὴν ὧν ἔφαγον οἱ νεανίσκοι, καὶ τῆς μερίδος τῶν ἀνδρῶν τῶν συμπορευθέντων μετʼ ἐμοῦ Ἐσχὼλ, Αὐνᾶν, Μαμβρῆ· οὗτοι λήψονται μερίδα.

\ch{15}
Μετὰ δὲ τὰ ῥήματα ταῦτα ἐγενήθη ῥῆμα Κυρίου πρὸς Ἅβραμ ἐν ὁράματι, λέγων, μὴ φοβοῦ Ἅβραμ· ἐγὼ ὑπερασπίζω σου· ὁ μισθός σου πολὺς ἔσται σφόδρα.
\vs{2}Δέγει δὲ Ἅβραμ, Δέσποτα Κύριε, τί μοι δώσεις; ἐγὼ δὲ ἀπολύομαι ἄτεκνος· ὁ δὲ υἱὸς Μασὲκ τῆς οἰκογενοῦς μου, οὗτος Δαμασκὸς Ἐλιέζερ.
\vs{3}Καὶ εἶπεν Ἅβραμ, ἐπειδὴ ἐμοὶ οὐκ ἔδωκας σπέρμα, ὁ δὲ οἰκογενής μου κληρονομήσει με.
\vs{4}Καὶ εὐθὺς φωνὴ Κυρίου ἐγένετο πρὸς αὐτὸν, λέγουσα, οὐ κληρονομήσει σε οὗτος· ἀλλʼ ὃς ἐξελεύσεται ἐκ σοῦ, οὗτος κληρονομήσει σε.
\vs{5}Ἐξήγαγε δὲ αὐτὸν ἔξω, καὶ εἶπεν αὐτῷ, ἀνάβλεψον δὴ εἰς τὸν οὐρανὸν, καὶ ἀρίθμησον τοὺς ἀστέρας, εἰ δυνήσῃ ἐξαριθμῆσαι αὐτούς· καὶ εἶπεν, οὕτως ἔσται τὸ σπέρμα σου.
\vs{6}Καὶ ἐπίστευσεν Ἅβραμ τῷ Θεῷ, καὶ ἐλογίσθη αὐτῷ εἰς δικαιοσύνην.
\vs{7}Εἶπε δὲ πρὸς αὐτὸν, ἐγὼ ὁ Θεὸς ὁ ἐξαγαγών σε ἐκ χώρας Χαλδαίων, ὥστε δοῦναί σοι τὴν γῆν ταύτην κληρονομῆσαι.
\vs{8}Εἶπε δέ, Δέσποτα Κύριε, κατὰ τί γνώσομαι, ὅτι κληρονομήσω αὐτήν;
\vs{9}Εἶπε δὲ αὐτῷ, λάβε μοι δάμαλιν τριετίζουσαν, καὶ αἶγα τριετίζουσαν, καὶ κριὸν τριετίζοντα, καὶ τρυγόνα, καὶ περιστεράν.
\vs{10}Ἔλαβε δὲ αὐτῷ πάντα ταῦτα, καὶ διεῖλεν αὐτὰ μέσα, καὶ ἔθηκεν αὐτὰ ἀντιπρόσωπα ἀλλήλοις· τὰ δὲ ὄρνεα οὐ διεῖλε.
\vs{11}Κατέβη δὲ ὄρνεα ἐπὶ τὰ σώματα, ἐπὶ τὰ διχοτομήματα αὐτῶν· καὶ συνεκάθισεν αὐτοῖς Ἅβραμ.
\vs{12}Περὶ δὲ ἡλίου δυσμὰς ἔκστασις ἐπέπεσε τῷ Ἅβραμ, καὶ ἰδοὺ φόβος σκοτεινὸς μέγας ἐπιπίπτει αὐτῷ.
\vs{13}Καὶ ἐῤῥέθη πρὸς Ἅβραμ· γινώσκων γνώσῃ ὅτι πάροικον ἔσται τὸ σπέρμα σου ἐν γῇ οὐκ ἰδίᾳ, καὶ δουλώσουσιν αὐτοὺς, καὶ κακώσουσιν αὐτοὺς, καὶ ταπεινώσουσιν αὐτοὺς, τετρακόσια ἔτη.
\vs{14}Τὸ δὲ ἔθνος, ᾧ ἐὰν δουλεύσωσι, κρινῶ ἐγώ· μετὰ δὲ ταῦτα, ἐξελεύσονται ὧδε μετὰ ἀποσκευῆς πολλῆς.
\vs{15}Σὺ δὲ ἀπελεύσῃ πρὸς τοὺς πατέρας σου ἐν εἰρήνῃ, τραφεὶς ἐν γήρᾳ καλῷ.
\vs{16}Τετάρτῃ δὲ γενεᾷ ἀποστραφήσονται ὧδε· οὔπω γὰρ ἀναπεπλήρωνται αἱ ἁμαρτίαι τῶν Ἀμοῤῥαίων ἕως τοῦ νῦν.
\vs{17}Ἐπεὶ δὲ ὁ ἥλιος ἐγένετο πρὸς δυσμὰς, φλὸξ ἐγένετο· καὶ ἰδοὺ κλίβανος καπνιζόμενος καὶ λαμπάδες πυρός, αἳ διῆλθον ἀνὰ μέσον τῶν διχοτομημάτων τούτων.
\vs{18}Ἐν τῇ ἡμέρᾳ ἐκείνῃ διέθετο Κύριος τῷ Ἅβραμ διαθήκην, λέγων, τῷ σπέρματί σου δώσω τὴν γῆν ταύτην, ἀπὸ τοῦ ποταμοῦ Αἰγύπτου ἕως τοῦ ποταμοῦ τοῦ μεγάλου Εὐφράτου·
\vs{19}Τοὺς Κεναίους, καὶ τοὺς Κενεζαίους, καὶ τοὺς Κεδμωναίους,
\vs{20}καὶ τοὺς Χετταίους, καὶ τοὺς Φερεζαίους, καὶ τοὺς ʼΡαφαεὶν,
\vs{21}καὶ τοὺς Ἀμοῤῥαίους, καὶ τοὺς Χαναναίους, καὶ τοὺς Εὐαίους, καὶ τοὺς Γεργεσαίους, καὶ τοὺς Ἰεβουσαίους.

\ch{16}
Σάρα δὲ ἡ γυνὴ Ἅβραμ οὐκ ἔτικτεν αὐτῷ· ἦν δὲ αὐτῇ παιδίσκη Αἰγυπτία, ᾗ ὄνομα Ἄγαρ.
\vs{2}Εἶπε δὲ Σάρα πρὸς Ἅβραμ, ἰδοὺ συνέκλεισέ με Κύριος τοῦ μὴ τίκτειν· εἴσελθε οὖν πρὸς τὴν παιδίσκην μου, ἵνα τεκνοποιήσωμαι ἐξ αὐτῆς· ὑπήκουσελ δὲ Ἅβραμ τῆς φωνῆς Σάρας.
\vs{3}Καὶ λαβοῦσα Σάρα ἡ γυνὴ Ἅβραμ Ἄγαρ τὴν Αἰγυπτίαν τὴν ἑαυτῆς παιδίσκην, μετὰ δέκα ἔτη τοῦ οἰκῆσαι Ἅβραμ ἐν γῇ Χαναὰν, ἔδωκεν αὐτὴν τῷ Ἅβραμ ἀνδρὶ αὐτῆς αὐτῷ γυναῖκα.
\vs{4}Καὶ εἰσῆλθε πρὸς Ἄγαρ, καὶ συνέλαβε· καὶ εἶδεν ὅτι ἐν γαστρὶ ἔχει, καὶ ἠτιμάσθη ἡ κυρία ἐναντίον αὐτῆς.
\vs{5}Εἶπε δὲ Σάρα πρὸς Ἅβραμ, ἀδικοῦμαι ἐκ σοῦ· ἐγὼ δέδωκα τὴν παιδίσκην μου εἰς τὸν κόλπον σου, ἰδοῦσα δὲ ὅτι ἐν γαστρὶ ἔχει, ἠτιμάσθην ἐναντίον αὐτῆς. κρίναι ὁ Θεὸς ἀνὰ μέσον ἐμοῦ καὶ σου.
\vs{6}Εἶπε δὲ Ἅβραμ πρὸς Σάραν, ἰδοὺ ἡ παιδίσκη σου ἐν ταῖς χερσί σου, χρῶ αὐτῇ ὡς ἄν σοι ἀρεστὸν ᾖ. καὶ ἐκάκωσεν αὐτὴν Σάρα, καὶ ἀπέδρα ἀπὸ προσώπου αὐτῆς.

\vs{7}Εὗρε δὲ αὐτὴν ἄγγελος Κυρίου ἐπὶ τῆς πηγῆς τοῦ ὕδατος ἐν τῇ ἐρήμῳ, ἐπὶ τῆς πηγῆς ἐν τῇ ὁδῷ Σούρ.
\vs{8}Καὶ εἶπεν αὐτῇ ὁ ἄγγελος Κυρίου, Ἄγαρ παιδίσκη Σάρας, πόθεν ἔρχῃ; καὶ ποῦ πορεύῃ; καὶ εἶπεν· ἀπὸ προσώπου Σάρας τῆς κυρίας μου ἐγὼ ἀποδιδράσκω.
\vs{9}Εἶπε δὲ αὐτῇ ὁ ἄγγελος Κυρίου, ἀποστράφηθι πρὸς τὴν κυρίαν σου, καὶ ταπεινώθητι ὑπὸ τὰς χεῖρας αὐτῆς.
\vs{10}Καὶ εἶπεν αὐτῇ ὁ ἄγγελος Κυρίου, πληθύνων πληθυνῶ τὸ σπέρμα σου, καὶ οὐκ ἀριθμηθήσεται ὑπὸ τοῦ πλήθους.
\vs{11}Καὶ εἶπεν αὐτῇ ὁ ἄγγελος Κυρίου, ἰδοὺ σὺ ἐν γαστρὶ ἔχεις, καὶ τέξῃ υἱὸν, καὶ καλέσεις τὸ ὄνομα αὐτοῦ Ἰσμαὴλ, ὅτι ἐπήκουσε Κύριος τῇ ταπεινώσει σου.
\vs{12}Οὗτος ἔσται ἄγροικος ἄνθρωπος· αἱ χεῖρες αὐτοῦ ἐπὶ πάντας, καὶ αἱ χεῖρες πάντων ἐπʼ αὐτόν· καὶ κατὰ πρόσωπον πάντων τῶν ἀδελφῶν αὐτοῦ κατοικήσει.
\vs{13}Καὶ ἐκάλεσε τὸ ὄνομα Κυρίου τοῦ λαλοῦντος πρὸς αὐτὴν, σὺ ὁ Θεὸς ὁ ἐτιδών με· ὅτι εἶπε, καὶ γὰρ ἐνώπιον εἶδον ὀφθέντα μοι.
\vs{14}Ἕνεκεν τούτου ἐκάλεσε τὸ φρέαρ, φρέαρ οὗ ἐνώπιον εἶδον· ἰδοὺ ἀνὰ μέσον Κάδης καὶ ἀνὰ μέσον Βαράδ.
\vs{15}Καὶ ἔτεκεν Ἄγαρ τῷ Ἅβραμ υἱὸν, καὶ ἐκάλεσεν Ἅβραμ τὸ ὄνομα τοῦ υἱοῦ αὐτοῦ, ὃν ἔτεκεν αὐτῷ Ἄγαρ, Ἰσμαήλ.
\vs{16}Ἅβραμ δὲ ἦν ἐτῶν ὀγδοηκονταὲξ, ἡνίκα ἔτεκεν Ἄγαρ τῷ Ἅβραμ τὸν Ἰσμαήλ.

\ch{17}
Ἐγένετο δὲ Ἅβραμ ἐτῶν ἐννενηκονταεννέα. Καὶ ὤφθη Κύριος τῷ Ἅβραμ, καὶ εἶπεν αὐτῷ, ἐγώ εἰμι ὁ Θεός σου· εὐαρέστει ἐνώπιον ἐμοῦ, καὶ γίνου ἄμεμπτος.
\vs{2}Καὶ θήσομαι τὴν διαθήκην μου ἀνὰ μέσον ἐμοῦ, καὶ ἀνὰ μέσον σου, καὶ πληθυνῶ σε σφόδρα.
\vs{3}Καὶ ἔπεσεν Ἅβραμ ἐπὶ πρόσωπον αὐτοῦ.
\vs{4}Καὶ ἐλάλησεν αὐτῷ ὁ Θεὸς, λέγων, Καὶ ἐγὼ ἰδοὺ ἡ διαθήκη μου μετὰ σοῦ· καὶ ἔσῃ πατὴρ πλήθους ἐθνῶν.
\vs{5}Καὶ οὐ κληθήσεται ἔτι τὸ ὄνομά σου Ἅβραμ, ἀλλʼ ἔσται τὸ ὄνομά σου Ἁβραὰμ, ὅτι πατέρα πολλῶν ἐθνῶν τέθεικά σε.
\vs{6}Καὶ αὐξανῶ σε σφόδρα σφόδρα, καὶ θήσω σε εἰς ἔθνη· καὶ βασιλεῖς ἐκ σοῦ ἐξελεύσονται.
\vs{7}Καὶ στήσω τὴν διαθήκην μου ἀνὰ μέσον σου, καὶ ἀνὰ μέσον τοῦ σπέρματός σου μετὰ σὲ εἰς τὰς γενεὰς αὐτῶν, εἰς διαθήκην αἰώνιον εἶναί σου Θεὸς, καὶ τοῦ σπέρματός σου μετὰ σέ.
\vs{8}Καὶ δώσω σοι καὶ τῷ σπέρματί σου μετὰ σὲ τὴν γῆν, ἣν παροικεῖς, πᾶσαν τὴν γῆν Χαναὰν, εἰς κατάσχεσιν αἰώνιον· καὶ ἔσομαι αὐτοῖς εἰς Θεόν.
\vs{9}Καὶ εἶπεν ὁ Θεὸς πρὸς Ἁβραὰμ, σὺ δὲ τὴν. διαθήκην μου διατηρήσεις, σὺ καὶ τὸ σπέρμα σου μετὰ σὲ εἰς τὰς γενεὰς αὐτῶν.
\vs{10}Καὶ αὕτη ἡ διαθήκη, ἣν διατηρήσεις, ἀνὰ μέσον ἐμοῦ καὶ ὑμῶν, καὶ ἀνὰ μέσον τοῦ σπέρματός σου μετὰ σὲ εἰς τὰς γενεὰς αὐτῶν· περιτμηθήσεται ὑμῶν πᾶν ἀρσενικόν.
\vs{11}Καὶ περιτμηθήσεσθε τὴν σάρκα τῆς ἀκροβυστίας ὑμῶν, καὶ ἔσται εἰς σημεῖον διαθήκης ἀνὰ μέσον ἐμοῦ καὶ ὑμῶν.
\vs{12}Καὶ παιδίον ὀκτὼ ἡμερῶν περιτμηθήσεται ὑμῖν, πᾶν ἀρσενικὸν εἰς τὰς γενεὰς ὑμῶν· καὶ οἰκογενὴς καὶ ὁ ἀργυρώνητος ἀπὸ παντὸς υἱοῦ ἀλλοτρίου, ὃς οὐκ ἔστιν ἐκ τοῦ σπέρματός σου,
\vs{13}Περιτομῇ περιτμηθήσεται ὁ οἰκογενὴς τῆς οἰκίας σου, καὶ ὁ ἀργυρώνητος· καὶ ἔσται ἡ διαθήκη μου ἐπὶ τῆς σαρκὸς ὑμῶν εἰς διαθήκην αἰώνιον.
\vs{14}Καὶ ἀπερίτμητος ἄρσην, ὃς οὐ περιτμηθήσεται τὴν σάρκα τῆς ἀκροβυστίας αὐτοῦ τῇ ἡμέρᾳ τῇ ὀγδόῃ, ἐξολοθρευθήσεται ἡ ψυχὴ ἐκείνη ἐκ τοῦ γένους αὐτῆς, ὅτι τὴν διαθήκην μου διεσκέδασε.
\vs{15}Καὶ εἶπεν ὁ Θεὸς τῷ Ἁβραὰμ, Σάρα ἡ γυνή σου, οὐ κληθήσεται τὸ ὄνομα αὐτῆς Σάρα, Σάῤῥα ἔσται τὸ ὄνομα αὐτῆς.
\vs{16}Εὐλογήσω δὲ αὐτὴν, καὶ δώσω σοι ἐξ αὐτῆς τέκνον, καὶ εὐλογήσω αὐτὸ, καὶ ἔσται εἰς ἔθνη, καὶ βασιλεῖς ἐθνῶν ἐξ αὐτοῦ ἔσονται.
\vs{17}Καὶ ἔπεσεν Ἁβραὰμ ἐπὶ πρόσωπον αὐτοῦ, καὶ ἐγέλασε· καὶ εἶπεν ἐν τῇ διανοίᾳ αὐτοῦ, λέγων, εἰ τῷ ἑκατονταετεῖ γενήσεται υἱός; καὶ εἰ ἡ Σάῤῥα ἐννενήκοντα ἐτῶν τέξεται;
\vs{18}Εἶπε δὲ Ἁβραὰμ πρὸς τὸν Θεόν· Ἰσμαὴλ οὗτος ζήτω ἐναντίον σου.
\vs{19}Εἶπε δὲ ὁ Θεὸς πρὸς Ἁβραὰμ, ναί· ἰδοὺ Σάῤῥα ἡ γυνή σου τέξεταί σοι υἱὸν, καὶ καλέσεις τὸ ὄνομα αὐτοῦ Ἰσαάκ· καὶ στήσω τὴν διαθήκην μου πρὸς αὐτὸν, εἰς διαθήκην αἰώνιον, εἶναι αὐτῷ Θεὸς καὶ τῷ σπέρματι αὐτοῦ μετʼ αὐτόν.
\vs{20}Περὶ δὲ Ἰσμαὴλ ἰδοὺ ἐπήκουσά σου· καὶ ἰδοὺ εὐλόγηκα αὐτὸν, καὶ αὐξανῶ αὐτὸν, καὶ πληθυνῶ αὐτὸν σφόδρα δώδεκα ἔθνη γεννήσει, καὶ δώσω αὐτὸν εἰς ἔθνος μέγα.
\vs{21}Τὴν δὲ διαθήκην μου στήσω πρὸς Ἰσαὰκ, ὃν τέξεταί σοι Σάῤῥα εἰς τὸν καιρὸν τοῦτον, ἐν τῷ ἐνιαυτῷ τῷ ἑτέρῳ.
\vs{22}Συνετέλεσε δὲ λαλῶν πρὸς αὐτὸν, καὶ ἀνέβη ὁ Θεὸς ἀπὸ Ἁβραάμ.

\vs{23}Καὶ ἔλαβεν Ἁβραὰμ Ἰσμαὴλ τὸν υἱὸν ἑαυτοῦ, καὶ πάντας τοὺς οἰκογενεῖς αὐτοῦ, καὶ πάντας τοὺς ἀργυρωνήτους, καὶ πᾶν ἄρσεν τῶν ἀνδρῶν τῶν ἐν τῷ οἴκῳ Ἁβραὰμ, καὶ περιέτεμε τὰς ἀκροβυστίας αὐτῶν, ἐν τῷ καιρῷ τῆς ἡμέρας ἐκείνης, καθὰ ἐλάλησεν αὐτῷ ὁ Θεός.
\vs{24}Ἁβραὰμ δὲ ἐννενηκονταεννέα ἦν ἐτῶν, ἡνίκα περιετέμετο τὴν σάρκα τῆς ἀκροβυστίας αὐτοῦ.
\vs{25}Ἰσμαὴλ δὲ ὁ υἱὸς αὐτοῦ ἦν ἐτῶν δεκατριῶν, ἡνίκα περιετέμετο τὴν σάρκα τῆς ἀκροβυστίας αὐτοῦ.
\vs{26}Ἐν δὲ τῷ καιρῷ τῆς ἡμέρας ἐκείνης, περιετμήθη Ἁβραὰμ, καὶ Ἰσμαὴλ ὁ υἱὸς αὐτοῦ,
\vs{27}καὶ πάντες οἱ ἄνδρες τοῦ οἴκου αὐτοῦ, καὶ οἱ οἰκογενεῖς αὐτοῦ, καὶ οἱ ἀργυρώνητοι ἐξ ἀλλογενῶν ἐθνῶν.

\ch{18}
Ὤφθη δὲ αὐτῷ ὁ Θεὸς πρὸς τῇ δρυῒ τῇ Μαμβρῇ, καθημένου αὐτοῦ ἐπὶ τῆς θύρας τῆς σκηνῆς αὐτοῦ μεσημβρίας.
\vs{2}Ἀναβλέψας δὲ τοῖς ὀφθαλμοῖς αὐτοῦ εἶδε, καὶ ἰδοὺ τρεῖς ἄνδρες εἱστήκεισαν ἐπάνω αὐτοῦ· καὶ ἰδὼν, προσέδραμεν εἰς συνάντησιν αὐτοῖς ἀπὸ τῆς θύρας τῆς σκηνῆς αὐτοῦ, καὶ προσεκύνησεν ἐπὶ τὴν γῆν.
\vs{3}Καὶ εἶπε, Κύριε, εἰ ἄρα εὗρον χάριν ἐναντίον σου, μὴ παρέλθῃς τὸν παῖδά σου.
\vs{4}Ληφθήτω δὴ ὕδωρ, καὶ νιψάτωσαν τοὺς πόδας ὑμῶν, καὶ καταψύξατε ὑπὸ τὸ δένδρον.
\vs{5}Καὶ λήψομαι ἄρτον, καὶ φάγεσθε. Καὶ μετὰ τοῦτο παρελεύσεσθε εἰς τὴν ὁδὸν ὑμῶν, οὗ ἕνεκεν ἐξεκλίνατε πρὸς τὸν παῖδα ὑμῶν. Καὶ εἶπεν, οὕτω ποίησον, καθὼς εἴρηκας.
\vs{6}Καὶ ἔσπευσεν Ἁβραὰμ ἐπὶ τὴν σκηνὴν πρὸς Σάῤῥαν, καὶ εἶπεν αὐτῇ, σπεῦσον, καὶ φύρασον τρία μέτρα σεμιδάλεως, καὶ ποίησον ἐγκρυφίας.
\vs{7}Καὶ εἰς τὰς βόας ἔδραμεν Ἁβραὰμ, καὶ ἔλαβεν ἁπαλὸν μοσχάριον καὶ καλὸν, καὶ ἔδωκε τῷ παιδὶ, καὶ ἐτάχυνε τοῦ ποιῆσαι αὐτό.
\vs{8}Ἔλαβε δὲ βούτυρον, καὶ γάλα, καὶ τὸ μοσχάριον ὃ ἐποίησε, καὶ παρέθηκεν αὐτοῖς, καὶ ἔφαγον· αὐτὸς δὲ παρειστήκει αὐτοῖς ὑπὸ τὸ δένδρον.

\vs{9}Εἶπε δὲ πρὸς αὐτὸν, ποῦ Σάῤῥα ἡ γυνή σου; ὁ δὲ ἀποκριθεὶς εἶπεν, ἰδοὺ ἐν τῇ σκηνῇ.
\vs{10}Εἶπε δὲ, ἐπαναστρέφων ἥξω πρὸς σὲ κατὰ τὸν καιρὸν τοῦτον εἰς ὥρας, καὶ ἕξει υἱὸν Σάῤῥα ἡ γυνή σου. Σάῤῥα δὲ ἤκουσε πρὸς τῇ θύρᾳ τῆς σκηνῆς οὖσα ὄπισθεν αὐτοῦ.
\vs{11}Ἁβραὰμ δὲ καὶ Σάῤῥα πρεσβύτεροι προβεβηκότες ἡμερῶν· ἐξέλιπε δὲ τῇ Σάῤῥᾳ γίνεσθαι τὰ γυναικεια.
\vs{12}Ἐγέλασε δὲ Σάῤῥα ἐν ἑαυτῇ λέγουσα, οὔπω μέν μοι γέγονεν ἕως τοῦ νῦν· ὁ δὲ κύριός μου πρεσβύτερος.
\vs{13}Καὶ εἶπε Κύριος πρὸς Ἁβραὰμ, τί ὅτι ἐγέλασε Σάῤῥα ἐν ἑαυτῇ, λέγουσα, ἆρά γε ἀληθῶς τέξομαι; ἐγὼ δὲ γεγήρακα.
\vs{14}Μὴ ἀδυνατήσει παρὰ τῷ Θεῷ ῥῆμα; εἰς τὸν καιρὸν τοῦτον ἀναστρέψω πρὸς σὲ εἰς ὥρας, καὶ ἔσται τῇ Σάῤῥᾳ υἱός.
\vs{15}Ἠρνήσατο δὲ Σάῤῥα, λέγουσα, οὐκ ἐγέλασα· ἐφοβήθη γάρ. Καὶ εἶπεν αὐτῇ, οὐχὶ, ἀλλὰ ἐγέλασας.

\vs{16}Ἐξαναστάντες δὲ ἐκεῖθεν οἱ ἄνδρες κατέβλεψαν ἐπὶ πρόσωπον Σοδόμων καὶ Γομόῤῥας. Ἁβραὰμ δὲ συνεπορεύετο μετʼ αὐτῶν, συμπροπέμπων αὐτούς.
\vs{17}Ὁ δὲ Κύριος εἶπε, οὐ μὴ κρύψω ἐγὼ ἀπὸ Ἁβραὰμ τοῦ παιδός μου ἃ ἐγὼ ποιῶ.
\vs{18}Ἁβραὰμ δὲ γινόμενος ἔσται εἰς ἔθνος μέγα καὶ πολὺ, καὶ ἐνευλογηθήσονται ἐν αὐτῷ πάντα τὰ ἔθνη τῆς γῆς.
\vs{19}Ἤδειν γὰρ ὅτι συντάξει τοῖς υἱοῖς αὐτοῦ, καὶ τῷ οἴκῳ αὐτοῦ μετʼ αὐτὸν, καὶ φυλάξουσι τὰς ὁδοὺς Κυρίου, ποιεῖν δικαιοσύνην καὶ κρίσιν, ὅπως ἂν ἐπαγάγῃ Κύριος ἐπὶ Ἁβραὰμ πάντα ὅσα ἐλάλησε πρὸς αὐτόν.
\vs{20}Εἶπε δὲ Κύριος, κραυγὴ Σοδόμων καὶ Γομόῤῥας πεπλήθυνται πρὸς μὲ, καὶ αἱ ἁμαρτίαι αὐτῶν μεγάλαι σφόδρα.
\vs{21}Καταβὰς οὖν ὄψομαι, εἰ κατὰ τὴν κραυγὴν αὐτῶν τὴν ἐρχομενεην πρὸς μὲ, συντελοῦνται· εἰ δὲ μὴ, ἵνα γνῶ.
\vs{22}Καὶ ἀποστρέψαντες ἐκεῖθεν οἱ ἄνδρες, ἦλθον εἰς Σόδομα· Ἁβραὰμ δὲ ἔτι ἦν ἑστηκὼς ἐναντίον Κυρίου.
\vs{23}Καὶ ἐγγίσας Ἁβραὰμ, εἶπε, μὴ συναπολέσῃς δίκαιον μετὰ ἀσεβοῦς· καὶ ἔσται ὁ δίκαιος ὡς ὁ ἀσεβής.
\vs{24}Ἐὰν ὦσι πεντήκοντα δίκαιοι ἐν τῇ πόλει, ἀπολεῖς αὐτούς; οὐκ ἀνήσεις πάντα τὸν τόπον ἕνεκεν τῶν πεντήκοντα δικαίων, ἐὰν ὦσιν ἐν αὐτῇ;
\vs{25}Μηδαμῶς σὺ ποιήσεις ὡς τὸ ῥῆμα τοῦτο, τοῦ ἀποκτεῖναι δίκαιον μετὰ ἀσεβοῦς, καὶ ἔσται ὁ δίκαιος ὡς ὁ ἀσεβής· μηδαμῶς· ὁ κρίνων πᾶσαν τὴν γῆν, οὐ ποιήσεις κρίσιν;
\vs{26}Εἶπε δὲ Κύριος, ἐὰν ὦσιν ἐν Σοδόμοις πεντήκοντα δίκαιοι ἐν τῇ πόλει, ἀφήσω ὅλην τὴν πόλιν, καὶ πάντα τὸν τόπον διʼ αὐτούς.
\vs{27}Καὶ ἀποκριθεὶς Ἁβραὰμ εἶπε, νῦν ἠρξάμην λαλῆσαι πρὸς τὸν Κύριόν μου· ἐγὼ δὲ εἰμὶ γῆ καὶ σποδός.
\vs{28}Ἐὰν δὲ ἐλαττονωθῶσιν οἱ πεντήκοντα δίκαιοι εἰς τεσσαρακονταπέντε, ἀπολεῖς ἕνεκεν τῶν πέντε πᾶσαν τὴν πόλιν; καὶ εἶπεν, οὐ μὴ ἀπολέσω, ἐὰν εὕρω ἐκεῖ τεσσαρακονταπέντε.
\vs{29}Καὶ προσέθηκεν ἔτι λαλῆσαι πρὸς αὐτὸν, καὶ εἶπεν, ἐὰν δὲ εὑρεθῶσιν ἐκεῖ τεσσαράκοντα· καὶ εἶπεν, οὐ μὴ ἀπολέσω ἕνεκεν τῶν τεσσαράκοντα.
\vs{30}Καὶ εἶπε, μή τι Κύριε ἐὰν λαλήσω; ἐὰν δὲ εὑρεθῶσιν ἐκεῖ τριάκοντα; καὶ εἶπεν, οὐ μὴ ἀπολέσω ἕνεκεν τῶν τριάκοντα.
\vs{31}Καὶ εἶπεν, ἐπειδὴ ἔχω λαλῆσαι πρὸς τὸν Κύριον, ἐὰν δὲ εὑρεθῶσιν ἐκεῖ εἴκοσι; καὶ εἶπεν, οὐ μὴ ἀπολέσω, ἐὰν εὕρω ἐκεῖ εἴκοσι.
\vs{32}Καὶ εἶπε, μή τι Κύριε ἐὰν λαλήσω ἔτι ἅπαξ; ἐὰν δὲ εὑρεθῶσιν ἐκεῖ δέκα; καὶ εἶπεν, οὐ μὴ ἀπολέσω ἕνεκεν τῶν δέκα.
\vs{33}Ἀπῆλθε δὲ ὁ Κύριος, ὡς ἐπαύσατο λαλῶν τῷ Ἁβραάμ· καὶ Ἁβραὰμ ἀπέστρεψεν εἰς τὸν τόπον αὐτοῦ.

\ch{19}
Ἦλθον δε οἱ δύο ἄγγελοι εἰς Σόδομα ἑσπέρας. Λὼτ δὲ ἐκάθητο παρὰ τὴν πύλην Σοδόμων· ἰδὸν δὲ Λὼτ, ἐξανέστη εἰς συνάντησιν αὐτοῖς, καὶ προσεκύνησε τῷ προσώπῳ ἐπὶ τὴν γῆν.
\vs{2}Καὶ εἶπεν, ἰδοὺ, Κύριοι, ἐκκλίνατε εἰς τὸν οἶκον τοῦ παιδὸς ὑμῶν, καὶ καταλύσατε, καὶ νίψασθε τοὺς πόδας ὑμῶν, καὶ ὀρθρίσαντες ἀπελεύσεσθε εἰς τὴν ὁδὸν ὑμῶν. Καὶ εἶπαν, οὐχὶ, ἀλλʼ ἐν τῇ πλατείᾳ καταλύσομεν.
\vs{3}Καὶ κατεβιάσατο αὐτοὺς, καὶ ἐξέκλιναν πρὸς αὐτὸν, καὶ εἰσῆλθον εἰς τὸν οἶκον αὐτοῦ· καὶ ἐποίησεν αὐτοῖς πότον, καὶ ἀζύμους ἔπεψεν αὐτοῖς, καὶ ἔφαγον.
\vs{4}Πρὸ τοῦ κοιμηθῆναι δὲ, οἱ ἄνδρες τῆς πόλεως, οἱ Σοδομῖται περιεκύκλωσαν τὴν οἰκίαν, ἀπὸ νεανίσκου ἕως πρεσβυτέρου, ἅπας ὁ λαὸς ἅμα.
\vs{5}Καὶ ἐξεκαλοῦντο τὸν Λὼτ, καὶ ἔλεγον πρὸς αὐτὸν, ποῦ εἰσιν οἱ ἄνδρες οἱ εἰσελθόντες πρὸς σὲ τὴν νύκτα; ἐξάγαγε αὐτοὺς πρὸς ἡμᾶς, ἵνα συγγενώμεθα αὐτοῖς.
\vs{6}Ἐξῆλθε δὲ Λὼτ πρὸς αὐτοὺς πρὸς τὸ πρόθυρον, τὴν δὲ θύραν προσέῳξεν ὀπίσω αὐτοῦ.
\vs{7}Εἶπε δὲ πρὸς αὐτοὺς, μηδαμῶς ἀδελφοὶ μὴ πονηρεύσησθε.
\vs{8}Εἰσὶ δέ μοι δύο θυγατέρες, αἳ οὐκ ἔγνωσαν ἄνδρα· ἐξάξω αὐτὰς πρὸς ὑμᾶς, καὶ χρᾶσθε αὐταῖς καθὰ ἂν ἀρέσκοι ὑμῖν· μόνον εἰς τοὺς ἄνδρας τούτους μὴ ποιήσητε ἄδικον, οὗ εἵνεκεν εἰσῆλθον ὑπὸ τὴν σκέπην τῶν δοκῶν μου.
\vs{9}Εἶπαν δὲ αὐτῷ, ἀπόστα ἐκεῖ· εἰσῆλθες παροικεῖν, μὴ καὶ κρίσιν κρίνειν; νῦν οὖν σε κακώσωμεν μᾶλλον ἢ ἐκείνους. Καὶ παρεβιάζοντο τὸν ἄνδρα τὸν Λὼτ σφόδρα, καὶ ἤγγισαν συντρίψαι τὴν θύραν.
\vs{10}Ἐκτείναντες δὲ οἱ ἄνδρες τὰς χεῖρας εἰσεσπάσαντο τὸν Λὼτ πρὸς ἑαυτοὺς εἰς τὸν οἶκον, καὶ τὴν θύραν τοῦ οἴκου ἀπέκλεισαν.
\vs{11}Τοὺς δὲ ἄνδρας τοὺς ὄντας ἐπὶ τῆς θύρας τοῦ οἴκου ἐπάταξαν ἐν ἀορασίᾳ ἀπὸ μικροῦ ἕως μεγάλου· καὶ παρελύθησαν ζητοῦντες τὴν θύραν.
\vs{12}Εἶπαν δὲ οἱ ἄνδρες πρὸς τὸν Λὼτ, εἰσί σοι ὧδε γαμβροὶ, ἢ υἱοὶ, ἢ θυγατέρες; ἢ εἴτις σοι ἄλλος ἐστὶν ἐν τῇ πόλει, ἐξάγαγε ἐκ τοῦ τόπου τούτου,
\vs{13}Ὅτι ἡμεῖς ἀπόλλυμεν τὸν τόπον τοῦτον· ὅτι ὑψώθη ἡ κραυγὴ αὐτῶν ἔναντι Κυρίου, καὶ ἀπέστειλεν ἡμᾶς Κύριος ἐκτρίψαι αὐτήν.
\vs{14}Ἐξῆλθε δὲ Λῶτ, καὶ ἐλάλησε πρὸς τοὺς γαμβροὺς αὐτοῦ τοὺς εἰληφότας τὰς θυγατέρας αὐτοῦ, καὶ εἶπεν, ἀνάστητε, καὶ ἐξέλθετε ἐκ τοῦ τόπου τούτου, ὅτι ἐκτρίβει Κύριος τὴν πόλιν· ἔδοξε δὲ γελοιάζειν ἐναντίον τῶν γαμβρῶν αὐτοῦ.
\vs{15}Ἡνίκα δὲ ὄρθρος ἐγένετο, ἐσπούδαζον οἱ ἄγγελοι τὸν Λὼτ, λέγοντες, ἀναστὰς λάβε τὴν γυναῖκά σου, καὶ τὰς δύο θυγατέρας σου, ἃς ἔχεις, καὶ ἔξελθε, ἵνα μὴ καὶ σὺ συναπόλῃ ταῖς ἀνομίαις τῆς πόλεως.
\vs{16}Καὶ ἐταράχθησαν, καὶ ἐκράτησαν οἱ ἄγγελοι τῆς χειρὸς αὐτοῦ, καὶ τῆς χειρὸς τῆς γυναικὸς αὐτοῦ, καὶ τῶν χειρῶν τῶν δύο θυγατέρων αὐτοῦ, ἐν τῷ φείσασθαι Κύριον αὐτοῦ.

\vs{17}Καὶ ἐγένετο ἡνίκα ἐξήγαγον αὐτοὺς ἔξω, καὶ εἶπαν, σώζων σῶζε τὴν σεαυτοῦ ψυχήν· μὴ περιβλέψῃ εἰς τὰ ὀπίσω, μηδὲ στῇς ἐν πάσῃ τῇ περιχώρῳ· εἰς τὸ ὄρος σώζου, μή ποτε συμπαραληφθῇς.
\vs{18}Εἶπε δὲ Λὼτ πρὸς αὐτοὺς, δέομαι
\vs{19}Κύριε, ἐπειδὴ εὗρεν ὁ παῖς σου ἔλεος ἐναντίον σου, καὶ ἐμεγάλυνας τὴν δικαιοσύνην σου, ὃ ποιεῖς ἐπʼ ἐμὲ, τοῦ ζῆν τὴν ψυχήν μου· ἐγὼ δὲ οὐ δυνήσομαι διασωθῆναι εἰς τὸ ὄρος, μή ποτε καταλάβῃ με τὰ κακὰ, καὶ ἀποθάνω.
\vs{20}Ἰδοὺ πόλις αὕτη ἐγγὺς τοῦ καταφυγεῖν με ἐκεῖ, ἥ ἐστι μικρά· καὶ ἐκεῖ διασωθήσομαι· οὐ μικρά ἐστι; καὶ ζήσεται ἡ ψυχή μου ἕνεκέν σου.
\vs{21}Καὶ εἶπεν αὐτῷ, ἰδοὺ ἐθαύμασά σου τὸ πρόσωπον καὶ ἐπὶ τῷ ῥήματι τούτῳ, τοῦ μὴ καταστρέψαι τὴν πόλιν περὶ ἧς ἐλάλησας.
\vs{22}Σπεῦσον οὖν τοῦ σωθῆναι ἐκεῖ, οὐ γὰρ δυνήσομαι ποιῆσαι πρᾶγμα, ἕως τοῦ ἐλθεῖν σε ἐκεῖ. διὰ τοῦτο ἐκάλεσε τὸ ὄνομα τῆς πόλεως ἐκείνης, Σηγώρ.
\vs{23}Ὁ ἥλιος ἐξῆλθεν ἐπὶ τὴν γῆν, καὶ Λὼτ εἰσῆλθεν εἰς Σηγώρ.
\vs{24}Καὶ Κύριος ἔβρεξεν ἐπὶ Σόδομα καὶ Γόμοῤῥα θεῖον καὶ πῦρ παρὰ Κυρίου ἐξ οὐρανοῦ.
\vs{25}Καὶ κατέστρεψε τὰς πόλεις ταύτας, καὶ πᾶσαν τὴν περίχωρον, καὶ πάντας τοὺς κατοικοῦντας ἐν ταῖς πόλεσι, καὶ τὰ ἀνατέλλοντα ἐκ τῆς γῆς.
\vs{26}Καὶ ἐπέβλεψεν ἡ γυνὴ αὐτοῦ εἰς τὰ ὀπίσω, καὶ ἐγένετο στήλη ἁλός.
\vs{27}Ὤρθρισε δὲ Ἁβραὰμ τῷ πρωῒ εἰς τὸν τόπον, οὗ εἱστήκει ἐναντίον Κυρίου.
\vs{28}Καὶ ἐπέβλεψεν ἐπὶ πρόσωπον Σοδόμων καὶ Γομόῤῥας, καὶ ἐπὶ πρόσωπον τῆς περιχώρου, καὶ εἶδε, καὶ ἰδοὺ ἀνέβαινεν φλὸξ ἐκ τῆς γῆς, ὡσεὶ ἀτμὶς καμίνου.
\vs{29}Καὶ ἐγένετο ἐν τῷ ἐκτρίψαι τὸν Θεὸν πάσας τὰς πόλεις τῆς περιοίκου, ἐμνήσθη ὁ Θεὸς τοῦ Ἁβραάμ· καὶ ἐξαπέστειλε τὸν Λὼτ ἐκ μέσου τῆς καταστροφῆς, ἐν τῷ καταστρέψαι Κύριον τὰς πόλεις, ἐν αἷς κατῴκει ἐν αὐταῖς Λώτ.

\vs{30}Ἀνέβη δὲ Λὼτ ἐκ Σηγὼρ, καὶ ἐκάθητο ἐν τῷ ὄρει αὐτὸς, καὶ αἱ δύο θυγατέρες αὐτοῦ μετʼ αὐτοῦ· ἐφοβήθη γὰρ κατοικῆσαι ἐν Σηγώρ· καὶ κατῴκησεν ἐν τῷ σπηλαίῳ αὐτὸς, καὶ αἱ δύο θυγατέρες αὐτοῦ μετʼ αὐτοῦ.
\vs{31}Εἶπε δὲ ἡ πρεσβυτέρα πρὸς τὴν νεωτέραν, ὁ πατὴρ ἡμῶν πρεσβύτερος, καὶ οὐδείς ἐστιν ἐπὶ τῆς γῆς, ὃς εἰσελεύσεται πρὸς ἡμᾶς, ὡς καθήκει πάσῃ τῇ γῇ.
\vs{32}Δεῦρο καὶ ποτίσωμεν τὸν πατέρα ἡμῶν οἶνον, καὶ κοιμηθῶμεν μετʼ αὐτοῦ, καὶ ἐξαναστήσωμεν ἐκ τοῦ πατρὸς ἡμῶν σπέρμα.
\vs{33}Ἐπότισαν δὲ τὸν πατέρα αὐτῶν οἶνον ἐν τῇ νυκτὶ ἐκείνῃ, καὶ εἰσελθοῦσα ἡ πρεσβυτέρα ἐκοιμήθη μετὰ τοῦ πατρὸς αὐτῆς ἐν τῇ νυκτὶ ἐκείνῃ· καὶ οὐκ ᾔδει ἐν τῷ κοιμηθῆναι αὐτὸν, καὶ ἐν τῷ ἀναστῆναι.
\vs{34}Ἐγένετο δὲ ἐν τῇ ἐπαύριον, καὶ εἶπεν ἡ πρεσβυτέρα πρὸς τὴν νεωτέραν, ἰδοὺ ἐκοιμήθην χθὲς μετὰ τοῦ πατρὸς ἡμῶν· ποτίσωμεν αὐτὸν οἶνον καὶ ἐν τῇ νυκτὶ ταύτῃ, καὶ εἰσελθοῦσα κοιμήθητι μετʼ αὐτοῦ, καὶ ἐξαναστήσωμεν ἐκ τοῦ πατρὸς ἡμῶν σπέρμα.
\vs{35}Ἐπότισαν δὲ καὶ ἐν τῇ νυκτὶ ἐκείνῃ τὸν πατέρα αὐτῶν οἶνον, καὶ εἰσελθοῦσα ἡ νεωτέρα ἐκοιμήθη μετὰ τοῦ πατρὸς αὐτῆς· καὶ οὐκ ᾔδει ἐν τῷ κοιμηθῆναι αὐτὸν, καὶ ἀναστῆναι.
\vs{36}Καὶ συνέλαβον αἱ δύο θυγατέρες Λὼτ ἐκ τοῦ πατρὸς αὐτῶν.
\vs{37}Καὶ ἔτεκεν ἡ πρεσβυτέρα υἱὸν, καὶ ἐκάλεσε τὸ ὄνομα αὐτοῦ Μωὰβ, λέγουσα, ἐκ τοῦ πατρός μου· οὗτος πατὴρ Μωαβιτῶν ἕως τῆς σήμερον ἡμέρας.
\vs{38}Ἔτεκε δὲ καὶ ἡ νεωτέρα υἱὸν, καὶ ἐκάλεσε τὸ ὄνομα αὐτοῦ Ἀμμὰν, λέγουσα, υἱὸς γένους μου· οὗτος πατὴρ Ἀμμανιτῶν ἕως τῆς σήμερον ἡμέρας.

\ch{20}
Καὶ ἐκίνησεν ἐκεῖθεν Ἁβραὰμ εἰς γῆν πρὸς Λίβα· καὶ ᾤκησεν ἀνὰ μέσον Κάδης, καὶ ἀνὰ μέσον Σούρ· καὶ παρῴκησεν ἐν Γεράροις.
\vs{2}Εἶπε δὲ Ἁβραὰμ περὶ Σάῤῥας τῆς γυναικὸς αὐτοῦ, ὅτι ἀδελφή μου ἐστίν· ἐφοβήθη γὰρ εἰπεῖν ὅτι γυνή μου ἐστὶ, μή ποτε ἀποκτείνωσιν αὐτὸν οἱ ἄνδρες τῆς πόλεως διʼ αὐτήν· ἀπέστειλε δὲ Ἀβιμέλεχ βασιλεὺς Γεράρων, καὶ ἔλαβε τὴν Σάῤῥαν.
\vs{3}Καὶ εἰσῆλθεν ὁ Θεὸς πρὸς Ἀβιμέλεχ ἐν ὕπνῳ τὴν νύκτα, καὶ εἶπεν, ἰδοὺ σὺ ἀποθνήσκεις περὶ τῆς γυναικὸς, ἧς ἔλαβες· αὕτη δέ ἐστι συνῳκηκυῖα ἀνδρί.
\vs{4}Ἀβιμέλεχ δὲ οὐχ ἥψατο αὐτῆς· καὶ εἶπε, Κύριε, ἔθνος ἀγνοοῦν καὶ δίκαιον ἀπολεῖς;
\vs{5}Οὐκ αὐτός μοι εἶπεν, ἀδφή μου ἐστί; καὶ αὕτη μοι εἶπεν, ἀδελφός μου ἐστίν; ἐν καθαρᾷ καρδίᾳ καὶ ἐν δικαιοσύνῃ χειρῶν ἐποίησα τοῦτο.
\vs{6}Εἶπε δὲ αὐτῷ ὁ Θεὸς καθʼ ὕπνον, κᾀγὼ ἔγνων ὅτι ἐν καθαρᾷ καρδίᾳ ἐποίησας τοῦτο, καὶ ἐφεισάμην σου τοῦ μὴ ἁμαρτεῖν σε εἰς ἐμέ· ἕνεκα τούτου οὐκ ἀφῆκά σε ἅψασθαι αὐτῆς.
\vs{7}Νῦν δὲ ἀπόδος τὴν γυναῖκα τῷ ἀνθρώπῳ, ὅτι προφήτης ἐστὶ, καὶ προσεύξεται περὶ σοῦ, καὶ ζήσῃ· εἰ δὲ μὴ ἀποδίδως, γνώσῃ ὅτι ἀποθανῇ σὺ καὶ πάντα τὰ σὰ.
\vs{8}Καὶ ὤρθρισεν Ἀβιμέλεχ τῷ πρωῒ, καὶ ἐκάλεσε πάντας τοὺς παῖδας αὐτοῦ, καὶ ἐλάλησε πάντα τὰ ῥήματα ταῦτα εἰς τὰ ὦτα αὐτῶν· ἐφοβήθησαν δὲ πάντες οἱ ἄνθρωποι σφόδρα.
\vs{9}Καὶ ἐκάλεσεν Ἀβιμέλεχ τὸν Ἁβραὰμ καὶ εἶπεν αὐτῷ, τί τοῦτο ἐποίησας ἡμῖν; μήτι ἡμάρτομεν εἰς σὲ, ὅτι ἐπήγαγες ἐπʼ ἐμὲ καὶ ἐπὶ τὴν βασιλείαν μου ἁμαρτίαν μεγάλην; ἔργον ὃ οὐδεὶς ποιήσει, πεποίηκάς μοι.
\vs{10}Εἶπε δὲ Ἀβιμέλεχ τῷ Ἁβραὰμ, τί ἐνιδὼν ἐποίησας τοῦτο;
\vs{11}Εἶπε δὲ Ἁβραὰμ, εἶπα γὰρ, ἄρα οὐκ ἔστι θεοσέβεια ἐν τῷ τόπῳ τούτῳ, ἐμέ τε ἀποκτενοῦσιν ἕνεκεν τῆς γυναικός μου.
\vs{12}Καὶ γὰρ ἀληθῶς, ἀδελφή μου ἐστὶν ἐκ πατρὸς, ἀλλʼ οὐκ ἐκ μητρός· ἐγενήθη δέ μοι εἰς γυναῖκα.
\vs{13}Ἐγένετο δὲ ἡνίκα ἐξήγαγέ με ὁ Θεὸς ἐκ τοῦ οἴκου τοῦ πατρός μου, καὶ εἶπα αὐτῇ, ταύτην τὴν δικαιοσύνην ποιήσεις εἰς ἐμὲ, εἰς πάντα τόπον οὗ ἐὰν εἰσέλθωμεν ἐκεῖ, εἶπον ἐμὲ, ὅτι ἀδελφός μου ἐστίν.
\vs{14}Ἔλαβε δὲ Ἀβιμέλεχ χίλια δίδραγμα, καὶ πρόβατα, καὶ μόσχους, καὶ παῖδας, καὶ παιδίσκας, καὶ ἔδωκε τῷ Ἁβραάμ· καὶ ἀπέδωκεν αὐτῷ Σάῤῥαν τὴν γυναῖκα αὐτοῦ.
\vs{15}Καὶ εἶπεν Ἀβιμέλεχ τῷ Ἁβραὰμ, ἰδοὺ ἡ γῆ μου ἐναντίον σου· οὗ ἄν σοι ἀρέσκῃ, κατοίκει.
\vs{16}Τῇ δὲ Σάῤῥᾳ εἶπεν, ἰδοὺ δέδωκα χίλια δίδραγμα τῷ ἀδελφῷ σου· ταῦτα ἔσται σοι εἰς τιμὴν τοῦ προσώπου σου, καὶ πάσαις ταῖς μετὰ σοῦ· καὶ πάντα ἀλήθευσον.
\vs{17}Προσηύξατο δὲ Ἁβραὰμ πρὸς τὸν Θεὸν, καὶ ἰάσατο ὁ Θεὸς τὸν Ἀβιμέλεχ, καὶ τὴν γυναῖκα αὐτοῦ, καὶ τὰς παιδίσκας αὐτοῦ· καὶ ἔτεκον.
\vs{18}Ὅτι συγκλείων συνέκλεισε Κύριος ἔξωθεν πᾶσαν μήτραν ἐν τῷ οἴκῳ Ἀβιμέλεχ, ἕνεκεν Σάῤῥας τῆς γυναικὸς Ἁβραάμ.

\ch{21}
Καὶ Κύριος ἐπεσκέψατο τὴν Σάῤῥαν, καθὰ εἶπε· καὶ ἐποίησε Κύριος τῇ Σάῥῥᾳ, καθὰ ἐλάλησε.
\vs{2}Καὶ συλλαβοῦσα ἔτεκε τῷ Ἁβραὰμ υἱὸν εἰς τὸ γῆρας, εἰς τὸν καιρὸν καθὰ ἐλάλησεν αὐτῷ Κύριος.
\vs{3}Καὶ ἐκάλεσεν Ἁβραὰμ τὸ ὄνομα τοῦ υἱοῦ αὐτοῦ τοῦ γενομένου αὐτῷ, ὃν ἔτεκεν αὐτῷ Σάῤῥα, Ἰσαάκ·
\vs{4}Περιέτεμε δὲ Ἁβραὰμ τὸν Ἰσαὰκ τῇ ἡμέρᾳ τῇ ὀγδόῃ, καθὰ ἐνετείλατο αὐτῷ ὁ Θεός.
\vs{5}Καὶ Ἁβραὰμ ἦν ἑκατὸν ἐτῶν, ηνίκα ἐγένετο αὐτῷ Ἰσαὰκ ὁ υἱὸς αὐτοῦ.
\vs{6}Εἶπε δὲ Σάῤῥα, γέλωτά μοι ἐποίησε Κύριος· ὃς γὰρ ἂν ἀκούσῃ συγχαρεῖταί μοι.
\vs{7}Καὶ εἶπε τίς ἀναγγελεῖ τῷ Ἁβραὰμ ὅτι θηλάζει παιδίον Σάῤῥα; ὅτι ἔτεκον υἱὸν ἐν τῷ γήρᾳ μου.
\vs{8}Καὶ ηὐξήθη τὸ παιδίον, καὶ ἀπεγαλακτίσθη· καὶ ἐποίησεν Ἁβραὰμ δοχὴν μεγάλην, ᾗ ἡμέρᾳ ἀπεκγαλακτίσθη Ἰσαὰκ ὁ υἱὸς αὐτοῦ.
\vs{9}Ἰδοῦσα δὲ Σάῥῥα τὸν υἱὸν Ἄγαρ τῆς Αἰγυπτίας, ὃς ἐγένετο τῷ Ἁβραὰμ, παίζοντα μετὰ Ἰσαὰκ τοῦ υἱοῦ αὐτῆς,
\vs{10}καὶ εἶπε τῷ Ἁβραὰμ, ἔκβαλε τὴν παιδίσκην ταύτην, καὶ τὸν υἱὸν αὐτῆς· οὐ γὰρ μὴ κληρονομήσει ὁ υἱὸς τῆς παιδίσκης ταύτης μετὰ τοῦ υἱοῦ μου Ἰσαάκ.
\vs{11}Σκληρὸν δὲ ἐφάνη τὸ ῥῆμα σφόδρα ἐνατίον Ἁβραὰμ περὶ τοῦ υἱοῦ αὐτοῦ.
\vs{12}Εἶπε δὲ ὁ Θεὸς τῷ Ἁβραὰμ, μὴ σκληρὸν ἔστω ἐναντίον σου περὶ τοῦ παιδίου, καὶ περὶ τῆς παιδίσκης· πάντα ὅσα ἂν εἴπῃ σοι Σάῤῥα, ἄκουε τῆς φωνῆς αὐτῆς· ὅτι ἐν Ἰσαὰκ κληθήσεταί σοι σπέρμα.
\vs{13}Καὶ τὸν υἱὸν δὲ τῆς παιδίσκης ταύτης εἰς ἔθνος μέγα ποιήσω αὐτὸν, ὅτι σπέρμα σόν ἐστιν.
\vs{14}Ἀνέστη δὲ Ἁβραὰμ τὸ πρωῒ, καὶ ἔλαβεν ἄρτους καὶ ἀσκὸν ὕδατος, καὶ ἔδωκεν τῇ Ἄγαρ· καὶ ἐπέθηκεν ἐπὶ τὸν ὦμον αὐτῆς τὸ παιδίον, καὶ ἀπέστειλεν αὐτήν· Ἀπελθοῦσα δὲ ἐπλανᾶτο κατὰ τὴν ἔρημον, κατὰ τὸ φρέαρ τοῦ ὅρκου.
\vs{15}Ἐξέλιπε δὲ τὸ ὕδωρ ἐκ τοῦ ἀσκου· καὶ ἔῤῥιψε τὸ παιδίον ὑποκάτω μιᾶς ἐλάτης·
\vs{16}Ἀπελθοῦσα δὲ ἐκάθητο ἀπέναντι αὐτοῦ μακρόθεν, ὡσεὶ τόξου βολήν· εἶπε γὰρ, οὐ μὴ ἴδω τὸν θάνατον τοῦ παιδίου μου. καὶ ἐκάθισεν ἀπέναντι αὐτοῦ· ἀναβοῆσαν δὲ τὸ παιδίον ἔκλαυσεν.
\vs{17}Εἰσήκουσε δὲ ὁ Θεὸς τῆς φωνῆς τοῦ παιδίου ἐκ τοῦ τόπου οὗ ἦν· καὶ ἐκάλεσεν ἄγγελος Θεοῦ τὴν Ἄγαρ ἐκ τοῦ οὐρανοῦ, καὶ εἶπεν αὐτῇ, τί ἐστιν Ἄγαρ; μὴ φοβοῦ· ἐπακήκοε γὰρ ὁ Θεὸς τῆς φωνῆς τοῦ παιδίου ἐκ τοῦ τόπου οὗ ἐστιν.
\vs{18}Ἀνάστηθι καὶ λάβε τὸ παιδίον, καὶ κράτησον τῇ χειρί σου αὐτό· εἰς γὰρ ἔθνος μέγα ποιήσω αὐτό.
\vs{19}Καὶ ἀνέῳξεν ὁ Θεὸς τοὺς ὀφθαλμοὺς αὐτῆς· καὶ εἶδε φρέαρ ὕδατος ζῶντος, καὶ ἐπορεύθη, καὶ ἔπλησε τὸν ἀσκὸν ὕδατος, καὶ ἐπότισε τὸ παιδίον.
\vs{20}Καὶ ἦν ὁ Θεὸς μετὰ τοῦ παιδίου· καὶ ηὐξήθη, καὶ κατῴκησεν ἐν τῇ ἐρήμῳ· ἐγένετο δὲ τοξότης.
\vs{21}Καὶ κατῴκησεν ἐν τῇ ἐρήμῳ· καὶ ἔλαβεν αὐτῷ ἡ μήτηρ γυναῖκα ἐκ Φαρὰν Αἰγύπτου.

\vs{22}Ἐγένετο δὲ ἐν τῷ καιρῷ ἐκείνῳ, καὶ εἶπεν Ἀβιμέλεχ, καὶ Ὁχοζὰθ ὁ νυμφαγωγὸς αὐτοῦ, καὶ Φιχὸλ ὁ ἀρχιστράτηγος τῆς δυνάμεως αὐτοῦ, πρὸς Ἁβραὰμ, λέγων, ὁ Θεὸς μετὰ σοῦ ἐν πᾶσιν, οἷς ἐὰν ποιῇς.
\vs{23}Νῦν οὖν ὄμοσόν μοι τὸν Θεὸν μὴ ἀδικήσειν με, μηδὲ τὸ σπέρμα μου, μηδὲ τὸ ὄνομά μου· ἀλλὰ κατὰ τὴν δικαιοσύνην ἣν ἐποίησα μετὰ σοῦ, ποιήσεις μετʼ ἐμοῦ, καὶ τῇ γᾗ, ᾗ σὺ παρῴκησας ἐν αὐτῇ.
\vs{24}Καὶ εἶπεν Ἁβραὰμ, ἐγὼ ὀμοῦμαι.
\vs{25}Καὶ ἤλεγξεν Ἁβραὰμ τὸν Ἀβιμέλεχ περὶ τῶν φρεάτων τοῦ ὕδατος, ὧν ἀφείλοντο οἱ παῖδες τοῦ Ἀβιμέλεχ.
\vs{26}Καὶ εἶπεν αὐτῷ Ἀβιμέλεχ, οὐκ ἔγνων τίς ἐποίησέ σοι τὸ ῥῆμα τοῦτο· οὐδὲ σύ μοι ἀπήγγειλας, οὐδὲ ἐγὼ ἤκουσα, ἀλλʼ ἢ σήμερον.
\vs{27}Καὶ ἔλαβεν Ἁβραὰμ πρόβατα καὶ μόσχους, καὶ ἔδωκε τῷ Ἀβιμέλεχ· καὶ διέθεντο ἀμφότεροι διαθήκην.
\vs{28}Καὶ ἔστησεν Ἁβραὰμ, ἑπτὰ ἀμνάδας προβάτων μόνας.
\vs{29}Καὶ εἶπεν Ἀβιμέλεχ τῷ Ἁβραὰμ, τί εἰσιν αἱ ἑπτὰ ἀμνάδες τῶν προβάτων τούτων, ἃς ἔστησας μόνας;
\vs{30}Καὶ εἶπεν Ἁβραὰμ, ὅτι τὰς ἑπτὰ ἀμνάδας λήψῃ παρʼ ἐμοῦ, ἵνα ὦσι μοι εἰς μαρτύριον, ὅτι ἐγὼ ὤρυξα τό φρέαρ τοῦτο.
\vs{31}Διὰ τοῦτο ἐπωνόμασε τὸ ὄνομα τοῦ τόπου ἐκείνου, Φρέαρ ὁρκισμοῦ· ὅτι ἐκεῖ ὤμοσαν ἀμφότεροι.
\vs{32}Καὶ διέθεντο διαθήκην ἐν τῷ φρέατι τοῦ ὁρκισμου· ἀνέστη δὲ Ἀβιμέλεχ, Ὁχοζὰθ ὁ νυμφαγωγὸς αὐτοῦ, καὶ Φίχολ ὁ ἀρχιστράτηγος τῆς δυνάμεως αὐτοῦ, καὶ ἐπέστρεψαν εἰς τὴν γῆν τῶν Φυλιστιείμ.
\vs{33}Καὶ ἐφύτευσεν Ἁβραὰμ ἄρουραν ἐπὶ τῷ φρέατι τοῦ ὅρκου· καὶ ἐπεκαλέσατο ἐκεῖ τὸ ὄνομα Κυρίου, Θεὸς αἰώνιος.
\vs{34}Παρῴκησε δὲ Ἁβραὰμ ἐν τῇ γῇ τῶν Φυλιστιεὶμ ἡμέρας πολλάς.

\ch{22}
Καὶ ἐγένετο μετὰ τὰ ῥήματα ταῦτα ὁ Θεὸς ἐπείρασε τὸν Ἁβραὰμ, καὶ εἶπεν αὐτῷ, Ἁβραὰμ, Ἁβραάμ· καὶ εἶπεν, ἰδοὺ ἐγώ.
\vs{2}Καὶ εἶπε, λάβε τὸν υἱόν σου τὸν ἀγαπητὸν, ὃν ἠγάπησας, τὸν Ἰσαὰκ, καὶ πορεύθητι εἰς τὴν γῆν τὴν ὑψηλὴν, καὶ ἀνένεγκε αὐτὸν ἐκεῖ εἰς ὁλοκάρπωσιν ἐφʼ ἓν τῶν ὀρέων ὧν ἄν σοι εἴπω.
\vs{3}Ἀναστὰς δὲ Ἁβραὰμ τὸ πρωῒ, ἐπέσαξε τὴν ὄνον αὐτοῦ· παρέλαβε δὲ μεθʼ ἑαυτοῦ δύο παῖδας, καὶ Ἰσαὰκ τὸν υἱὸν αὐτοῦ· καὶ σχίσας ξύλα εἰς ὁλοκάρπωσιν, ἀναστὰς ἐπορεύθη, καὶ ἦλθεν ἐπὶ τὸν τόπον, ὃν εἶπεν αὐτῷ ὁ Θεὸς, τῇ ἡμέρᾳ τῇ τρίτῃ.
\vs{4}Καὶ ἀναβλέψας Ἁβραὰμ τοῖς ὀφθαλμοῖς αὐτοῦ, εἶδε τὸν τόπον μακρόθεν.
\vs{5}Καὶ εἶπεν Ἁβραὰμ τοῖς παισὶν αὐτοῦ, καθίσατε αὐτοῦ μετὰ τῆς ὄνου· ἐγὼ δὲ καὶ τὸ παιδάριον διελευσόμεθα ἕως ὧδε· καὶ προσκυνήσαντες ἀναστρέψομεν πρὸς ὑμᾶς.
\vs{6}Ἔλαβε δὲ Ἁβραὰμ τὰ ξύλα τῆς ὁλοκαρπώσεως, καὶ ἐπέθηκεν Ἰσαὰκ τῷ υἱῷ αὐτοῦ· ἔλαβε δὲ μετὰ χεῖρας καὶ τὸ πῦρ καὶ τὴν μάχαιραν, καὶ ἐπορεύθησαν οἱ δύο ἅμα.
\vs{7}Εἶπε δὲ Ἰσαὰκ πρὸς Ἁβραὰμ τὸν πατέρα αὐτοῦ, πάτερ· ὁ δὲ εἶπε, τί ἐστι, τέκνον; εἶπε, δὲ, ἰδοὺ τὸ πῦρ καὶ τὰ ξύλα, ποῦ ἐστὶ τὸ πρόβατον τὸ εἰς ὁλοκάρπωσιν;
\vs{8}Εἶπε δὲ Ἁβραὰμ, ὁ Θεὸς ὄψεται ἑαυτῷ πρόβατον εἰς ὁλοκάρπωσιν, τέκνον. πορευθέντες δὲ ἀμφότεροι ἅμα,
\vs{9}ἦλθον ἐπὶ τὸν τόπον, ὃν εἶπεν αὐτῷ ὁ Θεός· καὶ ᾠκοδόμησεν ἐκεῖ Ἁβραὰμ τὸ θυσιαστήριον, καὶ ἐπέθηκε τὰ ξύλα· καὶ συμποδίσας Ἰσαὰκ τὸν υἱὸν αὐτοῦ, ἐπέθηκεν αὐτὸν ἐπὶ τὸ θυσιαστήριον ἐπάνω τῶν ξύλων.
\vs{10}Καὶ ἐξέτεινεν Ἁβραὰμ τὴν χεῖρα αὐτοῦ λαβεῖν τὴν μάχαιραν, σφάξαι τὸν υἱὸν αὐτοῦ.
\vs{11}Καὶ ἐκάλεσεν αὐτὸν Ἄγγελος Κυρίου ἐκ τοῦ οὐρανοῦ, καὶ εἶπεν, Ἁβραὰμ, Ἁβραάμ· ὁ δὲ εἶπεν, ἰδοὺ ἐγώ.
\vs{12}Καὶ εἶπε, μὴ ἐπιβάλῃς τὴν χεῖρά σου ἐπὶ τὸ παιδάριον, μηδὲ ποιήσῃς αὐτῷ μηδέν· νῦν γὰρ ἔγνων, ὅτι φοβῇ σὺ τὸν Θεόν· καὶ οὐκ ἐφείσω τοῦ υἱοῦ σου τοῦ ἀγαπητοῦ διʼ ἐμέ.
\vs{13}Καὶ ἀναβλέψας Ἁβραὰμ τοῖς ὀφθαλμοῖς αὐτοῦ εἶδε, καὶ ἰδοὺ κριὸς εἷς κατεχόμενος ἐν φυτῷ Σαβὲκ τῶν κεράτων. Καὶ ἐπορεύθη Ἁβραὰμ, καὶ ἔλαβε τὸν κριὸν, καὶ ἁνήνεγκεν αὐτὸν εἰς ὁλοκάρπωσιν ἀντὶ Ἰσαὰκ τοῦ υἱοῦ αὐτοῦ.

\vs{14}Καὶ ἐκάλεσεν Ἁβραὰμ τὸ ὄνομα τοῦ τόπου ἐκείνου, Κύριος εἶδεν· ἵνα εἴπωσιν σήμερον, ἐν τῷ ὄρει Κύριος ὤφθη.
\vs{15}Καὶ ἐκάλεσεν Ἄγγελος Κυρίου τὸν Ἁβραὰμ δεύτερον ἐκ τοῦ οὐρανοῦ,
\vs{16}λέγων, Κατʼ ἐμαυτοῦ ὤμοσα, λέγει Κύριος, οὗ εἵνεκεν ἐποίησας τὸ ῥῆμα τοῦτο, καὶ οὐκ ἐφείσω τοῦ υἱοῦ σου τοῦ ἀγαπτοῦ διʼ ἐμὲ,
\vs{17}Ἦ μὴν εὐλογῶν εὐλογήσω σε, καὶ πληθύνων πληθυνῶ τὸ σπέρμα σου, ὡς τοὺς ἀστέρας τοῦ οὐρανοῦ, καὶ ὡς τὴν ἄμμον τὴν παρὰ τὸ χεῖλος τῆς θαλάσσης· καὶ κληρονομήσει τὸ σπέρμα σου τὰς πόλεις τῶν ὑπεναντίων.
\vs{18}Καὶ ἐνευλογηθήσονται ἐν τῷ σπέρματί σου πάντα τὰ ἔθνη τῆς γῆς, ἀνθʼ ὧν ὑπήκουσας τῆς ἐμῆς φωνῆς.
\vs{19}Ἀπεστράφη δὲ Ἁβραὰμ πρὸς τοὺς παῖδας αὐτοῦ· καὶ ἀναστάντες ἐπορεύθησαν ἅμα ἐπὶ τὸ φρέαρ τοῦ ὅρκου. Καὶ κατῴκησεν Ἁβραὰμ ἐπὶ τὸ φρέαρ τοῦ ὅρκου.

\vs{20}Ἐγένετο δὲ μετὰ τὰ ῥήματα ταῦτα, καὶ ἀνηγγέλη τῷ Ἁβραὰμ, λέγοντες, ἰδοὺ τέτοκε Μελχὰ καὶ αὐτὴ υἱοὺς τῷ Ναχὼρ τῷ ἀδελφῷ σου,
\vs{21}τὸν Οὒζ πρωτότοκον, καὶ τὸν Βαὺξ ἀδελφὸν αὐτοῦ, καὶ τὸν Καμουὴλ πατέρα Σύρων,
\vs{22}καὶ τὸν Χαζὰδ, καὶ Ἀζαῦ, καὶ τὸν Φαλδὲς, καὶ τὸν Ἰελδὰφ, καὶ τὸν Βαθουήλ.
\vs{23}Βαθουὴλ δὲ ἐγέννησε τὴν Ῥεβέκκαν. ὀκτὼ οὗτοι υἱοὶ, οὓς ἔτεκε Μελχὰ τῷ Ναχὼρ τῷ ἀδελφῷ Ἁβραάμ.
\vs{24}Καὶ ἡ παλλακὴ αὐτοῦ, ᾗ ὄνομα Ῥεύμα, ἔτεκε καὶ αὐτὴ τὸν Ταβὲκ, καὶ τὸν Ταὰμ, καὶ τὸν Τοχός, καὶ τὸν Μοχά.

\ch{23}
Ἐγένετο δὲ ἡ ζωὴ Σάῤῥας, ἔτη ἑκατὸν εἰκοσιεπτά.
\vs{2}Καὶ ἀπέθανε Σάῤῥα ἐν πόλει Ἀρβὸκ, ἥ ἐστιν ἐν τῷ κοιλώματι· αὕτη ἔστι Χεβρὼν ἐν τῇ γῇ Χαναάν ἦλθε δὲ Ἁβραὰμ κόψασθαι Σάῤῥαν, καὶ πενθῆσαι.
\vs{3}Καὶ ἀνέστη Ἁβραὰμ ἀπὸ τοῦ νεκροῦ αὐτοῦ· καὶ εἶπεν Ἁβραὰμ τοῖς υἱοῖς τοῦ Χὲτ, λέγων,
\vs{4}Πάροικος καὶ παρεπίδημος ἐγώ εἰμι μεθʼ ὑμῶν· δότε μοι οὖν κτῆσιν τάφου μεθʼ ὑμῶν, καὶ θάψω τὸν νεκρόν μου ἀπʼ ἐμοῦ.
\vs{5}Ἀπεκρίθησαν δὲ οἱ υἱοὶ Χὲτ πρὸς Ἁβραὰμ, λέγοντες, μὴ, κύριε.
\vs{6}Ἄκουσον δὲ ἡμῶν· βασιλεὺς παρὰ Θεοῦ σὺ εἶ ἐν ἡμῖν· ἐν τοῖς ἐκλεκτοῖς μνημείοις ἡμῶν θάψον τὸν νεκρόν σου· οὐδεὶς γὰρ ἡμῶν οὐ μὴ κωλύσει τὸ μνημεῖον αὐτοῦ ἀπὸ σοῦ, τοῦ θάψαι τὸν νεκρόν σου ἐκεῖ.
\vs{7}Ἀναστὰς δὲ Ἁβραὰμ προσεκύνησε τῷ λαῷ τῆς γῆς, τοῖς υἱοῖς τοῦ Χέτ.
\vs{8}Καὶ ἐλάλησε πρὸς αὐτοὺς Ἁβραὰμ, λέγων, εἰ ἔχετε τῇ ψυχῇ ὑμῶν, ὥστε θάψαι τὸν νεκρόν μου ἀπὸ προσώπου μου, ἀκούσατέ μου, καὶ λαλήσατε περὶ ἐμοῦ Ἐφρὼν τῷ τοῦ Σαάρ.
\vs{9}Καὶ δότω μοι τὸ σπήλαιον τὸ διπλοῦν, ὅ ἐστιν αὐτῷ, τὸ ὂν ἐν μέρει τοῦ ἀγροῦ αὐτοῦ· ἀργυρίου τοῦ ἀξίου δότε μοι αὐτὸ ἐν ὑμῖν εἰς κτῆσιν μνημείου.
\vs{10}Ἐφρὼν δὲ ἐκάθητο ἐν μέσῳ τῶν υἱῶν Χέτ· ἀποκριθεὶς δὲ Ἐφρὼν ὁ Χετταῖος πρὸς Ἁβραὰμ εἶπεν, ἀκουόντων τῶν υἱῶν Χὲτ, καὶ τῶν εἰσπορευομένων εἰς τὴν πόλιν πάντων, λέγων,
\vs{11}Παρʼ ἐμοὶ γενοῦ, κύριε, καὶ ἄκουσόν μου· τὸν ἀγρὸν, καὶ τὸ σπήλαιον τὸ ἐν αὐτῷ, σοὶ δίδωμι· ἐναντίον πάντων τῶν πολιτῶν μου δέδωκά σοι· θάψον τὸν νεκρόν σου.
\vs{12}Καὶ προσεκύνησεν Ἁβραὰμ ἐναντίον τοῦ λαοῦ τῆς γῆς.
\vs{13}Καὶ εἶπε τῷ Ἐφρὼν εἰς τὰ ὦτα ἐναντίον τοῦ λαοῦ τῆς γῆς, ἐπειδὴ πρὸς ἐμοῦ εἶ, ἄκουσόν μου· τὸ ἀργύριον τοῦ ἀγροῦ λάβε παρʼ ἐμοῦ, καὶ θάψω τὸν νεκρόν μου ἐκεῖ.
\vs{14}Ἀπεκρίθη δὲ Ἐφρὼν τῷ Ἁβραὰμ, λέγων,
\vs{15}Οὐχὶ, κύριε· ἀκήκοα γὰρ, γῆ τετρακοσίων διδράχμων ἀργύριου· ἀλλὰ τί ἂν εἴη τοῦτο ἀνὰ μέσον ἐμοῦ καὶ σοῦ; σὺ δὲ τὸν νεκρόν σου θάψον.
\vs{16}καὶ ἤκουσεν Ἁβραὰμ τοῦ Ἐφρών· καὶ ἀπεκατέστησεν Ἁβραὰμ τῷ Ἐφρὼν τὸ ἀργύριον, ὃ ἐλάλησεν εἰς τὰ ὦτα τῶν υἱῶν Χὲτ, τετρακόσια δίδραχμα ἀργυρίου δοκίμου ἐμπόροις.
\vs{17}Καὶ ἔστη ὁ ἀγρὸς Ἐφρών, ὃς ἦν ἐν τῷ διπλῷ σπηλαίῳ, ὅς ἐστι κατὰ πρόσωπον Μαμβρῆ, ὁ ἀγρὸς καὶ τὸ σπήλαιον, ὃ ἦν ἐν αὐτῷ, καὶ πᾶν δένδρον, ὃ ἦν ἐν τῷ ἀγρῷ, καὶ πᾶν ὅ ἐστιν ἐν τοῖς ὁρίοις αὐτοῦ κύκλῳ,
\vs{18}τῷ Ἁβραὰμ, εἰς κτῆσιν ἐναντίον τῶν υἱῶν Χὲτ, καὶ πάντων τῶν εἰσπορευομένων εἰς τὴν πόλιν.
\vs{19}Μετὰ ταῦτα ἔθαψεν Ἁβραὰμ Σάῤῥαν τὴν γυναῖκα αὐτοῦ ἐν τῷ σπηλαίῳ τοῦ ἀγροῦ τῷ διπλῷ, ὅ ἐστιν ἀπέναντι Μαμβρῆ· αὕτη ἐστὶ Χεβρὼν ἐν τῇ γῇ Χαναάν.
\vs{20}Καὶ ἐκυρώθη ὁ ἀγρὸς καὶ τὸ σπήλαιον ὃ ἦν ἐν αὐτῷ τῷ Ἁβραὰμ εἰς κτῆσιν τάφου, παρὰ τῶν υἱῶν Χέτ.

\ch{24}Καὶ Ἁβραὰμ ἦν πρεσβύτερος προβεβηκὼς ἡμερῶν· καὶ Κύριος ηὐλόγησε τὸν Ἁβραὰμ κατὰ πάντα.

\vs{2}Καὶ εἶπεν Ἁβραὰμ τῷ παιδὶ αὐτοῦ τῷ πρεσβυτέρῳ τῆς οἰκίας αὐτοῦ, τῷ ἄρχοντι πάντων τῶν αὐτοῦ, θὲς τὴν χεῖρά σου ὑπὸ τὸν μηρόν μου.
\vs{3}Καὶ ἐξορκιῶ σε Κύριον τὸν Θεὸν τοῦ οὐρανοῦ καὶ τὸν Θεὸν τῆς γῆς, ἵνα μὴ λάβῃς γυναῖκα τῷ υἱῷ μου Ἰσαὰκ ἀπὸ τῶν θυγατέρων τῶν Χαναναίων, μεθʼ ὧν ἐγὼ οἰκῶ ἐν αὐτοις.
\vs{4}Ἀλλʼ ἢ εἰς τὴν γῆν μου, οὗ ἐγεννήθην, πορεύσῃ, καὶ εἰς τὴν φυλήν μου, καὶ λήψῃ γυναῖκα τῷ υἱῷ μου Ἰσαὰκ ἐκεῖθεν.
\vs{5}Εἶπε δὲ πρὸς αὐτὸν ὁ παῖς, μή ποτε οὐ βούληται ἡ γυνὴ πορευθῆναι μετʼ ἐμοῦ ὀπίσω εἰς τὴν γῆν ταύτην, ἀποστρέψω τὸν υἱόν σου εἰς τὴν γῆν, ὅθεν ἐξῆλθες ἐκεῖθεν;
\vs{6}Εἶπε δὲ πρὸς αὐτὸν Ἁβραάμ, πρόσεχε σεαυτῷ μὴ ἀποστρέψῃς τὸν υἱόν μου ἐκεῖ.
\vs{7}Κύριος ὁ Θεὸς τοῦ οὐρανοῦ καὶ ὁ Θεὸς τῆς γῆς, ὃς ἔλαβέ με ἐκ τοῦ οἴκου τοῦ πατρός μου, καἰ ἐκ τῆς γῆς ἧς ἐγεννήθην, ὃς ἐλάλησέ μοι, καὶ ὃς ὤμοσέ μοι, λέγων, σοὶ δώσω τὴν γῆν ταύτην καὶ τῷ σπέρματί σου, αὐτὸς ἀποστελεῖ τὸν Ἄγγελον αὐτοῦ ἔμπροσθέν σου, καὶ λήψῃ γυναῖκα τῷ υἱῷ μου ἐκεῖθεν.
\vs{8}Ἐὰν δὲ μὴ θέλῃ ἡ γυνὴ πορευθῆναι μετὰ σοῦ εἰς τὴν γῆν ταύτην, καθαρὸς ἔσῃ ἀπὸ τοῦ ὅρκου μου· μόνον τὸν υἱόν μου μὴ ἀποστρέψῃς ἐκεῖ.
\vs{9}Καὶ ἔθηκεν ὁ παῖς τὴν χεῖρα αὐτοῦ ὑπὸ τὸν μηρὸν Ἁβραὰμ τοῦ κυρίου αὐτοῦ, καὶ ὤμοσεν αὐτῷ περὶ τοῦ ῥήματος τούτου.
\vs{10}Καὶ ἔλαβεν ὁ παῖς δέκα καμήλους ἀπὸ τῶν καμήλων τοῦ κυρίου αὐτοῦ, καὶ ἀπὸ πάντων τῶν ἀγαθῶν τοῦ κυρίου αὐτοῦ μεθʼ ἑαυτοῦ· καὶ ἀναστὰς ἐπορεύθη εἰς τὴν Μεσοποταμίαν εἰς τὴν πόλιν Ναχώρ.
\vs{11}Καὶ ἐκοίμησε τὰς καμήλους ἔξω τῆς πόλεως παρὰ τὸ φρέαρ τοῦ ὕδατος τὸ πρὸς ὀψέ, ἡνίκα ἐκπορεύονται αἱ ὑδρευόμεναι.

\vs{12}Καὶ εἶπε, Κύριε ὁ Θεὸς τοῦ κυρίου μου Ἁβραάμ, εὐόδωσον ἐναντίον ἐμοῦ σήμερον, καὶ ποίησον ἔλεος μετὰ τοῦ κυρίου μου Ἁβραάμ.
\vs{13}Ἰδοὺ ἐγὼ ἕστηκα ἐπὶ τῆς πηγῆς τοῦ ὕδατος· αἱ δὲ θυγατέρες τῶν οἰκούντων τὴν πόλιν ἐκπορεύονται ἀντλῆσαι ὕδωρ.
\vs{14}Καὶ ἔσται ἡ παρθένος ᾗ ἂν ἐγὼ εἴπω, ἐπίκλινον τὴν ὑδρίαν σου, ἵνα πίω, καὶ εἴπῃ μοι, πίε σύ, καὶ τὰς καμήλους σου ποτιῶ, ἕως ἂν παύσωνται πίνουσαι, ταύτην ἡτοίμασας τῷ παιδί σου τῷ Ἰσαάκ· καὶ ἐν τούτῳ γνώσομαι, ὅτι ἐποίησας ἔλεος μετὰ τοῦ κυρίου μου Ἁβραάμ.

\vs{15}Καὶ ἐγένετο πρὸ τοῦ συντελέσαι αὐτὸν λαλοῦντα ἐν τῇ διανοίᾳ αὐτοῦ, καὶ ἰδοὺ Ῥεβέκκα ἐξεπορεύετο ἡ τεχθεῖσα Βαθουήλ, υἱῷ Μελχὰς τῆς γυναικὸς Ναχώρ, ἀδελφοῦ δὲ Ἁβραάμ, ἔχουσα τὴν ὑδρίαν ἐπὶ τῶν ὤμων αὐτῆς.
\vs{16}Ἡ δὲ παρθένος ἦν καλὴ τῇ ὄψει σφόδρα· παρθένος ἦν, ἀνὴρ οὐκ ἔγνω αὐτήν· καταβᾶσα δὲ ἐπὶ τὴν πηγὴν, ἔπλησε τὴν ὑδρίαν αὐτῆς, καὶ ἀνέβη.
\vs{17}Ἐπέδραμε δὲ ὁ παῖς εἰς συνάντησιν αὐτῆς, καὶ εἶπε, Πότισόν με μικρὸν ὕδωρ ἐκ τῆς ὑδρίας σου.
\vs{18}Ἡ δὲ εἶπε, πίε, κύριε· καὶ ἔσπευσε καὶ καθεῖλε τὴν ὑδρίαν ἐπὶ τὸν βραχίονα αὐτῆς, καὶ ἐπότισεν αὐτὸν, ἕων ἐπαύσατο πίνων.
\vs{19}Καὶ εἶπε, καὶ ταῖς καμήλοις σου ὑδρεύσομαι, ἕως ἂν πᾶσαι πίωσι.
\vs{20}Καὶ ἔσπευσε καὶ ἐξεκένωσε τὴν ὑδρίαν εἰς τὸ ποτιστήριον· καὶ ἔδραμεν ἐπὶ τὸ φρέαρ ἀντλῆσαι πάλιν· καὶ ὑδρεύσατο πάσαις ταῖς καμήλοις.
\vs{21}Ὁ δὲ ἄνθρωπος κατεμάνθανεν αὐτήν· καὶ παρεσιώπα τοῦ γνῶναι εἰ εὐώδωκε Κύριος τὴν ὁδὸν αὐτοῦ, ἢ οὔ.
\vs{22}Ἐγένετο δὲ ἡνίκα ἐπαύσαντο πᾶσαι αἱ κάμηλοι πίνουσαι, ἔλαβεν ὁ ἄνθρωπος ἐνώτια χρυσᾶ ἀνὰ δραχμὴν ὁλκῆς, καὶ δύο ψέλλια ἐπὶ τὰς χεῖρας αὐτῆς, δέκα χρυσῶν ὁλκὴ αὐτῶν.
\vs{23}Καὶ ἐπηρώτησεν αὐτὴν, καὶ εἶπε, θυγάτηρ τίνος εἶ; ἀνάγγειλόν μοι, εἰ ἔστι παρὰ τῷ πατρί σου τόπος ἡμῖν του καταλῦσαι.
\vs{24}Ἡ δὲ εἶπεν αὐτῷ, θυγάτηρ Βαθουήλ εἰμι τοῦ Μελχάς, ὃν ἔτεκε τῷ Ναχώρ.
\vs{25}Καὶ εἶπεν αὐτῷ, Καὶ ἄχυρα καὶ χορτάσματα πολλὰ παρʼ ἡμῖν, καὶ τόπος τοῦ καταλῦσαι.
\vs{26}Καὶ εὐδοκήσας ὁ ἄνθρωπος προσεκύνησε τῷ Κυρίῳ
\vs{27}Καὶ εἶπεν, εὐλογητὸς Κύριος ὁ Θεὸς τοῦ κυρίου μου Ἁβραάμ, ὃς οὐκ ἐγκατέλειπε τὴν δικαιοσύνην αὐτοῦ, καὶ τὴν ἀλήθειαν, ἀπὸ τοῦ κυρίου μου· ἐμὲ τʼ εὐώδωκε Κύριος εἰς οἶκον τοῦ ἀδελφοῦ τοῦ κυρίου μου.
\vs{28}Καὶ δραμοῦσα ἡ παῖς ἀνήγγειλεν εἰς τὸν οἶκον τῆς μητρὸς αὐτῆς, κατὰ τὰ ῥήματα ταῦτα.
\vs{29}Τῇ δὲ Ῥεβέκκᾷ ἀδελφὸς ἦν, ᾧ ὄνομα Λάβαν· καὶ ἔδραμε Λάβαν πρὸς τὸν ἄνθρωπον ἔξω ἐπὶ τὴν πηγήν.
\vs{30}Καὶ ἐγένετο ἡνίκα εἶδε τὰ ἐνώτια, καὶ τὰ ψέλλια ἐν ταῖς χερσὶ τῆς ἀδελφῆς αὐτοῦ, καὶ ὅτε ἤκουσε τὰ ῥήματα Ῥεβέκκας τῆς ἀδελφῆς αὐτοῦ, λεγούσης, οὕτω λελάληκέ μοι ὁ ἄνθρωπος, καὶ ἦλθε πρὸς τὸν ἄνθρωπον, ἑστηκότος αὐτοῦ ἐπὶ τῶν καμήλων ἐπὶ τῆς πηγῆς.
\vs{31}Καὶ εἶπεν αὐτῷ, δεῦρο εἴσελθε, εὐλογητὸς Κυροίυ· ἱνατί ἕστηκας ἔξω; ἐγὼ δὲ ἡτοίμασα τὴν οἰκίαν, καὶ τόπον ταῖς καμήλοις.
\vs{32}Εἰσῆλθε δὲ ὁ ἄνθρωπος εἰς τὴν οἰκίαν, καὶ ἀπέσαξε τὰς καμήλους· καὶ ἔδωκεν ἄχυρα καὶ χορτάσματα ταῖς καμήλοις, καὶ ὕδωρ νίψασθαι τοῖς ποσὶν αὐτοῦ, καὶ τοῖς ποσὶ τῶν ἀνδρῶν τῶν μετʼ αὐτοῦ.
\vs{33}Καὶ παρέθηκεν αὐτοῖς ἄρτους φαγεῖν· καὶ εἶπεν, οὐ μὴ φάγω, ἕως τοῦ λαλῆσαί με τὰ ῥήματά μου· καὶ εἶπεν, λάλησον.

\vs{34}Καὶ εἶπε, παῖς Ἁβραὰμ ἐγώ εἰμι.
\vs{35}Κύριος δὲ ηὐλόγησε τὸν κύριόν μου σφόδρα, καὶ ὑψώθη· καὶ ἔδωκεν αὐτῷ πρόβατα, καὶ μόσχους, καὶ ἀργύριον, καὶ χρυσίον, παῖδας, καὶ παιδίσκας, καμήλους, καὶ ὄνους.
\vs{36}Καὶ ἔτεκε Σάῤῥα ἡ γυνὴ τοῦ κυρίου μου υἱὸν ἕνα τῷ κυρίῳ μου μετὰ τὸ γηράσαι αὐτόν· καὶ ἔδωκεν αὐτῷ ὅσα ἦν αὐτῷ.
\vs{37}Καὶ ὥρκισέ με ὁ κύριός μου, λέγων, οὐ λήμψῃ γυναῖκα τῷ υἱῷ μου ἀπὸ τῶν θυγατέρων τῶν Χαναναίων, ἐν οἷς ἐγὼ παροικῶ ἐν τῇ γῇ αὐτῶν.
\vs{38}Ἀλλʼ εἰς τὸν οἶκον τοῦ πατρός μου πορεύσῃ, καὶ εἰς τὴν φυλήν μου, καὶ λήψῃ γυναῖκα τῷ υἱῷ μου ἐκεῖθεν.
\vs{39}Εἶπα δὲ τῷ κυρίῳ μου, μήποτε οὐ πορεύσεται ἡ γυνὴ μετʼ ἐμοῦ.
\vs{40}Καὶ εἶπέ μοι, Κύριος ὁ Θεὸς ᾧ εὐηρέστησα ἐναντίον αὐτοῦ, αὐτὸς ἐξαποστελεῖ τὸν Ἀγγελον αὐτοῦ μετὰ σοῦ, καὶ εὐοδώσει τὴν ὁδόν σου· καὶ λήψῃ γυναῖκα τῷ υἱῷ μου ἐκ τῆς φυλῆς μου, καὶ ἐκ τοῦ οἴκου τοῦ πατρός μου.
\vs{41}Τότε ἀθῷος ἔσῃ ἀπὸ τῆς ἀρᾶς μου· ἡνίκα γὰρ ἐὰν ἔλθῃς εἰς τὴν φυλήν μου, καὶ μή σοι δῶσι, καὶ ἔσῃ ἀθῷος ἀπὸ τοῦ ὁρκισμοῦ μου.
\vs{42}Καὶ ἐλθὼν σήμερον ἐπὶ τὴν πηγὴν εἶπα, Κύριε ὁ Θεὸς τοῦ κυρίου μου Ἁβραὰμ, εἰ σὺ εὐοδοῖς τὴν ὁδόν μου, ἐν ᾗ νῦν ἐγὼ πορεύομαι ἐν αὐτῇ,
\vs{43}ἰδοὺ ἐγὼ ἐφέστηκα ἐπὶ τῆς πηγῆς τοῦ ὕδατος, καὶ αἱ θυγατέρες τῶν ἀνθρώπων τῆς πόλεως ἐκπορεύονται ἀντλῆσαι ὕδωρ· καὶ ἔσται ἡ παρθένος, ᾗ ἂν ἐγὼ εἴπω, πότισόν με ἐκ τῆς ὑδρίας σου μικρὸν ὕδωρ,
\vs{44}καὶ εἴπῃ μοι, καὶ σὺ πίε, καὶ ταῖς καμήλοις σου ὑδρεύσομαι, αὕτη ἡ γυνὴ ἣν ἡτοίμασε Κύριος τῷ ἑαυτοῦ θεράποντι Ἰσαάκ· καὶ ἐν τούτῳ γνώσομαι, ὅτι πεποίηκας ἔλεος τῷ κυρίῳ μου Ἁβραάμ.
\vs{45}Καὶ ἐγένετο πρὸ τοῦ συντελέσαι με λαλοῦντα ἐν τῇ διανοίᾳ μου, εὐθὺς Ῥεβέκκα ἐξεπορεύετο, ἔχουσα τὴν ὑδρίαν ἐπὶ τῶν ὤμων· καὶ κατέβη ἐπὶ τὴν πηγὴν, καὶ ὑδρεύσατο· εἶπα δὲ αὐτῇ, πότισόν με.
\vs{46}Καὶ σπεύσασα καθεῖλε τὴν ὑδρίαν ἐπὶ τὸν βραχίονα αὐτῆς ἀφʼ ἑαυτῆς, καὶ εἶπε, πίε σὺ, καὶ τὰς καμήλους σου ποτιῶ· καὶ ἔπιον, καὶ τὰς καμήλους ἐπότισε.
\vs{47}Καὶ ἠρώτησα αὐτὴν, καὶ εἶπα, θυγάτηρ τίνος εἶ, ἀναγγειλόν μοι· ἡ δὲ ἔφη, θυγάτηρ Βαθουὴλ εἰμὶ υἱοῦ τοῦ Ναχὼρ, ὃν ἔτεκεν αὐτῷ Μελχά· καὶ περιέθηκα αὐτῇ τὰ ἐνώτια, καὶ τὰ ψέλλια περὶ τὰς χεῖρας αὐτῆς.
\vs{48}Καὶ εὐδοκήσας προσεκύνησα τῷ Κυρίῳ, καὶ εὐλόγησα Κύριον τὸν Θεὸν τοῦ κυρίου μου Ἁβραὰμ, ὃς εὐώδωσέ με ἐν ὁδῷ ἀληθείας λαβεῖν τὴν θυγατέρα τοῦ ἀδελφοῦ τοῦ κυρίου μου τῷ υἱῷ αὐτοῦ.
\vs{49}Εἰ οὖν ποιεῖτε ὑμεῖς ἔλεος καὶ δικαιοσύνην πρὸς τὸν κύριόν μου· εἰ δὲ μὴ, ἀπαγγείλατέ μοι, ἵνα ἐπιστρέψω εἰς δεξιὰν ἤ ἀριστεράν.

\vs{50}Ἀποκριθεὶς δὲ Λάβαν καὶ Βαθουὴλ εἶπαν, παρὰ κυρίου ἐξῆλθε τὸ πρᾶγμα τοῦτο· οὐ δυνησόμεθά σοι ἀντειπεῖν κακὸν ἢ καλόν.
\vs{51}Ἰδοὺ Ῥεβέκκα ἐνώπιόν σου· λαβὼν ἀπότρεχε· καὶ ἔστω γυνὴ τῷ υἱῷ τοῦ κυρίου σου, καθὰ ἐλάλησε Κύριος.
\vs{52}Ἐγένετο δὲ ἐν τῷ ἀκοῦσαι τὸν παῖδα τοῦ Ἁβραὰμ τῶν ῥημάτων αὐτῶν, προσεκύνησεν ἐπὶ τὴν γῆν τῷ κυρίῳ.
\vs{53}καὶ ἐξενέγκας ὁ παῖς σκεύη ἀργυρᾶ καὶ χρυσᾶ καὶ ἱματισμὸν, ἔδωκε τῇ Ῥεβέκκᾳ· καὶ δῶρα ἔδωκε τῷ ἀδελφῷ αὐτῆς, καὶ τῇ μητρὶ αὐτῆς.
\vs{54}Καὶ ἔφαγον καὶ ἔπιον καὶ αὐτὸς καὶ οἱ ἄνδρες οἱ μετʼ αὐτοῦ ὄντες, καὶ ἐκοιμήθησαν· καὶ ἀναστὰς τὸ πρωῒ εἶπεν, ἐκπέμψατέ με, ἵνα ἀπέλθω πρὸς τὸν κύριόν μου.
\vs{55}Εἶπαν δὲ οἱ ἀδελφοὶ αὐτῆς, καὶ ἡ μήτηρ, μεινάτω ἡ παρθένος μεθʼ ἡμῶν ἡμέρας ὡσεὶ δέκα, καὶ μετὰ ταῦτα ἀπελεύσεται.
\vs{56}Ὁ δὲ εἶπε πρὸς αὐτοὺς, μὴ κατέχετέ με· καὶ Κύριος εὐώδωσε τὴν ὁδόν μου ἐν ἐμοί· ἐκπέμψατέ με, ἵνα ἀπέλθω πρὸς τὸν κύριόν μου.
\vs{57}Οἱ δὲ εἶπαν, Καλέσωμεν τὴν παῖδα, καὶ ἐρωτήσωμεν τὸ στόμα αὐτῆς.
\vs{58}Καὶ ἐκάλεσαν τὴν Ῥεβέκκαν, καὶ εἶπαν αὐτῇ, πορεύσῃ μετὰ τοῦ ἀνθρώπου τούτου; ἡ δὲ εἶπε, πορεύσομαι.
\vs{59}Καὶ ἐξέπεμψαν Ῥεβέκκαν τὴν ἀδελφὴν αὐτῶν, καὶ τὰ ὑπάρχοντα αὐτῆς, καὶ τὸν παῖδα τοῦ Ἁβραὰμ, καὶ τοὺς μετʼ αὐτοῦ.
\vs{60}Καὶ εὐλόγησαν Ῥεβέκκαν, καὶ εἶπαν αὐτῇ, ἀδελφὴ ἡμῶν εἶ, γίνου εἰς χιλιάδας μυριάδων, καὶ κληρονομησάτω τὸ σπέρμα σου τὰς πόλεις τῶν ὑπεναντίων.
\vs{61}Ἀναστᾶσα δὲ Ῥεβέκκα καὶ αἱ ἅβραι αὐτῆς, ἐπέβησαν ἐπὶ τὰς καμήλους, καὶ ἐπορεύθησαν μετὰ τοῦ ἀνθρώπου· καὶ ἀναλαβὼν ὁ παῖς τὴν Ῥεβέκκαν ἀπῆλθεν.

\vs{62}Ἰσαὰκ δὲ διεπορεύετο διὰ τῆς ἐρήμου κατὰ τὸ φρέαρ τῆς ὁράσεως· αὐτὸς δὲ κατῴκει ἐν τῇ γῇ τῇ πρὸς Λίβα.
\vs{63}Καὶ ἐξῆλθεν Ἰσαὰκ ἀδολεσχῆσαι εἰς τὸ πεδίον τὸ πρὸς δείλης, καὶ ἀναβλέψας τοῖς ὀφθαλμοῖς αὐτοῦ εἶδε καμήλους ἐρχομένας.
\vs{64}Καὶ ἀναβλέψασα Ῥεβέκκα τοῖς ὀφθαλμοῖς εἶδε τὸν Ἰσαάκ· καὶ κατεπήδησεν ἀπὸ τῆς καμήλου.
\vs{65}Καὶ εἶπε τῷ παιδὶ, τίς ἐστιν ὁ ἄνθρωπος ἐκεῖνος ὁ πορευόμενος ἐν τῷ πεδίῳ εἰς συνάντησιν ἡμῖν; εἶπε δὲ ὁ παῖς, οὗτός ἐστιν ὁ κύριός μου· ἡ δὲ λαβοῦσα τὸ θέριστρον, περιεβάλετο.
\vs{66}Καὶ διηγήσατο ὁ παῖς τῷ Ἰσαὰκ πάντα τὰ ῥήματα, ἃ ἐποίησεν.
\vs{67}Εἰσῆλθε δὲ Ἰσαὰκ εἰς τὸν οἶκον τῆς μητρὸς αὐτοῦ, καὶ ἔλαβε τὴν Ῥεβέκκαν, καὶ ἐγένετο αὐτοῦ γυνὴ, καὶ ἠγάπησεν αὐτήν· καὶ παρεκλήθη Ἰσαὰκ περὶ Σάῤῥας τῆς μητρὸς αὐτοῦ.

\ch{25}
Προσθέμενος δὲ Ἁβραὰμ ἔλαβε γυναῖκα, ᾗ ὄνομα Χεττούρα.
\vs{2}Ἔτεκε δὲ αὐτῷ τὸν Ζομβρᾶν, καὶ τὸν Ἰεζὰν, καὶ τὸν Μαδὰλ, καὶ τὸν Μαδιὰμ, καὶ τὸν Ἰεσβὼκ, καὶ τὸν Σωίε.
\vs{3}Ἰεζὰν δὲ ἐγέννησε τὸν Σαβὰ, καὶ τὸν Δεδάν· υἱοὶ δὲ Δεδὰν Ἀσσουριεὶμ, καὶ Λατουσιεὶμ, καὶ Λαωμείμ.
\vs{4}Υἱοὶ δὲ Μαδιὰμ Γεφὰρ, καὶ Ἀφεὶρ, καὶ Ἐνὼχ, καὶ Ἀβειδὰ, καὶ Ἐλδαγά· πάντες οὗτοι ἦσαν υἱοὶ Χεττούρας.
\vs{5}Ἔδωκε δὲ Ἁβραὰμ πάντα τὰ ὑπάρχοντα αὐτοῦ Ἰσαὰκ τῷ υἱῷ αὐτοῦ.
\vs{6}Καὶ τοῖς υἱοῖς τῶν παλλακῶν αὐτοῦ ἔδωκεν Ἁβραὰμ δόματα, καὶ ἐξαπέστειλεν αὐτοὺς ἀπὸ Ἰσαὰκ τοῦ υἱοῦ αὐτοῦ, ἔτι ζῶντος αὐτοῦ, πρὸς ἀνατολὰς εἰς γῆν ἀνατολῶν.
\vs{7}Ταῦτα δὲ τὰ ἔτη ἡμερῶν τῆς ζωῆς Ἁβραὰμ ὅσα ἔζησεν, ἑκατὸν ἑβδομηκονταπεντε ἔτη.
\vs{8}Καὶ ἐκλείπων ἀπέθανεν Ἁβραὰμ ἐν γήρᾳ καλῷ πρεσβύτης, καὶ πλήρης ἡμερῶν, καὶ προσετέθη πρὸς τὸν λαὸν αὐτοῦ.
\vs{9}Καὶ ἔθαψαν αὐτὸν Ἰσαὰκ καὶ Ἰσμαὴλ οἱ υἱοὶ αὐτοῦ εἰς τὸ σπήλαιον τὸ διπλοῦν, εἰς τὸν ἀγρὸν Ἐφρων τοῦ Σαὰρ τοῦ Χετταίου, ὅς ἐστιν ἀπέναντι Μαμβρῆ,
\vs{10}τὸν ἀγρὸν καὶ τὸ σπήλαιον, ὃ ἐκτήσατο Ἁβραὰμ παρὰ τῶν υἱῶν τοῦ Χέτ· ἐκεῖ ἔθαψαν Ἁβραὰμ, καὶ Σάῤῥαν τὴν γυναῖκα αὐτοῦ.
\vs{11}Ἐγένετο δὲ μετὰ τὸ ἀποθανεῖν Ἁβραὰμ, εὐλόγησεν ὁ Θεὸς τὸν Ἰσαὰκ υἱὸν αὐτοῦ· καὶ κατῴκησεν Ἰσαὰκ παρὰ τὸ φρέαρ τῆς ὁράσεως.
\vs{12}Αὗται δὲ αἱ γενέσεις Ἰσμαὴλ τοῦ υἱοῦ Ἁβραὰμ, ὃν ἔτεκεν Ἄγαρ ἡ Αἰγυπτία, ἡ παιδίσκη Σάῤῥας, τῷ Ἁβραάμ.
\vs{13}Καὶ ταῦτα τὰ ὀνόματα τῶν υἱῶν Ἰσμαὴλ, κατʼ ὀνόματα τῶν γενεῶν αὐτοῦ· πρωτότοκος Ἰσμαὴλ, καὶ Ναβαϊὼθ, καὶ Κηδὰρ, καὶ Ναβδεὴλ, καὶ Μασσὰμ,
\vs{14}καὶ Μασμὰ, καὶ Δουμὰ, καὶ Μασσῆ,
\vs{15}καὶ Χοδδὰν, καὶ Θαιμὰν, καὶ Ἰετοὺρ, καὶ Ναφὲς, καὶ Κεδμά.
\vs{16}οὗτοί εἰσιν οἱ υἱοὶ Ἰσμαὴλ, καὶ ταῦτα τὰ ὀνόματα αὐτῶν ἐν ταῖς σκηναῖς αὐτῶν, καὶ ἐν ταῖς ἐπαύλεσιν αὐτῶν· δώδεκα ἄρχοντες κατὰ ἔθνη αὐτῶν.
\vs{17}Καὶ ταῦτα τὰ ἔτη τῆς ζωῆς Ἰσμαὴλ, ἑκατὸν τριακονταεπτὰ ἔτη· καὶ ἐκλείπων ἀπέθανε, καὶ προσετέθη πρὸς τὸ γένος αὐτοῦ.
\vs{18}Κατῴκησε δὲ ἀπὸ Εὐϊλὰτ ἕως Σοὺρ, ἥ ἐστι κατὰ πρόσωπον Αἰγύπτου ἕως ἐλθεῖν πρὸς Ἀσσυρίους· κατὰ πρόσωπον πάντων τῶν ἀδελφῶν αὐτοῦ κατῴκησε.

\vs{19}Καὶ αὗται αἱ γενέσεις Ἰσαὰκ τοῦ υἱοῦ Ἁβραάμ· Ἁβραάμ ἐγέννησε τὸν Ἰσαάκ.
\vs{20}Ἦν δὲ Ἰσαὰκ ἐτῶν τεσσαράκοντα ὅτε ἔλαβε τὴν Ῥεβέκκαν θυγατέρα Βαθουὴλ τοῦ Σύρου ἐκ τῆς Μεσοποταμίας Συρίας, ἀδελφὴν Λάβαν τοῦ Σύρου, ἑαυτῷ εἰς γυναῖκα.
\vs{21}Ἐδέετο δὲ Ἰσαὰκ Κυρίου περὶ Ῥεβέκκας τῆς γυναικὸς αὐτοῦ, ὅτι στεῖρα ἦν· ἐπήκουσε δὲ αὐτοῦ ὁ Θεὸς, καὶ συνέλαβεν ἐν γαστρὶ Ῥεβέκκα ἡ γυνὴ αὐτοῦ.
\vs{22}Ἐσκίρτων δὲ τὰ παιδία ἐν αὐτῇ· εἶπε δὲ, εἰ οὕτω μοι μέλλει γίνεσθαι, ἵνα τί μοι τοῦτο; ἐπορεύθη δὲ πυθέσθαι παρὰ Κυρίου.
\vs{23}Καὶ εἶπε Κύριος αὐτῇ, δύο ἔθνη ἐν γαστρί σου εἰσὶ, καὶ δύο λαοὶ ἐκ τῆς κοιλίας σου διασταλήσονται· καὶ λαὸς λαοῦ ὑπερέξει, καὶ ὁ μείζων δουλεύσει τῷ ἐλάσσονι.
\vs{24}Καὶ ἐπληρώθησαν αἱ ἡμέραι τοῦ τεκεῖν αὐτήν· καὶ τῇδε ἦν δίδυμα ἐν τῇ κοιλίᾳ αὐτῆς.
\vs{25}Ἐξῆλθε δὲ ὁ πρωτότοκος πυῤῥάκης· ὅλος, ὡσεὶ δορὰ, δασύς· ἐπωνόμασε δὲ τὸ ὄνομα αὐτοῦ, Ἡσαῦ.
\vs{26}Καὶ μετὰ τοῦτο ἐξῆλθεν ὁ ἀδελφὸς αὐτοῦ, καὶ ἡ χεὶρ αὐτοῦ ἐπειλημμένη τῆς πτέρνης Ἡσαῦ· καὶ ἐκάλεσε τὸ ὄνομα αὐτοῦ, Ἰακώβ. Ἰσαὰκ δὲ ἦν ἐτῶν ἑξήκοντα, ὅτε ἔτεκεν αὐτοὺς Ῥεβέκκα.
\vs{27}Ηὐξήθησαν δὲ οἱ νεανίσκοι· καὶ ἦν Ἡσαῦ ἄνθρωπος εἰδὼς κυνηγεῖν, ἄγροικος· Ἰακὼβ δὲ ἄνθρωπος ἄπλαστος, οἰκῶν οἰκίαν.
\vs{28}Ἠγάπησε δὲ Ἰσαὰκ τὸν Ἡσαῦ, ὅτι ἡ θήρα αὐτοῦ βρῶσις αὐτῷ· Ῥεβέκκα δὲ ἠγάπα τὸν Ἰακώβ.

\vs{29}Ἥψησε δὲ Ἰακὼβ ἕψημα· ἦλθε δὲ Ἡσαῦ ἐκ τοῦ πεδίου ἐκλείπων.
\vs{30}Καὶ εἶπεν Ἡσαῦ τῷ Ἰακὼβ, γεῦσόν με ἀπὸ τοῦ ἑψήματος πυῤῥου τούτου, ὅτι ἐκλείπω· διὰ τοῦτο ἐκλήθη τὸ ὄνομα αὐτοῦ, Ἐδώμ.
\vs{31}Εἶπε δὲ Ἰακὼβ τῷ Ἡσαῦ, ἀπόδου μοι σήμερον τὰ πρωτοτόκιά σου ἐμοί.
\vs{32}Καὶ εἶπεν Ἡσαῦ, ἰδοὺ ἐγὼ πορεύομαι τελευτᾷν· καὶ ἵνα τί μοι ταῦτα τὰ πρωτοτόκια;
\vs{33}Καὶ εἶπεν αὐτῷ Ἰακὼβ, ὄμοσόν μοι σήμερον· καὶ ὤμοσεν αὐτῷ· ἀπέδοτο δὲ Ἡσαῦ τὰ πρωτοτόκια τῷ Ἰακώβ.
\vs{34}Ἰακὼβ δὲ ἔδωκε τῷ Ἠσαῦ ἄρτον, καὶ ἕψημα φακοῦ· καὶ ἔφαγε καὶ ἔπιε, καὶ ἀναστὰς ᾤχετο· καὶ ἐφαύλισεν Ἡσαῦ τὰ πρωτοτόκια.

\ch{26}
Ἐγένετο δὲ λιμὸς ἐπὶ τῆς γῆς, χωρὶς τοῦ λιμοῦ τοῦ πρότερον, ὃς ἐγένετο ἐν τῷ καιρῷ τοῦ Ἁβραάμ· ἐπορεύθη δὲ Ἰσαὰκ πρὸς Ἀβιμέλεχ βασιλέα Φυλιστιεὶμ εἰς Γέραρα.
\vs{2}Ὤφθη δὲ αὐτῷ Κύριος, καὶ εἶπε, μὴ καταβῇς εἰς Αἴγυπτον· κατοίκησον δὲ ἐν τῇ γῇ, ᾗ ἄν σοι εἴπω.
\vs{3}Καὶ παροίκει ἐν τῇ γῇ ταύτῃ, καὶ ἔσομαι μετὰ σοῦ, καὶ εὐλογήσω σε· σοὶ γὰρ καὶ τῷ σπέρματί σου δώσω πᾶσαν τὴν γῆν ταύτην· καὶ στήσω τὸν ὅρκον μου, ὅν ὤμοσα τῷ Ἁβραὰμ τῷ πατρί σου.
\vs{4}Καὶ πληθυνῶ τὸ σπέρμα σου, ὡς τοὺς ἀστέρας τοῦ οὐρανοῦ· καὶ δώσω τῷ σπέρματί σου πᾶσαν τὴν γῆν ταύτην· καὶ εὐλογηθήσονται ἐν τῷ σπέρματί σου πάντα τὰ ἔθνη τῆς γῆς.
\vs{5}Ἀνθʼ ὧν ὑπήκουσεν Ἁβραὰμ ὁ πατήρ σου τῆς ἐμῆς φωνῆς, καὶ ἐφύλαξε τὰ προστάγματά μου, καὶ τὰς ἐντολάς μου, καὶ τὰ δικαιώματά μου, καὶ τὰ νόμιμά μου.
\vs{6}Κατῴκησε δὲ Ἰσαὰκ ἐν Γεράροις.
\vs{7}Ἐπηρώτησαν δὲ οἱ ἄνδρες τοῦ τόπου περὶ Ῥεβέκκας τῆς γυναικὸς αὐτοῦ, καὶ εἶπεν, ἀδελφή μου ἐστίν· ἐφοβήθη γὰρ εἰπεῖν, ὅτι γυνή μου ἐστὶ, μή ποτε ἀποκτείνωσιν αὐτὸν οἱ ἄνδρες τοῦ τόπου περὶ Ῥεβέκκας, ὅτι ὡραία τῇ ὄψει ἦν.
\vs{8}Ἐγένετο δὲ πολυχρόνιος ἐκεῖ· καὶ παρακύψας Ἀβιμέλεχ ὁ βασιλεὺς Γεράρων διὰ τῆς θυρίδος, εἶδε τὸν Ἰσαὰκ παίζοντα μετὰ Ῥεβέκκας τῆς γυναικὸς αὐτοῦ.
\vs{9}Ἐκάλεσε δὲ Ἀβιμέλεχ τὸν Ἰσαὰκ, καὶ εἶπεν αὐτῷ, ἆρά γε γυνή σου ἐστί; τί ὅτι εἶπας, ἀδελφή μου ἐστίν; εἶπε δὲ αὐτῷ Ἰσαὰκ, εἶπα γὰρ, μή ποτε ἀποθάνω διʼ αὐτήν.
\vs{10}Εἶπε δὲ αὐτῷ Ἀβιμέλεχ, τί τοῦτο ἐποίησας ἡμῖν; μικροῦ ἐκοιμήθη τις ἐκ τοῦ γένους μου μετὰ τῆς γυναικός σου, καὶ ἐπήγαγες ἂν ἐφʼ ἡμᾶς ἄγνοιαν.
\vs{11}Συνέταξε δὲ Ἀβιμέλεχ παντὶ τῷ λαῷ αὐτοῦ, λέγων, πᾶς ὁ ἁψάμενος τοῦ ἀνθρώπου τούτου καὶ τῆς γυναικὸς αὐτοῦ, θανάτῳ ἔνοχος ἔσται.
\vs{12}Ἔσπειρε δὲ Ἰσαὰκ ἐν τῇ γῇ ἐκείνῃ, καὶ εὗρεν ἐν τῷ ἐνιαυτῷ ἐκείνῳ ἑκατοστεύουσαν κριθήν· εὐλόγησε δὲ αὐτὸν Κύριος.
\vs{13}Καὶ ὑψώθη ὁ ἄνθρωπος, καὶ προβαίνων μείζων ἐγένετο, ἕως οὗ μέγας ἐγένετο σφόδρα.
\vs{14}Ἐγένετο δὲ αὐτῷ κτήνη προβάτων, καὶ κτήνη βοῶν, καὶ γεώργια πολλά. ἐζήλωσαν δὲ αὐτὸν οἱ Φυλιστιείμ.
\vs{15}Καὶ πάντα τὰ φρέατα, ἃ ὤρυξαν οἱ παῖδες τοῦ πατρὸς αὐτοῦ ἐν τῷ χρόνῳ τοῦ πατρὸς αὐτοῦ, ἐνέφραξαν αὐτὰ οἱ Φυλιστιεὶμ, καὶ ἔπλησαν αὐτὰ γῆς.
\vs{16}Εἶπε δὲ Ἀβιμέλεχ πρὸς Ἰσαὰκ, ἄπελθε ἀφʼ ἡμῶν, ὅτι δυνατώτερος ἡμῶν ἐγένου σφόδρα.
\vs{17}Καὶ ἀπῆλθεν ἐκεῖθεν Ἰσαάκ· καὶ κατέλυσεν ἐν τῇ φάραγγι Γεράρων, καὶ κατῴκησεν ἐκεῖ.

\vs{18}Καὶ πάλιν Ἰσαὰκ ὤρυξε τὰ φρέατα τοῦ ὕδατος, ἃ ὤρυξαν οἱ παῖδες Ἁβραὰμ τοῦ πατρὸς αὐτοῦ, καὶ ἐνέφραξαν αὐτὰ οἱ Φυλιστιεὶμ μετὰ τὸ ἀποθανεῖν Ἁβραὰμ τὸν πατέρα αὐτοῦ· καὶ ἐπωνόμασεν αὐτοῖς ὀνόματα κατὰ τὰ ὀνόματα, ἃ ὠνόμασεν ὁ πατὴρ αὐτοῦ.
\vs{19}Καὶ ὤρυξαν οἱ παῖδες Ἰσαὰκ ἐν τῇ φάραγγι Γεράρων· καὶ εὗρον ἐκεῖ φρέαρ ὕδατος ζῶντος.
\vs{20}Καὶ ἐμαχέσαντο οἱ ποιμένες Γεράρων μετὰ τῶν ποιμένων Ἰσαὰκ, φάσκοντες αὐτῶν εἶναι τὸ ὕδωρ· καὶ ἐκάλεσαν τὸ ὄνομα τοῦ φρέατος, Ἀδικία· ἠδίκησαν γὰρ αὐτόν.
\vs{21}Ἀπᾴρας δὲ ἐκεῖθεν ὤρυξε φρέαρ ἕτερον· ἐκρίνοντο δὲ καὶ περὶ ἐκείνου· καὶ ἐπωνόμασε τὸ ὄνομα αὐτοῦ, Ἐχθρία.
\vs{22}Ἀπᾴρας δὲ ἐκεῖθεν ὤρυξε φρέαρ ἕτερον· καὶ οὐκ ἐμαχέσαντο περὶ αὐτοῦ· καὶ ἐπωνόμασε τὸ ὄνομα αὐτοῦ, Εὐρυχωρία, λέγων, διότι νῦν ἐπλάτυνε Κύριος ἡμῖν, καὶ ηὔξησεν ἡμᾶς ἐπὶ τῆς γῆς.

\vs{23}Ἀνέβη δὲ ἐκεῖθεν ἐπὶ τὸ φρέαρ τοῦ ὅρκου.
\vs{24}Καὶ ὤφθη αὐτῷ Κύριος ἐν τῇ νυκτὶ ἐκείνῃ, καὶ εἶπεν, ἐγώ εἰμι ὁ Θεὸς Ἁβραὰμ τοῦ πατρός σου· μὴ φοβοῦ, μετὰ σοῦ γάρ εἰμι, καὶ εὐλογήσω σε, καὶ πληθυνῶ τὸ σπέρμα σου διʼ Ἁβραὰμ τὸν πατέρα σου.
\vs{25}Καὶ ᾠκοδόμησεν ἐκεῖ θυσιαστήριον, καὶ ἐπεκαλέσατο τὸ ὄνομα Κυρίου, καὶ ἔπηξεν ἐκεῖ τὴν σκηνὴν αὐτοῦ· ὤρυξαν δὲ ἐκεῖ οἱ παῖδες Ἰσαὰκ φρέαρ ἐν τῇ φάραγγι Γεράρων.
\vs{26}Καὶ Ἀβιμέλεχ ἐπορεύθη πρὸς αὐτὸν ἀπὸ Γεράρων, καὶ Ὁχοζὰθ ὁ νυμφαγωγὸς αὐτοῦ, καὶ Φιχὼλ ὁ ἀρχιστράτηγος τῆς δυνάμεως αὐτοῦ.
\vs{27}Καὶ εἶπεν αὐτοῖς Ἰσαὰκ, ἵνα τί ἤλθετε πρός με; ὑμεῖς δὲ ἐμισήσατέ με, καὶ ἐξαπεστείλατέ με ἀφʼ ὑμῶν.
\vs{28}Οἱ δὲ εἶπαν, ἰδόντες ἑωράκαμεν ὅτι ἦν Κύριος μετὰ σοῦ· καὶ εἴπαμεν, γενέσθω ἀρὰ ἀνὰ μέσον ἡμῶν καὶ ἀνὰ μέσον σου, καὶ διαθησόμεθα μετὰ σοῦ διαθήκην,
\vs{29}Μὴ ποιήσαι μεθʼ ἡμῶν κακὸν, καθότι οὐκ ἐβδελυξάμεθά σε ἡμεῖς, καὶ ὃν τρόπον ἐχρησάμεθά σοι καλῶς, καὶ ἐξαπεστείλαμέν σε μετʼ εἰρήνης· καὶ νῦν εὐλογημένος σὺ ὑπὸ Κυρίου.
\vs{30}Καὶ ἐποίησεν αὐτοῖς δοχὴν, καὶ ἔφαγον καὶ ἔπιον.
\vs{31}Καὶ ἀναστάντες τὸ πρωῒ, ὤμοσεν ἕκαστος τῷ πλησίον· καὶ ἐξαπέστειλεν αὐτοὺς Ἰσαάκ· καὶ ἀπῴχοντο ἀπʼ αὐτοῦ μετὰ σωτηρίας.
\vs{32}Ἐγένετο δὲ ἐν τῇ ἡμέρᾳ ἐκείνῃ, καὶ παραγενόμενοι οἱ παῖδες Ἰσαὰκ ἀπήγγειλαν αὐτῷ περὶ τοῦ φρέατος οὗ ὤρυξαν, καὶ εἶπαν, οὐχ εὕρομεν ὕδωρ.
\vs{33}Καὶ ἐκάλεσεν αὐτὸ, Ὅρκος· διὰ τοῦτο ἐκάλεσεν ὄνομα τῇ πόλει ἐκείνῃ, Φρέαρ Ὅρκου, ἕως τῆς σήμερον ἡμέρας.

\vs{34}Ἦν δὲ Ἡσαῦ ἐτῶν τεσσαράκοντα, καὶ ἔλαβε γυναῖκα Ἰουδὶθ, θυγατέρα Βεὼχ τοῦ Χετταίου, καὶ τὴν Βασεμὰθ, θυγατέρα Ἑλὼν Χετταίου.
\vs{35}Καὶ ἦσαν ἐρίζουσαι τῷ Ἰσαὰκ καὶ τῇ Ῥεβέκκᾳ.

\ch{27}
Ἐγένετο δὲ μετὰ τὸ γηράσαι τὸν Ἰσαὰκ, καὶ ἠμβλύνθησαν οἱ ὀφθαλμοὶ αὐτοῦ τοῦ ὁρᾷν, καὶ ἐκάλεσεν Ἡσαῦ τὸν υἱὸν αὐτοῦ τὸν πρεσβύτερον, καὶ εἶπεν αὐτῷ, υἱέ μου· καὶ εἶπεν, ἰδοὺ ἐγώ.
\vs{2}Καὶ εἶπεν, ἰδοὺ γεγήρακα, καὶ οὐ γινώσκω τὴν ἡμέραν τῆς τελευτῆς μου.
\vs{3}Νῦν οὖν λάβε τὸ σκεῦός σου, τήν τε φαρέτραν, καὶ τὸ τόξον, καὶ ἔξελθε εἰς τὸ πεδίον, καὶ θήρευσόν μοι θήραν.
\vs{4}Καὶ ποίησόν μοι ἐδέσματα, ὡς φιλῶ ἐγὼ, καὶ ἔνεγκέ μοι, ἵνα φάγω, ὅπως εὐλογήσῃ σε ἡ ψυχή μου πρὶν ἀποθανεῖν με.
\vs{5}Ῥεβέκκα δὲ ἤκουσε λαλοῦντος Ἰσαὰκ πρὸς Ἡσαῦ τὸν υἱὸν αὐτοῦ· ἐπορεύθη δὲ Ἡσαῦ εἰς τὸ πεδίον θηρεῦσαι θήραν τῷ πατρὶ αὐτοῦ.
\vs{6}Ῥεβέκκα δὲ εἶπε πρὸς τὸν Ἰακὼβ τὸν υἱὸν αὐτῆς τὸν ἐλάσσω, ἴδε, ἤκουσα τοῦ πατρός σου λαλοῦντος πρὸς Ἡσαῦ τὸν ἀδελφόν σου, λέγοντος,
\vs{7}Ἔνεγκόν μοι θήραν, καὶ ποίησόν μοι ἐδέσματα, ἵνα φαγὼν εὐλογήσω σε ἐναντίον Κυρίου πρὸ τοῦ ἀποθανεῖν με.
\vs{8}Νῦν οὖν, υἱέ μου, ἄκουσόν μου, καθὰ ἐγώ σοι ἐντέλλομαι.
\vs{9}Καὶ πορευθεὶς εἰς τὰ πρόβατα, λάβε μοι ἐκεῖθεν δύο ἐρίφους ἁπαλοὺς καὶ καλοὺς, καὶ ποιήσω αὐτοὺς ἐδέσματα τῷ πατρί σου, ὡς φιλεῖ.
\vs{10}Καὶ εἰσοίσεις τῷ πατρί σου, καὶ φάγεται, ὅπως εὐλογήσῃ σε ὁ πατήρ σου πρὸ τοῦ ἀποθανεῖν αὐτόν.
\vs{11}Εἶπε δὲ Ἰακὼβ πρὸς Ῥεβέκκαν τὴν μητέρα αὐτοῦ, ἔστιν Ἡσαῦ ὁ ἀδελφός μου ἀνὴρ δασὺς, ἐγὼ δὲ ἀνὴρ λεῖος.
\vs{12}Μή ποτε ψηλαφήσῃ με ὁ πατὴρ, καὶ ἔσομαι ἐναντίον αὐτοῦ ὡς καταφρονῶν, καὶ ἐπάξω ἐπʼ ἐμαυτὸν κατάραν, καὶ οὐκ εὐλογίαν.
\vs{13}Εἶπε δὲ αὐτῷ ἡ μήτηρ, ἐπʼ ἐμὲ ἡ κατάρα σου, τέκνον· μόνον ἐπάκουσόν μου τῆς φωνῆς, καὶ πορευθεὶς ἔνεγκέ μοι.
\vs{14}Πορευθεὶς δὲ ἔλαβε, καὶ ἤνεγκε τῇ μητρί· καὶ ἐποίησεν ἡ μήτηρ αὐτοῦ ἐδέσματα, καθὰ ἐφίλει ὁ πατὴρ αὐτοῦ.

\vs{15}Καὶ λαβοῦσα Ῥεβέκκα τὴν στολὴν Ἡσαῦ τοῦ υἱοῦ αὐτῆς τοῦ πρεσβυτέρου τὴν καλὴν, ἣ ἦν παρʼ αὐτῇ ἐν τῷ οἴκῳ, ἐνέδυσεν αὐτὴν Ἰακὼβ τὸν υἱὸν αὐτῆς τὸν νεώτερον.
\vs{16}Καὶ τὰ δέρματα τῶν ἐρίφων περιέθηκεν ἐπὶ τοὺς βραχίονας αὐτοῦ, καὶ ἐπὶ τὰ γυμνὰ τοῦ τραχήλου αὐτοῦ.
\vs{17}Καὶ ἔδωκε τὰ ἐδέσματα, καὶ τοὺς ἄρτους οὓς ἐποίησεν, εἰς τὰς χεῖρας Ἰακὼβ τοῦ υἱοῦ αὐτῆς.
\vs{18}Καὶ εἰσήνεγκε τῷ πατρὶ αὐτοῦ· εἶπε δὲ, πάτερ· ὁ δὲ εἶπεν, ἰδοὺ ἐγώ· τίς εἶ σὺ, τέκνον;
\vs{19}Καὶ εἶπεν Ἰακὼβ τῷ πατρὶ, ἐγὼ Ἡσαῦ ὁ πρωτότοκός σου πεποίηκα καθὰ ἐλάλησάς μοι· ἀναστὰς κάθισον, καὶ φάγε ἀπὸ τῆς θήρας μου, ὅπως εὐλογήσῃ με ἡ ψυχή σου.
\vs{20}Εἶπε δὲ Ἰσαὰκ τῷ υἱῷ αὐτοῦ, τί τοῦτο, ὃ ταχὺ εὗρες, ὦ τέκνον; ὁ δὲ εἶπεν, ὃ παρέδωκε Κύριος ὁ Θεός σου ἐναντίον μου.
\vs{21}Εἶπε δὲ Ἰσαὰκ τῷ Ἰακὼβ, ἔγγισόν μοι, καὶ ψηλαφήσω σε, τέκνον, εἰ σὺ εἶ ὁ υἱός μου Ἡσαῦ, ἢ οὔ.
\vs{22}Ἤγγισε δὲ Ἰακὼβ πρὸς Ἰσαὰκ τὸν πατέρα αὐτοῦ· καὶ ἐψηλάφησεν αὐτὸν, καὶ εἶπεν, ἡ μὲν φωνὴ, φωνὴ Ἰακὼβ, αἱ δὲ χεῖρες, χεῖρες Ἡσαῦ.
\vs{23}Καὶ οὐκ ἐπέγνω αὐτὸν, ἦσαν γὰρ αἱ χεῖρες αὐτοῦ, ὡς αἱ χεῖρες Ἡσαῦ τοῦ ἀδελφοῦ αὐτοῦ, δασεῖαι· καὶ εὐλόγησεν αὐτὸν,
\vs{24}καὶ εἶπε, σὺ εἶ ὁ υἱός μου Ἡσαῦ; ὁ δὲ εἶπεν, ἐγώ.
\vs{25}Καὶ εἶπε, προσάγαγέ μοι, καὶ φάγομαι ἀπὸ τῆς θήρας σου, τέκνον, ἵνα εὐλογήσῃ σε ἡ ψυχή μου· καὶ προσήνεγκεν αὐτῷ, καὶ ἔφαγε· καὶ εἰσήνεγκεν αὐτῷ οἶνον, καὶ ἔπιε.
\vs{26}Καὶ εἴπεν αὐτῷ Ἰσαὰκ ὁ πατὴρ αὐτοῦ, ἔγγισόν μοι, καὶ φίλησόν με, τέκνον.
\vs{27}Καὶ ἐγγίσας ἐφίλησεν αὐτόν· καὶ ὠσφράνθη τὴν ὀσμὴν τῶν ἱματίων αὐτοῦ, καὶ εὐλόγησεν αὐτὸν, καὶ εἶπεν, ἰδοὺ ὀσμὴ τοῦ υἱοῦ μου, ὡς ὀσμὴ ἀγροῦ πλήρους, ὃν εὐλόγησε Κύριος.
\vs{28}Καὶ δῴη σοι ὁ Θεὸς ἀπὸ τῆς δρόσου τοῦ οὐρανοῦ, καὶ ἀπὸ τῆς πιότητος τῆς γῆς, καὶ πλῆθος σίτου καὶ οἴνου.
\vs{29}Καὶ δουλευσάτωσάν σοι ἔθνη, καὶ προσκυνησάτωσάν σοι ἄρχοντες· καὶ γίνου κύριος τοῦ ἀδελφοῦ σου, καὶ προσκυνήσουσί σοι οἱ υἱοὶ τοῦ πατρός σου· ὁ καταρώμενός σε, ἐπικατάρατος· ὁ δὲ εὐλογῶν σε, εὐλογημένος.

\vs{30}Καὶ ἐγένετο μετὰ τὸ παύσασθαι Ἰσαὰκ εὐλογοῦντα Ἰακὼβ τὸν υἱὸν αὐτοῦ, καὶ ἐγένετο, ὡς ἂν ἐξῆλθεν Ἰακὼβ ἀπὸ προσώπου Ἰσαὰκ τοῦ πατρὸς αὐτοῦ, καὶ Ἡσαῦ ὁ ἀδελφὸς αὐτοῦ ἦλθεν ἀπὸ τῆς θήρας.
\vs{31}Καὶ ἐποίησε καὶ αὐτὸς ἐδέσματα, καὶ προσήνεγκε τῷ πατρὶ αὐτοῦ· καὶ εἶπε τῷ πατρὶ, ἀναστήτω ὁ πατήρ μου, καὶ φαγέτω ἀπὸ τῆς θήρας τοῦ υἱοῦ αὐτοῦ, ὅπως εὐλογήσῃ με ἡ ψυχή σου.
\vs{32}Καὶ εἶπεν αὐτῷ Ἰσαὰκ ὁ πατὴρ αὐτοῦ, τίς εἶ σύ; ὁ δὲ εἶπεν, ἐγώ εἰμι ὁ υἱός σου ὁ πρωτότοκος Ἡσαῦ.
\vs{33}Ἐξέστη δὲ Ἰσαὰκ ἔκστασιν μεγάλην σφόδρα, καὶ εἶπε, τίς οὖν ὁ θηρεύσας μοι θήραν καὶ εἰσενέγκας μοι, καὶ ἔφαγον ἀπὸ πάντων πρὸ τοῦ ἐλθεῖν σε; καὶ εὐλόγησα αὐτὸν, καὶ εὐλογημένος ἔσται.
\vs{34}Ἐγένετο δὲ ἡνίκα ἤκουσεν Ἡσαῦ τὰ ῥήματα τοῦ πατρὸς αὐτοῦ Ἰσαὰκ, ἀνεβόησε φωνὴν μεγάλην καὶ πικρὰν σφόδρα· καὶ εἶπεν, εὐλόγησον δὴ κᾀμὲ, πάτερ.
\vs{35}Εἶπε δὲ αὐτῷ, ἐλθὼν ὁ ἀδελφός σου μετὰ δόλου ἔλαβε τὴν εὐλογίαν σου.
\vs{36}Καὶ εἶπε, δικαίως ἐκλήθη τὸ ὄνομα αὐτοῦ Ἰακὼβ, ἐπτέρνικε γάρ με ἰδοὺ δεύτερον τοῦτο· τά τε πρωτοτόκιά μου εἴληφε, καὶ νῦν ἔλαβε τὴν εὐλογίαν μου· καὶ εἶπεν Ἡσαῦ τῷ πατρὶ αὐτοῦ, οὐχ ὑπελίπου μοι εὐλογίαν, πάτερ;
\vs{37}Ἀποκριθεὶς δὲ Ἰσαὰκ εἶπε τῷ Ἡσαῦ, εἰ κύριον αὐτὸν πεποίηκά σου, καὶ πάντας τοὺς ἀδελφοὺς αὐτοῦ πεποίηκα αὐτοῦ οἰκέτας· σίτῳ καὶ οἴνῳ ἐστήριξα αὐτόν· σοὶ δὲ τί ποιήσω, τέκνον;
\vs{38}Εἶπε δὲ Ἡσαῦ πρὸς τὸν πατέρα αὐτοῦ, μὴ εὐλογία μία σοι ἔστι, πάτερ; εὐλόγησον δὴ κᾀμὲ, πάτερ· κατανυχθέντος δὲ Ἰσαὰκ, ἀνεβόησε φωνῇ Ἡσαῦ, καὶ ἔκλαυσεν.
\vs{39}Ἀποκοιθεὶς δὲ Ἰσαὰκ ὁ πατὴρ αὐτοῦ εἶπεν αὐτῷ, ἰδοὺ ἀπὸ τῆς πιότητος τῆς γῆς ἔσται ἡ κατοίκησίς σου, καὶ ἀπὸ τῆς δρόσου τοῦ οὐρανοῦ ἄνωθεν.
\vs{40}Καὶ ἐπὶ τῇ μαχαίρᾳ σου ζήσῃ, καὶ τῷ ἀδελφῷ σου δουλεύσεις· ἔσται δὲ ἡνίκα ἐὰν καθέλῃς καὶ ἐκλύσῃς τὸν ζυγὸν αὐτοῦ ἀπὸ τοῦ τραχήλου σου.

\vs{41}Καὶ ἐνεκότει Ἡσαῦ τῷ Ἰακὼβ περὶ τῆς εὐλογίας, ἧς εὐλόγησεν αὐτὸν ὁ πατὴρ αὐτοῦ· εἶπε δὲ Ἡσαῦ ἐν τῇ διανοίᾳ αὐτοῦ, ἐγγισάτωσαν αἱ ἡμέραι τοῦ πένθους τοῦ πατρός μου, ἵνα ἀποκτείνω Ἰακὼβ τὸν ἀδελφόν μου.
\vs{42}Ἀπηγγέλη δὲ Ῥεβέκκᾳ τὰ ῥήματα Ἡσαῦ τοῦ υἱοῦ αὐτῆς τοῦ πρεσβυτέρου· καὶ πέμψασα ἐκάλεσεν Ἰακὼβ τὸν υἱὸν αὐτῆς τὸν νεώτερον, καὶ εἶπεν αὐτῷ, ἰδοὺ Ἡσαῦ ὁ ἀδελφός σου ἀπειλεῖ σοι τοῦ ἀποκτεῖναί σε.
\vs{43}Νῦν οὖν, τέκνον, ἄκουσόν μου τῆς φωνῆς, καὶ ἀναστὰς ἀπόδραθι εἰς τὴν Μεσοποταμίαν πρὸς Λάβαν τὸν ἀδελφόν μου εἰς Χαῤῥάν.
\vs{44}Καὶ οἴκησον μετʼ αὐτοῦ ἡμέρας τινὰς, ἕως τοῦ ἀποστρέψαι τὸν θυμὸν,
\vs{45}καὶ τὴν ὀργὴν τοῦ ἀδελφοῦ σου ἀπὸ σοῦ, καὶ ἐπιλάθηται ἃ πεποίηκας αὐτῷ· καὶ ἀποστείλασα μεταπέμψομαί σε ἐκεῖθεν, μή ποτε ἀποτεκνωθῶ ἀπὸ τῶν δύο ὑμῶν ἐν ἡμέρᾳ μιᾷ.
\vs{46}Εἶπε δὲ Ῥεβέκκα πρὸς Ἰσαὰκ, προσώχθικα τῇ ζωῇ μου διὰ τὰς θυγατέρας τῶν υἱῶν Χέτ· εἰ λήψεται Ἰακὼβ γυναῖκα ἀπὸ τῶν θυγατέρων τῆς γῆς ταύτης, ἵνα τί μοι τὸ ζῇν;

\ch{28}
Προσκαλεσάμενος δὲ Ἰσαὰκ τὸν Ἰακὼβ, εὐλόγησεν αὐτὸν, καὶ ἐνετείλατο αὐτῷ, λέγων, οὐ λήψῃ γυναῖκα ἐκ τῶν θυγατέρων τῶν Χαναναίων.
\vs{2}Ἀναστὰς ἀπόδραθι εἰς τὴν Μεσοποταμίαν, εἰς τὸν οἶκον Βαθουὴλ τοῦ πατρὸς τῆς μητρός σου, καὶ λάβε σεαυτῷ ἐκεῖθεν γυναῖκα ἐκ τῶν θυγατέρων Λάβαν τοῦ ἀδελφοῦ τῆς μητρός σου.
\vs{3}Ὁ δὲ Θεός μου εὐλογήσαι σε, καὶ αὐξήσαι σε, καὶ πληθύναι σε· καὶ ἔσῃ εἰς συναγωγὰς ἐθνῶν.
\vs{4}Καὶ δῴη σοι τὴν εὐλόγιαν Ἁβραὰμ τοῦ πατρός μου, σοὶ καὶ τῷ σπέρματί σου μετὰ σὲ, κληρονομῆσαι τὴν γῆν τῆς παροικήσεώς σου, ἣν ἔδωκεν ὁ Θεὸς τῷ Ἁβραάμ.
\vs{5}Καὶ ἀπέστειλεν Ἰσαὰκ τὸν Ἰακώβ· καὶ ἐπορεύθη εἰς τὴν Μεσοποταμίαν πρὸς Λάβαν τὸν υἱὸν Βαθουὴλ τοῦ Σύρου, ἀδελφὸν Ῥεβέκκας τῆς μητρὸς Ἰακὼβ καὶ Ἡσαῦ.

\vs{6}Ἴδε δὲ Ἡσαῦ ὅτι εὐλόγησεν Ἰσαὰκ τὸν Ἰακὼβ, καὶ ἀπέστειλεν εἰς τὴν Μεσοποταμίαν Συρίας, λαβεῖν ἑαυτῷ γυναῖκα ἐκεῖθεν, ἐν τῷ εὐλογεῖν αὐτόν· καὶ ἐνετείλατο αὐτῷ, λέγων, οὐ λήψῃ γυναῖκα ἐκ τῶν θυγατέρων τῶν Χαναναίων.
\vs{7}Καὶ ἤκουσεν Ἰακὼβ τοῦ πατρὸς καὶ τῆς μητρὸς αὐτοῦ· καὶ ἐπορεύθη εἰς τὴν Μεσοποταμίαν Συρίας.
\vs{8}Ἰδὼν δὲ καὶ Ἡσαῦ ὅτι πονηραί εἰσιν αἱ θυγατέρες Χαναὰν ἐναντίον Ἰσαὰκ τοῦ πατρὸς αὐτοῦ,
\vs{9}ἐπορεύθη Ἡσαῦ πρὸς Ἰσμαήλ· καὶ ἔλαβε τὴν Μαελὲθ, θυγατέρα Ἰσμαὴλ τοῦ υἱοῦ Ἁβραὰμ, ἀδελφὴν Ναβεὼθ, πρὸς ταῖς γυναιξὶν αὐτοῦ γυναῖκα.

\vs{10}Καὶ ἐξῆλθεν Ἰακὼβ ἀπὸ τοῦ φρέατος τοῦ ὅρκου, καὶ ἐπορεύθη εἰς Χαῤῥάν.
\vs{11}Καὶ ἀπήντησε τόπῳ, καὶ ἐκοιμήθη ἐκεῖ, ἔδυ γὰρ ὁ ἥλιος· καὶ ἔλαβεν ἀπὸ τῶν λίθων τοῦ τόπου, καὶ ἔθηκε πρὸς κεφαλῆς αὐτοῦ· καὶ ἐκοιμήθη ἐν τῷ τόπῳ ἐκείνῳ.
\vs{12}Καὶ ἐνυπνιάσθη· καὶ ἰδοὺ κλίμαξ ἐστηριγμένη ἐν τῇ γῇ, ἧς ἡ κεφαλὴ ἀφικνεῖτο εἰς τὸν οὐρανόν· καὶ οἱ ἄγγελοι τοῦ θεοῦ ἀνέβαινον καὶ κατέβαινον ἐπʼ αὐτῇ.
\vs{13}Ὁ δὲ Κύριος ἐπεστήρικτο ἐπʼ αὐτῆς· καὶ εἶπεν, ἐγώ εἰμι ὁ Θεὸς Ἁβραὰμ τοῦ πατρός σου, καὶ ὁ Θεὸς Ἰσαάκ· μὴ φοβοῦ· ἡ γῆ ἐφʼ ἧς σὺ καθεύδεις ἐπʼ αὐτῆς, σοὶ δώσω αὐτὴν, καὶ τῷ σπέρματί σου.
\vs{14}Καὶ ἔσται τὸ σπέρμα σου ὡς ἡ ἄμμος τῆς γῆς, καὶ πλατυνθήσεται ἐπὶ θάλασσαν, καὶ Λίβα, καὶ Βοῤῥὰν, καὶ ἐπὶ ἀνατολάς· καὶ ἐνευλογηθήσονται ἐν σοὶ πᾶσαι αἱ φυλαὶ τῆς γῆς, καὶ ἐν τῷ σπέρματί σου.
\vs{15}Καὶ ἰδοὺ ἐγώ εἰμι μετὰ σοῦ, διαφυλάσσων σε ἐν τῇ ὁδῷ πάσῃ, οὗ ἂν πορευθῇς· καὶ ἀποστρέψω σε εἰς τὴν γῆν ταύτην· ὅτι οὐ μή σε ἐγκαταλίπω, ἕως τοῦ ποιῆσαί με πάντα ὅσα ἐλάλησά σοι.
\vs{16}Καὶ ἐξηγέρθη Ἰακὼβ ἐκ τοῦ ὕπνου αὐτοῦ, καὶ εἶπεν, ὅτι ἔστι Κύριος ἐν τῷ τόπῳ τούτῳ, ἐγὼ δὲ οὐκ ᾔδειν.
\vs{17}Καὶ ἐφοβήθη, καὶ εἶπεν, ὡς φοβερὸς ὁ τόπος οὗτος· οὐκ ἔστι τοῦτο ἀλλʼ ἢ οἶκος Θεοῦ, καὶ αὕτη ἡ πύλη τοῦ οὐρανοῦ.
\vs{18}Καὶ ἀνέστη Ἰακὼβ τὸ πρωῒ, καὶ ἔλαβε τὸν λίθον, ὃν ὑπέθηκεν ἐκεῖ πρὸς κεφαλῆς αὐτοῦ, καὶ ἔστησεν αὐτὸν στήλην, καὶ ἐπέχεεν ἔλαιον ἐπὶ τὸ ἄκρον αὐτῆς.
\vs{19}Καὶ ἐκάλεσε τὸ ὄνομα τοῦ τόπου ἐκείνου, οἶκος Θεοῦ· καὶ Οὐλαμλοὺζ ἦν ὄνομα τῇ πόλει τὸ πρότερον.
\vs{20}Καὶ ηὔξατο Ἰακὼβ εὐχὴν, λέγων, ἐὰν ᾖ Κύριος ὁ Θεὸς μετʼ ἐμοῦ, καὶ διαφυλάξῃ με ἐν τῇ ὁδῷ ταύτῃ, ᾗ ἐγὼ πορεύομαι, καὶ δῷ μοι ἄρτον φαγεῖν, καὶ ἱμάτιον περιβαλέσθαι,
\vs{21}καὶ ἀποστρέψῃ με μετὰ σωτηρίας εἰς τὸν οἶκον τοῦ πατρός μου, καὶ ἔσται Κύριός μοι εἰς Θεόν.
\vs{22}Καὶ ὁ λίθος οὗτος, ὃν ἔστησα στήλην, ἔσται μοι οἶκος Θεοῦ· καὶ πάντων ὧν ἐάν μοι δῷς, δεκάτην ἀποδεκατώσω αὐτά σοι.

\ch{29}
Καὶ ἐξᾴρας Ἰακὼβ τοὺς πόδας ἐπορεύθη εἰς γῆν ἀνατολῶν, πρὸς Λάβαν τὸν υἱὸν Βαθουὴλ τοῦ Σύρου, ἀδελφὸν δὲ Ῥεβέκκας, μητρὸς Ἰακὼβ καὶ Ἡσαῦ.
\vs{2}Καὶ ὁρᾷ, καὶ ἰδοὺ φρέαρ ἐν τῷ πεδίῳ· ἦσαν δὲ ἐκεῖ τρία ποίμνια προβάτων ἀναπαυόμενα ἐπʼ αὐτοῦ· ἐκ γὰρ τοῦ φρέατος ἐκείνου ἐπότιζον τὰ ποίμνια· λίθος δὲ ἦν μέγας ἐπὶ τῷ στόματι τοῦ φρέατος.
\vs{3}Καὶ συνήγοντο ἐκεῖ πάντα τὰ ποίμνια· καὶ ἀπεκύλιον τὸν λίθον ἀπὸ τοῦ στόματος τοῦ φρέατος, καὶ ἐπότιζον τὰ πρόβατα, καὶ ἀπεκαθίστων τὸν λίθον ἐπὶ τὸ στόμα τοῦ φρέατος εἰς τὸν τόπον αὐτοῦ.
\vs{4}Εἶπε δὲ αὐτοῖς Ἰακὼβ, ἀδελφοὶ, πόθεν ἐστὲ ὑμεῖς; οἱ δὲ εἶπαν, ἐκ Χαῤῥὰν ἐσμέν.
\vs{5}Εἶπε δὲ αὐτοῖς, γινώσκετε Λάβαν τὸν υἱὸν Ναχώρ; οἱ δὲ εἶπαν, γινώσκομεν·
\vs{6}Εἶπε δὲ αὐτοῖς, ὑγιαίνει; οἱ δὲ εἶπαν, ὑγιαίνει· καὶ ἰδοὺ Ῥαχὴλ ἡ θυγάτηρ αὐτοῦ ἤρχετο μετὰ τῶν προβάτων.
\vs{7}Καὶ εἶπεν Ἰακὼβ, ἔτι ἐστὶν ἡμέρα πολλὴ· οὔπω ὥρα συναχθῆναι τὰ κτήνη· ποτίσαντες τὰ πρόβατα, ἀπελθόντες βόσκετε.
\vs{8}Οἱ δὲ εἶπαν, οὐ δυνησόμεθα, ἕως τοῦ συναχθῆναι πάντας τοὺς ποιμένας, καὶ ἀποκυλίσουσι τὸν λίθον ἀπὸ τοῦ στόματος τοῦ φρέατος, καὶ ποτιοῦμεν τὰ πρόβατα.
\vs{9}Ἔτι αὐτοῦ λαλοῦντος αὐτοῖς, καὶ ἰδοὺ Ῥαχὴλ ἡ θυγάτηρ Λάβαν ἤρχετο μετὰ τῶν προβάτων τοῦ πατρὸς αὐτῆς· αὐτὴ γὰρ ἔβοσκε τὰ πρόβατα τοῦ πατρὸς αὐτῆς.
\vs{10}Ἐγένετο δὲ ὡς εἶδεν Ἰακὼβ τὴν Ῥαχὴλ τὴν θυγατέρα Λάβαν, τοῦ ἀδελφοῦ τῆς μητρὸς αὐτοῦ, καὶ τὰ πρόβατα Λάβαν τοῦ ἀδελφοῦ τῆς μητρὸς αὐτοῦ, καὶ προσελθὼν Ἰακὼβ ἀπεκύλισε τὸν λίθον ἀπὸ τοῦ στόματος τοῦ φρέατος, καὶ ἐπότιζε τὰ πρόβατα Λάβαν τοῦ ἀδελφοῦ τῆς μητρὸς αὐτοῦ.
\vs{11}Καὶ ἐφίλησεν Ἰακὼβ τὴν Ῥαχὴλ, καὶ βοήσας τῇ φωνῇ αὐτοῦ ἔκλαυσε.
\vs{12}Καὶ ἀπήγγειλε τῇ Ῥαχὴλ, ὅτι ἀδελφὸς τοῦ πατρὸς αὐτῆς ἐστι, καὶ ὅτι υἱὸς Ῥεβέκκας ἐστί· καὶ δραμοῦσα ἀπήγγειλε τῷ πατρὶ αὐτῆς κατὰ τὰ ῥήματα ταῦτα.
\vs{13}Ἐγένετο δὲ ὡς ἤκουσε Λάβαν τὸ ὄνομα Ἰακὼβ τοῦ υἱοῦ τῆς ἀδελφῆς αὐτοῦ, ἔδραμεν εἰς συνάντησιν αὐτῷ, καὶ περιλαβὼν αὐτὸν ἐφίλησε, καὶ εἰσήγαγεν αὐτὸν εἰς τὸν οἶκον αὐτοῦ· καὶ διηγήσατο τῷ Λάβαν πάντας τοὺς λόγους τούτους.
\vs{14}Καὶ εἶπεν αὐτῷ Λάβαν, ἐκ τῶν ὀστῶν μου καὶ ἐκ τῆς σαρκός μου εἶ σύ· καὶ ἦν μετʼ αὐτοῦ μῆνα ἡμερῶν.

\vs{15}Εἶπε δὲ Λάβαν τῷ Ἰακὼβ, ὅτι γὰρ ἀδελφός μου εἶ, οὐ δουλεύσεις μοι δωρεάν· ἀπάγγειλόν μοι τίς ὁ μισθός σου ἐστί;
\vs{16}Τῷ δὲ Λάβαν ἦσαν δύο θυγατέρες· ὄνομα τῇ μείζονι, Λεία, καὶ ὄνομα τῇ νεωτέρᾳ, Ῥαχήλ.
\vs{17}Οἱ δὲ ὀφθάλμοὶ Λείας, ἀσθενεῖς· Ῥαχῆλ δὲ ἦν καλὴ τῷ εἴδει, καὶ ὡραία τῇ ὄψει σφάδρα.
\vs{18}Ἠγάπησε δὲ Ἰακὼβ τὴν Ῥαχήλ· καὶ εἶπε, δουλεύσω σοι ἑπτὰ ἔτη περὶ τῆς Ῥαχὴλ τῆς θυγατρός σου τῆς νεωτέρας.
\vs{19}Εἶπε δὲ αὐτῷ Λάβαν, βέλτιον δοῦναί με αὐτήν σοι, ἢ δοῦναί με αὐτὴν ἀνδρὶ ἑτέρῳ· οἴκησον μετʼ ἐμοῦ.
\vs{20}Καὶ ἐδούλευσεν Ἰακὼβ περὶ Ῥαχὴλ ἑπτὰ ἔτη· καὶ ἤσαν ἐναντίον αὐτοῦ ὡς ἡμέραι ὀλίγαι, παρὰ τὸ ἀγαπᾷν αὐτὸν αὐτήν.
\vs{21}Εἶπε δὲ Ἰακὼβ τῷ Λάβαν, δός μοι τὴν γυναῖκά μου, πεπλήρωνται γὰρ αἱ ἡμέραι ὅπως εἰσέλθω πρὸς αὐτήν.
\vs{22}Συνήγαγε δὲ Λάβαν πάντας τοὺς ἄνδρας τοῦ τόπου, καὶ ἐποίησε γάμον.
\vs{23}Καὶ ἐγένετο ἑσπέρα, καὶ λαβὼν Λείαν τὴν θυγατέρα αὐτοῦ, εἰσήγαγεν πρὸς Ἰακὼβ, καὶ εἰσῆλθε πρὸς αὐτὴν Ἰακώβ.
\vs{24}Ἔδωκε δὲ Λάβαν Λείᾳ τῇ θυγατρὶ αὐτοῦ Ζελφὰν τὴν παιδίσκην αὐτοῦ, αὐτῇ παιδίσκην.
\vs{25}Ἐγένετο δὲ πρωῒ, καὶ ἰδοὺ ἦν Λεία· εἶπε δὲ Ἰακὼβ τῷ Λάβαν, τί τοῦτο ἐποίησάς μοι; οὐ περὶ Ῥαχὴλ ἐδούλευσα παρὰ σοι; καὶ ἱνατί παρελογίσω με;
\vs{26}Ἀπεκρίθη δὲ Λάβαν, οὐκ ἔστιν οὕτως ἐν τῷ τόπῳ ἡμῶν, δοῦναι τὴν νεωτέραν πρινὴ τὴν πρεσβυτέραν.
\vs{27}Συντέλεσον οὖν τὰ ἕβδομα ταύτης, καὶ δώσω σοι καὶ ταύτην ἀντὶ τῆς ἐργασίας, ἧς ἐργᾷ παρʼ ἐμοὶ ἔτι ἑπτὰ ἔτη ἕτερα.
\vs{28}Ἐποίησε δὲ Ἰακὼβ οὕτως, καὶ ἀνεπλήρωσε τὰ ἕβδομα ταύτης· καὶ ἔδωκεν αὐτῷ Λάβαν Ῥαχὴλ τὴν θυγατέρα αὐτοῦ αὐτῷ γυναῖκα.
\vs{29}Ἔδωκε δὲ Λάβαν τῇ θυγατρὶ αὐτοῦ Βαλλὰν τὴν παιδίσκην αὐτοῦ, αὐτῇ παιδίσκην.
\vs{30}Καὶ εἰσῆλθε πρὸς Ῥαχήλ· ἠγάπησε δὲ Ῥαχὴλ μᾶλλον ἢ Λείαν· καὶ ἐδούλευσεν αὐτῷ ἑπτὰ ἔτη ἕτερα.

\vs{31}Ἰδὼν δὲ Κύριος ὁ Θεὸς ὅτι ἐμισεῖτο Λεία, ἤνοιξε τὴν μήτραν αὐτῆς· Ῥαχὴλ δὲ ἦν στεῖρα.
\vs{32}Καὶ συνέλαβε Λεία, καὶ ἔτεκεν υἱὸν τῷ Ἰακώβ· ἐκάλεσε δὲ τὸ ὄνομα αὐτοῦ Ῥουβὴν, λέγουσα, διότι εἶδέ μου Κύριος τὴν ταπείνωσιν, καὶ ἔδωκέ μοι υἱόν· νῦν οὖν ἀγαπήσει με ὁ ἀνήρ μου.
\vs{33}Καὶ συνέλαβε πάλιν, καὶ ἔτεκεν υἱὸν δεύτερον τῷ Ἰακὼβ, καὶ εἶπεν, ὅτι ἤκουσε Κύριος ὅτι μισοῦμαι, καὶ προσέδωκέ μοι καὶ τοῦτον· καὶ ἐκάλεσε τὸ ὄνομα αὐτοῦ, Συμεών.
\vs{34}Καὶ συνέλαβεν ἔτι, καὶ ἔτεκεν υἱὸν, καὶ εἶπεν, ἐν τῷ νῦν καιρῷ πρὸς ἐμοῦ ἔσται ὁ ἀνήρ μου, τέτοκα γὰρ αὐτῷ τρεῖς υἱούς· διὰ τοῦτο ἐκάλεσε τὸ ὄνομα αὐτοῦ, Λευεί.
\vs{35}Καὶ συλλαβοῦσα ἔτι ἔτεκεν υἱὸν, καὶ εἶπε, νῦν ἔτι τοῦτο ἐξομολογήσομαι Κυρίῳ· διὰ τοῦτο ἐκάλεσε τὸ ὄνομα αὐτοῦ, Ἰούδαν· καὶ ἔστη τοῦ τίκτειν.

\ch{30}
Ἰδοῦσα δὲ Ῥαχὴλ, ὅτι οὐ τέτοκε τῷ Ἱακώβ· καὶ ἐζήλωσε Ῥαχὴλ τὴν ἀδελφὴν αὐτῆς· καὶ εἶπε τῷ Ἰακὼβ, δός μοι τέκνα· εἰ δὲ μὴ, τελευτήσω ἐγώ.
\vs{2}Θυμωθεὶς δὲ Ἰακὼβ τῇ Ῥαχὴλ εἶπεν αὐτῇ, μὴ ἀντὶ Θεοῦ ἐγώ εἰμι, ὃς ἐστέρησέ σε καρπὸν κοιλίας;
\vs{3}Εἶπε δὲ Ῥαχὴλ τῷ Ἰακὼβ, ἰδοὺ ἡ παιδίσκη μου Βαλλά· εἴσελθε πρὸς αὐτήν· καὶ τέξεται ἐπὶ τῶν γονάτων μου, καὶ τεκνοποιήσομαι κᾀγὼ ἐξ αὐτῆς.
\vs{4}Καὶ ἔδωκεν αὐτῷ Βαλλὰν τὴν παιδίσκην αὐτῆς, αὐτῷ γυναῖκα· καὶ εἰσῆλθε πρὸς αὐτὴν Ἰακώβ.
\vs{5}Καὶ συνέλαβε Βαλλὰ ἡ παιδίσκη Ῥαχὴλ, καὶ ἔτεκε τῷ Ἰακὼβ υἱόν.
\vs{6}Καὶ εἶπε Ῥαχὴλ, ἔκρινέ μοι ὁ Θεὸς, καὶ ἐπήκουσε τῆς φωνῆς μου, καὶ ἔδωκε μοι υἱόν· διὰ τοῦτο ἐκάλεσε τὸ ὄνομα αὐτοῦ, Δάν.
\vs{7}Καὶ συνέλαβεν ἔτι Βαλλὰ ἡ παιδίσκη Ῥαχὴλ, καὶ ἔτεκεν υἱὸν δεύτερον τῷ Ἰακώβ.
\vs{8}Καὶ εἶπε Ῥαχὴλ, συναντελάβετό μου ὁ Θεὸς, καὶ συνανεστράφην τῇ ἀδελφῇ μου, καὶ ἠδυνάσθην· καὶ ἐκάλεσε τὸ ὄνομα αὐτοῦ, Νεφθαλεί.
\vs{9}Εἶδε δὲ Λεία ὅτι ἔστη τοῦ τίκτειν· καὶ ἔλαβε Ζελφὰν τὴν παιδίσκην αὐτῆς, καὶ ἔδωκεν αὐτὴν τῷ Ἰακὼβ γυναῖκα· καὶ εἰσῆλθε πρὸς αὐτήν.
\vs{10}Καὶ συνέλαβε Ζελφὰ ἡ παιδίσκη Λείας, καὶ ἔτεκε τῷ Ἰακὼβ υἱόν.
\vs{11}Καὶ εἶπε Λεία, ἐν τύχῃ· καὶ ἐπωνόμασε τὸ ὄνομα αὐτοῦ, Γάδ.
\vs{12}Καὶ συνέλαβεν ἔτι Ζελφὰ ἡ παιδίσκη Λείας, καὶ ἔτεκε τῷ Ἰακὼβ υἱὸν δεύτερον.
\vs{13}Καὶ εἶπε Λεία, μακαρία ἐγὼ, ὅτι μακαριοῦσί με αἱ γυναῖκες· καὶ ἐκάλεσε τὸ ὄνομα αὐτοῦ, Ἀσήρ.
\vs{14}Ἐπορεύθη δὲ Ῥουβὴν ἐν ἡμέρᾳ θερισμοῦ πυρῶν, καὶ εὗρε μῆλα μανδραγορῶν ἐν τῷ ἀγρῷ, καὶ ἤνεγκεν αὐτὰ πρὸς Λείαν τὴν μητέρα αὐτοῦ· εἶπε δὲ Ῥαχὴλ τῇ Λείᾳ τῇ ἀδελφῇ αὐτῆς, δός μοι τῶν μανδραγορῶν τοῦ υἱοῦ σου.
\vs{15}Εἶπε δὲ Λεία, οὐχ ἱκανόν σοι ὅτι ἔλαβες τὸν ἄνδρα μου; μὴ καὶ τοὺς μανδραγόρας τοῦ υἱοῦ μου λήψῃ; εἶπε δὲ Ῥαχὴλ, οὐχ οὕτως· κοιμηθήτω μετὰ σοῦ τὴν νύκτα ταύτην ἀντὶ τῶν μανδραγορῶν τοῦ υἱοῦ σου.
\vs{16}Εἰσῆλθεν δὲ Ἰακὼβ ἐξ ἀγροῦ ἑσπέρας· καὶ ἐξῆλθε Λεία εἰς συνάντησιν αὐτῷ, καὶ εἶπε, πρὸς ἐμὲ εἰσελεύσῃ σήμερον· μεμίσθωμαι γάρ σε ἀντὶ τῶν μανδραγορῶν τοῦ υἱοῦ μου· καὶ ἐκοιμήθη μετʼ αὐτῆς τὴν νύκτα ἐκείνην.
\vs{17}Καὶ ἐπήκουσεν ὁ Θεὸς Λείας· καὶ συλλαβοῦσα ἔτεκε τῷ Ἰακὼβ υἱὸν πέμπτον.
\vs{18}Καὶ εἶπε Λεία, δέδωκέ μοι ὁ Θεὸς τὸν μισθόν μου, ἀνθʼ οὗ ἔδωκα τὴν παιδίσκην μου τῷ ἀνδρί μου· καὶ ἐκάλεσε τὸ ὄνομα αὐτοῦ, Ἰσσάχαρ, ὅ ἐστι μισθός.
\vs{19}Καὶ συνέλαβεν ἔτι Λεία, καὶ ἔτεκεν υἱὸν ἕκτον τῷ Ἰακώβ.
\vs{20}Καὶ εἶπε Λεία, δεδώρηται ὁ Θεός μοι δῶρον καλὸν ἐν τῷ νῦν καιρῷ· αἱρετιεῖ με ὁ ἀνήρ μου, τέτοκα γὰρ αὐτῷ υἱοὺς ἕξ· καὶ ἐκάλεσε τὸ ὄνομα αὐτοῦ, Ζαβουλών.
\vs{21}Καὶ μετὰ τοῦτο ἔτεκε θυγατέρα, καὶ ἐκάλεσε τὸ ὄνομα αὐτῆς, Δεῖνα.
\vs{22}Ἐμνήσθη δὲ ὁ Θεὸς τῆς Ῥαχὴλ, καὶ ἔπήκουσεν αὐτῆς ὁ Θεός· καὶ ἀνέῳξεν αὐτῆς τὴν μήτραν.
\vs{23}Καὶ συλλαβοῦσα ἔτεκε τῷ Ἰακὼβ υἱόν· εἶπε δὲ Ῥαχὴλ, ἀφεῖλεν ὁ Θεός μου τὸ ὄνειδος.
\vs{24}Καὶ ἐκάλεσε τὸ ὄνομα αὐτοῦ Ἰωσὴφ, λέγουσα, προσθέτω ὁ Θεός μοι υἱὸν ἕτερον.

\vs{25}Ἐγένετο δὲ ὡς ἔτεκε Ῥαχὴλ τὸν Ἰωσὴφ, εἶπεν Ἰακὼβ τῷ Λάβαν, ἀπόστειλόν με, ἵνα ἀπέλθω εἰς τὸν τόπον μου, καὶ εἰς τὴν γῆν μου.
\vs{26}Ἀπόδος τὰς γυναῖκας μου, καὶ τὰ παιδία μου, περὶ ὧν δεδούλευκά σοι, ἵνα ἀπέλθω· σὺ γὰρ γινώσκεις τὴν δουλείαν, ἣν δεδούλευκά σοι.
\vs{27}Εἶπε δὲ αὐτῷ Λάβαν, εἰ εὗρον χάριν ἐναντίον σου, οἰωνισάμην ἄν· εὐλόγησε γάρ με ὁ Θεὸς ἐπὶ τῇ σῇ εἰσόδῳ.
\vs{28}Διάστειλον τὸν μισθόν σου πρός με, καὶ δώσω.
\vs{29}Εἶπε δὲ Ἰακὼβ, σὺ γινώσκεις ἃ δεδούλευκά σοι, καὶ ὅσα ἦν κτήνη σου μετʼ ἐμοῦ.
\vs{30}Μικρὰ γὰρ ἦν ὅσα σοι ἐναντίον ἐμοῦ, καὶ ηὐξήθη εἰς πλῆθος· καὶ εὐλόγησέ σε Κύριος ὁ Θεὸς ἐπὶ τῷ ποδί μου· νῦν οὖν πότε ποιήσω κᾀγὼ ἐμαυτῷ οἶκον;
\vs{31}Καὶ εἶπεν αὐτῷ Λάβαν, τί σοι δώσω; Εἶπε δὲ αὐτῷ Ἰακὼβ, οὐ δώσεις μοι οὐθὲν, ἐὰν ποιήσῃς μοι τὸ ῥῆμα τοῦτο, πάλιν ποιμανῶ τὰ πρόβατά σου, καὶ φυλάξω.
\vs{32}Παρελθέτω πάντα τὰ πρόβατά σου σήμερον, καὶ διαχώρισον ἐκεῖθεν πᾶν πρόβατον φαιὸν ἐν τοῖς ἄρνασι, καὶ πᾶν διάλευκον καὶ ῥαντὸν ἐν ταῖς αἰξὶν, ἔσται μοι μισθός.
\vs{33}Καὶ ἐπακούσεταί μοι ἡ δικαιοσύνη μου ἐν τῇ ἡμέρᾳ τῇ ἐπαύριον, ὅτι ἐστὶν ὁ μισθός μου ἐνώπιόν σου· πᾶν ὃ ἐὰν μὴ ᾖ ῥαντὸν καὶ διάλευκον ἐν ταῖς αἰξὶ, καὶ φαιὸν ἐν τοῖς ἄρνασι, κεκλεμμένον ἔσται παρʼ ἐμοί.
\vs{34}Εἶπε δὲ αὐτῷ Λάβαν, ἔστω κατὰ τὸ ῥῆμά σου.
\vs{35}Καὶ διέστειλεν ἐν τῇ ἡμέρᾳ ἐκείνῃ τοὺς τράγους τοὺς ῥαντοὺς καὶ τοὺς διαλεύκους, καὶ πάσας τὰς αἶγας τὰς ῥαντὰς καὶ τὰς διαλεύκους, καὶ πᾶν ὃ ἦν φαιὸν ἐν τοῖς ἄρνασι, καὶ πᾶν ὃ ἦν λευκὸν ἐν αὐτοῖς, καὶ ἔδωκε διὰ χειρὸς τῶν υἱῶν αὐτοῦ.
\vs{36}Καὶ ἀπέστησεν ὁδὸν τριῶν ἡμερῶν, καὶ ἀνὰ μέσον αὐτῶν καὶ ἀνὰ μέσον Ἰακώβ· Ἰακὼβ δὲ ἐποίμαινε τὰ πρόβατα Λάβαν τὰ ὑπολειφθέντα.
\vs{37}Ἔλαβε δὲ ἑαυτῷ Ἰακὼβ ῥάβδον στυρακίνην χλωρὰν καὶ καρυΐνην καὶ πλατάνου· καὶ ἐλέπισεν αὐτὰς Ἰακὼβ λεπίσματα λευκά· καὶ περισύρων τὸ χλωρὸν, ἐφαίνετο ἐπὶ ταῖς ῥάβδοις τὸ λευκὸν, ὃ ἐλέπισε, ποικίλον.
\vs{38}Καὶ παρέθηκε τὰς ῥάβδους, ἃς ἐλέπισεν, ἐν τοῖς ληνοῖς τῶν ποτιστηρίων τοῦ ὕδατος, ἵνα ὡς ἂν ἔλθωσι τὰ πρόβατα πιεῖν, ἐνώπιον τῶν ῥάβδων ἐλθόντων αὐτῶν εἰς τὸ πιεῖν, ἐγκισσήσωσι τὰ πρόβατα εἰς τὰς ῥάβδους.
\vs{39}Καὶ ἐνεκίσσων τὰ πρόβατα εἰς τὰς ῥάβδους· καὶ ἔτικτον τὰ πρόβατα διάλευκα καὶ ποικίλα καὶ σποδοειδῆ ῥαντά.
\vs{40}Τοὺς δὲ ἀμνοὺς διέστειλεν Ἰακὼβ, καὶ ἔστησεν ἐναντίον τῶν προβάτων κριὸν διάλευκον, καὶ πᾶν ποικίλον ἐν τοῖς ἀμνοῖς· καὶ διεχώρισεν ἑαυτῷ ποίμνια καθʼ ἑαυτὸν, καὶ οὐκ ἔμιξεν αὐτὰ εἰς τὰ πρόβατα Λάβαν.
\vs{41}Ἐγένετο δὲ ἐν τῷ καιρῷ ᾧ ἐνεκίσσων τὰ πρόβατα ἐν γαστρὶ λαμβάνοντα, ἔθηκεν Ἰακὼβ τὰς ῥάβδους ἐναντίον τῶν προβάτων ἐν τοῖς ληνοῖς, τοῦ ἐγκισσῆσαι αὐτὰ κατὰ τὰς ῥάβδους.
\vs{42}Ἡνίκα δʼ ἂν ἔτεκε τὰ πρόβατα, οὐκ ἐτίθει· ἐγένετο δὲ τὰ μὲν ἄσημα τοῦ Λάβαν, τὰ δὲ ἐπίσημα τοῦ Ἰακώβ.
\vs{43}Καὶ ἐπλούτησεν ὁ ἄνθρωπος σφόδρα σφόδρα· καὶ ἐγένετο αὐτῷ κτήνη πολλὰ, καὶ βόες, καὶ παῖδες, καὶ παιδίσκαι, καὶ κάμηλοι, καὶ ὄνοι.

\ch{31}
Ἤκουσε δὲ Ἰακὼβ τὰ ῥήματα τῶν υἱῶν Λάβαν, λεγόντων, εἴληφεν Ἰακὼβ πάντα τὰ τοῦ πατρὸς ἡμῶν, καὶ ἐκ τῶν τοῦ πατρὸς ἡμῶν πεποίηκε πᾶσαν τὴν δόξαν ταύτην.
\vs{2}Καὶ εἶδεν Ἰακὼβ τὸ πρόσωπον τοῦ Λάβαν, καὶ ἰδοὺ οὐκ ἦν πρὸς αὐτὸν ὡσεὶ χθὲς καὶ τρίτην ἡμέραν.
\vs{3}Εἶπε δὲ Κύριος πρὸς Ἰακὼβ, ἀποστρέφου εἰς τὴν γῆν τοῦ πατρός σου, καὶ εἰς τὴν γενεάν σου, καὶ ἔσομαι μετὰ σοῦ.
\vs{4}Ἀποστείλας δὲ Ἰακὼβ ἐκάλεσε Λείαν καὶ Ῥαχὴλ εἰς τὸ πεδίον, οὗ ἦν τὰ ποίμνια.
\vs{5}Καὶ εἶπεν αὐταῖς, ὁρῶ ἐγὼ τὸ πρόσωπον τοῦ πατρὸς ὑμῶν, ὅτι οὐκ ἔστι πρὸς ἐμοῦ, ὡς ἐχθὲς καὶ τρίτην ἡμέραν· ὁ δὲ Θεὸς τοῦ πατρός μου ἦν μετʼ ἐμοῦ.
\vs{6}Καὶ αὐταὶ δὲ οἴδατε, ὅτι ἐν πάσῃ τῇ ἰσχύϊ μου δεδούλευκα τῷ πατρὶ ὑμῶν.
\vs{7}Ὁ δὲ πατὴρ ὑμῶν παρεκρούσατό με, καὶ ἤλλαξε τὸν μισθόν μου τῶν δέκα ἀμνῶν· καὶ οὐκ ἔδωκεν αὐτῷ ὁ Θεὸς κακοποιῆσαί με.
\vs{8}Ἐὰν οὕτως εἴπῃ, τὰ ποικίλα ἔσται σου μισθὸς, καὶ τέξεται πάντα τὰ πρόβατα ποικίλα· ἐὰν δὲ εἴπῃ, τὰ λευκὰ ἔσται σου μισθὸς, καὶ τέξεται πάντα τὰ πρόβατα λευκά.
\vs{9}Καὶ ἀφείλετο ὁ Θεὸς πάντα τὰ κτήνη τοῦ πατρὸς ὑμῶν, καὶ ἔδωκέ μοι αὐτά.
\vs{10}Καὶ ἐγένετο ἡνίκα ἐνεκίσσων τὰ πρόβατα ἐν γαστρὶ λαμβάνοντα, καὶ εἶδον τοῖς ὀφθαλμοῖς μου ἐν τῷ ὕπνῳ· καὶ ἰδοὺ οἱ τράγοι καὶ οἱ κριοὶ ἀναβαίνοντες ἐπὶ τὰ πρόβατα καὶ τὰς αἶγας, διάλευκοι καὶ ποικίλοι καὶ σποδοειδεῖς ῥαντοί.
\vs{11}Καὶ εἶπέ μοι ὁ Ἄγγελος τοῦ Θεοῦ καθʼ ὕπνον, Ἰακώβ· ἐγὼ δὲ εἶπα, τί ἐστι;
\vs{12}Καὶ εἶπεν, ἀνάβλεψον τοῖς ὀφθαλμοῖς σου, καὶ ἴδε τοὺς τράγους καὶ τοὺς κριοὺς ἀναβαίνοντας ἐπὶ τὰ πρόβατα καὶ τὰς αἶγας διαλεύκους καὶ ποικίλους καὶ σποδοειδεῖς ῥαντούς· ἑώρακα γὰρ ὅσα σοι Λάβαν ποιεῖ.
\vs{13}Ἐγώ εἰμι ὁ Θεὸς ὁ ὀφθείς σοι ἐν τόπῳ Θεοῦ, οὗ ἤλειψάς μοι ἐκεῖ στήλην, καὶ ηὔξω μοι ἐκεῖ εὐχήν· νῦν οὖν ἀνάστηθι, καὶ ἔξελθε ἐκ τῆς γῆς ταύτης, καὶ ἄπελθε εἰς τὴν γῆν τῆς γενέσεώς σου, καὶ ἔσομαι μετὰ σοῦ.
\vs{14}Καὶ ἀποκριθεῖσαι Ῥαχὴλ καὶ Λεία εἶπαν αὐτῷ, μὴ ἔστιν ἡμῖν ἔτι μερὶς ἢ κληρονομία ἐν τῷ οἴκῳ τοῦ πατρὸς ἡμῶν;
\vs{15}Οὐχ ὡς αἱ ἀλλότριαι λελογίσμεθα αὐτῷ; πέπρακε γὰρ ἡμᾶς, καὶ καταβρώσει κατέφαγε τὸ ἀργύριον ἡμῶν.
\vs{16}Πάντα τὸν πλοῦτον καὶ τὴν δόξαν, ἣν ἀφείλετο ὁ Θεὸς τοῦ πατρὸς ἡμῶν, ἡμῖν ἔσται καὶ τοῖς τέκνοις ἡμῶν· νῦν οὖν ὅσα σοι εἴρηκεν ὁ Θεὸς, ποίει.
\vs{17}Ἀναστὰς δὲ Ἰακὼβ ἔλαβε τὰς γυναῖκας αὐτοῦ καὶ τὰ παιδία αὐτοῦ ἐπὶ τὰς καμήλους·
\vs{18}Καὶ ἀπήγαγε πάντα τὰ ὑπάρχοντα αὐτῷ, καὶ πᾶσαν τὴν ἀποσκευὴν αὐτοῦ, ἣν περιεποιήσατο ἐν τῇ Μεσοποταμίᾳ, καὶ πάντα τὰ αὐτοῦ, ἀπελθεῖν πρὸς Ἰσαὰκ τὸν πατέρα αὐτοῦ εἰς γῆν Χαναάν.
\vs{19}Λάβαν δὲ ᾤχετο κεῖραι τὰ πρόβατα αὐτοῦ· ἔκλεψε δὲ Ῥαχὴλ τὰ εἴδωλα τοῦ πατρὸς αὐτῆς.
\vs{20}Ἔκρυψε δὲ Ἰακὼβ Λάβαν τὸν Σύρον, τοῦ μὴ ἀναγγεῖλαι αὐτῷ, ὅτι ἀποδιδράσκει.
\vs{21}Καὶ ἀπέδρα αὐτὸς, καὶ τὰ αὐτοῦ πάντα, καὶ διέβη τὸν ποταμὸν, καὶ ὥρμησεν εἰς τὸ ὄρος Γαλαάδ.
\vs{22}Ἀνηγγέλη δὲ Λάβαν τῷ Σύρῳ τῇ ἡμέρᾳ τῇ τρίτῃ, ὅτι ἀπέδρα Ἰακώβ.
\vs{23}Καὶ παραλαβὼν τοὺς ἀδελφοὺς αὐτοῦ μεθʼ ἑαντοῦ, ἐδίωξεν ὀπίσω αὐτοῦ ὁδὸν ἡμερῶν ἑπτά· καὶ κατέλαβεν αὐτὸν ἐν τῷ ὄρει Γαλαάδ.
\vs{24}Ἦλθε δὲ ὁ Θεὸς πρὸς Λάβαν τὸν Σύρον καθʼ ὕπνον τὴν νύκτα, καὶ εἶπεν αὐτῷ, Φύλαξαι σεαυτὸν μή ποτε λαλήσῃς μετὰ Ἰακὼβ πονηρά.
\vs{25}Καὶ κατέλαβε Λάβαν τὸν Ἰακώβ· Ἰακὼβ δὲ ἔπηξεν τὴν σκηνὴν αὐτοῦ ἐν τῷ ὄρει· Λάβαν δὲ ἔστησε τοὺς ἀδελφοὺς αὐτοῦ ἐν τῷ ὄρει Γαλαάδ.
\vs{26}Εἶπε δὲ Λάβαν τῷ Ἰακὼβ, τί ἐποίησας; ἱνατί κρυφῇ ἀπέδρας, καὶ ἐκλοποφόρησάς με, καὶ ἀπήγαγες τὰς θυγατέρας μου, ὡς αἰχμαλώτιδας μαχαίρᾳ;
\vs{27}Καὶ εἰ ἀνήγγειλάς μοι, ἐξαπέστειλα ἄν σε μετʼ εὐφροσύνης, καὶ μετὰ μουσικῶν, καὶ τυμπάνων, καὶ κιθάρας.
\vs{28}Καὶ οὐκ ἠξιώθην καταφιλῆσαι τὰ παιδία μου, καὶ τὰς θυγατέρας μου· νῦν δὲ ἀφρόνως ἔπραξας.
\vs{29}Καὶ νῦν ἰσχύει ἡ χείρ μου κακοποιῆσαί σε· ὁ δὲ Θεὸς τοῦ πατρός σου χθὲς εἶπε πρός με, λέγων, Φύλαξαι σεαυτὸν μή ποτε λαλήσῃς μετὰ Ἰακὼβ πονηρά.
\vs{30}Νῦν οὖν πεπόρευσαι· ἐπιθυμίᾳ γὰρ ἐπεθύμησας ἀπελθεῖν εἰς τὸν οἶκον τοῦ πατρός σου· ἱνατί ἔκλεψας τοὺς θεούς μου;
\vs{31}Ἀποκριθεὶς δὲ Ἰακὼβ εἶπε τῷ Λάβαν, ὅτι ἐφοβήθην· εἶπα γὰρ, μή ποτε ἀφέλῃ τὰς θυγατέρας σου ἀπʼ ἐμοῦ, καὶ πάντα τὰ ἐμά.
\vs{32}Καὶ εἶπεν Ἰακὼβ, παρʼ ᾧ ἂν εὕρῃς τοὺς θεούς σου, οὐ ζήσεται ἐναντίον τῶν ἀδελφῶν ἡμῶν· ἐπίγνωθι τί ἐστι παρʼ ἐμοὶ τῶν σῶν, καὶ λάβε· καὶ οὐκ ἐπέγνω παρʼ αὐτῷ οὐθέν· οὐκ ᾔδει δὲ Ἰακὼβ, ὅτι Ῥαχὴλ ἡ γυνὴ αὐτοῦ ἔκλεψεν αὐτούς.
\vs{33}Εἰσελθὼν δὲ Λάβαν ἠρεύνησεν εἰς τὸν οἶκον Λείας, καὶ οὐχ εὗρεν· καὶ ἐξῆλθεν ἐκ τοῦ οἴκου Λείας, καὶ ἠρεύνησε τὸν οἶκον Ἰακὼβ, καὶ ἐν τῷ οἴκῳ τῶν δύο παιδισκῶν, καὶ οὐχ εὗρεν· εἰσῆλθε δὲ καὶ εἰς τὸν οἶκον Ῥαχήλ.
\vs{34}Ῥαχὴλ δὲ ἔλαβε τὰ εἴδωλα, καὶ ἐνέβαλεν αὐτὰ εἰς τὰ σάγματα τῆς καμήλου, καὶ ἐπεκάθισεν αὐτοῖς.
\vs{35}Καὶ εἶπε τῷ πατρὶ αὐτῆς, μὴ βαρέως φέρε, κύριε· οὐ δυνάμαι ἀναστῆναι ἐνώπιόν σου, ὅτι τὰ κατʼ ἐθισμὸν τῶν γυναικῶν μοι ἐστίν· ἠρεύνησε Λάβαν ἐν ὅλῳ τῷ οἴκῳ, καὶ οὐχ εὗρε τὰ εἴδωλα.
\vs{36}Ὠργίσθη δὲ Ἰακὼβ, καὶ ἐμαχέσατο τῷ Λάβαν· ἀποκριθεὶς δὲ Ἰακὼβ εἶπε τῷ Λάβαν, τί τὸ ἀδίκημά μου; καὶ τί τὸ ἁμάρτημά μου, ὅτι κατεδίωξας ὀπίσω μου,
\vs{37}καὶ ὅτι ἠρεύνησας πάντα τὰ σκεύη τοῦ οἴκου μου; τί εὗρες ἀπὸ πάντων τῶν σκευῶν τοῦ οἴκου σου; θὲς ὧδε ἐνώπιον τῶν ἀδελφῶν σου καὶ τῶν ἀδελφῶν μου, καὶ ἐλεγξάτωσαν ἀνὰ μέσον τῶν δύο ἡμῶν.
\vs{38}Ταῦτά μοι εἴκοσι ἔτη ἐγώ εἰμι μετὰ σοῦ· τὰ πρόβατά σου καὶ αἱ αἶγές σου οὐκ ἠτεκνώθησαν· κριοὺς τῶν προβάτων σου οὐ κατέφαγον.
\vs{39}Θηριάλωτον οὐκ ἐνήνοχά σοι· ἐγὼ ἀπετίννυον παρʼ ἐμαυτοῦ κλέμματα ἡμέρας, καὶ κλέμματα νυκτός.
\vs{40}Ἐγενόμην τῆς ἡμέρας συγκαιόμενος τῷ καύματι, καὶ τῷ παγετῷ τῆς νυκτός· καὶ ἀφίστατο ὁ ὕπνος μου ἀπὸ τῶν ὀφθαλμῶν μου.
\vs{41}Ταῦτά μοι εἴκοσι ἔτη ἐγώ εἰμι ἐν τῇ οἰκίᾳ σου· ἐδούλευσά σοι δεκατέσσαρα ἔτη ἀντὶ τῶν δύο θυγατέρων σου, καὶ ἓξ ἔτη ἐν τοῖς προβάτοις σου, καὶ παρελογίσω τὸν μισθόν μου δέκα ἀμνάσιν.
\vs{42}Εἰ μὴ ὁ Θεὸς τοῦ πατρός μου Ἁβραὰμ, καὶ ὁ φόβος Ἰσαὰκ, ἦν μοι, νῦν ἂν κενόν με ἐξαπέστειλας· τὴν ταπείνωσίν μου, καὶ τὸν κόπον τῶν χειρῶν μου, εἶδεν ὁ Θεός· καὶ ἤλεγξέ σε χθές.

\vs{43}Ἀποκριθεὶς δὲ Λάβαν εἶπε τῷ Ἰακὼβ, αἱ θυγατέρες, θυγατέρες μου, καὶ υἱοὶ, υἱοί μου, καὶ τὰ κτήνη, κτήνη μου· καὶ πάντα ὅσα σὺ ὁρᾷς, ἐμά ἐστι, καὶ τῶν θυγατέρων μου· τί ποιήσω ταύταις σήμερον ἢ τοῖς τέκνοις αὐτῶν, οἷς ἔτεκον;
\vs{44}Νῦν οὖν δεῦρο διαθῶμαι διαθήκην ἐγώ τε καὶ σύ· καὶ ἔσται εἰς μαρτύριον ἀνὰ μέσον ἐμοῦ καὶ σοῦ· εἶπε δὲ αὐτῷ, ἰδοὺ οὐθεὶς μεθʼ ἡμῶν ἐστιν· ἴδε ὁ Θεὸς μάρτυς ἀνὰ μέσον ἐμοῦ καὶ σοῦ.
\vs{45}Λαβὼν δὲ Ἰακὼβ λίθον, ἔστησεν αὐτὸν στήλην.
\vs{46}Εἶπε δὲ Ἰακὼβ τοῖς ἀδελφοῖς αὐτοῦ, συλλέγετε λίθους· καὶ συνέλεξαν λίθους, καὶ ἐποίησαν βουνόν· καὶ ἔφαγον ἐκεῖ ἐπὶ τοῦ βουνοῦ· καὶ εἶπεν αὐτῷ Λάβαν, ὁ βουνὸς οὗτος μαρτυρεῖ ἀνὰ μέσον ἐμοῦ καὶ σοῦ σήμερον.
\vs{47}Καὶ ἐκάλεσεν αὐτὸν Λάβαν, βουνὸς τῆς μαρτυρίας· Ἰακὼβ δὲ ἐκάλεσεν αὐτὸν, βουνὸς μάρτυς.
\vs{48}Εἶπε δὲ Λάβαν τῷ Ἰακὼβ, ἰδοὺ ὁ βουνὸς οὗτος καὶ ἡ στήλη, ἣν ἔστησα ἀνὰ μέσον ἐμοῦ καὶ σοῦ· μαρτυρεῖ ὁ βουνὸς οὗτος, καὶ μαρτυρεῖ ἡ στήλη αὕτη· διὰ τοῦτο ἐκλήθη τὸ ὄνομα, βουνὸς μαρτυρεῖ.
\vs{49}Καὶ ἡ ὅρασις, ἣν εἶπεν, ἐπίδοι ὁ Θεὸς ἀνὰ μέσον ἐμοῦ καὶ σοῦ· ὅτι ἀποστησόμεθα ἕτερος ἀφʼ ἑτέρου.
\vs{50}Εἰ ταπεινώσεις τὰς θυγατέρας μου, εἰ λάβῃς γυναῖκας πρὸς ταῖς θυγατράσι μου, ὅρα, οὐθεὶς μεθʼ ἡμῶν ἐστιν ὁρῶν· Θεὸς μάρτυς μεταξὺ ἐμοῦ καὶ μεταξὺ σοῦ.
\vs{50a}Καὶ εἶπε Λάβαν τῷ Ἰακὼβ, ἰδοὺ ὁ βουνὸς οὗτος καὶ μάρτυς ἡ στήλη αὕτη.
\vs{52}Ἐάν τε γὰρ ἐγὼ μὴ διαβῶ πρός σε, μήτε σὺ διαβῇς πρός με τὸν βουνὸν τοῦτον καὶ τὴν στήλην ταύτην ἐπὶ κακίᾳ.
\vs{53}Ὁ Θεὸς Ἁβραὰμ καὶ ὁ Θεὸς Ναχὼρ κρίναι ἀνὰ μέσον ἡμῶν· καὶ ὤμοσεν Ἰακὼβ κατὰ τοῦ φόβου τοῦ πατρὸς αὐτοῦ Ἰσαάκ.
\vs{54}Καὶ ἔθυσεν θυσίαν ἐν τῷ ὄρει· καὶ ἐκάλεσε τοὺς ἀδελφοὺς αὐτοῦ, καὶ ἔφαγον καὶ ἔπιον, καὶ ἐκοιμήθησαν ἐν τῷ ὄρει.

\ch{32}Ἀναστὰς δὲ Λάβαν τὸ πρωῒ, κατεφίλησε τοὺς υἱοὺς καὶ τὰς θυγατέρας αὐτοῦ, καὶ εὐλόγησεν αὐτούς· καὶ ἀποστραφεὶς Λάβαν ἀπῆλθεν εἰς τὸν τόπον αὐτοῦ.

\vs{2}Καὶ Ἰακὼβ ἀπῆλθεν εἰς τὴν ὁδὸν ἑαυτοῦ· καὶ ἀναβλέψας εἶδε παρεμβολὴν Θεοῦ παρεμβεβληκυῖαν· καὶ συνήντησαν αὐτῷ οἱ Ἄγγελοι τοῦ Θεοῦ.
\vs{3}Εἶπε δὲ Ἰακὼβ, ἡνίκα εἶδεν αὐτοὺς, παρεμβολὴ Θεοῦ αὕτη· καὶ ἐκάλεσε τὸ ὄνομα τοῦ τόπου ἐκείνου, Παρεμβολαί.

\vs{4}Ἀπέστειλε δὲ Ἰακὼβ ἀγγέλους ἔμπροσθεν αὐτοῦ πρὸς Ἡσαῦ τὸν ἀδελφὸν αὐτοῦ εἰς γῆν Σηεὶρ, εἰς χώραν Ἐδώμ.
\vs{5}Καὶ ἐνετείλατο αὐτοῖς, λέγων, οὕτως ἐρεῖτε τῷ κυρίῳ μου Ἡσαῦ· οὕτως λέγει ὁ παῖς σου Ἰακώβ· μετὰ Λάβαν παρῴκησα, καὶ ἐχρόνισα ἕως τοῦ νῦν.
\vs{6}Καὶ ἐγένοντό μοι βόες, καὶ ὄνοι, καὶ πρόβατα, καὶ παῖδες, καὶ παιδίσκαι· καὶ ἀπέστειλα ἀναγγεῖλαι τῷ κυρίῳ μου Ἡσαῦ, ἵνα εὕρῃ ὁ παῖς σου χάριν ἐναντίον σου.
\vs{7}Καὶ ἀνέστρεψαν οἱ ἄγγελοι πρὸς Ἰακὼβ, λέγοντες, ἤλθομεν πρὸς τὸν ἀδελφόν σου Ἡσαυ· καὶ ἰδοὺ αὐτὸς ἔρχεται εἰς συνάντησίν σου, καὶ τετρακόσιοι ἄνδρες μεθʼ αὐτοῦ.
\vs{8}Ἐφοβήθη δὲ Ἰακὼβ σφόδρα, καὶ ἠπορεῖτο· καὶ διεῖλε τὸν λαὸν τὸν μεθʼ ἑαυτοῦ, καὶ τοὺς βόας, καὶ τὰς καμήλους, καὶ τὰ πρόβατα, εἰς δύο παρεμβολάς.
\vs{9}Καὶ εἶπεν Ἰακὼβ, ἐὰν ἔλθῃ Ἡσαῦ εἰς παρεμβολὴν μίαν, καὶ κόψῃ αὐτὴν, ἔσται ἡ παρεμβολὴ ἡ δευτέρα εἰς τὸ σώζεσθαι.
\vs{10}Εἶπε δὲ Ἰακὼβ, ὁ Θεὸς τοῦ πατρός μου Ἁβραὰμ, καὶ ὁ Θεὸς τοῦ πατρός μου Ἰσαὰκ, Κύριε σὺ ὁ εἰπών μοι, ἀπότρεχε εἰς τὴν γῆν τῆς γενέσεώς σου, καὶ εὖ σε ποιήσω·
\vs{11}Ἱκανούσθω μοι ἀπὸ πάσης δικαιοσύνης, καὶ ἀπὸ πάσης ἀληθείας, ἧς ἐποίησας τῷ παιδί σου· ἐν γὰρ τῇ ῥάβδῳ μου ταύτῃ διέβην τὸν Ἰορδάνην τοῦτον· νυνὶ δὲ γέγονα εἰς δύο παρεμβολάς.
\vs{12}Ἐξελοῦ με ἐκ χειρὸς τοῦ ἀδελφοῦ μου, ἐκ χειρὸς Ἡσαῦ· ὅτι φοβοῦμαι ἐγὼ αὐτὸν, μή ποτε ἐλθὼν πατάξῃ με, καὶ μητέρα ἐπὶ τέκνοις.
\vs{13}Σὺ δὲ εἶπας, εὐ σε ποιήσω, καὶ θήσω τὸ σπέρμα σου ὡς τὴν ἄμμον τῆς θαλάσσης, ἣ οὐκ ἀριθμηθήσεται ὑπὸ τοῦ πλήθους.
\vs{14}Καὶ ἐκοιμήθη ἐκεῖ τὴν νύκτα ἐκείνην· καὶ ἔλαβεν ὧν ἔφερεν δῶρα· καὶ ἐξαπέστειλεν Ἡσαῦ τῷ ἀδελφῷ αὐτοῦ,
\vs{15}αἶγας διακοσίας, τράγους εἴκοσι, πρόβατα διακόσια, κριοὺς εἴκοσι,
\vs{16}καμήλους θηλαζούσας καὶ τὰ παιδία αὐτῶν τριάκοντα, βόας τεσσαράκοντα, ταύρους δέκα, ὄνους εἴκοσι, καὶ πώλους δέκα.
\vs{17}Καὶ ἔδωκεν αὐτὰ τοῖς παισὶν αὐτοῦ ποίμνιον κατὰ μόνας· εἶπε δὲ τοῖς παισὶν αὐτοῦ, προπορεύεσθε ἔμπροσθέν μου, καὶ διάστημα ποιεῖτε ἀνὰ μέσον ποίμνης καὶ ποίμνης.
\vs{18}Καὶ ἐνετείλατο τῷ πρώτῳ, λέγων, ἐάν σοι συναντήσῃ Ἡσαῦ ὁ ἀδελφός μου, καὶ ἐρωτᾷ σε, λέγων, τίνος εἶ; καὶ ποῦ πορεύῃ; καὶ τίνος ταῦτα τὰ προπορευόμενά σου;
\vs{19}Ἐρεῖς, τοῦ παιδός σου Ἰακώβ· δῶρα ἀπέσταλκε τῷ κυρίῳ μου Ἡσαῦ· καὶ ἰδοὺ αὐτὸς ὀπίσω ἡμῶν.
\vs{20}Καὶ ἐνετείλατο τῷ πρώτῳ, καὶ τῷ δευτέρῳ, καὶ τῷ τρίτῳ, καὶ πᾶσι τοῖς προπορευομένοις ὀπίσω τῶν ποιμνίων τούτων, λέγων, κατὰ τὸ ῥῆμα τοῦτο λαλήσατε Ἡσαῦ ἐν τῷ εὑρεῖν ὑμᾶς αὐτόν·
\vs{21}Καὶ ἐρεῖτε, ἰδοὺ ὁ παῖς σου Ἰακὼβ παραγίνεται ὀπίσω ἡμῶν· εἶπε γὰρ, ἐξιλάσομαι τὸ πρόσωπον αὐτοῦ ἐν τοῖς δώροις τοῖς προπορευομένοις αὐτοῦ, καὶ μετὰ τοῦτο ὄψομαι τὸ πρόσωπον αὐτοῦ· ἴσως γὰρ προσδέξεται τὸ πρόσωπόν μου.
\vs{22}Καὶ προεπορεύετο τὰ δῶρα κατὰ πρόσωπον αὐτοῦ· αὐτὸς δὲ ἐκοιμήθη τὴν νύκτα ἐκείνην ἐν τῇ παρεμβολῇ.
\vs{23}Ἀναστὰς δὲ τὴν νύκτα ἐκείνην, ἔλαβε τὰς δύο γυναῖκας, καὶ τὰς δύο παιδίσκας, καὶ τὰ ἕνδεκα παιδία αὐτοῦ, καὶ διέβη τὴν διάβασιν τοῦ Ἰαβώχ.
\vs{24}Καὶ ἔλαβεν αὐτοὺς, καὶ διέβη τὸν χειμάῤῥουν, καὶ διεβίβασε πάντα τὰ αὐτοῦ.

\vs{25}Ὑπελείφθη δὲ Ἰακὼβ μόνος· καὶ ἐπάλαιεν ἄνθρωπος μετʼ αὐτοῦ ἕως πρωΐ.
\vs{26}Εἶδε δὲ ὅτι οὐ δύναται πρὸς αὐτόν· καὶ ἥψατο τοῦ πλάτους τοῦ μηροῦ αὐτοῦ, καὶ ἐνάρκησε τὸ πλάτος τοῦ μηροῦ Ἰακὼβ ἐν τῷ παλαίειν αὐτὸν μετʼ αὐτοῦ.
\vs{27}Καὶ εἶπεν αὐτῷ, ἀπόστειλόν με, ἀνέβη γὰρ ὁ ὄρθρος. ὁ δὲ εἶπεν, οὐ μή σε ἀποστείλω, ἐὰν μή με εὐλογήσῃς.
\vs{28}Εἶπε δὲ αὐτῷ, τί τὸ ὄνομά σου ἐστίν; ὁ δὲ εἶπεν, Ἰακώβ.
\vs{29}Καὶ εἶπεν αὐτῷ, οὐ κληθήσεται ἔτι τὸ ὄνομά σου Ἰακὼβ, ἀλλʼ Ἰσραὴλ ἔσται τὸ ὄνομά σου· ὅτι ἐνίσχυσας μετὰ Θεοῦ, καὶ μετὰ ἀνθρώπων δυνατὸς ἔσῃ.
\vs{30}Ἠρώτησε δὲ Ἰακὼβ, καὶ εἶπεν, ἀνάγγειλόν μοι τὸ ὄνομά σου· καὶ εἶπεν, ἱνατί τοῦτο ἐρωτᾷς σὺ τὸ ὄνομά μου; καὶ εὐλόγησεν αὐτὸν ἐκεῖ.
\vs{31}Καὶ ἐκάλεσεν Ἰακὼβ τὸ ὄνομα τοῦ τόπου ἐκείνου, εἶδος Θεοῦ· εἶδον γὰρ Θεὸν πρόσωπον πρὸς πρὸσωπον, καὶ ἐσώθη μου ἡ ψυχή.
\vs{32}Ἀνέτειλεν δὲ αὐτῷ ὁ ἥλιος, ἡνίκα παρῆλθε τὸ εἶδος τοῦ Θεοῦ· αὐτὸς δὲ ἐπέσκαζε τῷ μηρῷ αὐτοῦ.
\vs{33}Ἕνεκεν τούτου οὐ μὴ φάγωσιν υἱοὶ Ἰσραὴλ τὸ νεῦρον, ὃ ἐνάρκησεν, ὅ ἐστιν ἐπὶ τοῦ πλάτους τοῦ μηροῦ, ἕως τῆς ἡμέρας ταύτης, ὅτι ἥψατο τοῦ πλάτους τοῦ μηροῦ Ἰακὼβ τοῦ νεύρου, ὃ ἐνάρκησεν.

\ch{33}
Ἀναβλέψας δὲ Ἰακὼβ τοῖς ὀφθαλμοῖς αὐτοῦ εἶδε· καὶ ἰδοὺ Ἡσαῦ ὁ ἀδελφὸς αὐτοῦ ἐρχόμενος, καὶ τετρακόσιοι ἄνδρες μετʼ αὐτοῦ· καὶ διεῖλεν Ἰακὼβ τὰ παιδία ἐπὶ Λείαν, καὶ ἐπὶ Ῥαχὴλ, καὶ τὰς δύο παιδίσκας.
\vs{2}Καὶ ἔθετο τὰς δύο παιδίσκας καὶ τοὺς υἱοὺς αὐτῶν ἐν πρώτοις, καὶ Λείαν καὶ τὰ παιδία αὐτῆς ὀπίσω, καὶ Ῥαχὴλ καὶ Ἰωσὴφ ἐσχάτους.
\vs{3}Αὐτὸς δὲ προῆλθεν ἔμπροσθεν αὐτῶν· καὶ προσεκύνησεν ἐπὶ τὴν γῆν ἑπτάκις, ἕως τοῦ ἐγγίσαι τῷ ἀδελφῷ αὐτοῦ.
\vs{4}Καὶ προσέδραμεν Ἡσαῦ εἰς συνάντησιν αὐτῷ· καὶ περιλαβὼν αὐτὸν προσέπεσεν ἐπὶ τὸν τράχηλον αὐτοῦ, καὶ κατεφίλησεν αὐτόν· καὶ ἔκλαυσαν ἀμφότεροι.
\vs{5}Καὶ ἀναβλέψας Ἡσαῦ εἶδε τὰς γυναῖκας καὶ τὰ παιδία· καὶ εἶπε, τί ταῦτά σοι ἐστίν; ὁ δὲ εἶπε, τὰ παιδία, οἷς ἠλέησεν ὁ Θεὸς τὸν παῖδά σου.
\vs{6}Καὶ προσήγγισαν αἱ παιδίσκαι καὶ τὰ τέκνα αὐτῶν, καὶ προσεκύνησαν.
\vs{7}Καὶ προσήγγισε Λεία καὶ τὰ τέκνα αὐτῆς, καὶ προσεκύνησαν· καὶ μετὰ ταῦτα προσήγγισε Ῥαχὴλ καὶ Ἰωσὴφ, καὶ προσεκύνησαν.
\vs{8}Καὶ εἶπε, τί ταῦτά σοι ἐστὶν, πᾶσαι αἱ παρεμβολαὶ αὗται, αἷς ἀπήντηκα; ὁ δὲ εἶπεν, ἵνα εὕρῃ ὁ παῖς σου χάριν ἐναντίον σου, κύριε.
\vs{9}Εἶπε δὲ Ἡσαῦ, ἔστι μοι πολλὰ, ἀδελφέ· ἔστω σοι τὰ σά.
\vs{10}Εἶπε δὲ Ἰακὼβ, εἰ εὓρον χάριν ἐναντίον σου, δέξαι τὰ δῶρα διὰ τῶν ἐμῶν χειρῶν· ἕνεκεν τούτου εἶδον τὸ πρόσωπόν σου, ὡς ἄν τις ἴδοι πρόσωπον Θεοῦ, καὶ εὐδοκήσεις με.
\vs{11}Λάβε τὰς εὐλογίας μου, ἃς ἤνεγκά σοι, ὅτι ἠλέησέ με ὁ Θεὸς, καὶ ἔστι μοι πάντα· καὶ ἐβιάσατο αὐτὸν, καὶ ἔλαβε.
\vs{12}Καὶ εἶπεν, ἀπάραντες πορευσώμεθα ἐπʼ εὐθεῖαν.
\vs{13}Εἶπε δὲ αὐτῷ, ὁ κύριός μου γινώσκει, ὅτι τὰ παιδία ἁπαλώτερα, καὶ τὰ πρόβατα καὶ αἱ βόες λοχεύονται ἐπʼ ἐμέ· ἐὰν οὖν καταδιώξω αὐτὰ ἡμέραν μίαν, ἀποθανοῦνται πάντα τὰ κτήνη.
\vs{14}Προελθέτω ὁ κύριός μου ἔμπροσθεν τοῦ παιδὸς αὐτοῦ· ἐγὼ δὲ ἐνισχύσω ἐν τῇ ὁδῷ κατὰ σχολὴν τῆς πορεύσεως τῆς ἐναντίον μου, καὶ κατὰ πόδα τῶν παιδαρίων, ἕως τοῦ ἐλθεῖν με πρὸς τὸν κύριόν μου εἰς Σηείρ.
\vs{15}Εἶπε δὲ Ἡσαῦ, καταλείψω μετὰ σοῦ ἀπὸ τοῦ λαοῦ τοῦ μετʼ ἐμοῦ· ὁ δὲ εἶπεν, ἱνατί τοῦτο; ἱκανὸν ὅτι εὗρον χάριν ἐναντίον σου, κύριε.
\vs{16}Ἀπέστρεψε δὲ Ἡσαῦ ἐν τῇ ἡμέρᾳ ἐκείνῃ εἰς τὴν ὁδὸν αὐτοῦ εἰς Σηείρ.
\vs{17}Καὶ Ἰακὼβ ἀπαίρει εἰς σκηνὰς, καὶ ἐποίησεν ἑαυτῷ ἐκεῖ οἰκίας, καὶ τοῖς κτήνεσιν αὐτοῦ ἐποίησε σκηνάς· διὰ τοῦτο ἐκάλεσε τὸ ὄνομα τοῦ τόπου ἐκείνου, Σκηναί.

\vs{18}Καὶ ἦλθεν Ἰακὼβ εἰς Σαλὴμ, πόλιν Σηκίμων, ἥ ἐστιν ἐν γῇ Χαναὰν, ὅτε ἐπανῆλθεν ἐκ τῆς Μεσοποταμίας Συρίας· καὶ παρενέλαβε κατὰ πρόσωπον τῆς πόλεως.
\vs{19}Καὶ ἐκτήσατο τὴν μερίδα τοῦ ἀγροῦ, οὗ ἔστησεν ἐκεῖ τὴν σκηνὴν αὐτοῦ, παρὰ Ἐμμὼρ πατρὸς Συχὲμ, ἑκατὸν ἀμνῶν.
\vs{20}Καὶ ἔστησεν ἐκεῖ θυσιαστήριον, καὶ ἐπεκαλέσατο τὸν Θεὸν Ἰσραήλ.

\ch{34}
Ἐξῆλθε δὲ Δείνα, ἡ θυγάτηρ Λείας, ἣν ἔτεκε τῷ Ἰακώβ, καταμαθεῖν τὰς θυγατέρας τῶν ἐγχωρίων.
\vs{2}Καὶ εἶδεν αὐτὴν Συχὲμ ὁ υἱὸς Ἐμμὼρ ὁ Εὐαῖος, ὁ ἄρχων τῆς γῆς· καὶ λαβὼν αὐτὴν, ἐκοιμήθη μετʼ αὐτῆς, καὶ ἐταπείνωσεν αὐτήν.
\vs{3}Καὶ προσέσχε τῇ ψυχῇ Δείνας τῆς θυγατρὸς Ἰακώβ· καὶ ἠγάπησε τὴν παρθένον· καὶ ἐλάλησε κατὰ τὴν διάνοιαν τῆς παρθένου αυτῇ.
\vs{4}Εἶπε Συχὲμ πρὸς Ἐμμὼρ τὸν πατέρα αὐτοῦ, λέγων, λάβε μοι τὴν παῖδα ταύτην εἰς γυναῖκα.
\vs{5}Ἰακὼβ δὲ ἤκουσεν, ὅτι ἐμίανεν ὁ υἱὸς Ἐμμὼρ Δείναν τὴν θυγατέρα αὐτοῦ· οἱ δὲ υἱοὶ αὐτοῦ ἦσαν μετὰ τῶν κτηνῶν αὐτοῦ ἐν τῷ πεδίῳ· παρεσιώπησε δὲ Ἰακὼβ, ἕως τοῦ ἐλθεῖν αὐτούς.
\vs{6}Ἐξῆλθε δὲ Ἐμμὼρ ὁ πατὴρ Συχὲμ πρὸς Ἰακὼβ, λαλῆσαι αὐτῷ.
\vs{7}Οἱ δὲ υἱοὶ Ἰακὼβ ἦλθον ἐκ τοῦ πεδίου· ὡς δὲ ἤκουσαν, κατενύγησαν οἱ ἄνδρες, καὶ λυπηρὸν ἦν αὐτοῖς σφόδρα· ὅτι ἄσχημον ἐποίησεν ἐν Ἰσραὴλ, κοιμηθεὶς μετὰ τῆς θυγατρὸς Ἰακώβ· καὶ οὐχ οὕτως ἔσται.
\vs{8}Καὶ ἐλάλησεν Ἐμμὼρ αὐτοῖς, λέγων, Συχὲμ ὁ υἱός μου προείλετο τῇ ψυχῇ τὴν θυγατέρα ὑμῶν· δότε οὖν αὐτὴν αὐτῷ γυναῖκα,
\vs{9}καὶ ἐπιγαμβρεύσασθε ἡμῖν· τὰς θυγατέρας ὑμῶν δότε ἡμῖν, καὶ τὰς θυγατέρας ἡμῶν λάβετε τοῖς υἱοῖς ὑμῶν.
\vs{10}Καὶ ἐν ἡμῖν κατοικεῖτε· καὶ ἡ γῆ ἰδοὺ πλατεῖα ἐναντίον ὑμῶν· κατοικεῖτε, καὶ ἐμπορεύεσθε ἐπʼ αὐτῆς, καὶ ἐγκτᾶσθε ἐν αὐτῇ.
\vs{11}Εἶπε δὲ Συχὲμ πρὸς τὸν πατέρα αὐτῆς, καὶ πρὸς τοὺς ἀδελφοὺς αὐτῆς, εὕροιμι χάριν ἐναντίον ὑμῶν· καὶ ὃ ἐὰν εἴπητε, δώσομεν.
\vs{12}Πληθύνατε τὴν φερνὴν σφόδρα, καὶ δώσω καθότι ἂν εἴπητέ μοι, καὶ δώσετέ μοι τὴν παῖδα ταύτην εἰς γυναῖκα.

\vs{13}Ἀπεκρίθησαν δὲ οἱ υἱοὶ Ἰακὼβ τῷ Συχὲμ, καὶ Ἐμμὼρ τῷ πατρὶ αὐτοῦ, μετὰ δόλου· καὶ ἐλάλησαν αὐτοῖς, ὅτι ἐμίαναν Δείναν τὴν ἀδελφὴν αὐτῶν.
\vs{14}Καὶ εἶπαν αὐτοῖς Συμεὼν καὶ Λευὶ οἱ ἀδελφοὶ Δείνας, οὐ δυνησόμεθα ποιῆσαι τὸ ῥῆμα τοῦτο, δοῦναι τὴν ἀδελφὴν ἡμῶν ἀνθρώπῳ, ὃς ἔχει ἀκροβυστίαν· ἔστι γὰρ ὄνειδος ἡμῖν.
\vs{15}Μόνον ἐν τούτῳ ὁμοιωθησόμεθα ὑμῖν, καὶ κατοικήσομεν ἐν ὑμῖν, ἐὰν γένησθε ὡς ἡμεῖς καὶ ὑμεῖς, ἐν τῷ περιτμηθῆναι ὑμῶν πᾶν ἀρσενικόν.
\vs{16}Καὶ δώσομεν τὰς θυγατέρας ἡμῶν ὑμῖν, καὶ ἀπὸ τῶν θυγατέρων ὑμῶν ληψόμεθα ἡμῖν γυναῖκας, καὶ οἰκήσομεν παρʼ ὑμῖν, καὶ ἐσόμεθα ὡς γένος ἕν.
\vs{17}Ἐὰν δὲ μὴ εἰσακούσητε ἡμῶν τοῦ περιτεμέσθαι, λαβόντες τὴν θυγατέρα ἡμῶν ἀπελευσόμεθα.
\vs{18}Καὶ ἤρεσαν οἱ λόγοι ἐναντίον Ἐμμὼρ, καὶ ἐναντίον Συχὲμ τοῦ υἱοῦ Ἐμμώρ.
\vs{19}Καὶ οὐκ ἐχρόνισεν ὁ νεανίσκος τοῦ ποιῆσαι τὸ ῥῆμα τοῦτο· ἐνέκειτο γὰρ τῇ θυγατρὶ Ἰακώβ· αὐτὸς δὲ ἦν ἐνδοξότατος πάντων τῶν ἐν τῷ οἴκῳ τοῦ πατρὸς αὐτοῦ.
\vs{20}Ἦλθε δὲ Ἐμμὼρ καὶ Συχὲμ ὁ υἱὸς αὐτοῦ πρὸς τὴν πύλην τῆς πόλεως αὐτῶν, καὶ ἐλάλησαν πρὸς τοὺς ἄνδρας τῆς πόλεως αὐτῶν, λέγοντες,
\vs{21}Οἱ ἄνθρωποι οὗτοι εἰρήνικοί εἰσι, μεθʼ ἡμῶν οἰκείτωσαν επὶ τῆς γῆς, καὶ ἐμπορευέσθωσαν αὐτήν· ἡ δὲ γῆ ἰδοὺ πλατεῖα ἐναντίον αὐτῶν· τὰς θυγατέρας αὐτῶν ληψόμεθα ἡμῖν γυναῖκας, καὶ τὰς θυγατέρας ἡμῶν δώσομεν αὐτοῖς.
\vs{22}Ἐν τούτῳ μόνον ὁμοιωθήσονται ἡμῖν οἱ ἄνθρωποι τοῦ κατοικεῖν μεθʼ ἡμῶν, ὥστε εἶναι λαὸν ἕνα, ἐν τῷ περιτεμέσθαι ἡμῶν πᾶν ἀρσενικὸν, καθὰ καὶ αὐτοὶ περιτέτμηνται.
\vs{23}Καὶ τὰ κτήνη αὐτῶν, καὶ τὰ τετράποδα, καὶ τὰ ὑπάρχοντα αὐτῶν, οὐχ ἡμῶν ἔσται; μόνον ἐν τούτῳ ὁμοιωθῶμεν αὐτοῖς, καὶ οἰκήσουσι μεθʼ ἡμῶν.
\vs{24}Καὶ εἰσήκουσαν Ἐμμὼρ καὶ Συχὲμ τοῦ υἱοῦ αὐτοῦ πάντες οἱ ἐμπορευόμενοι τὴν πύλην τῆς πόλεως αὐτῶν· καὶ περιετέμοντο τὴν σάρκα τῆς ἀκροβυστίας αὐτῶν πᾶς ἄρσην.

\vs{25}Ἐγένετο δὲ ἐν τῇ ἡμέρᾳ τῇ τρίτῃ, ὅτε ἦσαν ἐν τῷ πόνῳ, ἔλαβον οἱ δύο υἱοὶ Ἰακὼβ Συμεὼν καὶ Λευὶ, ἀδελφοὶ Δείνας, ἕκαστος τὴν μάχαιραν αὐτοῦ, καὶ εἰσῆλθον εἰς τὴν πόλιν ἀσφαλὼς, καὶ ἀπέκτειναν πᾶν ἀρσενικόν.
\vs{26}Τόν τε Ἐμμὼρ καὶ Συχὲμ τὸν υἱὸν αὐτοῦ ἀπέκτειναν ἐν στόματι μαχαίρας· καὶ ἔλαβον τὴν Δείναν ἐκ τοῦ οἴκου τοῦ Συχὲμ, καὶ ἐξῆλθον.
\vs{27}Οἱ δὲ υἱοὶ Ἰακὼβ εἰσῆλθον ἐπὶ τοὺς τραυματίας, καὶ διήρπασαν τὴν πόλιν, ἐν ᾗ ἐμίαναν Δείναν τὴν ἀδελφὴν αὐτῶν.
\vs{28}Καὶ τὰ πρόβατα αὐτῶν, καὶ τοὺς βόας αὐτῶν, καὶ τοὺς ὄνους αὐτῶν, ὅσα τε ἦν ἐν τῇ πόλει, καὶ ὅσα ἦν ἐν τῷ πεδίῳ, ἔλαβον.
\vs{29}Καὶ πάντα τὰ σώματα αὐτῶν, καὶ πᾶσαν τὴν ἀποσκευὴν αὐτῶν, καὶ τὰς γυναῖκας αὐτῶν ἠχμαλώτευσαν· καὶ διήρπασαν ὅσα τε ἦν ἐν τῇ πόλει, καὶ ὅσα ἦν ἐν ταῖς οἰκίαις.
\vs{30}Εἶπε δὲ Ἰακὼβ πρὸς Συμεὼν καὶ Λευὶ, μισητόν με πεποιήκατε, ὥστε πονηρόν με εἶναι πᾶσι τοῖς κατοικοῦσι τὴν γῆν, ἔν τε τοῖς Χαναναίοις, καὶ ἐν τοῖς Φερεζαίοις· ἐγὼ δὲ ὀλιγοστός εἰμι ἐν ἀριθμῷ· καὶ συναχθέντες ἐπʼ ἐμὲ συγκόψουσί με, καὶ ἐκτριβήσομαι ἐγὼ, καὶ ὁ οἶκός μου.
\vs{31}Οἱ δὲ εἶπαν, ἀλλʼ ὡσεὶ πόρνῃ χρήσονται τῇ ἀδελφῇ ἡμῶν;

\ch{35}
Εἶπε δὲ ὁ Θεὸς πρὸς Ἰακὼβ, ἀναστὰς ἀνάβηθι εἰς τὸν τόπον Βαιθὴλ, καὶ οἴκει ἐκεῖ· καὶ ποίησον ἐκεῖ θυσιαστήριον τῷ Θεῷ τῷ ὀφθέντι σοι, ἐν τῷ ἀποδιδράσκειν σε ἀπὸ προσώπου Ἡσαῦ τοῦ ἀδελφοῦ σου.
\vs{2}Εἶπε δὲ Ἰακὼβ τῷ οἴκῳ αὐτοῦ, καὶ πᾶσι τοῖς μετʼ αὐτοῦ, ἄρατε τοὺς θεοὺς τοὺς ἀλλοτρίους τοὺς μεθʼ ὑμῶν ἐκ μέσου ὑμῶν, καὶ καθαρίσθητε, καὶ ἀλλάξατε τὰς στολὰς ὑμῶν.
\vs{3}Καὶ ἀναστάντες ἀναβῶμεν εἰς Βαιθὴλ, καὶ ποιήσωμεν ἐκεῖ θυσιαστήριον τῷ Θεῷ τῷ ἐπακούσαντί μου ἐν ἡμέρᾳ θλίψεως, ὃς ἦν μετʼ ἐμοῦ, καὶ διέσωσέ με ἐν τῇ ὁδῷ, ᾗ ἐπορεύθην.
\vs{4}Καὶ ἔδωκαν τῷ Ἰακὼβ τοὺς θεοὺς τοὺς ἀλλοτρίους, οἳ ἦσαν ἐν ταῖς χερσὶν αὐτῶν, καὶ τὰ ἐνώτια τὰ ἐν τοῖς ὠσὶν αὐτῶν· καὶ κατέκρυψεν αὐτὰ Ἰακὼβ ὑπὸ τὴν τερέβινθον τὴν ἐν Σηκίμοις· καὶ ἀπώλεσεν αὐτὰ, ἕως τῆς σήμερον ἡμέρας.
\vs{5}Καὶ ἐξῇρεν Ἰσραὴλ ἐκ Σηκίμων· καὶ ἐγένετο φόβος Θεοῦ ἐπὶ τὰς πόλεις τὰς κύκλῳ αὐτῶν, καὶ οὐ κατεδίωξαν ὀπίσω τῶν υἱῶν Ἰσραήλ.
\vs{6}Ἦλθε δὲ Ἰακὼβ εἰς Λουζὰ ἥ ἐστιν ἐν γῇ Χαναὰν, ἥ ἐστι Βαιθὴλ, αὐτὸς, καὶ πᾶς ὁ λαὸς, ὃς ἦν μετʼ αὐτοῦ.
\vs{7}Καὶ ᾠκοδόμησεν ἐκεῖ θυσιαστήριον, καὶ ἐκάλεσε τὸ ὄνομα τοῦ τόπου, Βαιθήλ· ἐκεῖ γὰρ ἐφάνη αὐτῷ ὁ Θεὸς, ἐν τῷ ἀποδιδράσκειν αὐτὸν ἀπὸ προσώπου Ἡσαῦ τοῦ ἀδελφοῦ αὐτοῦ.

\vs{8}Ἀπέθανε δὲ Δεβόῤῥα, ἡ τρόφος Ῥεβέκκας, καὶ ἐτάφη κατώτερον Βαιθὴλ ὑπὸ τὴν βάλανον· καὶ ἐκάλεσεν Ἰακὼβ τὸ ὄνομα αὐτῆς, βάλανος πένθους.
\vs{9}Ὤφθη δὲ ὁ Θεὸς τῷ Ἰακὼβ ἔτι ἐν Λουζᾷ, ὅτε παρεγένετο ἐκ Μεσοποταμίας τῆς Συρίας· καὶ εὐλόγησεν αὐτὸν ὁ Θεὸς.
\vs{10}Καὶ εἶπεν αὐτῷ ὁ Θεὸς, τὸ ὄνομά σου οὐ κληθήσεται ἔτι Ἰακὼβ, ἀλλʼ Ἰσραὴλ ἔσται τὸ ὄνομά σου· καὶ ἐκάλεσε τὸ ὄνομα αὐτοῦ Ἰσραήλ.
\vs{11}Εἶπε δὲ αὐτῷ ὁ Θεὸς, ἐγὼ ὁ Θεός σου· αὐξάνου, καὶ πληθύνου· ἔθνη καὶ συναγωγαὶ ἐθνῶν ἔσονται ἐκ σοῦ, καὶ βασιλεῖς ἐκ τῆς ὀσφύος σου ἐξελεύσονται.
\vs{12}Καὶ τὴν γῆν, ἣν ἔδωκα Ἁβραὰμ καὶ Ἰσαὰκ, σοὶ δέδωκα αὐτήν· σοὶ ἔσται· καὶ τῷ σπέρματί σου μετὰ σὲ δώσω τῆν γῆν ταύτην.
\vs{13}Ἀνέβη δὲ ὁ Θεὸς ἀπʼ αὐτοῦ ἐκ τοῦ τόπου, οὗ ἐλάλησε μετʼ αὐτοῦ.
\vs{14}Καὶ ἔστησεν Ἰακὼβ στήλην ἐν τῷ τόπῳ, ᾧ ἐλάλησε μετʼ αὐτοῦ ὁ Θεὸς, στήλην λιθίνην· καὶ ἔσπεισεν ἐπʼ αὐτὴν σπονδὴν, καὶ ἐπέχεεν ἐπʼ αὐτὴν ἔλαιον.
\vs{15}Καὶ ἐκάλεσεν Ἰακὼβ τὸ ὄνομα τοῦ τόπου, ἐν ᾧ ἐλάλησε μετʼ αὐτοῦ ἐκεῖ ὁ Θεὸς, Βαιθήλ.
\vs{16}Ἀπάρας δὲ Ἰακὼβ ἐκ Βαιθὴλ, ἔπηξε τὴν σκηνὴν αὐτοῦ ἐπέκεινα τοῦ πύργου Γαδέρ· ἐγένετο δὲ ἡνίκα ἤγγισεν εἰς Χαβραθὰ τοῦ ἐλθεῖν εἰς τὴν Ἐφραθᾶ, ἔτεκε Ῥαχήλ· καὶ ἐδυστόκησεν ἐν τῷ τοκετῷ.
\vs{17}Ἐγένετο δὲ ἐν τῷ σκληρὼς αὐτὴν τίκτειν, εἶπεν αὐτῇ ἡ μαῖα, θάρσει, καὶ γὰρ οὗτός σοι ἐστὶν υἱός.
\vs{18}Ἐγένετο δὲ ἐν τῷ ἀφιέναι αὐτὴν τὴν ψυχὴν, ἀπέθνησκε γὰρ, ἐκάλεσε τὸ ὄνομα αὐτοῦ, υἱὸς ὀδύνης μου· ὁ δὲ πατὴρ ἐκάλεσεν τὸ ὄνομα αὐτοῦ, Βενιαμίν.
\vs{19}Ἀπέθανε δὲ Ῥαχὴλ, καὶ ἐτάφη ἐν τῇ ὁδῷ τοῦ ἱπποδρόμου Ἐφραθᾶ· αὕτη ἐστὶ Βηθλεέμ.
\vs{20}Καὶ ἔστησεν Ἰακὼβ στήλην ἐπὶ τοῦ μνημείου αὐτῆς· αὕτη ἐστὶν ἡ στήλη ἐπὶ τοῦ μνημείου Ῥαχὴλ ἕως τῆς ἡμέρας ταύτης.
\vs{22}Ἐγένετο δὲ ἡνίκα κατῴκησεν Ἰσραὴλ ἐν τῇ γῇ ἐκείνῃ, ἐπορεύθη Ῥουβὴν, καὶ ἐκοιμήθη μετὰ Βαλλὰς, τῆς παλλακῆς τοῦ πατρὸς αὐτοῦ Ἰακώβ· καὶ ἤκουσεν Ἰσραὴλ, καὶ πονηρὸν ἐφάνη ἐναντίον αὐτοῦ.

Ἦσαν δὲ οἱ υἱοὶ Ἰακὼβ, δώδεκα.
\vs{23}Υἱοὶ Λείας, πρωτότοκος Ἰακὼβ, Ῥουβὴν, Συμεὼν, Λευὶ, Ἰούδας, Ἰσσάχαρ, Ζαβουλών.
\vs{24}Υἱοὶ δὲ Ῥαχὴλ, Ἰωσὴφ, καὶ Βενιαμίν.
\vs{25}Υἱοὶ δὲ Βαλλᾶς παιδίσκης Ῥαχὴλ, Δαν, καὶ Νεφθαλείμ.
\vs{26}Υἱοὶ δὲ Ζελφᾶς παιδίσκης Λείας, Γὰδ, καὶ Ἀσήρ· οὗτοι υἱοὶ Ἰακὼβ, οἳ ἐγένοντο αὐτῷ ἐν Μεσοποταμίᾳ τῆς Συρίας.
\vs{27}Ἦλθε δὲ Ἰακὼβ πρὸς Ἰσαὰκ τὸν πατέρα αὐτοῦ εἰς Μαμβρῆ, εἰς πόλιν τοῦ πεδίου· αὕτη ἐστὶ Χεβρὼν ἐν γῇ Χαναὰν, οὗ παρῴκησεν Ἁβραὰμ καὶ Ἰσαάκ.
\vs{28}Ἐγένοντο δὲ αἱ ἡμέραι Ἰσαὰκ, ἃς ἔζησεν, ἔτη ἑκατὸν ὀγδοήκοντα.
\vs{29}Καὶ ἐκλείπων Ἰσαὰκ ἀπέθανε, καὶ προσετέθη πρὸς τὸ γένος αὐτοῦ πρεσβύτερος καὶ πλήρης ἡμερῶν· καὶ ἔθαψαν αὐτὸν Ἡσαῦ καὶ Ἰακὼβ οἱ υἱοὶ αὐτοῦ.

\ch{36}
Αὗται δὲ αἱ γενέσεις Ἡσαῦ· αὐτός ἐστιν Ἐδώμ.
\vs{2}Ἡσαῦ δὲ ἔλαβε τὰς γυναῖκας ἑαυτῷ ἀπὸ τῶν θυγατέρων τῶν Χαναναίων· τὴν Ἀδὰ, θυγατέρα Αἰλὼμ τοῦ Χετταίου· καὶ τὴν Ὀλιβεμὰ, θυγατέρα Ἀνὰ τοῦ υἱοῦ Σεβεγὼν τοῦ Εὐαίου.
\vs{3}Καὶ τὴν Βασεμὰθ, θυγατέρα Ἰσμαὴλ, ἀδελφὴν Ναβαιώθ.
\vs{4}Ἔτεκε δὲ αὐτῷ Ἀδὰ τὸν Ἑλιφάς· καὶ Βασεμὰθ ἔτεκε τὸν Ῥαγουήλ.
\vs{5}Καὶ Ὀλιβεμὰ ἔτεκε τὸν Ἰεοὺς, καὶ τὸν Ἰεγλὸμ, καὶ τὸν Κορέ· οὗτοι υἱοὶ Ἡσαῦ, οἳ ἐγένοντο αὐτῷ ἐν γῇ Χαναάν.
\vs{6}Ἔλαβε δὲ Ἡσαῦ τὰς γυναῖκας αὐτοῦ, καὶ τοὺς υἱοὺς αὐτοῦ, καὶ τὰς θυγατέρας αὐτοῦ, καὶ πάντα τὰ σώματα τοῦ οἴκου αὐτοῦ, καὶ πάντα τὰ ὑπάρχοντα αὐτοῦ, καὶ πάντα τὰ κτήνη, καὶ πάντα ὅσα ἐκτήσατο, καὶ πάντα ὅσα περιεποιήσατο ἐν γῇ Χαναάν· καὶ ἐπορεύθη Ἡσαῦ ἐκ τῆς γῆς Χαναὰν ἀπὸ προσώπου Ἰακὼβ τοῦ ἀδελφοῦ αὐτοῦ.
\vs{7}Ἦν γὰρ αὐτῶν τὰ ὑπάρχοντα πολλὰ, τοῦ οἰκεῖν ἅμα· καὶ οὐκ ἠδύνατο ἡ γῆ τῆς παροικήσεως αὐτῶν φέρειν αὐτοὺς, ἀπὸ τοῦ πλήθους τῶν ὑπαρχόντων αὐτῶν.
\vs{8}Κατῴκησε δὲ Ἡσαῦ ἐν τῷ ὄρει Σηείρ· Ἡσαῦ αὐτός ἐστιν Ἐδώμ.
\vs{9}Αὗται δὲ αἱ γενέσεις Ἡσαῦ πατρὸς Ἐδὼμ ἐν τῷ ὄρει Σηείρ.
\vs{10}Καὶ ταῦτα τὰ ὀνόματα τῶν υἱῶν Ἡσαῦ· Ἑλιφὰς υἱὸς Ἀδὰς γυναικὸς Ἡσαῦ· καὶ Ῥαγουὴλ υἱὸς Βασεμὰθ γυναικὸς Ἡσαῦ.
\vs{11}Ἐγένοντο δὲ Ἑλιφὰς υἱοὶ, Θαιμὰν, Ὠμὰρ, Σωφὰρ, Γοθὼμ, καὶ Κενέζ.
\vs{12}Θαμνὰ δὲ ἦν παλλακὴ Ἑλιφὰς τοῦ υἱοῦ Ἡσαῦ· καὶ ἔτεκε τῷ Ἑλιφὰς τὸν Ἀμαλήκ· οὗτοι υἱοὶ Ἀδὰς γυναικὸς Ἡσαῦ.
\vs{13}Οὗτοι δὲ υἱοὶ Ῥαγουὴλ, Ναχὼθ, Ζαρὲ, Σομὲ, καὶ Μοζέ· οὗτοι ἦσαν υἱοὶ Βασεμὰθ γυναικὸς Ἡσαῦ.
\vs{14}Οὗτοι δὲ υἱοὶ Ὀλιβεμὰς θυγατρὸς Ἀνὰ τοῦ υἱοῦ Σεβεγὼν, γυναικὸς Ἡσαῦ· ἔτεκε δὲ τῷ Ἡσαῦ τὸν Ἰεοὺς, καὶ τὸν Ἰεγλὸμ, καὶ τὸν Κορέ.
\vs{15}Οὗτοι ἡγεμόνες υἱοὶ Ἡσαῦ· υἱοὶ Ἑλιφὰς πρωτοτόκου Ἡσαῦ· ἡγεμὼν Θαιμὰν, ἡγεμὼν Ὠμὰρ, ἡγεμὼν Σωφὰρ, ἡγεμὼν Κενὲζ,
\vs{16}ἡγεμὼν Κορὲ, ἡγεμὼν Γοθὼμ, ἡγεμὼν Ἀμαλήκ· οὗτοι ἡγεμόνες Ἑλιφὰς ἐν γῇ Ἰδουμαίᾳ· οὗτοι υἱοὶ Ἀδάς.
\vs{17}Καὶ οὗτοι υἱοὶ Ῥαγουὴλ υἱοῦ Ἡσαῦ· ἡγεμὼν Ναχὼθ, ἡγεμὼν Ζαρὲ, ἡγεμὼν Σομὲ, ἡγεμὼν Μοζέ· οὗτοι ἡγεμόνες Ῥαγουὴλ ἐν γῇ Ἐδώμ· οὗτοι υἱοὶ Βασεμὰθ γυναικὸς Ἡσαῦ.
\vs{18}Οὗτοι δὲ υἱοὶ Ὀλιβεμὰς γυναικὸς Ἡσαῦ· ἡγεμὼν Ἰεοὺς, ἡγεμὼν Ἰεγλὸμ, ἡγεμὼν Κορέ· οὗτοι ἡγεμόνες Ὀλιβεμὰς θυγατρὸς Ἀνὰ γυναικὸς Ἡσαῦ.
\vs{19}Οὗτοι υἱοὶ Ἡσαῦ, καὶ οὗτοι ἡγεμόνες αὐτῶν· οὗτοί εἰσιν υἱοὶ Ἐδώμ.
\vs{20}Οὗτοι δὲ υἱοὶ Σηεὶρ τοῦ Χοῤῥαίου, τοῦ κατοικοῦντος τὴν γῆν· Λωτὰν, Σωβὰλ, Σεβεγὼν, Ἀνὰ,
\vs{21}καὶ Δησὼν, καὶ Ἀσὰρ, καὶ Ῥισών· οὗτοι ἡγεμόνες τοῦ Χοῤῥαίου, τοῦ υἱοῦ Σηεὶρ ἐν τῇ γῇ Ἐδώμ.
\vs{22}Ἐγένοντο δὲ υἱοὶ Λωτάν· Χοῤῥὶ, καὶ Αἱμάν· ἀδελφὴ δὲ Λωτὰν, Θαμνά.
\vs{23}Οὗτοι δὲ υἱοὶ Σωβάλ· Γωλὰμ, καὶ Μαναχὰθ, καὶ Γαιβὴλ, καὶ Σωφὰρ, καὶ Ὠμάρ.
\vs{24}Καὶ οὗτοι υἱοὶ Σεβεγὼν, Ἀϊὲ, καὶ Ἀνά· οὗτός ἐστιν Ἀνὰ, ὃς εὗρε τὸν Ἰαμεὶν ἐν τῇ ἐρήμῳ, ὅτε ἔνεμε τὰ ὑποζύγια Σεβεγὼν τοῦ πατρὸς αὐτοῦ·
\vs{25}Οὗτοι δὲ υἱοὶ Ἀνά· Δησὼν, καὶ Ὀλιβεμὰ θυγάτηρ Ἀνά.
\vs{26}Οὗτοι δὲ υἱοὶ Δησών· Ἀμαδὰ, καὶ Ἀσβὰν, καὶ Ἰθρὰν, καὶ Χαῤῥάν.
\vs{27}Οὗτοι δὲ υἱοὶ Ἀσάρ· Βαλαὰμ, καὶ Ζουκὰμ, καὶ Ἰουκάμ.
\vs{28}Οὗτοι δὲ υἱοὶ Ῥισὼν, Ὧς, καὶ Ἀράν.
\vs{29}Οὗτοι δὲ ἡγεμόνες Χοῤῥί· ἡγεμὼν Λωτὰν, ἡγεμὼν Σωβὰλ, ἡγεμὼν Σεβεγὼν, ἡγεμὼν Ἀνὰ,
\vs{30}ἡγεμὼν Δησὼν, ἡγεμὼν Ἀσὰρ, ἡγεμὼν Ῥισών· οὗτοι ἡγεμόνες Χοῤῥὶ ἐν ταῖς ἡγεμονίαις αὐτῶν ἐν γῇ Ἐδώμ.

\vs{31}Καὶ οὗτοι οἱ βασιλεῖς οἱ βασιλεύσαντες ἐν Ἐδὼμ, πρὸ τοῦ βασιλεῦσαι βασιλέα ἐν Ἰσραήλ.
\vs{32}Καὶ ἐβασίλευσεν ἐν Ἐδὼμ Βαλὰκ υἱὸς Βεώρ· καὶ ὄνομα τῇ πόλει αὐτοῦ, Δενναβά.
\vs{33}Ἀπέθανε δὲ Βαλὰκ, καὶ ἐβασίλευσεν ἀντʼ αὐτοῦ Ἰωβὰβ υἱὸς Ζαρὰ ἐκ Βοσόῤῥας.
\vs{34}Ἀπέθανε δὲ Ἰωβὰβ, καὶ ἐβασίλευσεν ἀντʼ αὐτοῦ Ἀσὼμ ἐκ τῆς γῆς Θαιμανών.
\vs{35}Ἀπέθανε δὲ Ἀσὼμ, καὶ ἐβασίλευσεν ἀντʼ αὐτοῦ Ἀδὰδ υἱὸς Βαρὰδ ὁ ἐκκόψας Μαδιὰμ ἐν τῷ πεδίῳ Μωάβ· καὶ ὄνομα τῇ πόλει αὐτοῦ Γετθαίμ.
\vs{36}Ἀπέθανε δὲ Ἀδὰδ, καὶ ἐβασίλευσεν ἀντʼ αὐτοῦ Σαμαδὰ ἐκ Μασσεκκάς.
\vs{37}Ἀπέθανε δὲ Σαμαδὰ, καὶ ἐβασίλευσεν ἀντʼ αὐτοῦ Σαοὺλ ἐκ Ῥοωβὼθ τῆς παρὰ ποταμόν.
\vs{38}Ἀπέθανε δὲ Σαοὺλ, καὶ ἐβασίλευσεν ἀντʼ αὐτοῦ Βαλλενὼν υἱὸς Ἀχοβώρ.
\vs{39}Ἀπέθανε δὲ Βαλλενὼν υἱὸς Ἀχοβὼρ, καὶ ἐβασίλευσεν ἀντʼ αὐτοῦ Ἀρὰδ υἱὸς Βαράδ· καὶ ὄνομα τῇ πόλει αὐτοῦ Φογώρ· ὄνομα δὲ τῇ γυναικὶ αὐτοῦ Μετεβεὴλ, θυγάτηρ Ματραῒθ, υἱοῦ Μαιζοώβ.
\vs{40}Ταῦτα τὰ ὀνόματα τῶν ἡγεμόνων Ἡσαῦ, ἐν ταῖς φυλαῖς αὐτῶν, κατὰ τόπον αὐτῶν, ἐν ταῖς χώραις αὐτῶν, καὶ ἐν τοῖς ἔθνεσιν αὐτῶν· ἡγεμὼν Θαμνὰ, ἡγεμὼν Γωλὰ, ἡγεμὼν Ἰεθὲρ,
\vs{41}ἡγεμὼν Ὁλιβεμὰς, ἡγεμὼν Ἡλὰς, ἡγεμὼν Φινὼν,
\vs{42}ἡγεμὼν Κενὲζ, ἡγεμὼν Θαιμὰν, ἡγεμὼν Μαζὰρ,
\vs{43}ἡγεμὼν Μαγεδιὴλ, ἡγεμὼν Ζαφωίν· οὗτοι ἡγεμόνες Ἐδὼμ, ἐν ταῖς κατῳκοδομημέναις ἐν τῇ γῇ τῆς κτήσεως αὐτῶν· οὗτος Ἡσαῦ πατὴρ Ἐδώμ.

\ch{37}
Κατῴκει δὲ Ἰακὼβ ἐν τῇ γῇ, οὗ παρῴκησεν ὁ πατὴρ αὐτοῦ ἐν γῇ Χαναάν· αὗται δὲ αἱ γενέσεις Ἰακώβ.
\vs{2}Ἰωσὴφ δὲ δέκα καὶ ἑπτὰ ἐτῶν ἦν, ποιμαίνων τὰ πρόβατα τοῦ πατρὸς αὐτοῦ μετὰ τῶν ἀδελφῶν αὐτοῦ, ὢν νέος, μετὰ τῶν υἱῶν Βαλλᾶς, καὶ μετὰ τῶν υἱῶν Ζελφᾶς, τῶν γυναικῶν τοῦ πατρὸς αὐτοῦ· κατήνεγκαν δὲ Ἰωσὴφ ψόγον πονηρὸν πρὸς Ἰσραὴλ τὸν πατέρα αὐτῶν.
\vs{3}Ἰακὼβ δὲ ἠγάπα τὸν Ἰωσὴφ παρὰ πάντας τοὺς υἱοὺς αὐτοῦ, ὅτι υἱὸς γήρως ἦν αὐτῷ· ἐποίησε δὲ αὐτῷ χιτῶνα ποικίλον.
\vs{4}Ἰδόντες δὲ οἱ ἀδελφοὶ αὐτοῦ, ὅτι αὐτὸν ὁ πατὴρ φιλεῖ ἐκ πάντων τῶν υἱῶν αὐτοῦ, ἐμίσησαν αὐτὸν, καὶ οὐκ ἠδύναντο λαλεῖν αὐτῷ οὐδὲν εἰρηνικόν.
\vs{5}Ἐνυπνιασθεὶς δὲ Ἰωσὴφ ἐνύπνιον, ἀπήγγειλεν αὐτὸ τοῖς ἀδελφοῖς αὐτοῦ.
\vs{6}Καὶ εἶπεν αὐτοῖς, ἀκούσατε τοῦ ἐνυπνίου τούτου, οὗ ἐνυπνιάσθην.
\vs{7}Ὤμην ὑμᾶς δεσμεύειν δράγματα ἐν μέσῳ τῷ πεδίῳ· καὶ ἀνέστη τὸ ἐμὸν δράγμα, καὶ ὠρθώθη· περιστραφέντα δὲ τὰ δράγματα ὑμῶν, προσεκύνησαν τὸ ἐμὸν δράγμα.
\vs{8}Εἶπαν δὲ αὐτῷ οἱ ἀδελφοὶ αὐτοῦ, μὴ βασιλεύων βασιλεύσεις ἐφʼ ἡμᾶς, ἢ κυριεύων κυριεύσεις ἡμῶν, καὶ προσέθεντο ἔτι μισεῖν αὐτὸν ἕνεκεν τῶν ἐνυπνίων αὐτοῦ, καὶ ἕνεκεν τῶν ῥημάτων αὐτοῦ.
\vs{9}Εἶδε δὲ ἐνύπνιον ἕτερον, καὶ διηγήσατο αὐτὸ τῷ πατρὶ αὐτοῦ, καὶ τοῖς ἀδελφοῖς αὐτοῦ· καὶ εἶπεν, ἰδοὺ ἐνυπνιασάμην ἐνύπνιον ἕτερον· ὥσπερ ὁ ἥλιος, καὶ ἡ σελήνη, καὶ ἕνδεκα ἀστέρες προσεκύνουν με.
\vs{10}Καὶ ἐπετίμησεν αὐτῷ ὁ πατὴρ αὐτοῦ, καὶ εἶπεν αὐτῷ, τί τὸ ἐνύπνιον τοῦτο, ὃ ἐνυπνιάσθης; ἆρά γε ἐλθόντες ἐλευσόμεθα ἐγώ τε καὶ ἡ μήτηρ σου καὶ οἱ ἀδελφοί σου προσκυνῆσαί σοι ἐπὶ τὴν γῆν;
\vs{11}Ἐζήλωσαν δὲ αὐτὸν οἱ ἀδελφοὶ αὐτοῦ· ὁ δὲ πατὴρ αὐτοῦ διετήρησε τὸ ῥῆμα.
\vs{12}Ἐπορεύθησαν δὲ οἱ ἀδελφοὶ αὐτοῦ βόσκειν τὰ πρόβατα τοῦ πατρὸς αὐτῶν εἰς Συχέμ.
\vs{13}Καὶ εἶπεν Ἰσραὴλ πρὸς Ἰωσὴφ, οὐχὶ οἱ ἀδελφοί σου ποιμαίνουσιν εἰς Συχέμ; δεῦρο ἀποστείλω σε πρὸς αὐτούς· εἶπε δὲ αὐτῷ, ἰδοὺ ἐγώ.
\vs{14}Εἶπε δὲ αὐτῷ Ἰσραὴλ, πορευθεὶς ἴδε, εἰ ὑγιαίνουσιν οἱ ἀδελφοί σου, καὶ τὰ πρόβατα, καὶ ἀνάγγειλόν μοι· καὶ ἀπέστειλεν αὐτὸν ἐκ τῆς κοιλάδος τῆς Χεβρών· καὶ ἦλθεν εἰς Συχέμ.
\vs{15}Καὶ εὗρεν αὐτὸν ἄνθρωπος πλανώμενον ἐν τῷ πεδίῳ· ἠρώτησε δὲ αὐτὸν ὁ ἄνθρωπος, λέγων, τί ζητεῖς;
\vs{16}Ὁ δὲ εἶπε, τοὺς ἀδελφούς μου ζητῶ· ἀπάγγειλόν μοι ποῦ βόσκουσιν.
\vs{17}Εἶπε δὲ αὐτῷ ὁ ἄνθρωπος, ἀπῄρκασιν ἐντεῦθεν· ἤκουσα γὰρ αὐτῶν λεγόντων, πορευθῶμεν εἰς Δωθαείμ· καὶ ἐπορεύθη Ἰωσὴφ κατόπισθε τῶν ἀδελφῶν αὐτοῦ, καὶ εὗρεν αὐτοὺς ἐν Δωθαείμ.

\vs{18}Προεῖδον δὲ αὐτὸν μακρόθεν πρὸ τοῦ ἐγγίσαι αὐτὸν πρὸς αὐτούς· καὶ ἐπονηρεύοντο τοῦ ἀποκτεῖναι αὐτόν.
\vs{19}Εἶπε δὲ ἕκαστος πρὸς τὸν ἀδελφὸν αὐτοῦ, ἰδοὺ ὁ ἐνυπνιαστὴς ἐκεῖνος ἔρχεται.
\vs{20}Νῦν οὖν δεῦτε ἀποκτείνωμεν αὐτὸν, καὶ ῥίψωμεν αὐτὸν εἰς ἕνα τῶν λάκκων· καὶ ἐροῦμεν, θηρίον πονηρὸν κατέφαγεν αὐτόν· καὶ ὀψόμεθα, τί ἔσται τὰ ἐνύπνια αὐτοῦ.
\vs{21}Ἀκούσας δὲ Ῥουβὴν, ἐξείλετο αὐτὸν ἐκ τῶν χειρῶν αὐτῶν· καὶ εἶπεν, οὐ πατάξωμεν αὐτὸν εἰς ψυχήν.
\vs{22}Εἶπε δὲ αὐτοῖς Ῥουβὴν, μὴ ἐκχέητε αἷμα· ἐμβάλλετε αὐτὸν εἰς ἕνα τῶν λάκκων τούτων τῶν ἐν τῇ ἐρήμῳ, χεῖρα δὲ μὴ ἐπενέγκητε αὐτῷ· ὅπως ἐξέληται αὐτὸν ἐκ τῶν χειρῶν αὐτῶν, καὶ ἀποδῷ αὐτὸν τῷ πατρὶ αὐτοῦ.
\vs{23}Ἐγένετο δὲ ἡνίκα ἦλθεν Ἰωσὴφ πρὸς τοὺς ἀδελφοὺς αὐτοῦ, ἐξέδυσαν Ἰωσὴφ τὸν χιτῶνα τὸν ποικίλον τὸν περὶ αὐτόν.
\vs{24}Καὶ λαβόντες αὐτὸν, ἔῤῥιψαν εἰς τὸν λάκκον· ὁ δὲ λάκκος κενὸς, ὕδωρ οὐκ εἶχε.
\vs{25}Ἐκάθισαν δὲ φαγεῖν ἄρτον· καὶ ἀναβλέψαντες τοῖς ὀφθαλμοῖς εἶδον, καὶ ἰδοὺ ὁδοιπόροι Ἰσμαηλῖται ἤρχοντο ἐκ Γαλαάδ· καὶ αἱ κάμηλοι αὐτῶν ἔγεμον θυμιαμάτων καὶ ῥητίνης καὶ στακτῆς. ἐπορεύοντο δὲ καταγαγεῖν εἰς Αἴγυπτον.

\vs{26}Εἶπε δὲ Ἰούδας πρὸς τοὺς ἀδελφοὺς αὐτοῦ, τί χρήσιμον, ἐὰν ἀποκτείνωμεν τὸν ἀδελφὸν ἡμῶν, καὶ κρύψωμεν τὸ αἷμα αὐτοῦ;
\vs{27}Δεῦτε ἀποδώμεθα αὐτὸν τοῖς Ἰσμαηλίταις τούτοις· αἱ δὲ χεῖρες ἡμῶν μὴ ἔστωσαν ἐπʼ αὐτὸν, ὅτι ἀδελφὸς ἡμῶν καὶ σὰρξ ἡμῶν ἐστίν. Ἤκουσαν δὲ οἱ ἀδελφοὶ αὐτοῦ.
\vs{28}Καὶ παρεπορεύοντο οἱ ἄνθρωποι οἱ Μαδιηναῖοι ἔμποροι, καὶ ἐξείλκυσαν καὶ ἀνεβίβασαν τὸν Ἰωσὴφ ἐκ τοῦ λάκκου· καὶ ἀπέδοντο τὸν Ἰωσὴφ τοῖς Ἰσμαηλίταις εἴκοσι χρυσῶν. Καὶ κατήγαγον τὸν Ἰωσὴφ εἰς Αἴγυπτον.
\vs{29}Ἀνέστρεψε δὲ Ῥουβὴν ἐπὶ τὸν λάκκον, καὶ οὐχ ὁρᾷ τὸν Ἰωσὴφ ἐν τῷ λάκκῳ· καὶ διέῤῥηξε τὰ ἱμάτια αὐτοῦ.
\vs{30}Καὶ ἐπέστρεψε πρὸς τοὺς ἀδελφοὺς αὐτοῦ, καὶ εἶπε, τὸ παιδάριον οὐκ ἔστιν· ἐγὼ δὲ ποῦ πορεύομαι ἔτι;
\vs{31}Λαβόντες δὲ τὸν χιτῶνα τοῦ Ἰωσὴφ, ἔσφαξαν ἔριφον αἰγῶν, καὶ ἐμόλυναν τὸν χιτῶνα τῷ αἵματι.
\vs{32}Καὶ ἀπέστειλαν τὸν χιτῶνα τὸν ποικίλον, καὶ εἰσήνεγκαν τῷ πατρὶ αὐτῶν· καὶ εἶπαν, τοῦτον εὕρομεν, ἐπίγνωθι εἰ χιτὼν τοῦ υἱοῦ σου ἐστὶν, ἢ οὔ.
\vs{33}Καὶ ἐπέγνω αὐτὸν, καὶ εἶπε, χιτὼν τοῦ υἱοῦ μου ἐστί· θηρίον πονηρὸν κατέφαγεν αὐτόν· θηρίον ἥρπασε τὸν Ἰωσήφ.
\vs{34}Διέῤῥηξε δὲ Ἰακὼβ τὰ ἱμάτια αὐτοῦ, καὶ ἐπέθετο σάκκον ἐπὶ τὴν ὀσφῦν αὐτοῦ, καὶ ἐπένθει τὸν υἱὸν αὐτοῦ ἡμέρας πολλάς.
\vs{35}Συνήχθησαν δὲ πάντες οἱ υἱοὶ αὐτοῦ καὶ αἱ θυγατέρες, καὶ ἦλθον παρακαλέσαι αὐτόν· καὶ οὐκ ἤθελε παρακαλεῖσθαι, λέγων, ὅτι καταβήσομαι πρὸς τὸν υἱόν μου πενθῶν εἰς ᾅδου· καὶ ἔκλαυσεν αὐτὸν ὁ πατὴρ αὐτοῦ.
\vs{36}Οἱ δὲ Μαδιηναῖοι ἀπέδοντο τὸν Ἰωσὴφ εἰς Αἴγυπτον τῷ Πετεφρῇ τῷ σπάδοντι Φαραὼ ἀρχιμαγείρῳ.

\ch{38}
Ἐγένετο δὲ ἐν τῷ καιρῷ ἐκείνῳ, κατέβη Ἰούδας ἀπὸ τῶν ἀδελφῶν αὐτοῦ, καὶ ἀφίκετο ἕως πρὸς ἄνθρωπον τινὰ Ὀδολλαμίτην, ᾧ ὄνομα Εἰράς.
\vs{2}Καὶ εἶδεν ἐκεῖ Ἰούδας θυγατέρα ἀνθρώπου Χαναναίου, ᾗ ὄνομα Σαυά· καὶ ἔλαβεν αὐτὴν, καὶ εἰσῆλθε πρὸς αὐτήν.
\vs{3}Καὶ συλλαβοῦσα ἔτεκεν υἱὸν, καὶ ἐκάλεσε τὸ ὄνομα αὐτοῦ, Ἤρ.
\vs{4}Καὶ συλλαβοῦσα ἔτεκεν υἱὸν ἔτι, καὶ ἐκάλεσε τὸ ὄνομα αὐτοῦ, Αὐνάν.
\vs{5}Καὶ προσθεῖσα ἔτεκεν υἱὸν, καὶ ἐκάλεσε τὸ ὄνομα αὐτοῦ, Σηλώμ· αὕτη δὲ ἦν ἐν Χασβὶ, ἡνίκα ἔτεκεν αὐτούς.
\vs{6}Καὶ ἔλαβεν Ἰούδας γυναῖκα Ἢρ τῷ πρωτοτόκῳ αὐτοῦ, ᾗ ὄνομα Θάμαρ.
\vs{7}Ἐγένετο δὲ Ἢρ πρωτότοκος Ἰούδα πονηρὸς ἔναντι Κυρίου· καὶ ἀπέκτεινεν αὐτὸν ὁ Θεός.
\vs{8}Εἶπε δὲ Ἰούδας τῷ Αὐνάν· εἴσελθε πρὸς τὴν γυναῖκα τοῦ ἀδελφοῦ σου, καὶ ἐπιγάμβρευσαι αὐτὴν, καὶ ἀνάστησον σπέρμα τῷ ἀδελφῷ σου.
\vs{9}Γνοὺς δὲ Αὐνὰν, ὅτι οὐκ αὐτῷ ἔσται τὸ σπέρμα, ἐγένετο ὅταν εἰσήρχετο πρὸς τὴν γυναῖκα τοῦ ἀδελφοῦ αὐτου, ἐξέχεεν ἐπὶ τὴν γῆν, τοῦ μὴ δοῦναι σπέρμα τῷ ἀδελφῷ αὐτοῦ.
\vs{10}Πονηρὸν δὲ ἐφάνη ἐναντίον τοῦ Θεοῦ, ὅτι ἐποίησε τοῦτο· καὶ ἐθανάτωσε καὶ τοῦτον.

\vs{11}Εἶπε δὲ Ἰούδας Θάμαρ τῇ νύμφῃ αὐτοῦ, κάθου χήρα ἐν τῷ οἴκῳ τοῦ πατρός σου, ἕως μέγας γένηται Σηλὼμ ὁ υἱός μου· εἶπε γάρ, μή ποτε ἀποθάνῃ καὶ οὗτος, ὥσπερ καὶ οἱ ἀδελφοὶ αὐτοῦ. Ἀπελθοῦσα δὲ Θάμαρ ἐκάθητο ἐν τῷ οἴκῳ τοῦ πατρὸς αὐτῆς.
\vs{12}Ἐπληθύνθησαν δὲ αἱ ἡμέραι, καὶ ἀπέθανε Σαυὰ ἡ γυνὴ Ἰούδα· καὶ παρακληθεὶς Ἰούδας ἀνέβη ἐπὶ τοὺς κείροντας τὰ πρόβατα αὐτοῦ, αὐτὸς καὶ Εἰρὰς ὁ ποιμὴν αὐτοῦ ὁ Ὀδολλαμίτης εἰς Θαμνά.
\vs{13}Καὶ ἀπηγγέλε Θάμαρ τῇ νύμφῃ αὐτοῦ, λέγοντες, ἰδοὺ ὁ πενθερός σου ἀναβαίνει εἰς Θαμνὰ, κεῖραι τὰ πρόβατα αὐτοῦ.
\vs{14}Καὶ περιελομένη τὰ ἱμάτια τῆς χηρεύσεως ἀφʼ ἑαυτῆς, περιέβαλε τὸ θέριστρον, καὶ ἐκαλλωπίσατο, καὶ ἐκάθισε πρὸς ταῖς πύλαις Αἰνὰν, ἥ ἐστιν ἐν παρόδῳ Θαμνά· ἴδε γὰρ ὅτι μέγας γέγονε Σηλὼμ, αὐτὸς δὲ οὐκ ἔδωκεν αὐτὴν αὐτῷ γυναῖκα.
\vs{15}Καὶ ἰδὼν αὐτὴν Ἰούδας ἔδοξεν αὐτὴν πόρνην εἶναι· κατεκαλύψατο γὰρ τὸ πρόσωπον αὐτῆς καὶ οὐκ ἐπέγνω αὐτήν.
\vs{16}Ἐξέκλινε δὲ πρὸς αὐτὴν τὴν ὁδόν· καὶ εἶπεν αὐτῇ, ἔασόν με εἰσελθεῖν πρός σε· οὐ γὰρ ἔγνω, ὅτι νύμφη αὐτοῦ ἐστίν· ἡ δὲ εἶπε, τί μοι δώσεις, ἐὰν εἰσέλθῃς πρός με;
\vs{17}Ὁ δὲ εἶπεν, ἐγώ σοι ἀποστελλῶ ἔριφον αἰγῶν ἐκ τῶν προβάτων μον· ἡ δὲ εἶπεν, ἐὰν δῷς μοι ἀῤῥαβῶνα, ἕως τοῦ ἀποστεῖλαί σε.
\vs{18}Ὁ δὲ εἶπε, τίνα τὸν ἀῤῥαβῶνά σοι δώσω; ἡ δὲ εἶπε, τὸν δακτύλιόν σου, καὶ τὸν ὁρμίσκον, καὶ τὴν ῥάβδον τὴν ἐν τῇ χειρίσου. Καὶ ἔδωκεν αὐτῇ, καὶ εἰσῆλθε πρὸς αὐτήν· καὶ ἐν γαστρὶ ἔλαβεν ἐξ αὐτοῦ.
\vs{19}Καὶ ἀναστᾶσα ἀπῆλθε, καὶ περιείλετο τὸ θέριστρον αὐτῆς ἀφʼ ἑαυτῆς, καὶ ἐνεδύσατο τὰ ἱμάτια τῆς χηρεύσεως αὐτῆς.
\vs{20}Ἀπέστειλε δὲ Ἰούδας τὸν ἔριφον ἐξ αἰγῶν ἐν χειρὶ τοῦ ποιμένος αὐτοῦ τοῦ Ὀδολλαμείτου, κομίσασθαι παρὰ τῆς γυναικὸς τὸν ἀῤῥαβῶνα· καὶ οὐχ εὗρεν αὐτήν.
\vs{21}Ἐπηρώτησε δὲ τοὺς ἄνδρας τοὺς ἐκ τοῦ τόπου, ποῦ ἐστιν ἡ πόρνη ἡ γενομένη ἐν Αἰνὰν ἐπὶ τῆς ὁδοῦ; καὶ εἶπαν, οὐκ ἦν ἐνταῦθα πόρνη.
\vs{22}Καὶ ἀπεστράφη πρὸς Ἰούδαν, καὶ εἶπεν, οὐχ εὗρον· καὶ οἱ ἄνθρωποι οἱ ἐκ τοῦ τόπου λέγουσι, μὴ εἶναι ὧδε πόρνην.
\vs{23}Εἶπε δὲ Ἰούδας, ἐχέτω αὐτά· ἀλλὰ μή ποτε καταγελασθῶμεν· ἐγὼ μὲν ἀπέσταλκα τὸν ἔριφον τοῦτον, σὺ δὲ οὐχ εὕρηκας.
\vs{24}Ἐγένετο δὲ μετὰ τρίμηνον ἀνηγγέλη τῷ Ἰούδα, λέγοντες, ἐκπεπόρνευκε Θάμαρ ἡ νύμφη σου, καὶ ἰδοὺ ἐν γαστρὶ ἔχει ἐκ πορνείας· Εἶπε δὲ Ἰούδας, ἐξαγάγετε αὐτὴν, καὶ κατακαυθήτω.
\vs{25}Αὐτὴ δὲ ἀγομένη ἀπέστειλε πρὸς τὸν πενθερὸν αὐτὴς, λέγουσα, ἐκ τοῦ ἀνθρώπου οὕτινος ταῦτά ἐστιν, ἐγὼ ἐν γαστρὶ ἔχω· καὶ εἶπεν, ἐπίγνωθι τίνος ὁ δακτύλιος, καὶ ὁ ὁρμίσκος καὶ ἡ ῥάβδος αὕτη.
\vs{26}Ἐπέγνω δὲ Ἰούδας, καὶ εἶπε, δεδικαίωται Θάμαρ ἢ ἐγώ· οὗ ἕνεκεν οὐκ ἔδωκα αὐτὴν Σηλὼμ τῷ υἱῷ μου· Καὶ οὐ προσέθετο ἔτι τοῦ γνῶναι αὐτήν.
\vs{27}Ἐγένετο δὲ ἡνίκα ἔτικτε, καὶ τῇδε ἦν δίδυμα ἐν τῇ γαστρὶ αὐτῆς.
\vs{28}Ἐγένετο δὲ ἐν τῷ τίκτειν αὐτὴν, ὁ εἷς προεξήνεγκεν τὴν χεῖρα· λαβοῦσα δὲ ἡ μαῖα, ἔδησεν ἐπὶ τὴν χεῖρα αὐτοῦ κόκκινον, λέγουσα, οὗτος ἐξελεύσεται πρότερος.
\vs{29}Ὡς δὲ ἐπισυνήγαγε τὴν χεῖρα, καὶ εὐθὺς ἐξῆλθεν ὁ ἀδελφὸς αὐτοῦ· ἡ δὲ εἶπε, τί διεκόπη διὰ σὲ φραγμός; καὶ ἐκάλεσε τὸ ὄνομα αὐτοῦ, Φαρές.
\vs{30}Καὶ μετὰ τοῦτο ἐξῆλθεν ὁ ἀδελφὸς αὐτοῦ, ἐφʼ ᾧ ἦν ἐπὶ τῇ χειρὶ αὐτοῦ τὸ κόκκινον· καὶ ἐκάλεσε τὸ ὄνομα αὐτοῦ, Ζαρά.

\ch{39}
Ἰωσὴφ δὲ κατήχθη εἰς Αἴγυπτον· καὶ ἐκτήσατο αὐτὸν Πετεφρὴς ὁ εὐνοῦχος Φαραὼ, ὁ ἀρχιμάγειρος, ἀνὴρ Αἰγύπτιος, ἐκ χειρῶν τῶν Ἰσμαηλιτῶν, οἳ κατήγαγον αὐτὸν ἐκεῖ.
\vs{2}Καὶ ἦν Κύριος μετὰ Ἰωσήφ· καὶ ἦν ἀνὴρ ἐπιτυγχάνων· καὶ ἐγένετο ἐν τῷ οἴκῳ παρὰ τῷ κυρίῳ αὐτοῦ τῷ Αἰγυπτίῳ.
\vs{3}Ἤδει δὲ ὁ κύριος αὐτοῦ, ὅτι ὁ Κύριος ἦν μετʼ αὐτοῦ, καὶ ὅσα ἐὰν ποιῇ, Κύριος εὐοδοῖ ἐν ταῖς χερσὶν αὐτοῦ.
\vs{4}Καὶ εὗρεν Ἰωσὴφ χάριν ἐναντίον τοῦ κυρίου αὐτοῦ, καὶ εὐηρέστησεν αὐτῷ. Καὶ κατέστησε αὐτὸν ἐπὶ τοῦ οἴκου αὐτοῦ· καὶ πάντα ὅσα ἦν αὐτῷ, ἔδωκε διὰ χειρὸς Ἰωσήφ.
\vs{5}Ἐγένετο δὲ μετὰ τὸ καταστῆναι αὐτὸν ἐπὶ τοῦ οἴκου αὐτοῦ, καὶ ἐπὶ πάντα ὅσα ἦν αὐτῷ, καὶ ηὐλόγησε Κύριος τὸν οἶκον τοῦ Αἰγυπτίου διὰ Ἰωσήφ· καὶ ἐγενήθη εὐλογία Κυρίου ἐν πᾶσι τοῖς ὑπάρχουσιν αὐτῷ ἐν τῷ οἴκῳ, καὶ ἐν τῷ ἀγρῷ αὐτοῦ.
\vs{6}Καὶ ἐπέτρεψε πάντα ὅσα ἦν αὐτῷ, εἰς χεῖρας Ἰωσήφ· καὶ οὐκ ᾔδει τῶν καθʼ αὑτὸν οὐδὲν, πλὴν τοῦ ἄρτου, οὗ ἤσθιεν αὐτός. Καὶ ἦν Ἰωσὴφ καλὸς τῷ εἴδει, καὶ ὡραῖος τῇ ὄψει σφόδρα.
\vs{7}Καὶ ἐγένετο μετὰ τὰ ῥήματα ταῦτα, καὶ ἐπέβαλεν ἡ γυνὴ τοῦ κυρίου αὐτοῦ τοὺς ὀφθαλμοὺς αὐτῆς ἐπὶ Ἰωσήφ· καὶ εἶπεν, κοιμήθητι μετʼ ἐμοῦ.
\vs{8}Ὁ δὲ οὐκ ἤθελεν· εἶπε δὲ τῇ γυναικὶ τοῦ κυρίου αὐτοῦ, εἰ ὁ κύριός μου οὐ γινώσκει διʼ ἐμὲ οὐδὲν ἐν τῷ οἴκῳ αὐτοῦ, καὶ πάντα ὅσα ἐστὶν αὐτῷ ἔδωκεν εἰς τὰς χεῖράς μου,
\vs{9}καὶ οὐχ ὑπερέχει ἐν τῇ οἰκίᾳ ταύτῆ οὐθὲν ἐμοῦ, οὐδὲ ὑπεξῄρηται ἀπʼ ἐμοῦ οὐδὲν, πλὴν σοῦ, διὰ τὸ σὲ γυναῖκα αὐτοῦ εἶναι, καὶ πῶς ποιήσω τὸ ῥῆμα τὸ πονηρὸν τοῦτο, καὶ ἁμαρτήσομαι ἐναντίον τοῦ Θεοῦ;
\vs{10}Ἡνίκα δὲ ἐλάλει τῷ Ἰωσὴφ ἡμέραν ἐξ ἡμέρας, καὶ οὐχ ὑπήκουεν αὐτῇ καθεύδειν μετʼ αὐτῆς, τοῦ συγγενέσθαι αὐτῇ.
\vs{11}Ἐγένετο δὲ τοιαύτη τις ἡμέρα, καὶ εἰσῆλθεν Ἰωσὴφ εἰς τὴν οἰκίαν ποιεῖν τὰ ἔργα αὐτοῦ, καὶ οὐθεὶς ἦν τῶν ἐν τῇ οἰκίᾳ ἔσω.
\vs{12}Καὶ ἐπεσπάσατο αὐτὸν τῶν ἱματίων, λέγουσα, κοιμήθητι μετʼ ἐμοῦ· καὶ καταλιπὼν τὰ ἱμάτια αὐτοῦ ἐν ταῖς χερσὶν αὐτῆς ἔφυγε, καὶ ἐξῆλθεν ἔξω.
\vs{13}Καὶ ἐγένετο ὡς εἶδεν ὅτι καταλιπὼν τὰ ἱμάτια αὐτοῦ ἐν ταῖς χερσὶν αὐτῆς ἔφυγε, καὶ ἐξῆλθεν ἔξω,
\vs{14}καὶ ἐκάλεσε τοὺς ὄντας ἐν τῇ οἰκίᾳ, καὶ εἶπεν αὐτοῖς, λέγουσα, ἴδετε, εἰσήγαγε ἡμῖν παῖδα Ἐβραῖον, ἐμπαίζειν ἡμῖν· εἰσῆλθε πρός με, λέγων, κοιμήθητι μετʼ ἐμοῦ· καὶ ἐβόησα φωνῇ μεγάλῃ.
\vs{15}Ἐν δὲ τῷ ἀκοῦσαι αὐτὸν, ὅτι ὕψωσα τὴν φωνήν μου καὶ ἐβόησα, καταλιπὼν τὰ ἱμάτια αὐτοῦ παρʼ ἐμοὶ ἔφυγε, καὶ ἐξῆλθεν ἔξω.
\vs{16}Καὶ καταλιμπάνει τὰ ἱμάτια παρʼ ἑαυτῇ, ἕως ἦλθεν ὁ κύριος εἰς τὸν οἶκον αὐτοῦ.
\vs{17}Καὶ ἐλάλησεν αὐτῷ κατὰ τὰ ῥήματα ταῦτα, λέγουσα, εἰσῆλθε πρός με ὁ παῖς ὁ Ἑβραῖος, ὃν εἰσήγαγες πρὸς ἡμᾶς, ἐμπαῖξαί μοι· καὶ εἶπέ μοι, κοιμηθήσομαι μετὰ σοῦ.
\vs{18}Ὡς δὲ ἤκοῦσεν, ὅτι ὕψωσα τὴν φωνήν μου καὶ ἐβόησα, καταλιπὼν τὰ ἱμάτια αὐτοῦ παρʼ ἐμοὶ ἔφυγε, καὶ ἐξῆλθεν ἔξω.
\vs{19}Ἐγένετο δὲ, ὡς ἤκουσεν ὁ κύριος τὰ ῥήματα τῆς γυναικὸς αὐτοῦ, ὅσα ἐλάλησε πρὸς αὐτὸν, λέγουσα, οὕτως ἐποίησέ μοι ὁ παῖς σου, καὶ ἐθυμώθη ὀργῇ.

\vs{20}Καὶ λαβὼν ὁ κύριος Ἰωσὴφ, ἐνέβαλε αὐτὸν εἰς τὸ ὀχύρωμα, εἰς τὸν τόπον ἐν ᾧ οἱ δεσμῶται τοῦ βασιλέως κατέχονται ἐκεῖ ἐν τῷ ὀχυρώματι.
\vs{21}Καὶ ἦν Κύριος μετὰ Ἰωσὴφ, καὶ κατέχεεν αὐτοῦ ἔλεος· καὶ ἔδωκεν αὐτῷ χάριν ἐναντίον τοῦ ἀρχιδεσμοφύλακος.
\vs{22}Καὶ ἔδωκεν ὁ ἀρχιδεσμοφύλαξ τὸ δεσμωτήριον διὰ χειρὸς Ἰωσὴφ, καὶ πάντας τοὺς ἀπηγμένους ὅσοι ἐν τῷ δεσμωτηρίῳ, καὶ πάντα ὅσα ποιοῦσιν ἐκεῖ, αὐτὸς ἦν ποιῶν.
\vs{23}Οὐκ ἦν ὁ ἀρχιδεσμοφύλαξ τοῦ δεσμωτηρίου γινώσκον διʼ αὐτὸν οὐθέν· πάντα γὰρ ἦν διὰ χειρὸς Ἰωσὴφ, διὰ τὸ τὸν Κύριον μετʼ αὐτοῦ εἶναι· καὶ ὅσα αὐτὸς ἐποίει, ὁ Κύριος εὐώδο ἐν ταῖς χερσὶν αὐτοῦ.

\ch{40}
Ἐγένετο δὲ μετὰ τὰ ῥήματα ταῦτα, ἥμαρτεν ὁ ἀρχιοινοχόος τοῦ βασιλέως Αἰγύπτου, καὶ ὁ ἀρχισιτοποιὸς, τῷ κυρίῳ αὐτῶν βασιλεῖ Αἰγύπτου.
\vs{2}Καὶ ὠργίσθη Φαραὼ ἐπὶ τοῖς δυσὶν εὐνούχοις αὐτοῦ, ἐπὶ τῷ ἀρχιοινοχόῳ, καὶ ἐπὶ τῷ ἀρχισιτοποιῷ·
\vs{3}Καὶ ἔθετο αὐτοὺς ἐν φυλακῇ εἰς τὸ δεσμωτήριον, εἰς τὸν τόπον, οὗ Ἰωσὴφ ἀπῆκτο ἐκεῖ.
\vs{4}Καὶ συνέστησεν ὁ ἀρχιδεσμώτης τῷ Ἰωσὴφ αὐτούς· καὶ παρέστη αὐτοῖς· ἦσαν δὲ ἡμέρας ἐν τῇ φυλακῇ.
\vs{5}Καὶ εἶδον ἀμφότεροι ἐνύπνιον ἐν μιᾷ νυκτί· ἡ δὲ ὅρασις τοῦ ἐνυπνίου τοῦ ἀρχιοινοχόου καὶ ἀρχισιτοποιοῦ, οἳ ἦσαν τῷ βασιλεῖ Αἰγύπτου, οἱ ὄντες ἐν τῷ δεσμωτηρίῳ, ἦν αὕτη.
\vs{6}Εἰσῆλθε πρὸς αὐτοὺς Ἰωσὴφ τὸ πρωῒ, καὶ εἶδεν αὐτοὺς, καὶ ἦσαν τεταραγμένοι.
\vs{7}Καὶ ἠρώτα τοὺς εὐνούχους Φαραὼ, οἳ ἦσαν μετʼ αὐτοῦ ἐν τῇ φυλακῇ παρὰ τῷ κυρίῳ αὐτοῦ, λέγων, τί ὅτι τὰ πρόσωπα ὑμῶν σκυθρωπὰ σήμερον;
\vs{8}Οἱ δὲ εἶπαν αὐτῷ, ἐνύπνιον εἴδομεν, καὶ ὁ συγκρίνων οὐκ ἔστιν αὐτό· εἶπε δὲ αὐτοῖς Ἰωσὴφ, οὐχὶ διὰ τοῦ Θεοῦ ἡ διασάφησις αὐτῶν ἐστι; διηγήσασθε οὖν μοὶ.
\vs{9}Καὶ διηγήσατο ὁ ἀρχιοινοχόος τὸ ἐνύπνιον αὐτοῦ τῷ Ἰωσήφ· καὶ εἶπεν, ἐν τῷ ὕπνῳ μου ἦν ἄμπελος ἐναντίον μου.
\vs{10}Ἐν δὲ τῇ ἀμπέλῳ τρεῖς πυθμένες, καὶ αὐτὴ θάλλουσα, ἀνενηνοχυῖα βλαστούς· πέπειροι οἱ βότρυες σταφυλῆς.
\vs{11}Καὶ τὸ ποτήριον Φαραὼ ἐν τῇ χειρί μου· καὶ ἔλαβον τὴν σταφυλὴν, καὶ ἐξέθλιψα αὐτὴν εἰς τὸ ποτήριον, καὶ ἔδωκα τὸ ποτήριον εἰς τὴν χεῖρα Φαραώ.
\vs{12}Καὶ εἶπεν αὐτῷ Ἰωσὴφ, τοῦτο ἡ σύγκρίσις αὐτοῦ· οἱ τρεῖς πυθμένες, τρεῖς ἡμέραι εἰσίν.
\vs{13}Ετι τρεῖς ἡμέραι, καὶ μνησθήσεται Φαραὼ τῆς ἀρχῆς σου, καὶ ἀποκαταστήσει σε ἐπὶ τὴν ἀρχιοινοχοΐαν σου, καὶ δώσεις τὸ ποτήριον Φαραὼ εἰς τὴν χεῖρα αὐτοῦ κατὰ τὴν ἀρχήν σου τὴν προτέραν, ὡς ἦσθα οἰνοχοῶν.
\vs{14}Ἀλλὰ μνήσθητί μου διὰ σεαυτοῦ, ὅταν εὖ γενηταί σοι· καὶ ποιήσεις ἐν ἐμοὶ ἔλεος· καὶ μνησθήσῃ περὶ ἐμοῦ πρὸς Φαραὼ, καὶ ἐξάξεις με ἐκ τοῦ ὀχυρώματος τούτου.
\vs{15}Ὅτι κλοπῇ ἐκλάπην ἐκ γῆς Ἑβραίων, καὶ ὧδε οὐκ ἐποίησα οὐδὲν, ἀλλʼ ἐνέβαλόν με εἰς τὸν λάκκον τοῦτον.
\vs{16}Καὶ εἶδεν ὁ ἀρχισιτοποιὸς ὅτι ὀρθῶς συνέκριεν· καὶ εἶπε τῷ Ἰωσὴφ, κᾀγὼ εἶδον ἐνύπνιον· καὶ ᾤμην τρία κανᾶ χονδριτῶν αἴρειν ἐπὶ τῆς κεφαλῆς μου·
\vs{17}Ἐν δὲ κανῷ τῷ ἐπάνω ἀπὸ πάντων τῶν γενῶν, ὧν Φαραὼ ἐσθίει, ἔργον σιτοποιοῦ, καὶ τὰ πετεινὰ τοῦ οὐρανου κατήσθιεν αὐτὰ ἀπὸ τοῦ κανοῦ τοῦ ἐπάνω τῆς κεφαλῆς μου.
\vs{18}Ἀποκριθεὶς δὲ Ἰωσὴφ εἶπεν αὐτῷ, αὕτη ἡ σύγκρισις αὐτοῦ· τὰ τρία κανᾶ, τρεῖς ἡμέραι εἰσίν·
\vs{19}Ἔτι τριῶν ἡμερῶν, καὶ ἀφελεῖ Φαραὼ τὴν κεφαλήν σου ἀπὸ σου· καὶ κρεμάσει σε ἐπὶ ξύλου, καὶ φάγεται τὰ ὄρνεα τοῦ οὐρανοῦ τὰς σάρκας σου ἀπὸ σοῦ.
\vs{20}Ἐγένετο δὲ ἐν τῇ ἡμέρᾳ τῇ τρίτῃ, ἡμέρα γενέσεως ἦν Φαραὼ, καὶ ἐποίει πότον πᾶσι τοῖς παισὶν αὐτοῦ· καὶ ἐμνήσθη τῆς ἀρχῆς τοῦ οἰνοχόου καὶ τῆς ἀρχῆς τοῦ σιτοποιοῦ ἐν μέσῳ τῶν παίδων αὐτοῦ.
\vs{21}Καὶ ἀποκατέστησε τὸν ἀρχιοινοχόον ἐπὶ τὴν ἀρχὴν αὐτοῦ· καὶ ἔδωκε τὸ ποτήριον εἰς τὴν χεῖρα Φαραώ.
\vs{22}Τὸν δὲ ἀρχισιτοποιὸν ἐκρέμασεν, καθὰ συνέκρινεν αὐτοῖς Ἰωσήφ.
\vs{23}Καὶ οὐκ ἐμνήσθη ὁ ἀρχιοινοχόος τοῦ Ἰωσὴφ, ἀλλαʼ ἐπελάθετο αὐτοῦ.

\ch{41}
Ἐγένετο δὲ μετὰ δύο ἔτη ἡμερῶν, Φαραὼ εἶδεν ἐνύπνιον· ᾤετο ἑστάναι ἐπὶ τοῦ ποταμοῦ.
\vs{2}Καὶ ἰδοὺ ὥσπερ ἐκ τοῦ ποταμοῦ ἀνέβαινον ἐπτὰ βόες, καλαὶ τῷ εἴδει, καὶ ἐκλεκταὶ ταῖς σαρξὶ, καὶ ἐβόσκοντο ἐν τῷ Ἄχει.
\vs{3}Ἄλλαι δὲ ἑπτὰ βόες ἀνέβαινον μετὰ ταύτας ἐκ τοῦ ποταμοῦ, αἰσχραὶ τῷ εἴδει, καὶ λεπταὶ ταῖς σαρξὶ, καὶ ἐνέμοντο παρὰ τὰς βόας ἐπὶ τὸ χεῖλος τοῦ ποταμοῦ.
\vs{4}Καὶ κατέφαγον αἱ ἑπτὰ βόες αἱ αἰσχραὶ καὶ λεπταὶ ταῖς σαρξὶ τὰς ἑπτὰ βόας τὰς καλὰς τῷ εἴδει καὶ τὰς ἐκλεκτὰς ταῖς σαρξί· ἠγέρθη δὲ Φαραώ.
\vs{5}Καὶ ἐνυπνιάσθη τὸ δεύτερον· καὶ ἰδοὺ ἑπτὰ στάχυες ἀνέβαινον ἐν τῷ πυθμένι ἑνὶ ἐκλεκτοὶ καὶ καλοί.
\vs{6}Καὶ ἰδοὺ ἑπτὰ στάχυες λεπτοὶ καὶ ἀνεμόφθοροι ἀνεφύοντο μετʼ αὐτούς.
\vs{7}Καὶ κατέπιον οἱ ἑπτὰ στάχυες οἱ λεπτοὶ καὶ ἀνεμόφθοροι τοὺς ἑπτὰ στάχυας τοὺς ἐκλεκτοὺς καὶ τοὺς πλήρεις· ἠγέρθη δὲ Φαραὼ, καὶ ἦν ἐνύπνιον.
\vs{8}Ἐγένετο δὲ πρωῒ, καὶ ἐταράχθη ἡ ψυχὴ αὐτοῦ, καὶ ἀποστείλας ἐκάλεσε πάντας τοὺς ἐξηγητὰς Αἰγύπτου, καὶ πάντας τοὺς σοφοὺς αὐτῆς· καὶ διηγήσατο αὐτοῖς Φαραὼ τὸ ἐνύπνιον αὐτοῦ, καὶ οὐκ ἦν ὁ ἀπαγγέλλων αὐτὸ τῷ Φαραώ.
\vs{9}Καὶ ἐλάλησεν ὁ ἀρχιοινοχόος πρὸς Φαραὼ, λέγων, τὴν ἁμαρτίαν μου ἀναμιμνήσκω σήμερον.
\vs{10}Φαραὼ ὠργίσθη τοῖς παισὶν αὐτοῦ, καὶ ἔθετο ἡμᾶς ἐν φυλακῇ, ἐν τῷ οἴκῳ τοῦ ἀρχιμαγείρου, ἐμέ τε καὶ τὸν ἀρχισιτοποιόν.
\vs{11}Καὶ εἴδομεν ἐνύπνιον ἀμφότεροι ἐν νυκτὶ μιᾷ ἐγὼ καὶ αὐτὸς, ἕκαστος κατὰ τὸ αὐτοῦ ἐνύπνιον εἴδομεν.
\vs{12}Ἦν δὲ ἐκεῖ μεθʼ ἡμῶν νεανίσκος παῖς Ἑβραῖος τοῦ ἀρχιμαγείρου, καὶ διηγησάμεθα αὐτῷ, καὶ συνέκρινεν ἡμῖν.
\vs{13}Ἐγενήθη δὲ, καθὼς συνέκρινεν ἡμῖν οὕτω καὶ συνέβη, ἐμέ τε ἀποκατασταθῆναι ἐπὶ τὴν ἀρχήν μου, ἐκεῖνον δὲ κρεμασθῆναι.
\vs{14}Ἀποστείλας δὲ Φαραὼ ἐκάλεσε τὸν Ἰωσήφ· καὶ ἐξήγαγον αὐτὸν ἀπὸ τοῦ ὀχυρώματος, καὶ ἐξύρησαν αὐτὸν, καὶ ἤλλαξαν τὴν στολὴν αὐτοῦ· καὶ ἦλθε πρὸς Φαραώ.
\vs{15}Εἶπε δὲ Φαραὼ πρὸς Ἰωσὴφ, ἐνύπνιον ἑώρακα, καὶ ὁ συγκρίνων οὐκ ἔστιν αὐτό· ἐγὼ δὲ ἀκήκοα περὶ σοῦ λεγόντων, ἀκούσαντά σε ἐνύπνια, συγκρῖναι αὐτά.
\vs{16}Ἀποκριθεὶς δὲ Ἰωσὴφ τῷ Φαραὼ εἶπεν, ἄνευ τοῦ Θεοῦ οὐκ ἀποκριθήσεται τὸ σωτήριον Φαραώ.
\vs{17}Ἐλάλησε δὲ Φαραὼ τῷ Ἰωσὴφ, λέγων, ἐν τῷ ὕπνῳ μου ᾤμην ἑστάναι παρὰ τὸ χεῖλος τοῦ ποταμοῦ.
\vs{18}Καὶ ὥσπερ ἐκ τοῦ ποταμοῦ ἀνέβαινον ἑπτὰ βόες καλαὶ τῷ εἴδει καὶ ἐκλεκταὶ ταῖς σαρξὶ, καὶ ἐνέμοντο ἐν τῷ Ἄχει.
\vs{19}Καὶ ἰδοὺ ἑπτὰ βόες ἕτεραι ἀνέβαινον ὀπίσω αὐτῶν ἐκ τοῦ ποταμοῦ, πονηραὶ καὶ αἰσχραὶ τῷ εἴδει, καὶ λεπταὶ ταῖς σαρξὶν, οἵας οὐκ εἶδον τοιαύτας ἐν ὅλῃ γῇ Αἰγύπτου αἰσχροτέρας.
\vs{20}Καὶ κατέφαγον αἱ ἑπτὰ βόες αἱ αἰσχραὶ καὶ λεπταὶ τὰς ἑπτὰ βόας τὰς πρώτας τὰς καλὰς καὶ τὰς ἐκλεκτάς.
\vs{21}Καὶ εἰσῆλθον εἰς τὰς κοιλίας αὐτῶν· καὶ οὑ διάδηλοι ἐγένοντο, ὅτι εἰσῆλθον εἰς τὰς κοιλίας αὐτῶν· καὶ αἱ ὄψεις αὐτῶν αἰσχραὶ, καθὰ καὶ τὴν ἀρχήν· ἐξεγερθεὶς δὲ ἐκοιμήθην.
\vs{22}Καὶ εἶδον πάλιν ἐν τῷ ὕπνῳ μου, καὶ ὥσπερ ἑπτὰ στάχυες ἀνέβαινον ἐν πυθμένι ἑνὶ πλήρεις καὶ καλοί·
\vs{23}Ἄλλοι δὲ ἑπτὰ στάχυες λεπτοὶ καὶ ἀνεμόφθοροι ἀνεφύοντο ἐχόμενοι αὐτῶν.
\vs{24}Καὶ κατέπιον οἱ ἑπτὰ στάχυες οἱ λεπτοὶ καὶ ἀνεμόφθοροι τοὺς ἑπτὰ στάχυας τοὺς καλοὺς καὶ τοὺς πλήρεις· εἶπα οὖν τοῖς ἐξῆγηταῖς, καὶ οὐκ ἦν ὁ ἀπαγγέλλων μοι αὐτό.

\vs{25}Καὶ εἶπεν Ἰωσὴφ τῷ Φαραὼ, τὸ ἐνύπνιον Φαραὼ ἕν ἐστιν· ὅσα ὁ Θεὸς ποιεῖ, ἔδειξε τῷ Φαραώ.
\vs{26}Αἱ ἑπτὰ βόες αἱ καλαὶ, ἑπτὰ ἔτη ἐστί· καὶ οἱ ἑπτὰ στάχυες οἱ καλοὶ, ἑπτὰ ἔτη ἐστί· τὸ ἐνύπνιον Φαραὼ ἕν ἐστι.
\vs{27}Καὶ αἱ ἑπτὰ βόες αἱ λεπταὶ, αἱ ἀναβαίνουσαι ὀπίσω αὐτῶν, ἑπτὰ ἔτη ἐστί· καὶ οἱ ἑπτὰ στάχυες οἱ λεπτοὶ καὶ ἀνεμόφθοροι, ἑπτὰ ἔτη ἐστί· ἔσονται ἑπτὰ ἔτη λιμοῦ.
\vs{28}Τὸ δὲ ῥῆμα ὃ εἴρηκα Φαραὼ, ὅσα ὁ Θεὸς ποιεῖ, ἔδειξε τῷ Φαραώ.
\vs{29}Ἰδοὺ ἑπτὰ ἔτη ἔρχεται εὐθηνία πολλὴ ἐν πάσῃ γῇ Αἰγύπτου.
\vs{30}Ἥξει δὲ ἑπτὰ ἔτη λιμοῦ μετὰ ταῦτα· καὶ ἐπιλήσονται τῆς πλησμονῆς τῆς ἐσομένης ἐν ὅλῃ Αἰγύπτῳ· καὶ ἀναλώσει ὁ λιμὸς τῆν γῆν.
\vs{31}Καὶ οὐκ ἐπιγνωσθήσεται ἡ εὐθηνία ἐπὶ τῆς γῆς ἀπὸ τοῦ λιμοῦ τοῦ ἐσομένου μετὰ ταῦτα· ἰσχυρὸς γὰρ ἔσται σφόδρα.
\vs{32}Περὶ δὲ τοῦ δευτερῶσαι τὸ ἐνύπνιον Φαραὼ δὶς, ὅτι ἀληθὲς ἔσται τὸ ῥῆμα τὸ παρὰ τοῦ Θεοῦ· καὶ ταχυνεῖ ὁ Θεὸς τοῦ ποιῆσαι αὐτό.
\vs{33}Νῦν οὖν σκέψαι ἄνθρωπον φρόνιμον καὶ συνετὸν, καὶ κατάστησον αὐτὸν ἐπὶ γῆς Αἰγύπτου.
\vs{34}Καὶ ποιησάτω Φαραὼ καὶ καταστησάτω τοπάρχας ἐπὶ τῆς γῆς· καὶ ἀποπεμπτωσάτωσαν πάντα τὰ γεννήματα τῆς γῆς Αἰγύπτου τῶν ἑπτὰ ἐτῶν τῆς εὐθηνίας,
\vs{35}καὶ συναγαγέτωσαν πάντα τὰ βρώματα τῶν ἑπτὰ ἐτῶν τῶν ἐρχομένων τῶν καλῶν τούτων· καὶ συναχθήτω ὁ σῖτος ὑπὸ χεῖρα Φαραώ· βρώματα ἐν ταῖς πόλεσι φυλαχθήτω.
\vs{36}Καὶ ἔσται τὰ βρώματα τὰ πεφυλαγμένα τῇ γῇ εἰς τὰ ἑπτὰ ἔτη τοῦ λιμοῦ, ἃ ἔσονται ἐν γῇ Αἰγύπτου, καὶ οὐκ ἐκτριβήσεται ἡ γῆ ἐν τῷ λιμῷ.
\vs{37}Ἤρεσε δὲ τὸ ῥῆμα ἐναντίον Φαραὼ, καὶ ἐναντίον πάντων τῶν παίδων αὐτοῦ.

\vs{38}Καὶ εἶπε Φαραὼ πᾶσι τοῖς παισὶν αὐτοῦ, μῆ εὑρήσομεν ἄνθρωπον τοιοῦτον, ὃς ἔχει πνεῦμα Θεοῦ ἐν αὐτῷ;
\vs{39}Εἶπε δὲ Φαραὼ τῷ Ἰωσὴφ, ἐπειδὴ ἔδειξεν ὁ Θεός σοι πάντα ταῦτα, οὐκ ἔστιν ἄνθρωπος φρονιμώτερος καὶ συνετώτερός σου.
\vs{40}Σὺ ἔσῃ ἐπὶ τῷ οἴκῳ μου, καὶ ἐπὶ τῷ στόματί σου ὑπακούσεται πᾶς ὁ λαός μου· πλὴν τὸν θρόνον ὑπερέξω σου ἐγώ.
\vs{41}Εἶπε δὲ Φαραὼ τῷ Ἰωσὴφ, ἰδοὺ καθίστημί σε σήμερον ἐπὶ πάσῃ γῇ Αἰγύπτου.
\vs{42}Καὶ περιελόμενος Φαραὼ τὸν δακτύλιον ἀπὸ τῆς χειρὸς αὐτοῦ, περίεθηκεν αὐτὸν ἐπὶ τὴν χεῖρα Ἰωσὴφ, καὶ ἐνέδυσεν αὐτὸν στολὴν βυσσίνην, καὶ περιέθηκε κλοιὸν χρυσοῦν περὶ τὸν τράχηλον αὐτοῦ.
\vs{43}Καὶ ἀνεβίβασεν αὐτὸν ἐπὶ τὸ ἅρμα τὸ δεύτερον τῶν αὐτοῦ· καὶ ἐκήρυξεν ἔμπροσθεν αὐτοῦ κήρυξ· καὶ κατέστησεν αὐτὸν ἐφʼ ὅλης γῆς Αἰγύπτου.
\vs{44}Εἶπε δὲ Φαραὼ τῷ Ἰωσὴφ, ἐγὼ Φαραώ· ἄνευ σοῦ οὐκ ἐξαρεῖ οὐθεὶς τὴν χεῖρα αὐτοῦ ἐπὶ πάσης γῆς Αἰγύπτου.
\vs{45}Καὶ ἐκάλεσε Φαραὼ τὸ ὄνομα Ἰωσὴφ, Ψονθομφανήχ· καὶ ἔδωκεν αὐτῷ τὴν Ἀσενὲθ θυγατέρα Πετεφρῆ ἱερέως Ἡλιουπόλεως αὐτῷ εἰς γυναῖκα.
\vs{46}Ἰωσὴφ δὲ ἦν ἐτῶν τριάκοντα, ὅτε ἔστη ἐναντίον Φαραὼ βασιλέως Αἰγύπτου· ἐξῆλθε δὲ Ἰωσὴφ ἀπὸ προσώπου Φαραὼ, καὶ διῆλθε πᾶσαν γῆν Αἰγύπτου.
\vs{47}Καὶ ἐποίησεν ἡ γῆ ἐν τοῖς ἑπτὰ ἔτεσι τῆς εὐθηνίας δράγματα.
\vs{48}Καὶ συνήγαγε πάντα τὰ βρώματα τῶν ἑπτὰ ἐτῶν, ἐν οἷς ἦν ἡ εὐθηνία ἐν τῇ γῇ Αἰγύπτου· καὶ ἔθηκε τὰ βρώματα ἐν ταῖς πόλεσι· βρώματα τῶν πεδίων τῆς πόλεως τῶν κύκλῳ αὐτῆς ἔθηκεν ἐν αὐτῇ.
\vs{49}Καὶ συνήγαγεν Ἰωσὴφ σῖτον ὡσεὶ τὴν ἄμμον τῆς θαλάσσης πολὺν σφόδρα, ἕως οὐκ ἠδύνατο ἀριθμηθῆναι, οὐ γὰρ ἦν ἀριθμός.

\vs{50}Τῷ δὲ Ἰωσὴφ ἐγένοντο υἱοὶ δύο πρὸ τοῦ ἐλθεῖν τὰ ἑπτὰ ἔτη τοῦ λιμοῦ, οὓς ἔτεκεν αὐτῷ Ἀσενὲθ ἡ θυγάτηρ Πετεφρῆ ἱερέως Ἡλιουπόλεως.
\vs{51}Ἐκάλεσε δὲ Ἰωσὴφ τὸ ὄνομα τοῦ πρωτοτόκου, Μανασσῆ· ὅτι ἐπιλαθέσθαι με ἐποίησεν ὁ Θεὸς πάντων τῶν πόνων μου, καὶ πάντων τῶν τοῦ πατρός μου·
\vs{52}Τὸ δὲ ὄνομα τοῦ δευτέρου ἐκάλεσεν, Ἐφραίμ· ὅτι ηὔξησέ με ὁ Θεὸς ἐν γῇ ταπεινώσεώς μου.
\vs{53}Παρῆλθον δὲ τὰ ἑπτὰ ἔτη τῆς εὐθηνίας, ἃ ἐγένοντο ἐν τῇ γῇ Αἰγύπτου.
\vs{54}Καὶ ἤρξατο τὰ ἑπτὰ ἔτη τοῦ λιμοῦ ἔρχεσθαι, καθὰ εἶπεν Ἰωσήφ· καὶ ἐγένετο λιμὸς ἐν πάσῃ τῇ γῇ· ἐν δὲ πάσῃ τῇ γῇ Αἰγύπτου ἦσαν ἄρτοι.
\vs{55}Καὶ ἐπείνασε πᾶσα ἡ γῆ Αἰγύπτου· ἔκραξε δὲ ὁ λαὸς πρὸς Φαραὼ περὶ ἄρτων· εἶπε δὲ Φαραὼ πᾶσι τοῖς Αἰγυπτίοις, πορεύεσθε πρὸς Ἰωσὴφ, καὶ ὃ ἐὰν εἴπῃ ὑμῖν, ποιήσατε.
\vs{56}Καὶ ὁ λιμὸς ἦν ἐπὶ προσώπου πάσης τῆς γῆς· ἀνέῳξε δὲ Ἰωσὴφ πάντας τοὺς σιτοβολῶνας, καὶ ἐπώλει πᾶσι τοῖς Αἰγυπτίοις.
\vs{57}Καὶ πᾶσαι αἱ χῶραι ἦλθον εἰς Αἴγυπτον, ἀγοράζειν πρὸς Ἰωσήφ· ἐπεκράτησε γὰρ ὁ λιμὸς ἐν πάσῃ τῇ γῇ·

\ch{42}
Ἰδὼν δὲ Ἰακὼβ ὅτι ἐστὶ πράσις ἐν Αἰγύπτῳ, εἶπε τοῖς υἱοῖς αὐτοῦ, ἱνατί ῥαθυμεῖτε;
\vs{2}Ἰδοὺ ἀκήκοα, ὅτι ἐστὶ σῖτος ἐν Αἰγύπτῳ· κατάβητε ἐκεὶ, καὶ πρίασθε ἡμῖν μικρὰ βρώματα, ἵνα ζήσωμεν καὶ μὴ ἀποθάνωμεν.

\vs{3}Κατέβησαν δὲ οἱ ἀδελφοὶ Ἰωσὴφ οἱ δέκα, πρίασθαι σῖτον ἐξ Αἰγύπτου·
\vs{4}Τὸν δὲ Βενιαμὶν, τὸν ἀδελφὸν Ἰωσὴφ, οὐκ ἀπέστειλε μετὰ τῶν ἀδελφῶν αὐτοῦ· εἶπε γὰρ, μή ποτε συμβῇ αὐτῷ μαλακία.
\vs{5}Ἦλθον δὲ οἱ υἱοὶ Ἰσραὴλ ἀγοράζειν μετὰ τῶν ἐρχομένων· ἦν γὰρ ὁ λιμὸς ἐν γῇ Χαναάν.
\vs{6}Ἰωσὴφ δὲ ἦν ὁ ἄρχων τῆς γῆς· οὗτος ἐπώλει παντὶ τῷ λαῷ τῆς γῆς· ἐλθόντες δὲ οἱ ἀδελφοὶ Ἰωσὴφ προσεκύνησαν αὐτῷ ἐπὶ πρόσωπον ἐπὶ τὴν γῆν.
\vs{7}Ἰδὼν δὲ Ἰωσὴφ τοὺς ἀδελφοὺς αὐτοῦ, ἐπέγνω· καὶ ἠλλοτριοῦτο ἀπʼ αὐτῶν, καὶ ἐλάλησεν αὐτοῖς σκληρά· καὶ εἶπεν αὐτοῖς, πόθεν ἥκατε; οἱ δὲ εἶπον, ἐκ γῆς Χαναὰν, ἀγοράσαι βρώματα.
\vs{8}Ἐπέγνω δὲ Ἰωσὴφ τοὺς ἀδελφοὺς αὐτοῦ· αὐτοὶ δὲ οὐκ ἐπέγνωσαν αὐτόν·
\vs{9}Καὶ ἐμνήσθη Ἰωσὴφ τῶν ἐνυπνίων αὐτοῦ, ὧν εἶδεν αὐτός· καὶ εἶπεν αὐτοῖς, κατάσκοποί ἐστε, κατανοῆσαι τὰ ἴχνη τῆς χώρας ἥκατε.
\vs{10}Οἱ δὲ εἶπαν, οὐχὶ, κύριε· οἱ παῖδές σου ἤλθομεν πρίασθαι βρώματα.
\vs{11}Πάντες ἐσμὲν υἱοὶ ἑνὸς ἀνθρώπου· εἰρηνικοί ἐσμεν, οὐκ εἰσιν οἱ παῖδές σου κατάσκοποι.
\vs{12}Εἶπε δὲ αὐτοῖς, οὐχί· ἀλλὰ τὰ ἴχνη τῆς γῆς ἤλθετε ἰδεῖν.
\vs{13}Οἱ δὲ εἶπαν, δώδεκά ἐσμεν οἱ παῖδές σου ἀδελφοὶ ἐν γῇ Χαναάν· καὶ ἰδοὺ ὁ νεώτερος μετὰ τοῦ πατρὸς ἡμῶν σήμερον· ὁ δὲ ἕτερος οὐχ ὑπάρχει.
\vs{14}Εἶπε δὲ αὐτοῖς Ἰωσὴφ, τοῦτό ἐστιν ὃ εἴρηκα ὑμῖν, λέγων, ὅτι κατάσκοποί ἐστε.
\vs{15}Ἐν τούτῳ φανεῖσθε· νὴ τὴν ὑγίειαν Φαραὼ, οὐ μὴ ἐξέλθητε ἐντεῦθεν, ἐὰν μὴ ὁ ἀδελφὸς ὑμῶν ὁ νεώτερος ἔλθῃ ὧδε.
\vs{16}Ἀποστείλατε ἐξ ὑμῶν ἕνα, καὶ λάβετε τὸν ἀδελφὸν ὑμῶν· ὑμεῖς δὲ ἀπάχθητε ἕως τοῦ φανερὰ γενέσθαι τὰ ῥήματα ὑμῶν, εἰ ἀληθεύετε ἢ οὔ· εἰ δὲ μὴ, νὴ τὴν ὑγίειαν Φαραὼ, ἦ μὴν κατάσκοποί ἐστε.
\vs{17}Καὶ ἔθετο αὐτοὺς ἐν φυλακῇ ἡμέρας τρεῖς.
\vs{18}Εἶπε δὲ αὐτοῖς τῇ ἡμέρᾳ τῇ τρίτῃ, τοῦτο ποιήσατε, καὶ ζήσεσθε· τὸν Θεὸν γὰρ ἐγὼ φοβοῦμαι.
\vs{19}Εἰ εἰρηνικοί ἐστε, ἀδελφὸς ὑμῶν κατασχεθήτω εἷς ἐν τῇ φυλακῇ· αὐτοὶ δὲ βαδίσατε, καὶ ἀπαγάγετε τὸν ἀγορασμὸν τῆς σιτοδοσίας ὑμῶν.
\vs{20}Καὶ τὸν ἀδελφὸν ὑμῶν τὸν νεώτερον ἀγάγετε πρός με, καὶ πιστευθήσονται τὰ ῥήματα ὑμῶν· εἰ δὲ μὴ, ἀποθανεῖσθε. Ἐποίησαν δὲ οὕτως.
\vs{21}Καὶ εἶπεν ἕκαστος πρὸς τὸν ἀδελφὸν αὐτοῦ, ναὶ, ἐν ἁμαρτίαις γάρ ἐσμεν περὶ τοῦ ἀδελφοῦ ἡμῶν, ὅτι ὑπερίδομεν τὴν θλίψιν τῆς ψυχῆς αὐτοῦ, ὅτε κατεδέετο ἡμῶν, καὶ οὐκ εἰσηκούσαμεν αὐτοῦ· καὶ ἕνεκεν τούτου ἐπῆλθεν ἐφʼ ἡμᾶς ἡ θλίψις αὕτη.
\vs{22}Ἀποκριθεὶς δὲ Ῥουβὴν εἶπεν αὐτοῖς, οὐκ ἐλάλησα ὑμῖν, λέγων, μὴ ἀδικήσητε τὸ παιδάριον, καὶ οὐκ εἰσηκούσατέ μου; καὶ ἰδοὺ τὸ αἷμα αὐτοῦ ἐκζητεῖται.
\vs{23}Αὐτοὶ δὲ οὐκ ᾔδεισαν, ὅτι ἀκούει Ἰωσήφ· ὁ γὰρ ἑρμηνευτὴς ἀνὰ μέσον αὐτῶν ἦν·
\vs{24}Ἀποστραφεὶς δὲ ἀπʼ αὐτῶν ἔκλαυσεν Ἰωσήφ· καὶ πάλιν προσῆλθε πρὸς αὐτοὺς, καὶ εἶπεν αὐτοῖς· καὶ ἔλαβε τὸν Συμεὼν ἀπʼ αὐτῶν, καὶ ἔδησεν αὐτὸν ἐναντίον αὐτῶν.

\vs{25}Ἐνετείλατο δὲ Ἰωσὴφ ἐμπλῆσαι τὰ ἀγγεῖα αὐτῶν σίτου, καὶ ἀποδοῦναι τὸ ἀργύριον αὐτῶν ἑκάστῳ εἰς τὸν σάκκον αὐτοῦ, καὶ δοῦναι αὐτοῖς ἐπισιτισμὸν εἰς τὴν ὁδόν· καὶ ἐγενήθη αὐτοῖς οὕτως.
\vs{26}Καὶ ἐπιθέντες τὸν σῖτον ἐπὶ τοῦς ὄνους αὐτῶν, ἀπῆλθον ἐκεῖθεν.
\vs{27}Λύσας δὲ εἷς τὸν μάρσιππον αὐτοῦ, δοῦναι χορτάσματα τοῖς ὄνοις αὐτοῦ, οὗ κατέλυσαν, καὶ εἶδε τὸν δεσμὸν τοῦ ἀργυρίου αὐτοῦ, καὶ ἦν ἐπάνω τοῦ στόματος τοῦ μαρσίππου.
\vs{28}Καὶ εἶπε τοῖς ἀδελφοῖς αὐτοῦ, ἀπεδόθη μοι τὸ ἀργύριον, καὶ ἰδοὺ τοῦτο ἐν τῷ μαρσίππῳ μου· καὶ ἐξέστη ἡ καρδία αὐτῶν, καὶ ἐταράχθησαν πρὸς ἀλλήλους, λέγοντες, τί τοῦτο ἐποίησεν ὁ Θεὸς ἡμῖν;
\vs{29}Ἦλθον δὲ πρὸς Ἰακὼβ τὸν πατέρα αὐτῶν εἰς γὴν Χαναὰν, καὶ ἀπήγγειλαν αὐτῷ πάντα τὰ συμβάντα αὐτοῖς, λέγοντες,
\vs{30}Λελάληκεν ὁ ἄνθρωπος ὁ κύριος τῆς γῆς πρὸς ἡμᾶς σκληρὰ, καὶ ἔθετο ἡμᾶς ἐν φυλακῇ, ὡς κατασκοπεύοντας τὴν γῆν.
\vs{31}Εἴπαμεν δὲ αὐτῷ, εἰρήνικοί ἐσμεν, οὐκ ἐσμὲν κατάσκοποι.
\vs{32}Δώδεκα ἀδελφοί ἐσμεν, υἱοὶ τοῦ πατρὸς ἡμῶν· ὁ εἷς οὐχ ὑπάρχει· ὁ δὲ μικρὸς μετὰ τοῦ πατρὸς ἡμῶν σήμερον ἐν γῇ Χαναάν.
\vs{33}Εἶπε δὲ ἡμῖν ὁ ἄνθρωπος ὁ κύριος τῆς γῆς, ἐν τούτῳ γνώσομαι, ὅτι εἰρηνικοί ἐστε· ἀδελφὸν ἕνα ἄφετε ὧδε μετʼ ἐμοῦ· τὸν δὲ ἀγορασμὸν τῆς σιτοδοσίας τοῦ οἴκου ὑμῶν λαβόντες ἀπέλθατε.
\vs{34}Καὶ ἀγάγετε πρός με τὸν ἀδελφὸν ὑμῶν τὸν νεώτερον· καὶ γνώσομαι ὅτι οὐ κατάσκοποί ἐστε, ἀλλʼ ὅτι εἰρηνικοί ἐστε· καὶ τὸν ἀδελφὸν ὑμῶν ἀποδώσω ὑμῖν, καὶ τῇ γῇ ἐμπορεύσεσθε.
\vs{35}Ἐγένετο δὲ ἐν τῷ κατακενοῦν αὐτοὺς τοὺς σάκκους αὐτῶν, καὶ ἦν ἑκάστου ὁ δεσμὸς τοῦ ἀργυρίου ἐν τῷ σάκκῳ αὐτῶν· καὶ εἶδον τοὺς δεσμοὺς τοῦ ἀργυρίου αὐτῶν αὐτοὶ, καὶ ὁ πατὴρ αὐτῶν, καὶ ἐφοβήθησαν.
\vs{36}Εἶπε δὲ αὐτοῖς Ἰακὼβ ὁ πατὴρ αὐτῶν, ἐμὲ ἠτεκνώσατε· Ἰωσὴφ οὐκ ἔστι, Συμεὼν οὐκ ἔστι, καὶ τὸν Βενιαμὶν λήψεσθε; ἐπʼ ἐμὲ ἐγένετο ταῦτα πάντα.
\vs{37}Εἶπε δὲ Ῥουβὴν τῷ πατρὶ αὐτῶν, λέγων, τοὺς δύο υἱούς μου ἀπόκτεινον, ἐὰν μὴ ἀγάγω αὐτὸν πρὸς σέ· δὸς αὐτὸν εἰς τὴν χεῖρά μου, κᾀγὼ ἀνάξω αὐτὸν πρὸς σέ.
\vs{38}Ὁ δὲ εἶπεν, οὐ καταβήσεται ὁ υἱός μου μεθʼ ὑμῶν, ὅτι ὁ ἀδελφὸς αὐτοῦ ἀπέθανε, καὶ αὐτὸς μόνος καταλέλειπται· καὶ συμβήσεται αὐτὸν μαλακισθῆναι ἐν τῇ ὁδῷ, ᾗ ἐὰν πορεύησθε, καὶ κατάξετέ μου τὸ γῆρας μετὰ λύπης εἰς ᾅδοῦ.

\ch{43}
Ὁ δὲ λιμὸς ἐνίσχυσεν ἐπὶ τῆς γῆς.
\vs{2}Ἐγένετο δὲ ἡνίκα συνετέλεσαν καταφαγεῖν τὸν σῖτον, ὃν ἤνεγκαν ἐξ Αἰγύπτου, καὶ εἶπεν αὐτοῖς ὁ πατὴρ αὐτῶν, πάλιν πορευθέντες πρίασθε ἡμῖν μικρὰ βρώματα.
\vs{3}Εἶπε δὲ αὐτῷ Ἰούδας, λέγων, διαμαρτυρίᾳ μεμαρτύρηται ἡμῖν ὁ ἄνθρωπος ὁ κύριος τῆς γῆς, λέγων, οὐκ ὄψεσθε τὸ πρόσωπόν μου, ἐὰν μὴ ὁ ἀδελφὸς ὑμῶν ὁ νεώτερος μεθʼ ὑμῶν ᾖ.
\vs{4}Εἰ μὲν οὖν ἀποστέλλῃς τὸν ἀδελφὸν ἡμῶν μεθʼ ἡμῶν, καταβησόμεθα, καὶ ἀγοράσομέν σοι βρώματα.
\vs{5}Εἰ δὲ μὴ ἀποστέλλῃς τὸν ἀδελφὸν ἡμῶν μεθʼ ἡμῶν, οὐ πορευσόμεθα· ὁ γὰρ ἄνθρωπος εἶπεν ἡμῖν, λέγων, οὐκ ὄψεσθέ μου τὸ πρόσωπον, ἐὰν μὴ ὁ ἀδελφὸς ὑμῶν ὁ νεώτερος μεθʼ ὑμῶν ᾖ.
\vs{6}Εἶπε δὲ Ἰσραὴλ, τί ἐκακοποιήσατέ με, ἀναγγείλαντες τῷ ἀνθρώπῳ ὅτι ἐστὶν ὑμῖν ἀδελφός;
\vs{7}Οἱ δὲ εἶπαν, ἐρωτῶν ἐπηρώτησεν ἡμᾶς ὁ ἄνθρωπος καὶ τὴν γενεὰν ἡμῶν, λέγων, εἰ ἔτι ὁ πατὴρ ὑμῶν ζῇ, καὶ εἰ ἔστιν ὑμῖν ἀδελφός· καὶ ἀπηγγείλαμεν αὐτῷ κατὰ τὴν ἐπερώτησιν ταύτην· μὴ ᾔδειμεν ὅτι ἐρεῖ ἡμῖν, ἀγάγετε τὸν ἀδελφὸν ὑμῶν;
\vs{8}Εἶπε δὲ Ἰούδας πρὸς Ἰσραὴλ τὸν πατέρα αὐτοῦ, ἀπόστειλον τὸ παιδάριον μετʼ ἐμοῦ· καὶ ἀναστάντες πορευσόμεθα, ἵνα ζῶμεν καὶ μὴ ἀποθάνωμεν καὶ ἡμεῖς, καὶ σὺ, καὶ ἡ ἀποσκευὴ ἡμῶν.
\vs{9}Ἐγὼ δὲ ἐκδέχομαι αὐτόν· ἐκ χειρός μου ζήτησον αὐτόν· ἐὰν μὴ ἀγάγω αὐτὸν πρός σε, καὶ στήσω αὐτὸν ἐναντίον σου, ἡμαρτηκὼς ἔσομαι εἰς σὲ πάσας τὰς ἡμέραν.
\vs{10}Εἰ μὴ γὰρ ἐβραδύναμεν, ἤδη ἂν ὑπεστρέψαμεν δίς.
\vs{11}Εἶπε δὲ αὐτοῖς Ἰσραὴλ ὁ πατὴρ αὐτῶν, εἰ οὕτως ἐστὶ, τοῦτο ποιήσατε· λάβετε ἀπὸ τῶν καρπῶν τῆς γῆς ἐν τοῖς ἀγγείοις ὑμῶν, καὶ καταγάγετε τῷ ἀνθρώπῳ δῶρα τῆς ῥητίνης, καὶ τοῦ μέλιτος, θυμίαμά τε καὶ στακτὴν, καὶ τερέινθον, καὶ κάρυα.
\vs{12}Καὶ τὸ ἀργύριον δισσὸν λάβετε ἐν ταῖς χερσὶν ὑμῶν· τὸ ἀργύριον τὸ ἀποστραφὲν ἐν τοῖς μαρσίπποις ὑμῶν ἀποστρέψατε μεθʼ ὑμῶν· μή ποτε ἀγνόημά ἐστι.
\vs{13}Καὶ τὸν ἀδελφὸν ὑμῶν λάβετε· καὶ ἀναστάντες κατάβητε πρὸς τὸν ἄνθρωπον.
\vs{14}Ὁ δὲ Θεός μου δώῃ ὑμῖν χάριν ἐναντίον τοῦ ἀνθρώπου καὶ ἀποστείλαι τὸν ἀδελφὸν ὑμῶν τὸν ἕνα, καὶ τὸν Βενιαμίν· ἐγὼ μὲν γὰρ καθάπερ ἠτέκνωμαι, ἠτέκνωμαι.

\vs{15}Λαβόντες δὲ οἱ ἄνδρες τὰ δῶρα ταῦτα καὶ τὸ ἀργύριον διπλοῦν, ἔλαβον ἐν ταῖς χερσὶν αὐτῶν καὶ τὸν Βενιαμείν· καὶ ἀναστάντες κατέβησαν εἰς Αἴγυπτον· καὶ ἔστησαν ἐναντίον Ἰωσήφ.
\vs{16}Εἶδε δὲ Ἰωσὴφ αὐτοὺς, καὶ τὸν Βενιαμὶν τὸν ἀδελφὸν αὐτοῦ τὸν ὁμομήτριον· καὶ εἶπε τῷ ἐπὶ τῆς οἰκίας αὐτοῦ, εἰσάγαγε τοὺς ἀνθρώπους εἰς τὴν οἰκίαν, καὶ σφάξον θύματα, καὶ ἑτοίμασον· μετʼ ἐμοῦ γὰρ φάγονται οἱ ἄνθρωποι ἄρτους τὴν μεσημβρίαν.
\vs{17}Ἐποίησε δὲ ὁ ἄνθρωπος καθὰ εἶπεν Ἰωσήφ· καὶ εἰσήγαγε τοὺς ἀνθρώπους εἰς τὸν οἶκον Ἰωσήφ.
\vs{18}Ἰδόντες δὲ οἱ ἄνδρες ὅτι εἰσήχθησαν εἰς τὸν οἶκον τοῦ Ἰωσήφ, εἶπαν, διὰ τὸ ἀργύριον τὸ ἀποστραφὲν ἐν τοῖς μαρσίπποις ἡμῶν τὴν ἀρχὴν, ἡμεῖς εἰσαγόμεθα, τοῦ συκοφαντῆσαι ἡμᾶς καὶ ἐπιθέσθαι ἡμῖν, τοῦ λαβεῖν ἡμᾶς εἰς παῖδας, καὶ τοὺς ὄνους ἡμῶν.
\vs{19}Προσελθόντες δὲ πρὸς τὸν ἄνθρωπον τὸν ἐπὶ τοῦ οἴκου τοῦ Ἰωσὴφ, ἐλάλησαν αὐτῷ ἐν τῷ πυλῶνι τοῦ οἴκου,
\vs{20}λέγοντες, δεόμεθα, κύριε· κατέβηεν τὴν ἀρχὴν πρίασθαι βρώματα.
\vs{21}Ἐγένετο δὲ ἡνίκα ἤλθομεν εἰς τὸ καταλῦσαι, καὶ ἠνοίξαμεν τοὺς μαρσίππους ἡμῶν, καὶ τόδε τὸ ἀργύριον ἑκάστου ἐν τῷ μαρσίππῳ αὐτοῦ· τὸ ἀργύριον ἡμῶν ἐν σταθμῷ ἀπεστρέψαμεν νῦν ἐν ταῖς χερσὶν ἡμῶν.
\vs{22}Καὶ ἀργύριον ἕτερον ἠνέγκαμεν μεθʼ ἑαυτῶν, ἀγοράσαι βρώματα· οὐκ οἴδαμεν τίς ἐνέβαλεν τὸ ἀργύριον εἰς τοὺς μαρσίππους ἡμῶν.
\vs{23}Εἶπε δὲ αὐτοῖς, ἵλεως ὑμῖν, μὴ φοβεῖσθε· ὁ Θεὸς ὑμῶν, καὶ ὁ Θεὸς τῶν πατέρων ὑμῶν, ἔδωκεν ὑμῖν θησαυροὺς ἐν τοῖς μαρσίπποις ὑμῶν· καὶ τὸ ἀργύριον ὑμῶν εὐδοκιμοῦν ἀπέχω· καὶ ἐξήγαγε πρὸς αὐτοὺς τὸν Συμεών.
\vs{24}Καὶ ἤνεγκεν ὕδωρ νίψαι τοὺς πόδας αὐτῶν· καὶ ἔδωκε χορτάσματα τοῖς ὄνοις αὐτῶν.
\vs{25}Ἡτοίμασαν δὲ τὰ δῶρα, ἕως τοῦ ἐλθεῖν τὸν Ἰωσὴφ μεσημβρίας· ἤκουσαν γὰρ ὅτι ἐκεῖ μέλλει ἀριστᾷν.
\vs{26}Εἰσῆλθε δὲ Ἰωσὴφ εἰς τὴν οἰκίαν, καὶ προσήνεγκαν αὐτῷ τὰ δῶρα, ἃ εἶχον ἐν ταῖς χερσὶν αὐτῶν, εἰς τὸν οἶκον· καὶ προσεκύνησαν αὐτῷ ἐπὶ πρόσωπον ἐπὶ τὴν γῆν.
\vs{27}Ἠρώτησε δὲ αὐτοὺς, πῶς ἔχετε; καὶ εἶπεν αὐτοῖς, εἰ ὑγιαίνει ὁ πατὴρ ὑμῶν ὁ πρεσβύτης, ὃν εἴπατε; ἔτι ζῇ;
\vs{28}Οἱ δὲ εἶπαν, ὑγιαίνει ὁ παῖς σου ὁ πατὴρ ἡμῶν, ἔτι ζῇ. Καὶ εἶπεν, εὐλογημένος ὁ ἄνθρωπος ἐκεῖνος τῷ Θεῷ· καὶ κύψαντες προσεκύνησαν αὐτῷ.
\vs{29}Ἀναβλέψας δὲ τοῖς ὀφθαλμοῖς αὐτοῦ Ἰωσὴφ, εἶδε Βενιαμὶν τὸν ἀδελφὸν αὐτοῦ τὸν ὁμομήτριον· καὶ εἶπεν, οὗτος ὁ ἀδελφὸς ὑμῶν ὁ νεώτερος, ὃν εἴπατε πρός με ἀγαγεῖν; καὶ εἶπεν, ὁ Θεὸς ἐλεήσαι σε, τέκνον.
\vs{30}Ἐταράχθη δὲ Ἰωσήφ· συνεστρέφετο γὰρ τὰ ἔγκατα αὐτοῦ ἐπὶ τῷ ἀδελφῷ αὐτοῦ, καὶ ἐζήτει κλαῦσαι· εἰσελθὼν δὲ εἰς τὸ ταμεῖον, ἔκλαυσεν ἐκεῖ.

\vs{31}Καὶ νιψάμενος τὸ πρόσωπον, ἐξελθὼν ἐνεκρατεύσατο· καὶ εἶπε, παράθετε ἄρτους.
\vs{32}Καὶ παρέθηκαν αὐτῷ μόνῳ, καὶ αὐτοῖς καθʼ ἑαυτούς, καὶ τοῖς Αἰγυπτίοις τοῖς συνδειπνοῦσι μετʼ αὐτοῦ καθʼ ἑαυτούς· οὐ γὰρ ἐδύναντο οἱ Αἰγύπτιοι συνεσθίειν μετὰ τῶν Ἐβραίων ἄρτους· βδέλυγμα γάρ ἐστι τοῖς Αἰγυπτίοις.
\vs{33}Ἐκάθισαν δὲ ἐναντίον αὐτοῦ, ὁ πρωτότοκος κατὰ τὰ πρεσβεῖα αὐτοῦ, καὶ ὁ νεώτερος κατὰ τὴν νεότητα αὐτοῦ· ἐξίσταντο δὲ οἱ ἄνθρωποι ἕκαστος πρὸς τὸν ἀδελφὸν αὐτοῦ.
\vs{34}Ἦραν δὲ μερίδα παρʼ αὐτοῦ πρὸς ἑαυτούς· ἐμεγαλύνθη δὲ ἡ μερὶς Βενιαμεὶν παρὰ τὰς μερίδας πάντων πενταπλασίως πρὸς τὰς ἐκείνων· ἔπιον δὲ καὶ ἐμεθύσθησαν μετʼ αὐτοῦ.

\ch{44}Καὶ ἐνετείλατο ὁ Ἰωσὴφ τῷ ὄντι ἐπὶ τῆς οἰκίας αὐτοῦ, λέγων, πλήσατε τοὺς μαρσίππους τῶν ἀνθρώπων βρωμάτων, ὅσα ἐὰν δύνωνται ἆραι· καὶ ἐμβάλετε ἑκάστου τὸ ἀργύριον ἐπὶ τοῦ στόματος τοῦ μαρσίππου.
\vs{2}Καὶ τὸ κόνδυ μου τὸ ἀργυροῦν ἐμβάλετε εἰς τὸν μάρσιππον τοῦ νεωτέρου, καὶ τὴν τιμὴν τοῦ σίτου αὐτοῦ· ἐγενήθη δὲ κατὰ τὸ ῥῆμα Ἰωσὴφ, καθὼς εἶπε.

\vs{3}Τὸ πρωῒ διέφαυσε· καὶ οἱ ἄνθρωποι ἀπεστάλησαν, αὐτοὶ καὶ οἱ ὄνοι αὐτῶν.
\vs{4}Ἐξελθόντων δὲ αὐτῶν τὴν πόλιν, οὐκ ἀπέσχον μακράν· καὶ Ἰωσὴφ εἶπε τῷ ἐπὶ τῆς οἰκίας αὐτοῦ, ἀναστὰς ἐπιδίωξον ὀπίσω τῶν ἀνθρώπων, καὶ καταλήμψῃ αὐτοὺς, καὶ ἐρεῖς αὐτοῖς τί ὅτι ἀνταπεδώκατε πονηρὰ ἀντὶ καλῶν;
\vs{5}Ἱνατί ἐκλέψατέ μου τὸ κόνδυ τὸ ἀργυροῦν; οὐ τοῦτό ἐστιν, ἐν ᾧ πίνει ὁ κύριός μου; αὐτὸς δὲ οἰωνισμῷ οἰωνίζεται ἐν αὐτῷ. πονηρὰ συντετελέκατε ἃ πεποιήκατε.
\vs{6}Εὑρὼν δὲ αὐτοὺς, εἶπεν αὐτοῖς κατὰ τὰ ῥήματα ταῦτα.
\vs{7}Οἱ δὲ εἶπαν αὐτῷ, ἱνατί λαλεῖ ὁ κύριος κατὰ τὰ ῥήματα ταῦτα; μὴ γένοιτο τοῖς παισίν σου ποιῆσαι κατὰ τὸ ῥῆμα τοῦτο.
\vs{8}Εἰ τὸ μὲν ἀργύριον, ὃ εὕρομεν ἐν τοῖς μαρσίπποις ἡμῶν, ἀπεστρέψαμεν πρὸς σὲ ἐκ γῆς Χαναὰν, πῶς ἂν κλέψαιμεν ἐκ τοῦ οἴκου τοῦ κυρίου σου ἀργύριον ἢ χρυσίον;
\vs{9}Παρʼ ᾧ ἂν εὕρῃς τὸ κόνδυ τῶν παιδων σου, ἀποθνησκέτω· καὶ ἡμεῖς δὲ ἐσόμεθα παῖδες τῷ κυρίῳ ἡμῶν.
\vs{10}Ὁ δὲ εἶπε, καὶ νῦν, ὡς λέγετε, οὕτως ἔσται· παρʼ ᾧ ἂν εὑρεθῇ τὸ κόνδυ, ἔσται μου παῖς ὑμεῖς δὲ ἔσεσθε καθαροί.
\vs{11}Καὶ ἔσπευσαν, καὶ καθεῖλαν ἕκαστος τὸν μάρσιππον αὐτοῦ ἐπὶ τὴν γῆν· καὶ ἤνοιξαν ἕκαστος τὸν μάρσιππον αὐτοῦ.
\vs{12}Ἠρεύνησε δὲ ἀπὸ τοῦ πρεσβυτέρου ἀρξάμενος, ἕως ἦλθεν ἐπὶ τὸν νεώτερον. καὶ εὗρε τὸ κόνδυ ἐν τῷ μαρσίππῳ τοῦ Βενιαμίν.
\vs{13}Καὶ διέῤῥηξαν τὰ ἱμάτια αὐτῶν, καὶ ἐπέθηκαν ἕκαστος τὸν μαρσίππον αὐτοῦ ἐπὶ τὸν ὄνον αὐτοῦ, καὶ ἐπέστρεψαν εἰς τὴν πόλιν.

\vs{14}Εἰσῆλθε δὲ Ἰούδας καὶ οἱ ἀδελφοὶ αὐτοῦ πρὸς Ἰωσὴφ ἔτι αὐτοῦ ὄντος ἐκεῖ, καὶ ἔπεσον ἐναντίον αὐτοῦ ἐπὶ τὴν γῆν.
\vs{15}Εἶπε δὲ αὐτοῖς Ἰωσὴφ, τί τὸ πρᾶγμα τοῦτο ἐποιήσατε; οὐκ οἴδατε ὅτι οἰωνισμῷ οἰωνιεῖται ὁ ἄνθρωπος, οἷος ἐγώ;
\vs{16}Εἶπε δὲ Ἰούδας, τί ἀντεροῦμεν τῷ κυρίῳ, ἢ τί λαλήσομεν, ἢ τί δικαιωθῶμεν; ὁ Θεὸς δὲ εὗρε τὴν ἀδικίαν τῶν παίδων σου· ἰδού ἐσμεν οἰκέται τῷ κυρίῳ ἡμῶν, καὶ ἡμεῖς, καὶ παρʼ ᾧ εὑρέθη τὸ κόνδυ.
\vs{17}Εἶπε δὲ Ἰωσὴφ, μή μοι γένοιτο ποιῆσαι τὸ ῥῆμα τοῦτο· ὁ ἄνθρωπος παρʼ ᾧ εὑρέθη τὸ κόνδυ, αὐτὸς ἔσται μου παῖς· ὑμεῖς δὲ ἀνάβητε μετὰ σωτηρίας πρὸς τὸν πατέρα ὑμῶν.
\vs{18}Ἐγγίσας δὲ αὐτῷ Ἰούδας εἶπε, δέομαι, κύριε· λαλησάτω ὁ παῖς σου ῥῆμα ἐναντίον σου, καὶ μὴ θυμωθῇς τῷ παιδί σου, ὅτι σὺ εἶ μετὰ Φαραώ.
\vs{19}Κύριε, σὺ ἠρώτησας τοὺς παῖδάς σου, λέγων, εἰ ἔχετε πατέρα ἢ ἀδελφόν.
\vs{20}Καὶ εἴπαμεν τῷ κυρίῳ, ἔστιν ἡμῖν πατὴρ πρεσβύτερος, καὶ παιδίον γήρως νεώτερον αὐτῷ, καὶ ὁ ἀδελφὸς αὐτοῦ ἀπέθανεν, αὐτὸς δὲ μόνος ὑπελείφθη τῇ μητρὶ αὐτοῦ, ὁ δὲ πατὴρ αὐτὸν ἠγάπησεν·
\vs{21}Εἶπας δὲ τοῖς παισί σου, καταγάγετε αὐτὸν πρὸς μέ, καὶ ἐπιμελοῦμαι αὐτοῦ.
\vs{22}Καὶ εἴπαμεν τῷ κυρίῳ, οὐ δυνήσεται τὸ παιδίον καταλιπεῖν τὸν πατέρα αὐτοῦ· ἐὰν δὲ καταλείπῃ τὸν πατέρα, ἀποθανεῖται.
\vs{23}Σὺ δὲ εἶπας τοῖς παισί σου, ἐὰν μὴ καταβῇ ὁ ἀδελφὸς ὑμῶν ὁ νεώτερος μεθʼ ὑμῶν, οὐ προσθήσεσθε ἰδεῖν τὸ πρόσωπόν μου.
\vs{24}Ἐγένετο δὲ ἡνίκα ἀνέβημεν πρὸς τὸν παῖδά σου πατέρα ἡμῶν, ἀπηγγείλαμεν αὐτῷ τὰ ῥήματα τοῦ κυρίου ἡμῶν.
\vs{25}Εἶπε δὲ ὁ πατὴρ ἡμῶν, βαδίσατε πάλιν καὶ ἀγοράσατε ἡμῖν μικρὰ βρώματα.
\vs{26}Ἡμεῖς δὲ εἴπομεν, οὐ δυνησόμεθα καταβῆναι· ἀλλʼ εἰ μὲν ὁ ἀδελφὸς ἡμῶν ὁ νεώτερος καταβαίνει μεθʼ ἡμῶν, καταβησόμεθα· οὐ γὰρ δυνησόμεθα ἰδεῖν τὸ πρόσωπον τοῦ ἀνθρώπου, τοῦ ἀδελφοῦ ἡμῶν τοῦ νεωτέρου μὴ ὄντος μεθʼ ἡμῶν.
\vs{27}Εἶπε δὲ ὁ παῖς σου πατὴρ ἡμῶν πρὸς ἡμᾶς, ὑμεῖς γινώσκετε ὅτι δύο ἔτεκέ μοι ἡ γυνὴ,
\vs{28}καὶ ἐξῆλθεν ὁ εἷς ἀπʼ ἐμοῦ· καὶ εἴπατε ὅτι θηριόβρωτος γέγονεν, καὶ οὐκ ἴδον αὐτὸν ἄχρι νῦν.
\vs{29}Ἐὰν οὖν λάβητε καὶ τοῦτον ἐκ τοῦ προσώπου μου, καὶ συμβῇ αὐτῷ μαλακία ἐν τῇ ὁδῷ, καὶ κατάξετέ μου τὸ γῆρας μετὰ λύπης εἰς ᾅδου.
\vs{30}Νῦν οὖν ἐὰν εἰσπορεύωμαι πρὸς τὸν παῖδά σου, πατέρα δὲ ἡμῶν, καὶ τὸ παιδίον μὴ ᾖ μεθʼ ἡμῶν, ἡ δὲ ψυχὴ αὐτοῦ ἐκκρέμαται ἐκ τῆς τούτου ψυχῆς,
\vs{31}καὶ ἔσται ἐν τῷ ἰδεῖν αὐτὸν μὴ ὂν τὸ παιδίον μεθʼ ἡμῶν, τελευτήσει, καὶ κατάξουσιν οἱ παῖδές σου τὸ γῆρας τοῦ παιδός σου, πατρὸς δὲ ἡμῶν, μετὰ λύπης εἰς ᾅδου.
\vs{32}Ὁ γὰρ παῖς σου παρὰ τοῦ πατρὸς ἐκδέδεκται τὸ παιδίον, λέγων, ἐὰν μὴ ἀγάγω αὐτὸν πρὸς σὲ, καὶ στήσω αὐτὸν ἐνώπιόν σου, ἡμαρτηκὼς ἔσομαι εἰς τὸν πατέρα πάσας τὰς ἡμέρας.
\vs{33}Νῦν οὖν παραμενῶ σοι παῖς ἀντὶ τοῦ παιδίου, οἰκέτης τοῦ κυρίου· τὸ δὲ παιδίον ἀναβήτω μετὰ τῶν ἀδελφῶν αὐτοῦ.
\vs{34}Πῶς γὰρ ἀναβήσομαι πρὸς τὸν πατέρα, τοῦ παιδίου μὴ ὄντος μεθʼ ἡμῶν; ἵνα μὴ ἴδω τὰ κακὰ, ἃ εὑρήσει τὸν πατέρα μου.

\ch{45}
Καὶ οὐκ ἠδύνατο Ἰωσὴφ ἀνέχεσθαι πάντων τῶν παρεστηκότων αὐτῷ, ἀλλʼ εἶπεν, ἐξαποστείλατε πάντας ἀπʼ ἐμοῦ· καὶ οὐ παρειστήκει οὐδεὶς τῷ Ἰωσὴφ, ἡνίκα ἀνεγνωρίζετο τοῖς ἀδελφοῖς αὐτοῦ.
\vs{2}Καὶ ἀφῆκε φωνὴν μετὰ κλαυθμοῦ· ἤκουσαν δὲ πάντες οἱ Αἰγύπτιοι, καὶ ἀκουστὸν ἐγένετο εἰς τὸν οἶκον Φαραώ.
\vs{3}Εἶπε δὲ Ἰωσὴφ πρὸς τοὺς ἀδελφοὺς αὐτοῦ, ἐγώ εἰμι Ἰωσήφ· ἔτι ὁ πατήρ μου ζῇ; καὶ οὐκ ἠδύναντο οἱ ἀδελφοὶ ἀποκριθῆναι αὐτῷ· ἐταράχθησαν γάρ.
\vs{4}Εἶπε δὲ Ἰωσὴφ πρὸς τοὺς ἀδελφοὺς αὐτοῦ, ἐγγίσατε πρὸς μέ· καὶ ἤγγισαν· καὶ εἶπεν, ἐγώ εἰμι Ἰωσὴφ ὁ ἀδελφὸς ὑμῶν, ὃν ἀπέδοσθε εἰς Αἴγυπτον.
\vs{5}Νῦν οὖν μὴ λυπεῖσθε, μηδὲ σκληρὸν ὑμῖν φανήτω, ὅτι ἀπέδοσθέ με ὧδε· εἰς γὰρ ζωὴν ἀπέστειλέ με ὁ Θεὸς ἔμπροσθεν ὑμῶν.
\vs{6}Τοῦτο γὰρ δεύτερον ἔτος λιμὸς ἐπὶ τῆς γῆς, καὶ ἔτι λοιπὰ πέντε ἔτη, ἐν οἷς οὐκ ἔσται ἀροτρίασις, οὐδὲ ἀμητός·
\vs{7}Ἀπέστειλε γάρ με ὁ Θεὸς ἔμπροσθεν ὑμῶν, ὑπολείπεσθαι ὑμῶν κατάλειμμα ἐπὶ τῆς γῆς, καὶ ἐκθρέψαι ὑμῶν κατάλειψιν μεγάλην.
\vs{8}Νῦν οὖν οὐχ ὑμεῖς με ἀπεστάλκατε ὧδε, ἀλλὰ ὁ Θεός· καὶ ἐποίησέ με ὡς πατέρα Φαραὼ, καὶ κύριον παντὸς τοῦ οἴκου αὐτοῦ, καὶ ἄρχοντα πάσης γῆς Αἰγύπτου.
\vs{9}Σπεύσαντες οὖν ἀνάβητε πρὸς τὸν πατέρα μου, καὶ εἴπατε αὐτῷ, τάδε λέγει ὁ υἱός σου Ἰωσήφ· ἐποίησέ με ὁ Θεὸς κύριον πάσης γῆς Αἰγύπτου· κατάβηθι οὖν πρός με, καὶ μὴ μείνῃς·
\vs{10}Καὶ κατοικήσεις ἐν γῇ Γεσὲμ Ἀραβίας· καὶ ἔσῃ ἐγγύς μου σὺ, καὶ οἱ υἱοί σου, καὶ οἱ υἱοὶ τῶν υἱῶν σου, τὰ πρόβατά σου, καὶ αἱ βόες σου, καὶ ὅσα σοι ἐστί.
\vs{11}Καὶ ἐκθρέψω σε ἐκεῖ· ἔτι γὰρ πέντε ἔτη λιμός· ἵνα μὴ ἐκτριβῇς σὺ, καὶ οἱ υἱοί σου, καὶ πάντα τὰ ὑπάρχοντά σου.
\vs{12}Ἰδοὺ οἱ ὀφθαλμοὶ ὑμῶν βλέπουσι, καὶ οἱ ὀφθαλμοὶ Βενιαμεὶν τοῦ ἀδελφοῦ μου, ὅτι τὸ στόμα μου τὸ λαλοῦν πρὸς ὑμᾶς.
\vs{13}Ἀπαγγείλατε οὖν τῷ πατρί μου πᾶσαν τὴν δόξαν μου τὴν ἐν Αἰγύπτῳ, καὶ ὅσα ἴδετε· καὶ ταχύναντες καταγάγετε τὸν πατέρα μου ὧδε.
\vs{14}Καὶ ἐπιπεσὼν ἐπὶ τὸν τράχηλον Βενιαμὶν τοῦ ἀδελφοῦ αὐτοῦ, ἔκλαυσεν ἐπʼ αὐτῷ· καὶ Βενιαμὶν ἔκλαυσεν ἐπὶ τῷ τραχήλῳ αὐτοῦ.
\vs{15}Καὶ καταφιλήσας πάντας τοὺς ἀδελφοὺς αὐτοῦ, ἔκλαυσεν ἐπʼ αὐτοῖς· καὶ μετὰ ταῦτα ἐλάλησαν οἱ ἀδελφοὶ αὐτοῦ πρὸς αὐτόν.

\vs{16}Καὶ διεβοήθη ἡ φωνὴ εἰς τὸν οἶκον Φαραὼ, λέγοντες, ἥκασιν οἱ ἀδελφοὶ Ἰωσήφ· ἐχάρη δὲ Φαραὼ καὶ ἡ θεραπεία αὐτοῦ.
\vs{17}Εἶπε δὲ Φαραὼ πρὸς Ἰωσὴφ, εἰπον τοῖς ἀδελφοῖς σου, τοῦτο ποιήσατε, γεμίσατε τὰ φορεῖα ὑμῶν, καὶ ἀπέλθετε εἰς γῆν Χαναάν.
\vs{18}Καὶ ἀναλαβόντες τὸν πατέρα ὑμῶν, καὶ τὰ ὑπάρχοντα ὑμῶν, ἥκετε πρός με· καὶ δώσω ὑμῖν πάντων τῶν ἀγαθῶν Αἰγύπτου, καὶ φάγεσθε τὸν μυελὸν τῆς γῆς.
\vs{19}Σὺ δὲ ἔντειλαι ταῦτα· λαβεῖν αὐτοῖς ἁμάξας ἐκ γῆς Αἰγύπτου τοῖς παιδίοις ὑμῶν, καὶ ταῖς γυναιξὶν ὑμῶν· καὶ ἀναλαβόντες τὸν πατέρα ὑμῶν παραγίνεσθε.
\vs{20}Καὶ μὴ φείσησθε τοῖς ὀφθαλμοῖς τῶν σκευῶν ὑμῶν· τὰ γὰρ πάντα ἀγαθὰ Αἰγύπτου ὑμῖν ἔσται.
\vs{21}Ἐποίησαν δὲ οὕτως οἱ υἱοὶ Ἰσραήλ· ἔδωκε δὲ Ἰωσὴφ αὐτοῖς ἁμάξας κατὰ τὰ εἰρημένα ὑπὸ Φαραὼ τοῦ βασιλέως· καὶ ἔδωκεν αὐτοῖς ἐπισιτισμὸν εἰς τὴν ὁδόν·
\vs{22}Καὶ πᾶσιν ἔδωκε δισσὰς στολάς· τῷ δὲ Βενιαμὶν ἔδωκε τριακοσίους χρυσοὺς, καὶ πέντε ἐξαλλασσούσας στολάς.
\vs{23}Καὶ τῷ πατρὶ αὐτοῦ ἀπέστειλε κατὰ τὰ αὐτά· καὶ δέκα ὄνους, αἴροντας ἀπὸ πάντων τῶν ἀγαθῶν Αἰγύπτου, καὶ δέκα ἡμιόνους, αἰρούσας ἄρτους τῷ πατρὶ αὐτοῦ εἰς ὁδόν.
\vs{24}Ἐξαπέστειλε δὲ τοὺς ἀδελφοὺς αὐτοῦ, καὶ ἐπορεύθησαν· καὶ εἶπεν αὐτοῖς, μὴ ὀργίζεσθε ἐν τῇ ὁδῷ.
\vs{25}Καὶ ἀνέβησαν ἐξ Αἰγυπτου, καὶ ἦλθον εἰς γῆν Χαναὰν πρὸς Ἰακὼβ τὸν πατέρα αὐτῶν.
\vs{26}Καὶ ἀνήγγειλαν αὐτῷ λέγοντες, ὅτι ὁ υἱός σου Ἰωσὴφ ζῇ, καὶ αὐτὸς ἄρχει πάσης γῆς Αἰγύπτου· καὶ ἐξέστη τῇ διανοίᾳ Ἰακὼβ, οὐ γὰρ ἐπίστευσεν αὐτοῖς.
\vs{27}Ἐλάλησαν δὲ αὐτῷ πάντα τὰ ῥηθέντα ὑπὸ Ἰωσὴφ, ὅσα εἶπεν αὐτοῖς. Ἰδὼν δὲ τὰς ἁμάξας, ἃς ἀπέστειλεν Ἰωσὴφ ὥστε ἀναλαβεῖν αὐτὸν, ἀνεζωπύρησε τὸ πνεῦμα Ἰακὼβ τοῦ πατρὸς αὐτῶν.
\vs{28}Εἶπε δὲ Ἰσραὴλ, μέγα μοι ἐστὶν, εἰ ἔτι Ἰωσὴφ ὁ υἱός μου ζῇ· πορευθεὶς ὄψομαι αὐτὸν πρὸ τοῦ ἀποθανεῖν με.

\ch{46}
Ἀπᾴρας δὲ Ἰσραὴλ, αὐτὸς καὶ πάντα τὰ αὐτοῦ, ἦλθεν ἐπὶ τὸ φρέαρ τοῦ ὅρκου· καὶ ἔθυσε θυσίαν τῷ Θεῷ τοῦ πατρὸς αὐτοῦ Ἰσαάκ.
\vs{2}Εἶπε δὲ ὁ Θεὸς τῷ Ἰσραὴλ ἐν ὁράματι τῆς νυκτὸς, εἰπὼν, Ἰακὼβ, Ἰακώβ· ὁ δὲ εἶπε, τί ἐστιν;
\vs{3}Ὁ δὲ λέγει αὐτῷ, ἐγώ εἰμι ὁ Θεὸς τῶν πατέρων σου· μὴ φοβοῦ καταβῆναι εἰς Αἴγυπτον· εἰς γὰρ ἔθνος μέγα ποιήσω σε ἐκεῖ.
\vs{4}Καὶ ἐγὼ καταβήσομαι μετὰ σοῦ εἰς Αἴγυπτον, καὶ ἐγὼ ἀναβιβάσω σε εἰς τέλος· καὶ Ἰωσὴφ ἐπιβαλεῖ τὰς χεῖρας ἐπὶ τοὺς ὀφθαλμούς σου.
\vs{5}Ἀνέστη δὲ Ἰακὼβ ἀπὸ τοῦ φρέατος τοῦ ὅρκου· καὶ ἀνέλαβον οἱ υἱοὶ Ἰσραὴλ τὸν πατέρα αὐτῶν, καὶ τὴν ἀποσκευὴν, καὶ τὰς γυναῖκας αὐτῶν, ἐπὶ τὰς ἁμάξας, ἃς ἀπέστειλεν Ἰωσὴφ ἆραι αὐτόν.
\vs{6}Καὶ ἀναλαβόντες τὰ ὑπάρχοντα αὐτῶν, καὶ πᾶσαν τὴν κτῆσιν, ἣν ἐκτήσαντο ἐκ γῇ Χαναὰν, εἰσῆλθον εἰς Αἴγυπτον, Ἰακὼβ, καὶ πᾶν τὸ σπέρμα αὐτοῦ μετʼ αὐτοῦ.
\vs{7}Υἱοὶ, καὶ υἱοὶ τῶν υἱῶν αὐτοῦ μετʼ αὐτοῦ· θυγατέρες, καὶ θυγατέρες τῶν θυγατέρων αὐτοῦ· καὶ πᾶν τὸ σπέρμα αὐτοῦ ἤγαγεν εἰς Αἴγυπτον·
\vs{8}Ταῦτα δὲ τὰ ὀνόματα τῶν υἱῶν Ἰσραὴλ τῶν εἰσελθόντων εἰς Αἴγυπτον ἅμα Ἰακὼβ τῷ πατρὶ αὐτῶν. Ἰακὼβ καὶ οἱ υἱοὶ αὐτοῦ· πρωτότοκος Ἰακὼβ, Ῥουβήν.
\vs{9}γἱοὶ δὲ Ῥουβὴν, Ἑνὼχ, καὶ Φαλλὸς, Ἀσρὼν, καὶ Χαρμί.
\vs{10}Υἱοὶ δὲ Συμεὼν, Ἰεμουὴλ, καὶ Ἰαμεὶν, καὶ Ἀὼδ, καὶ Ἀχεὶν, καὶ Σαὰρ, καὶ Σαοὺλ υἱὸς τῆς Χανανίτιδος.
\vs{11}Υἱοὶ δὲ Λευὶ, Γηρσὼν, Κὰθ, καὶ Μεραρί.
\vs{12}Υἱοὶ δὲ Ἰούδα, Ἢρ, καὶ Αὐνὰν, καὶ Σηλὼμ, καὶ Φαρὲς, καὶ Ζαρά· ἀπέθανε δὲ Ἢρ καὶ Αὐνὰν ἐν γῇ Χαναάν· ἐγένοντο δὲ υἱοὶ Φαρὲς, Ἑσρὼν, καὶ Ἰεμουήλ.
\vs{13}Υἱοὶ δὲ Ἰσσάχαρ, Θωλὰ, καὶ Φουὰ, καὶ Ἀσοὺμ, καὶ Σαμβράν.
\vs{14}Υἱοὶ δὲ Ζαβουλὼν, Σερὲδ, καὶ Ἀλλὼν, καὶ Ἀχοήλ.
\vs{15}Οὗτοι υἱοὶ Λείας, οὓς ἔτεκε τῷ Ἰακὼβ ἐν Μεσοποταμίᾳ τῆς Συρίας, καὶ Δείναν τὴν θυγατέρα αὐτοῦ· πᾶσαι αἱ ψυχαί, υἱοὶ καὶ θυγατέρες, τριάκοντα τρεῖς.
\vs{16}Υἱοὶ δὲ Γάδ, Σαφὼν, καὶ Ἀγγὶς, καὶ Σαννὶς, καὶ Θασοβὰν, καὶ Ἀηδεὶς, καὶ Ἀροηδεὶς, καὶ Ἀρεηλείς.
\vs{17}Υἱοὶ δὲ Ἀσὴρ, Ἰεμνα, Ἰεσσουὰ, καὶ Ἰεοὺλ, καὶ βαριὰ, καὶ Σάρα ἀδελφὴ αὐτῶν. Υἱοὶ δὲ βαριὰ, Χοβὸρ, καὶ Μελχιΐλ.
\vs{18}Οὗτοι υἱοὶ Ζελφᾶς, ἣν ἔδωκε Λάβαν Λείᾳ τῇ θυγατρὶ αὐτοῦ, ἣ ἔτεκε τούτους τῷ Ἰακὼβ, δεκαὲξ ψυχάς.
\vs{19}Υἱοὶ δὲ Ῥαχὴλ γυναικὸς Ἰακὼβ, Ἰωσὴφ, καὶ Βενιαμείν.
\vs{20}Ἐγένοντο δὲ υἱοὶ Ἰωσὴφ ἐν γῇ Αἰγύπτου, οὓς ἔτεκεν αὐτῷ Ἀσενὲθ θυγάτηρ Πετεφρῆ ἱερέως Ἡλιουπόλεως, τὸν Μανασσῆ, καὶ τὸν Ἐφραίμ· ἐγένοντο δὲ υἱοὶ Μανασσῆ, οὓς ἔτεκεν αὐτῷ ἡ παλλακὴ ἡ Σύρα, τὸν Μαχίρ· Μαχὶρ δὲ ἐγέννησε τὸν Γαλαάδ· υἱοὶ δὲ Ἐφραὶμ ἀδελφοῦ Μανασσῆ, Σουταλαὰμ, καὶ Ταάμ· υἱοὶ δὲ Σουταλαὰμ, Ἐδώμ.
\vs{21}Υἱοὶ δὲ Βενιαμεὶν, Βαλὰ καὶ Βοχὸρ, καὶ Ἀσβήλ. Ἐγένοντο δὲ υἱοὶ Βαλὰ, Γηρὰ, καὶ Νοεμὰν, καὶ Ἀγχὶς, καὶ Ῥὼς, καὶ Μαμφίμ· Γηρὰ δὲ ἐγέννησε τὸν Ἀράδ.
\vs{22}Οὗτοι υἱοὶ Ῥαχὴλ, οὓς ἔτεκε τῷ Ἰακώβ· πᾶσαι αἱ ψυχαὶ δεκαοκτώ.
\vs{23}Υἱοὶ δὲ Δὰν, Ἀσόμ.
\vs{24}Καὶ υἱοὶ Νεφθαλὶ, Ἀσιὴλ, καὶ Γωνὶ, καὶ Ἰσσάαρ, καὶ Σολλήμ.
\vs{25}Οὗτοι υἱοὶ Βαλλὰς, ἣν ἔδωκε Λάβαν Ῥαχὴλ τῇ θυγατρὶ αὐτοῦ, ἣ ἔτεκε τούτους τῷ Ἰακὼβ, πᾶσαι αἱ ψυχαὶ ἑπτά.
\vs{26}Πᾶσαι δὲ ψυχαὶ αἱ εἰσελθοῦσαι μετὰ Ἰακὼβ εἰς Αἴγυπτον, οἱ ἐξελθόντες ἐκ τῶν μηρῶν αὐτοῦ, χωρὶς τῶν γυναικῶν υἱῶν Ἰακὼβ, πᾶσαι αἱ ψυχαὶ, ἑξηκονταέξ·
\vs{27}Υἱοὶ δὲ Ἰωσὴφ οἱ γενόμενοι αὐτῷ ἐν γῇ Αἰγύπτῳ, ψυχαὶ ἐννέα. Πᾶσαι ψυχαὶ οἴκου Ἰακὼβ, αἱ εἰσελθοῦσαι μετὰ Ἰακὼβ εἰς Αἴγυπτον, ψυχαὶ ἑβδομηκονταπέντε.

\vs{28}Τὸν δὲ Ἰούδαν ἀπέστειλεν ἔμπροσθεν αὐτοῦ πρὸς Ἰωσὴφ, συναντῆσαι αὐτῷ καθʼ Ἡρώων πόλιν, εἰς γῆν Ῥαμεσσῆ.
\vs{29}Ζεύξας δὲ Ἰωσὴφ τὰ ἅρματα αὐτοῦ, ἀνέβη εἰς συνάντησιν Ἰσραὴλ τῷ πατρὶ αὐτοῦ, καθʼ Ἡρώων πόλιν· καὶ ὀφθεὶς αὐτῷ ἐπέπεσεν ἐπὶ τὸν τράχηλον αὐτοῦ, καὶ ἔκλαυσε κλαυθμῷ πίονι.
\vs{30}Καὶ εἶπεν Ἰσραήλ πρὸς Ἰωσὴφ, ἀποθανοῦμαι ἀπὸ τοῦ νῦν, ἐπεὶ ἑώρακα τὸ πρόσωπόν σου· ἔτι γὰρ σὺ ζῇς.
\vs{31}Εἶπε δὲ Ἰωσὴφ πρὸς τοὺς ἀδελφοὺς αὐτοῦ, ἀναβὰς ἀπαγγελῶ τῷ Φαραῷ, καὶ ἐρῶ αὐτῷ, οἱ ἀδελφοί μου, καὶ ὁ οἶκος τοῦ πατρός μου, οἳ ἦσαν ἐν γῇ χαναὰν, ἥκασι πρός με.
\vs{32}Οἱ δὲ ἄνδρες εἰσὶ ποιμένες· ἄνδρες γὰρ κτηνοτρόφοι ἦσαν· καὶ τὰ κτήνη, καὶ τοὺς βόας, καὶ πάντα τὰ αὐτῶν ἀγηόχασιν.
\vs{33}Ἐὰν οὖν καλέσῃ ὑμᾶς Φαραὼ, καὶ εἴπῃ ὑμῖν, τί τὸ ἔργον ὑμῶν ἐστι;
\vs{34}Ἐρεῖτε, ἄνδρες κτηνοτρόφοι ἐσμὲν οἱ παῖδές σου ἐκ παιδὸς ἕως τοῦ νῦν, καὶ ἡμεῖς, καὶ οἱ πατέρες ἡμῶν· ἵνα κατοικήσητε ἐν γῇ Γεσὲμ Ἀραβίας· βδέλυγμα γάρ ἐστιν Αἰγυπτίοις πᾶς ποιμὴν προβάτων.

\ch{47}
Ἐλθῶν δὲ Ἰωσὴφ ἀπήγγειλε τῷ Φαραῶ, λέγων, ὁ πατὴρ μου, καὶ οἱ ἀδελφοί μου, καὶ τὰ κτήνη, καὶ οἱ βόες αὐτῶν, καὶ πάντα τὰ αὐτῶν, ἦλθον ἐκ γῆς Χαναάν· καὶ ἰδού εἰσιν ἐν γῇ Γεσέμ.
\vs{2}Ἀπὸ δὲ τῶν ἀδελφῶν αὐτοῦ παρέλαβε πέντε ἄνδρας, καὶ ἔστησεν αὐτοὺς ἐναντίον Φαραώ.
\vs{3}Καὶ εἶπε Φαραὼ τοῖς ἀδελφοῖς Ἰωσὴφ, Τί τὸ ἔργον ὑμῶν; οἱ δὲ εἶπαν τῷ Φαραῷ, ποιμένες προβάτων οἱ παῖδές σου, καὶ ἡμεῖς καὶ οἱ πατέρες ἡμῶν.
\vs{4}Εἶπαν δὲ τῷ Φαραῷ, παροικεῖν ἐν τῇ γῇ ἥκαμεν, οὐ γάρ ἐστι νομὴ τοῖς κτήνεσι τῶν παιδων σου, ἐνίσχυσε γὰρ ὁ λιμὸς ἐν γῇ Χανάαν· νῦν οὖν κατοικήσομεν ἐν γῇ Γεσέμ. Εἶπε δὲ Φαραὼ τῷ Ἰωσὴφ, Κατοικείτωσαν ἐν γῇ Γεσέμ· εἰ δὲ ἐπίστῃ, ὅτι εἰσὶν ἐν αὐτοῖς ἄνδρες δυνατοὶ, κατάστησον αὐτοὺς ἄρχοντας τῶν ἐμῶν κτηνῶν. Ἦλθον δὲ εἰς Αἴγυπτον πρὸς Ἰωσὴφ Ἰακὼβ, καὶ οἱ υἱοὶ αὐτοῦ· καὶ ἤκουσε Φαραὼ βασιλεὺς Αἰγύπτου.
\vs{5}Καὶ εἶπε Φαραὼ πρὸς Ἰωσὴφ, λέγων, ὁ πατήρ σου, καὶ οἱ ἀδελφοί σου, ἥκασι πρὸς σέ.
\vs{6}Ἰδοὺ ἡ γῆ Αἰγύπτου ἐναντίον σου ἐστίν· ἐν τῇ βελτίστῃ γῇ κατοίκισον τὸν πατέρα σου, καὶ τοὺς ἀδελφούς σου.
\vs{7}Εἰσήγαγε δὲ Ἰωσὴφ Ἰακὼβ τὸν πατέρα αὐτοῦ, καὶ ἔστησεν αὐτὸν ἐναντίον Φαραώ· καὶ ηὐλόγησεν Ἰακὼβ τὸν Φαραώ.
\vs{8}Εἶπε δὲ Φαραὼ τῷ Ἰακὼβ, πόσα ἔτη ἡμερῶν τῆς ζωῆς σου;
\vs{9}Καὶ εἶπεν Ἰακὼβ τῷ Φαραῷ, αἱ ἡμέραι τῶν ἐτῶν τῆς ζωῆς μου, ἃς παροικῶ, ἑκατὸν τριάκοντα ἔτη· μικραὶ καὶ πονηραὶ γεγόνασιν αἱ ἡμέραι τῶν ἐτῶν τῆς ζωῆς μου· οὐκ ἀφίκοντο εἰς τὰς ἡμέρας τῶν ἐτῶν τῆς ζῶης τῶν πατέρων μου, ἃς ἡμέρας παρῴκησαν.
\vs{10}Καὶ εὐλογήσας Ἰακὼβ τὸν Φαραὼ, ἐξῆλθεν ἀπʼ αὐτοῦ.

\vs{11}Καὶ κατῴκισεν Ἰωσὴφ τὸν πατέρα αὐτοῦ, καὶ τοὺς ἀδελφοὺς αὐτοῦ, καὶ ἔδωκεν αὐτοῖς κατάσχεσιν ἐν γῇ Αἰγύπτῳ, ἐν τῇ βελτίστῃ γῇ, ἐν γῇ Ῥαμεσσῆ, καθὰ προσέταξε Φαραώ.
\vs{12}Καὶ ἐσιτομέτρει Ἰωσὴφ τῷ πατρὶ αὐτοῦ, καὶ τοῖς ἀδελφοῖς, καὶ παντὶ τῷ οἴκῳ τοῦ πατρὸς αὐτοῦ, σῖτον κατὰ σῶμα.

\vs{13}Σῖτος δὲ οὐκ ἦν ἐν πάσῃ τῇ γῇ, ἐνίσχυσε γὰρ ὁ λιμὸς σφόδρα· ἐξέλιπε δὲ ἡ γῆ Αἰγύπτου καὶ ἡ γῆ Χαναὰν ἀπὸ τοῦ λιμοῦ.
\vs{14}Συνήγαγε δὲ Ἰωσὴφ πᾶν τὸ ἀργύριον τὸ εὑρεθὲν ἐν γῇ Αἰγύπτου καὶ ἐν γῇ Χαναὰν, τοῦ σίτου, οὗ ἠγόραζον, καὶ ἐσιτομέτρει αὐτοῖς, καὶ εἰσήνεγκεν Ἰωσὴφ πᾶν τὸ ἀργύριον εἰς τὸν οἶκον Φαραώ.
\vs{15}Καὶ ἐξέλιπε πᾶν τὸ ἀργύριον ἐκ γῆς Αἰγύπτου καὶ ἐκ γῆς Χαναάν· ἦλθον δὲ πάντες οἱ Αἰγύπτιοι πρὸς Ἰωσὴφ, λέγοντες, δὸς ἡμῖν ἄρτους, καὶ ἱνατί ἀποθνήσκομεν ἐναντίον σου; ἐκλέλοιπε γὰρ τὸ ἀργύριον ἡμῶν.
\vs{16}Εἶπε δὲ αὐτοῖς Ἰωσὴφ, φέρετε τὰ κτήνη ὑμῶν, καὶ δώσω ὑμῖν ἄρτους, ἀντὶ τῶν κτηνῶν ὑμῶν, εἰ ἐκλέλοιπε τὸ ἀργύριον ὑμῶν.
\vs{17}Ἤγαγον δὲ τὰ κτήνη αὐτῶν πρὸς Ἰωσήφ· καὶ ἔδωκεν αὐτοῖς Ἰωσὴφ ἄρτους ἀντὶ τῶν ἵππων, καὶ ἀντὶ τῶν προβάτων, καὶ ἀντὶ τῶν βοῶν, καὶ ἀντὶ τῶν ὄνων· καὶ ἐξέθρεψεν αὐτοὺς ἐν ἄρτοις ἀντὶ πάντων τῶν κτηνῶν αὐτῶν ἐν τῷ ἐνιαυτῷ ἐκείνῳ.
\vs{18}Ἐξῆλθε δὲ τὸ ἔτος ἐκεῖνο, καὶ ἦλθον πρὸς αὐτὸν ἐν τῷ ἔτει τῷ δευτέρῳ, καὶ εἶπαν αὐτῷ, μή ποτε ἐκτριβῶμεν ἀπὸ τοῦ κυρίου ἡμῶν; εἰ γὰρ ἐκλέλοιπε τὸ ἀργύριον ἡμῶν, καὶ τὰ ὑπάρχοντα καὶ τὰ κτήνη πρός σε τὸν κύριον, καὶ οὐχ ὑπολέλειπται ἡμῖν ἐναντίον τοῦ κυρίου ἡμῶν, ἀλλʼ ἢ τὸ ἴδιον σῶμα καὶ ἡ γῆ ἡμῶν,
\vs{19}ἵνα οὖν μὴ ἀποθάνωμεν ἐναντίον σου, καὶ ἡ γῆ ἐρημωθῇ, κτῆσαι ἡμᾶς καὶ τὴν γῆν ἡμῶν ἀντὶ ἄρτων, καὶ ἐσόμεθα ἡμεῖς καὶ ἡ γῆ ἡμῶν παῖδες τῷ Φαραώ· δὸς σπέρμα, ἵνα σπείρωμεν, καὶ ζῶμεν καὶ μὴ ἀποθάνωμεν, καὶ ἡ γῆ οὐκ ἐρημωθήσεται.
\vs{20}Καὶ ἐκτήσατο Ἰωσὴφ πᾶσαν τὴν γῆν τῶν Αἰγυπτίων τῷ Φαραώ· ἀπέδοντο γὰρ οἱ Αἰγύπτιοι τὴν γῆν αὐτῶν τῷ Φαραώ· ἐπεκράτησε γὰρ αὐτῶν ὁ λιμός· καὶ ἐγένετο ἡ γῆ τῷ Φαραώ.
\vs{21}Καὶ τὸν λαὸν κατεδουλώσατο αὐτῷ εἰς παῖδας, ἀπʼ ἄκρων ὁρίων Αἰγύπτου ἕως τῶν ἄκρων,
\vs{22}χωρὶς τῆς γῆς τῶν ἱερέων μόνον· οὐκ ἐκτήσατο ταύτην Ἰωσήφ· ἐν δόσει γὰρ ἔδωκε δόμα τοῖς ἱερεῦσι Φαραὼ, καὶ ἤσθιον τὴν δόσιν, ἣν ἔδωκεν αὐτοῖς Φαραώ· διὰ τοῦτο οὐκ ἀπέδοντο τὴν γῆν αὐτῶν.
\vs{23}Εἶπε δὲ Ἰωσὴφ πᾶσι τοῖς Αἰγυπτίοις, ἰδοὺ κέκτημαι ὑμᾶς καὶ τὴν γῆν ὑμῶν σήμερον τῷ Φαραῷ· λάβετε ἑαυτοῖς σπέρμα, καὶ σπείρατε τὴν γῆν.
\vs{24}Καὶ ἔσται τὰ γεννήματα αὐτῆς· καὶ δώσετε τὸ πεμπτὸν μέρος τῷ Φαραώ· τὰ δὲ τέσσαρα μέρη ἔσται ὑμῖν αὐτοῖς εἰς σπέρμα τῇ γῇ, καὶ εἰς βρῶσιν ὑμῖν, καὶ πᾶσι τοῖς ἐν τοῖς οἴκοις ὑμῶν.
\vs{25}Καὶ εἶπαν, σέσωκας ἡμᾶς· εὕρομεν χάριν ἐναντίον τοῦ κυρίου ἡμῶν, καὶ ἐσόμεθα παῖδες τῷ Φαραώ.
\vs{26}Καὶ ἔθετο αὐτοῖς Ἰωσὴφ εἰς πρόσταγμα ἕως τῆς ἡμέρας ταύτης, ἐπὶ γῆς Αἰγύπτου τῷ Φαραὼ ἀποπεμπτοῦν, χωρὶς τῆς γῆς τῶν ἱερέων μόνον· οὐκ ἧν τῷ Φαραώ.

\vs{27}Κατῶκησε δὲ Ἰσραὴλ ἐν γῇ Αἰγύπτῳ ἐπὶ γῆς Γεσὲμ, καὶ ἐκληρονόμησαν ἐπʼ αὐτῆς· καὶ ηὐξήθησαν καὶ ἐπληθύνθησαν σφόδρα.
\vs{28}Ἐπέζησε δὲ Ἰακὼβ ἐν γῇ Αἰγύπτῳ δεκαεπτὰ ἔτη· καὶ ἐγένοντο αἱ ἡμέραι Ἰακὼβ ἐνιαυτῶν τῆς ζωῆς αὐτοῦ ἑκατὸν τεσσαρακονταεπτὰ ἔτη.
\vs{29}Ἤγγισαν δὲ αἱ ἡμέραι Ἰσραὴλ τοῦ ἀποθανεῖν· καὶ ἐκάλεσε τὸν υἱὸν αὐτοῦ Ἰωσὴφ, καὶ εἶπεν αὐτῷ, εἰ εὕρηκα χάριν ἐναντίον σου, ὑπόθες τὴν χεῖρά σου ὑπὸ τὸν μηρόν μου, καὶ ποιήσεις ἐπʼ ἐμὲ ἐλεημοσύνην, καὶ ἀλήθειαν, τοῦ μή με θάψαι ἐν Αἰγύπτῳ·
\vs{30}Ἀλλὰ κοιμηθήσομαι μετὰ τῶν πατέρων μου· καὶ ἀρεῖς με ἐξ Αἰγύπτου, καὶ θάψεις με ἐν τῷ τάφῳ αὐτῶν· ὁ δὲ εἶπεν, ἐγὼ ποιήσω κατὰ τὸ ῥῆμά σου.
\vs{31}Εἶπε δὲ, ὄμοσόν μοι· καὶ ὤμοσεν αὐτῷ· καὶ προσεκύνησεν Ἰσραὴλ ἐπὶ τὸ ἄκρον τῆς ῥάβδου αὐτοῦ.

\ch{48}
Ἐγένετο δὲ μετὰ τὰ ῥήματα ταῦτα, καὶ ἀπηγγέλη τῷ Ἰωσὴφ, ὅτι ὁ πατήρ σου ἐνοχλεῖται· καὶ ἀναλαβὼν τοὺς δύο υἱοὺς αὐτοῦ τὸν Μανασσῆ καὶ τὸν Ἐφραὶμ, ἦλθε πρὸς Ἰακώβ.
\vs{2}Ἀπηγγέλη δὲ τῷ Ἰακὼβ, λέγοντες, ἰδοὺ ὁ υἱός σου Ἰωσὴφ ἔρχεται πρὸς σέ· καὶ ἐνισχύσας Ἰσραὴλ ἐκάθισεν ἐπὶ τὴν κλίνην.
\vs{3}Καὶ εἶπεν Ἰακὼβ τῷ Ἰωσὴφ, ὁ Θεός μου ὤφθη μοι ἐν Λουζᾷ ἐν γῇ Χαναὰν, καὶ εὐλόγησέ με,
\vs{4}καὶ εἶπέ μοι, ἰδοὺ ἐγὼ αὐξανῶ σε, καὶ πληθυνῶ σε, καὶ ποιήσω σε εἰς συναγωγὰς ἐθνῶν· καὶ δώσω σοι τὴν γῆν ταύτην, καὶ τῷ σπέρματί σου μετὰ σὲ, εἰς κατάσχεσιν αἰώνιον.
\vs{5}Νῦν οὖν οἱ δύο υἱοί σου, οἱ γενόμενοί σοι ἐν γῇ Αἰγύπτῳ πρὸ τοῦ με ἐλθεῖν πρός σε εἰς Αἴγυπτον, ἐμοί εἰσιν, Ἐφραὶμ καὶ Μανασσῆ· ὡς Ῥουβὴν καὶ Συμεὼν ἔσονταί μοι.
\vs{6}Τὰ δὲ ἔκγονα, ἃ ἐὰν γεννήσῃς μετὰ ταῦτα, ἔσονται ἐπὶ τῷ ὀνόματι τῶν ἀδελφῶν αὐτῶν· κληθήσονται ἐπὶ τοῖς ἐκείνων κλήροις.
\vs{7}Ἐγὼ δὲ ἡνίκα ἠρχόμην ἐκ Μεσοποταμίας τῆς Συρίας, ἀπέθανε Ῥαχὴλ ἡ μήτηρ σου ἐν γῇ Χαναὰν, ἐγγίζοντός μου κατὰ τὸν ἱππόδρομον Χαβραθὰ τῆς γῆς, τοῦ ἐλθεῖν Ἐφραθά· καὶ κατώρυξα αὐτὴν ἐν τῇ ὁδῷ τοῦ ἱπποδρόμου· αὕτη ἐστὶ Βηθλεέμ.

\vs{8}Ἰδὼν δὲ Ἰσραὴλ τοὺς υἱοὺς Ἰωσὴφ, εἶπε, τίνες σοι οὗτοι;
\vs{9}Εἶπε δὲ Ἰωσὴφ τῷ πατρὶ αὐτοῦ, υἱοί μου εἰσὶν, οὓς ἔδωκε μοι ὁ Θεὸς ἐνταῦθα. Καὶ εἶπεν Ἰακὼβ, προσάγαγέ μοι αὐτοὺς, ἵνα εὐλογήσω αὐτούς.
\vs{10}Οἱ ὀφθαλμοὶ δὲ Ἰσραὴλ ἐβαρυώπησαν ἀπὸ τοῦ γήρως, καὶ οὐκ ἠδύνατο βλέπειν· καὶ ἤγγισεν αὐτοὺς πρὸς αὐτὸν, καὶ ἐφίλησεν αὐτοὺς, καὶ περιέλαβεν αὐτους.
\vs{11}Καὶ εἶπεν Ἰσραὴλ πρὸς Ἰωσὴφ, ἰδοὺ τοῦ προσώπου σου οὐκ ἐστερήθην, καὶ ἰδοὺ ἔδειξέ μοι ὁ Θεὸς καὶ τὸ σπέρμα σου.
\vs{12}Καὶ ἐξήγαγεν αὐτοὺς Ἰωσὴφ ἀπὸ τῶν γονάτων αὐτοῦ· καὶ προσεκύνησαν αὐτῷ ἐπὶ πρόσωπον ἐπὶ τῆς γῆς.
\vs{13}Λαβὼν δὲ Ἰωσὴφ τοὺς δύο υἱοὺς αὐτοῦ, τόν τε Ἐφραὶμ ἐν τῇ δεξιᾷ, ἐξ ἀριστερῶν δὲ Ἰσραὴλ, τὸν δὲ Μανασσῆ ἐξ ἀριστερῶν, ἐκ δεξιῶν δὲ Ἰσραὴλ, ἤγγισεν αὐτοὺς αὐτῷ.
\vs{14}Ἐκτείνας δὲ Ἰσραὴλ τὴν χεῖρα τὴν δεξιὰν, ἐπέβαλεν ἐπὶ τὴν κεφαλὴν Ἐφραὶμ, οὗτος δὲ ἦν ὁ νεώτερος, καὶ τὴν ἀριστερὰν ἐπὶ τὴν κεφαλὴν Μανασσῆ, ἐναλλὰξ τὰς χεῖρας.

\vs{15}Καὶ εὐλόγησεν αὐτοὺς, καὶ εἶπεν, ὁ Θεὸς, ᾧ εὐηρέστησαν οἱ πατέρες μου ἐνώπιον αὐτοῦ, Ἁβραὰμ καὶ Ἰσαὰκ, ὁ Θεὸς ὁ τρέφων με ἐκ νεότητος ἕως τῆς ἡμέρας ταύτης,
\vs{16}ὁ Ἄγγελος ὁ ῥυόμενός με ἐκ πάντων τῶν κακῶν, εὐλογήσαι τὰ παιδία ταῦτα· καὶ ἐπικληθήσεται ἐν αὐτοις τὸ ὄνομά μου, καὶ τὸ ὄνομα τῶν πατέρων μου Ἁβραὰμ καὶ Ἰσαάκ· καὶ πληθυνθείησαν εἰς πλῆθος πολὺ ἐπὶ τῆς γῆς.
\vs{17}Ἰδὼν δὲ Ἰωσὴφ ὅτι ἐπέβαλεν ὁ πατὴρ αὐτοῦ τὴν χεῖρα τὴν δεξιὰν αὐτοῦ ἐπὶ τὴν κεφαλὴν Ἐφραὶμ, βαρὺ αὐτῷ κατεφάνη· καὶ ἀντελάβετο Ἰωσὴφ τῆς χειρὸς τοῦ πατρὸς αὐτοῦ, ἀφελεῖν αὐτὴν ἀπὸ τῆς κεφαλῆς Ἐφραὶμ ἐπὶ τὴν κεφαλὴν Μανασσῆ.
\vs{18}Εἶπε δὲ Ἰωσὴφ τῷ πατρὶ αὐτοῦ, οὐχ οὕτως, πατὴρ, οὗτος γὰρ ὁ πρωτότοκος· ἐπίθες τὴν δεξιάν σου ἐπὶ τὴν κεφαλὴν αὐτοῦ.
\vs{19}Καὶ οὐκ ἠθέλησεν, ἀλλʼ εἶπεν, οἶδα, τέκνον, οἶδα· καὶ οὗτος ἔσται εἰς λαὸν, καὶ οὗτος ὑψωθήσεται· ἀλλʼ ὁ ἀδελφὸς αὐτοῦ ὁ νεώτερος μείζον αὐτοῦ ἔσται, καὶ τὸ σπέρμα αὐτοῦ ἔσται εἰς πλῆθος ἐθνῶν.
\vs{20}Καὶ εὐλόγησεν αὐτοὺς ἐν τῇ ἡμέρᾳ ἐκείνῃ, λέγων, ἐν ὑμῖν εὐλογηθήσεται Ἰσραὴλ, λέγοντες, ποιήσαι σε ὁ Θεὸς ὡς Ἐφραὶμ καὶ ὡς Μανασσῆ· καὶ ἔθηκε τὸν Ἐφραὶμ ἔμπροσθεν τοῦ Μανασσῆ.
\vs{21}Εἶπε δὲ Ἰσραὴλ τῷ Ἰωσὴφ, ἰδοὺ ἐγὼ ἀποθνήσκω· καὶ ἔσται ὁ Θεὸς μεθʼ ὑμῶν, καὶ ἀποστρέψει ὑμᾶς εἰς τὴν γῆν τῶν πατέρων ὑμῶν.
\vs{22}Ἐγὼ δὲ δίδωμί σοι Σίκιμα ἐξαίρετον ὑπὲρ τοὺς ἀδελφούς σου, ἣν ἔλαβον ἐκ χειρὸς Ἀμοῤῥαίων ἐν μαχαίρᾳ μου καὶ τόξῳ.

\ch{49}
Ἐκάλεσε δὲ Ἰακὼβ τοὺς υἱοὺς αὐτοῦ, καὶ εἶπεν αὐτοῖς, συνάχθητε, ἵνα ἀναγγείλω ὑμῖν, τί ἀπαντήσει ὑμῖν ἐπʼ ἐσχάτων τῶν ἡμέρων.
\vs{2}Συνάχθητε, καὶ ἀκούσατέ μου, υἱοὶ Ἰακώβ· ἀκούσατε Ἰσραὴλ, ἀκούσατε τοῦ πατρὸς ὑμῶν.
\vs{3}Ῥουβὴν πρωτότοκός μου, σὺ ἰσχύς μου, καὶ ἀρχὴ τέκνων μου, σκληρὸς φέρεσθαι, καὶ σκληρὸς αὐθάδης.
\vs{4}Ἐξύβρισας ὡς ὕδωρ, μὴ ἐκζέσῃς, ἀνέβης γὰρ ἐπὶ τὴν κοίτην τοῦ πατρός σου· τότε ἐμίανας τὴν στρωμνὴν, οὗ ἀνέβης.
\vs{5}Συμεὼν καὶ Λευὶ ἀδελφοὶ συνετέλεσαν ἀδικίαν ἐξαιρέσεως αὐτῶν·
\vs{6}Εἰς βουλὴν αὐτῶν μὴ ἔλθοι ἡ ψυχή μου, καὶ ἐπὶ τῇ συστάσει αὐτῶν μὴ ἐρίσαι τὰ ἥπατά μου· ὅτι ἐν τῷ θυμῷ αὐτῶν ἀπέκτειναν ἀνθρώπους, καὶ ἐν τῇ ἐπιθυμίᾳ αὐτῶν ἐνευροκόπησαν ταῦρον.
\vs{7}Ἐπικατάρατος ὁ θυμὸς αὐτὼν, ὅτι αὐθάδης· καὶ ἡ μῆνις αὐτῶν, ὅτι ἐσκληρύνθη· διαμεριῷ αὐτοὺς ἐν Ἰακὼβ, καὶ διασπερῷ αὐτοὺς ἐν Ἰσραήλ.
\vs{8}Ἰούδα, σὲ αἰνέσαισαν οἱ ἀδελφοί σου· αἱ χεῖρές σου ἐπὶ νώτου τῶν ἐχθρῶν σου· προσκυνήσουσί σοι οἱ υἱοὶ τοῦ πατρός σου.
\vs{9}Σκύμνος λέοντος Ἰούδα· ἐκ βλαστοῦ, υἱέ μου, ἀνέβης· ἀναπεσὼν ἐκοιμήθης ὡς λέων καὶ ὡς σκύμνος· τίς ἐγερεῖ αὐτόν;
\vs{10}Οὐκ ἐκλείψει ἄρχων ἐξ Ἰούδα, καὶ ἡγούμενος ἐκ τῶν μηρῶν αὐτοῦ, ἕως ἐὰν ἔλθῃ τὰ ἀποκείμενα αὐτῷ· καὶ αὐτὸς προσδοκία ἐθνῶν.
\vs{11}Δεσμεύων πρὸς ἄμπελον τὸν πῶλον αὐτοῦ, καὶ τῇ ἕλικι τὸν πῶλον τῆς ὄνου αὐτοῦ, πλυνεῖ ἐν οἴνῳ τὴν στολὴν αὐτοῦ, καὶ ἐν αἵματι σταφυλῆς τὴν περιβολὴν αὐτοῦ.
\vs{12}Χαροποιοὶ οἱ ὀφθαλμοὶ αὐτοῦ ὑπὲρ οἶνον· καὶ λευκοὶ οἱ ὀδόντες αὐτοῦ ἢ γάλα.
\vs{13}Ζαβουλὼν παράλιος κατοικήσει καὶ αὐτὸς παρʼ ὅρμον πλοίων, καὶ παρατενεῖ ἕως Σιδῶνος.
\vs{14}Ἰσσάχαρ τὸ καλὸν ἐπεθύμησεν, ἀναπαυόμενος ἀνὰ μέσον τῶν κλήρων.
\vs{15}Καὶ ἰδὼν τὴν ἀνάπαυσιν ὅτι καλὴ, καὶ τὴν γῆν ὅτι πίων, ὑπέθηκε τὸν ὦμον αὐτοῦ εἰς τὸ πονεῖν, καὶ ἐγενήθη ἀνὴρ γεωργός.
\vs{16}Δὰν κρινεῖ τὸν λαὸν αὐτοῦ, ὡσεὶ καὶ μία φυλὴ ἐν Ἰσραήλ.
\vs{17}Καὶ γενηθητω Δὰν ὄφις ἐφʼ ὁδοῦ, ἐγκαθήμενος ἐπὶ τρίβου, δάκνων πτέρναν ἵππου· καὶ πεσεῖται ὁ ἱππεὺς εἰς τὰ ὀπίσω,
\vs{18}τὴν σωτηρίαν περιμένων Κυρίου.
\vs{19}Γὰδ, πειρατήριον πειρατεύσει αὐτόν· αὐτὸς δὲ πειράτεύσει αὐτὸν κατὰ πόδας.
\vs{20}Ἀσὴρ, πίων αὐτοῦ ὁ ἄρτος· καὶ αὐτὸς δώσει τρυφὴν ἄρχουσι.
\vs{21}Νεφθαλὶ στέλεχος ἀνειμένον, ἐπιδιδοὺς ἐν τῷ γεννήματι κάλλος.
\vs{22}Υἱὸς ηὐξημένος Ἰωσὴφ, υἱὸς ηὐξημένος μου ζηλωτὸς, υἱός μου νεώτατος· πρός με ἀνάστρεψον.
\vs{23}Εἰς ὃν διαβουλευόμενοι ἐλοιδόρουν, καὶ ἐνεῖχον αὐτῷ κύριοι τοξευμάτων.
\vs{24}Καὶ συνετρίβη μετὰ κράτους τὰ τόξα αὐτῶν· καὶ ἐξελύθη τὰ νεῦρα βραχιόνων χειρὸς αὐτῶν, διὰ χεῖρα δυνάστου Ἰακώβ· ἐκεῖθεν ὁ κατισχύσας Ἰσραὴλ παρὰ Θεοῦ τοῦ πατρός σου.
\vs{25}Καὶ ἐβοήθησέ σοι ὁ Θεὸς ὁ ἐμὸς, καὶ εὐλόγησέ σε εὐλογίαν οὐρανοῦ ἄνωθεν, καὶ εὐλογίαν γῆς ἐχούσης πάντα, εἵνεκεν εὐλογίας μαστῶν καὶ μήτρας,
\vs{26}εὐλογίας πατρός σου καὶ μητρός σου· ὑπερίσχυσεν ὑπὲρ εὐλογίας ὀρέων μονίμων, καὶ ἐπʼ εὐλογίαις θινῶν ἀενάων· ἔσονται ἐπὶ κεφαλὴν Ἰωσὴφ, καὶ ἐπὶ κορυφῆς ὧν ἡγήσατο ἀδελφῶν.
\vs{27}Βενιαμὶν λύκος ἅρπαξ, τὸ πρωϊνὸν ἔδεται ἔτι, καὶ εἰς τὸ ἑσπέρας δίδωσι τροφήν.
\vs{28}Πάντες οὕτοι υἱοὶ Ἰακὼβ δώδεκα· καὶ ταῦτα ἐλάλησεν αὐτοῖς ὁ πατὴρ αὐτῶν· καὶ εὐλόγησεν αὐτούς· ἕκαστον κατὰ τὴν εὐλογίαν αὐτοῦ εὐλόγησεν αὐτούς.
\vs{29}Καὶ εἶπεν αὐτοῖς, ἐγὼ προστίθεμαι πρὸς τὸν ἐμὸν λαόν· θάψτέ με μετὰ τῶν πατέρων μου ἐν τῷ σπηλαίῳ, ὅ ἐστιν ἐν τῷ ἀγρῷ Ἐφρὼν τοῦ Χετταίου,
\vs{30}ἐν τῷ σπηλαίῳ τῷ διπλῷ, τῷ ἀπέναντι Μαμβρῆ, ἐν γῇ Χαναὰν, ὃ ἐκτήσατο Ἁβραὰμ τὸ σπήλαιον παρὰ Ἐφρὼν τοῦ Χετταίου ἐν κτήσει μνημείου.
\vs{31}Ἐκεῖ ἔθαψαν Ἁβραὰμ καὶ Σάῤῥαν τὴν γυναῖκα αὐτοῦ· ἐκεῖ ἔθαψαν Ἰσαὰκ καὶ Ῥεβέκκαν τὴν γυναῖκα αὐτοῦ· ἐκεῖ ἔθαψαν Λείαν·
\vs{32}Ἐν κτήσει τοῦ ἀγροῦ καὶ τοῦ σπηλαίου τοῦ ὄντος ἐν αὐτῷ, παρὰ τῶν υἱῶν Χέτ.
\vs{33}Καὶ κατέπαυσεν Ἰακὼβ ἐπιτάσσων τοῖς υἱοῖς αὐτοῦ· καὶ ἐξᾴρας τοὺς πόδας αὐτοῦ ἐπὶ τὴν κλίνην, ἐξέλιπε· καὶ προσετέθη πρὸς τὸν λαὸν αὐτοῦ.

\ch{50}
Καὶ ἐπιπεσὼν Ἰωσὴφ ἐπὶ πρόσωπον τοῦ πατρὸς αὐτοῦ ἔκλαυσεν αὐτὸν, καὶ ἐφίλησεν αὐτόν.
\vs{2}Καὶ προσέταξεν Ἰωσὴφ τοῖς παισὶν αὐτοῦ τοῖς ἐνταφιασταῖς, ἐνταφιάσαι τὸν πατέρα αὐτοῦ· καὶ ἐνεταφίασαν οἱ ἐνταφιασταὶ τὸν Ἰσραήλ.
\vs{3}Καὶ ἐπλήρωσαν αὐτοῦ τεσσαράκοντα ἡμέρας· οὕτω γὰρ καταριθμοῦνται αἱ ἡμέραι τῆς ταφῆς· καὶ ἐπένθησεν αὐτὸν Αἴγυπτος ἑβδομήκοντα ἡμέρας.
\vs{4}Ἐπεὶ δὲ παρῆλθον αἱ ἡμέραι τοῦ πένθους, ἐλάλησεν Ἰωσὴφ πρὸς τοὺς δυνάστας Φαραὼ, λέγων, εἰ εὗρον χάριν ἐναντίον ὑμῶν, λαλήσατε περὶ ἐμοῦ εἰς τὰ ὦτα Φαραὼ, λέγοντες,
\vs{5}ὁ πατήρ μου ὥρκισέ με, λέγων, ἐν τῷ μνημείῳ, ᾧ ὤρυξα ἐμαυτῷ ἐν γῇ Χαναὰν, ἐκεῖ με θάψεις· νῦν οὖν ἀναβὰς· θάψω τὸν πατέρα μου, καὶ ἐπανελεύσομαι·
\vs{6}Καὶ εἶπε Φαραὼ τῷ Ἰωσὴφ, ἀνάβηθι, θάψον τὸν πατέρα σου, καθάπερ ὥρκισέ σε.
\vs{7}Καὶ ἀνέβη Ἰωσὴφ θάψαι τὸν πατέρα αὐτοῦ· καὶ συνανέβησαν μετʼ αὐτοῦ πάντες οἱ παῖδες Φαραὼ, καὶ οἱ πρεσβύτεροι τοῦ οἴκου αὐτοῦ, καὶ πάντες οἱ πρεσβύτεροι τῆς γῆς Αἰγύπτου,
\vs{8}καὶ πᾶσα ἡ πανοικία Ἰωσὴφ, καὶ οἱ ἀδελφοὶ αὐτοῦ, καὶ πᾶσα ἡ οἰκία ἡ πατρικὴ αὐτοῦ, καὶ ἡ συγγένεια αὐτοῦ· καὶ τὰ πρόβατα, καὶ τοὺς βόας ὑπελίποντο ἐν γῇ Γεσέμ.
\vs{9}Καὶ συνανέβησαν μετʼ αὐτοῦ καὶ ἅρματα καὶ ἱππεῖς, καὶ ἐγένετο ἡ παρεμβολὴ μεγάλη σφόδρα.
\vs{10}Καὶ παρεγένοντο εἰς ἅλωνα Ἀτὰδ, ὅ ἐστι πέραν τοῦ Ἰορδάνου· καὶ ἐκόψαντο αὐτὸν κοπετὸν μέγαν καὶ ἰσχυρὸν σφόδρα· καὶ ἐποίησε τὸ πένθος τῷ πατρὶ αὐτοῦ ἑπτὰ ἡμέρας.
\vs{11}Καὶ εἶδον οἱ κάτοικοι τῆς γῆς Χαναὰν τὸ πένθος ἐπὶ ἅλωνι Ἀτὰδ, καὶ εἶπαν, πένθος μέγα τοῦτό ἐστι τοῖς Αἰγυπτίοις· διὰ τοῦτο ἐκάλεσε τὸ ὄνομα αὐτοῦ, Πένθος Αἰγύπτου, ὅ ἐστι πέραν τοῦ Ἰορδάνου.
\vs{12}Καὶ ἐποίησαν αὐτῷ οὕτως οἱ υἱοὶ αὐτοῦ.
\vs{13}Καὶ ἀνέλαβον αὐτὸν οἱ υἱοὶ αὐτοῦ εἰς γῆν Χαναάν· καὶ ἔθαψαν αὐτὸν εἰς τὸ σπήλαιον τὸ διπλοῦν, ὃ ἐκτήσατο Ἁβραὰμ τὸ σπήλαιον ἐν κτήσει μνημείου παρὰ Ἐφρὼν τοῦ Χετταίου, κατέναντι Μαμβρή.
\vs{14}Καὶ ὑπέστρεψεν Ἰωσὴφ εἰς Αἴγυπτον, αὐτὸς καὶ οἱ ἀδελφοὶ αὐτοῦ, καὶ οἱ συναναβάντες θάψαι τὸν πατέρα αὐτοῦ.

\vs{15}Ἰδόντες δὲ οἱ ἀδελφοὶ Ἰωσὴφ, ὅτι τέθνηκεν ὁ πατὴρ αὐτῶν, εἶπαν, μή ποτε μνησικακήσῃ ἡμῖν Ἰωσὴφ, καὶ ἀνταπόδομα ἀνταποδῷ ἡμῖν πάντα τὰ κακὰ, ἃ ἐνεδειξάμεθα εἰς αὐτὸν.
\vs{16}Καὶ παραγενόμενοι πρὸς Ἰωσὴφ εἶπαν, ὁ πατήρ σου ὥρκισε πρὸ τοῦ τελευτῆσαι αὐτὸν, λέγων,
\vs{17}οὕτως εἴπατε Ἰωσήφ· ἄφες αὐτοῖς τὴν ἀδικίαν καὶ τὴν ἁμαρτίαν αὐτῶν, ὅτι πονηρά σοι ἐνεδείξαντο· καὶ νῦν δέξαι τὴν ἀδικίαν τῶν θεραπόντων τοῦ Θεοῦ τοῦ πατρός σου· καὶ ἔκλαυσεν Ἰωσὴφ λαλούντων αὐτῶν πρὸς αὐτόν.
\vs{18}Καὶ ἐλθόντες πρὸς αὐτὸν εἶπαν, οἵδε ἡμεῖς σοὶ οἰκέται.
\vs{19}Καὶ εἶπεν αὐτοῖς Ἰωσὴφ, μὴ φοβεῖσθε, τοῦ γὰρ Θεοῦ εἰμι ἐγώ.
\vs{20}Ὑμεῖς ἐβουλεύσασθε κατʼ ἐμοῦ εἰς πονηρὰ, ὁ δὲ Θεὸς ἐβουλεύσατο περὶ ἐμοῦ εἰς ἀγαθὰ, ὅπως ἂν γενηθῇ ὡς σήμερον, καὶ τραφῇ λαὸς πολύς.
\vs{21}Καὶ εἶπεν αὐτοῖς, μὴ φοβεῖσθε· ἐγὼ διαθρέψω ὑμᾶς, καὶ τὰς οἰκίας ὑμῶν· καὶ παρεκάλεσεν αὐτοὺς, καὶ ἐλάλησεν αὐτῶν εἰς τὴν καρδίαν.
\vs{22}Καὶ κατῴκησεν Ἰωσὴφ ἐν Αἰγύπτῳ, αὐτὸς καὶ οἱ ἀδελφοὶ αὐτοῦ, καὶ πᾶσα ἡ πανοικία τοῦ πατρὸς αὐτοῦ· καὶ ἔζησεν Ἰωσὴφ ἔτη ἑκατὸν δέκα.
\vs{23}Καὶ εἶδεν Ἰωσὴφ Ἐφραὶμ παιδία, ἕως τρίτης γενεᾶς· καὶ οἱ υἱοὶ Μαχεὶρ τοῦ υἱοῦ Μανασσῆ ἐτέχθησαν ἐπὶ μηρῶν Ἰωσήφ.
\vs{24}Καὶ εἶπεν Ἰωσὴφ τοῖς ἀδελφοῖς αὐτοῦ, λέγων, ἐγὼ ἀποθνήσκω· ἐπισκοπῇ δὲ ἐπισκέψεται ὁ Θεὸς ὑμᾶς, καὶ ἀνάξει ὑμᾶς ἐκ τῆς γῆς ταύτης εἰς τὴν γῆν, ἣν ὤμοσεν ὁ Θεὸς τοῖς πατράσιν ἡμῶν, Ἁβραὰμ, Ἰσαὰκ, καὶ Ἰακώβ.
\vs{25}Καὶ ὥρκισεν Ἰωσὴφ τοὺς υἱοὺς Ἰσραὴλ, λέγων, ἐν τῇ ἐπισκοπῇ ᾗ ἐπισκέψηται ὁ Θεὸς ὑμᾶς, καὶ συνανοίσετε τὰ ὀστᾶ μου ἐντεῦθεν μεθʼ ὑμῶν.
\vs{26}Καὶ ἐτελεύτησεν Ἰωσὴφ ἐτῶν ἑκατὸν δέκα· καὶ ἔθαψαν αὐτὸν, καὶ ἔθηκαν ἐν τῇ σορῷ ἐν Αἰγύπτῳ.


\def\book{ΕΞΟΔΟΣ}
\biblebook{ΕΞΟΔΟΣ}


\lettrine[lines=2, loversize=0.2, nindent=0em, findent=.25em]{\textcolor{bookheadingcolor}{Τ}}{ΑΥΤΑ} τὰ ὀνόματα τῶν υἱῶν Ἰσραὴλ τῶν εἰσπεπορευμένων εἰς Αἴγυπτον ἅμα Ἰακὼβ τῷ πατρὶ αὐτῶν, ἕκαστος πανοικὶ αὐτῶν εἰσήλθοσαν.
\vs{2}Ῥουβὴν, Συμεών, Λευὶ, Ἰούδας,
\vs{3}Ἰσσάχαρ, Ζαβουλὼν, Βενιαμὶν,
\vs{4}Δὰν, καὶ Νεφθαλὶ, Γὰδ, καὶ Ἀσήρ.
\vs{5}Ἰωσὴφ δὲ ἦν ἐν Αἰγύπτῳ· ἦσαν δὲ πᾶσαι ψυχαὶ ἐξ Ἰακὼβ, πέντε καὶ ἑβδομήκοντα.
\vs{6}Ἐτελεύτησε δὲ Ἰωσὴφ, καὶ πάντες οἱ ἀδελφοὶ αὐτοῦ, καὶ πᾶσα ἡ γενεὰ ἐκείνη.
\vs{7}Οἱ δὲ υἱοὶ Ἰσραὴλ ηὐξήθησαν, καὶ ἐπληθύνθησαν, καὶ χυδαῖοι ἐγένοντο, καὶ κατίσχυον σφόδρα σφόδρα· ἐπλήθυνε δὲ ἡ γῆ αὐτούς.
\vs{8}Ἀνέστη δὲ βασιλεὺς ἕτερος ἐπʼ Αἴγυπτον, ὃς οὐκ ᾔδει τὸν Ἰωσήφ.
\vs{9}Εἶπε δὲ τῷ ἔθνει αὐτοῦ, ἰδοὺ τὸ γένος τῶν υἱῶν Ἰσραὴλ μέγα πλῆθος, καὶ ἰσχύει ὑπὲρ ἡμᾶς.
\vs{10}Δεῦτε οὖν κατασοφισώμεθα αὐτοὺς, μήποτε πληθυνθῇ, καὶ ἡνίκα ἂν συμβῇ ἡμῖν πόλεμος, προστεθήσονται καὶ οὗτοι πρὸς τοὺς ὑπεναντίους, καὶ ἐκπολεμήσαντες ἡμᾶς, ἐξελεύσονται ἐκ τῆς γῆς.
\vs{11}Καὶ ἐπέστησεν αὐτοῖς ἐπιστάτας τῶν ἔργων, ἵνα κακώσωσιν αὐτοὺς ἐν τοῖς ἔργοις. Καὶ ᾠκοδόμησαν πόλεις ὀχυρὰς τῷ Φαραῷ, τήν τε Πειθὼ, καὶ Ῥαμεσσῆ, καὶ Ὢν, ἥ ἐστιν Ἡλιού πολις.
\vs{12}Καθότι δὲ αὐτοὺς ἐταπείνουν, τοσούτῳ πλείους ἐγίνοντο, καὶ ἴσχυον σφόδρα σφόδρα· καὶ ἐβδελύσσοντο οἱ Αἰγύπτιοι ἀπὸ τῶν υἱῶν Ἰσραήλ.
\vs{13}Καὶ κατεδυνάστευον οἱ Αἰγύπτιοι τοὺς υἱοὺς Ἰσραὴλ βίᾳ.
\vs{14}Καὶ κατωδύνων αὐτῶν τὴν ζωὴν ἐν τοῖς ἔργοις τοῖς σκληροῖς, τῷ πηλῷ καὶ τῇ πλινθείᾳ, καὶ πᾶσι τοῖς ἔργοις τοῖς ἐν τοῖς πεδίοις, κατὰ πάντα τὰ ἔργα, ὧν κατεδουλοῦντο αὐτοὺς μετὰ βίας.

\vs{15}Καὶ εἶπεν ὁ βασιλεὺς τῶν Αἰγυπτίων ταῖς μαίαις τῶν Ἐβραίων, τῇ μιᾷ αὐτῶν ὄνομα Σεπφώρα, καὶ τὸ ὄνομα τῆς δευτέρας Φουά·
\vs{16}Καὶ εἶπεν, ὅταν μαιοῦσθε τὰς Ἐβραίας, καὶ ὦσι πρὸς τῷ τίκτειν, ἐὰν μὲν ἄρσεν ᾖ, ἀποκτείνατε αὐτό· ἐὰν δὲ θῆλυ, περιποιεῖσθε αὐτό.
\vs{17}Ἐφοβήθησαν δὲ αἱ μαῖαι τὸν Θεὸν, καὶ οὐκ ἐποίησαν καθότι συνέταξεν αὐταῖς ὁ βασιλεὺς Αἰγύπτου, καὶ ἐζωογόνουν τὰ ἄρσενα.
\vs{18}Ἐκάλεσε δὲ ὁ βασιλεὺς Αἰγύπτου τὰς μαίας, καὶ εἶπεν αὐταῖς, τί ὅτι ἐποιήσατε τὸ πρᾶγμα τοῦτο, καὶ ἐζωογονεῖτε τὰ ἄρσενα;
\vs{19}Εἶπαν δὲ αἱ μαῖαι τῷ Φαραῷ, οὐχ ὡς γυναῖκες Αἰγύπτου αἱ Ἐβραῖαι· τίκτουσι γὰρ πρὶν ἢ εἰσελθεῖν πρὸς αὐτὰς τὰς μαίας· καὶ ἔτικτον.
\vs{20}Εὖ δὲ ἐποίει ὁ Θεὸς ταῖς μαίαις· καὶ ἐπλήθυνεν ὁ λαὸς, καὶ ἴσχυε σφόδπα.
\vs{21}Ἐπεὶ δὲ ἐφοβοῦντο αἱ μαῖαι τὸν Θεὸν, ἐποίησαν ἑαυταῖς οἰκίας.
\vs{22}Συνέταξε δὲ Φαραὼ παντὶ τῷ λαῷ αὐτοῦ, λέγων, πᾶν ἄρσεν, ὃ ἐὰν τεχθῇ τοῖς Ἑβραίοις, εἰς τὸν ποταμὸν ῥίψατε, καὶ πᾶν θῆλυ, ζωογονεῖτε αὐτό.

\ch{2}
Ἦν δέ τις ἐκ τῆς φυλῆς Λευὶ, ὃς ἔλαβεν τῶν θυγατέρων Λευί.
\vs{2}Καὶ ἐν γαστρὶ ἔλαβε, καὶ ἔτεκεν ἄρσεν· ἰδόντες δὲ αὐτὸ ἀστεῖον, ἐσπέπασαν αὐτὸ μῆνας τρεῖς.
\vs{3}Ἐπεὶ δὲ οὐκ ἐδύναντο αὐτὸ ἔτι κρύπτειν, ἔλαβεν αὐτῷ ἡ μήτηρ αὐτοῦ θῖβιν, καὶ κατέχρισεν αὐτὴν ἀσφαλτοπίσσῃ, καὶ ἐνέβαλε τὸ παιδίον εἰς αὐτήν, καὶ ἔθηκεν αὐτὴν εἰς τὸ ἕλος παρὰ τὸν ποταμόν.
\vs{4}Καὶ κατεσκόπευεν ἡ ἀδελφὴ αὐτοῦ μακρόθεν, μαθεῖν τί τὸ ἀποβησόμενον αὐτῷ.

\vs{5}Κατέβη δὲ ἡ θυγάτηρ Φαραὼ λούσασθαι ἐπὶ τὸν ποταμὸν, καὶ αἱ ἅβραι αὐτῆς παρεπορεύοντο παρὰ τὸν ποταμόν· καὶ ἰδοῦσα τὴν θίβιν ἐν τῷ ἕλει, ἀποστείλασα τὴν ἅβραν, ἀνείλατο αὐτήν.
\vs{6}Ἀνοίξασα δὲ ὁρᾷ παιδίον κλαῖον ἐν τῇ θίβει· καὶ ἐφείσατο αὐτοῦ ἡ θυγάτηρ Φαραὼ, καὶ ἔφη, ἀπὸ τῶν παιδίων τῶν Ἐβραίων τοῦτο.
\vs{7}Καὶ εἶπεν ἡ ἀδελφὴ αὐτοῦ τῇ θυγατρὶ Φαραὼ, θέλεις καλέσω σοι γυναῖκα τροφεύουσαν ἐκ τῶν Ἐβραίων, καὶ θηλάσει σαι τὸ παιδὶον σοι τὸ παιδίον;
\vs{8}Ἡ δὲ εἶπεν ἡ θυγάτηρ Φαραὼ, πορεύου· ἐλθοῦσα δὲ νεᾶνις ἐκάλεσε τὴν μητέρα τοῦ παιδίου.
\vs{9}Εἶπεν δὲ πρὸς αὐτὴν ἡ θυγάτηρ Φαραὼ, διατήρησόν μοι τὸ παιδίον τοῦτο, καὶ θήλασόν μοι αὐτὸ, ἐγὼ δὲ δώσω σοι τὸν μισθόν· ἔλαβε δὲ ἡ γυνὴ τὸ παιδίον, καὶ ἐθήλαζεν αὐτό.
\vs{10}Ἁδρυνθέντος δὲ τοῦ παιδίου, εἰσήγαγεν αὐτὸ πρὸς τὴν θυγατέρα Φαραὼ, καὶ ἐγενήθη αὐτῇ εἰς υἱόν· ἐπωνόμασε δὲ τὸ ὄνομα αὐτοῦ Μωυσῆν, λέγουσα, ἐκ τοῦ ὕδατος αὐτὸν ἀνειλόμην.

\vs{11}Ἐγένετο δὲ ἐν ταῖς ἡμέραις ταῖς πολλαῖς ἐκείναις μέγας γενόμενος Μωυσῆς, ἐξῆλθε πρὸς τοὺς ἀδελφοὺς αὐτοῦ τοὺς υἱοὺς Ἰσραήλ· κατανοήσας δὲ τὸν πόνον αὐτῶν, ὁρᾷ ἄνθρωπον Αἰγύπτιον τύπτοντα τινὰ Ἐβραῖον, τῶν ἑαυτοῦ ἀδελφῶν τῶν υἱῶν Ἰσραήλ.
\vs{12}Περιβλεψάμενος δὲ ὧδε καὶ ὧδε οὐχ ὁρᾷ οὐδένα, καὶ πατάξας τὸν Αἰγύπτιον, ἔκρυψεν αὐτὸν ἐν τῇ ἄμμῳ.
\vs{13}Ἐξελθὼν δὲ τῇ ἡμέρᾳ τῇ δευτέρᾳ, ὁρᾷ δύο ἄνδρας Ἐβραίους διαπληκτιζομένους· καὶ λέγει τῷ ἀδικοῦντι, διὰ τί σὺ τύπτεις τὸν πλησίον;
\vs{14}Ὁ δὲ εἶπε, τίς σε κατέστησεν ἄρχοντα καὶ δικαστὴν ἐφʼ ἡμῶν; μὴ ἀνελεῖν με σὺ θέλεις, ὃν τρόπον ἀνεῖλες χθὲς τὸν Αἰγύπτιον; ἐφοβήθη δὲ Μωυσῆς, καὶ εἶπεν, εἰ οὕτως ἐμφανὲς γέγονε τὸ ῥῆμα τοῦτο.
\vs{15}Ἤκουσε δὲ Φαραὼ τὸ ῥῆμα τοῦτο, καὶ ἐζήτει ἀνελεῖν Μωυσῆν. Ἀνεχώρησε δὲ Μωυσῆς ἀπὸ προσώπου Φαραὼ, καὶ ᾤκησεν ἐν γῇ Μαδιάμ· ἐλθὼν δὲ εἰς γῆν Μαδιὰμ, ἐκάθισεν ἐπὶ τοῦ φρέατος.
\vs{16}Τῷ δὲ ἱερεῖ Μαδιὰμ ἦσαν ἑπτὰ θυγατέρες, ποιμαίνουσαι τὰ πρόβατα τοῦ πατρὸς αὐτῶν Ἰοθόρ· παραγενόμεναι δὲ ἤντλουν, ἕως ἔπλησαν τὰς δεξαμενάς, ποτίσαι τὰ πρόβατα τοῦ πατρὸς αὐτῶν Ἰοθόρ.
\vs{17}Παραγενόμενοι δὲ οἱ ποιμένες ἐξέβαλλον αὐτάς· ἀναστὰς δὲ Μωυσῆς ἐῤῥύσατο αὐτὰς, καὶ ἤντλησεν αὐταῖς, καὶ ἐπότισε τὰ πρόβατα αὐτῶν.
\vs{18}Παρεγένοντο δὲ πρὸς Ῥαγουὴλ τὸν πατέρα αὐτῶν· ὁ δὲ εἶπεν αὐταῖς, διατί ἐταχύνατε τοῦ παραγενέσθαι σήμερον;
\vs{19}Αἱ δὲ εἶπαν, ἄνθρωπος Αἰγύπτιος ἐῤῥύσατο ἡμᾶς ἀπὸ τῶν ποιμένων, καὶ ἤντλησεν ἡμῖν, καὶ ἐπότισε τὰ πρόβατα ἡμῶν.
\vs{20}Ὁ δὲ εἶπε ταῖς θυγατράσιν αὐτοῦ, καὶ ποῦ ἐστιν; καὶ ἱνατί καταλελοίπατε τὸν ἄνθρωπον; καλέσατε οὖν αὐτὸν, ὅπως φάγῃ ἄρτον.
\vs{21}Κατῳκίσθη δὲ Μωυσῆς παρὰ τῷ ἀνθρώπῳ· καὶ ἐξέδοτο Σεπφώραν τὴν θυγατέρα αὐτοῦ Μωυσῇ γυναῖκα.
\vs{22}Ἐν γαστρὶ δὲ λαβοῦσα ἡ γυνὴ ἔτεκεν υἱόν· καὶ ἐπωνόμασε Μωυσῆς τὸ ὄνομα αὐτοῦ Γηρσάμ, λέγων, ὅτι παροικός εἰμι ἐν γῇ ἀλλοτρίᾳ.
\vs{23}Μετὰ δὲ τὰς ἡμέρας τὰς πολλὰς ἐκείνας, ἐτελεύτησεν ὁ βασιλεὺς Αἰγύπτου, καὶ κατεστέναξαν οἱ υἱοὶ Ἰσραὴλ ἀπὸ τῶν ἔργων, καὶ ἀνεβόησαν· καὶ ἀνέβη ἡ βοὴ αὐτῶν πρὸς τὸν Θεὸν ἀπὸ τῶν ἔργων.
\vs{24}Καὶ εἰσήκουσεν ὁ Θεὸς τὸν στεναγμὸν αὐτῶν· καὶ ἐμνήσθη ὁ Θεὸς τῆς διαθήκης αὐτοῦ τῆς πρὸς Ἀβραὰμ, καὶ Ἰσαὰκ, καὶ Ἰακώβ.
\vs{25}Καὶ ἐπεῖδεν ὁ Θεὸς τοὺς υἱοὺς Ἰσραὴλ, καὶ ἐγνώσθη αὐτοῖς.

\ch{3}
Καὶ Μωυσῆς ἦν ποιμαίνων τὰ πρόβατα Ἰοθὸρ τοῦ γαμβροῦ αὐτοῦ, τοῦ ἱερέως Μαδιὰμ, καὶ ἤγαγεν τὰ πρόβατα ὑπὸ τὴν ἔρημον, καὶ ἦλθεν εἰς τὸ ὄρος Χωρήβ.
\vs{2}Ὤφθη δὲ αὐτῷ Ἄγγελος Κυρίου ἐν πυρὶ φλογὸς ἐκ τοῦ βάτου· καὶ ὁρᾷ ὅτι ὁ βάτος καίεται πυρί, ὁ δὲ βάτος οὐ κατεκαίετο.
\vs{3}Εἶπε δὲ Μωυσῆς, παρελθὼν ὄψομαι τὸ ὅραμα τὸ μέγα τοῦτο, ὅτι οὐ κατακαίεται ὁ βάτος.
\vs{4}Ὡς δὲ εἶδεν Κύριος ὅτι προσάγει ἰδεῖν, ἐκάλεσεν αὐτὸν Κύριος ἐκ τοῦ βάτου, λέγων, Μωυσῆ, Μωυσῆ· ὁ δὲ εἶπε, τί ἐστιν;
\vs{5}Ὁ δὲ εἶπε, μὴ ἐγγίσῃς ὧδε· λύσαι τὸ ὑπόδημα ἐκ τῶν ποδῶν σου, ὁ γὰρ τόπος. ἐν ᾧ σὺ ἕστηκας, γῆ ἁγία ἐστί.
\vs{6}Καὶ εἶπεν, ἐγώ εἰμι ὁ Θεὸς τοῦ πατρός σου, Θεὸς Ἁβραὰμ, καὶ Θεὸς Ἰσαὰκ, καὶ Θεὸς Ἰακώβ· ἀπέστρεψε δὲ Μωυσῆς τὸ πρόσωπον αὐτοῦ, εὐλαβεῖτο γὰρ κατεμβλέψαι ἐνώπιον τοῦ Θεοῦ.
\vs{7}Εἶπε δὲ Κύριος πρὸς Μωυσῆν, ἰδὼν εἶδον τὴν κάκωσιν τοῦ λαοῦ μου τοῦ ἐν Αἰγύπτῳ, καὶ τῆς κραυγῆς αὐτῶν ἀκήκοα ἀπὸ τῶν ἐργοδιωκτῶν· οἶδα γὰρ τὴν ὀδύνην αὐτων,
\vs{8}καὶ κατέβην ἐξελέσθαι αὐτοὺς ἐκ χειρὸς τῶν Αἰγυπτίων, καὶ ἐξαγαγεῖν αὐτοὺς ἐκ τῆς γῆς ἐκείνης, καὶ εἰσαγαγεῖν αὐτοὺς εἰς γῆν ἀγαθὴν καὶ πολλήν, εἰς γῆν ῥέουσαν γάλα καὶ μέλι, εἰς τὸν τόπον τῶν Χαναναίων, καὶ Χετταίων, καὶ Ἀμοῤῥαίων, καὶ Φερεζαίων, καὶ Γεργεσαὶων, καὶ Εὐαίων, καὶ Ἰεβουσαίων.
\vs{9}Καὶ νῦν ἰδοὺ κραυγὴ τῶν υἱῶν Ἰσραὴλ ἥκει πρὸς με· κᾀγὼ ἑώρακα τὸν θλιμμὸν, ὃν οἱ Αἰγύπτιοι θλίβουσιν αὐτούς.
\vs{10}Καὶ νῦν δεῦρο, ἀποστείλω σε πρὸς Φαραὼ βασιλέα Αἰγύπτου, καὶ ἐξάξεις τὸν λαόν μου τοὺς υἱοὺς Ἰσραὴλ ἐκ γῆς Αἰγύπτου.

\vs{11}Καὶ εἶπε Μωυσῆς πρὸς τὸν Θεὸν, τίς εἰμι ἐγὼ, ὅτι πορεύσομαι πρὸς Φαραὼ βασιλέα Αἰγύπτου, καὶ ὅτι ἐξάξω τοὺς υἱοὺς Ἰσραὴλ ἐκ γῆς Αἰγύπτου;
\vs{12}Εἶπε δὲ ὁ Θεὸς Μωυσῇ, λέγων, ὅτι ἔσομαι μετὰ σοῦ· καὶ τοῦτό σοι τὸ σημεῖον ὅτι ἐγώ σε ἐξαποστελῶ, ἐν τῷ ἐξαγαγεῖν σε τὸν λαόν μου ἐξ Αἰγύπτου, καὶ λατρεύσετε τῷ Θεῷ ἐν τῷ ὄρει τοῦτῳ.
\vs{13}Καὶ εἶπε Μωυσῆς πρὸς τὸν Θεὸν, ἰδοὺ ἐγὼ ἐξελεύσομαι πρὸς τοὺς υἱοὺς Ἰσραὴλ, καὶ ἐρῶ πρὸς αὐτοὺς, ὁ Θεὸς τῶν πατέρων ἡμῶν ἀπέσταλκέ με πρὸς ὑμᾶς· ἐρωτήσουσί με, τί ὄνομα αὐτῷ; τί ἐρῶ πρὸς αὐτούς;
\vs{14}Καὶ εἶπεν ὁ Θεὸς πρὸς Μωυσῆν, λέγων, ἐγώ εἰμι ὁ Ὤν· καὶ εἶπεν, οὕτως ἐρεῖς τοῖς υἱοῖς Ἰσραὴλ, ὁ Ὢν ἀπέσταλκέ με πρὸς ὑμᾶς.
\vs{15}Καὶ εἶπεν ὁ Θεὸς πάλιν πρὸς Μωυσῆν, οὕτως ἐρεῖς τοῖς υἱοῖς Ἰσραήλ, Κύριος ὁ Θεὸς τῶν πατέρων ἡμῶν, Θεὸς Ἀβραὰμ, καὶ Θεὸς Ἰσαὰκ, καὶ Θεὸς Ἰακὼβ, ἀπέσταλκέ με πρὸς ὑμᾶς· τοῦτό μου ἐστὶν ὄνομα αἰώνιον, καὶ μνημόσυνον γενεῶν γενεαῖς.
\vs{16}Ἐλθὼν οὐν συνάγαγε τὴν γερουσίαν τῶν υἱῶν Ἰσραὴλ, καὶ ἐρεῖς πρὸς αὐτοὺς, Κύριος ὁ Θεὸς τῶν πατέρων ἡμων ὦπταί μοι, Θεὸς Ἀβραὰμ, καὶ Θεὸς Ἰσαὰκ, καὶ Θεὸς Ἰακὼβ, λέγων, ἐπισκοπῇ ἐπέσκεμμαι ὑμᾶς, καὶ ὅσα συμβέβηκεν ὑμῖν ἐν Αἰγύπτῳ.
\vs{17}Καὶ εἶπεν, ἀναβιβάσω ὑμᾶς ἐκ τῆς κακώσεως τῶν Αἰγυπτίων, εἰς τὴν γῆν τῶν Χαναναίων, καὶ Χετταίων, καὶ Ἀμοῤῥαίων, καὶ Φερεζαίων, καὶ Γεργεσαίων, καὶ Εὑαίων, καὶ Ἰεβουσαίων, εἰς γῆν ῥέουσαν γάλα καὶ μέλι.
\vs{18}Καὶ εἰσακούσονταί σου τῆς φωνῆς· καὶ εἰσελεύσῃ σὺ, καὶ ἡ γερουσία Ἰσραὴλ, πρὸς Φαραὼ βασιλέα Αἰγύπτου, καὶ ἐρεῖς πρὸς αὐτὸν ὁ Θεὸς τῶν Ἑβραίων προσκέκληται ἡμᾶς· πορευσόμεθα οὖν ὁδὸν τριῶν ἡμερῶν εἰς τὴν ἔρημον, ἵνα θύσωμεν τῷ Θεῷ ἡμῶν.
\vs{19}Ἐγὼ δὲ οἶδα ὅτι οὐ προήσεται ὑμᾶς Φαραὼ βασιλεὺς Αἰγύπτου πορευθῆναι, ἐὰν μὴ μετὰ χειρὸς κραταιᾶς.
\vs{20}Καὶ ἐκτείνας τὴν χεῖρα, πατάξω τοὺς Αἰγυπτίους ἐν πᾶσι τοῖς θαυμασίοις μου, οἷς ποιήσω ἐν αὐτοῖς· καὶ μετὰ ταῦτα ἐξαποστελεῖ ὑμᾶς.
\vs{21}Καὶ δώσω χάριν τῷ λαῷ τούτῳ ἐναντίον τῶν Αἰγυπτίων· ὅταν δὲ ἀποτρέχητε, οὐκ ἀπελεύσεσθε κενοί·
\vs{22}Ἀλλὰ αἰτήσει γυνὴ παρὰ γείτονος καὶ συσκήνου αὐτῆς σκεύη ἀργυρᾶ, καὶ χρυσᾶ, καὶ ἱματισμόν· καὶ ἐπιθήσετε ἐπὶ τοὺς υἱοὺς ὑμῶν, καὶ ἐπὶ τὰς θυγατέρας ὑμῶν, καὶ σκυλεύσατε τοὺς Αἰγυπτίους.

\ch{4}
Ἀπεκρίθη δὲ Μωυσῆς, καὶ εἶπεν, ἐὰν μὴ πιστεύσωσί μοι, μηδὲ εἰσακούσωσι τῆς φωνῆς μου, ἐροῦσι γὰρ, ὅτι οὐκ ὦπταί σοι ὁ Θεὸς, τί ἐρῶ πρὸς αὐτούς;
\vs{2}Εἶπε δὲ αὐτῳ Κύριος, τί τοῦτό ἐστι τὸ ἐν τῇ χειρί σου; ὁ δὲ εἶπε, ῥάβδος.
\vs{3}Καὶ εἶπε, ῥίψον αὐτὴν ἐπὶ τὴν γῆν· καὶ ἔῤῥιψεν αὐτὴν ἐπὶ τὴν γῆν, καὶ ἐγένετο ὄφις· καὶ ἔφυγε Μωυσῆς ἀπʼ αὐτοῦ.
\vs{4}Καὶ εἶπε Κύριος πρὸς Μωυσῆν, ἔκτεινον τῆν χεῖρα, καὶ ἐπιλαβοῦ τῆς κέρκου· ἐκτείνας οὖν τὴν χεῖρα ἐπελάβετο τῆς κέρκου· καὶ ἐγένετο ῥάβδος ἐν τῇ χειρὶ αὐτοῦ.
\vs{5}Ἵνα πιστεύσωσί σοι, ὅτι ὦπταί σοι ὁ Θεὸς τῶν πατέρων αὐτῶν, Θεὸς Ἀβραὰμ, καὶ Θεὸς Ἰσαὰκ, καὶ Θεὸς Ἰακώβ.
\vs{6}Εἶπε δὲ αὐτῷ Κύριος πάλιν, εἰσένεγκον τὴν χεῖρά σου εἰς τὸν κόλπον σου· καὶ εἰσήνεγκε τὴν χεῖρα αὐτοῦ εἰς τὸν κόλπον αὐτοῦ· καὶ ἐξήνεγκεν τὴν χεῖρα αὐτοῦ ἐκ τοῦ κόλπου αὐτοῦ, καὶ ἐγενήθη ἡ χεὶρ αὐτοῦ ὡσεὶ χιών.
\vs{7}Καὶ εἶπεν πάλιν, εἰσένεγκον τὴν χεῖρά σου εἰς τὸν κόλπον σου· καὶ εἰσήνεγκε τὴν χεῖρα εἰς τὸν κόλπον αὐτοῦ· καὶ ἐξήνεγκεν αὐτὴν ἐκ τοῦ κόλπου αὐτοῦ, καὶ πάλιν ἀπεκατέστη εἰς τὴν χρόαν τῆς σαρκὸς αὐτῆς.
\vs{8}Ἐὰν δὲ μὴ πιστεύσωσί σοι, μηδὲ εἰσακούσωσι τῆς φωνῆς τοῦ σημείου τοῦ πρώτου, πιστεύσουσί σοι τῆς φωνῆς τοῦ σημείου τοῦ δετέρου.
\vs{9}Καὶ ἔσται ἐὰν μὴ πιστεύσωσί σοι τοῖς δυσὶ σημείοις τούτοις, μηδὲ εἰσακούσωσι τῆς φωνῆς σου, λήψῃ ἀπὸ τοῦ ὕδατος τοῦ ποταμοῦ, καὶ ἐκχεεῖς ἐπὶ τὸ ξηρόν· καὶ ἔσται τὸ ὕδωρ, ὃ ἐὰν λάβῃς ἀπὸ τοῦ ποταμοῦ, αἷμα ἐπὶ τοῦ ξηροῦ.
\vs{10}Εἶπε δὲ Μωυσῆς πρὸς Κύριον, δέομαι, Κύριε· οὐχ ἱκανός εἰμι πρὸ τῆς χθὲς οὐδὲ πρὸ τῆς τρίτης ἡμέρας, οὐδὲ ἀφʼ οὗ ἤρξω λαλεῖν τῷ θεράποντί σου· ἰσχνόφωνος καὶ βραδύγλωσσος ἐγώ εἰμι.
\vs{11}Εἶπε δὲ Κύριος πρὸς Μωυσῆν, τίς ἔδωκε στόμα ἀνθρώπῳ; καὶ τίς ἐποιήσε δύσκωφον καὶ κωφὸν, βλέποντα καὶ τυφλόν; οὐκ ἐγὼ ὁ Θεός;
\vs{12}Καὶ νῦν πορεύου, καὶ ἐγὼ ἀνοίξω τὸ στόμα σου, καὶ συμβιβάσω σε ὃ μέλλεις λαλῆσαι.
\vs{13}Καὶ εἶπε Μωυσῆς, δέομαι, Κύριε· προχείρισαι δυνάμενον ἄλλον, ὃν ἀποστελεῖς.
\vs{14}Καὶ θυμωθεὶς ὀργῇ Κύριος ἐπὶ Μωυσῆν, εἶπεν, οὐκ ἰδοὺ Ἀαρὼν ὁ ἄδελφός σου ὁ Λευίτης; ἐπίσταμαι ὅτι λαλῶν λαλήσει αὐτός σοι· καὶ ἰδοὺ αὐτὸς ἐξελεύσεται εἰς συνάντησίν σοι, καὶ ἰδών σε χαρήσεται ἐν ἑαυτῷ.
\vs{15}Καὶ ἐρεῖς πρὸς αὐτὸν, καὶ δώσεις τὰ ῥήματά μου εἰς τὸ στόμα αὐτοῦ, καὶ ἐγὼ ἀνοίξω τὸ στόμα σου καὶ τὸ στόμα αὐτοῦ, καί συμβιβάσω ὑμᾶς ἃ ποιήσετε.
\vs{16}Καὶ αὐτός σοι λαλήσει πρὸς τὸν λαὸν, καὶ αὐτὸς ἔσται σου στόμα· σὺ δὲ αὐτῷ ἔσῃ τὰ πρὸς τὸν Θεόν.
\vs{17}Καὶ τὴν ῥάβδον ταύτην, τὴν στραφεῖσαν εἰς ὄφιν, λήψῃ ἐν τῇ χειρί σου, ἐν ᾗ ποιήσεις ἐν αὐτῇ τὰ σημεῖα.

\vs{18}Ἐπορεύθη δὲ Μωυσῆς, καὶ ἀπέστρεψε πρὸς Ἰοθὸρ τὸν γαμβρὸν αὐτοῦ, καὶ λέγει, πορεύσομαι καὶ ἀποστρέψω πρὸς τοὺς ἀδελφούς μου τοὺς ἐν Αἰγύπτῳ, καὶ ὄψομαι εἰ ἔτι ζῶσι· καὶ εἶπεν Ἰοθὸρ Μωυσῇ, βάδιζε ὑγιαίνων· μετὰ δὲ τὰς ἡμέρας τὰς πολλὰς ἐκείνας ἐτελεύτησεν ὁ βασιλεὺς Αἰγύπτου.
\vs{19}Εἶπε δὲ Κύριος πρὸς Μωυσῆν ἐν Μαδιὰμ, βάδιζε, ἄπελθε εἰς Αἴγυπτον, τεθνήκασι γὰρ πάντες οἱ ζητοῦντες σου τὴν ψυχήν.
\vs{20}Ἀναλαβὼν δὲ Μωυσῆς τὴν γυναῖκα καὶ τὰ παιδία, ἀνεβίβασεν αὐτὰ ἐπὶ τὰ ὑποζύγια, καὶ ἐπέστρεψεν εἰς Αἴγυπτον· ἔλαβε δὲ Μωυσῆς τὴν ῥάβδον τὴν παρὰ τοῦ Θεοῦ ἐν τῇ χειρὶ αὐτοῦ.
\vs{21}Εἶπε δὲ Κύριος πρὸς Μωυσῆν, πορευομένου σου καὶ ἀποστρέφοντος εἰς Αἴγυπτον, ὅρα πάντα τὰ τέρατα ἃ δέδωκα ἐν ταῖς χερσί σου, ποιήσεις αὐτὰ ἐναντίον Φαραώ· ἐγὼ δὲ σκληρυνῶ τὴν καρδίαν αὐτοῦ, καὶ οὐ μὴ ἐξαποστείλῃ τὸν λαόν.
\vs{22}Σὺ δὲ ἐρεῖς τῷ Φαραῴ, τάδε λέγει Κύριος, υἱὸς πρωτότοκός μου Ἰσραήλ.
\vs{23}Εἶπα δέ σοι, ἐξαπόστειλον τὸν λαόν μου, ἵνα μοι λατρεύσῃ· εἰ μὲν οὖν μὴ βούλει ἐξαποστεῖλαι αὐτούς, ὅρα οὖν, ἐγὼ ἀποκτένῶ τὸν υἱόν σου τὸν πρωτότοκον.
\vs{24}Ἐγένετο δὲ ἐν τῇ ὁδῷ ἐν τῷ καταλύματι συνήντησεν αὐτῷ Ἄγγελος Κυρίου, καὶ ἐζήτει αὐτὸν ἀποκτεῖναι.
\vs{25}Καὶ λαβοῦσα Σεπφώρα ψῆφον, περιέτεμε τὴν ἀκροβυστίαν τοῦ υἱοῦ αὐτῆς· καὶ προσέπεσε πρὸς τοὺς πόδας αὐτοῦ, καὶ εἶπεν, ἔστη τὸ αἷμα τῆς περιτομῆς τοῦ παιδίου μου.
\vs{26}Καὶ ἀπῆλθεν ἀπʼ αὐτοῦ, διότι εἶπεν, ἔστη τὸ αἷμα τῆς περιτομῆς τοῦ παιδίου μου.
\vs{27}Εἶπε δὲ Κύριος πρὸς Ἀαρὼν, πορεύθητι εἰς συνάντησιν Μωυσῇ εἰς τὴν ἔρημον· καὶ ἐπορεύθη, καὶ συνήντησεν αὐτῷ ἐν τῷ ὄρει τοῦ Θεοῦ, καὶ κατεφίλησαν ἀλλήλους.
\vs{28}Καὶ ἀνήγγειλε Μωυσῆς τῷ Ἀαρὼν πάντας τοὺς λόγους Κυρίου, οὓς ἀπέστειλε, καὶ πάντα τὰ ῥήματα, ἃ ἐνετείλατο αὐτῷ.
\vs{29}Ἐπορεύθη δὲ Μωυσῆς καὶ Ἀαρὼν, καὶ συνήγαγον τὴν γερουσίαν τῶν υἱῶν Ἰσραήλ.
\vs{30}Καὶ ἐλάλησεν Ἀαρὼν πάντα τὰ ῥήματα ταῦτα, ἃ ἐλάλησεν ὁ Θεὸς πρὸς Μωυσῆν, καὶ ἐποίησε τὰ σημεῖα ἐναντίον τοῦ λαοῦ.
\vs{31}Καὶ ἐπίστευσεν ὁ λαὸς καὶ ἐχάρη, ὅτι ἐπεσκέψατο ὁ Θεὸς τοὺς υἱοὺς Ἰσραὴλ, καὶ ὅτι εἶδεν αὐτῶν τὴν θλίψιν· κύψας δὲ ὁ λαὸς προσεκύνησε.

\ch{5}
Καὶ μετὰ ταῦτα εἰσῆλθε Μωυσῆς καὶ Ἀαρὼν πρὸς Φαραὼ, καὶ εἶπαν αὐτῷ, τάδε λέγει Κύριος ὁ Θεὸς Ἰσραὴλ, ἐξαπόστειλον τὸν λαόν μου, ἵνα μοι ἑορτάσωσιν ἐν τῇ ἐρήμῳ.
\vs{2}Καὶ εἶπε Φαραὼ, τίς ἐστιν οὗ εἰσακούσομαι τῆς φωνῆς αὐτοῦ, ὥστε ἐξαποστεῖλαι τοὺς υἱοὺς Ἰσραήλ; οὐκ οἶδα τὸν Κύριον, καὶ τὸν Ἰσραὴλ οὐκ ἐξαποστέλλω.
\vs{3}Καὶ λέγουσιν αὐτῷ, ὁ Θεὸς τῶν Ἐβραίων προσκέκληται ἡμᾶς· πορευσόμεθα οὖν ὁδὸν τριῶν ἡμερῶν εἰς τὴν ἔρημον, ὅπως θύσωμεν Κυρίῳ τῷ Θεῷ ἡμῶν, μή ποτε συναντήσῃ ἡμῖν θάνατος ἢ φόνος.
\vs{4}Καὶ εἶπεν αὐτοῖς ὁ βασιλεὺς Αἰγύπτου, ἱνατί Μωυσῆς καὶ Ἀαρών διαστρέφετε τὸν λαὸν ἀπὸ τῶν ἔργων; ἀπέλθατε ἕκαστος ὑμῶν πρὸς τὰ ἔργα αὐτοῦ.
\vs{5}Καὶ εἶπεν Φαραὼ, ἰδοὺ νῦν πολυπληθεῖ ὁ λαὸς, μὴ οὖν καταπαύσωμεν αὐτοὺς ἀπὸ τῶν ἔργων.
\vs{6}Συνέταξε δὲ Φαραὼ τοῖς ἐργοδιώκταις τοῦ λαοῦ, καὶ τοῖς γραμματεῦσι, λέγων,
\vs{7}οὐκέτι προστεθήσεσθε διδόναι ἄχυρον τῷ λαῷ εἰς τὴν πλινθουργίαν, καθάπερ χθὲς καὶ τρίτην ἡμέραν· ἀλλ αὐτοὶ πορευέσθωσαν καὶ συναγαγέτωσαν ἑαυτοῖς ἄχυρα.
\vs{8}Καὶ τὴν σύνταξιν τῆς πλινθείας, ἧς αὐτοὶ ποιοῦσι, καθʼ ἑκάστην ἡμέραν ἐπιβαλεῖς αὐτοῖς· οὐκ ἀφελεῖς οὐδέν· σχολάζουσι γάρ· διὰ τοῦτο κεκράγασι, λέγοντες, ἐγερθῶμεν, καὶ θύσωμεν τῷ Θεῷ ἡμῶν.
\vs{9}Βαρυνέσθω τὰ ἔργα τῶν ἀνθρώπων τούτων, καὶ μεριμνάτωσαν ταῦτα, καὶ μὴ μεριμνάτωσαν ἐν λόγοις κενοῖς.

\vs{10}Κατέσπευδον δὲ αὐτοὺς οἱ ἐργοδιῶκται καὶ οἱ γραμματεῖς, καὶ ἔλεγον πρὸς τὸν λαὸν, λέγοντες, τάδε λέγει Φαραὼ, οὐκέτι δίδωμι ὑμῖν ἄχυρα.
\vs{11}Αὐτοὶ ὑμεῖς πορευόμενοι συλλέγετε ἑαυτοῖς ἄχυρα, ὅθεν ἐὰν εὕρητε· οὐ γὰρ ἀφαιρεῖται ἀπὸ τῆς συντάξεως ὑμῶν οὐθέν.
\vs{12}Καὶ διεσπάρη ὁ λαὸς ἐν ὅλῃ γῇ Αἰγύπτῳ συναγαγεῖν καλάμην εἰς ἄχυρα.
\vs{13}Οἱ δὲ ἐργοδιῶκται κατέσπευδον αὐτοὺς, λέγοντες, συντελεῖτε τὰ ἔργα τὰ καθήκοντα καθʼ ἡμέραν, καθάπερ καὶ ὅτε τὸ ἄχυρον ἐδίδοτο ὑμῖν.
\vs{14}Καὶ ἐμαστιγώθησαν οἱ γραμματεῖς τοῦ γένους τῶν υἱῶν Ἰσραὴλ, οἱ κατασταθέντες ἐπʼ αὐτοὺς, ὑπὸ τῶν ἐπιστατῶν τοῦ Φαραὼ, λέγοντες, διατί οὐ συνετελέσατε τὰς συντάξεις ὑμῶν τῆς πλινθείας καθάπερ χθὲς καὶ τρίτην ἡμέραν, καὶ τὸ τῆς σήμερον;
\vs{15}Εἰσελθόντες δὲ οἱ γραμματεῖς τῶν υἱῶν Ἰσραὴλ κατεβόησαν πρὸς Φαραὼ, λέγοντες, ἱνατί σὺ οὕτως ποιεῖς τοῖς σοῖς οἰκέταις;
\vs{16}Ἄχυρον οὐ δίδοται τοῖς οἰκέταις σου, καὶ τὴν πλίνθον ἡμῖν λέγουσι ποιεῖν· καὶ ἰδοὺ οἱ παῖδές σου μεμαστίγωνται, ἀδικήσεις οὖν τὸν λαόν σου.
\vs{17}Καὶ εἶπεν αὐτοῖς, σχολάζετε, σχολασταί ἐστε· διὰ τοῦτο λέγετε, πορευθῶμεν, θύσωμεν τῷ Θεῷ ἡμῶν.
\vs{18}Νῦν οὖν πορευθέντες, ἐργάζεσθε· τὸ γὰρ ἄχυρον οὐ δοθήσεται ὑμῖν, καὶ τὴν σύνταξιν τῆς πλινθείας ἀποδώσετε.
\vs{19}Ἑώρων δὲ οἱ γραμματεῖς τῶν υἱῶν Ἰσραὴλ ἑαυτοὺς ἐν κακοῖς, λέγοντες, οὐκ ἀπολείψετε τῆς πλινθείας τὸ καθῆκον τῇ ἡμέρᾳ.
\vs{20}Συνήντησαν δὲ Μωυσῇ καὶ Ἀαρὼν ἐρχομένοις εἰς συνάντησιν αὐτοῖς, ἐκπορευομένων αὐτῶν ἀπὸ Φαραώ,
\vs{21}Καὶ εἶπαν αὐτοῖς, ἴδοι ὁ Θεὸς ὑμᾶς καὶ κρίναι, ὅτι ἐβδελύξατε τὴν ὀσμὴν ἡμῶν ἐναντίον Φαραὼ, καὶ ἐναντίον τῶν θεραπόντων αὐτοῦ, δοῦναι ῥομφαίαν εἰς τὰς χεῖρας αὐτοῦ, ἀποκτεῖναι ἡμᾶς.
\vs{22}Ἐπέστρεψε δὲ Μωυσῆς πρὸς Κύριον, καὶ εἶπε, δέομαι, Κύριε· τί ἐκάκωσας τὸν λαὸν τοῦτον; καὶ ἱνατί ἀπέσταλκάς με;
\vs{23}Καὶ ἀφʼ οὗ πεπόρευμαι πρὸς Φαραὼ, λαλῆσαι ἐπὶ τῷ σῷ ὀνόματι, ἐκάκωσε τὸν λαὸν τοῦτον· καὶ οὐκ ἐῤῥύσω τὸν λαόν σου.

\ch{6}
Καὶ εἶπε Κύριος πρὸς Μωυσῆν, ἤδη ὄψει ἃ ποιήσω τῷ Φαραῷ· ἐν γὰρ χειρὶ κραταίᾳ ἐξαποστελεῖ αὐτούς, καὶ ἐν βραχίονι ὑψηλῷ ἐκβαλεῖ αὐτοὺς ἐκ τῆς γῆς αὐτοῦ.
\vs{2}Ἐλάλησε δὲ ὁ Θεὸς πρὸς Μωυσῆν, καὶ εἶπε πρὸς αὐτὸν, ἐγὼ Κύριος.
\vs{3}Καὶ ὤφθην πρὸς Ἀβραὰμ καὶ Ἰσαὰκ καὶ Ἰακὼβ, Θεὸς ὢν αὐτῶν· καὶ τὸ ὄνομά μου Κύριος οὐκ ἐδήλωσα αὐτοῖς.
\vs{4}Καὶ ἔστησα τὴν διαθήκην μου πρὸς αὐτοὺς, ὥστε δοῦναι αὐτοῖς τὴν γῆν τῶν Χαναναίων, τὴν γῆν ἣν παρῳκήκασιν, ἐν ᾗ καὶ παρῴκησαν ἐπʼ αὐτῆς.
\vs{5}Καὶ ἐγὼ εἰσήκουσα τὸν στεναγμὸν τῶν υἱῶν Ἰσραήλ, ὃν οἱ Αἰγύπτιοι καταδουλοῦνται αὐτούς, καὶ ἐμνήσθην τῆς διαθήκης ὑμῶν.
\vs{6}Βάδιζε, εἶπον τοῖς υἱοῖς Ἰσραὴλ, λέγων, ἐγὼ Κύριος· καὶ ἐξάξω ὑμᾶς ἀπὸ τῆς δυναστείας τῶν Αἰγυπτίων, καὶ ῥύσομαι ὑμᾶς ἐκ τῆς δουλείας, καὶ λυτρώσομαι ὑμᾶς ἐν βραχίονι ὑψηλῷ καὶ κρίσει μεγάλῃ·
\vs{7}Καὶ λὴψομαι ἐμαυτῷ ὑμᾶς λαὸν ἐμοὶ, καὶ ἔσομαι ὑμῶν Θεός· καὶ γνώσεσθε ὅτι ἐγὼ Κύριος ὁ Θεὸς ὑμῶν, ὁ ἐξαγαγὼν ὑμᾶς ἐκ τῆς καταδυναστείας τῶν Αἰγυπτίων.
\vs{8}Καὶ εἰσάξω ὑμᾶς εἰς τὴν γῆν, εἰς ἣν ἐξέτεινα τὴν χεῖρά μου, δοῦναι αὐτὴν τῷ Ἀβραὰμ, καὶ Ἰσαὰκ, καὶ Ἰακὼβ, καὶ δώσω ὑμῖν αὐτὴν ἐν κληρῷ· ἐγὼ Κύριος.
\vs{9}Ἐλάλησε δὲ Μωυσῆς οὕτω τοῖς υἱοῖς Ἰσραήλ· καὶ οὐκ εἰσήκουσαν Μωυσῇ ἀπὸ τῆς ὀλιγοψυχίας, καὶ ἀπὸ τῶν ἔργων τῶν σκληρῶν.
\vs{10}Εἶπε δὲ Κύριος πρὸς Μωυσῆν λέγων,
\vs{11}εἴσελθε, λάλησον Φαραῷ βασιλεῖ Αἰγύπτου, ἵνα ἐξαποστείλῃ τοὺς υἱοὺς Ἰσραὴλ ἐκ τῆς γῆς αὐτοῦ.
\vs{12}Ἐλάλησε δὲ Μωυσῆς ἔναντι Κυρίου, λέγων, ἰδοὺ οἱ υἱοὶ Ἰσραὴλ οὐκ εἰσήκουσάν μου, καὶ πῶς εἰσακούσεταί μου Φαραώ; ἐγὼ δὲ ἄλογός εἰμι.
\vs{13}Εἶπε δὲ Κύριος πρὸς Μωυσῆν καὶ Ἀαρὼν, καὶ συνέταξεν αὐτοῖς πρὸς Φαραὼ βασιλέα Αἰγύπτου, ὥστε ἐξαποστεῖλαι τοὺς υἱοὺς Ἰσραὴλ ἐκ γῆς Αἰγύπτου.

\vs{14}Καὶ οὗτοι ἀρχηγοὶ οἴκων πατριῶν αὐτῶν· υἱοὶ Ῥουβὴν, πρωτοτόκου Ἰσραήλ· Ἑνὼχ, καὶ Φαλλοὺς, Ἀσρὼν, καὶ Χαρμεί· αὕτη ἡ συγγένεια Ῥουβήν.
\vs{15}Καὶ υἱοὶ Συμεών· Ἰεμουὴλ, καὶ Ἰαμεὶμ, καὶ Ἀὼδ, καὶ Ἰαχεὶν, καὶ Σαὰρ, καὶ Σαοὺλ ὁ ἐκ τῆς Φοινίσσης· αὗται αἱ πατριαὶ τῶν υἱῶν Συμεών.
\vs{16}Καὶ ταῦτα τὰ ὀνόματα τῶν υἱῶν Λευὶ κατὰ συγγενείας αὐτῶν· Γεδσὼν, Καὰθ, καὶ Μεραρεί· καὶ τὰ ἔτη τῆς ζωῆς Λευὶ ἑκατὸν τριάκοντα ἑπτά.
\vs{17}Καὶ οὗτοι υἱοὶ Γεδσών· Λοβενεὶ, καὶ Σεμεεί· οἶκοι πατριᾶς αὐτῶν.
\vs{18}Καὶ υἱοὶ Καάθ· Ἀμβρὰμ, καὶ Ἰσσαάρ, Χεβρὼν, καὶ Ὀζειήλ· καὶ τὰ ἔτη τῆς ζωῆς Καὰθ ἑκατὸν τριάκοντα τρία ἔτη.
\vs{19}Καὶ υἱοὶ Μεραρεί· Μοολεὶ, καὶ Ὀμουσεί. οὗτοι οἱ οἶκοι πατριῶν Λευὶ κατὰ συγγενείας αὐτῶν.
\vs{20}Καὶ ἔλαβεν Ἀμβρὰν τὴν Ἰωχαβὲδ, θυγατέρα τοῦ ἀδελφοῦ τοῦ πατρὸς αὐτοῦ, ἑαυτῷ εἰς γυναῖκα· καὶ ἐγέννησεν αὐτῷ τόν τε Ἀαρὼν καὶ τὸν Μωυσῆν, καὶ Μαριὰμ τὴν ἀδελφὴν αὐτῶν· τὰ δὲ ἔτη τῆς ζωῆς Ἀμβρὰμ, ἑκατὸν τριάκοντα δύο ἔτη.
\vs{21}Καὶ υἱοὶ Ἰσσαάρ· Κορὲ, καὶ Ναφὲκ, καὶ Ζεχρεί.
\vs{22}Καὶ υἱοὶ Ὀζειήλ· Μισαὴλ, καὶ Ἐλισαφὰν, καὶ Σεγρεί.
\vs{23}Ἔλαβε δὲ Ἀαρὼν τὴν Ἐλισαβὲθ θυγατέρα Ἀμιναδὰβ, ἀδελφὴν Ναασσὼν, αὐτῷ γυναῖκα· καὶ ἔτεκεν αὐτῷ τόν τε Ναδὰβ, καὶ Ἀβιοὺδ, καὶ τὸν Ἐλεάζαρ, καὶ Ἰθάμαρ.
\vs{24}Υἱοὶ δὲ Κορέ· Ἀσεὶρ, καὶ Ἑλκανὰ, καὶ Ἀβιασάρ· αὗται αἱ γενέσεις Κορέ.
\vs{25}Καὶ Ἐλεάζαρ ὁ τοῦ Ἀαρὼν ἔλαβε τῶν θυγατέρων Φουτιὴλ αὐτῷ γυναῖκα· καὶ ἔτεκεν αὐτῷ τὸν Φινεές· αὗται αἱ ἀρχαὶ πατριᾶς Λευιτῶν, κατὰ γενέσεις αὐτῶν.
\vs{26}Οὗτος Ἀαρὼν καὶ Μωυσῆς, οἷς εἶπεν αὐτοῖς ὁ Θεὸς ἐξαγαγεῖν τοὺς υἱοὺς Ἰσραὴλ ἐκ γῆς Αἰγύπτου σὺν δυνάμει αὐτῶν.
\vs{27}Οὗτοί εἰσιν οἱ διαλεγόμενοι πρὸς Φαραὼ βασιλέα Αἰγύπτου· καὶ ἐξήγαγον τοὺς υἱοὺς Ἰσραὴλ ἐκ γῆς Αἰγύπτου αὐτὸς Ἀαρὼν καὶ Μωυσὴς,
\vs{28}ᾗ ἡμέρᾳ ἐλάλησε Κύριος Μωυσῇ ἐν γῇ Αἰγύπτῳ.
\vs{29}Καὶ ἐλάλησε Κύριος πρὸς Μωυσῆν, λέγων, ἐγὼ Κύριος· λάλησον πρὸς Φαραὼ βασιλέα Αἰγύπτου ὅσα ἐγὼ λέγω πρὸς σέ.
\vs{30}Καὶ εἶπε Μωυσῆς ἐναντίον Κυρίου, ἰδοὺ ἐγὼ ἰσχνόφωνός εἰμι, καὶ πῶς εἰσακούσεταί μου Φαραώ,

\ch{7}
Καὶ εἶπε Κύριος πρὸς Μωυσῆν, λέγων, ἰδοὺ δέδωκά σε θεὸν Φαραὼ, καὶ Ἀαρὼν ὁ ἀδελφός σου ἔσται σου προφήτης.
\vs{2}Σὺ δὲ λαλήσεις αὐτῷ πάντα ὅσα σοι ἐντέλλομαι· ὁ δὲ Ἀαρὼν ὁ ἀδελφός σου λαλήσει πρὸς Φαραὼ, ὥστε ἐξαποστεῖλαι τοὺς υἱοὺς Ἰσραὴλ ἐκ τῆς γῆς αὐτοῦ.
\vs{3}Ἐγὼ δὲ σκληρυνῶ τὴν καρδίαν Φαραὼ, καὶ πληθυνῶ τὰ σημεῖά μου καὶ τὰ τέρατα ἐν γῇ Αἰγύπτῳ.
\vs{4}Καὶ οὐκ εἰσακούσεται ὑμῶν Φαραώ· καὶ ἐπιβαλῶ τὴν χεῖρά μου ἐπʼ Αἴγυπτον, καὶ ἐξάξω σὺν δυνάμει μου τὸν λαόν μου τοὺς υἱοὺς Ἰσραὴλ ἐκ γῆς Αἰγύπτου σὺν ἐκδικήσει μεγάλῃ.
\vs{5}Καὶ γνώσονται πάντες οἱ Αἰγύπτιοι ὅτι ἐγώ εἰμι Κύριος, ἐκτείνων τὴν χεῖρά μου ἐπʼ Αἴγυπτον, καὶ ἐξάξω τοὺς υἱοὺς Ἰσραὴλ ἐκ μέσον αὐτῶν.
\vs{6}Ἐποίησε δὲ Μωυσῆς καὶ Ἀαρὼν καθάπερ ἐνετείλατο αὐτοῖς Κύριος, οὕτως ἐποίησαν.
\vs{7}Μωυσῆς δὲ ἦν ἐτῶν ὀγδοήκοντα, Ἀαρὼν δὲ ὁ ἀδελφὸς αὐτοῦ ἐτῶν ὀγδοήκοντατριῶν, ἡνίκα ἐλάλησεν πρὸς Φαραώ.
\vs{8}Καὶ εἶπε Κύριος πρὸς Μωυσῆν καὶ Ἀαρὼν, λέγων,
\vs{9}καὶ ἐὰν λαλήσῃ πρὸς ὑμᾶς Φαραὼ, λέγων, δότε ἡμῖν σημεῖον ἢ τέρας, καὶ ἐρεῖς Ἀαρὼν τῷ ἀδελφῷ σου, λάβε τὴν ῥάβδον, καὶ ῥίψον ἐπὶ τὴν γῆν ἐναντίον Φαραὼ, καὶ ἐναντίον τῶν θεραπόντων αὐτοῦ, καὶ ἔσται δράκων.
\vs{10}Εἰσῆλθε δὲ Μωυσῆς καὶ Ἀαρὼν ἐναντίον Φαραὼ, καὶ τῶν θεραπόντων αὐτοῦ· καὶ ἐποίησαν οὕτως, καθάπερ ἐνετείλατο αὐτοῖς Κύριος· καὶ ἔῤῥιψεν Ἀαρὼν τὴν ῥάβδον ἐναντίον Φαραὼ, καὶ ἐναντίον τῶν θεραποντων αὐτοῦ, καὶ ἐγένετο δράκων.
\vs{11}Συνεκάλεσε δὲ Φαραὼ τοὺς σοφιστὰς Αἰγύπτου, καὶ τοὺς φαρμακούς· καὶ ἐποίησαν καὶ οἱ ἐπαοιδοὶ τῶν Αἰγυπτίων ταῖς φαρμακίαις αὐτῶν ὡσαύτως.
\vs{12}Καὶ ἔῤῥιψαν ἔκαστος τὴν ῥάβδον αὐτῶν, καὶ ἐγένοντο δράκοντες· καὶ κατέπιεν ἡ ῥάβδος ἡ Ἀαρὼν τὰς ἐκείνων ῥάβδους.
\vs{13}Καὶ κατίσχυσεν ἡ καρδία Φαραὼ, καὶ οὐκ εἰσήκουσεν αὐτῶν, καθάπερ ἐνετείλατο αὐτοῖς Κύριος.

\vs{14}Εἶπε δὲ Κύριος πρὸς Μωυσῆν, βεβάρηται ἡ καρδία Φαραὼ, τοῦ μὴ ἐξαποστεῖλαι τὸν λαόν.
\vs{15}Βάδισον πρὸς Φαραὼ τὸ πρωΐ· ἰδοὺ αὐτὸς ἐκπορεύεται ἐπὶ τὸ ὕδωρ, καὶ ἔσῃ συναντῶν αὐτῷ ἐπὶ τὸ χεῖλος τοῦ ποταμοῦ· καὶ τὴν ῥάβδον τὴν στραφεῖσαν εἰς ὄφιν λήψῃ ἐν τῇ χειρί σου.
\vs{16}Καὶ ἐρεῖς πρὸς αὐτὸν, Κύριος ὁ Θεὸς τῶν Ἐβραίων ἀπέσταλκέ με πρὸς σὲ, λέγων, ἐξαπόστειλον τὸν λαόν μου, ἵνα μοι λατρεύσῃ ἐν τῇ ἐρήμῳ· καὶ ἰδοὺ οὐκ εἰσήκουσας ἕως τούτου.
\vs{17}Τάδε λέγει Κύριος, ἐν τούτῳ γνώσῃ ὅτι ἐγὼ Κύριος· ἰδοὺ ἑγὼ τύπτω τῇ ῥάβδῳ τῇ ἐν τῇ χειρί μου ἐπὶ τὸ ὕδωρ τὸ ἐν τῷ ποταμῷ, καὶ μεταβαλεῖ εἰς αἷμα.
\vs{18}Καὶ οἱ ἰχθύες οἱ ἐν τῷ ποταμῷ τελευτήσουσι· καὶ ἐποζέσει ὁ ποταμὸς, καὶ οὐ δυνήσονται οἱ Αἰγύπτιοι πιεῖν ὕδωρ ἀπὸ τοῦ ποταμοῦ.
\vs{19}Εἶπε δὲ Κύριος πρὸς Μωυσῆν, εἶπὸν Ἀαρὼν τῷ ἀδελφῷ σου, λάβε τὴν ῥάβδον σου ἐν τῇ χειρί σου, καὶ ἔκτεινον τὴν χεῖρά σου ἐπὶ τὰ ὕδατα Αἰγύπτου, καὶ ἐπὶ τοὺς ποταμοὺς αὐτῶν, καὶ ἐπὶ τὰς διώρυγας αὐτῶν, καὶ ἐπὶ τὰ ἕλη αὐτῶν, καὶ ἐπὶ πᾶν συνεστηκὸς ὕδωρ αὐτῶν, καὶ ἔσται αἷμα· καὶ ἐγένετο αἷμα ἐν πάσῃ γῇ Αἰγύπτου, ἔν τε τοῖς ξύλοις καὶ ἐν τοῖς λίθοις.
\vs{20}Καὶ ἐποίησαν οὕτως Μωυσῆς καὶ Ἀαρὼν, καθάπερ ἐνετείλατο αὐτοῖς Κύριος· καὶ ἐπάρας τῇ ῥάβδῳ αὐτοῦ ἐπάταξε τὸ ὕδωρ τὸ ἐν τῷ ποταμῷ ἐναντίον Φαραὼ, καὶ ἐναντίον τῶν θεραπόντων αὐτοῦ· καὶ μετέβαλε πᾶν τὸ ὕδωρ τὸ ἐν τῷ ποταμῷ εἰς αἷμα.
\vs{21}Καὶ οἱ ἰχθύες οἱ ἐν τῷ ποταμῷ ἐτελεύτησαν· καὶ ἐπώζεσεν ὁ ποταμὸς, καὶ οὐκ ἠδύναντο οἱ Αἰγύπτιοι πιεῖν ὕδωρ ἐκ τοῦ ποταμοῦ· καὶ ἦν τὸ αἷμα ἐν πάσῃ γῇ Αἰγύπτου.
\vs{22}Ἐποίησαν δὲ ὡσαύτως καὶ οἱ ἐπαοιδοὶ τῶν Αἰγυπτίων ταῖς φαρμακίαις αὐτῶν· καὶ ἐσκληρύνθη ἡ καρδία Φαραὼ, καὶ οὐκ εἰσήκουσεν αὐτῶν, καθάπερ εἶπε Κύριος.
\vs{23}Ἐπιστραφεὶς δὲ Φαραὼ εἰσῆλθεν εἰς τὸν οἶκον αὐτοῦ· καὶ οὐκ ἐπέστησε τὸν νοῦν αὐτοῦ οὐδὲ ἐπὶ τούτῳ.
\vs{24}Ὤρυξαν δὲ πάντες οἱ Αἰγύπτιοι κύκλῳ τοῦ ποταμοῦ, ὥστε πιεῖν ὕδωρ· καὶ οὐκ ἠδύναντο πιεῖν ὕδωρ ἀπὸ τοῦ ποταμοῦ.
\vs{25}Καὶ ἀνεπληρώθησαν ἑπτὰ ἡμέραι, μετὰ τὸ πατάξαι Κύριον τὸν ποταμόν.

\vs{26}Εἶπε δὲ Κύριος πρὸς Μωυσὴν, εἴσελθε πρὸς Φαραὼ, καὶ ἐρεῖς πρὸς αὐτὸν, τάδε λεγέι Κύριος, ἐξαπόστειλον τὸν λαόν μου, ἵνα μοι λατρεύσωσιν.
\vs{27}Εἰ δὲ μὴ βούλει σὺ ἐξαποστεῖλαι, ἰδοὺ ἐγὼ τύπτω πάντα τὰ ὅριά σου τοῖς βατράχοις.
\vs{28}Καὶ ἐξερεύξεται ὁ ποταμὸς βατράχους· καὶ ἀναβάντες εἰσελεύσονται εἰς τοὺς οἴκους σου, καὶ εἰς τὰ ταμιεῖα τῶν κοιτώνων σου, καὶ ἐπὶ τῶν κλινῶν σου, καὶ ἐπὶ τοὺς οἴκους τῶν θεραπόντων σου, καὶ τοῦ λαοῦ σου, καὶ ἐν τοῖς φυράμασί σου, καὶ ἐν τοῖς κλιβάνοις σου.
\vs{29}Καὶ ἐπὶ σὲ, καὶ ἐπὶ τοὺς θεράποντάς σου, καὶ ἐπὶ τὸν λαόν σου, ἀναβήσονται οἱ βάτραχοι.

\ch{8}Εἶπε δὲ Κύριος πρὸς Μωυσῆν, εἶπον Ἀαρὼν τῷ ἀδελφῷ σου, ἔκτεινον τῇ χειρὶ τὴν ῥάβδον σου ἐπὶ τοὺς ποταμοὺς, καὶ ἐπὶ τὰς διώρυγας, καὶ ἐπὶ τὰ ἕλη, καὶ ἀνάγαγε τοὺς βατράχους.
\vs{2}Καὶ ἐξέτεινεν Ἀαρὼν τὴν χεῖρα ἐπὶ τὰ ὕδατα Αἰγύπτου, καὶ ἀνήγαγε τοὺς βατράχους· καὶ ἀνεβιβάσθη ὁ βάτραχος, καὶ ἐκάλυψε τὴν γῆν Αἰγύπτου.
\vs{3}Ἐποίησαν δὲ ὡσαύτως καὶ οἱ ἐπαοιδοὶ τῶν Αἰγυπτίων ταῖς φαρμακίαις αὐτῶν, καὶ ἀνήγαγον τοὺς βατράχους ἐπὶ γῆν Αἰγύπτου.
\vs{4}Καὶ ἐκάλεσε Φαραὼ Μωυσῆν καὶ Ἀαρὼν, καὶ εἶπεν, εὔξασθε περὶ ἐμοῦ πρὸς Κύριον, καὶ περιελέτω τοὺς βατράχους ἀπʼ ἐμοῦ, καὶ ἀπὸ τοῦ ἐμοῦ λαοῦ· καὶ ἐξαποστελῶ αὐτοὺς, καὶ θύσωσι τῷ Κυρίῳ.
\vs{5}Εἶπε δὲ Μωυσῆς πρὸς Φαραὼ, τάξαι πρὸς με πότε εὔξομαι περὶ σοῦ, καὶ περὶ τῶν θεραπόντων σου, καὶ τοῦ λαοῦ σου, ἀφανίσαι τοὺς βατράχους ἀπὸ σοῦ, καὶ ἀπὸ τοῦ λαοῦ σου, καὶ ἐκ τῶν οἰκιῶν ὑμῶν, πλὴν ἐν τῷ ποταμῷ ὑπολειφθήσονται.
\vs{6}Ὁ δὲ εἶπεν, εἰς αὔριον· εἶπεν οὖν, ὡς εἴρηκας· ἵνα εἰδῇς ὅτι οὐκ ἔστιν ἄλλος πλὴν Κυρίου·
\vs{7}Καὶ περιαιρεθήσονται οἱ βάτραχοι ἀπὸ σοῦ, καὶ ἀπὸ τῶν οἰκιῶν ὑμῶν, καὶ ἀπὸ τῶν ἐπαύλεων, καὶ ἀπὸ τῶν θεραπόντων σου, καὶ ἀπὸ τοῦ λαοῦ σου, πλὴν ἐν τῷ ποταμῷ ὑπολειφθήσονται.
\vs{8}Ἐξῆλθε δὲ Μωυσῆς καὶ Ἀαρὼν ἀπὸ Φαραώ· καὶ ἐβόησε Μωυσῆς πρὸς Κύριον περὶ τοῦ ὁρισμοῦ τῶν βατράχων, ὡς ἐτάξατο Φαραώ.
\vs{9}Ἐποιήσε δὲ Κύριος καθάπερ εἶπε Μωυσῆς· καὶ ἐτελεύτησαν οἱ βάτραχοι ἐκ τῶν οἰκιῶν, καὶ ἐκ τῶν ἐπαύλεων, καὶ ἐκ τῶν ἀγρῶν.
\vs{10}Καὶ συνήγαγον αὐτοὺς, θημωνίας θημωνίας· καὶ ὤζεσεν ἡ γῆ.
\vs{11}Ἰδὼν δὲ Φαραὼ ὅτι γέγονεν ἀνάψυξις, ἐβαρύνθη ἡ καρδία αὐτοῦ, καὶ οὐκ εἰσήκουσεν αὐτῶν, καθάπερ ἐλάλησε Κύριος.
\vs{12}Εἶπε δὲ Κύριος πρὸς Μωυσῆν, εἶπον Ἀαρὼν, ἔκτεινον τῇ χειρὶ τὴν ῥάβδον σου, καὶ πάταξον τὸ χῶμα τῆς γῆς· καὶ ἔσονται σκνίφες ἔν τε τοῖς ἀνθρώποις, καὶ ἐν τοῖς τετράποσι, καὶ ἐν πάσῃ γῇ Αἰγύπτου.
\vs{13}Ἐξέτεινεν οὖν Ἀαρὼν τῇ χειρὶ τὴν ῥάβδον, καὶ ἐπάταξε τὸ χῶμα τῆς γῆς· καὶ ἐγένοντο οἱ σκνίφες ἐν τοῖς ἀνθρώποις, ἔν τε τοῖς τετράποσι, καὶ ἐν παντὶ χώματι τῆς γῆς ἐγένοντο οἱ σκνίφες.
\vs{14}Ἐποίησαν δὲ ὡσαύτως καὶ οἱ ἐπαοιδοὶ ταῖς φαρμακίαις αὐτῶν, ἐξαγαγεῖν τὸν σκνῖφα, καὶ οὐκ ἠδύναντο· καὶ ἐγένοντο οἱ σκνίφες ἔν τε τοῖς ἀνθρώποις, καὶ ἐν τοῖς τετράποσιν.
\vs{15}Εἶπαν οὖν οἱ ἐπαοιδοὶ τῷ Φαραῷ, δάκτυλος Θεοῦ ἐστι τοῦτο· καὶ ἐσκληρύνθη ἡ καρδία Φαραὼ, καὶ οὐκ εἰσήκουσεν αὐτῶν, καθάπερ ἐλάλησε Κύριος.
\vs{16}Εἶπε δὲ Κύριος πρὸς Μωυσῆν, ὄρθρισον τὸ πρωΐ, καὶ στῆθι ἐναντίον Φαραώ· καὶ ἰδοὺ αὐτὸς ἐξελεύσεται ἐπὶ τὸ ὕδωρ· καὶ ἐρεῖς πρὸς αὐτὸν, τάδε λέγει Κύριος, ἐξαπόστειλον τὸν λαόν μου, ἵνα μοι λατρεύσωσιν ἐν τῇ ἐρήμῳ.
\vs{17}Ἐὰν δὲ μὴ βούλει ἐξαποστεῖλαι τὸν λαόν μου, ἰδοὺ ἐγὼ ἐξαποστέλλω ἐπὶ σὲ, καὶ ἐπὶ τοὺς θεράποντάς σου, καὶ ἐπὶ τὸν λαόν σου, καὶ ἐπὶ τοὺς οἴκους ὑμῶν, κυνόμυιαν· καὶ πλησθήσονται αἱ οἰκίαι τῶν Αἰγυπτίων τῆς κυνομυίης, καὶ εἰς τὴν γῆν ἐφʼ ἧς εἰσιν ἐπʼ αὐτῆς.
\vs{18}Καὶ παραδοξάσω ἐν τῇ ἡμέρᾳ ἐκείνῃ τὴν γῆν Γεσὲμ, ἐφʼ ἧς ὁ λαός μου ἔπεστιν ἐπʼ αὐτῆς, ἐφʼ ἧς οὐκ ἔσται ἐκεῖ ἡ κυνόμυια· ἵνα εἰδῇς ὅτι ἐγώ εἰμι Κύριος ὁ Θεὸς πάσης τῆς γῆς.
\vs{19}Καὶ δώσω διαστολὴν ἀνὰ μέσον τοῦ ἐμοῦ λαοῦ, καὶ ἀνὰ μέσον τοῦ σου λαοῦ· ἐν δὲ τῇ αὔριον ἔσται τοῦτο ἐπὶ τῆς γῆς.
\vs{20}Ἐποίησε δὲ Κύριος οὕτως· καὶ παρεγένετο ἡ κυνόμυια πλῆθος εἰς τοὺς οἴκους Φαραὼ, καὶ εἰς τοὺς οἴκους τῶν θεραπόντων αὐτοῦ, καὶ εἰς πᾶσαν τὴν γῆν Αἰγύπτου· καὶ ἐξωλοθρεύθη ἡ γῆ ἀπὸ τῆς κυνομυίης.

\vs{21}Ἐκάλεσε δὲ Φαραὼ Μωυσῆν καὶ Ἀαρὼν, λέγων, ἐλθόντες θύσατε Κυρίῳ τῷ Θεῷ ὑμῶν ἐν τῇ γῇ.
\vs{22}Καὶ εἶπε Μωυσῆς, οὐ δυνατὸν γενέσθαι οὕτως· τὰ γὰρ βδελύγματα τῶν Αἰγυπτίων θύσομεν Κυρίῳ τῷ Θεῷ ἡμῶν· ἐὰν γὰρ θύσωμεν τὰ βδελύγματα τῶν Αἰγυπτίων ἐναντίον αὐτῶν, λιθοβοληθησόμεθα.
\vs{23}Ὁδὸν τριῶν ἡμερῶν πορευσόμεθα εἰς τὴν ἔρημον· καὶ θύσομεν τῷ Θεῷ ἡμῶν, καθάπερ εἶπεν Κύριος ἡμῖν.
\vs{24}Καὶ εἶπε Φαραὼ, ἐγὼ ἀποστέλλω ὑμᾶς, καὶ θύσατε τῷ Θεῷ ὑμῶν ἐν τῇ ἐρήμῳ· ἀλλʼ οὐ μακρὰν ἀποτενεῖτε πορευθῆναι· εὔξασθε οὖν περὶ ἐμοῦ πρὸς Κύριον.
\vs{25}Εἶπε δὲ Μωυσῆς, ὁ δὲ ἐγὼ ἐξελεύσομαι ἀπὸ σοῦ, καὶ εὔξομαι πρὸς τὸν Θεὸν, καὶ ἀπελεύσεται ἡ κυνόμυια καὶ ἀπὸ τῶν θεραπόντων σου, καὶ ἀπὸ τοῦ λαοῦ σου αὔριον· μὴ προσθῇς ἔτι Φαραὼ ἐξαπατῆσαι, τοῦ μὴ ἐξαποστεῖλαι τὸν λαὸν θῦσαι Κυρίῳ.
\vs{26}Ἐξῆλθε δὲ Μωυσῆς ἀπὸ Φαραὼ, καὶ ηὔξατο πρὸς τὸν Θεόν.
\vs{27}Ἐποίησε δὲ Κύριος καθάπερ εἶπε Μωυσῆς· καὶ περιεῖλε τὴν κυνόμυιαν ἀπὸ Φαραὼ, καὶ τῶν θεραπόντων αὐτοῦ, καὶ τοῦ λαοῦ αὐτοῦ, καὶ οὐ κατελείφθη οὐδεμία.
\vs{28}Καὶ ἐβάρυνε Φαραὼ τὴν καρδίαν αὐτοῦ καὶ ἐπὶ τοῦ καιροῦ τούτου, καὶ οὐκ ἠθέλησεν ἐξαποστεῖλαι τὸν λαόν.

\ch{9}
Εἶπε δὲ Κύριος πρὸς Μωυσῆν, εἴσελθε πρὸς Φαραὼ, καὶ ἐρεῖς αὐτῷ, τάδε λέγει Κύριος ὁ Θεὸς τῶν Ἑβραίων, ἐξαπόστειλον τὸν λαόν μου, ἵνα μοι λατρεύσωσι.
\vs{2}Εἰ μὲν οὖν μὴ βούλει ἐξαποστεῖλαι τὸν λαόν μου, ἀλλὰ ἔτι ἐγκρατεῖς αὐτοῦ,
\vs{3}Ἰδοὺ, χεὶρ Κυρίου ἐπέσται ἐν τοῖς κτήνεσί σου τοῖς ἐν τοῖς πεδίοις, ἔν τε τοῖς ἵπποις, καὶ ἐν τοῖς ὑποζυγίοις, καὶ ταῖς καμήλοις, καὶ βουσὶ, καὶ προβάτοις, θάνατος μέγας σφόδρα.
\vs{4}Καὶ παραδοξάσω ἐγὼ ἐν τῷ καιρῷ ἐκείνῳ ἀνὰ μέσον τῶν κτηνῶν τῶν Αἰγυπτίων, καὶ ἀνὰ μέσον τῶν κτηνῶν τῶν υἱῶν Ἰσραήλ· οὐ τελευτήσει ἀπὸ πάντων τῶν τοῦ Ἰσραὴλ υἱῶν ῥητόν.
\vs{5}Καὶ ἔδωκεν ὁ Θεὸς ὅρον, λέγων, ἐν τῇ αὔριον ποιήσει Κύριος τὸ ῥῆμα τοῦτο ἐπὶ τῆς γῆς.
\vs{6}Καὶ ἐποίησε Κύριος τὸ ῥῆμα τοῦτο τῇ ἐπαύριον· καὶ ἐτελεύτησε πάντα τὰ κτήνη τῶν Αἰγυπτίων· ἀπὸ δὲ τῶν κτηνῶν τῶν υἱῶν Ἰσραὴλ οὐκ ἐτελεύτησεν οὐδέν.
\vs{7}Ἰδὼν δὲ Φαραὼ ὅτι οὐκ ἐτελεύτησεν ἀπὸ πάντων τῶν κτηνῶν τῶν υἱῶν Ἰσραὴλ οὐδὲν, ἐβαρύνθη ἡ καρδία Φαραὼ, καὶ οὐκ ἐξαπέστειλε τὸν λαόν.
\vs{8}Εἶπε δὲ Κύριος πρὸς Μωυσῆν καὶ Ἀαρὼν, λέγων, λάβετε ὑμεῖς πληρεῖς τὰς χεῖρας αἰθάλης καμιναίας, καὶ πασάτω Μωυσῆς εἰς τὸν οὐρανὸν ἐναντίον Φαραὼ, καὶ ἐναντίον τῶν θεραπόντων αὐτοῦ.
\vs{9}Καὶ γενηθήτω κονιορτὸς ἐπὶ πᾶσαν τὴν γῆν Αἰγύπτου· καὶ ἔσται ἐπὶ τοὺς ἀνθρώπους, καὶ ἐπὶ τὰ τετράποδα, ἕλκη, φλυκτίδες ἀναζέουσαι ἔν τε τοῖς ἀνθρώποις, καὶ ἐν τοῖς τετράποσιν, ἐν πάσῃ γῇ Αἰγύπτου.
\vs{10}Καὶ ἔλαβεν τὴν αἰθάλην τῆς καμιναίας ἐναντίον Φαραὼ, καὶ ἔπασεν αὐτὴν Μωυσῆς εἰς τὸν οὐρανόν· καὶ ἐγένετο ἕλκη, φλυκτίδες ἀναζέουσαι, ἔν τε τοῖς ἀνθρώποις, καὶ ἐν τοῖς τετράποσι.
\vs{11}Καὶ οὐκ ἠδύναντο οἱ φαρμακοὶ στῆναι ἐναντίον Μωυσῆ διὰ τὰ ἕλκη· ἐγένετο γὰρ τὰ ἕλκη ἐν τοῖς φαρμακοῖς, καὶ ἐν πάσῃ γῇ Αἰγύπτου.
\vs{12}Ἐσκλήρυνε δὲ Κύριος τὴν καρδίαν Φαραὼ, καὶ οὐκ εἰσήκουσεν αὐτῶν, καθὰ συνέταξε Κύριος.
\vs{13}Εἶπε δὲ Κύριος πρὸς Μωυσῆν, ὄρθρισον τὸ πρωῒ, καὶ στῆθι ἐναντίον Φαραὼ, καὶ ἐρεῖς πρὸς αὐτὸν, τάδε λέγει Κύριος ὁ Θεὸς τῶν Ἑβραίων, ἐξαπόστειλον τὸν λαόν μου, ἵνα λατρεύσωσί μοι.
\vs{14}Ἐν τῷ γὰρ νῦν καιρῷ ἐγὼ ἐξαποστέλλω πάντα τὰ συναντήματά μου εἰς τὴν καρδίαν σου, καὶ τῶν θεραπόντων σου, καὶ τοῦ λαοῦ σου, ἵνα εἴδῇς ὅτι οὐκ ἔστιν, ὡς ἐγὼ, ἄλλος ἐν πάσῃ τῇ γῇ.
\vs{15}Νῦν γὰρ ἀποστείλας τὴν χεῖρα πατάξω σε, καὶ τὸν λαόν σου θανατώσω, καὶ ἐκτριβήσῃ ἀπὸ τῆς γῆς.
\vs{16}Καὶ ἕνεκεν τούτου διετηρήθης, ἵνα ἐνδείξωμαι ἐν σοὶ τὴν ἰσχύν μου, καὶ ὅπως διαγγελῇ τὸ ὄνομά μου ἐν πάσῃ τῇ γῇ.
\vs{17}Ἔτι οὖν σὺ ἐνποιῇ τοῦ λαοῦ μου, τοῦ μὴ ἐξαποστεῖλαι αὐτούς;
\vs{18}Ἰδοὺ ἐγὼ ὕω ταύτην τὴν ὥραν αὔριον χάλαζαν πολλὴν σφόδρα, ἥτις τοιαύτη οὐ γέγονεν ἐν Αἰγύπτῳ, ἀφʼ ἧς ἡμέρας ἔκτισται, ἕως τῆς ἡμέρας ταύτης.
\vs{19}Νῦν οὖν κατάσπευσον συναγαγεῖν τὰ κτήνη σου, καὶ ὅσα σοι ἐστὶν ἐν τῷ πεδίῳ· πάντες γὰρ οἱ ἄνθρωποι, καὶ τὰ κτήνη, ὅσα σοί ἐστιν ἐν τῷ πεδίῳ· πὰντες γὰρ οἱ ἄνθρωποι, καὶ τὰ κτήνη, ὅσα ἐὰν εὑρεθῇ ἐν τοῖς πεδίοις, καὶ μὴ εἰσέλθῃ εἰς οἰκίαν, πεσῇ δὲ ἐπʼ αὐτὰ ἡ χάλαζα, τελευτήσει.
\vs{20}Ὁ φοβούμενος τὸ ῥῆμα Κυρίου τῶν θεραπόντων Φαραὼ, συνήγαγε τὰ κτήνη αὐτοῦ εἰς τοὺς οἴκους.
\vs{21}Ὃς δὲ μὴ πρόσεσχεν τῇ διανοίᾳ εἰς τὸ ῥῆμα Κυρίου, ἀφῆκε τὰ κτήνη ἐν τοῖς πεδίοις.

\vs{22}Εἶπε δὲ Κύριος πρὸς Μωυσῆν, ἔκτεινον τὴν χεῖρά σου εἰς τὸν οὐρανὸν, καὶ ἔσται χάλαζα ἐπὶ πᾶσαν γῆν Αἰγύπτου, ἐπί τε τοὺς ἀνθρώπους, καὶ τὰ κτήνη, καὶ ἐπὶ πᾶσαν βοτάνην τὴν ἐπὶ τῆς γῆς.
\vs{23}Ἐξέτεινε δὲ Μωυσῆς τὴν χεῖρα εἰς τὸν οὐρανὸν, καὶ Κύριος ἔδωκε φωνὰς καὶ χάλαζαν· καὶ διέτρεχε τὸ πῦρ ἐπὶ τῆς γῆς· καὶ ἔβρεξε Κύριος χάλαζαν ἐπὶ πᾶσαν γῆν Αἰγύπτου.
\vs{24}Ἦν δὲ ἡ χάλαζα καὶ τὸ πῦρ φλογίζον ἐν τῇ χαλάζῃ· ἡ δὲ χάλαζα πολλὴ σφόδρα, ἥτις τοιαύτη οὐ γέγονεν ἐν Αἰγύπτῳ, ἀφʼ ἧς ἡμέρας γεγένηται ἐπʼ αὐτῆς ἔθνος.
\vs{25}Ἐπάταξε δὲ ἡ χάλαζα ἐν πάσῃ γῇ Αἰγύπτου, ἀπὸ ἀνθρώπου ἕως κτήνους· καὶ πᾶσαν βοτάνην τὴν ἐν τῷ πεδίῳ ἐπάταξεν ἡ χάλαζα· καὶ πάντα τὰ ξύλα τὰ ἐν τοῖς πεδίοις συνέτριψεν ἡ χάλαζα.
\vs{26}Πλὴν ἐν γῇ Γεσὲμ, οὗ ἦσαν οἱ υἱοὶ Ἰσραὴλ, οὐκ ἐγένετο ἡ χάλαζα.
\vs{27}Ἀποστείλας δὲ Φαραὼ ἐκάλεσε Μωυσῆν καὶ Ἀαρὼν, καὶ εἶπεν αὐτοῖς, ἡμάρτηκα τὸ νῦν· ὁ Κύριος δίκαιος, ἐγὼ δὲ καὶ ὁ λαός μου ἀσεβεῖς.
\vs{28}Εὔξασθε οὖν περὶ ἐμοῦ πρὸς Κύριον, καὶ παυσάσθω τοῦ γενηθῆναι φωνὰς Θεοῦ, καὶ χάλαζαν, καὶ πῦρ· καὶ ἐξαποστελῶ ὑμᾶς, καὶ οὐκέτι προστεθήσεσθε μένειν.
\vs{29}Εἶπε δὲ αὐτῷ Μωυσῆς, ὡς ἂν ἐξέλθω τὴν πόλιν, ἐκπετάσω τὰς χεῖράς μου πρὸς τὸν Κύριον, καὶ αἱ φωναὶ παύσονται, καὶ ἡ χὰλαζα καὶ ὁ ὑετὸς οὐκ ἔσται ἔτι, ἵνα γνῷς ὅτι τοῦ Κυρίου ἡ γῆ.
\vs{30}Καὶ σὺ καὶ οἱ θεράποντές σου, ἐπίσταμαι ὅτι οὐδέπω πεφόβησθε τὸν Κύριον.
\vs{31}Τὸ δὲ λίνον καὶ ἡ κριθὴ ἐπλήγη· ἡ γὰρ κριθὴ παρεστηκυῖα, τὸ δὲ λίνον σπερματίζον.
\vs{32}Ὁ δὲ πυρὸς καὶ ἡ ὀλύρα οὐκ ἐπληγησαν, ὄψιμα γὰρ ἦν.
\vs{33}Ἐξῆλθε δὲ Μωυσῆς ἀπὸ Φαραὼ ἐκτὸς τῆς πόλεως, καὶ ἐξέτεινε τὰς χεῖρας πρὸς Κύριον· καὶ αἱ φωναὶ ἐπαύσαντο, καὶ ἡ χάλαζα καὶ ὁ ὑετὸς οὐκ ἔσταξεν ἔτι ἐπὶ τὴν γῆν.
\vs{34}Ἰδὼν δὲ Φαραὼ ὅτι πέπαυται ὁ ὑετὸς καὶ ἡ χάλαζα καὶ αἱ φωναὶ, προσέθετο τοῦ ἁμαρτάνειν· καὶ ἐβάρυνεν αὐτοῦ τὴν καρδίαν, καὶ τῶν θεραπόντων αὐτοῦ.
\vs{35}Καὶ ἐσκληρύνθη ἡ καρδία Φαραὼ, καὶ οὐκ ἐξαπέστειλε τοὺς υἱοὺς Ἰσραὴλ, καθάπερ ἐλάλησε Κύριος τῷ Μωυσῇ.

\ch{10}
Εἶπε δὲ Κύριος πρὸς Μωυσῆν, λέγων, εἴσελθε πρὸς Φαραὼ, ἐγὼ γὰρ ἐσκλήρυνα αὐτοῦ τὴν καρδίαν καὶ τῶν θεραπόντων αὐτοῦ, ἵνα ἑξῆς ἐπέλθῃ τὰ σημεῖα ταῦτα ἐπʼ αὐτούς·
\vs{2}ὅπως διηγήσησθε εἰς τὰ ὦτα τῶν τέκνων ὑμῶν, καὶ τοῖς τέκνοις τῶν τέκνων ὑμῶν, ὅσα ἐμπέπαιχα τοῖς Αἰγυπτίοις, καὶ τὰ σημεῖά μου, ἃ ἐποίησα ἐν αὐτοῖς· καὶ γνώσεσθε ὅτι ἐγὼ Κύριος.
\vs{3}Εἰσῆλθε δὲ Μωυσῆς καὶ Ἀαρὼν ἐναντίον Φαραὼ, καὶ εἶπαν αὐτῷ, τάδε λέγει Κύριος ὁ Θεὸς τῶν Ἐβραίων, ἕως τίνος οὐ βούλει ἐντραπῆναί με; ἔξαπόστειλον τὸν λαόν μου, ἵνα λατρεύσωσί μοι.
\vs{4}Ἐὰν δὲ μὴ θέλῃς σὺ ἐξαποστεῖλαι τὸν λαόν μου, ἰδοὺ ἐγὼ ἐπάγω ταύτην τὴν ὥραν αὔριον ἀκρίδα πολλὴν ἐπὶ πάντα τὰ ὅριά σου.
\vs{5}Καὶ καλύψει τὴν ὄψιν τῆς γῆς, καὶ οὐ δυνήσῃ κατιδεῖν τὴν γῆν· καὶ κατέδεται πᾶν τὸ περισσὸν τῆς γῆς τὸ καταλειφθὲν, ὃ κατέλιπεν ὑμῖν ἡ χάλαζα, καὶ κατέδεται πᾶν ξύλον τὸ φυόμενον ὑμῖν ἐπὶ τῆς γῆς.
\vs{6}Καὶ πλησθήσονταί σου αἱ οἰκίαι, καὶ αἱ οἰκίαι τῶν θεραπόντων σου, καὶ πᾶσαι αἱ οἰκίαι ἐν πάσῃ γῇ τῶν Αἰγυπτίων· ἃ οὐδέποτε ἑωράκασιν οἱ πατέρες σου, οὐδʼ οἱ πρόπαπποι αὐτῶν, ἀφʼ ἧς ἡμέρας γεγόνασιν ἐπὶ τῆς γῆς, ἔως τῆς ἡμέρας ταύτης· καὶ ἐκκλίνας Μωυσῆς ἐξῆλθεν ἀπὸ Φαραώ.
\vs{7}Καὶ λέγουσιν οἱ θεράποντες Φαραὼ πρὸς αὐτὸν, ἕως τίνος ἔσται τοῦτο ἡμῖν σκῶλον; ἐξαπόστειλον τοὺς ἀνθρώπους, ὅπως λατρεύσωσι τῷ Θεῷ αὐτῶν· ἢ εἰδέναι βούλει ὅτι ἀπόλωλεν Αἴγυπτος;
\vs{8}Καὶ ἀπέστρεψαν τόν τε Μωυσῆν καὶ Ἀαρὼν πρὸς Φαραὼ, καὶ εἶπεν αὐτοῖς, πορεύεσθε καὶ λατρεύσατε Κυρίῳ τῷ Θεῷ ὑμῶν· τίνες δὲ καὶ τίνες εἰσιν οἱ πορευόμενοι;
\vs{9}Καὶ λέγει Μωυσῆς, σὺν τοῖς νεανίσκοις καὶ πρεσβυτέροις πορευσόμεθα, σὺν τοῖς υἱοῖς καὶ θυγατράσι, καὶ προβάτοις, καὶ βουσὶν ἡμῶν· ἔστι γὰρ ἑορτὴ Κυρίου.
\vs{10}Καὶ εἶπε πρὸς αὐτοὺς, ἔστω οὕτω Κύριος μεθʼ ὑμῶν· καθότι ἀποστέλλω ὑμᾶς, μὴ καὶ τὴν ἀποσκευὴν ὑμῶν; ἴδετε ὅτι πονηρία πρόσκειται ὑμῖν.
\vs{11}Μὴ οὕτως· πορευέσθωσαν δὲ οἱ ἄνδρες, καὶ λατρευσάτωσαν τῷ Θεῷ· τοῦτο γὰρ αὐτοὶ ἐκζητεῖτε· ἐξέβαλον δὲ αὐτοὺς ἀπὸ προσώπου Φαραώ.
\vs{12}Εἶπε δὲ Κύριος πρὸς Μωυσῆν, ἔκτεινον τὴν χεῖρα ἐπὶ γῆν Αἰγύπτου· καὶ ἀναβήτω ἀκρὶς ἐπὶ τὴν γῆν, καὶ κατέδεται πᾶσαν βοτάνην τῆς γῆς, καὶ πάντα τὸν καρπὸν τῶν ξύλων, ὃν ὑπελίπετο ἡ χάλαζα.
\vs{13}Καὶ ἐπῇρε Μωυσῆς τὴν ῥάβδον εἰς τὸν οὐρανὸν, καὶ Κύριος ἐπήγαγεν ἄνεμον νότον ἐπὶ τὴν γῆν, ὅλην τὴν ἡμέραν ἐκείνην, καὶ ὅλην τὴν νύκτα· τὸ πρωῒ ἐγενήθη, καὶ ὁ ἄνεμος ὁ νότος ἀνέλαβεν τὴν ἀκρίδα,
\vs{14}καὶ ἀνήγαγεν αὐτὴν ἐπὶ πᾶσαν γῆν Αἰγύπτου· καὶ κατέπαυσεν ἐπὶ πάντα τὰ ὅρια Αἰγύπτου πολλὴ σφόδρα· προτέρα αὐτῆς οὐ γέγονε τοιαύτη ἀκρὶς, καὶ μετὰ ταῦτα οὐκ ἔσται οὕτως.
\vs{15}Καὶ ἐκάλυψε τὴν ὄψιν τῆς γῆς, καὶ ἐφθάρη ἡ γῆ· καὶ κατέφαγε πᾶσαν βοτάνην τῆς γῆς, καὶ πάντα τόν καρπὸν τῶν ξύλων, ὃς ὑπελείφθη ἀπὸ τῆς χαλάζης· οὐχ ὑπελείφθη χλωρὸν οὐδὲν ἐν τοῖς ξύλοις, καὶ ἐν πάσῃ βοτάνῃ τοῦ πεδίου, ἐν πάσηῃ γῇ Αἰγύπτου.

\vs{16}Κατέσπευδε δὲ Φαραὼ καλέσαι Μωυσῆν καὶ Ἀαρὼν, λέγων, ἡμάρτηκα ἐναντίον Κυρίου τοῦ Θεοῦ ὑμῶν, καὶ εἰς ὑμᾶς.
\vs{17}Προσδέξασθε οὖν μου τὴν ἁμαρτίαν ἔτι νῦν, καὶ προσεύξασθε πρὸς Κύριον τὸν Θεὸν ὑμῶν, καὶ περιελέτω ἀπʼ ἐμοῦ τὸν θάνατον τοῦτον.
\vs{18}Ἐξῆλθε δὲ Μωυσῆς ἀπὸ Φαραὼ, καὶ ηὔξατο πρὸς τὸν Θεόν.
\vs{19}Καὶ μετέβαλε Κύριος ἄνεμον ἀπὸ θαλάσσης σφοδρὸν, καὶ ἀνέλαβε τὴν ἀκρίδα, καὶ ἔβαλεν αὐτὴν εἰς τὴν ἐρυθρὰν θαλάσσαν· καὶ οὐχ ὑπελείφη ἀκρὶς μία ἐν πάσῃ γῇ Αἰγύπτου.
\vs{20}Καὶ ἐσκλήρυνε Κύριος τὴν καρδίαν Φαραὼ, καὶ οὐκ ἐξαπέστειλε τοὺς υἱοὺς Ἰσραήλ.
\vs{21}Εἶπε δὲ Κύριος πρὸς Μωυσῆν, ἔκτεινον τὴν χεῖρά σου εἰς τὸν οὐρανὸν, καὶ γενηθήτω σκότος ἐπὶ γῆς Αἰγύπτου, ψηλαφητὸν σκότος.
\vs{22}Ἐξέτεινε δὲ Μωυσῆς τὴν χεῖρα εἰς τὸν οὐρανόν· καὶ ἐγένετο σκότος γνόφος, θύελλα ἐπὶ πᾶσαν γῆν Αἰγύπτου τρεῖς ἡμέρας·
\vs{23}Καὶ οὐκ εἶδεν οὐδεὶς τὸν ἀδελφὸν αὐτοῦ τρεῖς ἡμέρας· καὶ οὐκ ἐξανέστη οὐδεὶς ἐκ τῆς κοίτης αὐτοῦ τρεῖς ἡμέρας· πᾶσι δὲ τοῖς υἱοῖς Ἰσραὴλ φῶς ἦν ἐν πᾶσιν οἷς κατεγίνοντο.
\vs{24}Καὶ ἐκάλεσε Φαραὼ Μωυσῆν καὶ Ἀαρὼν, λέγων, Βαδίζετε, λατρεύσατε Κυρίῳ τῷ Θεῷ ὑμῶν, πλὴν τῶν προβάτων καὶ τῶν βοῶν ὑπολείπεσθε· καὶ ἡ ἀποσκευὴ ὑμῶν ἀποτρεχέτω μεθʼ ὑμῶν.
\vs{25}Καὶ εἶπε Μωυσῆς, ἀλλὰ καὶ σὺ δώσεις ἡμῖν ὁλοκαυτώματα καὶ θυσίας, ἂ ποιήσομεν Κυρίῳ τῷ Θεῷ ἡμῶν.
\vs{26}Καὶ τὰ κτήνη ἡμῶν πορεύσεται μεθʼ ἡμῶν, καὶ οὐχ ὑπολειψόμεθα ὁπλήν· ἀπʼ αὐτῶν γὰρ ληψόμεθα λατρεῦσαι Κυρίῳ τῷ Θεῷ ἡμῶν· ἡμεῖς δὲ οὐκ οἴδαμεν τί λατρεύσομεν Κυρίῳ τῷ Θεῷ ἡμῶν, ἕως τοῦ ἐλθεῖν ἡμᾶς ἐκεῖ.
\vs{27}Ἐσκλήρυνε δὲ Κύριος τὴν καρδίαν Φαραὼ, καὶ οὐκ ἐβουλήθη ἐξαποστεῖλαι αὐτούς.
\vs{28}Καὶ λέγει Φαραὼ, ἄπελθε ἀπʼ ἐμοῦ· πρόσεχε σεαυτῷ ἔτι προσθεῖναι ἰδεῖν μου τὸ πρόσωπον· ᾗ δʼ ἂν ἡμέρᾳ ὀφθῇς μοι, ἀποθανῇ.
\vs{29}Λέγει δὲ Μωυσῆς, εἴρηκας· οὐκ ἔτι ὀφθήσομαί σοι εἰς πρόσωπον.

\ch{11}
Εἶπε δὲ Κύριος πρὸς Μωυσῆν, ἔτι μίαν πληγὴν ἐγὼ ἐπάξω ἐπὶ Φαραὼ, καὶ ἐπʼ Αἴγυπτον, καὶ μετὰ ταῦτα ἐξαποστελεῖ ὑμᾶς ἐντεῦθεν· ὅταν δὲ ἐξαποστέλλῃ ὑμᾶς σὺν παντὶ, ἐκβαλεῖ ὑμᾶς ἐκβολῇ.
\vs{2}Λάλησον οὖν κρυφῇ εἰς τὰ ὦτα τοῦ λαοῦ, καὶ αἰτησάτω ἕκαστος παρὰ τοῦ πλησίον σκεύη ἀργυρᾶ καὶ χρυσὰ καὶ ἱματισμόν.
\vs{3}Κύριος δὲ ἔδωκε τὴν χάριν τῷ λαῷ αὐτοῦ ἐναντίον τῶν Αἰγυπτίων, καὶ ἔχρησαν αὐτοῖς· καὶ ὁ ἄνθρωπος Μωυσῆς μέγας ἐγενήθη σφόδρα ἐναντίον τῶν Αἰγυπτίων, καὶ ἐναντίον Φαραὼ, καὶ ἐναντίον τῶν θεραπόντων ἀτοῦ.
\vs{4}Καὶ εἶπε Μωυσῆς, τάδε λέγει Κύριος, περὶ μέσας νύκτας ἐγὼ εἰσπορεύομαι εἰς μέσον Αἰγύπτου·
\vs{5}Καὶ τελευτήσει πᾶν πρωτότοκον ἐν γῇ Αἰγύπτῳ, ἀπὸ πρωτοτόκου Φαραὼ, ὃς κάθηται ἐπὶ τοῦ θρόνου, καὶ ἕως πρωτοτόκου τῆς θεραπαίνης τῆς παρὰ τὸν μύλον, καὶ ἕως πρωτοτοκου παντος κτήνους.
\vs{6}Καὶ ἔσται κραυγὴ μεγάλη κατὰ πᾶσαν γῆν Αἰγύπτου, ἥτις τοιαύτη οὐ γέγονε, καὶ τοιαύτη οὐκ ἔτι προστεθήσεται.
\vs{7}Καὶ ἐν πᾶσι τοῖς υἱοῖς Ἰσραὴλ οὐ γρύξει κύων τῇ γλώσσῃ αὐτοῦ, ἀπὸ ἀνθρώπου ἕως κτήνους· ὅπως εἰδῇς ὅσα παραδοξάσει Κύριος ἀνὰ μέσον τῶν Αἰγυπτίων καὶ τοῦ Ἰσραήλ.
\vs{8}Καὶ καταβήσονται πάντες οἱ παῖδές σου οὗτοι πρός με, καὶ προσκυνήσουσί με, λέγοντες, ἔξελθε σὺ, καὶ πᾶς ὁ λαός σου, οὗ σὺ ἀφηγῇ· καὶ μετὰ ταῦτα ἐξελεύσομαι· ἐξῆλθε δὲ Μωυσῆς ἀπὸ Φαραὼ μετὰ θυμοῦ.
\vs{9}Εἶπε δὲ Κύριος πρὸς Μωυσῆν, οὐκ εἰσακούσεται ὑμῶν Φαραὼ, ἵνα πληθύνων πληθυνῶ μου τὰ σημεῖα, καὶ τὰ τέρατα ἐν γῇ Αἰγύπτῳ.
\vs{10}Μωσῆς δὲ καὶ Ἀαρὼν ἐποίησαν πάντα τὰ σημεῖα καὶ τὰ τέρατα ταῦτα ἐν γῇ Αἰγύπτῳ ἐναντίον Φαραώ· ἐσκλήρυνε δὲ Κύριος τὴν καρδίαν Φαραὼ, καὶ οὐκ εἰσήκουσεν ἐξαποστεῖλαι τοὺς υἱοὺς Ἰσραὴλ ἐκ γῆς Αἰγύπτου.

\ch{12}
Εἶπε δὲ Κύριος πρὸς Μευσῆν καὶ Ἀαρὼν ἐν γῇ Αἰγύπτου, λέγων,
\vs{2}ὁ μὴν οὗτος ὑμῖν ἀρχὴ μηνῶν· πρῶτός ἐστιν ὑμῖν ἐν τοῖς μησὶ τοῦ ἐνιαυτοῦ.
\vs{3}Λάλησον πρὸς πᾶσαν συναγωγὴν υἱῶν Ἰσραὴλ, λέγων, τῇ δεκάτῃ τοῦ μηνὸς τούτου λαβέτωσαν ἕκαστος πρόβατον κατʼ οἴκους πατριῶν, ἕκαστος πρόβατον κατʼ οἰκίαν.
\vs{4}Ἐὰν δὲ ὀλιγοστοὶ ὦσιν ἐν τῇ οἰκίᾳ, ὥστε μὴ εἶναι ἱκανοὺς εἰς πρόβατον, συλλήψεται μεθʼ ἑαυτοῦ τὸν γείτονα τὸν πλησίον αὐτοῦ· κατὰ ἀριθμὸν ψυχῶν, ἕκαστος τὸ ἀρκοῦν αὐτῷ συναριθμήσεται εἰς πρόβατον.
\vs{5}Πρόβατον τέλειον, ἄρσεν, ἐνιαύσιον ἔσται ὑμῖν· ἀπὸ τῶν ἀρνῶν καὶ τῶν ἐρίφων λὴψεσθε.
\vs{6}Καὶ ἔσται ὑμῖν διατετηρημένον ἕως τῆς τεσσαρεσκαιδεκάτης τοῦ μηνὸς τούτου· καὶ σφάξουσιν αὐτὸ πᾶν τὸ πλῆθος συναγωγῆς υἱῶν Ἰσραὴλ πρὸς ἑσπέραν.
\vs{7}Καὶ λήψονται ἀπὸ τοῦ αἵματος, καὶ θήσουσιν ἐπὶ τῶν δύο σταθμῶν καὶ ἐπὶ τὴν φλιὰν, ἐν τοῖς οἴκοις ἐν οἷς ἐὰν φάγωσιν αὐτὰ ἐν αὐτοῖς.
\vs{8}Καὶ φάγονται τὰ κρέα τῇ νυκτὶ ταύτῃ ὀπτὰ πυρὶ, καὶ ἄζυμα ἐπὶ πικρίδων ἔδονται.
\vs{9}Οὐκ ἔδεσθε ἀπʼ αὐτῶν ὠμὸν, οὐδὲ ἡψημένον ἐν ὕδατι, ἀλλʼ ἢ ὀπτὰ πυρὶ, κεφαλὴν σὺν τοῖς ποσὶ καὶ τοῖς ἐνδοσθίοις.
\vs{10}Οὐκ ἀπολείψεται ἀπʼ αὐτοῦ ἕως πρωΐ· καὶ ὀστοῦν οὐ συντρίψετε ἀπʼ αὐτοῦ· τὰ δὲ καταλειπόμενα ἀπʼ αὐτοῦ ἕως πρωῒ ἐν πυρὶ κατακαύσετε.
\vs{11}Οὕτω δὲ φάγεσθε αὐτό· αἱ ὀσφύες ὑμῶν περιεζωσμέναι, καὶ τὰ ὑποδήματα ἐν τοῖς ποσὶν ὑμῶν, καὶ αἱ βακτηρίαι ἐν ταῖς χερσὶν ὑμῶν· καὶ ἔδεσθε αὐτὸ μετὰ σπουδῆς· Πάσχα ἐστὶ Κυρίῳ.
\vs{12}Καὶ διελεύσομαι ἐν γῇ Αἰγύπτῳ ἐν τῇ νυκτὶ ταύτῃ, καὶ πατάξω πᾶν πρωτότοκον ἐν γῇ Αἰγύπτῳ ἀπὸ ἀνθρώπου ἕως κτήνους· καὶ ἐν πᾶσι τοῖς θεοῖς τῶν Αἰγυπτίων ποιήσω τὴν ἐκδίκησιν· ἐγὼ Κύριος.
\vs{13}Καὶ ἔσται τὸ αἷμα ὑμῖν ἐν σημείῳ ἐπὶ τῶν οἰκιῶν, ἐν αἷς ὑμεῖς ἔστε ἐκεῖ· καὶ ὄψομαι τὸ αἷμα, καὶ σκεπάσω ὑμᾶς, καὶ οὐκ ἔσται ἐν ὑμῖν πληγὴ τοῦ ἐκτριβῆναι ὅταν παίω ἐν γῇ Αἰγύπτῳ.

\vs{14}Καὶ ἔσται ἡ ἡμέρα ὑμῖν αὕτη μνημόσυνον, καὶ ἑορτάσετε αὐτὴν ἑορτὴν Κυρίῳ εἰς πάσας τὰς γενεὰς ὑμῶν· νόμιμον αἰώνιον ἑορτάσετε αὐτήν.
\vs{15}Ἑπτὰ ἡμέρας ἄζυμα ἔδεσθε· ἀπὸ δὲ τῆς ἡμέρας τῆς πρώτης, ἀφανιεῖτε ζύμην ἐκ τῶν οἰκιῶν ὑμῶν· πᾶς ὃς ἂν φάγῃ ζύμην, ἐξολοθρευθήσεται ἡ ψυχὴ ἐκείνη ἐξ Ἰσραήλ, ἀπὸ τῆς ἡμέρας τῆς πρώτης ἕως τῆς ἡμέρας τῆς ἑβδόμης.
\vs{16}Καὶ ἡ ἡμέρα ἡ πρώτη, κληθήσεται ἁγία· καὶ ἡ ἡμέρα ἡ ἑβδόμη, κλητὴ ἁγία ἔσται ὑμῖν· πᾶν ἔργον λατρευτὸν οὐ ποιήσετε ἐν αὐταῖς, πλὴν ὅσα ποιηθήσεται πάσῃ ψυχῇ, τοῦτο μόνον ποιηθήσεται ὑμῖν.
\vs{17}Καὶ φυλάξετε τὴν ἐντολὴν ταύτην· ἐν γὰρ τῇ ἡμέρᾳ ταύτῃ ἐξάξω τὴν δύναμιν ὑμῶν ἐκ γῆς Αἰγύπτου, καὶ ποιήσετε τὴν ἡμέραν ταύτην εἰς γενεὰς ὑμῶν νόμιμον αἰώνιον,
\vs{18}ἐναρχόμενοι τῇ τεσσαρεσκαιδεκάτῃ ἡμέρᾳ τοῦ μηνὸς τοῦ πρώτου, ἀφʼ ἑσπέρας ἔδεσθε ἄζυμα, ἕως ἡμέρας μίας καὶ εἰκάδος τοῦ μηνὸς, ἕως ἑσπέρας.
\vs{19}Ἑπτὰ ἡμέρας ζύμη οὐχ εὑρεθήσεται ἐν ταῖς οἰκιαῖς ὑμῶν· πᾶς ὃς ἂν φάγῃ ζυμωτὸν, ἐξολοθρευθήσεται ἡ ψυχὴ ἐκείνη ἐκ συναγωγῆς Ἰσραήλ· ἔν τε τοῖς γειώραις, καὶ αὐτόχθοσι τῆς γῆς.
\vs{20}Πᾶν ζυμωτὸν οὐκ ἔδεσθε, ἐν παντὶ δὲ κατοικητηρίῳ ὑμῶν ἔδεσθε ἄζυμα.

\vs{21}Ἐκάλεσε δὲ Μωυσῆς πᾶσαν γερουσίαν υἱῶν Ἰσραὴλ, καὶ εἶπε πρὸς αὐτοὺς, ἀπελθόντες λάβετε ὑμῖν αὐτοῖς πρόβατον κατὰ συγγενείας ὑμῶν, καὶ θύσατε τὸ πάσχα.
\vs{22}Λήψεσθε δὲ δέσμην ὑσσώπου, καὶ βάψαντες ἀπὸ τοῦ αἵματος τοῦ παρὰ τὴν θύραν, καθίξετε τῆς φλιᾶς, καὶ ἐπʼ ἀμφοτέρων τῶν σταθμῶν, ἀπὸ τοῦ αἵματος ὅ ἐστι παρὰ τὴν θύραν· ὑμεῖς δὲ οὐκ ἐξελεύσεσθε ἕκαστος τὴν θύραν τοῦ οἴκου αὐτοῦ ἕως πρωΐ.
\vs{23}Καὶ παρελεύσεται Κύριος πατάξαι τοὺς Αἰγυπτίους, καὶ ὄψεται τὸ αἷμα ἐπὶ τῆς φλιᾶς, καὶ ἐπʼ ἀμφοτέρων τῶν σταθμῶν· καὶ παρελεύσεται Κύριος τὴν θύραν, καὶ οὐκ ἀφήσει τὸν ὀλοθρεύοντα εἰσελθεῖν εἰς τὰς οἰκίας ὑμῶν πατάξαι.
\vs{24}Καὶ φυλάξασθε τὸ ῥῆμα τοῦτο νόμιμον σεαυτῷ, καὶ τοῖς υἱοῖς σου, ἕως αἰῶνος.
\vs{25}Ἐὰν δὲ εἰσέλθητε εἰς τὴν γῆν, ἣν ἂν δῷ Κύριος ὑμῖν, καθότι ἐλάλησε, φυλάξασθε τὴν λατρείαν ταύτην.
\vs{26}Καὶ ἐσται ἐὰν λέγωσι πρὸς ὑμᾶς οἱ υἱοὶ ὑμῶν, τίς ἡ λατρεία αὕτη;
\vs{27}Καὶ ἐρεῖτε αὐτοῖς, θυσία τὸ πάσχα τοῦτο Κυρίῳ, ὡς ἐσκέπασε τοὺς οἴκους τῶν υἱῶν Ἰσραὴλ ἐν Αἰγύπτῳ, ἡνίκα ἐπάταξε τοὺς Αἰγυπτίους, τοὺς δὲ οἴκους ἡμῶν ἐῤῥύσατο· καὶ κύψας ὁ λαὸς προσεκύνησε.
\vs{28}Καὶ ἀπελθόντες ἐποίησαν οἱ υἱοὶ Ἰσραὴλ, καθὰ ἐνετείλατο Κύριος τῷ Μωυσῇ καὶ Ἀαρῶν, οὕτως ἐποίησαν.

\vs{29}Ἐγενήθη δὲ μεσούσης τῆς νυκτὸς, καὶ Κύριος ἐπάταξε πᾶν πρωτότοκον ἐν γῇ Αἰγύπτῳ, ἀπὸ πρωτοτόκου Φαραὼ τοῦ καθημένου ἐπὶ τοῦ θρόνου, ἕως πρωτοτόκου τῆς αἰχμαλωτίδος τῆς ἐν τῷ λάκκῳ, καὶ ἕως πρωτοτόκου παντὸς κτήνους.
\vs{30}Καὶ ἀναστὰς Φαραὼ νυκτὸς, καὶ οἱ θεράποντες αὐτοῦ, καὶ πάντες οἱ Αἰγύπτιοι, καὶ ἔγενήθη κραυγὴ μεγάλη ἐν πάσῃ γῇ Αἰγύπτῳ· οὐ γὰρ ἦν οἰκία, ἐν ᾗ οὐκ ἦν ἐν αὐτῇ τεθνηκώς.
\vs{31}Καὶ ἐκάλεσε Φαραὼ Μωυσῆν καὶ Ἀαρὼν νυκτὸς, καὶ εἶπεν αὐτοῖς, ἀνάστητε, καὶ ἐξέλθατε ἐκ τοῦ λαοῦ μου, καὶ ὑμεῖς, καὶ οἱ υἱοὶ Ἰσραήλ· βαδίζετε καὶ λατρεύσατε Κυρίῳ τῷ Θεῷ ὑμῶν, καθὰ λέγετε.
\vs{32}Καὶ τὰ πρόβατα καὶ τοὺς βόας ὑμῶν ἀναλαβόντες πορεύεσθε· εὐλογήσατε δὴ κᾀμέ.
\vs{33}Καὶ κατεβιάζοντο οἱ Αἰγύπτιοι τὸν λαὸν σπουδῇ ἐκβαλεῖν αὐτοὺς ἐν τῆς γῆς· εἶπαν γὰρ, ὅτι πάντες ἡμεῖς ἀποθνήσκομεν.
\vs{34}Ἀνέλαβε δὲ ὁ λαὸς τὸ σταῖς αὐτῶν, πρὸ τοῦ ζυμωθῆναι τὰ φυράματα αὐτῶν, ἐνδεδεμένα ἐν τοῖς ἱματίοις αὐτῶν ἐπὶ τῶν ὤμαν.
\vs{35}Οἱ δὲ υἱοὶ Ἰσραὴλ ἐποίησαν, καθὰ συνέταξεν αὐτοῖς Μωυσῆς, καὶ ᾔτησαν παρὰ τῶν Αἰγυπτίων σκεύη ἀργυρᾶ καὶ χρυσᾶ καὶ ἱματισμόν.
\vs{36}Καὶ ἔδωκε Κύριος τὴν χάριν τῷ λαῷ αὐτοῦ ἐναντίον τῶν Αἰγυπτίων, καὶ ἔχρησαν αὐτοῖς· καὶ ἐσκύλευσαν τοὺς Αἰγυπτίους.

\vs{37}Ἀπάραντες δὲ υἱοὶ Ἰσραὴλ ἐκ Ῥαμεσσῆ εἰς Σοκχὼθ εἰς ἑξακοσίας χιλιάδας πεζῶν, οἱ ἄνδρες, πλὴν τῆς ἀποσκευῆς.
\vs{38}Καὶ ἐπίμικτος πολὺς συνανέβη αὐτοῖς, καὶ πρόβατα, καὶ βόες, καὶ κτήνη πολλὰ σφόδρα.
\vs{39}Καὶ ἔπεψαν τὸ σταῖς ὃ ἐξήνεγκαν ἐξ Αἰγύπτου, ἐγκρυφίας ἀζύμους, οὐ γὰρ ἐζυμώθη· ἐξέβαλον γὰρ αὐτοὺς οἱ Αἰγύπτιοι, καὶ οὐκ ἠδυνήθησαν ἐπιμεῖναι, οὐδὲ ἐπισιτισμὸν ἐποίησαν ἑαυτοῖς εἰς τὴν ὁδόν.
\vs{40}Ἡ δὲ κατοίκησις τῶν υἱῶν Ἰσραὴλ, ἣν κατῴκησαν ἐν γῇ Αἰγύπτῳ καὶ ἐν γῇ Χαναὰν, ἔτη τετρακόσια τριάκοντα.
\vs{41}Καὶ ἐγένετο μετὰ τὰ τετρακόσια τριάκοντα ἔτη, ἐξῆλθε πᾶσα ἡ δύναμις Κυρίου ἐκ γῆς Αἰγύπτου νυκτός.
\vs{42}Προφυλακή ἐστι τῷ Κυρίῳ, ὥστε ἐξαγαγεῖν αὐτοὺς ἐκ γῆς Αἰγύπτου· ἐκείνη ἡ νὺξ αὕτη, προφυλακὴ Κυρίῳ, ὥστε πᾶσι τοῖς υἱοῖς Ἰσραὴλ εἶναι εἰς γενεὰς αὐτῶν.
\vs{43}Εἶπε δὲ Κύριος πρὸς Μωυσῆν καὶ Ἀαρὼν, οὗτος ὁ νόμος τοῦ πάσχα· πᾶς ἀλλογενὴς οὐκ ἔδεται ἀπʼ αὐτοῦ·
\vs{44}Καὶ πάντα οἰκέτην ἢ ἀργυρώνητον περιτεμεῖς αὐτόν· καὶ τότε φάγεται ἀπʼ αὐτοῦ.
\vs{45}Πάροικος ἢ μισθωτὸς οὐκ ἔδεται ἀπʼ αὐτοῦ.
\vs{46}Ἐν οἰκίᾳ μιᾷ βρωθήσεται, καὶ οὐκ ἐξοίσετε ἐκ τῆς οἰκίας τῶν κρεῶν ἔξω· καὶ ὀστοῦν οὐ συντρίψετε ἀπʼ αὐτοῦ.
\vs{47}Πᾶσα συναγωγὴ υἱῶν Ἰσραὴλ ποιήσει αὐτό.
\vs{48}Ἐὰν δέ τις προσέλθῃ πρὸς ὑμᾶς προσήλυτος ποιῆσαι τὸ πάσχα Κυρίῳ, περιτεμεῖς αὐτοῦ πᾶν ἀρσενικόν, καὶ τότε προσελεύσεται ποιῆσαι αὐτό· καὶ ἔσται ὥσπερ καὶ ὁ αὐτόχθων τῆς γῆς· πᾶς ἀπερίτμητος οὐκ ἔδεται ἀπʼ αὐτοῦ.
\vs{49}Νόμος εἷς ἔσται τῷ ἐγχωρίῳ, καὶ τῷ προσελθόντι προσηλύτῳ ἐν ὑμῖν.
\vs{50}Καὶ ἐποίησαν οἱ υἱοὶ Ἰσραὴλ καθὰ ἐνετείλατο Κύριος τῷ Μωυσῇ καὶ Ἀαρὼν πρὸς αὐτούς, οὕτως ἐποίησαν.
\vs{51}Καὶ ἐγένετο ἐν τῇ ἡμέρᾳ ἐκείνῃ, ἐξήγαγε Κύριος τοὺς υἱοὺς Ἰσραὴλ ἐκ γῆς Αἰγύπτου σὺν δυνάμει αὐτῶν.

\ch{13}
Εἶπε δὲ Κύριος πρὸς Μωυσῆν, λέγων,
\vs{2}ἁγίασόν μοι πᾶν πρωτότοκον πρωτογενὲς διανοῖγον πᾶσαν μήτραν ἐν τοῖς υἱοῖς Ἰσραὴλ ἀπὸ ἀνθρώπου ἕως κτήνους, ἐμοί ἐστιν.
\vs{3}Εἶπε δὲ Μωυσῆς πρὸς τὸν λαὸν, μνημονεύετε τὴν ἡμέραν ταύτην, ἐν ᾗ ἐξήλθατε ἐκ γῆς Αἰγύπτου, ἐξ οἴκου δουλείας· ἐν γὰρ χειρὶ κραταιᾷ ἐξήγαγεν ὑμᾶς Κύριος ἐντεῦθεν· καὶ οὐ βρωθήσεται ζύμη.
\vs{4}Ἐν γὰρ τῇ σήμερον ὑμεῖς ἐκπορεύεσθε ἐν μηνὶ τῶν νέων.
\vs{5}Καὶ ἔσται ἡνίκα ἐὰν εἰσαγάγῃ σε Κύριος ὁ Θεός σου εἰς τὴν γῆν τῶν Χαναναίων, καὶ Χετταίων, καὶ Ἀμοῤῥαίων, καὶ Εὐαίων, καὶ Ἰεβουσαίων, καὶ Γεργεσαίων, καὶ Φερεζαίων, ἣν ὤμοσε τοῖς πατράσι σου, δοῦναί σοι γῆν ῥέουσαν γάλα καὶ μέλι· καὶ ποιήσεις τὴν λατρείαν ταύτην ἐν τῷ μηνὶ τούτῳ.
\vs{6}Ἓξ ἡμέρας ἔδεσθε ἄζυμα, τῇ δὲ ἡμέρᾳ τῇ ἑβδόμῃ ἑορτὴ Κυρίου.
\vs{7}Ἄζυμα ἔδεσθε ἑπτὰ ἡμέρας· οὐκ ὀφθήσεταί σοι ζυμωτὸν, οὐδὲ ἔσται σοι ζύμη ἐν πᾶσι τοῖς ὁρίοις σου.
\vs{8}Καὶ ἀναγγελεῖς τῷ υἱῷ σου ἐν τῇ ἡμέρᾳ ἐκείνῃ, λέγων, διὰ τοῦτο ἐποίησε Κύριος ὁ Θεός μοι, ὡς ἐξεπορευόμην ἐξ Αἰγύπτου.
\vs{9}Καὶ ἔσται σοι σημεῖον ἐπὶ τῆς χειρός σου, καὶ μνημόσυνον πρὸ ὀφθαλμῶν σου, ὅπως ἂν γένηται ὁ νόμος Κυρίου ἐν τῷ στόματί σου· ἐν γὰρ χειρὶ κραταιᾷ ἐξήγαγέ σε Κύριος ὁ Θεὸς ἐξ Αἰγύπτου.
\vs{10}Καὶ φυλάξασθε τὸν νόμον τοῦτον κατὰ καιροὺς ὡρῶν, ἀφʼ ἡμερῶν εἰς ἡμέρας.

\vs{11}Καὶ ἔσται ὡς ἂν εἰσαγάγῃ σε Κύριος ὁ Θεός σου εἰς τὴν γῆν τῶν Χαναναίων, ὃν τρόπον ὤμοσε τοῖς πατράσι σου, καὶ δώσει σοι αὐτήν.
\vs{12}Καὶ ἀφελεῖς πᾶν διανοῖγον μήτραν, τὰ ἀρσενικὰ τῷ Κυρίῳ· πᾶν διανοῖγον μήτραν ἐκ βουκολίων ἢ ἐν τοῖς κτήνεσί σου, ὅσα ἐὰν γένηταί σοι, τὰ ἀρσενικὰ ἁγιάσεις τῷ Κυρίῳ.
\vs{13}Πᾶν διανοῖγον μήτραν ὄνου, ἀλλάξεις προβάτῳ· ἐὰν δὲ μὴ ἀλλάξῃς, λυτρώσῃ αὐτό· πᾶν πρωτότοκον ἀνθρώπου τῶν υἱῶν σου λυτρώσῃ.
\vs{14}Ἐὰν δὲ ἐρωτήσῃ σε ὁ υἱός σου μετὰ ταῦτα, λέγων, τί τοῦτο; καὶ ἐρεῖς αὐτῷ, ὅτι ἐν χειρὶ κραταιᾷ ἐξήγαγεν Κύριος ἡμᾶς ἐκ γῆς Αἰγύπτου, ἐξ οἴκου δουλείας.
\vs{15}Ἡνίκα δὲ ἐσκλήρυνε Φαραὼ ἐξαποστεῖλαι ἡμᾶς, ἀπέκτεινε πᾶν πρωτότοκον ἐν γῇ Αἰγύπτῳ, ἀπὸ πρωτοτόκων ἀνθρώπων ἕως πρωτοτόκων κτηνῶν· διὰ τοῦτο ἐγὼ θύω πᾶν διανοῖγον μήτραν, τὰ ἀρσενικὰ τῷ Κυρίῳ, καὶ πᾶν πρωτότοκον τῶν υἱῶν μου λυτρώσομαι.
\vs{16}Καὶ ἔσται εἰς σημεῖον ἐπὶ τῆς χειρός σου, καὶ ἀσαλευτον πρὸ ὀφθαλμων σου· ἐν γὰρ χειρὶ κραταιᾷ ἐξήγαγέ σε Κύριος ἐξ Αἰγύπτου.

\vs{17}Ὡς δὲ ἐξαπέστειλε Φαραὼ τὸν λαὸν, οὐχ ὡδήγησεν αὐτοὺς ὁ Θεὸς ὁδὸν γῆς· Φυλιστιεὶμ, ὅτι ἐγγὺς ἦν· εἶπε γὰρ ὁ Θεὸς, μήποτε μεταμελήσῃ τῷ λαῷ ἰδόντι πόλεμον, καὶ ἀποστρέψῃ εἰς Αἴγυπτον.
\vs{18}Καὶ ἐκύκλωσεν ὁ Θεὸς τὸν λαὸν ὁδὸν τὴν εἰς τὴν ἔρημον, εἰς τὴν ἐρυθρὰν θάλασσαν· πέμπτῃ δὲ γενεᾷ ἀνέβησαν οἱ υἱοὶ Ἰσραὴλ ἐκ γῆς Αἰγύπτου.
\vs{19}Καὶ ἔλαβε Μωυσῆς τὰ ὀστᾶ Ἰωσὴφ μεθʼ ἑαυτοῦ· ὅρκῳ γὰρ ὥρκισεν τοὺς υἱοὺς Ἰσραὴλ, λέγων, ἐπισκοπῇ ἐπισκέψεται ὑμᾶς Κύριος, καὶ συνανοίσετε μου τὰ ὀστᾶ ἐντεῦθεν μεθʼ ὑμῶν.

\vs{20}Ἐξάραντες δὲ οἱ υἱοὶ Ἰσραὴλ ἐκ Σοκχὼθ, ἐστρατοπέδευσαν ἐν Ὀθὼμ παρὰ τὴν ἔρημον.
\vs{21}Ὁ δὲ Θεὸς ἡγεῖτο αὐτῶν, ἡμέρας μὲν ἐν στύλῳ νεφέλης, δεῖξαι αὐτοῖς τὴν ὁδόν· τὴν δὲ νύκτα ἐν στύλῳ πυρός.
\vs{22}Οὐκ ἐξέλιπεν δὲ ὁ στύλος τῆς νεφέλης ἡμέρας, καὶ ὁ στύλος τοῦ πυρὸς νυκτὸς, ἐναντίον τοῦ λαοῦ παντός.

\ch{14}
Καὶ ἐλάλησε Κύριος πρὸς Μωυσῆν, λέγων,
\vs{2}Λάλησον τοῖς υἱοῖς Ἰσραὴλ, καὶ ἀποστρέψαντες στρατοπεδευσάτωσαν ἀπέναντι τῆς ἐπαύλεως, ἀνὰ μέσον Μαγδώλου καὶ ἀνὰ μέσον τῆς θαλάσσης, ἐξεναντίας Βεελσεπφῶν· ἐνώπιον αὐτῶν στρατοπεδεύσεις ἐπὶ τῆς θαλάσσης.
\vs{3}Καὶ ἐρεῖ Φαραὼ τῷ λαῷ αὐτοῦ, οἱ υἱοὶ Ἰσραὴλ πλανῶνται οὗτοι ἐν τῇ γῇ, συγκέκλεικε γὰρ αὐτοὺς ἡ ἔρημος.
\vs{4}Ἐγὼ δὲ σκληρυνῶ τὴν καρδίαν Φαραὼ, καὶ καταδιώξεται ὀπίσω αὐτῶν· καὶ ἐνδοξασθήσομαι ἐν Φαραῷ, καὶ ἐν πάσῃ τῇ στρατίᾳ αὐτοῦ· καὶ γνώσονται πάντες οἱ Αἰγύπτιοι ὅτι ἐγώ εἰμι Κύριος· καὶ ἐποίησαν οὕτως.
\vs{5}Καὶ ἀνηγγέλη τῷ βασιλεῖ τῶν Αἰγυπτίων ὅτι πέφευγεν ὁ λαός· καὶ μετεστράφη ἡ καρδία Φαραὼ, καὶ τῶν θεραπόντων αὐτοῦ, ἐπὶ τὸν λαὸν, καὶ εἶπαν, τί τοῦτο ἐποιήσαμεν, τοῦ ἐξαποστεῖλαι τοὺς υἱοὺς Ἰσραὴλ, τοῦ μὴ δουλεύειν ἡμῖν;
\vs{6}Ἔζευξεν οὖν Φαραὼ τὰ ἅρματα αὐτοῦ, καὶ πάντα τὸν λαὸν αὐτοῦ συναπήγαγε μεθʼ ἑαυτοῦ,
\vs{7}καὶ λαβὼν ἑξακόσια ἅρματα ἐκλεκτὰ, καὶ πᾶσαν τὴν ἵππον τῶν Αἰγυπτίων, καὶ τριστάτας ἐπὶ πάντων.
\vs{8}Καὶ ἐσκλήρυνε Κύριος τὴν καρδίαν Φαραὼ βασιλέως Αἰγύπτου, καὶ τῶν θεραπόντων αὐτοῦ, καὶ κατεδίωξεν ὀπίσω τῶν υἱῶν Ἰσραήλ· οἱ δὲ υἱοὶ Ἰσραὴλ ἐξεπορεύοντο ἐν χειρὶ ὑψηλῇ.
\vs{9}Καὶ κατεδίωξαν οἱ Αἰγύπτιοι ὀπίσω αὐτῶν, καὶ εὕροσαν αὐτοὺς παρεμβεβληκότας παρὰ τὴν θάλασσαν· καὶ πᾶσα ἡ ἵππος καὶ τὰ ἅρματα Φαραὼ, καὶ οἱ ἱππεῖς, καὶ ἡ στρατιὰ αὐτοῦ ἀπέναντι τῆς ἐπαύλεως, ἐξεναντίας Βεελσεπφῶν.
\vs{10}Καὶ Φαραὼ προσῆγε· καὶ ἀναβλέψαντες οἱ υἱοὶ Ἰσραὴλ τοῖς ὀφθαλμοῖς ὁρῶσι, καὶ οἱ Αἰγύπτιοι ἐστρατοπέδευσαν ὀπίσω αὐτῶν· καὶ ἐφοβήθησαν σφόδρα· ἀνεβόησαν δὲ οἱ υἱοὶ Ἰσραὴλ πρὸς Κύριον.
\vs{11}Καὶ εἶπαν πρὸς Μωυσῆν, παρὰ τὸ μὴ ὑπάρχειν μνήματα ἐν γῇ Αἰγύπτῳ, ἐξήγαγες ἡμᾶς θανατῶσαι ἐν τῇ ἐρήμῳ· τί τοῦτο ἐποίησας ἡμῖν, ἐξαγαγὼν ἐξ Αἰγύπτου;
\vs{12}Οὐ τοῦτο ἦν τὸ ῥῆμα, ὃ ἐλαλήσαμεν πρὸς σὲ ἐν Αἰγύπτῳ, λέγοντες, πάρες ἡμᾶς, ὅπως δουλεύσωμεν τοῖς Αἰγυπτίοις; κρεῖσσον γὰρ ἡμᾶς δουλεύειν τοῖς Αἰγυπτίοις, ἢ ἀποθανεῖν ἐν τῇ ἐρήμῳ ταύτῃ.

\vs{13}Εἶπε δὲ Μωυσῆς πρὸς τὸν λαὸν, θαρσεῖτε, στῆτε καὶ ὁρᾶτε τὴν σωτηρίαν τὴν παρὰ τοῦ Κυρίου, ἣν ποιήσει ἡμῖν σήμερον· ὃν τρόπον γὰρ ἑωράκατε τοὺς Αἰγυπτίους σήμερον, οὐ προσθήσεσθε ἔτι ἰδεῖν αὐτοὺς εἰς τὸν αἰῶνα χρόνον·
\vs{14}Κύριος πολεμήσει περὶ ὑμῶν, καὶ ὑμεῖς σιγήσετε.
\vs{15}Εἶπε δὲ Κύριος πρὸς Μωυσῆν, τί βοᾷς πρός με; λάλησον τοῖς υἱοῖς Ἰσραὴλ, καὶ ἀναζευξάτωσαν.
\vs{16}Καὶ σὺ ἔπαρον τῇ ῥάβδῳ σου, καὶ ἔκτεινον τὴν χεῖρά σου ἐπὶ τὴν θάλασσαν, καὶ ῥῆξον αὐτήν· καὶ εἰσελθάτωσαν οἱ υἱοὶ Ἰσραὴλ εἰς μέσον τῆς θαλάσσης κατὰ τὸ ξηρόν.
\vs{17}Καὶ ἰδοὺ ἐγὼ σκληρυνῶ τὴν καρδίαν Φαραὼ, καὶ τῶν Αἰγυπτίων πάντων, καὶ εἰσελεύσονται ὀπίσω αὐτῶν· καὶ ἐνδοξασθήσομαι ἐν Φαραῷ, καὶ ἐν πάσῃ τῇ στρατιᾷ αὐτοῦ, καὶ ἐν τοῖς ἅρμασι, καὶ ἐν τοῖς ἵπποις αὐτοῦ.
\vs{18}Καὶ γνώσονται πάντες οἱ Αἰγύπτιοι ὅτι ἐγώ εἰμὶ Κύριος, ἐνδοξαζομένου μου ἐν Φαραῷ, καὶ ἐν τοῖς ἅρμασι, καὶ ἵπποις αὐτοῦ.
\vs{19}Ἐξῇρε δὲ ὁ Ἄγγελος τοῦ Θεοῦ ὁ προπορευόμενος τῆς παρεμβολῆς τῶν υἱῶν Ἰσραὴλ, καὶ ἐπορεύθη ἐκ τῶν ὄπισθεν· ἐξῇρε δὲ καὶ ὁ στύλος τὴς νεφέλης ἀπὸ προσώπου αὐτῶν, καὶ ἔστη ἐκ τῶν ὀπίσω αὐτῶν.
\vs{20}Καὶ εἰσῆλθεν ἀνὰ μέσον τῆς παρεμβολῆς τῶν Αἰγυπτίων, καὶ ἀνὰ μέσον τῆς παρεμβολῆς τῶν Αἰγυπίων, καὶ ἀνὰ μέσον τῆς παρεμβολῆς Ἰσραὴλ, καὶ ἔστη· καὶ ἐγένετο σκότος καὶ γνόφος· καὶ διῆλθεν ἡ νύξ· καὶ οὐ συνέμιξαν ἀλλήλοις ὅλην τὴν νύκτα.
\vs{21}Ἐξέτεινε δὲ Μωυσῆς τὴν χεῖρα ἐπὶ τὴν θάλασσαν· καὶ ὑπήγαγε Κύριος τὴν θάλασσαν ἐν ἀνέμῳ νότῳ βιαίῳ ὅλην τὴν νύκτα, καὶ ἐποίησε τὴν θάλασσαν ξηράν· καὶ ἐσχίσθη τὸ ὕδωρ.
\vs{22}Καὶ εἰσῆλθον οἱ υἱοὶ Ἰσραὴλ εἰς μέσον τῆς θαλάσσης κατὰ τὸ ξηρόν· καὶ τὸ ὕδωρ αὐτῆς τεῖχος ἐκ δεξιῶν, καὶ τεῖχος ἐξ εὐωνύμων.

\vs{23}Καὶ κάτεδίωξαν οἱ Αἰγύπτιοι, καὶ εἰσῆλθον ὀπίσω αὐτῶν καὶ πᾶς ἵππος Φαραὼ, καὶ τὰ ἅρματα, καὶ οἱ ἀναβάται, εἰς μέσον τῆς θαλάσσης.
\vs{24}Ἐγενήθη δὲ ἐν τῇ φυλακῇ τῇ ἑωθινῇ, καὶ ἐπίβλεψε Κύριος ἐπὶ τὴν παρεμβολὴν τῶν Αἰγυπτίων ἐν στύλῳ πυρὸς καὶ νεφέλης, καὶ συνετάραξε τὴν παρεμβολὴν τῶν Αἰγυπτίων,
\vs{25}καὶ συνέδησε τοὺς ἄξονας τῶν ἁρμάτων αὐτῶν, καὶ ἤγαγεν αὐτοὺς μετὰ βίας· καὶ εἶπαν οἱ Αἰγύπτιοι, φυγωμεν ἀπὸ προσώπου Ἰσραήλ· ὁ γὰρ Κύριος πολεμεῖ περὶ αὐτῶν τοὺς Αἰγυπτίους.
\vs{26}Εἶπε δὲ Κύριος πρὸς Μωυσῆν, ἔκτεινον τὴν χεῖρά σου ἐπὶ τὴν θάλασσαν, καὶ ἀποκαταστήτω τὸ ὕδωρ, καὶ ἐπικαλυψάτω τοὺς Αἰγυπτίους, ἐπί τε τὰ ἅρματα καὶ τοὺς ἀναβάτας.
\vs{27}Ἐξέτεινε δὲ Μωυσῆς τὴν χεῖρα ἐπὶ τὴν θάλασσαν, καὶ ἀπεκατέστη τὸ ὕδωρ πρὸς ἡμέραν ἐπὶ χώρας· οἱ δὲ Αἰγύπτιοι ἔφυγον ὑπὸ τὸ ὕδωρ· καὶ ἐξετίναξε Κύριος τοὺς Αἰγυπτίους μέσον τῆς θαλάσοης.
\vs{28}Καὶ ἐπαναστραφὲν τὸ ὕδωρ ἐκάλυψε τὰ ἅρματα καὶ τοὺς ἀναβάτας, καὶ πᾶσαν τὴν δύναμιν Φαραὼ, τοὺς εἰσπεπορευμένους ὀπίσω αὐτῶν εἰς τὴν θάλασσαν· καὶ οὐ κατελείφθη ἐξ αὐτῶν οὐδὲ εἷς.
\vs{29}Οἱ δὲ υἱοὶ Ἰσραὴλ ἐπορεύθησαν διὰ ξηρᾶς ἐν μέσῳ τῆς θάλασσης· τὸ δὲ ὕδωρ αὐτοῖς τεῖχος ἐκ δεξιῶν, καὶ τεῖχος ἐξ εὐωνύμων.
\vs{30}Καὶ ἐῤῥύσατο Κύριος τὸν Ἰσραὴλ ἐν τῇ ἡμέρᾳ ἐκείνῃ ἐκ χειρὸς τῶν Αἰγυπτίων· καὶ εἶδεν Ἰσραὴλ τοὺς Αἰγυπτίους τεθνηκότας παρὰ τὸ χεῖλος τῆς θαλάσσης.
\vs{31}Εἶδε δὲ Ἰσραὴλ τὴν χεῖρα τὴν μεγάλην, ἃ ἐποίησε Κύριος τοῖς Αἰγυπτίοις· ἐφοβήθη δὲ ὁ λαὸς τὸν Κύριον, καὶ ἐπίστευσαν τῷ Θεῷ, καὶ Μωυσῇ τῷ θεράποντι αὐτοῦ.

\ch{15}
Τότε ᾖσε Μωυσῆς καὶ οἱ υἱοὶ Ἰσραὴλ τὴν ᾠδὴν ταύτην τῷ Θεῷ, καὶ εἶπαν, λέγοντες, ᾄσωμεν τῷ Κυρίῳ, ἐνδόξως γὰρ δεδόξασται· ἵππον καὶ ἀναβάτην ἔῤῥιψεν εἰς θάλασσαν.
\vs{2}Βοηθὸς καὶ σκεπαστὴς ἐγένετό μοι εἰς σωτηρίαν· οὗτός μου Θεὸς, καὶ δοξάσω αὐτόν· Θεὸς τοῦ πατρός μου, καὶ ὑψώσω αὐτόν.
\vs{3}Κύριος συντρίβων πολέμους, Κύριος ὄνομα αὐτῷ.
\vs{4}Ἅρματα Φαραὼ, καὶ τὴν δύναμιν αὐτοῦ, ἔῤῥιψεν εἰς θάλασσαν, ἐπιλέκτους ἀναβάτας τριστάτας· κατεπόθησαν ἐν ἐρυθρᾷ θαλάσσῃ·
\vs{5}Πόντῳ ἐκάλυψεν αὐτούς· κατέδυσαν εἰς βυθὸν ὡσεὶ λίθος.
\vs{6}Ἡ δεξιά σου, Κύριε, δεδόξασται ἐν ἰσχύϊ· ἡ δεξιά σου χεὶρ, Κύριε, ἔθραυσεν ἐχθρούς.
\vs{7}Καὶ τῷ πλήθει τῆς δόξης σου συνέτριψας τοὺς ὑπεναντίους· ἀπέστειλας τὴν ὀργήν σου κατέφαγεν αὐτοὺς ὡς καλάμην.
\vs{8}Καὶ διὰ πνεύματος τοῦ θυμοῦ σου διέστη τὸ ὕδωρ· ἐπάγη ὡσεὶ τεῖχος τὰ ὕδατα· ἐπάγη τὰ κύματα ἐν μέσῳ τῆς θαλάσσης.
\vs{9}Εἶπεν ὁ ἐχθρὸς, διώξας καταλήψομαι, μεριῶ σκῦλα· ἐμπλήσω ψυχήν μου, ἀνελῶ τῇ μαχαίρῃ μου, κυριεύσει ἡ χείρ μου.
\vs{10}Ἀπέστειλας τὸ πνεῦμά σου· ἐκάλυψεν αὐτοὺς θάλασσα· ἔδυσαν ὡσεὶ μόλιβος ἐν ὕδατι σφοδρῷ.
\vs{11}Τίς ὅμοιός σοι ἐν θεοῖς, Κύριε; τίς ὅμοιός σοι; δεδοξασμένος ἐν ἁγίοις, θαυμαστὸς ἐν δόξαις, ποιῶν τέρατα.
\vs{12}Ἐξέτεινας τὴν δεξιάν σου· κατέπιεν αὐτοὺς γῆ.
\vs{13}Ὡδήγησας τῇ δικαιοσύνῃ σου τὸν λαόν σου τοῦτον, ὃν ἐλυτρώσω· παρεκάλεσας τῇ ἰσχύϊ σου εἰς κατάλυμα ἅγιόν σου.
\vs{14}Ἤκουσαν ἔθνη, καὶ ὠργίσθησαν· ὠδῖνες ἔλαβον κατοικοῦντας Φυλιστιείμ.
\vs{15}Τότε ἔσπευσαν ἡγεμόνες Ἐδὼμ, καὶ ἄρχοντες Μωαβιτῶν· ἔλαβεν αὐτοὺς τρόμος· ἐτάκησαν πάντες οἱ κατοικοῦντες Χαναάν.
\vs{16}Ἐπιπέσοι ἐπʼ αὐτοὺς τρόμος καὶ φόβος· μεγέθει βραχίονός σου ἀπολιθωθήτωσαν, ἕως ἂν παρέλθῃ ὁ λαός σου, Κύριε· ἕως ἂν παρέλθῃ ὁ λαός σου οὗτος, ὃν ἐκτήσω.
\vs{17}Εἰσαγαγὼν καταφύτευσον αὐτοὺς εἰς ὄρος κληρονομίας σου, εἰς ἕτοιμον κατοικητήριόν σου, ὃ κατηρτίσω, Κύριε, ἁγίασμα, Κύριε, ὃ ἡτοίμασαν αἱ χεῖρές σου.
\vs{18}Κύριος βασιλεύων τὸν αἰῶνα, καὶ ἐπʼ αἰῶνα, καὶ ἔτι.
\vs{19}Ὅτι εἰσῆλθεν ἵππος Φαραὼ σὺν ἅρμασιν καὶ ἀναβάταις εἰς θάλασσαν, καὶ ἐπήγαγεν ἐπʼ αὐτοὺς Κύριος τὸ ὕδωρ τῆς θαλάσσης· οἱ δὲ υἱοὶ Ἰσραὴλ ἐπορεύθησαν διὰ ξηρᾶς ἐν μέσῳ τῆς θαλάσσης.

\vs{20}Λαβοῦσα δὲ Μαριὰμ ἡ προφῆτις ἡ ἀδελφὴ Ἀαρὼν τὸ τύμπανον ἐν τῇ χειρὶ αὐτῆς, καὶ ἐξήλθοσαν πᾶσαι αἱ γυναῖκες ὀπίσω αὐτῆς μετὰ τυμπάνων καὶ χορῶν.
\vs{21}Ἐξῆρχε δὲ αὐτῶν Μαριὰμ, λέγουσα, ᾄσωμεν τῷ Κυρίῳ, ἐνδόξως γὰρ δεδόξασται· ἵππον καὶ ἀναβάτην ἔῤῥιψεν εἰς θάλασσαν.
\vs{22}Ἐξῆρε δὲ Μωυσῆς τοὺς υἱοὺς Ἰσραὴλ ἀπὸ θαλάσσης ἐρυθρᾶς, καὶ ἤγαγεν αὐτοὺς εἰς τὴν ἔρημον Σούρ· καὶ ἐπορεύοντο τρεῖς ἡμέρας ἐν τῇ ἐρήμῳ, καὶ οὐχ ηὕρισκον ὕδωρ, ὥστε πιεῖν.
\vs{23}Ἦλθον δὲ εἰς Μεῤῥᾶ, καὶ οὐκ ἠδύναντο πιεῖν ἐκ Μεῤῥᾶς· πικρὸν γὰρ ἦν· διὰ τοῦτο ἐπωνόμασε τὸ ὄνομα τοῦ τόπου ἐκείνου, Πικρία.
\vs{24}Καὶ διεγόγγυζεν ὁ λαὸς ἐπὶ Μωυσῇ, λέγοντες, τί πιόμεθα;
\vs{25}Ἐβόησε δὲ Μωυσῆς πρὸς Κύριον· καὶ ἔδειξεν αὐτῷ Κύριος ξύλον, καὶ ἐνέβαλεν αὐτὸ εἰς τὸ ὕδωρ, καὶ ἐγλυκάνθη τὸ ὕδωρ· ἐκεῖ ἔθετο αὐτῷ δικαιώματα καὶ κρίσεις· καὶ ἐκεῖ αὐτὸν ἐπείρασε,
\vs{26}καὶ εἶπεν, ἐὰν ἀκοῇ ἀκούσῃς τῆς φωνῆς Κυρίου τοῦ Θεοῦ σου, καὶ τὰ ἀρεστὰ ἐναντίον αὐτοῦ ποιήσῃς, καὶ ἐνωτίσῃ ταῖς ἐντολαῖς αὐτοῦ, καὶ φυλάξῃς πάντα τὰ δικαιώματα αὐτοῦ, πᾶσαν νόσον, ἣν ἐπήγαγον τοῖς Αἰγυπτίοις, οὐκ ἐπάξω ἐπὶ σέ· ἐγὼ γάρ εἰμι Κύριος ὁ Θεός σου ὁ ἰώμενός σε.
\vs{27}Καὶ ἤλθοσαν εἰς Αἰλείμ· καὶ ἦσαν ἐκεῖ δώδεκα πηγαὶ ὑδάτων, καὶ ἑβδομήκοντα στελέχη φοινίκων· παρενέβαλον δὲ ἐκεῖ παρὰ τὰ ὕδατα.

\ch{16}
Ἀπῄραν δὲ ἐξ Αἰλεὶμ, καὶ ἤλθοσαν πᾶσα συναγωγὴ υἱῶν Ἰσραὴλ εἰς τὴν ἔρημον Σὶν, ὅ ἐστιν ἀνὰ μέσον Αἰλεὶμ, καὶ ἀνὰ μέσον Σινά. τῇ δὲ πεντεκαιδεκάτῃ ἡμέρᾳ, τῷ μηνὶ τῷ δευτέρῳ ἐξεληλυθότων αὐτῶν ἐκ γῆς Αἰγύπτου,
\vs{2}διεγόγγυζε πᾶσα συναγωγὴ υἱῶν Ἰσραὴλ ἐπὶ Μωυσὴν καὶ Ἀαρών.
\vs{3}Καὶ εἶπεν πρὸς αὐτοὺς οἱ υἱοὶ Ἰσραήλ, ὄφελον ἀπεθάνομεν πληγέντες ὑπὸ Κυρίου ἐν γῇ Αἰγύπτῳ, ὅταν ἐκαθίσαμεν ἐπὶ τῶν λεβήτων τῶν κρεῶν, καὶ ἠσθίομεν ἄρτους εἰς πλησμονήν· ὅτι ἐξηγάγετε ἡμᾶς εἰς τὴν ἔρημον ταύτην, ἀποκτεῖναι πᾶσαν τὴν συναγωγὴν ταύτην ἐν λιμῷ.
\vs{4}Εἶπε δὲ Κύριος πρὸς Μωυσῆν, ἰδοὺ ἐγὼ ὕω ὑμῖν ἄρτους ἐκ τοῦ οὐρανοῦ· καὶ ἐξελεύσεται ὁ λαὸς, καὶ συλλέξουσι τὸ τῆς ἡμέρας εἰς ἡμέραν, ὅπως πειράσω αὐτοὺς εἰ πορεύσονται τῷ νόμῳ μου, ἢ οὔ·
\vs{5}Καὶ ἔσται ἐν τῇ ἡμέρᾳ τῇ ἕκτῃ, καὶ ἑτοιμάσουσιν ὃ ἐὰν εἰσενέγκωσι· καὶ ἔσται διπλοῦν ὃ ἐὰν συναγάγωσι τὸ καθʼ ἡμέραν εἰς ἡμέραν.
\vs{6}Καὶ εἶπε Μωυσῆς καὶ Ἀαρὼν πρὸς πάσαν συναγωγὴν υἱῶν Ἰσραὴλ, ἑσπέρας γνώσεσθε, ὅτι Κύριος ἐξήγαγεν ὑμᾶς ἐκ γῆς Αἰγύπτου,
\vs{7}καὶ πρωῒ ὄψεσθε τὴν δόξαν Κυρίου ἐν τῷ εἰσακοῦσαι τὸν γογγυσμὸν ὑμῶν ἐπὶ τῷ Θεῷ· ἡμεῖς δὲ τί ἐσμεν, ὅτι διαγογγύζετε καθʼ ἡμῶν;
\vs{8}Καὶ εἶπε Μωυσῆς, ἐν τῷ διδόναι Κύριον ὑμῖν ἑσπέρας κρέα φαγεῖν, καὶ ἄρτους τὸ πρωῒ εἰς πλησμονὴν, διὰ τὸ εἰσακοῦσαι Κύριον τὸν γογγυσμὸν ὑμῶν, ὃν ὑμεῖς διαγογγύζετε καθʼ ἡμῶν· ἡμεῖς δὲ τί ἐσμεν; οὐ γὰρ καθʼ ἡμῶν ἐστιν ὁ γογγυσμὸς ὑμῶν, ἀλλʼ ἢ κατὰ τοῦ Θεοῦ.

\vs{9}Εἶπε δὲ Μωυσῆς πρὸς Ἀαρὼν, εἶπον πάσῃ συναγωγῇ υἱῶν Ἰσραὴλ, προσέλθετε ἐναντίον τοῦ Θεοῦ· εἰσακήκοε γὰρ τὸν γογγυσμὸν ὑμῶν.
\vs{10}Ἡνίκα δὲ ἐλάλει Ἀαρὼν πάσῃ συναγωγῇ υἱῶν Ἰσραὴλ, καὶ ἐπεστράφησαν εἰς τὴν ἔρημον, καὶ ἡ δόξα Κυρίου ὤφθη ἐν νεφέλῃ.
\vs{11}καὶ ἐλάλησε Κύριος πρὸς Μωυσῆν, λέγων,
\vs{12}εἰσακήκοα τὸν γογγυσμὸν τῶν υἱῶν Ἰσραήλ· λάλησον πρὸς αὐτοὺς, λέγων, τὸ πρὸς ἑσπέραν ἔδεσθε κρέα, καὶ τὸ πρωῒ πλησθήσεσθε ἄρτων· καὶ γνώσεσθε, ὅτι ἐγὼ Κύριος ὁ Θεὸς ὑμῶν.
\vs{13}Ἐγένετο δὲ ἑσπέρα· καὶ ἀνέβη ὀρτυγομήτρα, καὶ ἐκάλυψε τὴν παρεμβολήν· τὸ πρωῒ ἐγένετο καταπαυομένης τῆς δρόσου κύκλῳ τῆς παρεμβολῆς.
\vs{14}Καὶ ἰδοὺ ἐπὶ πρόσωπον τῆς ἐρήμου λεπτὸν ὡσεὶ κόριον λευκὸν, ὡσεὶ πάγος ἐπὶ τῆς γῆς.
\vs{15}Ἰδόντες δὲ αὐτὸ οἱ υἱοὶ Ἰσραὴλ, εἶπαν ἕτερος τῷ ἑτέρῳ, τί ἐστι τοῦτο; οὐ γὰρ ᾔδεισαν τί ἦν· εἶπε δὲ Μωυσῆς αὐτοῖς, οὗτος ὁ ἄρτος, ὃν ἔδωκε Κύριος ὑμῖν φαγεῖν.
\vs{16}Τοῦτο τὸ ῥῆμα ὃ συνέταξε Κύριος· συναγάγετε ἀπʼ αὐτοῦ ἕκαστος εἰς τοὺς καθήκοντας γομὸρ, κατὰ κεφαλὴν κατὰ ἀριθμὸν ψυχῶν ὑμῶν, ἕκαστος σὺν τοῖς συσκηνίοις ὑμῶν συλλέξατε.
\vs{17}Ἐποίησαν δὲ οὕτως οἱ υἱοὶ Ἰσραήλ· καὶ συνέλεξαν ὁ τὸ πολὺ καὶ ὁ τὸ ἔλαττον.
\vs{18}Καὶ μετρήσαντες γομὸρ, οὐκ ἐπλεόνασεν ὁ τὸ πόλυ, καὶ ὁ τὸ ἔλαττον οὐκ ἠλαττόνησεν· ἕκαστος εἰς τοὺς καθήκοντας παρʼ ἑαυτῷ συνέλεξαν.
\vs{19}Εἶπε δὲ Μωυσῆς πρὸς αὐτοὺς, μηδεὶς καταλειπέτω ἀπʼ αὐτοῦ εἰς τὸ πρωΐ.

\vs{20}Καὶ οὐκ εἰσήκουσαν Μωυσῆ, ἀλλὰ κατέλιπόν τινες ἀπʼ αὐτοῦ εἰς τὸ πρωΐ· καὶ ἐξέζεσε σκώληκας, καὶ ἐπώζεσε· καὶ ἐπικράνθη ἐπʼ αὐτοῖς Μωυσῆς.
\vs{21}Καὶ συνέλεξαν αὐτὸ πρωῒ πρωῒ, ἕκαστος τὸ καθῆκον αὐτῷ· ἡνίκα δὲ διεθέρμαινεν ὁ ἥλιος, ἐτήκετο.
\vs{22}Ἐγένετο δὲ τῇ ἡμέρᾳ τῇ ἕκτῃ, συνέλεξαν τὰ δέοντα διπλᾶ, δύο γομὸρ τῷ ἑνί· εἰσήλθοσαν δὲ πάντες οἱ ἄρχοντες τῆς συναγωγῆς, καὶ ἀνήγγειλαν Μωυσῇ.
\vs{23}Εἶπε δὲ Μωυσῆς πρὸς αὐτούς, οὐ τοῦτο τὸ ῥῆμά ἐστιν ὃ ἐλάλησε Κύριος; σάββατα ἀνάπαυσις ἁγία τῷ Κυρίῳ αὔριον· ὅσα ἐὰν πέσσητε, πέσσετε· καὶ ὅσα ἐὰν ἕψητε, ἕψετε· καὶ πᾶν τὸ πλεονάζον καταλείπετε αὐτὸ εἰς ἀποθήκην εἰς τὸ πρωΐ.
\vs{24}Καὶ κατελίποσαν ἀπʼ αὐτοῦ εἰς ἕως πρωῒ, καθὼς συνέταξεν αὐτοῖς Μωυσῆς· καὶ οὐκ ἐπώζεσεν, οὐδὲ σκώληξ ἐγένετο ἐν αὐτῷ.
\vs{25}Εἶπε δὲ Μωυσῆς, φάγετε σήμερον· ἔστι γὰρ σάββατα αήμερον τῷ Κυρίῳ· οὐχ εὑρεθήσεται ἐν τῷ πεδίῳ.
\vs{26}Ἓξ ἡμέρας συλλέξετε· τῇ δὲ ἡμέρᾳ τῇ ἑβδόμῃ σάββατα, ὅτι οὐκ ἔσται ἐν αὐτῇ.
\vs{27}Ἐγένετο δὲ ἐν τῇ ἡμέρᾳ τῇ ἑβδόμῃ ἐξήθλοσάν τινες ἐκ τοῦ λαοῦ συλλέξαι, καὶ οὐχ εὗρον.
\vs{28}Εἶπε δὲ Κύριος πρὸς Μωυσῆν, ἕως τίνος οὐ βούλεσθε εἰσακούειν τὰς ἐντολάς μου, καὶ τὸν νόμον μου;
\vs{29}Ἴδετε, ὁ γὰρ Κύριος ἔδωκεν ὑμῖν σάββατα τὴν ἡμέραν ταύτην· διὰ τοῦτο αὐτὸς ἔδωκεν ὑμῖν τῇ ἡμέρᾳ τῇ ἕκτῃ ἄρτους δύο ἡμερῶν· καθίσεσθε ἕκαστος εἰς τοὺς οἴκους ὑμῶν· μηδεὶς ἐκπορευέσθω ἐκ τοῦ τόπου αὐτοῦ τῇ ἡμέρᾳ τῇ ἑβδόμῃ.
\vs{30}Καὶ ἐσαββάτισεν ὁ λαὸς τῇ ἡμέρᾳ τῇ ἑβδόμῃ.
\vs{31}Καὶ ἐπωνόμασαν αὐτὸ οἱ υἱοὶ Ἰσραὴλ τὸ ὄνομα αὐτοῦ, Μάν· ἦν δὲ ὡσεὶ σπέρμα κορίου λευκόν· τὸ δὲ γεῦμα αὐτοῦ ὡς ἐγκρὶς ἐν μέλιτι.
\vs{32}Εἶπε δὲ Μωυσῆς, τοῦτο τὸ ῥῆμα, ὃ συνέταξε Κύριος, πλήσατε τὸ γομὸρ τοῦ μὰν, εἰς ἀποθήκην εἰς τὰς γενεὰς ὑμῶν· ἵνα ἴδωσι τὸν ἄρτον, ὃν ἐφάγετε ὑμεῖς ἐν τῇ ἐρήμῳ, ὡς ἐξήγαγεν ὑμᾶς Κύριος ἐκ γῆς Αἰγύπτου.
\vs{33}Καὶ εἶπε Μωυσῆς πρὸς Ἀαρὼν, λάβε στάμνον χρυσοῦν ἕνα, καὶ ἔμβαλε εἰς αὐτὸν πλῆρες τὸ γομὸρ τοῦ μὰν, καὶ ἀποθήσεις αὐτὸ ἐναντίον τοῦ Θεοῦ, εἰς διατήρησιν εἰς τὰς γενεὰς ὑμῶν,
\vs{34}ὃν τρόπον συνέταξε Κύριος τῷ Μωυσῇ· καὶ ἀπέθηκεν Ἀαρὼν ἐναντίον τοῦ μαρτυρίου εἰς διατήρησιν.
\vs{35}Οἱ δὲ υἱοὶ Ἰσραὴλ ἔφαγον τὸ μὰν ἔτη τεσσαράκοντα, ἕως ἦλθον εἰς τὴν οἰκουμένην ἐφάγοσαν τὸ μὰν, ἕως παρεγένοντο εἰς μέρος τῆς Φοινίκης.
\vs{36}Τὸ δὲ γομὸρ τὸ δέκατον τῶν τριῶν μέτρων ἦν.

\ch{17}
Καὶ ἀπῇρε πᾶσα συναγωγὴ υἱῶν Ἰσραὴλ ἐκ τῆς ἐρήμου Σὶν κατὰ παρεμβολὰς αὐτῶν, διὰ ῥήματος Κυρίου· καὶ παρενεβάλοσαν ἐν Ῥαφιδείν· οὐκ ἦν δὲ ὕδωρ τῷ λαῷ πιεῖν.
\vs{2}Καὶ ἐλοιδορεῖτο ὁ λαὸς πρὸς Μωυσῆν, λέγοντες, δὸς ἡμῖν ὕδωρ, ἵνα πίωμεν· καὶ εἶπεν αὐτοῖς Μωυσῆς, τί λοιδορεῖσθέ μοι, καὶ τί πειράζετε Κύριον;
\vs{3}Ἐδίψησε δὲ ἐκεῖ ὁ λαὸς ὕδατι· καὶ διεγόγγυσεν ἐκεῖ ὁ λαὸς πρὸς Μωυσῆν, λέγοντες, ἱνατί τοῦτο; ἀνεβίβασας ἡμᾶς ἐξ Αἰγύπτου ἀποκτεῖναι ἡμᾶς καὶ τὰ τέκνα ἡμῶν καὶ τὰ κτήνη τῷ δίψει;
\vs{4}Ἐβόησε δὲ Μωυσῆς πρὸς Κύριον, λέγων, τί ποιήσω τῷ λαῷ τούτῳ; ἔτι μικρὸν, καὶ καταλιθοβολὴσουσί με.
\vs{5}Καὶ εἶπε Κύριος πρὸς Μωυσῆν, προπορεύου τοῦ λαοῦ τούτου· λάβε δὲ σεαυτῷ ἀπὸ τῶν πρεσβυτέρων τοῦ λαοῦ· καὶ τὴν ῥάβδον, ἐν ᾗ ἐπάταξας τὸν ποταμὸν, λάβε ἐν τῇ χειρί σου, καὶ πορεύσῃ.
\vs{6}Ὅδε ἐγὼ ἕστηκα ἐκεῖ πρὸ τοῦ σὲ ἐπὶ τῆς πέτρας ἐν Χωρήβ· καὶ πατάξεις τὴν πέτραν, καὶ ἐξελεύσεται ἐξ αὐτῆς ὕδωρ, καὶ πίεται ὁ λαός. Ἐποίησε δὲ Μωυσῆς οὕτως ἐναντίον τῶν υἱῶν Ἰσραήλ.
\vs{7}Καὶ ἐπωνόμασε τὸ ὄνομα τοῦ τόπου ἐκείνου, Πειρασμὸς, καὶ Λοιδόρησις, διὰ τὴν λοιδορίαν τῶν υἱῶν Ἰσραὴλ, καὶ διὰ τὸ πειράζειν Κύριον, λέγοντας, εἰ ἔστι Κύριος ἐν ἡμῖν, ἢ οὔ;

\vs{8}Ἦλθε δὲ Ἀμαλὴκ καὶ ἐπολέμει Ἰσραὴλ ἐν Ῥαφιδείν.
\vs{9}Εἶπε δὲ Μωυσῆς τῷ Ἰησοῖ, Ἐπίλεξον σεαυτῷ ἄνδρας δυνατοὺς, καὶ ἐξελθὼν παράταξαι τῷ Ἀμαλὴκ αὔριον· καὶ ἰδοὺ ἐγὼ ἕστηκα ἐπὶ τῆς κορυφῆς τοῦ βουνοῦ, καὶ ἡ ῥάβδος τοῦ Θεοῦ ἐν τῇ χειρί μου.
\vs{10}Καὶ ἐποίησεν Ἰησοῦς καθάπερ εἶπεν αὐτῷ Μωυσῆς, καὶ ἐξελθὼν παρετάξατο τῷ Ἀμαλήκ· καὶ Μωυσῆς καὶ Ἀαρὼν καὶ Ὢρ ἀνέβησαν ἐπὶ τὴν κορυφὴν τοῦ βουνοῦ.
\vs{11}Καὶ ἐγένετο ὅταν ἐπῇρε Μωυσῆς τὰς χεῖρας, κατίσχυεν Ἰσραήλ· ὅταν δὲ καθῆκε τὰς χεῖρας, κατίσχυεν Ἀμαλήκ.
\vs{12}Αἱ δὲ χεῖρες Μωυσῆ βαρεῖαι· καὶ λαβόντες λίθον ὑπέθηκαν ὑπʼ αὐτὸν, καὶ ἐκάθητο ἐπʼ αὐτοῦ· καὶ Ἀαρὼν καὶ Ὢρ ἐστήριζον τὰς χεῖρας αὐτοῦ ἐντεῦθεν εἷς, καὶ ἐντεῦθεν εἷς· καὶ ἐγένοντο αἱ χεῖρες Μωυσῆ ἐστηριγμέναι ἕως δυσμῶν ἡλίου.
\vs{13}Καὶ ἐτρέψατο Ἰησοῦς τὸν Ἀμαλὴκ, καὶ πάντα τὸν λαὸν αὐτοῦ ἐν φόνῳ μαχαίρας.
\vs{14}Εἶπε δὲ Κύριος πρὸς Μωυσῆν, Κατάγραψον τοῦτο εἰς μνημόσυνον εἰς βιβλίον, καὶ δὸς εἰς τὰ ὦτα Ἰησοῖ· ὅτι ἀλοιφῇ ἐξαλείψω τὸ μνημόσυνον Ἀμαλὴκ ἐκ τῆς ὑπὸ τὸν οὐρανόν.
\vs{15}Καὶ ᾠκοδόμησε Μωυσῆς θυσιαστήριον Κυρίῳ· καὶ ἐπωνόμασε τὸ ὄνομα αὐτοῦ, Κύριος καταφυγή μου.
\vs{16}Ὅτι ἐν χειρὶ κρυφαίᾳ πολεμεῖ Κύριος ἐπὶ Ἀμαλὴκ ἀπὸ γενεῶν εἰς γενεάς.

\ch{18}
Ἤκουσε δὲ Ἰοθὸρ ἱερεὺς Μαδιὰμ ὁ γαμβρὸς Μωυσῆ πάντα ὅσα ἐποίησε Κύριος Ἰσραὴλ τῷ ἑαυτοῦ λαῷ· ἐξήγαγε γὰρ Κύριος τὸν Ἰσραὴλ ἐξ Αἰγύπτου.
\vs{2}Ἔλαβε δὲ Ἰοθὸρ ὁ γαμβρὸς Μωυσῆ Σεπφώραν τὴν γυναῖκα Μωυσῆ μετὰ τὴν ἄφεσιν αὐτῆς,
\vs{3}καὶ τοὺς δύο υἱοὺς αὐτῆς· ὄνομα τῷ ἑνὶ αὐτῶν Γηρσάμ, λέγων, πάροικος ἤμην ἐν γῇ ἀλλοτρίᾳ·
\vs{4}καὶ τὸ ὄνομα τοῦ δευτέρου Ἐλίεζερ, λέγων, ὁ γὰρ Θεὸς τοῦ πατρός μου βοηθός μου, καὶ ἐξείλατό με ἐκ χειρὸς Φαραώ.
\vs{5}Καὶ ἐξῆλθεν Ἰοθὸρ ὁ γαμβρὸς Μωυσῆ καὶ οἱ υἱοὶ καὶ ἡ γυνὴ πρὸς Μωυσῆν εἰς τὴν ἔρημον, οὗ παρενέβαλεν ἐπʼ ὄρους τοῦ Θεοῦ.
\vs{6}Ἀνηγγέλη δὲ Μωυσῇ, λέγοντες, ἰδοὺ ὁ γαμβρός σου Ἰοθὸρ παραγίνεται πρὸς σέ, καὶ ἡ γυνὴ καὶ οἱ δύο υἱοί σου μετʼ αὐτοῦ.
\vs{7}Ἐξῆλθε δὲ Μωυσῆς εἰς συνάντησιν τῷ γαμβρῷ, καὶ προσεκύνησεν αὐτῷ, καὶ ἐφίλησεν αυτὸν, καὶ ἠσπάσαντο ἀλλήλους, καὶ εἰσήγαγεν αὐτοὺς εἰς τὴν σκηνήν.
\vs{8}Καὶ διηγήσατο Μωυσῆς τῷ γαμβρῷ πάντα ὅσα ἐποίησε Κύριος τῷ Φαραῷ καὶ πᾶσι τοῖς Αἰγυπτίοις ἕνεκεν τοῦ Ἰσραήλ, καὶ πάντα τὸν μόχθον τὸν γενόμενον αὐτοῖς ἐν τῇ ὁδῷ, καὶ ὅτι ἐξείλατο αὐτοὺς Κύριος ἐκ χειρὸς Φαραὼ, καὶ ἐκ χειρὸς τῶν Αἰγυπτίων.
\vs{9}Ἐξέστη δὲ Ἰοθὸρ ἐπὶ πᾶσι τοῖς ἀγαθοῖς οἷς ἐποίησεν αὐτοῖς Κύριος, ὅτι ἐξείλατο αὐτοὺς ἐκ χειρὸς Αἰγυπτίων καὶ ἐκ χειρὸς Φαραώ.
\vs{10}Καὶ εἶπεν Ἰοθὸρ, εὐλογητὸς Κύριος, ὅτι ἐξείλατο αὐτοὺς ἐκ χειρὸς Αἰγυπτίων καὶ ἐκ χειρὸς Φαραώ.
\vs{11}Νῦν ἔγνων ὅτι μέγας Κύριος παρὰ πάντας τοὺς θεούς ἕνεκεν τούτου, ὅτι ἐπέθεντο αὐτοῖς.
\vs{12}Καὶ ἔλαβεν Ἰοθὸρ ὁ γαμβρὸς Μωυσῆ ὁλοκαυτώματα καὶ θυσίας τῷ Θεῷ· παρεγένετο δὲ Ἀαρὼν καὶ πάντες οἱ πρεσβύτεροι Ἰσραὴλ συμφαγεῖν ἄρτον μετὰ τοῦ γαμβροῦ Μωυσῆ, ἐναντίον τοῦ Θεοῦ.

\vs{13}Καὶ ἐγένετο μετὰ τὴν ἐπαύριον συνεκάθισε Μωυσῆς κρίνειν τὸν λαόν· παρειστήκει δὲ πᾶς ὁ λαὸς Μωυσῇ ἀπὸ πρωΐθεν ἕως δείλης.
\vs{14}Καὶ ἰδὼν Ἰοθὸρ πάντα ὅσα ποιεῖ τῷ λαῷ, λέγει, τί τοῦτο ὃ σὺ ποιεῖς τῷ λαῷ; διατί σὺ κάθησαι μόνος, πᾶς δὲ ὁ λαὸς παρέστηκέ σοι ἀπὸ πρωΐθεν ἕως δείλης;
\vs{15}Καὶ λέγει Μωυσῆς τῷ γαμβρῶ, Ὅτι παραγίνεται πρός με ὁ λαὸς ἐκζητῆσαι κρίσιν παρὰ τοῦ Θεοῦ.
\vs{16}Ὅταν γὰρ γένηται αὐτοῖς ἀντιλογία, καὶ ἔλθωσι πρός με, διακρίνω ἕκαστον, καὶ συμβιβάζω αὐτοὺς τὰ προστάγματα τοῦ Θεοῦ καὶ τὸν νόμον αὐτοῦ.
\vs{17}Εἶπε δὲ ὁ γαμβρὸς Μωυσῆ πρὸς αὐτὸν, οὐκ ὀρθῶς σὺ ποιεῖς τὸ ῥῆμα τοῦτο.
\vs{18}Φθορᾷ καταφθαρήσῃ ἀνυπομονήτῳ καὶ σὺ, καὶ πᾶς ὁ λαὸς οὗτος, ὅς ἐστι μετὰ σοῦ· βαρύ σοι τὸ ῥῆμα τοῦτο· οὐ δυνήσῃ ποιεῖν σὺ μόνος.
\vs{19}Νῦν οὖν ἄκουσόν μου, καὶ συμβουλεύσω σοι, καὶ ἔσται ὁ Θεὸς μετὰ σοῦ· γίνου σὺ τῷ λαῷ τὰ πρὸς τὸν Θεὸν, καὶ ἀνοίσεις τοὺς λόγους αὐτῶν πρὸς τὸν Θεόν.
\vs{20}Καὶ διαμαρτύρῇ αὐτοῖς τὰ προστάγματα τοῦ Θεοῦ καὶ τὸν νόμον αὐτοῦ, καὶ σημανεῖς αὐτοῖς τὰς ὁδοὺς ἐν αἷς πορεύσονται ἐν αὐταῖς, καὶ τὰ ἔργα ἃ ποιήσουσι.
\vs{21}Καὶ σὺ σεαυτῷ σκέψαι ἀπὸ παντὸς τοῦ λαοῦ ἄνδρας δυνατοὺς, θεοσεβεῖς, ἄνδρας δικαίους, μισοῦντας ὑπερηφανίαν, καὶ καταστήσεις ἐπʼ αὐτῶν χιλιάρχους καὶ ἑκατοντάρχους καὶ πεντηκοντάρχους καὶ δεκαδάρχους.
\vs{22}Καὶ κρινοῦσι τὸν λαὸν πᾶσαν ὥραν· τὸ δὲ ῥῆμα τὸ ὑπέρογκον ἀνοίσουσιν ἐπὶ σὲ· τὰ δὲ βραχέα τῶν κριμάτων κρινοῦσιν αὐτοί· καὶ κουφιοῦσιν ἀπὸ σοῦ, καὶ συναντιλήψονταί σοι.
\vs{23}Ἐὰν τὸ ῥῆμα τοῦτο ποιήσῃς, κατισχύσει σε ὁ Θεὸς, καὶ δυνήσῃ παραστῆναι, καὶ πᾶς ὁ λαὸς οὗτος εἰς τὸν ἑαυτοῦ τόπον μετʼ εἰρήνης ἥξει.
\vs{24}Ἤκουσε δὲ Μωυσῆς τῆς φωνῆς τοῦ γαμβροῦ, καὶ ἐποίησεν ὅσα εἶπεν αὐτῷ.
\vs{25}Καὶ ἐπέλεξε Μωυσῆς ἄνδρας δυνατοὺς ἀπὸ παντὸς Ἰσραὴλ, καὶ ἐποίησεν αὐτοὺς ἐπʼ αὐτῶν χιλιάρχους καὶ ἑκατοντάρχους καὶ πεντηκοντάρχους καὶ δεκαδάρχους.
\vs{26}Καὶ ἐκρίνοσαν τὸν λαὸν πᾶσαν ὥραν· πᾶν δὲ ῥῆμα ὑπέρογκον ἀνεφέροσαν ἐπὶ Μωυσῆν· πᾶν δὲ ῥῆμα ἐλαφρὸν ἐκρίνοσαν αὐτοί.
\vs{27}Ἐξαπέστειλε δὲ Μωυσῆς τὸν ἑαυτοῦ γαμβρὸν, καὶ ἀπῆλθεν εἰς τὴν γῆν αὐτοῦ.

\ch{19}
Τοῦ δὲ μηνὸς τοῦ τρίτου τῆς ἐξόδου τῶν υἱῶν Ἰσραὴλ ἐκ γῆς Αἰγύπτου τῇ ἡμέρᾳ ταύτῃ, ἤλθοσαν εἰς τὴν ἔρημον τοῦ Σινά.
\vs{2}Καὶ ἀπῆραν ἐκ Ῥαφιδεὶν, καὶ ἤλθοσαν εἰς τὴν ἔρημον τοῦ Σινὰ, καὶ παρενέβαλεν ἐκεῖ Ἰσραὴλ κατέναντι τοῦ ὄρους.
\vs{3}Καὶ Μωυσῆς ἀνέβη εἰς τὸ ὄρος τοῦ Θεοῦ· καὶ ἐκάλεσεν αὐτὸν ὁ Θεὸς ἐκ τοῦ ὄρους, λέγων, τάδε ἐρεῖς τῷ οἴκῳ Ἰακὼβ, καὶ ἀναγγελεῖς τοῖς υἱοῖς Ἰσραήλ.
\vs{4}Αὐτοὶ ἑωράκατε ὅσα πεποίηκα τοῖς Αἱγυπτίοις, καὶ ἀνέλαβον ὑμᾶς ὡσεὶ ἐπὶ πτερύγων ἀετῶν, καὶ προσηγαγόμην ὑμᾶς πρὸς ἐμαυτόν.
\vs{5}Καὶ νῦν ἐὰν ἀκοῇ ἀκούσητε τῆς ἐμῆς φωνῆς, καὶ φυλάξητε τὴν διαθήκην μου, ἔσεσθέ μοι λαὸς περιούσιος ἀπὸ πάντων τῶν ἐθνῶν· ἐμὴ γάρ ἐστι πᾶσα ἡ γῆ.
\vs{6}Ὑμεῖς δὲ ἔσεσθέ μοι βασίλειον ἱεράτευμα καὶ ἔθνος ἅγιον· ταῦτα τὰ ῥήματα ἐρεῖς τοῖς υἱοῖς Ἰσραήλ.
\vs{7}Ἦλθε δὲ Μωυσῆς, καὶ ἐκάλεσεν τοὺς πρεσβυτέρους τοῦ λαοῦ· καὶ παρέθηκεν αὐτοῖς πάντας τοὺς λόγους τούτους, οὓς συνέταξεν αὐτοῖς ὁ Θεός.
\vs{8}Ἀπεκρίθη δὲ πᾶς ὁ λαὸς ὁμοθυμαδὸν, καὶ εἶπαν, πάντα ὅσα εἶπεν ὁ Θεὸς, ποιήσομεν καὶ ἀκουσόμεθα· ἀνήνεγκε δὲ Μωυσῆς τοὺς λόγους τούτους πρὸς τὸν Θεόν.
\vs{9}Εἶπε δὲ Κύριος πρὸς Μωυσῆν, ἰδοὺ ἐγὼ παραγίνομαι πρὸς σὲ ἐν στύλῳ νεφέλης, ἵνα ἀκούσῃ ὁ λαὸς λαλοῦντός μου πρὸς σὲ, καὶ σοὶ πιστεύσωσιν εἰς τὸν αἰῶνα· ἀνήγγειλε δὲ Μωυσῆς τὰ ῥήματα τοῦ λαοῦ πρὸς Κύριον.
\vs{10}Εἶπε δὲ Κύριος πρὸς Μωυσῆν, Καταβὰς διαμάρτυραι τῷ λαῷ, καὶ ἅγνισον αὐτοὺς σήμερον καὶ αὔριον, καὶ πλυνάτωσαν τὰ ἱμάτια,
\vs{11}καὶ ἔστωσαν ἕτοιμοι εἰς τὴν ἡμέραν τὴν τρίτην· τῇ γὰρ ἡμέρᾳ τῇ τρίτῃ καταβήσεται Κύριος ἐπὶ τὸ ὄρος τὸ Σινὰ, ἐναντίον παντὸς τοῦ λαοῦ.
\vs{12}Καὶ ἀφοριεῖς τὸν λαὸν κύκλῳ, λέγων, προσέχετε ἑαυτοῖς τοῦ ἀναβῆναι εἰς τὸ ὄρος, καὶ θίγειν τι αὐτοῦ· πᾶς ὁ ἁψάμενος τοῦ ὄρους, θανάτῳ τελευτήσει.
\vs{13}Οὐχ ἅψεται αὐτοῦ χείρ· ἐν γὰρ λίθοις λιθοβοληθήσεται, ἢ βολίδι κατατοξευθήσεται· ἐάν τε κτῆνος ἐάν τε ἄνθρωπος, οὐ ζήσεται· ὅταν αἱ φωναὶ καὶ αἱ σάλπιγγες καὶ ἡ νεφέλη ἀπέλθῃ ἀπὸ τοῦ ὄρους, ἐκεῖνοι ἀναβήσονται ἐπὶ τὸ ὄρος.

\vs{14}Κατέβη δὲ Μωυσῆς ἐκ τοῦ ὄρους πρὸς τὸν λαὸν, καὶ ἡγίασεν αὐτούς· καὶ ἔπλυναν τὰ ἱμάτια.
\vs{15}Καὶ εἶπε τῷ λαῷ, γίνεσθε ἕτοιμοι, τρεῖς ἡμέρας μὴ προσέλθητε γυναικί.
\vs{16}Ἐγένετο δὲ τῇ ἡμέρᾳ τῇ τρίτῃ γενηθέντος πρὸς ὄρθρον, καὶ ἐγένοντο φωναὶ καὶ ἀστραπαὶ καὶ νεφέλη γνοφώδης ἐπʼ ὄρους Σινά· φωνὴ τῆς σάλπιγγος ἤχει μέγα· καὶ ἐπτοήθη πᾶς ὁ λαὸς ὁ ἐν τῇ παρεμβολῇ.
\vs{17}Καὶ ἐξήγαγε Μωυσῆς τὸν λαὸν εἰς συνάντησιν τοῦ Θεοῦ ἐκ τῆς παρεμβολῆς· καὶ παρέστησαν ὑπὸ τὸ ὄρος.
\vs{18}Τὸ ὄρος τὸ Σινὰ ἐκαπνίζετο ὅλον, διὰ τὸ καταβεβηκέναι ἐπʼ αὐτὸ τὸν Θεὸν ἐν πυρί· καὶ ἀνέβαινεν ὁ καπνὸς, ὡσεὶ καπνὸς καμίνου· καὶ ἐξέστη πᾶς ὁ λαὸς σφόδρα.
\vs{19}Ἐγίνοντο δὲ αἱ φωναὶ τῆς σάλπιγγος προβαίνουσαι ἰσχυρότεραι σφόδρα. Μωυσῆς ἐλάλησεν, ὁ δὲ Θεὸς ἀπεκρίνατο αὐτῷ φωνῇ.
\vs{20}Κατέβη δὲ Κύριος ἐπὶ τὸ ὄρος τὸ Σινὰ ἐπὶ τὴν κορυφὴν τοῦ ὄρους· καὶ ἐκάλεσε Κύριος Μωυσῆν ἐπὶ τὴν κορυφὴν τοῦ ὄρους· καὶ ἀνέβη Μωυσῆς.
\vs{21}Καὶ εἶπεν ὁ Θεὸς πρὸς Μωυσῆν, λέγων, καταβὰς διαμάρτυραι τῷ λαῷ, μή ποτε ἐγγίσωσι πρὸς τὸν Θεὸν κατανοῆσαι, καὶ πέσωσιν ἐξ αὐτῶν πλῆθος·
\vs{22}Καὶ οἱ ἱερεῖς οἱ ἐγγίζοντες Κυρίῳ τῷ Θεῷ ἁγιασθήτωσαν, μήποτε ἀπαλλάξῃ ἀπʼ αὐτῶν Κύριος.

\vs{23}Καὶ εἶπε Μωυσῆς πρὸς τὸν Θεὸν, οὐ δυνήσεται ὁ λαὸς προσαναβῆναι πρὸς τὸ ὄρος τὸ Σινά· σὺ γὰρ διαμεμαρτύρησαι ἡμῖν, λέγων, ἀφόρισαι τὸ ὄρος, καὶ ἁγίασαι αὐτό.
\vs{24}Εἴπε δὲ αὐτῷ Κύριος, βάδιζε, κατάβηθι, καὶ ἀνάβηθι σὺ καὶ Ἀαρὼν μετὰ σοῦ· οἱ δὲ ἱερεῖς καὶ ὁ λαὸς μὴ βιαζέσθωσαν ἀναβῆναι πρὸς τὸν Θεὸν, μὴ ποτε ἀπολέσῃ ἀπʼ αὐτῶν Κύριος.
\vs{25}Κατέβη δὲ Μωυσῆς πρὸς τὸν λαὸν, καὶ εἶπεν αὐτοῖς.

\ch{20}
Καὶ ἐλάλησε Κύριος πάντας τοὺς λόγους τούτους, λέγων,
\vs{2}ἐγώ εἰμι Κύριος ὁ Θεός σου, ὅστις ἐξήγαγόν σε ἐκ γῆς Αἰγύπτου, ἐξ οἴκου δουλείας.
\vs{3}Οὐκ ἔσονταί σοι θεοὶ ἕτεροι πλὴν ἐμοῦ.
\vs{4}Οὐ ποιήσεις σεαυτῷ εἴδωλον, οὐδὲ παντὸς ὁμοίωμα, ὅσα ἐν τῷ οὐρανῷ ἄνω, καὶ ὅσα ἐν τῇ γῇ κάτω, καὶ ὅσα ἐν τοῖς ὕδασιν ὑποκάτω τῆς γῆς.
\vs{5}Οὐ προσκυνήσεις αὐτοῖς, οὐδὲ μὴ λατρεύσεις αὐτοῖς· ἐγὼ γάρ εἰμι Κύριος ὁ Θεός σου, Θεὸς ζηλωτὴς, ἀποδιδοὺς ἁμαρτίας πατέρων ἐπὶ τέκνα, ἕως τρίτης καὶ τετάρτης γενεᾶς τοῖς μισοῦσί με,
\vs{6}καὶ ποιῶν ἔλεος εἰς χιλιάδας τοῖς ἀγαπῶσί με, καὶ τοῖς φυλάσσουσι τὰ προστάγματά μου.
\vs{7}Οὐ λήψῃ τὸ ὄνομα Κυρίου τοῦ Θεοῦ σου ἐπὶ ματαίῳ· οὐ γὰρ μὴ καθαρίσῃ Κύριος ὁ Θεός σου τὸν λαμβάνοντα τὸ ὄνομα αὐτοῦ ἐπὶ ματαίῳ.
\vs{8}Μνήσθητι τὴν ἡμέραν τῶν σαββάτων ἁγιάζειν αὐτήν.
\vs{9}Ἓξ ἡμέρας ἐργᾷ, καὶ ποιήσεις πάντα τὰ ἔργα σου.
\vs{10}Τῇ δὲ ἡμέρᾳ τῇ ἑβδόμῃ, σάββατα Κυρίῳ τῷ Θεῷ σου· οὐ ποιήσεις ἐν αὐτῇ πᾶν ἔργον σὺ, καὶ ὁ υἱός σου, καὶ ἡ θυγάτηρ σου, ὁ παῖς σου, καὶ ἡ παιδίσκη σου, ὁ βοῦς σου, καὶ τὸ ὑποζύγιόν σου, καὶ πᾶν κτῆνός σου, καὶ ὁ προσήλυτος ὁ παροικῶν ἐν σοί.
\vs{11}Ἐν γὰρ ἓξ ἡμέραις ἐποίησε Κύριος τὸν οὐρανὸν καὶ τὴν γῆν καὶ τὴν θὰλασσαν καὶ πάντα τὰ ἐν αὐτοῖς, καὶ κατέπαυσε τῇ ἡμέρᾳ τῇ ἑβδόμῃ· διὰ τοῦτο εὐλόγησε Κύριος τὴν ἡμέραν τὴν ἑβδόμην, καὶ ἡγίασεν αὐτήν.
\vs{12}Τίμα τὸν πατέρα σου, καὶ τὴν μητέρα σου, ἵνα εὖ σοι γένηται, καὶ ἵνα μακροχρόνιος γένῃ ἐπὶ τῆς γῆς τῆς ἀγαθῆς, ἧς Κύριος ὁ Θεός σου δίδωσί σοι.
\vs{13}Οὐ μοιχεύσεις.
\vs{14}Οὐ κλέψεις.
\vs{15}Οὐ φονεύσεις.
\vs{16}Οὐ ψευδομαρτυρήσεις κατὰ τοῦ πλησίον σου μαρτυρίαν ψευδῆ.
\vs{17}Οὐκ ἐπιθυμήσεις τὴν γυναῖκα τοῦ πλησίον σου· οὐκ ἐπιθυμήσεις τὴν οἰκίαν τοῦ πλησίον σου, οὔτε τὸν ἀγρὸν αὐτοῦ, οὔτε τὸν παῖδα αὐτοῦ, οὔτε τὴν παιδίσκην αὐτοῦ, οὔτε τοῦ βοὸς αὐτοῦ, οὔτε τοῦ ὑποζυγίου αὐτοῦ, οὔτε παντὸς κτήνους αὐτοῦ, οὔτε ὅσα τῷ πλησίον σου ἐστί.

\vs{18}Καὶ πᾶς ὁ λαὸς ἑώρα τὴν φωνὴν, καὶ τὰς λαμπάδας, καὶ τὴν φωνὴν τῆς σάλπιγγος, καὶ τὸ ὄρος τὸ καπνίζον· φοβηθέντες δὲ πᾶς ὁ λαὸς ἔστησαν μακρόθεν.
\vs{19}Καὶ εἶπαν πρὸς Μωυσῆν, λάλησον σὺ ἡμῖν, καὶ μὴ λαλείτω πρὸς ἡμᾶς ὁ Θεὸς, μὴ ἀποθάνωμεν.
\vs{20}Καὶ λέγει αὐτοῖς Μωυσῆς, θαρσεῖτε· ἕνεκεν γὰρ τοῦ πειράσαι ὑμᾶς παρεγενήθη ὁ Θεὸς πρὸς ὑμᾶς, ὅπως ἂν γένηται ὁ φόβος αὐτοῦ ἐν ὑμῖν, ἵνα μὴ ἁμαρτάνητε.
\vs{21}Εἱστήκει δὲ ὁ λαὸς μακρόθεν, Μωυσῆς δὲ εἰσῆλθεν εἰς τὸν γνόφον, οὗ ἦν ὁ Θεός.
\vs{22}Εἶπε δὲ Κύριος πρὸς Μωυσῆν, τάδε ἐρεῖς τῷ οἴκῳ Ἰακὼβ, καὶ ἀναγγελεῖς τοῖς υἱοῖς Ἰσραήλ· ὑμεῖς ἑωράκατε, ὅτι ἐκ τοῦ οὐρανοῦ λελάληκα πρὸς ὑμᾶς.
\vs{23}Οὐ ποιήσετε ὑμῖν αὐτοῖς θεοὺς ἀργυροῦς, καὶ θεοὺς χρυσοῦς οὐ ποιήσετε ὑμῖν αὑτοῖς.
\vs{24}Θυσιαστήριον ἐκ γῆς ποιήσετέ μοι, καὶ θύσετε ἐπʼ αὐτοῦ τὰ ὁλοκαυτώματα ὑμῶν, καὶ τὰ σωτήρια ὑμῶν, καὶ τὰ πρόβατα, καὶ τοὺς μόσχους ὑμῶν ἐν παντὶ τόπῳ, οὗ ἐὰν ἐπονομάσω τὸ ὄνομά μου ἐκεῖ, καὶ ἥξω πρὸς σὲ, καὶ εὐλογήσω σε.
\vs{25}Ἐὰν δὲ θυσιαστήριον ἐκ λίθων ποιῇς μοι, οὐκ οἰκοδομήσεις αὐτοὺς τμητούς· τὸ γὰρ ἐγχειρίδιόν σου ἐπιβέβληκας ἐπʼ αὐτοὺς, καὶ μεμίανται.
\vs{26}Οὐκ ἀναβήσῃ ἐν ἀναβαθμίσιν ἐπὶ τὸ θυσιαστήριόν μου, ὅπως ἂν μὴ ἀποκαλύψῃς τὴν ἀσχημοσύνην σου ἐπʼ αὐτοῦ.

\ch{21}
Καὶ ταῦτα τὰ δικαιώματα, ἃ παραθήσῃ ἐνώπιον αὐτῶν.
\vs{2}Ἐὰν κτήσῃ παῖδα Ἐβραῖον, ἓξ ἔτη δουλεύσει σοι· τῷ δὲ ἑβδόμῳ ἔτει ἀπελεύσεται ἐλεύθερος δωρεάν.
\vs{3}Ἐὰν αὐτὸς μόνος εἰσέλθῃ, καὶ μόνος ἐξελεύσεται· ἐὰν δὲ γυνὴ συνεισέλθῃ μετʼ αὐτοῦ, ἐξελεύσεται καὶ ἡ γυνὴ αὐτοῦ.
\vs{4}Καὶ ἐὰν δὲ ὁ κύριος δῷ αὐτῷ γυναῖκα, καὶ τέκῃ αὐτῷ υἱοὺς ἢ θυγατέρας, ἡ γυνὴ καὶ τὰ παιδία ἔσται τῷ κυρίῳ αὐτοῦ, αὐτὸς δὲ μόνος ἐξελεύσεται.
\vs{5}Ἐὰν δὲ ἀποκριθεὶς εἴπῃ ὁ παῖς, ἠγάπηκα τὸν κύριόν μου, καὶ τὴν γυναῖκα, καὶ τὰ παιδία, οὐκ ἀποτρέχω ἐλεύθερος·
\vs{6}προσάξει αὐτὸν ὁ κύριος αὐτοῦ πρὸς τὸ κριτήριον τοῦ Θεοῦ, καὶ τότε προσάξει αὐτὸν ἐπὶ τὴν θύραν ἐπὶ τὸν σταθμὸν, καὶ τρυπήσει ὁ κύριος αὐτοῦ τὸ οὖς τῷ ὀπητίῳ, καὶ δουλεύσει αὐτῷ εἰς τὸν αἰῶνα.

\vs{7}Ἐὰν δέ τις ἀποδῶται τὴν ἑαυτοῦ θυγατέρα οἰκέτιν, οὐκ ἀπελεύσεται, ὥσπερ ἀποτρέχουσιν αἱ δοῦλαι.
\vs{8}Ἐὰν μὴ εὐαρεστήσῃ τῷ κυρίῳ αὐτῆς, ἣ αὐτῷ καθωμολογήσατο, ἀπολυτρώσει αὐτήν· ἔθνει δὲ ἀλλοτρίῳ οὐ κύριός ἐστι πωλεῖν αὐτὴν, ὅτι ἠθέτησεν ἐν αὐτῇ.
\vs{9}Ἐὰν δὲ τῷ υἱῷ καθομολογήσηται αὐτὴν, κατὰ τὸ δικαίωμα τῶν θυγατέρων ποιήσει αὐτῇ.
\vs{10}Ἐὰν δὲ ἄλλην λάβῃ ἑαυτῷ, τὰ δέοντα καὶ τὸν ἱματισμὸν καὶ τὴν ὁμιλίαν αὐτῆς οὐκ ἀποστερήσει.
\vs{11}Ἐὰν δὲ τὰ τρία ταῦτα μὴ ποιήσῃ αὐτῇ, ἐξελεύσεται δωρεὰν ἄνευ ἀργυρίου.
\vs{12}Ἐὰν δὲ πατάξῃ τις τινὰ, καὶ ἀποθάνῃ, θανάτῳ θανατούσθω.
\vs{13}Ὁ δὲ οὐχ ἑκὼν, ἀλλὰ ὁ Θεὸς παρέδωκεν εἰς τὰς χεῖρας αὐτοῦ, δώσω σοι τόπον οὗ φεύξεται ἐκεῖ ὁ φονεύσας.
\vs{14}Ἐὰν δέ τις ἐπιθῆται τῷ πλησίον ἀποκτεῖναι αὐτὸν δόλῳ, καὶ καταφύγῃ, ἀπὸ τοῦ θυσιαστηρίου μου λήψῃ αὐτὸν θανατῶσαι.
\vs{15}Ὃς τύπτει πατέρα αὐτοῦ ἢ μητέρα αὐτοῦ, θανάτῳ θανατούσθω.
\vs{16}Ὁ κακολογῶν πατέρα αὐτοῦ ἢ μητέρα αὐτοῦ, τελευτήσει θανάτῳ.
\vs{17}Ὃς ἐὰν κλέψῃ τις τινὰ τῶν υἱῶν Ἰσραὴλ, καὶ καταδυναστεύσας αὐτὸν ἀποδῶται, καὶ εὑρεθῇ ἐν αὐτῷ, θανάτῳ τελευτάτω.
\vs{18}Ἐὰν δὲ λοιδορῶνται δύο ἄνδρες, καὶ πατάξωσι τὸν πλησίον λίθῳ ἢ πυγμῇ, καὶ μὴ ἀποθάνῃ, κατακλιθῇ δὲ ἐπὶ τὴν κοίτην,
\vs{19}ἐὰν ἐξαναστὰς ὁ ἄνθρωπος περιπατήσῃ ἔξω ἐπὶ ῥάβδου, ἀθῶος ἔσται ὁ πατάξας· πλὴν τῆς ἀργείας αὐτοῦ ἀποτίσει, καὶ τὰ ἰατρεῖα.
\vs{20}Ἐὰν δέ τις πατάξῃ τὸν παῖδα αὐτοῦ ἢ τὴν παιδίσκην αὐτοῦ ἐν ῥάβδῳ, καὶ ἀποθάνῃ ὑπὸ τὰς χεῖρας αὐτοῦ, δίκῃ ἐκδικηθήσεται.
\vs{21}Ἐὰν δὲ διαβιώσῃ ἡμέραν μίαν ἢ δύο, οὐκ ἐκδικηθήτω· τὸ γὰρ ἀργύριον αὐτοῦ ἐστιν.
\vs{22}Ἐὰν δὲ μάχωνται δύο ἄνδρες, καὶ πατάξωσι γυναῖκα ἐν γαστρὶ ἔχουσαν, καὶ ἐξέλθῃ τὸ παιδίον αὐτῆς μὴ ἐξεικονισμένον, ἐπιζήμιον ζημιωθήσεται· καθότι ἂν ἐπιβάλῃ ὁ ἀνὴρ τῆς γυναικὸς, δώσει μετὰ ἀξιώματος.
\vs{23}Ἐὰν δὲ ἐξεικονισμένον ᾖ, δώσει ψυχὴν ἀντὶ ψυχῆς,
\vs{24}Ὀφθαλμὸν ἀντὶ ὀφθαλμοῦ, ὀδόντα ἀντὶ ὀδόντος, χεῖρα ἀντὶ χειρὸς, πόδα ἀντὶ ποδὸς,
\vs{25}κατάκαυμα ἀντὶ κατακαύματος, τραῦμα ἀντὶ τραύματος, μώλωπα ἀντὶ μώλωπος.
\vs{26}Ἐὰν δέ τις πατάξῃ τὸν ὀφθαλμὸν τοῦ οἰκέτου αὐτοῦ, ἢ τὸν ὀφθαλμὸν τῆς θεραπαίνης αὐτοῦ, καὶ ἐκτυφλώσῃ, ἐλευθέρους ἐξαποστελεῖ αὐτοὺς ἀντὶ τοῦ ὀφθαλμοῦ αὐτῶν.
\vs{27}Ἐὰν δὲ τὸν ὀδόντα τοῦ οἰκέτου, ἢ τὸν ὀδόντα τῆς θεραπαίνης αὐτοῦ ἐκκόψῃ, ἐλευθέρους ἐξαποστελεῖ αὐτοὺς ἀντὶ τοῦ ὀδόντος αὐτῶν.
\vs{28}Ἐὰν δὲ κερατίσῃ ταῦρος ἄνδρα ἢ γυναῖκα καὶ ἀποθάνῃ, λίθοις λιθοβοληθήσεται ὁ ταῦρος, καὶ οὐ βρωθήσεται τὰ κρέα αὐτοῦ· ὁ δὲ κύριος τοῦ ταύρου ἀθῶος ἔσται.
\vs{29}Ἐὰν δὲ ὁ ταῦρος κερατιστὴς ᾖ πρὸ τῆς χθὲς καὶ πρὸ τῆς τρίτης, καὶ διαμαρτύρωνται τῷ κυρίῳ αὐτοῦ, καὶ μὴ ἀφανίσῃ αὐτὸν, ἀνέλῃ δὲ ἄνδρα ἢ γυναῖκα, ὁ ταῦρος λιθοβοληθήσεται, καὶ ὁ κύριος αὐτοῦ προσαποθανεῖται.
\vs{30}Ἐὰν δὲ λύτρα ἐπιβληθῇ αὐτῷ, δώσει λύτρα τῆς ψυχῆς αὐτοῦ ὅσα ἐὰν ἐπιβάλωσιν αὐτῶ.
\vs{31}Ἐὰν δὲ υἱὸν ἢ θυγατέρα κερατίσῃ, κατὰ τὸ δικαίωμα τοῦτο ποιήσωσιν αὐτῷ.
\vs{32}Ἐὰν δὲ παῖδα κερατίσῃ ὁ ταῦρος ἢ παιδίσκην, ἀργυρίου τριάκοντα δίδραχμα δώσει τῷ κυρίῳ αὐτῶν, καὶ ὁ ταῦρος λιθοβοληθήσεται.
\vs{33}Ἐὰν δέ τις ἀνοίξῃ λάκκον ἢ λατομήσῃ λάκκον, καὶ μὴ καλύψῃ αὐτὸν, καὶ ἐμπέσῃ ἐκεῖ μόσχος ἢ ὄνος,
\vs{34}ὁ κύριος τοῦ λάκκου ἀποτίσει, ἀργύριον δώσει τῷ κυρίῳ αὐτῶν· τὸ δὲ τετελευτηκὸς αὐτῷ ἔσται.
\vs{35}Ἐὰν δὲ κερατίσῃ τινὸς ταῦρος τόν ταῦρον τοῦ πλησίον, καὶ τελευτήσῃ, ἀποδώσονται τὸν ταῦρον τὸν ζῶντα, καὶ διελοῦνται τὸ ἀργύριον αὐτοῦ, καὶ τὸν ταῦρον τὸν τεθνηκότα διελοῦνται.
\vs{36}Ἐὰν δὲ γνωρίζηται ὁ ταῦρος ὅτι κερατιστής ἐστι πρὸ τῆς χθὲς καὶ πρὸ τῆς τρίτης ἡμέρας, καὶ διαμεμαρτυρημένοι ὦσι τῷ κυρίῳ αὐτοῦ· καὶ μὴ ἀφανίσῃ αὐτὸν, ἀποτίσει ταῦρον ἀντὶ ταύρου, ὁ δὲ τετελευτηκὼς αὐτῷ ἔσται.

\vs{37}Ἐὰν δέ τις κλέψῃ μόσχον ἢ πρόβατον, καὶ σφάξῃ ἢ ἀποδῶται, πέντε μόσχους ἀποτίσει ἀντὶ τοῦ μόσχου, καὶ τέσσερα πρόβατα ἀντὶ τοῦ προβάτου.

\ch{22}Ἐὰν δὲ ἐν τῷ διορύγματι εὑρεθῇ ὁ κλέπτης, καὶ πληγεὶς ἀποθάνῃ, οὐκ ἔστιν αὐτῷ φόνος.
\vs{2}Ἐὰν δὲ ἀνατείλῃ ὁ ἥλιος ἐπʼ αὐτῷ, ἔνοχός ἐστιν, ἀνταποθανεῖται· ἐὰν δὲ μὴ ὑπάρχῃ αὐτῷ, πραθήτω ἀντὶ τοῦ κλέμματος.
\vs{3}Ἐὰν δὲ καταλειφθῇ καὶ εὑρεθῇ ἐν τῇ χειρὶ αὐτοῦ τὸ κλέμμα ἀπό τε ὄνου ἕως προβάτου ζῶντα, διπλᾶ αὐτὰ ἀποτίσει.
\vs{4}Ἐὰν δὲ καταβοσκήσῃ τις ἀγρὸν ἢ ἀμπελῶνα, καὶ ἀφῇ τὸ κτῆνος αὐτοῦ καταβοσκῆσαι ἀγρὸν ἕτερον, ἀποτίσει ἐκ τοῦ ἀγροῦ αὐτοῦ κατὰ τὸ γέννημα αὐτοῦ· ἐὰν δὲ πάντα τὸν ἀγρὸν καταβοσκήσῃ, τὰ βέλτιστα τοῦ ἀγροῦ αὐτοῦ καὶ τὰ βέλτιστα τοῦ ἀμπελῶνος αὐτοῦ ἀποτίσει.
\vs{5}Ἐὰν δὲ ἐξελθὸν πῦρ εὕρῃ ἀκάνθας, καὶ προσεμπρήσῃ ἅλωνας ἢ στάχυς ἢ πεδίον, ἀποτίσει ὁ τὸ πῦρ ἐκκαύσας.

\vs{6}Ἐὰν δέ τις δῷ τῷ πλησίον ἀργύριον ἢ σκεύη φυλάξαι, καὶ κλαπῇ ἐκ τῆς οἰκίας τοῦ ἀνθρώπου, ἐὰν εὑρεθῇ ὁ κλέψας, ἀποτίσει τὸ διπλοῦν.
\vs{7}Ἐὰν δὲ μὴ εὑρεθῇ ὁ κλέψας, προσελεύσεται ὁ κύριος τῆς οἰκίας ἐνώπιον τοῦ Θεοῦ, καὶ ὀμεῖται ἦ μὴν μὴ αὐτὸν πεπονηρεῦσθαι ἐφʼ ὅλης τῆς παρακαταθήκης τοῦ πλησίον,
\vs{8}κατὰ πᾶν ῥητὸν ἀδίκημα, περί τε μόσχου, καὶ ὑποζυγίου, καὶ προβάτου, καὶ ἱματίου, καὶ πάσης ἀπωλίας τῆς ἐνκαλουμένης· ὅ, τι οὖν ἂν ᾖ, ἐνώπιον τοῦ Θεοῦ ἐλεύσεται ἡ κρίσις ἀμφοτέρων, καὶ ὁ ἁλοὺς διὰ τοῦ Θεοῦ, ἀποτίσει διπλοῦν τῷ πλησίον.
\vs{9}Ἐὰν δέ τις δῷ τῷ πλησίον ὑποζύγιον ἢ μόσχον ἢ πρόβατον ἢ πᾶν κτῆνος φυλάξαι, καὶ συντριβῇ ἢ τελευτήσῃ ἢ αἰχμάλωτον γένηται, καὶ μηδεὶς γνῷ,
\vs{10}ὅρκος ἔσται τοῦ Θεοῦ ἀνὰ μέσον ἀμφοτέρων, ἦ μὴν μὴ αὐτὸν πεπονηρεῦσθαι καθόλου τῆς παρακαταθήκης τοῦ πλησίον· καὶ οὕτως προσδέξεται ὁ κύριος αὐτοῦ, καὶ οὐκ ἀποτίσει.
\vs{11}Ἐὰν δὲ κλαπῇ παρʼ αὐτοῦ, ἀποτίσει τῷ κυρίῳ·
\vs{12}Ἐὰν δὲ θηριάλωτον γένηται, ἄξει αὐτὸν ἐπὶ τὴν θήραν, καὶ οὐκ ἀποτίσει.
\vs{13}Ἐὰν δὲ αἰτήσῃ τις παρὰ τοῦ πλησίον, καὶ συντριβῇ ἢ ἀποθάνῃ ἢ αἰχμάλωτον γένηται, ὁ δὲ κύριος μὴ ᾖ μετʼ αὐτοῦ, ἀποτίσει.
\vs{14}Ἐὰν δὲ ὁ κύριος ᾖ μετʼ αὐτοῦ, οὐκ ἀποτίσει· ἐὰν δὲ μισθωτὸς ᾖ, ἔσται αὐτῷ ἀντὶ τοῦ μισθοῦ αὐτοῦ.

\vs{15}Ἐὰν δὲ ἀπατήσῃ τις παρθένον ἀμνήστευτον, καὶ κοιμηθῇ μετʼ αὐτῆς, φερνῇ φερνιεῖ αὐτὴν αὐτῷ γυναῖκα.
\vs{16}Ἐὰν δὲ ἀνανεύων ἀνανεύσῃ, καὶ μὴ βούληται ὁ πατὴρ αὐτῆς δοῦναι αὐτὴν αὐτῷ γυναῖκα, ἀργύριον ἀποτίσει τῷ πατρὶ καθʼ ὅσον ἐστὶν ἡ φερνὴ τῶν παρθένων.
\vs{17}Φαρμακοὺς οὐ περιποιήσετε.
\vs{18}Πᾶν κοιμώμενον μετὰ κτήνους θανάτῳ ἀποκτενεῖτε αὐτούς.
\vs{19}Ὁ θυσιάζων θεοῖς θανάτῳ ἐξολοθρευθήσεται, πλὴν Κυρίῳ μόνῳ.

\vs{20}Καὶ προσήλυτον οὐ κακώσετε, οὐδὲ μὴ θλίψητε αὐτόν· ἦτε γάρ προσήλυτοι ἐν γῇ Αἰγύπτῳ.
\vs{21}Πᾶσαν χήραν καὶ ὀρφανὸν οὐ κακώσετε.
\vs{22}Ἐὰν δὲ κακίᾳ κακώσητε αὐτοὺς, καὶ κεκράξαντες καταβοήσωσι πρός με, ἀκοῇ εἰσακούσομαι τῆς φωνῆς αὐτῶν,
\vs{23}καὶ ὀργισθήσομαι θυμῷ, καὶ ἀποκτενῶ ὑμᾶς μαχαίρᾳ, καὶ ἔσονται αἱ γυναῖκες ὑμῶν χῆραι, καὶ τὰ παιδία ὑμῶν ὀρφανά.
\vs{24}Ἐὰν δὲ ἀργύριον ἐκδανείσῃς τῷ ἀδελφῷ τῷ πενιχρῷ παρὰ σοὶ, οὐκ ἔσῃ αὐτὸν κατεπείγων, οὐκ ἐπιθήσεις αὐτῷ τόκον.
\vs{25}Ἐὰν δὲ ἐνεχύρασμα ἐνεχυράσῃς τὸ ἱμάτιον τοῦ πλησίον, πρὸ δυσμῶν ἡλίου ἀποδώσεις αὐτῷ·
\vs{26}Ἔστι γὰρ τοῦτο περιβόλαιον αὐτοῦ, μόνον τοῦτο τὸ ἱμάτιον ἀσχημοσύνης αὐτοῦ· ἐν τίνι κοιμηθήσεται; Ἐὰν οὖν καταβοήσῃ πρός μέ, εἰσακούσομαι αὐτοῦ· ἐλεήμων γάρ εἰμι.
\vs{27}Θεοὺς οὐ κακολογήσεις, καὶ ἄρχοντα τοῦ λαοῦ σου οὐ κακῶς ἐρεῖς.
\vs{28}Ἀπαρχὰς ἅλωνος καὶ ληνοῦ σου οὐ καθυστερήσεις· τὰ πρωτότοκα τῶν υἱῶν σου δώσεις ἐμοί.
\vs{29}Οὕτω ποιήσεις τὸν μόσχον σου καὶ τὸ πρόβατόν σου καὶ τὸ ὑποζύγιόν σου· ἑπτὰ ἡμέρας ἔσται ὑπὸ τὴν μητέρα, τῇ δὲ ὀγδόῃ ἡμέρᾳ ἀποδώσεις μοι αὐτό.
\vs{30}Καὶ ἄνδρες ἅγιοι ἔσεσθέ μοι· καὶ κρέας θηριάλωτον οὐκ ἔδεσθε, τῷ κυνὶ ἀποῤῥίψατε αὐτό.

\ch{23}
Οὐ παραδέξῃ ἀκοὴν ματαίαν· οὐ συγκαταθήσῃ μετὰ τοῦ ἀδίκου γενέσθαι μάρτυς ἄδικος.
\vs{2}Οὐκ ἔσῃ μετὰ πλειόνων ἐπὶ κακίᾳ· οὐ προστεθήσῃ μετὰ πλήθους ἐκκλῖναι μετὰ τῶν πλειόνων, ὥστε ἑκκλεῖσαι κρίσιν.
\vs{3}Καὶ πένητα οὐκ ἐλεήσεις ἐν κρίσει.
\vs{4}Ἐὰν δὲ συναντήσῃς τῷ βοῒ τοῦ ἐχθροῦ σου, ἢ τῷ ὑποζυγίῳ αὐτοῦ πλανωμένοις, ἀποστρέψας ἀποδώσεις αὐτῷ.
\vs{5}Ἐὰν δὲ ἴδῃς τὸ ὑποζύγιον τοῦ ἐχθροῦ σου πεπτωκὸς ὑπὸ τὸν γόμον αὐτοῦ, οὐ παρελεύσῃ αὐτὸ, ἀλλὰ συναρεῖς αὐτὸ μετʼ αὐτοῦ.

\vs{6}Οὐ διαστρέψεις κρίμα πένητος ἐν κρίσει αὐτοῦ.
\vs{7}Ἀπὸ παντὸς ῥήματος ἀδίκου ἀποστήσῃ· ἀθῷον καὶ δίκαιον οὐκ ἀποκτενεῖς· καὶ οὐ δικαιώσεις τὸν ἀσεβῆ ἕνεκεν δώρων.
\vs{8}Καὶ δῶρα οὐ λήψῃ· τὰ γὰρ δῶρα ἐκτυφλοῖ ὀφθαλμοὺς βλεπόντων, καὶ λυμαῖνεται ῥήματα δίκαια.
\vs{9}Καὶ προσήλυτον οὐ θλίψετε· ὑμεῖς γὰρ οἴδατε τὴν ψυχὴν τοῦ προσηλύτου· αὐτοὶ γὰρ προσήλυτοι ἦτε ἐν γῇ Αἰγύπτῳ.
\vs{10}Ἓξ ἔτη σπερεῖς τὴν γῆν σου, καὶ συνάξεις τὰ γεννήματα αὐτῆς.
\vs{11}Τῷ δὲ ἑβδόμῳ ἄφεσιν ποιήσεις, καὶ ἀνήσεις αὐτὴν, καὶ ἔδονται οἱ πτωχοὶ τοῦ ἔθνους σου· τὰ δὲ ὑπολειπόμενα ἔδεται τὰ ἄγρια θηρία· οὕτως ποιήσεις τὸν ἀμπελῶνά σου, καὶ τὸν ἐλαιῶνά σου.
\vs{12}Ἓξ ἡμέρας ποιήσεις τὰ ἔργα σου, τῇ δὲ ἡμέρᾳ τῇ ἑβδόμῃ, ἀνάπαυσις· ἵνα ἀναπαύσηται ὁ βοῦς σου, καὶ τὸ ὑποζύγιόν σου, καὶ ἵνα ἀναψύξῃ ὁ υἱὸς τῆς παιδίσκης σου καὶ ὁ προσήλυτος.
\vs{13}Πάντα ὅσα εἴρηκα πρὸς ὑμᾶς, φυλάξασθε· καὶ ὄνομα θεῶν ἑτέρων οὐκ ἀναμνησθήσεσθε, οὐδὲ μὴ ἀκουσθῇ ἐκ τοῦ στόματος ὑμῶν.

\vs{14}Τρεῖς καιροὺς τοῦ ἐνιαυτοῦ ἑορτάσατέ μοι.
\vs{15}Τὴν ἑορτὴν τῶν ἀζύμων φυλάξασθε ποιεῖν· ἑπτὰ ἡμέρας ἔδεσθε ἄζυμα, καθάπερ ἐνετειλάμην σοι κατὰ τὸν καιρὸν τοῦ μηνὸς τῶν νέων· ἐν γὰρ αὐτῷ ἐξῆλθες ἐξ Αἰγύπτου· οὐκ ὀφθήσῃ ἐνώπίον μου κενός.
\vs{16}Καὶ ἑορτὴν θερισμοῦ πρωτογεννημάτων ποιήσεις τῶν ἔργων σου, ὧν ἐὰν σπείρῃς ἐν τῷ ἀγρῷ σου, καὶ ἑορτὴν συντελείας ἐπʼ ἐξόδου τοῦ ἐνιαυτοῦ ἐν τῇ συναγωγῇ τῶν ἔργων σου τῶν ἐκ τοῦ ἀγροῦ σου.
\vs{17}Τρεῖς καιροὺς τοῦ ἐνιαυτοῦ ὀφθήσεται πᾶν ἀρσενικόν σου ἐνώπιον Κυρίου τοῦ Θεοῦ σου.
\vs{18}Ὅταν γὰρ ἐκβάλω τὰ ἔθνη ἀπὸ προσώπου σου, καὶ ἐμπλατύνω τὰ ὅριά σου, οὐ θύσεις ἐπὶ ζύμῃ αἷμα θυμιάματός μου, οὐδὲ μὴ κοιμηθῇ στέαρ τῆς ἑορτῆς μου ἕως πρωΐ.
\vs{19}Τὰς ἀπαρχὰς τῶν πρωτογενημάτων τῆς γνς σου εἰσοίσεις εἰς τὸν οἶκον Κυρίου τοῦ Θεοῦ σου· οὐχ ἑψήσεις ἄρνα ἐν γάλακτι μητρὸς αὐτοῦ.
\vs{20}Καὶ ἰδοὺ ἐγὼ ἀποστέλλω τὸν ἄγγελόν μου πρὸ προσώπου σου, ἵνα φυλάξῃ σε ἐν τῇ ὁδῷ, ὅπως εἰσαγάγῃ σε εἰς τὴν γῆν, ἣν ἡτοίμασά σοι.
\vs{21}Πρόσεχε σεαυτῷ, καὶ εἰσάκουε αὐτοῦ, καὶ μὴ ἀπείθει αὐτῷ, οὐ γὰρ μὴ ὑποστείληταί σε· τὸ γὰρ ὄνομά μου ἐστὶν ἐπʼ αὐτῷ.
\vs{22}Ἐὰν ἀκοῇ ἀκούσητε τῆς ἐμῆς φωνῆς, καὶ ποιήσῃς πάντα ὅσα ἂν ἐντείλωμαί σοι, καὶ φυλάξητε τὴν διαθήκην μου, ἔσεσθέ μοι λαὸς περιούσιος ἀπὸ πάντων τῶν ἐθνῶν· ἐμὴ γάρ ἐστι πᾶσα ἡ γῆ· ὑμεῖς δὲ ἔσεσθέ μοι βασίλειον ἱεράτευμα, καὶ ἔθνος ἅγιον· ταῦτα τὰ ῥήματα ἐρεῖς τοῖς υἱοῖς Ἰσραὴλ, ἐὰν ἀκοῇ ἀκούσητε τῆς φωνῆς μου, καὶ ποιήσητε πάντα ὅσα ἂν εἴπω σοι, ἐχθρεύσω τοῖς ἐχθροῖς σου, καὶ ἀντικείσομαι τοῖς ἀντικειμένοις σοι.
\vs{23}Πορεύσεται γὰρ ὁ ἄγγελός μου ἡγούμενός σου, καὶ εἰσάξει σε πρὸς τὸν Ἀμοῤῥαῖον, καὶ Χετταῖον, καὶ Φερεζαῖον, καὶ Χαναναῖον, καὶ Γεργεσαῖον, καὶ Εὑαῖον, καὶ Ἰεβουσαῖον, καὶ ἐκτρίψω αὐτούς.
\vs{24}Οὐ προσκυνήσεις τοῖς θεοῖς αὐτῶν, οὐδὲ μὴ λατρεύσῃς αὐτοῖς· οὐ ποιήσεις κατὰ τὰ ἔργα αὐτῶν· ἀλλὰ καθαιρέσει καθελεῖς, καὶ συντρίβων συντρίψεις τὰς στήλας αὐτῶν.
\vs{25}Καὶ λατρεύσεις Κυρίῳ τῷ Θεῷ σου· καὶ εὐλογήσω τὸν ἄρτον σου καὶ τὸν οἶνόν σου καὶ τὸ ὕδωρ σου, καὶ ἀποστρέψω μαλακίαν ἀφʼ ὑμῶν.
\vs{26}Οὐκ ἔσται ἄγονος, οὐδὲ στεῖρα ἐπὶ τῆς γῆς σου· τὸν ἀριθμὸν τῶν ἡμερῶν σου ἀναπληρῶν ἀναπληρώσω.
\vs{27}Καὶ τὸν φόβον ἀποστελῶ ἡγούμενόν σου, καὶ ἐκστήσω πάντα τὰ ἔθνη, εἰς οὓς σὺ εἰσπορεύῃ εἰς αὐτούς· καὶ δώσω πάντας τοὺς ὑπεναντίους σου φυγάδας.
\vs{28}Καὶ ἀποστελῶ τὰς σφηκίας προτέρας σου· καὶ ἐκβαλεῖς τοὺς Ἀμοῤῥαίους, καὶ τοὺς Εὑαίους, καὶ τοὺς Χαναναίους, καὶ τοὺς Χετταίους ἀπὸ σοῦ.
\vs{29}Οὐκ ἐκβαλῶ αὐτοὺς ἐν ἐνιαυτῷ ἑνὶ, ἵνα μὴ γένηται ἡ γῆ ἔρημος, καὶ πολλὰ γένηται ἐπὶ σὲ τὰ θηρία τῆς γῆς.
\vs{30}Κατὰ μικρὸν ἐκβαλῶ αὐτοὺς ἀπὸ σοῦ, ἕως ἂν αὐξηθῇς καὶ κληρονομήσῃς τὴν γῆν.
\vs{31}Καὶ θήσω τὰ ὅριά σου ἀπὸ τῆς ἐρυθρᾶς θαλάσσης, ἕως τῆς θαλάσσης τῆς Φυλιστιείμ· καὶ ἀπὸ τῆς ἐρήμου, ἕως τοῦ μεγάλου ποταμοῦ Εὐφράτου· καὶ παραδώσω εἰς τὰς χεῖρας ὑμῶν τοὺς ἐγκαθημένους ἐν τῇ γῇ, καὶ ἐκβαλῶ αὐτοὺς ἀπὸ σοῦ.
\vs{32}Οὐ συγκαταθήσῃ αὐτοῖς καὶ τοῖς θεοῖς αὐτῶν διαθήκην.
\vs{33}Καὶ οὐκ ἐνκαθήσονται ἐν τῇ γῇ σου, ἵνα μὴ ἁμαρτεῖν σε ποιήσωσι πρὸς μέ· ἐὰν γὰρ δουλεύσῃς τοῖς θεοῖς αὐτῶν, οὗτοι ἔσονταί σοι πρόσκομμα.

\ch{24}
Καὶ Μωυσῇ εἶπεν, ἀνάβηθι πρὸς τὸν Κύριον σὺ καὶ Ἀαρὼν, καὶ Ναδὰβ, καὶ Ἀβιοὺδ, καὶ ἑβδομήκοντα τῶν πρεσβυτέρων Ἰσραήλ· καὶ προσκυνήσουσι μακρόθεν τῷ Κυρίῳ.
\vs{2}Καὶ ἐγγιεῖ Μωσῆς μόνος πρὸς τὸν Θεὸν, αὐτοὶ δὲ οὐκ ἐγγιοῦσιν, ὁ δὲ λαὸς οὐ συναναβήσεται μετʼ αὐτῶν.
\vs{3}Εἰσῆλθε δὲ Μωυσῆς, καὶ διηγήσατο τῷ λαῷ πάντα τὰ ῥήματα τοῦ Θεοῦ καὶ τὰ δικαιώματα· ἀπεκρίθη δὲ πᾶς ὁ λαὸς φωνῇ μιᾷ, λέγοντες, πάντας τοὺς λόγους, οὓς ἐλάλησε Κύριος, ποιήσομεν, καὶ ἀκουσόμεθα.
\vs{4}Καὶ ἔγραψε Μωυσῆς πάντα τὰ ῥήματα Κυρίου· ὀρθρίσας δὲ Μωυσῆς τὸ πρωῒ ᾠκοδόμησε θυσιαστήριον ὑπὸ τὸ ὄρος, καὶ δώδεκα λίθους εἰς τὰς δώδεκα φυλὰς τοῦ Ἰσραήλ.
\vs{5}Καὶ ἐξαπέστειλε τοὺς νεανίσκους τῶν υἱῶν Ἰσραήλ, καὶ ἀνήνεγκαν ὁλοκαυτώματα· καὶ ἔθυσαν θυσίαν σωτηρίου τῷ Θεῷ μοσχάρια.
\vs{6}Λαβὼν δὲ Μωυσῆς τὸ ἥμισυ τοῦ αἵματος, ἐνέχεεν εἰς κρατῆρας, τὸ δὲ ἥμισυ τοῦ αἵματος προσέχεε πρὸς τὸ θυσιαστήριον.
\vs{7}Καὶ λαβὼν τὸ βιβλίον τῆς διαθήκης, ἀνέγνω εἰς τὰ ὦτα τοῦ λαοῦ· καὶ εἶπαν, πάντα ὅσα ἐλάλησε Κύριος, ποιήσομεν καὶ ἀκουσόμεθα.
\vs{8}Λαβὼν δὲ Μωυσῆς τὸ αἷμα, κατεσκέδασε τοῦ λαοῦ, καὶ εἶπεν, ἰδοὺ τὸ αἷμα τῆς διαθήκης, ἧς διέθετο Κύριος πρὸς ὑμᾶς περὶ πάντων τῶν λόγων τούτων.

\vs{9}Καὶ ἀνέβη Μωυσῆς καὶ Ἀαρὼν, καὶ Ναδὰβ, καὶ Ἀβιοῦδ, καὶ ἑβδομήκοντα τῆς γερουσίας Ἰσραήλ.
\vs{10}Καὶ εἶδον τὸν τόπον οὗ εἱστήκει ὁ Θεὸς τοῦ Ἰσραήλ· καὶ τὰ ὑπὸ τοὺς πόδας αὐτοῦ, ὡσεὶ ἔργον πλίνθου σαπφείρου, καὶ ὥσπερ εἶδος στερεώματος τοῦ οὐρανοῦ τῇ καθαριότητι.
\vs{11}Καὶ τῶν ἐπιλέκτων τοῦ Ἰσραὴλ οὐ διεφώνησεν οὐδὲ εἷς· καὶ ὤφθησαν ἐν τῷ τόπῳ τοῦ Θεοῦ, καὶ ἔφαγον καὶ ἔπιον.
\vs{12}Καὶ εἶπε Κύριος πρὸς Μωυσῆν, ἀνάβηθι πρὸς με εἰς τὸ ὄρος, καὶ ἴσθι ἐκεῖ· καὶ δώσω σοι τὰ πυξία τὰ λίθινα, τὸν νόμον καὶ τὰς ἐντολας, ἃς ἔγραψα νομοθετῆσαι αὐτοῖς.
\vs{13}Καὶ ἀναστὰς Μωυσῆς καὶ Ἰησοῦς ὁ παρεστηκὼς αὐτῷ, ἀνέβησαν εἰς τὸ ὄρος τοῦ Θεοῦ.
\vs{14}Καὶ τοῖς πρεσβυτέροις εἶπαν, ἡσυχάζετε αὐτοῦ, ἕως ἀναστρέψωμεν πρὸς ὑμᾶς· καὶ ἰδοὺ Ἀαρὼν καὶ Ὢρ μεθʼ ὑμῶν· ἐάν τινι συμβῇ κρίσις, προσπορευέσθωσαν αὐτοῖς.
\vs{15}Καὶ ἀνέβη Μωυσῆς καὶ Ἰησοῦς εἰς τὸ ὄρος· καὶ ἐκάλυψεν ἡ νεφέλη τὸ ὄρος.
\vs{16}Καὶ κατέβη ἡ δόξα τοῦ Θεοῦ ἐπὶ τὸ ὄρος τὸ Σινὰ, καὶ ἐκάλυψεν αὐτὸ ἡ νεφέλη ἓξ ἡμέρας· καὶ ἐκάλεσε Κύριος τὸν Μωυσῆν τῇ ἡμέρᾳ τῇ ἑβδόμῃ ἐκ μέσου τῆς νεφέλης.
\vs{17}Τὸ δὲ εἶδος τῆς δόξης Κυρίου, ὡσεὶ πῦρ φλέγον ἐπὶ τῆς κορυφῆς τοῦ ὄρους, ἐναντίον τῶν υἱῶν Ἰσραήλ.
\vs{18}Καὶ εἰσῆλθε Μωυσῆς εἰς τὸ μέσον τῆς νεφέλης, καὶ ἀνέβη εἰς τὸ ὄρος· καὶ ἦν ἐκεῖ ἐν τῷ ὄρει τεσσεράκοντα ἡμέρας καὶ τεσσαράκοντα νύκτας.

\ch{25}
Καὶ ἐλάλησε Κύριος πρὸς Μωυσῆν, λέγων,
\vs{2}εἶπον τοῖς υἱοῖς Ἰσραὴλ, καὶ λάβετε ἀπαρχὰς παρὰ πάντων, οἷς ἂν δόξῃ τῇ καρδίᾳ, καὶ λήψεσθε τὰς ἀπαρχάς μου.
\vs{3}Καὶ αὕτη ἐστὶν ἡ ἀπαρχὴ, ἣν λήψεσθε παρʼ αὐτῶν· χρυσίον, καὶ ἀργύριον, καὶ χαλκὸν,
\vs{4}καὶ ὑάκινθον, καὶ πορφύραν, καὶ κόκκινον διπλοῦν, καὶ βύσσον κεκλωσμένην, καὶ τρίχας αἰγείας,
\vs{5}καὶ δέρματα κριῶν ἠρυθροδανωμένα, καὶ δέρματα ὑακίνθινα, καὶ ξύλα ἄσηπτα,
\vs{5a}καὶ ἔλαιον εἰς τὴν φαῦσιν, θυμιάματα εἰς τὸ ἔλαιον τῆς χρίσεως, καὶ εἰς τὴν σύνθεσιν τοῦ θυμιάματος,
\vs{7}καὶ λίθους Σαρδίου, καὶ λίθους εἰς τὴν γλυφὴν εἰς τὴν ἐπωμίδα, καὶ τὸν ποδήρη.
\vs{8}Καὶ ποιήεις μοι ἁγίασμα, καὶ ὀφθήσομαι ἐν ὑμῖν.
\vs{9}Καὶ ποιήσεις μοι κατὰ πάντα ὅσα σοι δεικνύω ἐν τῷ ὄρει, τὸ παράδειγμα τῆς σκηνῆς, καὶ τὸ παράδειγμα πάντων τῶν σκευῶν αὐτῆς· οὕτω ποιήσεις.
\vs{10}Καὶ ποιήσεις κιβωτὸν μαρτυρίου ἐκ ξύλων ἀσήπτων, δύο πήχεων καὶ ἡμίσους τὸ μῆκος, καὶ πήχεος καὶ ἡμίσους τὸ πλάτος, καὶ πήχεως καὶ ἡμίσους τὸ ὕψος.
\vs{11}Καὶ καταχρυσώσεις αὐτὴν χρυσίῳ καθαρῷ, ἔσωθεν καὶ ἔξωθεν χρυσώσεις αὐτήν· καὶ ποιήσεις αὐτῇ κυμάτια χρυσᾶ στρεπτὰ κύκλῳ.
\vs{12}Καὶ ἐλάσεις αὐτῇ τέσσαρας δακτυλίους χρυσοῦς, καὶ ἐπιθήσεις ἐπὶ τὰ τέσσαρα κλίτη· δύο δακτυλίους ἐπὶ τὸ κλίτος τὸ ἓν, καὶ δύο δακτυλίους ἐπὶ τὸ κλίτος τὸ δεύτερον.
\vs{13}Ποιήσεις δὲ ἀναφορεῖς ξύλα ἄσηπτα, καὶ καταχρυσώσεις αὐτὰ χρυσίῳ·
\vs{14}Καὶ εἰσάξεις τοὺς ἀναφορεῖς εἰς τοὺς δακτυλίους τοὺς ἐν τοῖς κλίτεσι τῆς κιβωτοῦ, αἴρειν τὴν κιβωτὸν ἐν αὐτοῖς.
\vs{15}Ἐν τοῖς δακτυλίοις τῆς κιβωτοῦ ἔσονται οἱ ἀναφορεῖς ἀκίνητοι.
\vs{16}Καὶ ἐμβαλεῖς εἰς τὴν κιβωτὸν τὰ μαρτύρια, ἃ ἂν δῶ σοι.
\vs{17}Καὶ ποιήσεις ἱλαστήριον ἐπίθεμα χρυσίου καθαροῦ, δύο πήχεων καὶ ἡμίσους τὸ μῆκος, καὶ πήχεως καὶ ἡμίσους τὸ πλάτος.
\vs{18}Καὶ ποιήσεις δύο χερουβὶμ χρυσοτορευτὰ, καὶ ἐπιθήσεις αὐτὰ ἐξ ἀμφοτέρων τῶν κλιτῶν τοῦ ἱλαστηρίου.
\vs{19}Ποιηθήσονται χεροὺβ εἷς ἐκ τοῦ κλίτους τούτου, καὶ χεροὺβ εἷς ἐκ τοῦ κλίτους τοῦ δευτέρου τοῦ ἱλαστηρίου· καὶ ποιήσεις τοὺς δύο χερουβὶμ ἐπὶ τὰ δύο κλίτη.
\vs{20}Ἔσονται οἱ χερουβὶμ ἐκτείνοντες τὰς πτέρυγας ἐπάνωθεν, συσκιάζοντες ἐν ταῖς πτέρυξιν αὐτῶν ἐπὶ τοῦ ἱλαστηρίου, καὶ τὰ πρόσωπα αὐτῶν εἰς ἄλληλα, εἰς τὸ ἱλαστήριον ἔσονται τὰ πρόσωπα τῶν χερουβίμ.
\vs{21}Καὶ ἐπιθήσεις τὸ ἱλαστήριον ἐπὶ τὴν κιβωτὸν ἄνωθεν, καὶ εἰς τὴν κιβωτὸν ἐμβαλεῖς τὰ μαρτύρια, ἃ ἂν δῶ σοι.
\vs{22}Καὶ γνωσθήσομαί σοι ἐκεῖθεν, καὶ λαλήσω σοι ἄνωθεν τοῦ ἱλαστηρίου ἀνὰ μέσον τῶν δύο χερουβὶμ, τῶν ὄντων ἐπὶ τῆς κιβωτοῦ τοῦ μαρτυρίου, καὶ κατὰ πάντα ὅσα ἐὰν ἐντείλωμαί σοι πρὸς τοὺς υἱοὺς Ἰσραήλ.
\vs{23}Καὶ ποιήσεις τράπεζαν χρυσῆν χρυσίου καθαροῦ, δύο πήχεων τὸ μῆκος, καὶ πήχεως τὸ εὖρος, καὶ πήχεως καὶ ἡμίσους τὸ ὕψος.
\vs{24}Καὶ ποιήσεις αὐτῇ στρεπτὰ κυμάτια χρυσᾶ κύκλῳ· καὶ ποιήσεις αὐτῇ στεφάνην παλαιστοῦ κύκλῳ·

\vs{25}Καὶ ποιήσεις στρεπτὸν κυμάτιον τῇ στεφάνῃ κύκλῳ.
\vs{26}Καὶ ποιήσεις τέσσαρας δακτυλίους χρυσοῦς, καὶ ἐπιθήσεις τοὺς τέσσαρας δακτυλίους ἐπὶ τὰ τέσσαρα μέρη τῶν ποδῶν αὐτῆς ὑπὸ τὴν στεφάνην.
\vs{27}Καὶ ἔσονται οἱ δακτύλιοι εἰς θήκας τοῖς ἀναφορεῦσιν, ὥστε αἴρειν ἐν αὐτοῖς τὴν τράπεζαν.
\vs{28}Καὶ ποιήσεις τοὺς ἀναφορεῖς ἐκ ξύλων ἀσήπτων, καὶ καταχρυσώσεις αὐτοὺς χρυσίῳ καθαρῷ, καὶ ἀρθήσεται ἐν αὐτοῖς ἡ τράπεζα.
\vs{29}Καὶ ποιήσεις τὰ τρυβλία αὐτῆς, καὶ τὰς θυΐσκας, καὶ τὰ σπονδεῖα, καὶ τοὺς κυάθους, ἐν οἷς σπείσεις ἐν αὐτοῖς, ἐκ χρυσίου καθαροῦ ποιήσεις αὐτά.
\vs{30}Καὶ ἐπιθήσεις ἐπὶ τὴν τράπεζαν ἄρτους ἐνωπίους ἐναντίον μου διαπαντός.

\vs{31}Καὶ ποιήσεις λυχνίαν ἐκ χρυσίου καθαροῦ, τορευτὴν ποιήσεις τὴν λυχνίαν· ὁ καυλὸς αὐτῆς, καὶ ὁ καλαμίσκοι, καὶ οἱ κρατῆρες, καὶ οἱ σφαιρωτῆρες, καὶ τὰ κρίνα ἐξ αὐτῆς ἔσται.
\vs{32}Ἓξ δὲ καλαμίσκοι ἐκπορευόμενοι ἐκ πλαγίων, τρεῖς καλαμίσκοι τῆς λυχνίας ἐκ τοῦ κλίτους τοῦ ἑνὸς αὐτῆς, καὶ τρεῖς καλαμίσκοι τῆς λυχνίας ἐκ τοῦ κλίτους τοῦ δευτέρου.
\vs{33}Καὶ τρεῖς κρατῆρες ἐκτετυπωμένοι καρυΐσκους· ἐν τῷ ἑνὶ καλαμίσκῳ σφαιρωτὴρ καὶ κρίνον· οὕτω τοῖς ἓξ καλαμίσκοις τοῖς ἐκπορευομένοις ἐκ τῆς λυχνίας.
\vs{34}Καὶ ἐν τῇ λυχνίᾳ τέσσαρες κρατῆρες ἐκτετυπωμένοι καρυΐσκους· ἐν τῷ ἑνὶ καλαμίσκῳ σφαιρωτῆρες, καὶ τὰ κρίνα αὐτῆς.
\vs{35}Ὁ σφαιρωτὴρ ὑπὸ τοὺς δύο καλαμίσκους ἐξ αὐτῆς· καὶ σφαιρωτὴρ ὑπὸ τοὺς τέσσαρας καλαμίσκους ἐξ αὐτῆς· οὕτω τοῖς ἓξ καλαμίσκοις τοῖς ἐκπορευομένοις ἐκ τῆς λυχνίας· καὶ ἐν τῇ λυχνίᾳ τέσσαρες κρατῆρες ἐκτετυπωμένοι καρυΐσκους.
\vs{36}Οἱ σφαιρωτῆρες καὶ οἱ καλαμίσκοι ἐξ αὐτῆς ἔστωσαν· ὅλη τορευτὴ ἐξ ἑνὸς χρυσίου καθαροῦ.
\vs{37}Καὶ ποιήσεις τοὺς λύχνους αὐτῆς ἑπτά· καὶ ἐπιθήσεις τοὺς λύχνους, καὶ φανοῦσιν ἐκ τοῦ ἑνὸς προσώπου.
\vs{38}Καὶ τὸν ἐπαρυστῆρα αὐτῆς, καὶ τὰ ὑποθέματα αὐτῆς ἐκ χρυσίου καθαροῦ ποιήσεις.
\vs{39}Πάντα τὰ σκεύη ταῦτα τάλαντον χρυσίου καθαροῦ.
\vs{40}Ὅρα, ποιήσεις κατὰ τὸν τύπον τὸν δεδειγμένον σοι ἐν τῷ ὄρει.

\ch{26}
Καὶ τὴν σκηνὴν ποιήσεις, δέκα αὐλαίας ἐκ βύσσου κεκλωσμένης, καὶ ὑακίνθου, καὶ πορφύρας, καὶ κοκκίνου κεκλωσμένου χερουβὶμ· ἐργασίᾳ ὑφάντου ποιήσεις αὐτάς.
\vs{2}Μῆκος τῆς αὐλαίας τῆς μιᾶς ὀκτὼ καὶ εἴκοσι πήχεων, καὶ εὖρος τεσσάρων πήχεων ἡ αὐλαία ἡ μία ἔσται· μέτρον τὸ αὐτὸ ἔσται πάσαις ταῖς αὐλαίαις.
\vs{3}Πέντε δὲ αὐλαῖαι ἔσονται ἐξ ἀλλήλων ἐχόμεναι ἡ ἑτέρα ἐκ τῆς ἑτέρας· καὶ πέντε αὐλαῖαι ἔσονται συνεχόμεναι ἑτέρα τῇ ἑτέρᾳ.
\vs{4}Καὶ ποιήσεις αὐταῖς ἀγκύλας ὑακινθίνας ἐπὶ τοῦ χείλους τῆς αὐλαίας τῆς μιᾶς, ἐκ τοῦ ἑνὸς μέρους εἰς τὴν συμβολήν· καὶ οὕτω ποιήσεις ἐπὶ τοῦ χείλους τῆς αὐλαίας τῆς ἐξωτέρας πρὸς τῇ συμβολῇ τῇ δευτέρᾳ.
\vs{5}Πεντήκοντα ἀγκύλας ποιήσεις τῇ αὐλαίᾳ τῇ μιᾷ, καὶ πεντήκοντα ἀγκύλας ποιήσεις ἐκ τοῦ μέρους τῆς αὐλαίας κατὰ τὴν συμβολὴν τῆς δευτέρας, ἀντιπρόσωποι ἀντιπίπτουσαι ἀλλήλαις εἰς ἑκάστην.
\vs{6}Καὶ ποιήσεις κρίκους πεντήκοντα χρυσοῦς· καὶ συνάψεις τὰς αὐλαίας ἑτέραν τῇ ἑτέρα τοῖς κρίκοις· καὶ ἔσται ἡ σκηνὴ μία.
\vs{7}Καὶ ποιήσεις δέῤῥεις τριχίνας σκέπην ἐπὶ τῆς σκηνῆς, ἕνδεκα δέῤῥεις ποιήσεις αὐτάς.
\vs{8}Τὸ μῆκος τῆς δέῤῥεως τῆς μιᾶς, τριάκοντα πήχεων, καὶ τεσσάρων πήχεων τὸ εὖρος τῆς δέῤῥεως τῆς μιᾶς· τὸ αὐτὸ μέτρον ἔσται ταῖς ἕνδεκα δέῤῥεσι.
\vs{9}Καὶ συνάψεις τὰς πέντε δέῤῥεις ἐπὶ τὸ αὐτὸ, καὶ τὰς ἓξ δέῤῥεις ἐπὶ τὸ αὐτό· καὶ ἐπιδιπλώσεις τὴν δέῤῥιν τὴν ἕκτην κατὰ πρόσωπον τῆς σκηνῆς.
\vs{10}Καὶ ποιήσεις ἀγκύλας πεντήκοντα ἐπὶ τοῦ χείλους τῆς δέῤῥεως τῆς μιᾶς, τῆς ἀναμέσον κατὰ συμβολήν· καὶ πεντήκοντα ἀγκύλας ποιήσεις ἐπὶ τοῦ χείλους τῆς δέῤῥεως, τῆς συναπτούσης τῆς δευτέρας.

\vs{11}Καὶ ποιήσεις κρίκους χαλκοῦς πεντήκοντα· καὶ συνάψεις τοὺς κρίκους ἐκ τῶν ἀγκυλῶν, καὶ συνάψεις τὰς δέῤῥεις, καὶ ἔσται ἕν.
\vs{12}Καὶ ὑποθήσεις τὸ πλεονάζον ἐν ταῖς δέῤῥεσι τῆς σκηνῆς· τὸ ἥμισυ τῆς δέῤῥεως τὸ ὑπολελειμμένον ὑποκαλύψεις εἰς τὸ πλεονάζον τῶν δέῤῥεων τῆς σκηνῆς, ὑποκαλύψεις ὀπίσω τῆς σκηνῆς.
\vs{13}Πῆχυν ἐκ τούτου, καὶ πῆχυν ἐκ τούτου, ἐκ τοῦ ὑπερέχοντος τῶν δέῤῥεων, ἐκ τοῦ μήκους τῶν δέῤῥεων τῆς σκηνῆς· ἔσται συγκαλύπτον ἐπὶ τὰ πλάγια τῆς σκηνῆς ἔνθεν καὶ ἔνθεν, ἵνα καλύπτῃ.
\vs{14}Καὶ ποιήσεις κατακάλυμμα τῇ σκηνῇ δέρματα κριῶν ἠρυθροδανωμένα, καὶ ἐπικαλύμματα δέρματα ὑακίνθινα ἐπάνωθεν.

\vs{15}Καὶ ποιήσεις στύλους τῆς σκηνῆς ἐκ ξύλων ἀσήπτων.
\vs{16}Δέκα πήχεων ποιήσεις τὸν στύλον τὸν ἕνα, καὶ πήχεως ἑνὸς καὶ ἡμίσους τὸ πλάτος τοῦ στύλου τοῦ ἑνός.
\vs{17}Δύο ἀγκωνίσκους τῷ στύλῳ τῷ ἑνὶ, ἀντιπίπτοντας ἕτερον τῷ ἑτέρῳ· οὕτω ποιήσεις πᾶσι τοῖς στύλοις τῆς σκηνῆς.
\vs{18}Καὶ ποιήσεις στύλους τῇ σκηνῇ, εἴκοσι στύλους ἐκ τοῦ κλίτους τοῦ πρὸς Βοῤῥᾶν.
\vs{19}Καὶ τεσσαράκοντα βάσεις ἀργυρᾶς ποιήσεις τοῖς εἴκοσι στύλοις· δύο βάσεις τῷ στύλῳ τῷ ἑνὶ εἰς ἀμφότερα τὰ μέρη αὐτοῦ· και δύο βάσεις τῷ στύλῳ τῷ ἑνὶ εἰς ἀμφοτέρα τὰ μέρη αὐτοῦ.
\vs{20}Καὶ τὸ κλίτος τὸ δεύτερον τὸ πρὸς Νότον, εἴκοσι στύλους,
\vs{21}καὶ τεσσαράκοντα βάσεις αὐτῶν ἀργυρᾶς· δύο βάσεις τῷ στύλῳ τῷ ἑνὶ εἰς ἀμφότερα τὰ μέρη αὐτοῦ, καὶ δύο βάσεις τῷ στύλῳ τῷ ἑνὶ εἰς ἀμφότερα τὰ μέρη αὐτοῦ.
\vs{22}Καὶ ἐκ τῶν ὀπίσω τῆς σκηνῆς κατὰ τὸ μέρος τὸ πρὸς θάλασσαν ποιήσεις ἓξ στύλους.
\vs{23}Καὶ δύο στύλους ποιήσεις ἐπὶ τῶν γωνιῶν τῆς σκηνῆς ἐκ τῶν ὀπισθίων.
\vs{24}Καὶ ἔσται ἐξ ἴσου κάτωθεν· κατὰ τὸ αὐτὸ ἔσονται ἴσοι ἐκ τῶν κεφαλῶν εἰς σύμβλησιν μίαν· οὕτω ποιήσεῖς ἀμφοτέραις ταῖς δυσὶ γωνίαις· ἴσαι ἔστωσαν.
\vs{25}Καὶ ἔσονται ὀκτὼ στύλοι, καὶ αἱ βάσεις αὐτῶν ἀργυραῖ δεκαέξ· δύο βάσεις τῷ ἑνὶ στύλῳ εἰς ἀμφότερα τὰ μέρη αὐτοῦ, καὶ δύο βάσεις τῷ στύλῳ τῷ ἑνί.
\vs{26}Καὶ ποιήσεις μοχλοὺς ἐκ ξύλων ἀσήπτων· πέντε τῷ ἑνὶ στύλῳ ἐκ τοῦ ἑνὸς μέρους τῆς σκηνῆς,
\vs{27}καὶ πέντε μοχλοὺς τῷ στύλῳ τῷ ἑνὶ κλίτει τῆς σκηνῆς τῷ δευτέρῳ, καὶ πέντε μοχλοὺς τῷ στύλῳ τῷ ὀπισθίῳ τῷ κλίτει τῆς σκηνῆς τῷ πρὸς θάλασσαν.
\vs{28}Καὶ ὁ μοχλὸς ὁ μέσος ἀναμέσον τῶν στύλων διϊκνείσθω ἀπὸ τοῦ ἑνὸς κλίτους εἰς τὸ ἕτερον κλίτος.
\vs{29}Καὶ τοὺς στύλους καταχρυσώσεις χρυσίῳ· καὶ τοὺς δακτυλίους ποιήσεις χρυσοῦς, εἰς οὓς εἰσάξεις τούς μοχλούς· καὶ καταχρυσώσεις τοὺς μοχλοὺς χρυσίῳ.
\vs{30}Καὶ ἀναστήσεις τὴν σκηνὴν κατὰ τὸ εἶδος τὸ δεδειγμένον σοι ἐν τῷ ὄρει.

\vs{31}Καὶ ποιήσεις καταπέτασμα ἐξ ὑακίνθου, καὶ πορφύρας, καὶ κοκκίνου κεκλωσμένου, καὶ βύσσου νενησμένης· ἔργον ὑφαντὸν ποιήσεις αὐτὸ χερουβίμ.
\vs{32}Καὶ ἐπιθήσεις αὐτὸ ἐπὶ τεσσάρων στύλων ἀσήπτων κεχρυσωμένων χρυσίῳ· καὶ αἱ κεφαλίδες αὐτῶν χρυσαῖ, καὶ αἱ βάσεις αὐτῶν τέσσαρες ἀργυραῖ.
\vs{33}Καὶ θήσεις τὸ καταπέτασμα ἐπὶ τῶν στύλων· καὶ εἰσοίσεις ἐκεῖ ἐσώτερον τοῦ καταπετάσματος τὴν κιβωτὸν τοῦ μαρτυρίου· καὶ διοριεῖ τὸ καταπέτασμα ὑμῖν ἀναμέσον τοῦ ἁγίου καὶ ἀναμέσον τοῦ ἁγίου τῶν ἁγίων.
\vs{34}Καὶ κατακαλύψεις τῷ καταπετάσματι τὴν κιβωτὸν τοῦ μαρτυρίου ἐν τῷ ἁγίῳ τῶν ἁγίων.
\vs{35}Καὶ ἐπιθήσεις τὴν τράπεζαν ἔξωθεν τοῦ καταπετάσματος, καὶ τὴν λυχνίαν ἀπέναντι τῆς τραπέζης ἐπὶ μέρους τῆς σκηνῆς τὸ πρὸς Νότον· καὶ τὴν τράπεζαν θήσεις ἐπὶ μέρους τῆς σκηνῆς τὸ πρὸς Βοῤῥᾶν.
\vs{36}Καὶ ποιήσεις ἐπίσπαστρον τῇ θύρᾳ τῆς σκηνῆς ἐξ ὑακίνθου, καὶ πορφύρας, καὶ κοκκίνου κεκλωσμένου, καὶ βύσσου κεκλωσμένης, ἔργον ποικιλτοῦ.
\vs{37}Καὶ ποιήσεις τῷ καταπετάσματι πέντε στύλους, καὶ χρυσώσεις αὐτοὺς χρυσίῳ· καὶ αἱ κεφαλίδες αὐτῶν χρυσαῖ· καὶ χωνεύσεις αὐτοῖς πέντε βάσεις χαλκᾶς.

\ch{27}
Καὶ ποιήσεις θυσιαστήριον ἐκ ξύλων ἀσήπτων, πέντε πήχεων τὸ μῆκος, καὶ πέντε πήχεων τὸ εὖρος· τετράγωνον ἔσται τὸ θυσιαστήριον, καὶ τριῶν πήχεων τὸ ὕψος αὐτοῦ.
\vs{2}Καὶ ποιήσεις τὰ κέρατα ἐπὶ τῶν τεσσάρων γωνιῶν· ἐξ αὐτοῦ ἔσται τὰ κέρατα, καὶ καλύψεις αὐτὰ χαλκῷ.
\vs{3}Καὶ ποιήσεις στεφάνην τῷ θυσιαστηρίῳ· καὶ τὸν καλυπτῆρα αὐτοῦ, καὶ τὰς φιάλας αὐτοῦ, καὶ τὰς κρεάγρας αὐτοῦ, καὶ τὸ πυρεῖον αὐτοῦ, καὶ πάντα τὰ σκεύη αὐτοῦ ποιήσεις χαλκᾶ.
\vs{4}Καὶ ποιήσεις αὐτῷ ἐσχάραν ἔργῳ δικτυωτῷ χαλκῆν· καὶ ποιήσεις τῇ ἐσχάρᾳ τέσσαρες δακτυλίους χαλκοῦς ὑπὸ τὰ τέσσαρα κλίτη.
\vs{5}Καὶ ὑποθήσεις αὐτοὺς ὑπὸ τὴν ἐσχάραν τοῦ θυσιαστήριου κάτωθεν· ἔσται δὲ ἡ ἐσχάρα ἕως τοῦ ἡμίσους τοῦ θυσιαστηρίου.
\vs{6}Καὶ ποιήσεις τῷ θυσιαστηρίῳ ἀναφορεῖς ἐκ ξύλων ἀσήπτων, καὶ περιχαλκώσεις αὐτοὺς χαλκῷ.
\vs{7}Καὶ εἰσάξεις τοὺς ἀναφορεῖς εἰς τοὺς δακτυλίους· καὶ ἔστωσαν ἀναφορεῖς κατὰ πλευρὰ τοῦ θυσιαστηρίου ἐν τῷ αἴρειν αὐτό.
\vs{8}Κοῖλον συνιδωτὸν ποιήσεις αὐτό· κατὰ τὸ παραδειχθέν σοι ἐν τῷ ὄρει, οὕτω ποιήσεις αὐτό.
\vs{9}Καὶ ποιήσεις αὐλὴν τῇ σκηνῇ· εἰς τὸ κλίτος τὸ πρὸς Λίβα ἱστία τῆς αὐλῆς ἐκ βύσσου κεκλωσμένης· μῆκος ἑκατὸν πήχεων τῷ ἑνὶ κλίτει.
\vs{10}Καὶ οἱ στύλοι αὐτῶν εἴκοσι, καὶ αἱ βάσεις αὐτῶν εἴκοσι χαλκαῖ, καὶ οἱ κρίκοι αὐτῶν καὶ αἱ ψαλίδες ἀργυραῖ.
\vs{11}Οὕτως τῷ κλίτει τῷ πρὸς ἀπηλιώτην ἱστία ἑκατὸν πήχεων μῆκος· καὶ οἱ στύλοι αὐτῶν εἴκοσι, καὶ αἱ βάσεις αὐτῶν εἴκοσι χαλκαῖ· καὶ οἱ κρίκοι καὶ αἱ ψαλίδες τῶν στύλων, καὶ αἱ βάσεις αὐτῶν περιηργυρωμέναι ἀργυρίῳ.
\vs{12}Τὸ δὲ εὖρος τῆς αὐλῆς τὸ κατὰ θάλασσαν ἱστία πεντήκοντα πήχεων· στύλοι αὐτῶν δέκα, καὶ βάσεις αὐτῶν δέκα.
\vs{13}Καὶ εὖρος τῆς αὐλῆς τῆς πρὸς Νότον ἱστία πεντήκοντα πήχεων· στύλοι αὐτῶν δέκα, καὶ βάσεις αὐτῶν δέκα.
\vs{14}Καὶ πεντεκαίδεκα πήχεων τὸ ὕψος τῶν ἱστίων τῷ κλίτει τῷ ἑνί· στύλοι αὐτῶν τρεῖς, καὶ αἱ βάσεις αὐτῶν τρεῖς.
\vs{15}Καὶ τὸ κλίτος τὸ δεύτερον δεκαπέντε πήχεων τῶν ἱστίων τὸ ὕψος· στύλοι αὐτῶν τρεῖς, καὶ αἱ βάσεις αὐτῶν τρεῖς.
\vs{16}Καὶ τῇ πύλῃ τῆς αὐλῆς κάλυμμα· εἴκοσι πήχεων τὸ ὕψος ἐξ ὑακίνθου, καὶ πορφύρας, καὶ κοκκίνου κεκλωσμένου, καὶ βύσσου κεκλωσμένης τῇ ποικιλίᾳ τοῦ ῥαφιδευτοῦ· στύλοι αὐτῶν τέσσαρες, καὶ αἱ βάσεις αὐτῶν τέσσαρες.
\vs{17}Πάντες οἱ στύλοι τῆς αὐλῆς κύκλῳ κατηργυρωμένοι ἀργυρίῳ, καὶ αἱ κεφαλίδες αὐτῶν ἀργυραῖ, καὶ αἱ βάσεις αὐτῶν χαλκαῖ.
\vs{18}Τὸ δὲ μῆκος τῆς αὐλῆς ἑκατὸν ἐφʼ ἑκατόν· καὶ εὖρος πεντήκοντα ἐπὶ πεντήκοντα· καὶ ὕψος πέντε πήχεῶν ἐκ βύσσου κεκλωσμένης, καὶ βάσεις αὐτῶν χαλκαῖ.
\vs{19}Καὶ πᾶσα ἡ κατασκευὴ καὶ πάντα τὰ ἐργαλεῖα καὶ οἱ πάσσαλοι τῆς αὐλῆς χαλκοῖ.

\vs{20}Καὶ σὺ σύνταξον τοῖς υἱοῖς Ἰσραὴλ, καὶ λαβέτωσάν σοι ἔλαιον ἐξ ἐλαιῶν ἀτρυγον καθαρὸν κεκομμένον εἰς φῶς καῦσαι, ἵνα καίηται λύχνος διαπαντός
\vs{21}ἐν τῇ σκηνῇ τοῦ μαρτυρίου· ἔξωθεν τοῦ καταπετάσματος τοῦ ἐπὶ τῆς διαθήκης καύσει αὐτὸ Ἀαρὼν καὶ οἱ υἱοὶ αὐτοῦ ἀφʼ ἑσπέρας ἕως πρωῒ, ἐναντίον Κυρίου, νόμιμον αἰώνιον εἰς τὰς γενεὰς ὑμῶν παρὰ τῶν υἱῶν Ἰσραήλ.

\ch{28}
Καὶ σὺ προσαγάγου πρὸς σεαυτὸν τόν τε Ἀαρὼν τὸν ἀδελφόν σου, καὶ τοὺς υἱοὺς αὐτοῦ, καὶ ἐκ τῶν υἱῶν Ἰσραὴλ, ἱερατεύειν μοι Ἀαρὼν, καὶ Ναδὰβ, καὶ Ἀβιοὺδ, καὶ Ἐλεάζαρ, καὶ Ἰθάμαρ, υἱοὺς Ἀαρών.
\vs{2}Καὶ ποιήσεις στολὴν ἁγίαν Ἀαρὼν τῷ ἀδελφῷ σου εἰς τιμὴν καὶ δόξαν.
\vs{3}Καὶ σύ λάλησον πᾶσι τοῖς σοφοῖς τῇ διανοίᾳ, οὓς ἐνέπλησα πνεύματος σοφίας καὶ αἰσθήσεως· καὶ ποιήσουσι τὴν στολὴν τὴν ἁγίαν Ἀαρὼν εἰς τὸ ἅγιον, ἐν ᾗ ἱερατεύσει μοι.
\vs{4}Καὶ αὗται αἱ στολαὶ, ἃς ποιησουσι· τὸ περιστήθιον, καὶ τὴν ἐπωμίδα, καὶ τὸν ποδήρη, καὶ χιτῶνα κοσυμβωτὸν, καὶ κίδαριν, καὶ ζώνην· καὶ ποιήσουσι στολὰς ἁγίας Ἀαρὼν καὶ τοῖς υἱοῖς αὐτοῦ εἰς τὸ ἱερατεύειν μοι.
\vs{5}Καὶ αὐτοὶ λήψονται τὸ χρυσίον, καὶ τὸν ὑάκινθον, καὶ τὴν πορφύραν, καὶ τὸ κόκκινον, καὶ τὴν βύσσον.
\vs{6}Καὶ ποιήσουσι τὴν ἐπωμίδα ἐκ βύσσου κεκλωσμένης, ἔργον ὑφαντὸν ποικιλτοῦ.
\vs{7}Δύο ἐπωμίδες συνέχουσαι ἔσονται αὐτῷ ἑτέρα τὴν ἑτέραν, ἐπὶ τοῖς δυσὶ μέρεσιν ἐξηρτισμέναι.
\vs{8}Καὶ τὸ ὕφασμα τῶν ἐπωμίδων ὅ ἐστιν ἐπʼ αὐτῷ, κατὰ τὴν ποίησιν ἐξ αὐτοῦ ἔσται ἐκ χρυσίου καθαροῦ, καὶ ὑακίνθου, καὶ πορφύρας, καὶ κοκκίνου διανενησμένου, καὶ βύσσου κεκλωσμένης.
\vs{9}Καὶ λήψῃ τοὺς δύο λίθους, λίθους σμαράγδου, καὶ γλύψεις ἐν αὐτοῖς τὰ ὀνόματα τῶν υἱῶν Ἰσραήλ.
\vs{10}Ἓξ ὀνόματα ἐπὶ τὸν λίθον τὸν ἕνα, καὶ τὰ ἓξ ὀνόματα τὰ λοιπὰ ἐπὶ τὸν λίθον τὸν δεύτερον κατὰ τὰς γενέσεις αὐτῶν.
\vs{11}Ἔργον λιθουργικῆς τέχνης· γλύμμα σφραγίδος διαγλύψεις τοὺς δύο λίθους ἐπὶ τοῖς ὀνόμασι τῶς υἱῶν Ἰσρσήλ.
\vs{12}Καὶ θήσεις τοὺς δύο λίθους ἐπὶ τῶς ὤμων τῆς ἐπωμίδος· λίθοι μνημοσύνου εἰσὶ τοῖς υἱοῖς Ἰσραήλ· καὶ ἀναλήψεται Ἀαρὼν τὰ ὀνόματα τῶν υἱῶν Ἰσραὴλ ἔναντι Κυρίου ἐπὶ τῶν δύο ὤμων αὐτοῦ, μνημόσυνον πεπὶ αὐτῶν.
\vs{13}Καὶ ποιήσεις ἀσπιδίσκας ἐκ χρυσίου καθαροῦ.
\vs{14}Καὶ ποιήσεις δύο κροσωτὰ ἐκ χρυσίου καθαροῦ, καταμεμιγμένα ἐν ἄνθεσιν, ἔργον πλοκῆς· καὶ ἐπιθήσεις τὰ κροσσωτὰ τὰ πεπλεγμένα ἐπὶ τὰς ἀσπιδίσκας, κατὰ τὰς παρωμίδας αὐτῶν ἐκ τῶν ἐμπροσθίων.

\vs{15}Καὶ ποιήσεις λογεῖον τῶν κρίσεων, ἔργον ποικιλτοῦ· κατὰ τὸν ῥυθμὸν τῆς ἐπωμίδος ποιήσεις αὐτὸ ἐκ χρυσίου, καὶ ὑακίνθου, καὶ πορφύρας, καὶ κοκκίνου κεκλωσμένου, καὶ βύσσου κεκλωσμένης.
\vs{16}Ποιήσεις αὐτό τετράγωνον· ἔσται διπλοῦν, σπιθαμῆς τὸ μῆκος αὐτοῦ, καὶ σπιθαμῆς τὸ εὖρος.
\vs{17}Καὶ καθυφανεῖς ἐν αὐτῷ ὕφασμα κατάλιθον τετράστιχον· στίχος λίθων ἔσται, σάρδιον, τοπάζιον, καὶ σμαράγδος, ὁ στίχος ὁ εἷς.
\vs{18}Καὶ ὁ στίχος ὁ δεύτερος, ἄνθραξ, καὶ σάπφειρος, καὶ ἴασπις.
\vs{19}Καὶ ὁ στίχος ὁ τρίτος, λιγύριον, ἀχάτης, ἀμέθυστος.
\vs{20}Καὶ ὁ στίχος ὁ τέταρτος, χρυσόλιθος, καὶ βηρύλλιον, καὶ ὀνύχιον, περικεκαλυμμένα χρυσίῳ, συνδεδεμένα ἐν χρυσίῳ· ἔστωσαν κατὰ στίχον αὐτῶν.
\vs{21}Καὶ οἱ λίθοι ἔστωσαν ἐκ τῶν ὀνομάτων τῶν υἱῶν Ἰσραὴλ δεκαδύο κατὰ τὰ ὀνόματα αὐτῶν· γλυφαὶ σφραγίδων, ἕκαστος κατὰ τὸ ὄνομα ἔστωσαν εἰς δεκαδύο φυλάς.
\vs{22}Καὶ ποιήσεις ἐπὶ τὸ λογιον κρωσσοὺς συμπεπλεγμένους, ἔργον ἁλυσιδωτὸν ἐκ χρυσίου καθαροῦ.
\vs{29}Καὶ λήψεται Ἀαρὼν τὰ ὀνόματα τῶν υἱῶν Ἰσραὴλ ἐπὶ τοῦ λογείου τῆς κρίσεως ἐπὶ τοῦ στήθους, εἰσιόντι εἰς τὸ ἅγιον μνημόσυνου ἐναντίον τοῦ Θεοῦ.
\vs{29a}Καὶ θήσεις ἐπὶ τὸ λογεῖον τῆς κρίσεως τοὺς κρωσσούς· τὰ ἁλυσιδωτὰ ἐπʼ ἀμφοτέρων τῶν κλιτῶν τοῦ λογείου ἐπιθήσεις. Καὶ τὰς δύο ἀσπιδίσκας ἐπιθήσεις ἐπʼ ἀμφοτέρους τοὺς ὤμους τῆς ἐπωμίδος κατὰ πρόσωπον.
\vs{30}Καὶ ἐπιθήσεις ἐπὶ τὸ λογεῖον τῆς κρίσεως τὴν δήλωσιν καὶ τὴν ἀλήθειαν· καὶ ἔσται ἐπὶ τοῦ στήθους Ἀαρὼν, ὃταν εἰσπορεύεται εἰς τὸ ἅγιον ἔναντὶ Κυρίου· καὶ οἴσει Ἀαρὼν τὰς κρίσεις τῶν υἱῶν Ἰσραὴλ ἐπὶ τοῦ στήθους ἔναντι Κυρίου διαπαντός.
\vs{31}Καὶ ποιήσεις ὑποδύτην ποδήρη ὅλον ὑακίνθινον.
\vs{32}Καὶ ἔσται τὸ περιστόμιον ἐξ αὐτοῦ μέσον, ὤαν ἔχον κύκλῳ τοῦ περιστομίου, ἔργον ὑφαντου, τὴν συμβολὴν συνυφασμένην ἐξ αὐτοῦ, ἵνα μὴ ῥαγῇ.
\vs{33}Καὶ ποιήσεις ὑπὸ τὸ λῶμα τοῦ ὑποδύτου κάτωθεν, ὡσεὶ ἐξανθούσης ῥόας ῥοΐσκους ἐξ ὑακίνθου, καὶ πορφύρας, καὶ κοκκίνου διανενησμένου, καὶ βύσσου κεκλωσμένης, ὑπὸ τοῦ λώματος τοῦ ὑποδύτου κύκλῳ· τὸ αὐτὸ εἶδος ῥοΐσκους χρυσοῦς, καὶ κώδωνας ἀναμέσον τούτων περικύκλῳ.
\vs{34}Παρὰ ῥοΐσκον χρυσοῦν δώδωνα, καὶ ἄνθινον ἐπὶ τοῦ λώματος τοῦ ὑποδύτου κύκλῳ·
\vs{35}Καὶ ἔσται Ἀαρὼν ἐν τῷ λειτουργεῖν ἀκουστὴ ἡ φωνὴ αὐτοῦ, εἰσιόντι εἰς τὸ ἅγιον ἔναντι Κυρίου, καὶ ἐξιόντι, ἵνα μὴ ἀποθάνῃ.
\vs{36}Καὶ ποιήσεις πέταλον χρυσοῦν καθαρόν· καὶ ἐκτυπώσεις ἐν αὐτῷ ἐκτύπωμα σφραγίδος, Ἁγίασμα Κυρίου.
\vs{37}Καὶ ἐπιθήσεις αὐτὸ ἐπὶ ὑακίνθου κεκλωσμένης· καὶ ἔσται ἐπὶ τῆς μίτρας, κατὰ πρόσωπον τῆς μίτρας ἔσται.
\vs{38}Καὶ ἔσται ἐπὶ τοῦ μετώπου Ἀαρών· καὶ ἐξαρεῖ Ἀαρὼν τὰ ἁμαρτήματα τῶν ἁγίων, ὅσα ἂν ἁγιάσωσιν οἱ υἱοὶ Ἰσραὴλ παντὸς δόματος τῶν ἁγίων αὐτῶν· καὶ ἔσται ἐπὶ τοῦ μετώπου Ἀαρὼν διαπαντὸς δεκτὸν αὐτοῖς ἔναντι Κυρίου.

\vs{39}Καὶ οἱ κοσυμβωτοὶ τῶν χιτώνων ἐκ βύσσου· καὶ ποιήσεις κίδαριν βυσσίνην· καὶ ζώνην ποιήσεις, ἔργον ποικιλτοῦ.
\vs{40}Καὶ τοῖς υἱοῖς Ἀαρὼν ποιήσεις χιτῶνας καὶ ζώνας, καὶ κιδάρεις ποιήσεις αὐτοῖς εἰς τιμὴν καὶ δόξαν.
\vs{41}Καὶ ἐνδύσεις αὐτὰ Ἀαρὼν τὸν ἀδελφόν σου, καὶ τοὺς υἱοὺς αὐτοῦ μετʼ αὐτοῦ· καὶ χρίσεις αὐτοὺς, καὶ ἐμπλήσεις αὐτῶν τὰς χεῖρας· καὶ ἁγιάσεις αὐτοὺς, ἵνα ἱερατεύωσί μοι.
\vs{42}Καὶ ποιήσεις αὐτοῖς περισκελῆ λινᾶ καλύψαι ἀσχημοσύνην χρωτὸς αὐτῶν, ἀπὸ ὀσφύος ἕως μηρῶν ἔσται.
\vs{43}Καὶ ἕξει Ἀαρὼν αὐτὰ καὶ οἱ υἱοὶ αὐτοῦ, ὅταν εἰσπορεύωνται εἰς τὴν σκηνὴν τοῦ μαρτυρίου, ἢ ὅταν προσπορεύωνται λειτουργεῖν πρὸς τὸ θυσιαστήριον τοῦ ἁγίου· καὶ οὐκ ἐπάξονται πρὸς ἑαυτοὺς ἁμαρτίαν, ἵνα μὴ ἀποθάνωσι· νόμιμον αἰώνιον αὐτῷ, καὶ τῷ σπέρματι αὐτοῦ μετʼ αὐτόν.

\ch{29}
Καὶ ταῦτά ἐστιν, ἃ ποιήσεις αὐτοῖς· ἁγιάσεις αὐτοὺς, ὥστε ἱερατεύειν μοι αὐτούς· λήψῃ δὲ μοσχάριον ἐκ βοῶν ἓν, καὶ κριοὺς ἀμώμους δύο,
\vs{2}καὶ ἄρτους ἀζύμους πεφυραμένους ἑν ἐλαίῳ, καὶ λάγανα ἄζυμα κεχρισμένα ἐν ἐλαίῳ· σεμίδαλιν ἐκ πυρῶν ποιήσεις αὐτά.
\vs{3}Καὶ ἐπιθήσεις αὐτὰ ἐπὶ κανοῦν ἕν· καὶ προσοίσεις αὐτὰ ἐπὶ τῷ κανῷ· καὶ τὸ μοσχάριον, καὶ τοὺς δύο κριούς.
\vs{4}Καὶ Ἀαρὼν καὶ τοὺς υἱοὺς αὐτοῦ προσάξεις ἐπὶ τὰς θύρας τῆς σκηνῆς τοῦ μαρτυρίου, καὶ λούσεις αὐτοὺς ἐν ὕδατι.
\vs{5}Καὶ λαβὼν τὰς στολὰς, ἐνδύσεις Ἀαρὼν τὸν ἀδελφόν σου καὶ τὸν χιτῶνα τὸν ποδήρη, καὶ τὴν ἐπωμίδα, καὶ τὸ λογεῖον· καὶ συνάψεις αὐτῷ τὸ λογεῖον πρὸς τὴν ἐπωμίδα.
\vs{6}Καὶ ἐπιθήσεις τὴν μίτραν ἐπὶ τὴν κεφαλὴν αὐτοῦ, καὶ ἐπιθήσεις τὸ πέταλον τὸ ἁγίασμα ἐπὶ τὴν μίτραν.
\vs{7}Καὶ λήψῃ τοῦ ἐλαίου τοῦ χρίσματος· καὶ ἐπιχεεῖς αὐτὸ ἐπὶ τὴν κεφαλὴν αὐτοῦ, καὶ χρίσεις αὐτόν.
\vs{8}Καὶ τοὺς υἱοὺς αὐτοῦ προσάξεις, καὶ ἐνδύσεις αὐτοὺς χιτῶνας.
\vs{9}Καὶ ζώσεις αὐτοὺς ταῖς ζωναῖς, καὶ περιθήσεις αὐτοῖς τὰς κιδάρεις· καὶ ἔσται αὐτοῖς ἱερατῖα μοι εἰς τὸν αἰῶνα· καὶ τελειώσεις Ἀαρὼν τὰς χεῖρας αὐτοῦ, καὶ τὰς χεῖρας τῶν υἱῶν αὐτοῦ.
\vs{10}Καὶ προσάξεις τὸν μόσχον ἐπὶ τὰς θύρας τῆς σκηνῆς τοῦ μαρτυρίου· καὶ ἐπιθήσουσιν Ἀαρὼν καὶ οἱ υἱοὶ αὐτοῦ τὰς χεῖρας αὐτῶν ἐπὶ τὴν κεφαλὴν τοῦ μόσχου, ἔναντι Κυρίου, παρὰ τὰς θύρας τῆς σκηνῆς τοῦ μαρτυρίου.
\vs{11}Καὶ σφάξεις τὸν μόσχον ἔναντι Κυρίου, παρὰ τὰς θύρας τῆς σκηνῆς τοῦ μαρτυρίου.
\vs{12}Καὶ λήψῃ ἀπὸ τοῦ αἵματος τοῦ μόσχου, καὶ θήσεις ἐπὶ τῶν κεράτων τοῦ θυσιαστηρίου τῷ δακτύλῳ σου· τὸ δὲ λοιπὸν πᾶν αἷμα ἐκχεεῖς παρὰ τὴν βάσιν τοῦ θυσιαστηρίου.
\vs{13}Καὶ λήψῃ πᾶν τὸ στέαρ τὸ ἐπὶ τῆς κοιλίας, καὶ τὸν λοβὸν τοῦ ἥπατος, καὶ τοὺς δύο νεφροὺς, καὶ τὸ στέαρ τὸ ἐπʼ αὐτῶν, καὶ ἐπιθήσεις ἐπὶ τὸ θυσιαστήριον.
\vs{14}Τὰ δὲ κρέατα τοῦ μόσχου, καὶ τὸ δέρμα, καὶ τὴν κόπρον κατακαύσεις πυρὶ ἔξω τῆς παρεμβολῆς· ἁμαρτίας γάρ ἐστι.

\vs{15}Καὶ τὸν κριὸν λήψῃ τὸν ἕνα, καὶ ἐπιθήσουσιν Ἀαρὼν καὶ οἱ υἱοὶ αὐτοῦ τὰς χεῖρας αὐτῶν ἐπὶ τὴν κεφαλὴν τοῦ κριοῦ.
\vs{16}Καὶ σφάξεις αὐτὸν, καὶ λαβὼν τὸ αἷμα προσχεεῖς πρὸς τὸ θυσιαστήριον κύκλῳ.
\vs{17}Καὶ τὸν κριὸν διχοτομήσεις κατὰ μέλη· καὶ πλυνεῖς τὰ ἐνδόσθια καὶ τοὺς πόδας ὕδατι, καὶ ἐπιθήσεις ἐπὶ τὰ διχοτομήματα σὺν τῇ κεφαλῇ.
\vs{18}Καὶ ἀνοίσεις ὅλον τὸν κριὸν ἐπὶ τὸ θυσιαστήριον, ὁλοκαύτωμα τῷ Κυρίῳ εἰς ὀσμὴν εὐωδίας· θυμίαμα Κυρίῳ ἐστί.
\vs{19}Καὶ λήψῃ τὸν κριὸν τὸν δεύτερον, καὶ ἐπιθήσει Ἀαρὼν καὶ οἱ υἱοὶ αὐτοῦ τὰς χεῖρας αὐτῶν ἐπὶ τὴν κεφαλὴν τοῦ κριοῦ.
\vs{20}Καὶ σφάξεις αὐτὸν, καὶ λήψῃ τοῦ αἵματος αὐτοῦ, καὶ ἐπιθήσεις ἐπὶ τὸν λοβὸν τοῦ ὠτὸς Ἀαρὼν τοῦ δεξιοῦ, καὶ ἐπὶ τὸ ἄκρον τῆς δεξιᾶς χειρὸς, καὶ ἐπὶ τὸ ἄκρον τοῦ ποδὸς τοῦ δεξιοῦ, καὶ ἐπὶ τοὺς λοβοὺς τῶν ὤτων τῶν υἱῶν αὐτοῦ τῶν δεξιῶν, καὶ ἐπὶ τὰ ἄκρα τῶν χειρῶν αὐτῶν τῶν δεξιῶν, καὶ ἐπὶ τὰ ἄκρα τῶν ποδῶν αὐτῶν τῶν δεξιῶν.
\vs{21}Καὶ λήψῃ ἀπὸ τοῦ αἵματος τοῦ ἀπὸ τοῦ θυσιαστηρίου, καὶ ἀπὸ τοῦ ἐλαίου τῆς χρίσεως, καὶ ῥανεῖς ἐπὶ Ἀαρὼν καὶ ἐπὶ τὴν στολὴν αὐτοῦ, καὶ ἐπὶ τοὺς υἱοὺς αὐτοῦ καὶ ἐπὶ τὰς στολὰς τῶν υἱῶν αὐτοῦ μετʼ αὐτοῦ· καὶ ἁγιασθήσεται αὐτὸς καὶ ἡ στολὴ αὐτοῦ, καὶ οἱ υἱοὶ αὐτοῦ καὶ αἱ στολαὶ τῶν υἱῶν αὐτοῦ μετʼ αὐτοῦ· τὸ δὲ αἷμα τοῦ κριοῦ προσχεεῖς πρὸς τὸ θυσιαστήριον κύκλῳ.
\vs{22}Καὶ λήψῃ ἀπὸ τοῦ κριοῦ τὸ στέαρ αὐτοῦ, καὶ τὸ στέαρ τὸ κατακαλύπτον τὴν κοιλίαν, καὶ τὸν λοβὸν τοῦ ἥπατος, καὶ τοὺς δύο νεφροὺς, καὶ τὸ στέαρ τὸ ἐπʼ αὐτῶν, καὶ τὸν βραχίονα τὸν δεξιόν· ἔστι γὰρ τελείωσις αὕτη.
\vs{23}Καὶ ἄρτον ἕνα ἐξ ἐλαίου, καὶ λάγανον ἓν ἀπὸ τοῦ κανοῦ τῶν ἀζύμων τῶν προτεθειμένων ἔναντι Κυρίου.
\vs{24}Καὶ ἐπιθήσεις τὰ πάντα ἐπὶ τὰς χεῖρας Ἀαρὼν, καὶ ἐπὶ τὰς χεῖρας τῶν υἱῶν αὐτοῦ· καὶ ἀφοριεῖς αὐτὰ ἀφόρισμα ἔναντι Κυρίου.
\vs{25}Καὶ λήψῃ αὐτὰ ἐκ τῶν χειρῶν αὐτῶν, καὶ ἀνοίσεις ἐπὶ τὸ θυσιαστήριον τῆς ὁλοκαυτώσεως εἰς ὀσμὴν εὐωδίας ἔναντι Κύριου· κάρπωμά ἐστι Κυρίῳ.
\vs{26}Καὶ λήψῃ τὸ στηθύνιον ἀπὸ τοῦ κριοῦ τῆς τελειώσεως, ὅ ἐστιν Ἀαρών· καὶ ἀφοριεῖς αὐτὸ ἀφόρισμα ἔναντι Κυρίου· καὶ ἔσται σοι ἐν μερίδι.
\vs{27}Καὶ ἁγιάσεις τὸ στηθύνιον ἀφόρισμα, καὶ τὸν βραχίονα τοῦ ἀφαιρέματος, ὃς ἀφώρισται, καὶ ὃς ἀφῄρηται ἀπὸ τοῦ κριοῦ τῆς τελειώσεως ἀπὸ τοῦ Ἀαρὼν, καὶ ἀπὸ τῶν υἱῶν αὐτοῦ.
\vs{28}Καὶ ἔσται Ἀαρὼν καὶ τοῖς υἱοῖς αὐτοῦ νόμιμον αἰώνιον παρὰ τῶν υἱῶν Ἰσραήλ· ἔστι γὰρ ἀφόρισμα τοῦτο· καὶ ἀφαίρεμα ἕσται παρὰ τῶν υἱῶν Ἰσραὴλ ἀπὸ τῶν θυμάτων τῶν σωτηρίων τῶν υἱῶν Ἰσραὴλ, ἀφαίρεμα Κυρίῳ.

\vs{29}Καὶ ἡ στολὴ τοῦ ἁγίου, ἥ ἐστιν Ἀαρὼν, ἔσται τοῖς υἱοῖς αὐτοῦ μετʼ αὐτὸν, χρισθῆναι αὐτοὺς ἐν αὐτοῖς, καὶ τελειῶσαι τὰς χεῖρας αὐτῶν.
\vs{30}Ἑπτὰ ἡμέρας ἐνδύσεται αὐτὰ ὁ ἱερεὺς ὁ ἀντʼ αὐτοῦ ἐκ τῶν υἱῶν αὐτοῦ, ὃς εἰσελεύσεται εἰς τὴν σκηνὴν τοῦ μαρτυρίου λειτουργεῖν ἐν τοῖς ἁγίοις.
\vs{31}Καὶ τὸν κριὸν τῆς τελειώσεως λήψῃ· καὶ ἑψήσεις τὰ κρέα ἐν τόπῳ ἁγίῳ.
\vs{32}Καὶ ἔδονται Ἀαρὼν καὶ οἱ υἱοὶ αὐτοῦ τὰ κρέα τοῦ κριοῦ, καὶ τοὺς ἄρτους τοὺς ἐν τῷ κανῷ, παρὰ τὰς θύρας τῆς σκηνῆς τοῦ μαρτυρίου.
\vs{33}Ἔδονται αὐτὰ ἐν οἷς ἡγιάσθησαν ἐν αὐτοῖς τελειῶσαι τὰς χεῖρας αὐτῶν, ἁγιάσαι αὐτούς· καὶ ἀλλογενὴς οὐκ ἔδεται ἀπʼ αὐτῶν· ἔστι γὰρ ἅγια.
\vs{34}Ἐὰν δὲ καταλειφθῇ ἀπὸ τῶν κρεῶν τῆς θυσίας τῆς τελειώσεως καὶ τῶν ἄρτων ἕως πρωῒ, κατακαύσεις τὰ λοιπὰ πυρί· οὐ βρωθήσεται· ἁγίασμα γάρ ἐστι.

\vs{35}Καὶ ποιήσεις Ἀαρὼν καὶ τοῖς υἱοῖς αὐτοῦ οὕτω κατὰ πάντα ὅσα ἐνετειλάμην σοι· ἑπτὰ ἡμέρας τελειώσεις τὰς χεῖρας αὐτῶν.
\vs{36}Καὶ τὸ μοσχάριον τῆς ἁμαρτίας ποιήσεις τῇ ἡμέρᾳ τοῦ καθαρισμοῦ· καὶ καθαριεῖς τὸ θυσιαστήριον ἐν τῷ ἁγιάζειν σε ἐπʼ αὐτῷ· καὶ χρίσεις αὐτὸ ὥστε ἁγιάσαι αὐτό.
\vs{37}Ἑπτὰ ἡμέρας καθαριεῖς τὸ θυσιαστήριον, καὶ ἁγιάσεις αὐτό· καὶ ἔσται τὸ θυσιαστήριον, ἅγιον τοῦ ἁγίου· πᾶς ὁ ἁπτόμενος τοῦ θυσιαστηρίου, ἁγιασθήσεται.
\vs{38}Καὶ ταῦτά ἐστιν, ἃ ποιήσεις ἐπὶ τοῦ θυσιαστηρίου· ἀμνοὺς ἐνιαυσίους ἀμώμους δύο τὴν ἡμέραν ἐπὶ τὸ θυσιαστήριον ἐνδελεχῶς, κάρπωμα ἐνδελεχισμοῦ.

\vs{39}Τὸν ἀμνὸν τὸν ἕνα ποιήσεις τὸ πρωῒ, καὶ τὸν ἀμνὸν τὸν δεύτερον ποιήσεις τὸ δειλινόν.
\vs{40}Καὶ δέκατον σεμιδάλεως πεφυραμένης ἐν ἐλαίῳ κεκομμένῳ τῷ τετάρτῳ τοῦ εἴν· καὶ σπονδὴν τὸ τέταρτον τοῦ εἲν οἴνου τῷ ἀμνῷ τῷ ἑνί.
\vs{41}Καὶ τὸν ἀμνὸν τὸν δεύτερον ποιήσεις τὸ δειλινὸν, κατὰ τὴν θυσίαν τὴν πρωϊνὴν, καὶ κατὰ τὴν σπονδὴν αὐτοῦ· ποιήσεις εἰς ὀσμὴν εὐωδίας κάρπωμα Κυρίῳ,
\vs{42}θυσίαν ἐνδελεχισμοῦ εἰς γενεὰς ὑμῶν, ἐπὶ θύρας τῆς σκηνῆς τοῦ μαρτυρίου ἔναντι Κυρίου, ἐν οἷς γνωσθήσομαί σοι ἐκεῖθεν, ὥστε λαλῆσαί σοι.
\vs{43}Καὶ τάξομαι ἐκεῖ τοῖς υἱοῖς Ἰσραὴλ, καὶ ἁγιασθήσομαι ἐν δόξῃ μου.
\vs{44}Καὶ ἁγιάσω τὴν σκηνὴν τοῦ μαρτυρίου, καὶ τὸ θυσιαστήριον· καὶ Ἀαρὼν καὶ τοὺς υἱοὺς αὐτοῦ ἁγιάσω, ἱερατεύειν μοι.
\vs{45}Καὶ ἐπικληθήσομαι ἐν τοῖς υἱοῖς Ἰσραὴλ, καὶ ἔσομαι αὐτῶν Θεός.
\vs{46}Καὶ γνώσονται, ὅτι ἐγώ εἰμι Κύριος ὁ Θεὸς αὐτῶν, ὁ ἐξαγαγὼν αὐτοὺς ἐκ γῆς Αἰγύπτου, ἐπικληθῆναι αὐτοῖς, καὶ εἶναι αὐτῶν Θεός.

\ch{30}
Καὶ ποιήσεις θυσιαστήριον θυμιάματος ἐκ ξύλων ἀσήπτων.
\vs{2}Καὶ ποιήσεις αὐτὸ πήχεος τὸ μῆκος, καὶ πήχεος τὸ εὖρος· τετράγωνον ἔσται· καὶ δύο πήχεων τὸ ὕψος· ἐξ αὐτοῦ ἔσται τὰ κέρατα αὐτοῦ.
\vs{3}Καὶ καταχρυσώσεις χρυσίῳ καθαρῷ τὴν ἐσχάραν αὐτοῦ, καὶ τοὺς τοίχους αὐτοῦ κύκλῳ, καὶ τὰ κέρατα αὐτοῦ· καὶ ποιήσεις αὐτῷ στρεπτὴν στεφάνην χρυσῆν κύκλῳ.
\vs{4}Καὶ δύο δακτυλίους χρυσοῦς καθαροὺς ποιήσεις ὑπὸ τὴν στρεπτὴν στεφάνην αὐτοῦ, εἰς τὰ δύο κλίτη ποιήσεις ἐν τοῖς δυσὶ πλευροῖς· καὶ ἔσονται ψαλίδες ταῖς σκυτάλαις, ὥστε αἴρειν αὐτὸ ἐν αὐταῖς.
\vs{5}Καὶ ποιήσεις σκυτάλας ἐκ ξύλων ἀσήπτων, καὶ καταχρυσώσεις αὐτὰς χρυσίῳ.
\vs{6}Καὶ θήσεις αὐτὸ ἀπέναντι τοῦ καταπετάσματος, τοῦ ὄντος ἐπὶ τῆς κιβωτοῦ τῶν μαρτυρίων, ἐν οἷς γνωσθήσομαί σοι ἐκεῖθεν.
\vs{7}Καὶ θυμιάσει ἐπʼ αὐτοῦ Ἀαρὼν θυμίαμα σύνθετον λεπτὸν τὸ πρωῒ πρωΐ· ὅταν ἐπισκευάζῃ τοὺς λύχνους, θυμιάσει ἐπʼ αὐτοῦ.
\vs{8}Καὶ ὅταν ἐξάπτῃ Ἀαρὼν τοὺς λύχνους ὀψὲ, θυμιάσει ἐπʼ αὐτοῦ. θυμίαμα ἐνδελεχισμοῦ διαπαντὸς ἔναντι Κυρίου εἰς γενεὰς αὐτῶν.
\vs{9}Καὶ οὐκ ἀνοίσει ἐπʼ αὐτοῦ θυμίαμα ἕτερον· κάρπωμα, θυσίαν, και σπονδὴν οὐ σπείσεις ἐπʼ αὐτοῦ.
\vs{10}Καὶ ἐξιλάσεται ἐπʼ αὐτοῦ Ἀαρὼν ἐπὶ τῶν κεράτων αὐτοῦ ἅπαξ τοῦ ἐνιαυτοῦ· ἀπὸ τοῦ αἵματος τοῦ καθαρισμοῦ καθαριεῖ αὐτὸ εἰς γενεὰς αὐτῶν· ἅγιον τῶν ἁγίων ἐστὶ Κυρίῳ.

\vs{11}Καὶ ἑλάλησε Κύριος πρὸς Μωυσῆν, λέγων,
\vs{12}ἐὰν λάβῃς τὸν συλλογισμὸν τῶν υἱῶν Ἰσραὴλ ἐν τῇ ἐπισκοπῇ αὐτῶν, καὶ δώσουσιν ἕκαστος λύτρα τῆς ψυχῆς αὐτοῦ Κυρίῳ, καὶ οὐκ ἔσται ἐν αὐτοῖς πτῶσις ἐν τῇ ἐπισκοπῇ αὐτῶν.
\vs{13}Καὶ τοῦτό ἐστιν ὅ δώσουσιν ὅσοι ἂν παραπορεύωνται τὴν ἐπίσκεψιν· τὸ ἥμισυ τοῦ διδράχμου ὅ ἐστιν κατὰ τὸ δίδραχμον τὸ ἅγιον, εἴκοσι ὀβολοὶ τὸ δίδραχμον, τὸ δὲ ἥμισυ τοῦ διδράχμου εἰσφορὰ Κυρίῳ.
\vs{14}Πᾶς ὁ παραπορευόμενος εἰς τὴν ἐπίσκεψιν ἀπὸ εἰκοσαετοῦς καὶ ἐπάνω, δώσουσι τὴν εἰσφορὰν Κυρίῳ.
\vs{15}Ὁ πλουτῶν οὐ προσθήσει, καὶ ὁ πενόμενος οὐκ ἐλαττονήσει ἀπὸ τοῦ ἡμίσεως τοῦ διδράχμου ἐν τῷ διδόναι τὴν εἰσφορὰν Κυρίῳ, ἐξιλάσασθαι περὶ τῶν ψυχῶν ὑμῶν.
\vs{16}Καὶ λήψῃ τὸ ἀργύριον τῆς εἰσφορᾶς παρὰ τῶν υἱῶν Ἰσραήλ, καὶ δώσεις αὐτὸ εἰς τὸ κάτεργον τῆς σκηνῆς τοῦ μαρτυρίου· καὶ ἔσται τοῖς υἱοῖς Ἰσραὴλ μνημόσυνον ἔναντι Κυρίου, ἐξιλάσασθαι περὶ τῶν ψυχῶν ὑμῶν.
\vs{17}Καὶ ἐλάλησε Κύριος πρὸς Μωυσῆν, λέγων,
\vs{18}ποίησον λουτῆρα χαλκοῦν, καὶ βάσιν αὐτῷ χαλκῆν, ὥστε νίπτεσθαι· καὶ θήσεις αὐτὸν ἀνὰ μέσον τῆς σκηνῆς τοῦ μαρτυρίου καὶ ἀνὰ μέσον τοῦ θυσιαστηρίου· καὶ ἐκχεεῖς εἰς αὐτὸν ὕδωρ.
\vs{19}Καὶ νίψεται Ἀαρὼν καὶ οἱ υἱοὶ αὐτοῦ ἑξ αὐτοῦ τὰς χεῖρας, καὶ τοὺς πόδας ὕδατι.
\vs{20}Ὅταν εἰσπορεύωνται εἰς τὴν σκηνὴν τοῦ μαρτυρίου, νίψονται ὕδατι, καὶ οὐ μὴ ἀποθάνωσιν, ὅταν προσπορεύωνται πρὸς τὸ θυσιαστήριον λειτουργεῖν καὶ ἀναφέρειν τὰ ὁλοκαυτώματα Κυρίῳ.
\vs{21}Νίψονται τὰς χεῖρας καὶ τοὺς πόδας ὕδατι, ὅταν εἰσπορεύωνται εἰς τὴν σκηνὴν τοῦ μαρτυρίου, νίψονται ὕδατι, ἵνα μὴ ἀποθάνωσι· καὶ ἔσται αὐτοῖς νόμιμον αἰώνιον, αὐτῷ καὶ ταῖς γενεαῖς αὐτοῦ μετʼ αὐτόν.
\vs{22}Καὶ ἐλάλησε Κύριος πρὸς Μωυσῆν, λέγων,
\vs{23}καὶ σὺ λάβε ἡδύσματα, τὸ ἄνθος σμύρνης ἐκλεκτῆς πεντακοσίους σίκλους, καὶ κινναμώμου εὐώδους τὸ ἥμισυ τούτου διακοσίους πεντήκοντα, καὶ καλάμου εὐώδους διακοσίους πεντήκοντα,
\vs{24}καὶ ἴρεως πεντακοσίους σίκλους τοῦ ἁγίου, καὶ ἔλαιον ἐξ ἐλαιῶν εἵν.
\vs{25}Καὶ ποιήσεις αὐτὸ ἔλαιον χρίσμα ἅγιον, μύρον μυρεψικὸν τέχνῃ μυρεψοῦ· ἔλαιον χρίσμα ἅγιον ἔσται.
\vs{26}Καὶ χρίσεις ἐξ αὐτοῦ τὴν σκηνὴν τοῦ μαρτυρίου, καὶ τὴν κιβωτὸν τῆς σκηνῆς τοῦ μαρτυρίου, καὶ πάντα τὰ σκεύη αὐτῆς,
\vs{27}καὶ τὴν λυχνίαν καὶ πάντα τὰ σκεύη αὐτῆς, καὶ τὸ θυσιαστήριον τοῦ θυμιάματος,
\vs{28}καὶ τὸ θυσιαστήριον τῶν ὁλοκαυτωμάτων καὶ πάντα αὐτοῦ τὰ σκεύη, καὶ τὴν τράπεζαν καὶ πάντα τὰ σκεύη αὐτῆς, καὶ τὸν λουτῆρα.
\vs{29}Καὶ ἁγιάσεις αὐτά· καὶ ἔσται ἅγια τῶν ἁγίων· πᾶς ὁ ἁπτόμενος αὐτῶν, ἁγιασθήσεται.
\vs{30}Καὶ Ἀαρὼν καὶ τοὺς υἱοὺς αὐτοῦ χρίσεις, καὶ ἁγιάσεις αὐτοὺς ἱερατεύειν μοι.
\vs{31}Καὶ τοῖς υἱοῖς Ἰσραὴλ λαλήσεις, λέγων, ἔλαιον ἄλειμμα χρίσεως ἅγιον ἔσται τοῦτο ὑμῖν εἰς τὰς γενεὰς ὑμῶν.
\vs{32}Ἐπὶ σάρκα ἀνθρώπου οὐ χρισθήσεται· καὶ κατὰ τὴν σύνθεσιν ταύτην οὐ ποιήσετε ὑμῖν ἑαυτοῖς ὡσαύτως· ἅγιόν ἐστιν, καὶ ἁγίασμα ἔσται ὑμῖν.
\vs{33}Ὃς ἂν ποιήσῃ ὡσαύτως, καὶ ὃς ἂν δῷ ἀπʼ αὐτοῦ ἀλλογενεῖ, ἐξολοθρευθήσεται ἐκ τοῦ λαοῦ αὐτοῦ.

\vs{34}Καὶ εἶπε Κύριος πρὸς Μωυσῆν, λάβε σεαυτῷ ἡδύσματα, στακτήν, ὄνυχα, χαλβάνην ἡδυσμοῦ καὶ λίβανον διαφανῆ· ἴσον ἴσῳ ἔσται.
\vs{35}Καὶ ποιήσουσιν ἐν αὐτῷ θυμίαμα μυρεψικὸν ἔργον μυρεψοῦ μεμιγμένον, καθαρὸν ἔργον ἅγιον.
\vs{36}Καὶ συνκόψεις ἐκ τούτων λεπτόν, καὶ θήσεις ἀπέναντι τῶν μαρτυρίων ἐν τῇ σκηνῇ τοῦ μαρτυρίου, ὅθεν γνωσθήσομαί σοι ἐκεῖθεν· ἅγιον τῶν ἁγίων ἔσται ὑμῖν θυμίαμα.
\vs{37}Κατὰ τὴν σύνθεσιν ταύτην οὐ ποιήσετε ὑμῖν ἐαυτοῖς· ἁγίασμα ἔσται ὑμῖν Κυρίῳ·
\vs{38}Ὃς ἂν ποιήσῃ ὡσαύτως ὥστε ὀσφραίνεσθαι ἐν αὐτῷ, ἀπολεῖται ἐκ τοῦ λαοῦ αὐτοῦ.

\ch{31}
Καὶ ἐλάλησε Κύριος πρὸς Μωυσῆν, λέγων,
\vs{2}ἰδοὺ ἀνακέκλημαι ἐξ ὀνόματος τὸν Βεσελεὴλ τὸν τοῦ Οὐρείου τὸν Ὣρ, ἐκ τῆς φυλῆς Ἰούδα.
\vs{3}Καὶ ἐνέπλησα αὐτὸν πνεῦμα θεῖον σοφίας καὶ συνέσεως καὶ ἐπιστήμης, ἐν παντὶ ἔργῳ διανοεῖσθαι,
\vs{4}καὶ ἀρχιτεκτονῆσαι, ἐργάζεσθαι τὸ χρυσίον, καὶ τὸ ἀργύριον, καὶ τὸν χαλκὸν, καὶ τὴν ὑάκινθον, καὶ τὴν πορφύραν, καὶ τὸ κόκκινον τὸ νηστὸν,
\vs{5}καὶ τὰ λιθουργικὰ, καὶ εἰς τὰ ἔργα τὰ τεκτονικὰ τῶν ξύλων, ἐργάζεσθαι κατὰ πάντα τὰ ἔργα.
\vs{6}Καὶ ἐγὼ ἔδωκα αὐτὸν καὶ τὸν Ἐλιὰβ τὸν τοῦ Ἀχισαμὰχ ἐκ φυλῆς Δάν· καὶ παντὶ συνετῷ καρδίᾳ δέδωκα σύνεσιν·
\vs{7}καὶ πονήσουσι πάντα ὅσα συνέταξά σοι, τὴν σκηνὴν τοῦ μαρτυρίου, καὶ τὴν κιβωτὸν τῆς διαθήκης, καὶ τὸ ἱλαστήριον τὸ ἐπʼ αὐτῆς, καὶ τὴν διασκευὴν τῆς σκηνῆς,
\vs{8}καὶ τὰ θυσιαστήρια, καὶ τὴν τράπεζαν καὶ πάντα τὰ σκεύη αὐτῆς, καὶ τὴν λυχνίαν τὴν καθαρὰν καὶ πάντα τὰ σκεύη αὐτῆς
\vs{9}καὶ τὸν λουτῆρα καὶ τὴν βάσιν αὐτοῦ,
\vs{10}καὶ τὰς στολὰς τὰς λειτουργικὰς Ἀαρὼν, καὶ τὰς στολὰς τῶν υἱῶν αὐτοῦ ἱερατεύειν μοι,
\vs{11}καὶ τὸ ἔλαιον τῆς χρίσεως, καὶ τὸ θυμίαμα τῆς συνθέσεως τοῦ ἁγίου· κατὰ πάντα ὅσα ἐγὼ ἐνετειλάμην σοι, ποιήσουσι.

\vs{12}Καὶ ἐλάλησε Κύριος πρὸς Μωυσῆν, λέγων,
\vs{13}Καὶ σὺ σύνταξον τοῖς υἱοῖς Ἰσραὴλ, λέγων, Ὁρᾶτε, καὶ τὰ σάββατά μου φυλάξεσθε· σημεῖόν ἐστι παρʼ ἐμοὶ καὶ ἐν ὑμῖν εἰς τὰς γενεὰς ὑμῶν, ἵνα γνῶτε ὅτι ἐγὼ Κύριος ὁ ἁγιάξων ὑμᾶς.
\vs{14}καὶ φυλάξεσθε τὰ σάββατα, ὅτι ἅγιον τοῦτό ἐστι Κυρίῳ ὑμῖν· ὁ βεβηλῶν αὐτὸ, θανάτῳ θανατωθήσεται· πᾶς ὃς ποιήσει ἐν αὐτῷ ἔργον, ἐξολοθρευθήσεται ἡ ψυχὴ ἐκείνη ἐκ μέσου τοῦ λαοῦ αὐτοῦ.
\vs{15}ἓξ ἡμέρας ποιήσεις ἔργα, τῇ δὲ ἡμέρᾳ τῇ ἑβδόμῃ σάββατα, ἀνάπαυσις ἁγία τῷ κυρίῳ· πᾶς ὃς ποιήσει ἔργον τῇ ἡμέρᾳ τῇ ἑβδόμῃ θανατωθήσεται.
\vs{16}Καὶ φυλάξουσιν οἱ υἱοὶ Ἰσραὴλ τὰ σάββατα, ποιεῖν αὐτὰ εἰς τὰς γενεὰς αὐτῶν·
\vs{17}Διαθήκη αἰώνιος ἐν ἐμοὶ καὶ τοῖς υἱοῖς Ἰσραὴλ, σημεῖόν ἐστιν ἐν ἐμοὶ αἰώνιον· ὅτι ἓξ ἡμεραις ἐποίησε Κύριος τὸν οὐρανὸν καὶ τὴν γῆν, καὶ τῇ ἡμέρᾳ τῇ ἑβδόμῃ κατέπαυσε, καὶ ἐπαύσατο.
\vs{18}Καὶ ἔδωκε Μωυσῇ ἡνίκα κατέπαυσε λαλῶν αὐτῷ ἐν τῷ ὄρει τῷ Σινὰ, τὰς δύο πλάκας τοῦ μαρτυρίου, πλάκας λιθίνας γεγραμμένας τῷ δακτύλῳ τοῦ Θεοῦ.

\ch{32}
Καὶ ἰδὼν ὁ λαὸς, ὅτι κεχρόνικε Μωυσῆς καταβῆναι ἐκ τοῦ ὄρους, συνέστη ὁ λαὸς ἐπὶ Ἀαρὼν, καὶ λέγουσιν αὐτῷ ἀνάστηθι, καὶ ποίησον ἡμῖν θεοὺς, οἳ προπορεύσονται ἡμῶν· ὁ γὰρ Μωυσῆς οὗτος ὁ ἄνθρωπος ὃς ἐξήγαγεν ἡμᾶς ἐκ γῆς Αἰγύπτου, οὐκ οἴδαμεν τί γέγονεν αὐτῷ.
\vs{2}Καὶ λέγει αὐτοῖς Ἀαρὼν Περιέλεσθε τὰ ἐνώτια τὰ χρυσᾶ τὰ ἐν τοῖς ὠσὶ τῶν γυναικῶν ὑμῶν καὶ θυγατέρων, καὶ ἐνέγκατε πρός με.
\vs{3}Καὶ περιείλαντο πᾶς ὁ λαὸς τὰ ἐνώτια τὰ χρυσᾶ τὰ ἐν τοῖς ὠσὶν αὐτῶν, καὶ ἤνεγκαν πρὸς Ἀαρών.
\vs{4}Καὶ ἐδέξατο ἐκ τῶν χειρῶν αὐτῶν, καὶ ἔπλασεν αὐτὰ ἐν τῇ γραφίδι· καὶ ἐποίησεν αὐτὰ μόσχον χωνευτὸν καὶ εἶπεν, Οὗτοι οἱ θεοί σου Ἰσραὴλ, οἵτινες ἀνεβίβασάν σε ἐκ γῆς Αἰγύπτου.
\vs{5}Καὶ ἰδὼν Ἀαρὼν ᾠκοδόμησε θυσιαστήριον κατέναντι αὐτοῦ· καὶ ἐκήρυξεν Ἀαρὼν λέγων, ἑορτὴ τοῦ κυρίου αὔριον.
\vs{6}Καὶ ὀρθρίσας τῇ ἐπαύριον ἀνεβίβασεν ὁλοκαυτώματα, καὶ προσήνεγκε θυσίαν σωτηρίου· καὶ ἐκάθισεν ὁ λαὸς φαγεῖν καὶ πιεῖν, καὶ ἀνέστησαν παίζειν.

\vs{7}Καὶ ἐλάλησε Κύριος πρὸς Μωυσῆν, λέγων, βάδιζε τὸ τάχος, κατάβηθι ἐντεύθεν· ἠνόμησε γὰρ ὁ λαός σου ὃν ἐξήγαγες ἐκ γῆς Αἰγύπτου.
\vs{8}Παρέβησαν ταχὺ ἐκ τῆς ὁδοῦ, ἧς ἐνετείλω αὐτοῖς· ἐποίησαν ἑαυτοῖς μόσχον, καὶ προσκεκυνήκασιν αὐτῷ, καὶ τεθύκασιν αὐτῷ, καὶ εἶπαν, Οὗτοι οἱ θεοί σου Ἰσραὴλ, οἵτινες ἀνεβίβασάν σε ἐκ γῆς Αἰγύπτου.
\vs{10}καὶ νῦν ἔασόν με, καὶ θυμωθεὶς ὀργῇ εἰς αὐτοὺς, ἐκτρίψω αὐτούς· καὶ ποιήσω σὲ εἰς ἔθνος μέγα.
\vs{11}καὶ ἐδεήθη Μωυσῆς ἔναντι Κυρίου τοῦ Θεοῦ, καὶ εἶπεν, ἱνατί, Κύριε, θυμοῖ ὀργῇ εἰς τὸν λαόν σου, οὓς ἐξήγαγες ἐκ γῆς Αἰγύπτου ἐν ἰσχύϊ μεγάλῃ, καὶ ἐν τῷ βραχίονί σου τῷ ὑψηλῷ;
\vs{12}Μή ποτε εἴπωσιν οἱ Αἰγύπτιοι λέγοντες Μετὰ πονηρίας ἐξήγαγεν αὐτοὺς ἀποκτεῖναι ἐν τοῖς ὄρεσιν καὶ ἐξαναλῶσαι αὐτοὺς ἀπὸ τῆς γῆς. παῦσαι τῆς ὀργῆς τοῦ θυμοῦ σου, καὶ ἵλεως γενοῦ ἐπὶ τῇ κακίᾳ τοῦ λαοῦ σου,
\vs{13}μνησθεὶς Ἀβραὰμ καὶ Ἰσαὰκ καὶ Ἰακὼβ τῶν σῶν οἰκετῶν, οἷς ὤμοσας κατὰ σεαυτοῦ, καὶ ἐλάλησας πρὸς αὐτοὺς, λέγων, πολυπληθυνῶ τὸ σπέρμα ὑμῶν ὡσεὶ τὰ ἄστρα τοῦ οὐρανοῦ τῷ πλήθει· καὶ πᾶσαν τὴν γῆν ταύτην ἣν εἶπας δοῦναι αὐτοῖς, καὶ καθέξουσιν αὐτὴν εἰς τὸν αἰῶνα.
\vs{14}Καὶ ἱλάσθη Κύριος περιποιῆσαι τὸν λαὸν αὐτοῦ.

\vs{15}Καὶ ἀποστρέψας Μωυσῆς, κατέβη ἀπὸ τοῦ ὄρους· καὶ αἱ δύο πλάκες τοῦ μαρτυρίου ἐν ταῖς χερσὶν αὐτοῦ, πλάκες λίθιναι καταγεγραμμέναι ἐξ ἀμφοτέρων τῶν μερῶν αὐτῶν, ἔνθεν καὶ ἔνθεν ἦσανς γεγραμμέναι·
\vs{16}καὶ αἱ πλάκες ἔργον Θεοῦ ἦσαν, καὶ ἡ γραφὴ γραφὴ Θεοῦ κεκολαμμένη ἐν ταῖς πλαξί.
\vs{17}καὶ ἀκούσας Ἰησοῦς τῆς φωνῆς τοῦ λαοῦ κραζόντων, λέγει πρὸς Μωυσὴν, Φωνὴ πολέμου ἐν τῇ παρεμβολῇ.
\vs{18}καὶ λέγει Οὐκ ἔστι φωνὴ ἐξαρχόντων κατʼ ἰσχὺν, οὐδὲ φωνὴ ἐξαρχόντων τροπῆς, ἀλλὰ φωνὴν ἐξαρχόντων οἴνου ἐγὼ ἀκούω.

\vs{19}καὶ ἡνίκα ἤγγιζε τῇ παρεμβολῇ, ὁρᾷ τὸν μόσχον καὶ τοὺς χορούς· καὶ ὀργισθεὶς θυμῷ Μωυσῆς ἔῤῥιψεν ἀπὸ τῶν χειρῶν αὐτοῦ τὰς δύο πλάκας, καὶ συνέτριψεν αὐτὰς ὑπὸ τὸ ὄρος·
\vs{20}καὶ λαβὼν τὸν μόσχον ὃν ἐποίησαν, κατέκαυσεν αὐτὸν ἐν πυρὶ, καὶ κατήλεσεν αὐτὸν λεπτὸν, καὶ ἔσπειρεν αὐτὸν ὑπὸ τὸ ὕδωρ, καὶ ἐπότισεν αὐτὸ τοὺς υἱοὺς Ἰσραήλ.
\vs{21}καὶ εἶπε Μωυσῆς τῷ Ἀαρὼν, Τί ἐποίησέ σοι ὁ λαὸς οὗτος, ὅτι ἐπήγαγες ἐπʼ αὐτοὺς ἁμαρτίαν μεγάλην;
\vs{22}καὶ εἶπεν Ἀαρὼν πρὸς Μωυσῆν, μὴ ὀργίζου, κύριε· σὺ γὰρ οἶδας τὸ ὅρμημα τοῦ λαοῦ τούτου.
\vs{23}Λέγουσι γάρ μοι, ποιήσον ἡμῖν θεοὺς, οἳ προπορεύσονται ἡμῶν· ὁ γὰρ Μωυσῆς οὗτος ὁ ἄνθρωπος, ὃς ἐξήγαγεν ἡμᾶς ἐξ Αἰγύπτου, οὐκ οἴδαμεν τί γέγονεν αὐτῷ.
\vs{24}καὶ εἶπα αὐτοῖς, εἴ τινι ὑπάρχει χρυσία, περιέλεσθε· καὶ ἔδωκάν μοι· καὶ ἔῤῥιψα εἰς τὸ πῦρ· καὶ ἐξῆλθεν ὁ μόσχος οὗτος.
\vs{25}Καὶ ἰδὼν Μωυσῆς τὸν λαὸν ὅτι διεσκέδασται· (διεσκέδασε γὰρ αὐτοὺς Ἀαρών ἐπίχαρμα τοῖς ὑπεναντίοις αὐτῶν)
\vs{26}ἔστη δὲ Μωυσῆς ἑπὶ τῆς πύλης τῆς παρεμβολῆς, καὶ εἶπε, τίς πρὸς Κύριον; ἴτω πρός με. Συνῆλθον οὖν πρὸς αὐτὸν πάντες οἱ υἱοὶ Λευί.
\vs{27}Καὶ λέγει αὐτοῖς τάδε λέγει Κύριος ὁ Θεὸς Ἰσραήλ θέσθε ἕκαστος τὴν ἑαυτοῦ ῥομφαίανν ἐπὶ τὸν μηρὸν, καὶ διέλθατε καὶ ἀνακάμψατε ἀπὸ πύλης ἐπὶ πύλην διὰ τῆς παρεμβολῆς, καὶ ἀποκτείνατε ἕκαστος τὸν ἀδελφὸν αὐτοῦ, καὶ ἕκαστος τὸν πλησίον αὐτοῦ, καὶ ἓκαστος τὸν ἔγγιστα αὐτοῦ.
\vs{28}Καὶ ἐποίησαν οἱ υἱοὶ Λευεὶ καθὰ ἐλάλησεν αὐτοῖς Μωυσῆς· καὶ ἔπεσαν ἐκ τοῦ λαοῦ ἐν ἐκείνῃ τῇ ἡμέρᾳ εἰς τρισχιλίους ἄνδρας.
\vs{29}Καὶ εἶπεν αὐτοῖς Μωυσῆς, ἐπληρώσατε τὰς χεῖρας ὑμῶν σήμερον Κυρίῳ ἕκαστος ἐν τῷ υἱῷ ἢ ἐν τῷ ἀδελφῷ αὐτοῦ, δοθῆναι ἐφʼ ὑμᾶς εὐλογίαν.

\vs{30}Καὶ ἐγένετο μετὰ τὴν αὔριον εἶπε Μωυσῆς πρὸς τὸν λαὸν, ὑμεῖς ἡμαρτήκατε ἁμαρτίαν μεγάλη· καὶ νῦν ἀναβήσομαι πρὸς τὸν Θεὸν, ἵνα ἐξιλάσωμαι περὶ τῆς ἁμαρτίας ὑμῶν.
\vs{31}Ὑπέστρεψε δὲ Μωυσῆς πρὸς Κύριον, καὶ εἶπε, δέομαι, κύριε· ἡμάρτηκεν ὁ λαὸς οὗτος ἁμαρτίαν μεγάλην, καὶ ἐποίησαν ἑαυτοῖς θεοὺς χρυσοῦς.
\vs{32}Καὶ νῦν εἰ μὲν ἀφεῖς αὐτοῖς τὴν ἁμαρτίαν αὐτῶν, ἄφες· εἰ δὲ μή, ἐξάλειψόν με ἐκ τῆς βίβλου σου, ἧς ἔγραψας.
\vs{33}Καὶ εἶπε Κύριος πρὸς Μωυσῆν, εἴ τις ἡμάρτηκεν ἐνώπιόν μου, ἐξαλείψω αὐτοὺς ἐκ τῆς βίβλου μου.
\vs{34}Νυνὶ δὲ βάδιζε, κατάβηθι, καὶ ὁδήγησον τὸν λαὸν τοῦτον εἰς τὸν τόπον, ὃν εἶπά σοι· ἰδοὺ ὁ ἄγγελός μου προπορεύσεται πρὸ προσώπου σου· ᾖ δʼ ἂν ἡμέρᾳ ἐπισκέπτωμαι, ἐπάξω ἐπʼ αὐτοὺς τὴν ἁμαρτίαν αὐτῶν
\vs{35}Καὶ ἐπάταξε Κύριος τὸν λαὸν περὶ τῆς ποιήσεως τοῦ μόσχου, οὗ ἐποίησεν Ἀαρών.

\ch{33}
Καὶ εἶπε Κύριος πρὸς Μωυσῆς, προπορεύου, ἀνάβηθι ἐντεῦθεν σὺ καὶ ὁ λαός σου, οὓς ἐξήγαγες ἐκ γῆς Αἰγύπτου, εἰς τὴν γῆν, ἣν ὤμοσα τῷ Ἀβραὰμ, καὶ Ἰσαὰκ, καὶ Ἰακὼβ, λέγων, Τῷ σπέρματι ὑμῶν δώσω αὐτήν.
\vs{2}Καὶ συναποστελῶ τὸν ἄγγελόν μου πρὸ προσώπου σου· καὶ ἐκβαλεῖ τὸν Ἀμοῤῥαῖον, καὶ Χετταῖον, καὶ Φερεζαῖον, καὶ Γεργεσαῖον, καὶ Εὐαῖον, καὶ Ἰεβουσαῖον, καὶ Χαναναῖον.
\vs{3}Καὶ εἰσάξω σε εἰς γῆν ῥέουσαν γάλα καὶ μέλι· οὐ γὰρ μὴ συναναβῶ μετὰ σου, διὰ τὸ λαὸν σκληροτράχηλόν σε εἶναι, ἵνα μὴ ἐξαναλώσω σεε ἐν τῇ ὁδῷ.
\vs{4}Καὶ ἀκούσας ὁ λαὸς τὸ ῥῆμα τὸ πονηρὸν τοῦτο, κατεπένθησεν ἐν πενθικοῖς.
\vs{5}Καὶ εἶπε Κύριος τοῖς υἱοῖς Ἰσραὴλ, ὑμεῖς λαὸς σκληροτράχηλος· ὁρᾶτε, μὴ πληγὴν ἄλλην ἐπάξω ἐγὼ ἐφʼ ὑμᾶς, καὶ ἐξαναλώσω ὑμᾶς· νῦν οὖν ἀφέλεσθε τὰς στολὰς τῶν δοξῶν ὑμῶν, καὶ τὸν κόσμον, καὶ δείξω σοι ἃ ποιήσω σοι.
\vs{6}Καὶ περιέλαντο οἱ υἱοὶ Ἰσραὴλ τὸν κόσμον αὐτῶν, καὶ τὴν περιστολὴν ἀπὸ τοῦ ὄρους τοῦ Χωρήβ.
\vs{7}Καὶ λαβὼν Μωυσῆς τὴν σκηνὴν αὐτοῦ, ἔπηξεν ἔξω τῆς παρεμβολῆς, μακρὰν ἀπὸ τῆς παρεμβολῆς· καὶ ἐκλήθν Σκηνὴ μαρτυρίου· καὶ ἐγένετο, πᾶς ὁ ζητῶν Κύριον ἐξεπορεύετο εἰς τὴν σκηνὴν τὴν ἔξω τῆς παρεμβολῆς.
\vs{8}Ἡνίκα δʼ ἂν εἰσεπορεύετο Μωυσῆς εἰς τὴν σκηνὴν ἔξω τῆς παρεμβολῆς, εἱστήκει πᾶς ὁ λαὸς σκοπεύοντες ἕκαστος παρὰ τὰς θύρας τῆς σκηνῆς αὐτοῦ· καὶ κατενοοῦσαν ἀπιόντος Μωυσῆ ἕως τοῦ εἰσελθεῖν αὐτὸν εἰς τὴν σκηνὴν.
\vs{9}Ὡς δʼ ἂν εἰσῆλθε Μωσῆς εἰς τὴν σκηνήν, κατέβαινεν ὁ στύλος τῆς νεφέλης, καὶ ἵστατο ἐπὶ τὴν θύραν τῆς σκηνῆς, καὶ ἐλάλει Μωσῇ·
\vs{10}καὶ ἐλάλει Μωυσῇ. Καὶ ἑώρα πᾶς ὁ λαὸς τὸν στύλον τῆς νεφέλης ἑστῶτα ἐπὶ τῆς θύρας τῆς σκηνῆς· καὶ στάντες πᾶς ὁ λαὸς, προσεκύνησαν ἕκαστος ἀπὸ τῆς θύρας τῆς σκηνῆς αὐτοῦ.
\vs{11}Καὶ ἐλάλησε Κύριος πρὸς Μωυσῆν, ἐνώπιος ἐνωπίῳ, ὡς εἴτις λαλήσει πρὸς τὸν ἑαυτοῦ φίλον· καὶ ἀπελύετο εἰς τὴν παρεμβολήν· ὁ δὲ θεράπων Ἰησοῦς υἱὸς Ναυὴ νέος οὐκ ἐξεπορεύετο ἐκ τῆς σκηνῆς.

\vs{12}Καὶ εἶπε Μωυσῆς, πρὸς Κύριον, Ἰδοὺ σύ μοι λέγεις, ἀνάγαγε τὸν λαὸν τοῦτον, σὺ δὲ οὐκ ἐδήλωσάς μοι, ὃν συναποστελεῖς μετʼ ἐμοῦ· σὺ δέ μοι εἶπας, Οἶδά σε παρὰ πάντας, καὶ χάριν ἔχεις παρʼ ἐμοί.
\vs{13}Εἰ οὖν εὕρηκα χάριν ἐναντίον σου, ἐμφάνισόν μοι σεαυτόν· γνωστῶς ἴδω ἴδω σε, ὅπως ἂν ὦ εὑρηκὼς χάριν ἐναντίον σου, καὶ ἵνα γνῶ, ὅτι λαός σου τὸ ἔθνος τὸ μέγα τοῦτο.
\vs{14}Καὶ λέγει, αὐτὸς προπορεύσομαί σου, καὶ καταπαύσω σε.
\vs{15}Καὶ λέγει πρὸς αὐτόν, εἰ μὴ αὐτὸς σὺ σνμπορεύῃ, μή με ἀναγάγῃς ἐντεῦθεν.
\vs{16}Καὶ πῶς γνωστὸν ἔσται ἀληθῶς, ὅτι εὕρηκα χάριν παρὰ σοί ἐγώ τε καὶ ὁ λαός σου, ἀλλʼ ἢ συμπορευομένου σου μεθʼ ἡμῶν; καὶ ἐνδοξασθήσομαι ἐγώ τε καὶ ὁ λαός σου παρὰ πάντα τὰ ἔθνη, ὅσα ἐπὶ τῆς γῆς ἐστί.
\vs{17}Καὶ εἶπε Κύριος πρὸς Μωυσῆν, Καὶ τοῦτόν σοι τὸν λόγον, ὃν εἴρηκας ποιήσω· εὕρηκας, ποιήσω· εὕρηκας γὰρ χάριν ἐνώπιον ἐμοῦ, καὶ οἶδά σε παρὰ πάντας.
\vs{18}Καὶ λέγει, ἐμφάνισόν μοι σεαυτόν.
\vs{19}Καὶ εἶπεν, ἐγὼ παρελεύσομαι πρότερός σου τῇ δόξῃ μου, καὶ καλέσω τῷ ὀνόματί μου, Κύριος ἐναντίον σου· καὶ ἐλεήσω, ὃν ἂν ἐλεῶ, καὶ οἰκτειρήσω, ὃν ἂν οἰκτείρῶ.
\vs{20}Καὶ εἶπε, οὐ δυνήσῃ ἰδεῖν τὸ πρόσωπόν μου· οὐ γὰρ μὴ ἴδῃ ἄνθρωπος τὸ πρόσωπόν μου, καὶ ζήσεται.
\vs{21}Καὶ εἶπεν Κύριος, Ἰδοὺ τόπος παρʼ ἐμοί, στήσῃ ἐπὶ τῆς πέτρας·
\vs{22}Ἡνίκα δʼ ἂν παρέλθηνᾑ ᾑ δόξα μου, καί θήσω σε εἰς ὀπὴν τῆς πέτρας, καὶ σκεπάσω τῇ χειρί μου ἐπὶ σὲ, ἕως ἂν παρέλθω.
\vs{23}Καὶ ἀφελῶ τὴν χεῖρα, καὶ τότε ὄψει τὰ ὀπίσω μου· τὸ δὲ πρόσωπόν μου οὐκ ὀφθήσεταί σοι.

\ch{34}
Καὶ εἶπε Κύριος πρὸς Μωυσῆν λαξευσον σεαυτῷ δύο πλάκας λιθίνας, καθὼς καὶ αἱ πρῶται, καὶ ἀνάβηθι πρὸς μὲ εἰς τὸ ὄρος, καὶ γράψω ἐπὶ τῶν πλακῶν τὰ ῥήματα ἃ ἦν ἐν ταῖς πλαξὶ ταῖς πρώταις, αἷς συνέτριψας.
\vs{2}Καὶ γίνου ἕτοιμος εἰς τὸ πρωί, καὶ ἀναβήσῃ ἐπὶ τὸ ὄρος τὸ Σινά, καὶ στήσῃ μοι ἐκεῖ ἐπʼ ἄκρου τοῦ ὄρους.
\vs{3}Καὶ μηδεὶς ἀναβήτω μετὰ σοῦ μηδὲ ὀφθήτω ἐν παντὶ τῷ ὄρει· καὶ τὰ πρόβατα καὶ βόες μὴ νεμέσθωσαν πλησίον τοῦ ὄρους ἐκείνου.
\vs{4}Καὶ ἐλάξευσε δύο πλάκας λιθίνας, καθάπερ καὶ αἱ πρῶται· καὶ ὀρθρίσας Μωυσῆς, ἀνέβη εἰς τὸ ὄρος τὸ Σινὰ, καθότι συνέταξεν αὐτῷ Κύριος· καὶ ἔλαβε Μωυσῆς τὰς δύο πλάκας τὰς λιθίνας.
\vs{5}Καὶ κατέβη Κύριος ἐν νεφέλῃ, καὶ παρέστη αὐτῷ ἐκεῖ, καὶ ἐκάλεσε τῷ ὀνόματι Κυρίου.
\vs{6}Καὶ παρῆλθε Κύριος πρὸ προσώπου αὐτοῦ, καὶ ἐκάλεσε κύριος ὁ Θεὸς οἰκτείρμων καὶ ἐλεήμων, μακρόθυμος καὶ πολυέλεος καὶ ἀληθινός,
\vs{7}καὶ δικαιοσύνην διατηρῶν καὶ ἔλεος εἰς χιλιάδας, ἀφαιρῶν ἀνομίας καὶ ἀδικίας καὶ ἁμαρτίας, καὶ οὐ καθαριεῖ τὸν ἔνοχον, ἐπάγων ἀνομίας πατέρων ἐπὶ τέκνα καὶ ἐπὶ τέκνα τέκνων ἐπὶ τρίτην καὶ τετάρτην γενεάν.
\vs{8}Καὶ σπεύσας Μωσῆς κύψας ἐπὶ τὴν γῆν προσεκύνησε·
\vs{9}καὶ εἶπεν, εἰ εὕρηκα χάριν ἐνώπιόν σου, συμπορευθήτω ὁ Κύριός μου μεθʼ ἡμῶν· ὁ λαὸς γὰρ σκληροτράχηλός ἐστι, καὶ ἀφελεῖς σὺ τὰς ἁμαρτίας ἡμῶν, καὶ τὰς ἀνομίας ἡμῶν, καὶ ἐσόμεθά σοι.

\vs{10}Καὶ εἶπε Κύριος πρὸς Μωυσῆν, Ἰδοὺ, ἐγὼ τίθημί σοι διαθήκην ἐνώπιον παντὸς τοῦ λαοῦ σοῦ, ποιήσω ἔνδοξα, ἃ οὐ γέγονεν ἐν πάσῃ τῇ γῇ, καὶ ἐν παντὶ ἔθνει· καὶ ὄψεται πᾶς ὁ λαὸς, ἐν οἷς εἶ σὺ, τὰ ἔργα Κυρίου, ὅτι θαυμαστά ἐστιν, ἃ ἐγὼ ποιήσω σοι.
\vs{11}Πρόσεχε σὺ πάντα ὅσα ἐγὼ ἐντέλλομαί σοι· ἰδοὺ ἐγὼ ἐκβάλλω πρὸ προσώπου ὑμῶν τὸν Ἀμοῤῥαῖον, καὶ Χαναναῖον, καὶ Φερεζαῖον, καὶ Χετταῖον, καὶ Εὑαῖον, καὶ Γεργεσαῖον, καὶ Ἰεβουσαῖον.
\vs{12}Πρόσεχε σεαυτῷ, μή ποτε θῇς διαθήκην τοῖς ἐγκαθημένοις ἐπὶ τῆς γῆς, εἰς ἣν εἰσπορεύῃ εἰς αὐτὴν, μή σοι γένηται πρόσκομμα ἐν ὑμῖν.
\vs{13}Τοὺς βωμοὺς αὐτῶν καθελεῖτε, καὶ τὰς στήλας αὐτῶν συντρίψετε, καὶ τὰ ἄλση αὐτῶν ἐκκόψετε, καὶ τὰ γλυπτὰ τῶν θεῶν αὐτῶν κατακαύσετε ἐν πυρί.
\vs{14}Οὐ γὰρ μὴ προσκυνήσητε θεοῖς ἑτέροις· ὁ γὰρ Κύριος ὁ Θεὸς, ζηλωτὸν ὄνομα, Θεὸς ζηλωτής ἐστι.
\vs{15}Μή ποτε θῇς διαθήκην τοῖς ἔγκαθημένοις ἐπὶ τῆς γῆς, καὶ ἐκπορνεύσωσιν ὀπίσω τῶν θεῶν αὐτῶν, καὶ θύσωσι τοῖς θεοῖς αὐτῶν, καὶ καλέσωσίν σε, καὶ φάγῃς τῶν αὐτῶν,
\vs{16}καὶ λάβῃς τῶν θυγατέρων αὐτῶν τοῖς υἱοῖς σου, καὶ τῶν θυγατέρων σου δῷς τοῖς υἱοῖς αὐτῶν, καὶ ἐκπορνεύσωσιν αἱ θυγατέρες σου ὀπίσω τῶν θεῶν αὐτῶν, καὶ ἐκπορνεύσωσιν οἱ υἱοί σου ὀπίσω τῶν θεῶν αὐτῶν.
\vs{17}Καὶ θεοὺς χωνευτοὺς οὐ ποιήσεις σεαυτῷ.
\vs{18}Καὶ τὴν ἑσρτὴν τῶν ἀζύμων φυλάξῃ· ἑπτὰ ἡμέρας φαγῃ ἄζυμα, καθάπερ ἐντέταλμαί σοι, εἰς τὸν καιρὸν ἐν μηνὶ τῶν νέων· ἐν γὰρ μηνὶ τῶν νέων ἐξῆλθες ἐξ Αἰγύπτου.
\vs{19}Πᾶν διανοῖγον μήτραν, ἐμοὶ τὰ ἀρσενικὰ, πᾶν πρωτότοκον μόσχου, καὶ πρωτότοκον προβάτου.
\vs{20}Καὶ πρωτότοκον ὑποζυγίου λυτρώσῃ προβάτῳ· ἐὰν δὲ μὴ λυτρώσῃ αὐτὸ, τιμὴν δώσεις. πᾶν πρωτότοκον τῶν υἱῶν σου λυτρώσῃ· οὐκ ὀφθήσῃ ἐνώπιόν μου κενός.

\vs{21}Ἓξ ἡμέρας ἐργᾷ, τῇ δὲ ἑβδόμῃ καταπαύσεις· τῷ σπόρῳ καὶ τῷ ἀμητῷ κατάπαυσις.
\vs{22}Καὶ ἑορτὴν ἑβδομάδων ποιήσεις μοι, ἀρχὴν θερισμοῦ πυροῦ· καὶ ἐορτὴν συναγωγῆς μεσοῦντος τοῦ ἐνιαυτοῦ.
\vs{23}Τρεῖς καιροὺς τοῦ ἐνιαυτοῦ ὀφθήσεται πᾶν ἀρσενικόν σου ἐνώπιον Κυρίου τοῦ Θεοῦ Ἰσραήλ.
\vs{24}Ὅταν γὰρ ἐκβάλω τὰ ἔθνη πρὸ προσώπου σου, καὶ πλατυνῶ τὰ ὅριά σου, οὐκ ἐπιθυμήσει οὐδεὶς τῆς γῆς σου, ἡνίκα ἂν ἀναβαίνῃς ὀφθῆναι ἐναντίον Κυρίου τοῦ Θεοῦ σου, τρεῖς καιροὺς τοῦ ἐνιαυτοῦ.
\vs{25}Οὐ σφάξεις ἐπὶ ζύμῃ αἷμα θυμιαμάτων μου, καὶ οὐ κοιμηθήσεται εἰς τὸ πρωῒ θύματα ἑορτῆς τοῦ πάσχα.
\vs{26}Τὰ πρωτογεννήματα τῆς γῆς σου θήσεις εἰς τὸν οἶκον Κυρίου τοῦ Θεοῦ σου· οὐχ ἑψήσεις ἄρνα ἐν γάλακτι μητρὸς αὐτοῦ.
\vs{27}Καὶ εἶπε Κύριος πρὸς Μωυσῆν, γράψον σεαυτῷ τὰ ῥήματα ταῦτα· ἐπὶ γὰρ τῶν λόγων τούτων τέθειμαί σοι διαθήκην, καὶ τῷ Ἰσραήλ.
\vs{28}Καὶ ἦν ἐκεῖ Μωυσῆς ἐναντίον Κυρίου τεσσεράκοντα ἡμέρας, καὶ τεσσεράκοντα νύκτας· ἄρτον οὐκ ἔφαγε, καὶ ὕδωρ οὐκ ἔπιε· καὶ ἔγραψεν ἐπὶ τῶν πλακῶν τὰ ῥήματα ταῦτα τῆς διαθήκης, τοὺς δέκα λόγους.

\vs{29}Ὡς δὲ κατέβαινε Μωυσῆς ἐκ τοῦ ὄρους, καὶ αἱ δύο πλάκες ἐπὶ τῶν χειρῶν Μωυσῆ· καταβαίνοντος δὲ αὐτοῦ ἐκ τοῦ ὄρους, Μωυσῆς οὐκ ᾔδει ὅτι δεδόξασται ἡ ὄψις τοῦ χρώματος τοῦ προσώπου αὐτοῦ ἐν τῷ λαλεῖν αὐτὸν αὐτῷ.
\vs{30}Καὶ εἶδεν Ἀαρὼν καὶ πάντες οἱ πρεσβύτεροι Ἰσραὴλ τὸν Μωυσῆν, καὶ ἦν δεδοξασμένη ἡ ὄψις τοῦ χρώματος τοῦ προσώπου αὐτοῦ. καὶ ἐφοβήθησαν ἐγγίσαι αὐτῷ.
\vs{31}Καὶ ἐκάλεσεν αὐτοὺς Μωυσῆς, καὶ ἐπεστράφησαν πρὸς αὐτὸν Ἀαρὼν καὶ πάντες οἱ ἄρχοντες τῆς συναγωγῆς· καὶ ἐλάλησεν αὐτοῖς Μωυσῆς.

\vs{32}Καὶ μετὰ ταῦτα προσῆλθον πρὸς αὐτὸν πάντες οἱ υἱοὶ Ἰσραήλ. καὶ ἐνετείλατο αὐτοῖς πάντα, ὅσα ἐνετείλατο Κύριος πρὸς αὐτὸν ἐν τῷ ὄρει Σινά.
\vs{33}Καὶ ἐπειδὴ κατέπαυσε λαλῶν πρὸς αὐτοὺς, ἐπέθηκεν ἐπὶ τὸ πρόσωπον αὐτοῦ κάλυμμα.
\vs{34}Ἡνίκα δʼ ἂν εἰσεπορεύετο Μωυσῆς, ἔναντι Κυρίου λαλεῖν αὐτῷ, περιῃρεῖτο τὸ κάλυμμα ἕως τοῦ ἐκπορεύεσθαι· καὶ ἐξελθὼν ἐλάλει πᾶσι τοῖς υἱοῖς Ἰσραὴλ ὅσα ἐνετείλατο αὐτῷ Κύριος.
\vs{35}Καὶ εἶδον οἱ υἱοὶ Ἰσραὴλ τὸ πρόσωπον Μωυσέως, ὅτι δεδόξασται· καὶ περιέθηκε Μωυσῆς κάλυμμα ἐπὶ τὸ πρόσωπον ἑαυτοῦ, ἕως ἂν εἰσέλθῃ συλλαλεῖν αὐτῷ.

\ch{35}
Καὶ συνήθροισε Μωυσῆς πᾶσαν συναγωγὴν υἱῶν Ἰσραὴλ, καὶ εἶπεν, οὗτοι οἱ λόγοι, οὓς εἶπε Κύριος ποιῆσαι αὐτούς.
\vs{2}Ἓξ ἡμέρας ποιήσεις ἔργα, τῇ δὲ ἡμέρᾳ τῇ ἑβδόμῃ κατάπαυσις· ἅγια, σάββατα· ἀνάπαυσις Κυρίῳ· πᾶς ὁ ποιῶν ἔργον ἐν αὐτῇ, τελευτάτω.
\vs{3}Οὐ καύσετε πῦρ ἐν πάσῃ κατοικίᾳ ὑμῶν τῇ ἡμέρᾳ τῶν σαββάτων· ἐγὼ Κύριος.
\vs{4}Καὶ εἶπε Μωυσῆς πρὸς πᾶσαν συναγωγὴν υἱῶν Ἰσραὴλ, λέγων, τοῦτο τὸ ῥῆμα, ὃ συνέταξε Κύριος, λέγων,
\vs{5}λάβετε παρʼ ὑμῶν αὐτῶν ἀφαίρεμα Κυρίῳ· πᾶς ὁ καταδεχόμενος τῇ καρδίᾳ, οἴσουσι τὰς ἀπαρχὰς Κυρίῳ, χρυσίον, ἀργύριον, χαλκὸν,
\vs{6}ὑάκινθον, πορφύραν, κόκκινον διπλοῦν διανενησμένον, καὶ βύσσον κεκλωσμένην, καὶ τρίχας αἰγείας,
\vs{7}καὶ δέρματα κριῶν ἠρυθροδανωμένα, καὶ δέρματα ὑακινθινα, καὶ ξύλα ἄσηπτα,
\vs{9}καὶ λίθους σαρδίου, καὶ λίθους εἰς τὴν γλυφὴν εἰς τὴν ἐπωμίδα καὶ τὸν ποδήρη.
\vs{10}Καὶ πᾶς σοφὸς τῇ καρδίᾳ ἐν ὑμῖν, ἐλθὼν ἐργαζέσθω πάντα ὅσα συνέταξε Κύριος·
\vs{11}Τὴν σκηνὴν, καὶ τὰ παραρύματα, καὶ τὰ κατακαλύμματα, καὶ τὰ διατόνια, καὶ τοὺς μοχλοὺς, καὶ τοὺς στύλους,
\vs{12}καὶ τὴν κιβωτὸν τοῦ μαρτυρίου, καὶ τοὺς ἀναφορεῖς αὐτῆς, καὶ τὸ ἱλαστήριον αὐτῆς, καὶ τὸ καταπέτασμα,
\vs{12a}καὶ τὰ ἱστία τῆς αὐλῆς, καὶ τοὺς στύλους αὐτῆς, καὶ τοὺς λίθους τοὺς τῆς σμαράγδου, καὶ τὸ θυμίαμα, καὶ τὸ ἔλαιον τοῦ χρίσματος,
\vs{13}καὶ τὴν τράπεζαν καὶ πάντα τὰ σκεύη αὐτῆς,
\vs{14}καὶ τὴν λυχνίαν τοῦ φωτὸς καὶ πάντα τὰ σκεύη αὐτῆς,
\vs{16}καὶ τὸ θυσιαστήριον καὶ πάντα τὰ σκεύη αὐτοῦ,
\vs{19}καὶ τὰς στολὰς τὰς ἁγίας Ἀαρὼν τοῦ ἱερέως, καὶ τὰς στολὰς ἐν αἷς λειτουργήσουσιν ἐν αὐταῖς, καὶ τοὺς χιτῶνας τοῖς υἱοῖς Ἀαρὼν τῆς ἱερατίας, καὶ τὸ ἔλαιον τοῦ χρίσματος, καὶ τὸ θυμίαμα τῆς συνθέσεως.

\vs{20}Καὶ ἐξῆλθε πᾶσα συναγωγὴ υἱῶν Ἰσραὴλ ἀπὸ Μωυσῆ.
\vs{21}Καὶ ἤνεγκαν ἕκαστος, ὧν ἔφερεν ἡ καρδία αὐτῶν, καὶ ὅσοις ἔδοξε τῇ ψυχῇ αὐτῶν, ἀφαίρεμα· καὶ ἤνεγκαν ἀφαίρεμα Κυρίῳ εἰς πάντα τὰ ἔργα τῆς σκηνῆς τοῦ μαρτυρίου, καὶ εἰς πάντα τὰ κάτεργα αὐτῆς καὶ εἰς πάσας τὰς στολὰς τοῦ ἁγίου.
\vs{22}Καὶ ἤνεγκαν οἱ ἄνδρες παρὰ τῶν γυναικῶν, πᾶς ᾧ ἔδοξε τῇ διανοίᾳ, ἤνεγκαν σφραγίδας, καὶ ἐνώτια, καὶ δακτυλίους, καὶ ἐμπλόκια, καὶ περιδέξια, πᾶν σκεῦος χρυσοῦν.
\vs{23}Καὶ πάντες ὅσοι ἤνεγκαν ἀφαιρέματα χρυσίου Κυρίῳ, καὶ παρʼ ᾧ εὑρέθη βύσσος· καὶ δέρματα ὑακίνθινα καὶ δέρματα κριῶν ἠρυθροδανωμένα ἤνεγκαν.
\vs{24}Καὶ πᾶς ὁ ἀφαιρῶν ἀφαίρεμα, ἤνεγκαν ἀργύριον καὶ χαλκὸν, τὰ ἀφαιρέματα Κυρίῳ· καὶ παρʼ οἷς εὑρέθη ξύλα ἄσηπτα· καὶ εἰς πάντα τὰ ἔργα τῆς παρασκευῆς ἤνεγκαν.
\vs{25}Καὶ πᾶσα γυνὴ σοφὴ τῇ διανοίᾳ ταῖς χερσὶ νήθειν, ἤνεγκαν νενησμένα, τὴν ὑάκινθον, καὶ τὴν πορφύραν, καὶ τὸ κόκκινον, καὶ τὴν βύσσον.
\vs{26}Καὶ πᾶσαι αἱ γυναῖκες, αἷς ἔδοξε τῇ διανοίᾳ αὐτῶν ἐν σοφίᾳ, ἔνησαν τὰς τρίχας τὰς αἰγείας.
\vs{27}Καὶ οἱ ἄρχοντες ἤνεγκαν τοὺς λίθους τῆς σμαράγδου, καὶ τοὺς λίθους τῆς πληρώσεως εἰς τὴν ἐπωμίδα, καὶ τὸ λογεῖον,
\vs{28}καὶ τὰς συνθέσεις, καὶ εἰς τὸ ἔλαιον τῆς χρίσεως, καὶ τὴν συνθεσιν τοῦ θυμιάματος.
\vs{29}Καὶ πᾶς ἀνὴρ καὶ γυνὴ, ὧν ἔφερεν ἡ διάνοια αὐτῶν εἰσελθόντας ποιεῖν πάντα τὰ ἔργα, ὅσα συνέταξε Κύριος ποιῆσαι αὐτὰ διὰ Μωυσῆ, ἤνεγκαν οἱ υἱοὶ Ἰσραὴλ, ἀφαίρεμα Κυρίῳ.
\vs{30}Καὶ εἶπε Μωυσῆς τοῖς υἱοῖς Ἰσραὴλ, Ἰδοὺ ἀνακέκληκεν ὁ Θεὸς ἐξ ὀνόματος τὸν Βεσελεὴλ τὸν τοῦ Οὐρείου τὸν Ὢρ, ἐκ τῆς φυλῆς Ἰούδα,
\vs{31}καὶ ἐνέπλησεν αὐτὸν πνεῦμα θεῖον σοφίας καὶ συνέσεως, καὶ ἐπιστήμης πάντων,
\vs{32}ἀρχιτεκτονεῖν κατὰ πάντα τὰ ἔργα τῆς ἀρχιτεκτονίας, ποιεῖν τὸ χρυσίον καὶ τὸ ἀργύριον καὶ τὸν χαλκόν,
\vs{33}καὶ λιθουργῆσαι τὸν λίθον, καὶ κατεργάζεσθαι τὰ ξύλα, καὶ ποιεῖν ἐν παντὶ ἔργῳ σοφίας.
\vs{34}Καὶ προβιβάσαι γε ἔδωκεν ἐν τῇ διανοίᾳ αὐτῷ τε, καὶ τῷ Ἐλιὰβ τῷ τοῦ Ἀχισαμὰκ, ἐκ φυλῆς Δάν·
\vs{35}Καὶ ἐνέπλησεν αὐτοὺς σοφίας, συνέσεως, διανοίας, πάντα συνιέναι ποιῆσαι τὰ ἔργα τοῦ ἁγίου, καὶ τὰ ὑφαντὰ καὶ ποικιλτὰ ὑφάναι τῷ κοκκίνῳ, καὶ τῇ βύσσῳ, ποιεῖν πᾶν ἔργον ἀρχιτεκτονίας, ποικιλίας.

\ch{36}
Καὶ ἐποίησε Βεσελεὴλ καὶ Ἐλιὰβ, καὶ πᾶς σοφὸς τῇ διανοὶᾳ, ᾧ ἐδόθη σοφία καὶ ἐπιστήμη ἐν αὑτοῖς, συνιέναι ποιεῖν πάντα τὰ ἔργα, κατὰ τὰ ἅγια καθήκοντα, κατὰ πάντα ὅσα συνέταξε Κύριος.
\vs{2}Καὶ ἐκάλεσε Μωυσῆς Βεσελεὴλ καὶ Ἐλιὰβ, καὶ πάντας τοὺς ἔχοντας τὴν σοφίαν, ᾧ ἔδωκεν ὁ Θεὸς ἐπιστήμην ἐν τῇ καρδίᾳ, καὶ πάντας τοὺς ἑκουσίως βουλομένους προσπορεύεσθαι πρὸς τὰ ἔργα, ὥστε συντελεῖν αὐτά.
\vs{3}Καὶ ἔλαβον παρὰ Μωσῆ πάντα τὰ ἀφαιρέματα, ἃ ἤνεγκαν οἱ υἱοὶ Ἰσραὴλ εἰς πάντα τὰ ἔργα τοῦ ἁγίου ποιεῖν αὐτά· καὶ αὐτοὶ προσεδέχοντο ἔτι τὰ προσφερόμενα παρὰ τῶν φερόντων τὸ πρωΐ.
\vs{4}Καὶ παρεγίνοντο πάντες οἱ σοφοὶ οἱ ποιοῦντες τὰ ἔργα τοῦ ἁγίου, ἕκαστος κατὰ τὸ αὐτοῦ ἔργον, ὃ εἰργάζοντο αὐτοί.
\vs{5}Καὶ εἶπε πρὸς Μωυσῆν, ὅτι πλῆθος φέρει ὁ λαὸς κατὰ τὰ ἔργα ὅσα συνέταξε Κύριος ποιῆσαι.
\vs{6}Καὶ προσέταξε Μωυσῆς, καὶ ἐκήρυξεν ἐν τῇ παρεμβολῇ, λέγων, ἀνὴρ καὶ γυνὴ μηκέτι ἐργαζέσθωσαν εἰς τάς ἀπαρχὰς τοῦ ἁγίου· καὶ ἐκωλύθη ὁ λαὸς ἔτι προσφέρειν.
\vs{7}Καὶ τὰ ἔργα ἦν αὐτοῖς ἱκανὰ εἰς τὴν κατασκευὴν ποιῆσαι, καὶ προσκατέλιπον.
\vs{8}Καὶ ἐποίησε πᾶς σοφὸς ἐν τοῖς ἐργαζομένοις τὰς στολὰς τῶν ἁγίων, αἵ εἰσιν Ἀαρὼν τῷ ἱερεῖ, καθὰ συνέταξε Κύριος τῷ Μωυσῇ.
\vs{9}Καὶ ἐποίησε τὴν ἐπωμίδα ἐκ χρυσίου, καὶ ὑακίνθου, καὶ πορφύρας, καὶ κοκκίνου νενησμένου, καὶ βύσσου κεκλωσμένης·
\vs{10}καὶ ἐτμήθη τὰ πέταλα τοῦ χρυσίου τρίχες, ὥστε συνυφάναι σὺν τῇ ὑακίνθῳ, καὶ τῇ πορφύρᾳ, καὶ σὺν τῷ κοκκίνῳ τῷ διανενησμένῳ, καὶ τῇ βύσσῳ τῇ κεκλωσμένῃ· ἔργον ὑφαντὸν ἐποίησαν αὐτό·
\vs{11}ἐπωμίδας συνεχούσας ἐξ ἀμφοτέρων τῶν μερῶν, ἔργον ὑφαντὸν εἰς ἄλληλα συμπεπλεγμένα καθʼ ἑαυτό.
\vs{12}Ἐξ αὐτοῦ ἐποίησαν αὐτὸ κατὰ τὴν αὐτοῦ ποίησιν, ἐκ χρυσίου, καὶ ὑακίνθου, καὶ πορφύρας, καὶ κοκκίνου διανενησμένου, καὶ βύσσου κεκλωσμένης, καθὰ συνέταξε Κύριος τῷ Μωυσῇ·
\vs{13}καὶ ἐποίησαν ἀμφοτέρους τοὺς λίθους τῆς σμαράγδου συνπεπορπημένους καὶ περισεσιαλωμένους χρυσίῳ, γεγλυμμένους καὶ ἐκκεκολαμμένους ἐγκόλαμμα σφραγίδος ἐκ τῶν ὀνομάτων τῶν υἱῶν Ἰσραήλ·
\vs{14}καὶ ἐπέθηκεν αὐτοὺς ἐπὶ τοὺς ὤμους τῆς ἐπωμίδος, λίθους μνημοσύνου τῶν υἱῶν Ἰσραὴλ, καθὰ συνέταξε Κύριος τῷ Μωυσῇ.

\vs{15}Καὶ ἐποίησαν λογεῖον, ἔργον ὑφαντὸν ποικιλίᾳ κατὰ τὸ ἔργον τῆς ἐπωμίδος, ἐκ χρυσίου, καὶ ὑακίνθου, καὶ πορφύρας, καὶ κοκκίνου διανενησμένου, καὶ βύσσου κεκλωσμένης·
\vs{16}τετράγωνον διπλοῦν ἐπόησαν τὸ λογεῖον· σπιθαμῆς τὸ μῆκος, καὶ σπιθαμῆς τὸ εὖρος διπλοῦν.
\vs{17}Καὶ συνυφάνθη ἐν αὐτῷ ὕφασμα κατάλιθον τετράστιχον· στίχος λίθων, σάρδιον καὶ τοπάζιον καὶ σμάραγδος, ὁ στίχος ὁ εἷς·
\vs{18}καὶ ὁ στίχος ὁ δεύτερος, ἄνθραξ καὶ σάπφειρος καὶ ἴασπις·
\vs{19}καὶ ὁ στίχος ὁ τρίτος, λιγύριον καὶ ἀχάτης καὶ ἀμέθυστος·
\vs{20}καὶ ὁ στίχος ὁ τέταρτος, χρυσόλιθος καὶ βηρύλλιον καὶ ὀνύχιον περικεκυκλωμένα χρυσίῳ, καὶ συνδεδεμένα χρυσίῳ.
\vs{21}Καὶ οἱ λίθοι ἦσαν ἐκ τῶν ὀνομάτων τῶν υἱῶν Ἰσραὴλ δώδεκα, ἐκ τῶν ὀνομάτων αὐτῶν ἐγγεγλυμμένα εἰς σφραγίδας, ἕκαστος ἐκ τοῦ ἑαυτοῦ ὀνόματος εἰς τὰς δώδεκα φυλάς.
\vs{22}Καὶ ἐποίησαν ἐπὶ τὸ λογεῖον κρωσσοὺς συμπεπλεγμένους, ἔργον ἐμπλοκίου, ἐκ χρυσίου καθαροῦ.
\vs{23}Καὶ ἐποίησαν δύο ἀσπιδίσκας χρυσᾶς, καὶ δύο δακτυλίους χρυσοῦς· καὶ ἐπέθηκαν τοὺς δύο δακτυλίους τοὺς χρυσοῦς ἐπʼ ἀμφοτέρας τὰς ἀρχὰς τοῦ λογείου.
\vs{24}Καὶ ἐπέθηκαν τὰ ἐμπλόκια ἐκ χρυσίου ἐπὶ τοὺς δακτυλίους ἐπʼ ἀμφοτέρων τῶν μερῶν τοῦ λογείου·
\vs{25}καὶ εἰς τὰς δύο συμβολὰς τὰ δύο ἐμπλόκια. Καὶ ἐπέθηκαν ἐπὶ τὰς δύο ἀσπιδίσκας· καὶ ἐπέθηκαν ἐπὶ τοὺς ὤμους τῆς ἐπωμίδος ἐξεναντίας κατὰ πρόσωπον.
\vs{26}Καὶ ἐποίησαν δύο δακτυλίους χρυσοῦς, καὶ ἐπέθηκαν ἐπὶ τὰ δύο πτερύγια ἐπʼ ἄκρου τοῦ λογείου, καὶ ἐπὶ τὸ ἄκρον τοῦ ὀπισθίου τῆς ἐπωμίδος ἔσωθεν·
\vs{27}Καὶ ἐποίησαν δύο δακτυλίους χρυσοῦς, καὶ ἐπέθηκαν ἐπʼ ἀμφοτέρους τοὺς ὤμους τῆς ἐπωμίδος κάτωθεν αὐτοῦ, κατὰ πρόσωπον κατὰ τὴν συμβολὴν ἄνωθεν τῆς συνυφῆς τῆς ἐπωμίδος·
\vs{28}καὶ συνέσφιγξε τὸ λογεῖον ἀπὸ τῶν δακτυλίων τῶν ἐπʼ αὐτοῦ εἰς τοὺς δακτυλίους τῆς ἐπωμίδος, συνεχομένους ἐκ τῆς ὑακίνθου, συμπεπλεγμένους εἰς τὸ ὕφασμα τῆς ἐπωμίδος, ἵνα μὴ χαλᾶται τὸ λογεῖον ἀπὸ τῆς ἐπωμίδος, καθὰ συνέταξε Κύριος τῷ Μωυσῇ.
\vs{29}Καὶ ἐποίησαν τὸν ὑποδύτην ὑπὸ τὴν ἐπωμίδα, ἔργον ὑφαντὸν, ὅλον ὑακίνθινον·
\vs{30}τὸ δὲ περιστόμιον τοῦ ὑποδύτου ἐν τῷ μέσῳ διυφασμένον συμπλεκτὸν, ὤαν ἔχον κύκλῳ τὸ περιστόμιονν ἀδιάλυτον·
\vs{31}Καὶ ἐποίησαν ἐπὶ τοῦ λώματος τοῦ ὑποδύτου κάτωθεν ὡς ἐξανθούσης ῥόας ῥοΐσκους, ἐξ ὑακίνθου, καὶ πορφύρας, καὶ κοκκίνου νενησμένου, καὶ βύσσου κεκλωσμένης.
\vs{32}Καὶ ἐποίησαν κώδωνας χρυσοῦς, καὶ ἐπέθηκαν τοὺς κώδωνας ἐπὶ τὸ λῶμα τοῦ ὑποδύτου κύκλῳ ἀνὰ μέσον τῶν ῥοΐσκων·
\vs{33}κώδων χρυσοῦς καὶ ῥοΐσκος ἐπὶ τοῦ λώματος τοῦ ὑποδύτου κύκλῳ, εἰς τὸ λειτουργεῖν, καθὰ συνέταξε Κύριος τῷ Μωυσῇ.
\vs{34}Καὶ ἐποίησαν χιτῶνας βυσσίνους, ἔργον ὑφαντὸν, Ἀαρὼν καὶ τοῖς υἱοῖς αὐτοῦ,
\vs{35}καὶ τὰς κιδάρεις ἐκ βύσσου, καὶ τὴν μίτραν ἐκ βύσσου, καὶ τὰ περισκελῆ ἐκ βύσσου κεκλωσμένης,
\vs{36}καὶ τὰς ζώνας αὐτῶν ἐκ βύσσου, καὶ ὑακίνθου, καὶ πορφύρας, καὶ κοκκίνου νενησμένου, ἔργον ποικιλτοῦ, ὃν τρόπον συνέταξε Κύριος τῷ Μωυσῇ.
\vs{37}Καὶ ἐποίησαν τὸ πέταλον τὸ χρυσοῦν, ἀφόρισμα τοῦ ἁγίου, χρυσίου καθαροῦ· καὶ ἔγραψεν ἐπʼ αὐτοῦ γράμματα ἐκτετυπωμένα, σφραγίδος, Ἁγίασμα Κυρίῳ·
\vs{38}Καὶ ἐπέθηκαν ἐπὶ τὸ λῶμα ὑακίνθινον, ὥστε ἐπικεῖσθαι ἐπὶ τὴν μίτραν ἄνωθεν, ὅν τρόπον συνέταξε Κύριος τῷ Μωυσῇ.

\ch{37}
Καὶ ἐποίησαν τῇ σκηνῇ δέκα αὐλαίας·
\vs{2}ὀκτὼ καὶ εἴκοσι πήχεων μῆκος τῆς αὐλαίας τῆς μιᾶς· τὸ αὐτὸ ἦν πάσαις· καὶ τεσσάρων πήχεων τὸ εὖρος τῆς αὐλαίας τῆς μιᾶς.
\vs{3}καὶ ἐποίησαν τὸ καταπέτασμα ἐξ ὑακίνθου, καὶ πορφύρας, καὶ κοκκίνου νενησμένου, καὶ βύσσου κεκλωσμένης, ἔργον ὑφαντὸυ χερουβείμ·
\vs{4}καὶ ἐπέθηκαν αὐτὸ ἐπὶ τέσσαρας στύλους ἀσήπτους κατακεχρυσωμένους ἐν χρυσίῳ· καὶ αἱ κεφαλίδες αὐτῶν χρυσαῖ, καὶ αἱ βάσεις αὐτῶν τέσσαρες ἀργυραῖ.
\vs{5}καὶ ἐποίησαν τὸ καταπέτασμα τῆς θύρας τῆς σκηνῆς τοῦ μαρτυρίου ἐξ ὑακίνθου, καὶ πορφύρας, καὶ κοκκίνου νενησμένου, καὶ βύσσου κεκλωσμένης, ἔργον ὑφαντὸντοῦ χερουβείμ·
\vs{6}καὶ τοὺς στύλους αὐτῶν πέντε, καὶ τοὺς κρίκους· καὶ τὰς κεφαλίδας αὐτῶν, καὶ τὰς ψαλίδας αὐτῶν κατεχρύσωσαν χρυσίῳ· καὶ αἱ βάσεις αὐτῶν πέντε χαλκαῖ.

\vs{7}Καὶ ἐποίησαν τὴν αὐλῆν τὰ πρὸς Λίβα, ἱστία τῆς αὐλῆς ἐκ βύσσου κεκλωσμένης ἑκατὸν ἐφʼ ἑκατόν·
\vs{8}καὶ οἱ στύλοι αὐτῶν εἴκοσι, καὶ αἱ βάσεις αὐτῶν εἴκοσι.
\vs{9}καὶ τὸ κλίτος τὸ πρὸς Βοῤῥᾶν, ἑκατὸν ἐφʼ ἑκατόν· καὶ τὸ κλίτος τὸ πρὸς Νότον, ἑκατὸν ἐφʼ ἑκατόν· καὶ οἱ στύλοι αὐτῶν εἴκοσι, καὶ αἱ βάσεις αὐτῶν εἴκοσι·
\vs{10}Καὶ τὸ κλίτος τὸ πρὸς θάλασσαν αὐλαῖαι πεντήκοντα πήχεων· στύλοι αὐτῶν δέκα, καὶ αἱ βάσεις αὐτῶν δέκα·
\vs{11}Καὶ τὸ κλίτος τὸ πρὸς ἀνατολὰς πεντήκοντα πήχεων ἱστία, πεντεαίδεκα πήχεων τὸ κατὰ νώτου·
\vs{12}καὶ οἱ στύλοι αὐτῶν τρεῖς, καὶ αἱ βάσεις αὐτῶν τρεῖς·
\vs{13}Καὶ ἐπὶ τοῦ νώτου τοῦ δευτέρου ἔνθεν καὶ ἔνθεν κατὰ τὴν πύλην τῆς αὐλῆς, αὐλαῖαι πεντεκαίδεκα πήχεων· στύλοι αὐτῶν τρεῖς, καὶ αἱ βάσεις αὐτῶν τρεῖς·
\vs{14}πᾶσαι αἱ αὐλαῖαι τῆς σκηνῆς ἐκ βύσσου κεκλωσμένης.
\vs{15}Καὶ αἱ βάσεις τῶν στύλων αὐτῶν χαλκαῖ, καὶ αἱ ἀγκύλαι αὐτῶν ἀργυραῖ, καὶ αἱ κεφαλίδες αὐτῶν περιηργυρωμέναι ἀργυρίῳ, καὶ οἱ στύλοι περιηργυρωμένοι ἀργυρίῳ πάντες οἱ στύλοι τῆς αὐλῆς·
\vs{16}καὶ τὸ καταπέτασμα τῆς πύλης τῆς αὐλῆς ἔργον ποικιλτοῦ ἐξ ὑακίνθου, καὶ πορφύρας, καὶ κοκκίνου νενησμένου, καὶ βύσσου κεκλωσμένης· εἴκοσι πήχεων τὸ μῆκος, καὶ τὸ ὕψος καὶ τὸ εὖρος πέντε πήχεων ἐξισούμενον τοῖς ἱστίοις τῆς αὐλῆς·
\vs{17}καὶ οἱ στύλοι αὐτῶν τέσσαρες, καὶ αἱ βάσεις αὐτῶν τέσσαρες χαλκαῖ, καὶ αἱ ἀγκύλαι αὐτῶν ἀργυραῖ, καὶ αἱ κεφαλίδες αὐτῶν περιηργυρωμέναι ἀργυρίῳ.
\vs{18}Καὶ πάντες οἱ πάσσαλοι τῆς αὐλῆς κύκλῳ χαλκοῖ, καὶ αὐτοὶ περιηργυρωμένοι ἀργυρίῳ.
\vs{19}Καὶ αὕτη ἡ σύνταξις τῆς σκηνῆς τοῦ μαρτυρίου, καθὰ συνετάγη Μωυσῇ, τὴν λειτουργίαν εἶναι τῶν Λευιτῶν διὰ Ἰθάμαρ τοῦ υἱοῦ Ἀαρὼν τοῦ ἱερέως.

\vs{20}Καὶ Βεσελεὴλ ὁ τοῦ Οὐρείου, ἐκ φυλῆς Ἰούδα, ἐποίησε καθὰ συνέταξε Κύριος τῷ Μωυσῇ,
\vs{21}καὶ Ἐλιὰβ ὁ τοῦ Ἀχισαμὰχ ἐκ φυλῆς Δὰν, ὅς ἠρχιτεκτόνησε τὰ ὑφαντὰ καὶ τὰ ῥαφιδευτὰ καὶ ποικιλτικά, ὑφάναι τῷ κοκκίνῳ καὶ τῇ βύσσῳ.

\ch{38}
Καὶ ἐποίησε Βεσελεὴλ τὴν κιβωτόν,
\vs{2}καὶ κατεχρύσωσεν αὐτὴν χρωσίῳ καθαρῷ ἔσωθεν καὶ ἔξωθεν·
\vs{3}καὶ ἐχώνευσεν αὐτῇ τέσσαρας δακτυλίους χρυσοῦς· δύο ἐπὶ τὸ κλίτος τὸ ἓν, καὶ δύο ἐπὶ τὸ κλίτος τὸ δεύτερον,
\vs{4}εὐρεῖς τοῖς διωστῆρσιν, ὥστε αἴρειν αὐτὴν ἐν αὐτοῖς.
\vs{5}Καὶ ἐποίησε τὸ ἱλαστήριον ἐπάνωθεν τῆς κιβωτοῦ ἐκ χρυσίου καθαροῦ,
\vs{6}καὶ τοὺς δύο χερουβεὶμ χρυσοῦς·
\vs{7}χεροὺβ ἕνα ἐπὶ τὸ ἄκρον τοῦ ἱλαστηρίου τὸ ἓν, καὶ χεροὺβ ἕνα ἐπὶ τὸ ἄκρον τοῦ ἱλαστηρίου τὸ δεύτερον,
\vs{8}σκιάζοντα ταῖς πτέρυξιν αὐτῶν ἐπὶ τὸ ἱλαστήριον.
\vs{9}Καὶ ἐποίησε τὴν τράπεζαν τὴν προκειμένην ἐκ χρυσίου καθαροῦ,
\vs{10}καὶ ἐχώνευσεν αὐτῇ τέσσαρας δακτυλίους, δύο ἐπὶ τοῦ κλίτους τοῦ ἑνὸς, καὶ δύο ἐπὶ τοῦ κλίτους τοῦ δευτέρου, εὐρεῖς, ὥστε αἴρειν τοῖς διωστῆρσιν ἐν αὐτοῖς.
\vs{11}Καὶ τοὺς διωστῆρας τῆς κιβωτοῦ καὶ τῆς τραπέζης ἐποίησε, καὶ κατεχρύσωσεν αὐτοὺς χρυσίῳ.
\vs{12}Καὶ ἐποίησε τὰ σκεύη τῆς τραπέζης, τά τε τρυβλία, καὶ τὰς θυίσκας, καὶ τοὺς κυάθους, καὶ τὰ σπονδεῖα, ἐν οἷς σπείσει ἐν αὐτοῖς, χρυσᾶ.
\vs{13}Καὶ ἐποίησε τὴν λυχνίαν ἣ φωτίζει, χρυσῆν,
\vs{14}στερεὰν τὸν καυλόν, καὶ τοὺς καλαμίσκους ἐξ ἀμφοτέρων τῶν μερῶν αὐτῆς·
\vs{15}ἐκ τῶν καλαμίσκων αὐτῆς οἱ βλαστοὶ ἐκ τῶν καλαμίσκων αὐτῆς οἱ βλαστοὶ ἐξέχοντες· τρεῖς ἐκ τούτου, καὶ τρεῖς ἐκ τούτου, ἐξισούμενοι ἀλλήλοις.
\vs{16}Καὶ τὰ λαμπάδια αὐτῶν, ἅ ἐστιν ἐπὶ τῶν ἄκρων, καρυωτὰ ἐξ αὐτῶν· καὶ τὰ ἐνθέμια ἐξ αὐτῶν, ἵνα ὦσιν οἱ λύχνοι ἐπʼ αὐτῶν· καὶ τὸ ἐνθέμιον τὸ ἕβδομον, τὸ ἐπʼ ἄκρου τοῦ λαμπαδίου, ἐπὶ τῆς κορυφῆς ἄνωθεν, στερεὸν ὅλον χρυσοῦν·
\vs{17}Καὶ ἑπτὰ λύχνους ἐπʼ αὐτῆς χρυσοῦς, καὶ τὰς λαβίδας αὐτῆς χρυσᾶς, καὶ τὰς ἐπαρυστρίδας αὐτῶν χρυσᾶς.
\vs{18}Οὗτος περιηργύρωσε τοὺς στύλους, καὶ ἐχώνευσε τῷ στύλῳ δακτυλίους χρυσοῦς, καὶ ἐχρύσωσε τοὺς μοχλοὺς χρυσίῳ, καὶ κατεχρύσωσε τοὺς στύλους τοῦ καταπετάσματος χρυσίῳ, καὶ ἐποίησε τὰς ἀγκύλας χρυσᾶς.
\vs{19}Οὗτος ἐποίησε καὶ τοὺς κρίκους τῆς σκηνῆς χρυσοῦς, καὶ τοὺς κρίκους τῆς αὐλῆς, καὶ κρίκους εἰς τὸ ἐκτείνειν τὸ κατακάλυμμα ἄνωθεν χαλκοῦς·
\vs{20}Οὗτος ἐχώνευσε τὰς κεφαλίδας τὰς ἀργυρᾶς τῆς σκηνῆς, καὶ τὰς κεφαλίδας τὰς χαλκᾶς τῆς θύρας τῆς σκηνῆς, καὶ τὴν πύλην τῆς αὐλῆς· καὶ ἀγκύλας ἐποίησε τοῖς στύλοις ἀργυρᾶς, ἐπὶ τῶν στύλων οὗτος περιηργύρωσεν αὐτάς·
\vs{21}Οὗτος ἐποίησε τοὺς πασσάλους τῆς σκηνῆς, καὶ τοὺς πασσάλους τῆς αὐλῆς χαλκοῦς·
\vs{22}Οὗτος ἐποίησε τὸ θυσιαστήριον τὸ χαλκοῦν ἐκ τῶν πυρείων τῶν χαλκῶν, ἃ ἦσαν τοῖς ἀνδράσιν τοῖς καταστασιάσασι μετὰ τῆς Κορὲ συναγωγῆς·
\vs{23}Οὗτος ἐποίησε πάντα τὰ σκεύη τοῦ θυσιαστηρίου, καὶ τὸ πυρεῖον αὐτοῦ, καὶ τὴν βάσιν, καὶ τὰς φιάλας, καὶ τὰς κρεάγρας τὰς χαλκᾶς·
\vs{24}Οὗτος ἐποίησε θυσιαστηρίῳ παράθεμα, ἔργον δικτυωτὸν κάτωθεν τοῦ πυρείου ὑπὸ αὐτὸ ἕως τοῦ ἡμίσους αὐτοῦ· καὶ ἐπέθηκεν αὐτῷ τέσσαρας δακτυλίους ἐκ τῶν τεσσάρων μερῶν τοῦ παραθέματος τοῦ θυσιαστηρίου χαλκοῦς, εὐρεῖς τοῖς μοχλοῖς, ὥστε αἴρειν ἐν αὐτοῖς τὸ θυσιαστήριον·
\vs{25}Οὗτος ἐποίησε τὸ ἔλαιον τῆς χρίσεως τὸ ἅγιον, καὶ τὴν σύνθεσιν τοῦ θυμιάματος καθαρὸν ἔργον μυρεψοῦ·
\vs{26}Οὗτος ἐποίησε τὸν λουτῆρα τὸν χαλκοῦν, καὶ τὴν βάσιν αὐτοῦ χαλκῆν ἐκ τῶν κατόπτρων τῶν νηστευσασῶν, αἳ ἐνήστευσαν παρὰ τὰς θύρας τῆς σκηνῆς τοῦ μαρτυρίου, ἐν ᾗ ἡμέρᾳ ἔπηξεν αὐτήν.

\vs{27}Καὶ ἐποίησε τὸν λουτῆρα, ἵνα νίπτωνται ἐξ αὐτοῦ Μωυσῆς καὶ Ἀαρὼν καὶ οἱ υἱοὶ αὐτοῦ τὰς χεῖρας αὐτῶν καὶ τοὺς πόδας, εἰσπορευομένων αὐτῶν εἰς τὴν σκηνὴν τοῦ μαρτυρίου, ἢ ὅταν προσπορεύωνται πρὸς τὸ θυσιαστήριον λειτουργεῖν, ἐνίπτοντο ἐξ αὐτοῦ, καθάπερ συνέταξε Κύριος τῷ Μωυσῇ.

\ch{39}
Πᾶν τὸ χρυσίον, ὃ κατειργάσθη εἰς τὰ ἔργα κατὰ πᾶσαν τὴν ἐργασίαν τῶν ἁγίων, ἐγένετο χρυσίου τοῦ τῆς ἀπαρχῆς, ἐννέα καὶ εἴκοσι τάλαντα, καὶ ἑπτακόσιοι εἴκοσι σίκλοι κατὰ τὸν σίκλον τὸν ἅγιον·
\vs{2}Καὶ ἀργυρίου ἀφαίρεμα παρὰ τῶν ἐπεσκεμμένων ἀνδρῶν τῆς συναγωγῆς ἑκατὸν τάλαντα, καὶ χίλιοι ἑπτακόσιοι ἑβδομηκονταπέντε σίκλοι· δραχμὴ μία τῇ κεφαλῇ τὸ ἥμισυ τοῦ σίκλου, κατὰ τὸν σίκλον τὸν ἅγιον·
\vs{3}Πᾶς ὁ παραπορευόμενος τὴν ἐπίσκεψιν ἀπὸ εἰκοσαετοῦς καὶ ἐπάνω εἰς τὰς ἑξήκοντα μυριάδας, καὶ τρισχίλιοι πεντακόσιοι καὶ πεντήκοντα.
\vs{4}Καὶ ἐγενήθη τὰ ἑκατὸν τάλαντα τοῦ ἀργυρίου εἰς τὴν χώνευσιν τῶν ἑκατὸν κεφαλίδων τῆς σκηνῆς, καὶ εἰς τὰς κεφαλίδας τοῦ καταπετάσματος,
\vs{5}ἑκατὸν κεφαλίδες εἰς τὰ ἑκατὸν τάλαντα, τάλαντον τῇ κεφαλίδι·
\vs{6}Καὶ τοὺς χιλίους ἑπτακοσίους ἑβδομηκοντα πέντε σίκλους ἐποίησεν εἰς τὰς ἀγκύλας τοῖς στύλοις· καὶ κατεχρύσωσε τὰς κεφαλίδας αὐτῶν, καὶ κατεκόσμησεν αὐτούς.

\vs{7}Καὶ ὁ χαλκὸς τοῦ ἀφαιρέματος ἑβδομήκοντα τάλαντα, καὶ χίλιοι πεντακόσιοι σίκλοι·
\vs{8}Καὶ ἐποίησαν ἐξ αὐτον τὰς βάσεις τῆς θύρας τῆς σκηνῆς τοῦ μαρτυρίου,
\vs{9}καὶ τὰς βάσεις τῆς αὐλῆς κύκλῳ, καὶ τὰς βάσεις τῆς πύλης τῆς αὐλῆς, καὶ τοὺς πασσάλους τῆς σκηνῆς, καὶ τοὺς πασσάλους τῆς αὐλῆς κύκλῳ,
\vs{10}καὶ τὸ παράθεμα τὸ χαλκοῦν τοῦ θυσιαστηρίου, καὶ πάντα τὰ σκεύη τοῦ θυσιαστηρίου, καὶ πάντα τὰ ἐργαλεῖα τῆς σκηνῆς τοῦ μαρτυρίου·
\vs{11}Καὶ ἐποίησαν οἱ υἱοὶ Ἰσραὴλ, καθὰ συνέταξε Κύριος τῷ Μωυσῇ, οὕτως ἐποίησαν·
\vs{12}Τὸ δὲ λοιπὸν χρυσίον τοῦ ἀφαιρέματος ἐποίησαν σκεύη εἰς τὸ λειτουργεῖν ἐν αὐτοῖς ἔναντι Κυρίου·
\vs{13}Καὶ τὴν καταλειφθεῖσαν ὑάκινθον, καὶ πορφύραν, καὶ τὸ κόκκινον ἐποίησαν στολὰς λειτουργικὰς Ἀαρών, ὥστε λειτουργεῖν ἐν αὐταῖς ἐν τῷ ἁγίῳ·
\vs{14}Καὶ ἤνεγκαν τὰς στολὰς πρὸς Μωυσῆν, καὶ τὴν σκηνὴν, καὶ τὰ σκεύη αὐτῆς, τὰς βάσεις καὶ τοὺς μοχλοὺς αὐτῆς, καὶ τοὺς στύλους·
\vs{15}καὶ τὸ θυσιαστήριον, καὶ πάντα τὰ σκεύη αὐτοῦ.

\vs{16}Καὶ τὸ ἔλαιον τῆς χρίσεως, καὶ τὸ θυμίαμα τῆς συνθέσεως, καὶ τὴν λυχνίαν τὴν καθαρὰν,
\vs{17}καὶ τοὺς λύχνους αὐτῆς, λύχνους τῆς καύσεως, καὶ τὸ ἔλαιον τοῦ φωτός·
\vs{18}Καὶ τὴν τράπεζαν τῆς προθέσεως, καὶ πάντα τὰ σκεύη αὐτῆς· καὶ τοὺς ἄρτους τοὺς προκειμένους·
\vs{19}Καὶ τὰς στολὰς τοῦ ἁγίου, αἵ εἰσιν Ἀαρών, καὶ τὰς στολὰς τῶν υἱῶν αὐτοῦ, εἰς τὴν ἱερατείαν·
\vs{20}Καὶ τὰ ἱστία τῆς αὐλῆς, καὶ τοὺς στύλους· καὶ τὸ καταπέτασμα τῆς θύρας τῆς σκηνῆς, καὶ τῆς πύλης τῆς αὐλῆς·
\vs{21}Καὶ πάντα τὰ σκεύη τῆς σκηνῆς, καὶ πάντα τὰ ἐργαλεῖα αὐτῆς· καὶ τὰς διφθέρας δέρματα κριῶν ἠρυθροδανωμένα, καὶ τὰ καλύμματα ὑακίνθινα, καὶ τῶν λοιπῶν τὰ ἐπικαλύμματα· καὶ τοὺς πασσάλους, καὶ πάντα τὰ ἐργαλεῖα τὰ εἰς τὰ ἔργα τῆς σκηνῆς τοῦ μαρτυρίου·
\vs{22}Ὃσα συνέταξε Κύριος τῷ Μωυσῇ, οὕτως ἐποίησαν οἱ υἱοὶ Ἰσραὴλ πᾶσαν τὴν ἀποσκευήν·
\vs{23}Καὶ εἶδε Μωυσῆς πάντα τὰ ἔργα, καὶ ἦσαν πεποιηκότες αὐτὰ ὃν τρόπον συνέταξε Κύριος τῷ Μωυσῇ, οὕτως ἐποίησαν αὐτὰ, καὶ εὐλόγησεν αὐτοὺς Μωυσῆς.

\ch{40}
Καὶ ἐλάλησε Κύριος πρὸς Μωυσῆν, λέγων,
\vs{2}ἐν ἡμέρᾳ μιᾷ τοῦ μηνὸς τοῦ πρώτου νουμηνίᾳ, στήσεις τὴν σκηνὴν τοῦ μαρτυρίου.
\vs{3}Καὶ θήσεις τὴν κιβωτὸν τοῦ μαρτυρίου, καὶ σκεπάσεις τὴν κιβωτὸν τῷ καταπετάσματι.
\vs{4}Καὶ εἰσοίσεις τὴν τράπεζαν, καὶ προθήσεις τὴν πρόθεσιν αὐτῆς· καὶ εἰσοίσεις τὴν λυχνίαν, καὶ ἐπιθήσεις τοὺς λύχνους αὐτῆς.
\vs{5}Καὶ θήσεις τὸ θυσιαστήριον τὸ χρυσοῦν, εἰς τὸ θυμιᾷν ἐναντίον τῆς κιβωτοῦ· καὶ ἐπιθήσεις κάλυμμα καταπετάσματος ἐπὶ τὴν θύραν τῆς σκηνῆς τοῦ μαρτυρίου.
\vs{6}Καὶ τὸ θυσιαστήριον τῶν καρπωμάτων θήσεις παρὰ τὰς θύρας τῆς σκηνῆς τοῦ μαρτυρίου·
\vs{8}καὶ περιθήσεις τὴν σκηνήν, καὶ πάντα τὰ αὐτῆς ἁγιάσεις κύκλῳ.
\vs{9}Καὶ λήψῃ τὸ ἔλαιον τοῦ χρίσματος, καὶ χρίσεις τὴν σκηνὴν, καὶ πάντα τὰ ἐν αὐτῇ, καὶ ἁγιάσεις αὐτὴν, καὶ πάντα τὰ σκεύη αὐτῆς, καὶ ἔσται ἁγία.
\vs{10}Καὶ χρίσεις τὸ θυσιαστήριον τῶν καρπωμάτων, καὶ πάντα τὰ σκεύη αὐτοῦ· καὶ ἁγιάσεις τὸ θυσιαστήριον, καὶ ἔσται τὸ θυσιαστήριον ἅγιον τῶν ἁγίων.
\vs{12}Καὶ προσάξεις Ἀαρὼν καὶ τοὺς υἱοὺς αὐτοῦ ἐπὶ τὰς θύρας τῆς σκηνῆς τοῦ μαρτυρίου, καὶ λούσεις αὐτοὺς ὕδατι.
\vs{13}Καὶ ἐνδύσεις Ἀαρὼν τὰς στολὰς τὰς ἁγίας, καὶ χρίσεις αὐτὸν, καὶ ἁγιάσεις αὐτὸν, καὶ ἱερατεύει μοι.
\vs{14}Καὶ τοὺς υἱοὺς αὐτοῦ προσάξεις, καὶ ἐνδύσεις αὐτοὺς χιτῶνας.
\vs{15}Καὶ ἀλείψεις αὐτοὺς ὃν τρόπον ἤλειψας τὸν πατέρα αὐτῶν, καὶ ἱερατεύσουσί μοι· καὶ ἔσται, ὥστε εἶναι αὐτοῖς χρίσμα ἱερατείας εἰς τὸν αἰῶνα, εἰς τὰς γενεὰς αὐτῶν.
\vs{16}Καὶ ἐποίησε Μωυσῆς πάντα, ὅσα ἐνετείλατο αὐτῷ Κύριος, οὕτως ἐποίησε.

\vs{17}Καὶ ἐγένετο ἐν τῷ μηνὶ τῷ πρώτῳ, τῷ δευτέρῳ ἔτει, ἐκπορευομένων αὐτῶν ἐξ Αἰγύπτου, νουμηνίᾳ ἐστάθη ἡ σκηνή.
\vs{18}Καὶ ἔστησε Μωυσῆς τὴν σκηνὴν, καὶ ἐπέθηκε τὰς κεφαλιδας, καὶ διενέβαλε τοὺς μοχλοὺς, καὶ ἔστησε τοὺς στύλους.
\vs{19}Καὶ ἐξέτεινε τὰς αὐλαίας ἐπὶ τὴν σκηνὴν, καὶ ἐπέθηκε τὸ κατακάλυμμα τῆς σκηνῆς ἐπʼ αὐτὴν ἄνωθεν, καθὰ συνέταξε Κύριος τῷ Μωυσῇ.
\vs{20}Καὶ λαβὼν τὰ μαρτύρια ἐνέβαλεν εἰς τὴν κιβωτόν· καὶ ὑπέθηκε τοὺς διωστῆρας ὑπὸ τὴν κιβωτὸν,
\vs{21}καὶ εἰσήνεγκε τὴν κιβωτὸν εἰς τὴν σκηνὴν, καὶ ἐπέθηκε τὸ κατακάλυμμα τοῦ καταπετάσματος, καὶ ἐσκέπασε τὴν κιβωτὸν τοῦ μαρτυρίου, ὃν τρόπον συνέταξε Κύριος τῷ Μωυσῇ·
\vs{22}Καὶ ἐπέθηκε τὴν τράπεζαν εἰς τὴν σκηνὴν τοῦ μαρτυρίου, τὸ πρὸς Βοῤῥᾶν ἔξωθεν τοῦ καταπετάσματος τῆς σκηνῆς.
\vs{23}Καὶ προσέθηκεν ἐπʼ αὐτῆς ἄρτους τῆς προθέσεως ἔναντι Κυρίου, ὃν τρόπον συνέταξε Κύριος τῷ Μωυσῇ.
\vs{24}Καὶ ἔθηκε τὴν λυχνίαν εἰς τὴν σκηνὴν τοῦ μαρτυρίου, εἰς τὸ κλίτος τῆς σκηνῆς τὸ πρὸς Νότον.
\vs{25}Καὶ ἐπέθηκε τοὺς λύχνους αὐτῆς ἔναντι Κυρίου, ὃν τρόπον συνέταξε Κύριος τῷ Μωυσῇ.
\vs{26}Καὶ ἔθηκε τὸ θυσιαστήριον τὸ χρυσοῦν ἐν τῇ σκηνῇ τοῦ μαρτυρίου ἀπέναντι τοῦ καταπετάσματος,
\vs{27}καὶ ἐθυμίασεν ἐνʼ αὐτοῦ θυμίαμα τῆς συνθέσεως, καθάπερ συνέταξε Κύριος τῷ Μωυσῇ.
\vs{29}Καὶ τὸ θυσιαστήριον τῶν καρπωμάτων ἔθηκε παρὰ τὰς θύρας τῆς σκηνῆς.
\vs{33}Καὶ ἔστησε τὴν αὐλὴν κύκλῳ τῆς σκηνῆς, και τοῦ θυσιαστηρίου· καὶ συνετέλεσε Μωυσῆς πάντα τὰ ἔργα.

\vs{34}Καὶ ἐκάλυψεν ἡ νεφέλη τὴν σκηνὴν τοῦ μαρτυρίου· καὶ δόξης Κυρίου ἐπλήσθη ἡ σκηνή.
\vs{35}Καὶ οὐκ ἠδυνάσθη Μωυσῆς εἰσελθεῖν εἰς τὴν σκηνὴν τοῦ μαρτυρίου, ὅτι ἐπεσκίαζεν ἐπʼ αὐτὴν ἡ νεφέλη, καὶ δόξης Κυρίου ἐνεπλήσθη ἡ σκηνή.
\vs{36}Ἡνίκα δʼ ἂν ἀνέβη ἡ νεφέλη ἀπὸ τῆς σκηνῆς, ἀνεζεύγνυσαν οἱ υἱοὶ Ἰσραὴλ σὺν τῇ ἀπαρτίᾳ αὐτῶν.
\vs{37}Εἰ δὲ μὴ ἀνέβη ἡ νεφέλη, οὐκ ἀνεζεύγνυσαν ἕως ἡμέρας, ἧς ἀνέβη ἡ νεφέλη.
\vs{38}Νεφέλη γὰρ ἦν ἐπὶ τῆς σκηνῆς ἡμέρας, καὶ πῦρ ἦν ἐπʼ αὐτῆς νυκτὸς ἐναντίον παντὸς Ἰσραὴλ, ἐν πάσαις ταῖς ἀναζυγαῖς αὐτῶν.


\def\book{ΛΕΥΙΤΙΚΟΝ}
\biblebook{ΛΕΥΙΤΙΚΟΝ}


\lettrine[lines=2, loversize=0.2, nindent=0em, findent=.25em]{\textcolor{bookheadingcolor}{Κ}}{ΑΙ} ἀνεκάλεσε Μωυσῆν, καὶ ἐλάλησε Κύριος αὐτῷ ἐκ τῆς σκηνῆς τοῦ μαρτυρίου, λέγων,
\vs{2}λάλησον τοῖς υἱοῖς Ἰσραὴλ, καὶ ἐρεῖς πρὸς αὐτοὺς, ἄνθρωπος ἐξ ὑμῶν ἐὰν προσαγάγῃ δῶρα τῷ Κυρίῳ, ἀπὸ τῶν κτηνῶν καὶ ἀπὸ τῶν βοῶν καὶ ἀπὸ τῶν προβάτων προσοίσετε τὰ δῶρα ὑμῶν.
\vs{3}Ἐὰν ὁλοκαύτωμα τὸ δῶρον αὐτοῦ, ἐκ τῶν βοῶν ἄρσεν ἄμωμον προσάξει πρὸς τὴν θύραν τῆς σκηνῆς τοῦ μαρτυρίου, προσοίσει αὐτὸ δεκτὸν ἐναντίον Κυρίου.
\vs{4}Καὶ ἐπιθήσει τὴν χεῖρα ἐπὶ τὴν κεφαλὴν τοῦ καρπώματος δεκτὸν αὐτῷ, ἐξιλάσασθαι περὶ αὐτοῦ.
\vs{5}Καὶ σφάξουσι τὸν μόσχον ἔναντι Κυρίου· καὶ προσοίσουσιν οἱ υἱοὶ Ἀαρὼν οἱ ἱερεῖς τὸ αἷμα, καὶ προσχεοῦσι τὸ αἷμα ἐπὶ τὸ θυσιαστήριον κύκλῳ τὸ ἐπὶ τῶν θυρῶν τῆς σκηνῆς τοῦ μαρτυρίου·
\vs{6}καὶ ἐκδείραντες τὸ ὁλοκαύτωμα, μελιοῦσιν αὐτὸ κατὰ μέλη.
\vs{7}Καὶ ἐπιθήσουσιν οἱ υἱοὶ Ἀαρὼν οἱ ἱερεῖς πῦρ ἐπὶ τὸ θυσιαστήριον, καὶ ἐπιστοιβάσουσι ξύλα ἐπὶ τὸ πῦρ.
\vs{8}Καὶ ἐπιστοιβάσουσιν οἱ υἱοὶ Ἀαρὼν οἱ ἱερεῖς τὰ διχοτομήματα, καὶ τὴν κεφαλὴν, καὶ τὸ στέαρ ἐπὶ τὰ ξύλα τὰ ἐπὶ τοῦ πυρὸς τὰ ὄντα ἐπὶ τοῦ θυσιαστηρίου.
\vs{9}Τὰ δὲ ἐγκοίλια καὶ τοὺς πόδας πλυνοῦσιν ὕδατι· καὶ ἐπιθήσουσιν οἱ ἱερεῖς τὰ πάντα ἐπὶ τὸ θυσιαστήριον· κάρπωμά ἐστι θυσία ὀσμὴ εὐωδίας τῷ Κυρίῳ.
\vs{10}Ἐὰν δὲ ἀπὸ τῶν προβάτων τὸ δῶρον αὐτοῦ τῷ Κυρίῳ, ἀπό τε τῶν ἀρνῶν, καὶ τῶν ἐρίφων εἰς ὁλοκαυτώματα, ἄρσεν ἄμωμον προσάξει αὐτό.
\vs{11}Καὶ ἐπιθήσει τὴν χεῖρα ἐπὶ τὴν κεφαλὴν αὐτοῦ· καὶ σφάξουσιν αὐτὸ ἐκ πλαγίων τοῦ θυσιαστηρίου πρὸς βοῤῥᾶν ἔναντι Κυρίου· καὶ προσχεοῦσιν οἱ υἱοὶ Ἀαρὼν οἱ ἱερεῖς τὸ αἷμα αὐτοῦ ἐπὶ τὸ θυσιαστήριον κύκλῳ.
\vs{12}Καὶ διελουσιν αὐτὸ κατὰ μέλη, καὶ τὴν κεφαλὴν, καὶ τὸ στέαρ· καὶ ἐπιστοιβάσουσιν οἱ ἱερεῖς αὐτὰ ἐπὶ τὰ ξύλα τὰ ἐπὶ τοῦ πυρὸς τὰ ἐπὶ τοῦ θυσιαστηρίου.
\vs{13}Καὶ τὰ ἐγκοίλια, καὶ τοὺς πόδας πλυνοῦσιν ὕδατι· καὶ προσοίσει ὁ ἱερεὺς τὰ πάντα, καὶ ἐπιθήσει ἐπὶ τὸ θυσιαστήριον· κάρπωμά ἐστι θυσία ὀσμὴ εὐωδίας τῷ Κυρίῳ.
\vs{14}Ἐὰν δὲ ἀπὸ τῶν πετεινῶν κάρπωμα προσφέρει δῶρον αὐτοῦ τῷ κυρίῳ, καὶ προσοίσει ἀπὸ τῶν τρυγόνων, ἢ ἀπὸ τῶν περιστερῶν τὸ δῶρον αὐτοῦ.
\vs{15}Καὶ προσοίσει αὐτὸ ὁ ἱερεὺς πρὸς τὸ θυσιαστήριον, καὶ ἀποκνίσει τὴν κεφαλήν, καὶ ἐπιθήσει ὁ ἱερεὺς ἐπὶ τὸ θυσιαστήριον, καὶ στραγγιεῖ τὸ αἷμα πρὸς τὴν βάσιν τοῦ θυσιαστηρίου.
\vs{16}Καὶ ἀφελεῖ τὸν πρόλοβον σὺν τοῖς πτεροῖς, καὶ ἐκβαλεῖ αὐτὸ παρὰ τὸ θυσιαστήριον κατʼ ἀνατολὰς εἰς τὸν τόπον τῆς σποδοῦ·
\vs{17}Καὶ ἐκκλάσει αὐτὸ ἐκ τῶν πτερύγων, καὶ οὐ διελεῖ, καὶ ἐπιθήσει αὐτὸ ὁ ἱερεὺς ἐπὶ τὸ θυσιαστήριον ἐπὶ τὰ ξύλα τὰ ἐπὶ τοῦ πυρός· κάρπωμά ἐστι θυσία ὀσμὴ εὐωδίας τῷ Κυρίῳ.

\ch{2}
Ἐὰν δὲ ψυχὴ προσφέρῃ δῶρον θυσίαν τῷ Κυρίῳ, σεμίδαλις ἔσται τὸ δῶρον αὐτοῦ, καὶ ἐπιχεεῖ ἐπʼ αὐτὸ ἔλαιον, καὶ ἐπιθήσει ἐπʼ αὐτὸ λίβανον· θυσία ἐστί.
\vs{2}Καὶ οἴσει πρὸς τοὺς υἱοὺς Ἀαρὼν τοὺς ἱερεῖς· καὶ δραξάμενος ἀπʼ αὐτῆς πλήρη τὴν δράκα ἀπὸ τῆς σεμιδάλεως σὺν τῷ ἐλαίῳ, καὶ πάντα τὸν λίβανον αὐτῆς, καὶ ἐπιθήσει ὁ ἱερεὺς τὸ μνημόσυνον αὐτῆς ἐπὶ τὸ θυσιαστήριον· θυσία ὀσμὴ εὐωδίας τῷ Κυρίῳ.
\vs{3}Καὶ τὸ λοιπὸν ἀπὸ τῆς θυσίας Ἀαρὼν καὶ τοῖς υἱοῖς αὐτοῦ, ἅγιον τῶν ἁγίων ἀπὸ τῶν θυσιῶν Κυρίου.
\vs{4}Ἐὰν δὲ προσφέρῃ δῶρον θυσίαν πεπεμμένην ἐκ κλιβάνου δῶρον Κυρίῳ ἐκ σεμιδάλεως, ἄρτους ἀζύμους πεφυραμένους ἐν ἐλαίῳ, καὶ λάγανα ἄζυμα διακεχρισμένα ἐν ἐλαίῳ.
\vs{5}Ἐὰν δὲ θυσία ἀπὸ τηγάνου τὸ δῶρόν σου, σεμίδαλις πεφυραμένη ἐν ἐλαίῳ ἄζυμά ἐστι.
\vs{6}Καὶ διαθρύψεις αὐτὰ κλάσματα, καὶ ἐπιχεεῖς ἐπʼ αὐτὰ ἔλαιον· θυσία ἐστὶ Κυρίῳ.
\vs{7}Ἐὰν δὲ θυσία ἀπὸ ἐσχάρας τὸ δῶρόν σου, σεμίδαλις ἐν ἐλαίῳ ποιηθήσεται.
\vs{8}Καὶ προσοίσει τὴν θυσίαν ἣν ἂν ποιήσῃ ἐκ τούτων τῷ Κυρίῳ, καὶ προσοίσει πρὸς τὸν ἱερέα.
\vs{9}Καὶ προσεγγίσας πρὸς τὸ θυσιαστήριον, ἀφελεῖ ὁ ἱερεὺς ἀπὸ τῆς θυσίας τὸ μνημόσυνον αὐτῆς, καὶ ἐπιθήσει ὁ ἱερεὺς ἐπὶ τὸ θυσιαστήριον, κάρπωμα· ὀσμὴ εὐωδίας Κυρίῳ.
\vs{10}Τὸ δὲ καταλειφθὲν ἀπὸ τῆς θυσίας, Ἀαρὼν καὶ τοῖς υἱοῖς αὐτοῦ, ἅγια τῶν ἁγίων ἀπὸ τῶν καρπωμάτων Κυρίου.
\vs{11}Πᾶσαν θυσίαν, ἣν ἂν προσφέρητε Κυρίῳ, οὐ ποιήσετε ζυμωτόν· πᾶσαν γὰρ ζύμην, καὶ πᾶν μέλι οὐ προσοίσετε ἀπʼ αὐτοῦ, καρπῶσαι Κυρίῳ δῶρον.
\vs{12}Ἀπαρχῆς προσοίσετε αὐτὰ Κυρίῳ, ἐπὶ δὲ τὸ θυσιαστήριον οὐκ ἀναβιβασθήσεται εἰς ὀσμὴν εὐωδίας Κυρίῳ.
\vs{13}Καὶ πᾶν δῶρον θυσίας ὑμῶν ἁλὶ ἁλισθήσεται· οὐ διαπαύσατε ἅλας διαθήκης Κυρίου ἀπὸ θυσιασμάτων ὑμῶν· ἐπὶ παντὸς δώρου ὑμῶν προσοίσετε Κυρίῳ τῷ Θεῷ ὑμῶν ἅλας.
\vs{14}Ἐὰν δὲ προσφέρῃς θυσίαν πρωτογεννημάτων τῷ Κυρίῳ, νέα πεφρυγμένα χίδρα ἐρικτὰ τῷ Κυρίῳ· καὶ προσοίσεις τὴν θυσίαν τῶν πρωτογεννημάτων.
\vs{15}Καὶ ἐπιχεεῖς ἐπʼ αὐτὴν ἔλαιον, καὶ ἐπιθήσεις ἐπʼ αὐτὴν λίβανον· θυσία ἐστί.
\vs{16}Καὶ ἀνοίσει ὁ ἱερεὺς τὸ μνημόσυνον αὐτῆς ἀπὸ τῶν χίδρων σὺν τῷ ἐλαίῳ, καὶ πάντα τὸν λίβανον αὐτῆς· κάρπωμά ἐστι Κυρίῳ.

\ch{3}
Ἐὰν δὲ θυσία σωτηρίου τὸ δῶρον αὐτοῦ τῷ Κυρίῳ, ἐὰν μὲν ἐκ τῶν βοῶν αὐτὸ προσαγάγῃ, ἐάν τω ἄρσεν, ἐάν τε θῆλυ, ἄμωμον προσάξει αὐτὸ ἔναντι Κυρίου.
\vs{2}Καὶ ἐπιθήσει τὰς χεῖρας ἐπὶ τὴν κεφαλὴν τοῦ δώρου, καὶ σφάξει αὐτὸ ἐναντίον Κυρίου παρὰ τὰς θύρας τῆς σκηνῆς τοῦ μαρτυρίου· καὶ προσχεοῦσιν οἱ υἱοὶ Ἀαρὼν οἱ ἱερεῖς τὸ αἷμα ἐπὶ τὸ θυσιαστήριον τῶν ὁλοκαυτωμάτων κύκλῳ.
\vs{3}Καὶ προσάξουσιν ἀπὸ τῆς θυσίας τοῦ σωτηρίου κάρπωμα Κυρίῳ, τὸ στέαρ τὸ κατακαλύπτον τὴν κοιλίαν, καὶ πᾶν τὸ στέαρ τὸ ἐπὶ τῆς κοιλίας.
\vs{4}Καὶ τοὺς δύο νεφροὺς, καὶ τὸ στέαρ τὸ ἐπʼ αὐτῶν, τὸ ἐπὶ τῶν μηρίων, καὶ τὸν λοβὸν τὸν ἐπὶ τοῦ ἥπατος σὺν τοῖς νεφροῖς περιελεῖ.
\vs{5}Καὶ ἀνοίσουσιν αὐτὰ οἱ υἱοὶ Ἀαρὼν οἱ ἱερεῖς ἐπὶ τὸ θυσιαστήριον ἐπὶ τὰ ὁλοκαυτώματα ἐπὶ τὰ ξύλα, τὰ ἐπὶ τοῦ πυρὸς ἐπὶ τοῦ θυσιαστηρίου· κάρπωμα ὀσμὴ εὐωδίας Κυρίῳ.
\vs{6}Ἐὰν δὲ ἀπὸ τῶν προβάτων τὸ δῶρον αὐτοῦ θυσία σωτηρίου τῷ Κυρίῳ, ἄρσεν ἢ θῆλυ, ἄμωμον προσοίσει αὐτό.
\vs{7}Ἐὰν ἄρνα προσαγάγῃ τὸ δῶρον αὐτοῦ, προσάξει αὐτὸ ἔναντι Κυρίου.
\vs{8}Καὶ ἐπιθήσει τὰς χεῖρας ἐπὶ τὴν κεφαλὴν τοῦ δώρου αὐτοῦ, καὶ σφάξει αὐτὸ παρὰ τὰς θύρας τῆς σκηνῆς τοῦ μαρτυρίου· καὶ προσχεοῦσιν οἱ υἱοὶ Ἀαρὼν οἱ ἱερεῖς τὸ αἷμα ἐπὶ τὸ θυσιαστήριον κύκλῳ.
\vs{9}Καὶ προσοίσει ἀπὸ τῆς θυσίας τοῦ σωτηρίου κάρπωμα τῷ Κυρίῷ· τὸ στέαρ καὶ τὴν ὀσφὺν ἄμωμον σὺν ταῖς ψόαις περιελεῖ αὐτό· καὶ πᾶν τὸ στέαρ τὸ κατακαλύπτον τὴν κοιλίαν, καὶ πᾶν τὸ στέαρ τὸ ἐπὶ τῆς κοιλίας.
\vs{10}Καὶ ἀμφοτέρους τοὺς νεφροὺς, καὶ τὸ στέαρ τὸ ἐπʼ αὐτῶν, τὸ ἐπὶ τῶν μηρίων, καὶ τὸν λοβὸν τὸν ἐπὶ τοῦ ἥπατος σὺν τοῖς νεφροῖς περιελὼν,
\vs{11}ἀνοίσει ὁ ἱερεὺς ἐπὶ τὸ θυσιαστήριον· ὀσμὴ εὐωδίας κάρπωμα Κυρίῳ.

\vs{12}Ἐὰν δὲ ἀπὸ τῶν αἰγῶν τὸ δῶρον αὐτοῦ, καὶ προσάξει ἔναντι Κυρίου.
\vs{13}Καὶ ἐπιθήσει τὰς χεῖρας ἐπὶ τὴν κεφαλὴν αὐτοῦ, καὶ σφάξουσιν αὐτὸ ἔναντι Κυρίου παρὰ τὰς θύρας τῆς σκηνῆς τοῦ μαρτυρίου· καὶ προσχεοῦσιν οἱ υἱοὶ Ἀαρὼν οἱ ἱερεῖς τὸ αἷμα ἐπὶ τὸ θυσιαστήριον κύκλῳ.
\vs{14}Καὶ ἀνοίσει ἀπʼ αὐτοῦ κάρπωμα Κυρίῳ τὸ στέαρ τὸ κατακαλύπτον τὴν κοιλίαν, καὶ πᾶν τὸ στέαρ τὸ ἐπὶ τῆς κοιλίας.
\vs{15}Καὶ ἀμφοτέρους τοὺς νεφροὺς, καὶ πᾶν τὸ στέαρ τὸ ἐπʼ αὐτῶν, τὸ ἐπὶ τῶν μηρίων, καὶ τὸν λοβὸν τοῦ ἥπατος σὺν τοῖς νεφροῖς περιελεῖ,
\vs{16}καὶ ἀνοίσει ὁ ἱερεὺς ἐπὶ τὸ θυσιαστήριον· κάρπωμα ὀσμὴ εὐωδίας τῷ Κυρίῳ· πᾶν τὸ στέαρ τῷ Κυρίῳ.
\vs{17}Νόμιμον εἰς τὸν αἰῶνα εἰς τὰς γενεὰς ὑμῶν, ἐν πάσῃ κατοικίᾳ ὑμῶν· πᾶν στέαρ καὶ πᾶν αἷμα οὐκ ἔδεσθε.

\ch{4}
Καὶ ἐλάλησε Κύριος πρὸς Μωυσῆν, λέγων,
\vs{2}Λάλησον πρὸς τοὺς υἱοὺς Ἰσραὴλ, λέγων, ψυχὴ ἐὰν ἁμάρτῃ ἔναντι Κυρίου ἀκουσίως ἀπὸ πάντων τῶν προσταγμάτων Κυρίου, ὧν οὐ δεῖ ποιεῖν, καὶ ποιήσῃ ἕν τι ἀπʼ αὐτῶν·
\vs{3}Ἐὰν μὲν ὁ ἀρχιερεὺς ὁ κεχρισμένος ἁμάρτῃ τοῦ τὸν λαὸν ἁμαρτεῖν, καὶ προσάξει περὶ τῆς ἁμαρτίας αὐτοῦ, ἧς ἥμαρτε, μόσχον ἐκ βοῶν ἄμωμον τῷ Κυρίῳ περὶ τῆς ἁμαρτίας.
\vs{4}Καὶ προσάξει τὸν μόσχον παρὰ τὴν θύραν τῆς σκηνῆς τοῦ μαρτυρίου ἔναντι Κυρίου, καὶ ἐπιθήσει τὴν χεῖρα αὐτοῦ ἐπὶ τὴν κεφαλὴν τοῦ μόσχου ἔναντι Κυρίου, καὶ σφάξει τὸν μόσχον ἐνώπιον Κυρίου.
\vs{5}Καὶ λαβὼν ὁ ἱερεὺς ὁ χριστὸς ὁ τετελειωμένος τὰς χεῖρας ἀπὸ τοῦ αἵματος τοῦ μόσχου, καὶ εἰσοίσει αὐτὸ εἰς τὴν σκηνὴν τοῦ μαρτυρίου.
\vs{6}Καὶ βάψει ὁ ἱερεὺς τὸν δάκτυλον εἰς τὸ αἷμα, καὶ προσρανεῖ ἀπὸ τοῦ αἵματος ἑπτάκις ἔναντι Κυρίου, κατὰ τὸ καταπέτασμα τὸ ἅγιον.
\vs{7}Καὶ ἐπιθήσει ὁ ἱερεὺς ἀπὸ τοῦ αἵματος τοῦ μόσχου ἐπὶ τὰ κέρατα τοῦ θυσιαστηρίου τοῦ θυμιάματος τῆς συνθέσεως τοῦ ἐναντίον Κυρίου, ὅ ἐστιν ἐν τῇ σκηνῇ τοῦ μαρτυρίου· καὶ πᾶν τὸ αἷμα τοῦ μόσχου ἐκχεεῖ παρὰ τὴν βάσιν τοῦ θυσιαστηρίου τῶν ὁλοκαυτωμάτων, ὅ ἐστι παρὰ τὰς θύρας τῆς σκηνῆς τοῦ μαρτυρίου.
\vs{8}Καὶ πὰν τὸ στέαρ τοῦ μόσχου τοῦ τῆς ἁμαρτίας περιελεῖ ἀπʼ αὐτοῦ, τὸ στέαρ τὸ κατακαλύπτον τὰ ἐνδόσθια, καὶ πᾶν τὸ στέαρ τὸ ἐπὶ τῶν ἐνδοσθίων,
\vs{9}καὶ τοὺς δύο νεφροὺς, καὶ τὸ στέαρ τὸ ἐπʼ αὐτῶν, ὅ ἐστιν ἐπὶ τῶν μηρίων, καὶ τὸν λοβὸν τὸν ἐπὶ τοῦ ἥπατος σὺν τοῖς νεφροῖς περιελεῖ αὐτό,
\vs{10}ὃν τρόπον ἀφαιρεῖται αὐτὸ ἀπὸ τοῦ μόσχου τοῦ τῆς θυσίας τοῦ σωτηρίου, καὶ ἀνοίσει ὁ ἱερεὺς ἐπὶ τὸ θυσιαστήριον τῆς καρπώσεως.
\vs{11}Καὶ τὸ δέρμα τοῦ μόσχου, καὶ πᾶσαν αὐτοῦ τὴν σάρκα σὺν τῇ κεφαλῇ καὶ τοῖς ἀκρωτηρίοις καὶ τῇ κοιλίᾳ καὶ τῇ κόπρῳ·
\vs{12}καὶ ἐξοίσουσιν ὅλον τὸν μόσχον ἔξω τῆς παρεμβολῆς εἰς τόπον καθαρὸν, οὗ ἐκχεοῦσι τὴν σποδιὰν, καὶ κατακαύσουσιν αὐτὸν ἐπὶ ξύλων ἐν πυρί· ἐπὶ τῆς ἐκχύσεως τῆς σποδιᾶς καυθήσεται.

\vs{13}Ἐὰν δὲ πᾶσα συναγωγὴ Ἰσραὴλ ἀγνοήσῃ ἀκουσίως, καὶ λάθῃ ῥῆμα ἐξ ὀφθαλμῶν τῆς συναγωγῆς, καὶ ποιήσωσι μίαν ἀπὸ πασῶν τῶν ἐντολῶν Κυρίου, ἣ οὐ ποιηθήσεται, καὶ πλήμμελήσωσι,
\vs{14}καὶ γνωσθῇ αὐτοῖς ἡ ἁμαρτία, ἣν ἥμαρτον ἐν αὐτῇ, καὶ προσάξει ἡ συναγωγὴ μόσχον ἐκ βοῶν ἄμωμον περὶ τῆς ἁμαρτίας, καὶ προσάξει αὐτὸν παρὰ τὰς θύρας τῆς σκηνῆς τοῦ μαρτυρίου.
\vs{15}Καὶ ἐπιθήσουσιν οἱ πρεσβύτεροι τῆς συναγωγῆς τὰς χείρας αὐτῶν ἐπὶ τὴν κεφαλὴν τοῦ μόσχου ἔναντι Κυρίου, καὶ σφάξουσιν τὸν μόσχον ἔναντι Κυρίου.
\vs{16}Καὶ εἰσοίσει ὁ ἱερεὺς ὁ χριστὸς ἀπὸ τοῦ αἵματος τοὺ μόσχου εἰς τὴν σκηνὴν τοῦ μαρτυρίου.
\vs{17}Καὶ βάψει ὁ ἱερεὺς τὸν δάκτυλον ἀπὸ τοῦ αἵματος τοῦ μόσχου, καὶ ῥανεῖ ἑπτάκις ἔναντι Κυρίου, κατενώπιον τοῦ καταπετάσματος τοῦ ἁγίου.
\vs{18}Καὶ ἀπὸ τοῦ αἵματος ἐπιθήσει ὁ ἱερεὺς ἐπὶ τὰ κέρατα τοῦ θυσιαστηρίου τῶν θυμιαμάτων τῆς συνθέσεως, ὅ ἐστιν ἐνώπιον Κυρίου, ὅ ἐστιν ἐν τῇ σκηνῇ τοῦ μαρτυρίου· καὶ τὸ πᾶν αἷμα ἐκχεεῖ πρὸς τὴν βάσιν τοῦ θυσιαστηρίου τῶν καρπώσεων, τοῦ πρὸς τῇ θύρᾳ τῆς σκηνῆς τοῦ μαρτυρίου.
\vs{19}Καὶ τὸ πᾶν στέαρ περιελεῖ ἀπʼ αὐτοῦ, καὶ ἀνοίσει ἐπὶ τὸ θυσιαστήριον.
\vs{20}Καὶ ποιήσει τὸν μόσχον, ὃν τρόπον ἐποίησε τὸν μόσχον τὸν τῆς ἁμαρτίας, οὕτω ποιηθήσεται· καὶ ἐξιλάσεται περὶ αὐτῶν ὁ ἱερεὺς, καὶ ἀφεθήσεται αὐτοῖς ἡ ἁμαρτία.
\vs{21}Καὶ ἐξοίσουσι τὸν μόσχον ὅλον ἔξω τῆς παρεμβολῆς, καὶ κατακαύσουσι τὸν μόσχον, ὃν τρόπον κατέκαυσαν τὸν μόσχον τὸν πρότερον· ἁμαρτία συναγωγῆς ἐστιν.

\vs{22}Ἐὰν δὲ ὁ ἄρχων ἁμάρτῃ, καὶ ποιήσῃ μίαν ἀπὸ πασῶν τῶν ἐντολῶν Κυρίου τοῦ Θεοῦ αὐτοῦ, ἣ οὐ ποιηθήσεται, ἀκουσίως, καὶ ἁμάρτῃ καὶ πλημμελήσῃ,
\vs{23}καὶ γνωσθῇ αὐτῷ ἡ ἁμαρτία, ἣν ἥμαρτεν ἐν αὐτῇ, καὶ προσοίσει τὸ δῶρον αὐτοῦ χίμαρον ἐξ αἰγῶν, ἄρσεν ἄμωμον.
\vs{24}Καὶ ἐπιθήσει τὴν χεῖρα ἐπὶ τὴν κεφαλὴν τοῦ χιμάρου· καὶ σφάξουσιν αὐτὸν ἐν τόπῳ οὗ σφάζουσι τὰ ὁλοκαυτώματα ἐνώπιον Κυρίου· ἁμαρτία ἐστί.
\vs{25}Καὶ ἐπιθήσει ὁ ἱερεὺς ἀπὸ τοῦ αἵματος τοῦ τῆς ἁμαρτίας τῷ δακτύλῳ ἐπὶ τὰ κέρατα τοῦ θυσιαστηρίου τῶν ὁλοκαυτωμάτων· καὶ τὸ πᾶν αἷμα αὐτοῦ ἐκχεεῖ παρὰ τὴν βάσιν τοῦ θυσιαστηρίου τῶν ὁλοκαυτωμάτων.
\vs{26}Καὶ τὸ πᾶν στέαρ αὐτοῦ ἀνοίσει ἐπὶ τὸ θυσιαστήριον, ὥσπερ τὸ στέαρ θυσίας σωτηρίου· καὶ ἐξιλάσεται περὶ αὐτοῦ ὁ ἱερεὺς ἀπὸ τῆς ἁμαρτίας αὐτοῦ, καὶ ἀφεθήσεται αὐτῷ.

\vs{27}Ἐὰν δὲ ψυχὴ μία ἁμάρτῃ ἀκουσίως ἐκ τοῦ λαοῦ τῆς γῆς, ἐν τῷ ποιῆσαι μίαν ἀπὸ πασῶν τῶν ἐντολῶν Κυρίου, ἣ οὐ ποιηθήσεται, καὶ πλημμελήσῃ·
\vs{28}καὶ γνωσθῇ αὐτῷ ἡ ἁμαρτία, ἣν ἥμαρτεν ἐν αὐτῇ, καὶ οἴσει χίμαιραν ἐξ αἰγῶν, θήλειαν ἄμωμον οἴσει περὶ τῆς ἁμαρτίας, ἧς ἥμαρτε.
\vs{29}Καὶ ἐπιθήσει τὴν χεῖρα ἐπὶ τὴν κεφαλὴν τοῦ ἁμαρτήματος αὐτοῦ· καὶ σφάξουσιν τὴν χίμαιραν τὴν τῆς ἁμαρτίας ἐν τῷ τόπῳ, οὗ σφάζουσι τὰ ὁλοκαυτώματα.
\vs{30}Καὶ λήψεται ὁ ἱερεὺς ἀπὸ τοῦ αἵματος αὐτῆς τῷ δακτύλῳ, καὶ ἐπιθήσει ἐπὶ τὰ κέρατα τοῦ θυσιαστηρίου τῶν ὁλοκαυτωμάτων· καὶ πᾶν τὸ αἷμα αὐτῆς ἐκχεεῖ παρὰ τὴν βάσιν τοῦ θυσιαστηρίου.
\vs{31}Καὶ πᾶν τὸ στέαρ περιελεῖ, ὃν τρόπον περιαιρεῖται στέαρ ἀπὸ θυσίας σωτηρίου· καὶ ἀνοίσει ὁ ἱερεὺς ἐπὶ τὸ θυσιαστήριον εἰς ὀσμὴν εὐωδίας Κυρίῳ· καὶ ἐξιλάσεται περὶ αὐτοῦ ὁ ἱερεὺς, καὶ ἀφεθήσεται αὐτῷ.

\vs{32}Εὰν δὲ πρόβατον προσενέγκῃ τὸ δῶρον αὐτοῦ περὶ τῆς ἁμαρτίας, θῆλυ ἄμωμον προσοίσει αὐτό.
\vs{33}Καὶ ἐπιθήσει τὴν χεῖρα ἐπὶ τῆν κεφαλὴν τοῦ τῆς ἁμαρτίας· καὶ σφάξουσιν αὐτὸ ἐν τόπῳ, οὗ σφάζουσι τὰ ὁλοκαυτώματα.
\vs{34}Καὶ λαβὼν ὁ ἱερεὺς ἀπὸ τοῦ αἵματος τοῦ τῆς ἁμαρτίας τῷ δακτύλῳ, ἐπιθήσει ἐπὶ τὰ κέρατα τοῦ θυσιαστηρίου τῆς ὁλοκαρπώσεως· καὶ πᾶν αὐτοῦ τὸ αἷμα ἐκχεεῖ παρὰ τὴν βάσιν τοῦ θυσιαστηρίου τῆς ὁλοκαυτώσεως.
\vs{35}Καὶ πᾶν αὐτοῦ τὸ στέαρ περιελεῖ, ὃν τρόπον περιαιρεῖται στέαρ προβάτου ἐκ τῆς θυσίας τοῦ σωτηρίου· καὶ ἐπιθήσει αὐτὸ ὁ ἱερεὺς ἐπὶ τὸ θυσιαστήριον ἐπὶ τὸ ὁλοκαύτωμα Κυρίου· καὶ ἐξιλάσεται περὶ αὐτοῦ ὁ ἱερεὺς περὶ τῆς ἁμαρτίας ἧς ἥμαρτε, καὶ ἀφεθήσεται αὐτῷ.

\ch{5}
Ἐὰν δὲ ψυχὴ ἁμάρτῃ, καὶ ἀκούσῃ φωνὴν ὁρκισμοῦ, καὶ οὗτος μάρτυς ἢ ἑώρακεν ἢ σύνοιδεν, ἐὰν μὴ ἀπαγγείλῃ, λήψεται τὴν ἁμαρτίαν.
\vs{2}Ἡ ψυχὴ ἐκείνη ἥτις ἐὰν ἅψηται παντὸς πράγματος ἀκαθάρτου, ἢ θνησιμαίου, ἢ θηριαλώτου ἀκαθάρτου, ἢ τῶν θνησιμαίων βδελυγμάτων τῶν ἀκαθάρτων, ἢ τῶν θνησιμαίων κτηνῶν τῶν ἀκαθάρτων,
\vs{3}ἢ ἅψηται ἀπὸ ἀκαθαρσίας ἀνθρώπου, ἀπὸ πάσης ἀκαθαρσίας αὐτοῦ, ἧς ἂν ἁψάμενος μιανθῇ καὶ ἔλαθεν αὐτόν, μετὰ τοῦτο δὲ γνῷ, καὶ πλημμελήσῃ.
\vs{4}Ἡ ψυχὴ ἡ ἄνομος, ἡ διαστέλλουσα τοῖς χείλεσι κακοποιῆσαι ἢ καλῶς ποιῆσαι κατὰ πάντα ὅσα ἐὰν διαστείλῃ ὁ ἄνθρωπος μεθʼ ὅρκου, καὶ λάθῃ αὐτὸν πρὸ ὀφθαλμῶν, καὶ οὗτος γνῷ, καὶ ἁμάρτῃ ἕν τι τούτων.
\vs{5}Καὶ ἐξαγορεύσει τὴν ἁμαρτίαν περὶ ὧν ἡμάρτηκε κατʼ αὐτῆς.
\vs{6}Καὶ οἴσει περὶ ὧν ἐπλημμέλησε Κυρίῳ, περὶ τῆς ἁμαρτίας ἧς ἥμαρτε, θῆλυ ἀπὸ τῶν προβάτων ἀμνάδα, ἢ χίμαιραν ἐξ αἰγῶν, περὶ ἁμαρτίας· καὶ ἐξιλάσεται περὶ αὐτοῦ ὁ ἱερεὺς περὶ τῆς ἁμαρτίας αὐτοῦ, ἧς ἥμαρτε, καὶ ἀφεθήσεται αὐτῷ ἡ ἁμαρτία.
\vs{7}Ἐὰν δὲ μὴ ἰσχύῃ ἡ χεὶρ αὐτοῦ τὸ ἱκανὸν εἰς τὸ πρόβατον, οἴσει περὶ τῆς ἁμαρτίας αὐτοῦ, ἧς ἥμαρτε, δύο τρυγόνας, ἢ δύο νοσσοὺς περιστερῶν Κυρίῳ, ἕνα περὶ ἁμαρτίας, καὶ ἕνα εἰς ὁλοκαύτωμα.
\vs{8}Καὶ οἴσει αὐτὰ πρὸς τὸν ἱερέα· καὶ προσάξει ὁ ἱερεὺς τὸ περὶ τῆς ἁμαρτίας πρότερον· καὶ ἀποκνίσει ὁ ἱερεὺς τὴν κεφαλὴν αὐτοῦ ἀπὸ τοῦ σφονδύλου, καὶ οὐ διελεῖ.
\vs{9}Καὶ ῥανεῖ ἀπὸ τοῦ αἵματος τοῦ περὶ τῆς ἁμαρτίας ἐπὶ τὸν τοῖχον τοῦ θυσιαστηρίου· τὸ δὲ κατάλοιπον τοῦ αἵματος καταστραγγιεῖ ἐπὶ τὴν βάσιν τοῦ θυσιαστηρίου· ἁμαρτία γάρ ἐστι·
\vs{10}Καὶ τὸ δεύτερον ποιήσει ὁλοκάρπωμα, ὡς καθήκει· καὶ ἐξιλάσεται ὁ ἱερεὺς περὶ τῆς ἁμαρτίας αὐτοῦ, ἧς ἥμαρτε, καὶ ἀφεθήσεται αὐτῷ.

\vs{11}Ἐὰν δὲ μὴ εὑρίσκῃ ἡ χεὶρ αὐτοῦ ζεῦγος τρυγόνων, ἢ δύο νοσσούς περιστερῶν, καὶ οἴσει τὸ δῶρον αὐτοῦ, περὶ οὗ ἥμαρτε, τὸ δέκατον τοῦ οἰφὶ σεμιδάλεως περὶ ἁμαρτίας· οὐκ ἐπιχεεῖ ἐπʼ αὐτὸ ἔλαιον, οὐδὲ ἐπιθήσει ἐπʼ αὐτῷ λίβανον, ὅτι περὶ ἁμαρτίας ἐστί.
\vs{12}Καὶ οἴσει αὐτὸ πρὸς τὸν ἱερέα· καὶ δραξάμενος ὁ ἱερεὺς ἀπʼ αὐτῆς πλήρη τὴν δράκα, τὸ μνημόσυνον αὐτῆς ἐπιθήσει ἐπὶ τὸ θυσιαστήριον τῶν ὁλοκαυτωμάτων Κυρίῳ· ἁμαρτία ἐστί.
\vs{13}Καὶ ἐξιλάσεται περὶ αὐτοῦ ὁ ἱερεὺς περὶ τῆς ἁμαρτίας αὐτοῦ, ἧς ἥμαρτεν ἀφʼ ἑνὸς τούτων, καὶ ἀφεθήσεται αὐτῷ. τὸ δὲ καταλειφθὲν ἔσται τῷ ἱερεῖ, ὡς θυσία τῆς σεμιδάλεως.

\vs{14}Καὶ ἐλάλησε Κύριος πρὸς Μωυσῆν, λέγων,
\vs{15}ψυχὴ ἣ ἂν λάθῃ αὐτὸν λήθῃ, καὶ ἁμάρτῃ ἀκουσίως ἀπὸ τῶν ἁγίων Κυρίου, καὶ οἴσει τῆς πλημμελείας αὐτοῦ τῷ Κυρίῳ κριὸν ἄμωμον ἐκ τῶν προβάτων, τιμῆς ἀργυρίου σίκλων, τῷ σίκλῳ τῶν ἁγίων, περὶ οὗ ἐπλημμέλησε.
\vs{16}Καὶ ὃ ἥμαρτεν ἀπὸ τῶν ἁγίων ἀποτίσει αὐτὸ, καὶ τὸ ἐπίπεμπτον προσθήσει ἐπʼ αὐτὸ, καὶ δώσει αὐτὸ τῷ ἱερεῖ· καὶ ὁ ἱερεὺς ἐξιλάσεται περὶ αὐτοῦ ἐν τῷ κριῷ τῆς πλημμελείας, καὶ ἀφεθήσεται αὐτῷ.
\vs{17}Καὶ ἡ ψυχὴ ἣ ἂν ἁμάρτῃ, καὶ ποιήσει μίαν ἀπὸ πασῶν τῶν ἐντολῶν Κυρίου, ὧν οὐ δεῖ ποιεῖν, καὶ οὐκ ἔγνω, καὶ πλημμελήσῃ, καὶ λάβῃ τὴν ἁμαρτίαν,
\vs{18}καὶ οἴσει κριὸν ἄμωμον ἐκ τῶν προβάτων, τιμῆς ἀργυρίου εἰς πλημμέλειαν πρὸς τὸν ἱερέα· καὶ ἐξιλάσεται περὶ αὐτοῦ ὁ ἱερεὺς περὶ τῆς ἀγνοίας αὐτοῦ, ἧς ἠγνόησε, καὶ αὐτὸς οὐκ ᾔδει, καὶ ἀφεθήσεται αὐτῷ.
\vs{19}Ἐπλημμέλησε γὰρ πλημμελείᾳ ἔναντι Κυρίου.

\vs{20}Καὶ ἐλάλησε Κύριος πρὸς Μωυσῆν, λέγων,
\vs{21}ψυχὴ ἣ ἂν ἁμάρτῃ, καὶ παριδὼν παρίδῃ τὰς ἐντολὰς Κυρίου, καὶ ψεύσηται τὰ πρὸς τὸν πλησίον ἐν παραθήκῃ, ἢ περὶ κοινωνίας, ἢ περὶ ἁρπαγῆς, ἢ ἠδίκησέ τι τὸν πλησίον,
\vs{22}ἢ εὗρεν ἀπωλίαν, καὶ ψεύσηται περὶ αὐτῆς, καὶ ὀμόσῃ ἀδίκως περὶ ἑνὸς ἀπὸ πάντων, ὧν ἐὰν ποιήσῃ ὁ ἄνθρωπος, ὥστε ἁμαρτεῖν ἐν τούτοις·
\vs{23}Καὶ ἔσται ἡνίκα ἐὰν ἁμάρτῃ, καὶ πλημμελήσῃ, καὶ ἀποδῷ τὸ ἅρπαγμα, ὃ ἥρπασεν, ἢ τὸ ἀδίκημα, ὃ ἠδίκησεν, ἢ τὴν παραθήκην, ἥτις παρετέθη αὐτῷ, ἢ τὴν ἀπώλειαν, ἣν εὗρεν
\vs{24}ἀπὸ παντὸς πράγματος, οὗ ὤμοσε περὶ αὐτοῦ ἀδίκως, καὶ ἀποτίσει αὐτὸ τὸ κεφάλαιον, καὶ τὸ ἐπίπεμπτον προσθήσει ἐπʼ αὐτὸ, τίνος ἐστίν, αὐτῷ ἀποδώσει ᾗ ἡμέρᾳ ἐλεγχθῇ.
\vs{25}Καὶ τῆς πλημμελείας αὐτου οἴσει τῷ Κυρίῳ κριὸν ἀπὸ τῶν προβάτων ἄμωμον, τιμῆς, εἰς ὃ ἐπλημμέλησε.
\vs{26}Καὶ ἐξιλάσεται περὶ αὐτοῦ ὁ ἱερεὺς ἔναντι Κυρίου, καὶ ἀφεθήσεται αὐτῷ περὶ ἑνὸς ἀπὸ πάντων ὧν ἐποίησε καὶ ἐπλημμέλησεν αὐτῷ.

\ch{6}
Καὶ ἐλάλησε Κύριος πρὸς Μωυσῆν, λέγων,
\vs{2}ἔντειλαι τῷ Ἀαρὼν καὶ τοῖς υἱοῖς αὐτοῦ, λέγων, οὗτος ὁ νόμος τῆς ὁλοκαυτώσεως· αὕτη ἡ ὁλοκαύτωσις ἐπὶ τῆς καύσεως αὐτῆς ἐπὶ τοῦ θυσιαστηρίου ὅλην τὴν νύκτα ἕως τοπρωῒ, καὶ τὸ πῦρ τοῦ θυσιαστηρίου καυθήσεται ἐπʼ αὐτοῦ, οὐ σβεσθήσεται.
\vs{3}Καὶ ἐνδύσεται ὁ ἱερεὺς χιτῶνα λινοῦν, καὶ περισκελὲς λινοῦν ἐνδύσεται περὶ τὸ σῶμα αὐτοῦ, καὶ ἀφελεῖ τὴν κατακάρπωσιν, ἣν ἂν καταναλώσῃ τὸ πῦρ, τὴν ὁλοκαύτωσιν ἀπὸ τοῦ θυσιαστηρίου· καὶ παραθήσει αὐτὸ ἐχόμενον τοῦ θυσιαστηρίου.
\vs{4}Καὶ ἐκδύσεται τὴν στολὴν αὐτοῦ, καὶ ἐνδύσεται στολὴν ἄλλην· καὶ ἐξοίσει τὴν κατακάρπωσιν ἔξω τῆς παρεμβολῆς εἰς τόπον καθαρόν.
\vs{5}Καὶ πῦρ ἐπὶ τὸ θυσιαστήριον καυθήσεται ἀπʼ αὐτοῦ, καὶ οὐ σβεσθήσεται· καὶ καύσει ἐπʼ αὐτοῦ ὁ ἱερεὺς ξύλα τοπρωῒ πρωῒ, καὶ στοιβάσει ἐπʼ αὐτοῦ τὴν ὁλοκαύτωσιν, καὶ ἐπιθήσει ἐπʼ αὐτὸ τὸ στέαρ τοῦ σωτηρίου.
\vs{6}Καὶ πῦρ διαπαντὸς καυθήσεται ἐπὶ τὸ θυσιαστήριον, οὐ σβεσθήσεται.
\vs{7}Οὗτος ὁ νόμος τῆς θυσίας, ἣν προσάξουσιν αὐτὴν οἱ υἱοὶ Ἀαρὼν ἔναντι Κυρίου, ἀπέναντι τοῦ θυσιαστηρίου.
\vs{8}Καὶ ἀφελεῖ ἀπʼ αὐτοῦ τῇ δρακὶ ἀπὸ τῆς σεμιδάλεως τῆς θυσίας σὺν τῷ ἐλαίῳ αὐτῆς, καὶ σὺν παντὶ τῷ λιβάνῳ αὐτῆς, τὰ ὄντα ἐπὶ τῆς θυσίας· καὶ ἀνοίσει ἐπὶ τὸ θυσιαστήριον κάρπωμα ὀσμὴν εὐωδίας, τὸ μνημόσυνον αὐτῆς τῷ Κυρίῳ.
\vs{9}Τὸ δὲ καταλειφθὲν ἀπʼ αὐτῆς ἔδεται Ἀαρὼν καὶ οἱ υἱοὶ αὐτοῦ· ἄζυμα βρωθήσεται ἐν τόπῳ ἁγίῳ· ἐν αὐλῇ τῆς σκηνῆς τοῦ μαρτυρίου ἔδονται αὐτήν.
\vs{10}Οὐ πεφθήσεται ἐζυμωμένη· μερίδα αὐτὴν ἔδωκα αὐτοῖς ἀπὸ τῶν καρπωμάτων Κυρίου· ἅγια ἁγίων ἐστὶν, ὥσπερ τὸ τῆς ἁμαρτίας, καὶ ὥσπερ τὸ τῆς πλημμελείας.
\vs{11}Πᾶν ἀρσενικὸν τῶν ἱερέων ἔδονται αὐτήν· νόμιμον αἰώνιον εἰς τὰς γενεὰς ὑμῶν ἀπὸ τῶν καρπωμάτων Κυρίου· πᾶς ὃς ἐὰν ἅψηται αὐτῶν, ἁγιασθήσεται.

\vs{12}Καὶ ἐλάλησε Κύριος πρὸς Μωυσῆν, λέγων,
\vs{13}τοῦτο τὸ δῶρον Ἀαρὼν καὶ τῶν υἱῶν αὐτοῦ, ὃ προσοίσουσι Κυρίῳ ἐν τῇ ἡμέρᾳ, ᾗ ἂν χρίσῃς αὐτόν· τὸ δέκατον τοῦ οἰφὶ σεμιδάλεως εἰς θυσίαν διαπαντὸς, τὸ ἥμισυ αὐτῆς τὸπρωῒ, καὶ τὸ ἥμισυ αὐτῆς τοδειλινόν.
\vs{14}Ἐπὶ τηγάνου ἐν ἐλαίῳ ποιηθήσεται, πεφυραμένην οἴσει αὐτήν ἑλικτά, θυσίαν ἐκ κλασμάτων, θυσίαν εἰς ὀσμὴν εὐωδίας Κυρίῳ.
\vs{15}Ὁ ἱερεὺς ὁ χριστὸς ὁ ἀντʼ αὐτοῦ ἐκ τῶν υἱῶν αὐτοῦ ποιήσει αὐτήν· νόμος αἰώνιος· ἅπαν ἐπιτελεσθήσεται.
\vs{16}Καὶ πᾶσα θυσία ἱερέως ὁλόκαυτος ἔσται, καὶ οὐ βρωθήσεται.
\vs{17}Καὶ ἐλάλησε Κύριος πρὸς Μωυσῆν, λέγων,
\vs{18}λάλησον τῷ Ἀαρὼν καὶ τοῖς υἱοῖς αὐτοῦ, λέγων, οὗτος ὁ νόμος τῆς ἁμαρτίας· ἐν τόπῳ οὗ σφάζουσι τὸ ὁλοκαύτωμα, σφάξουσι τὰ περὶ τῆς ἁμαρτίας ἔναντι Κυρίου· ἅγια ἁγίων ἐστίν.
\vs{19}Ὁ ἱερεὺς ὁ ἀναφέρων αὐτὴν, ἔδεται αὐτήν· ἐν τόπῳ ἁγίῳ βρωθήσεται, ἐν αὐλῇ τῆς σκηνῆς τοῦ μαρτυρίου.
\vs{20}Πᾶς ὁ ἁπτόμενος τῶν κρεῶν αὐτῆς, ἁγιασθήσεται· καὶ ᾧ ἐὰν ἐπιῤῥαντισθῇ ἀπὸ τοῦ αἵματος αὐτῆς ἐπὶ τὸ ἱμάτιον, ὃς ἐὰν ῥαντισθῇ ἐπʼ αὐτὸ, πλυθήσεται ἐν τόπῳ ἁγίῳ.
\vs{21}Καὶ σκεῦος ὀστράκινον, οὗ ἐὰν ἑψεθῇ ἐν αὐτῷ, συντριβήσεται· ἐὰν δὲ ἐν σκεύει χαλκῷ ἑψηθῇ, ἐκτρίψει αὐτὸ, καὶ ἐκκλύσει ὕδατι.
\vs{22}Πᾶς ἄρσην ἐν τοῖς ἱερεῦσιν φάγεται αὐτά ἅγια ἁγίων ἐστὶ Κυρίῳ.
\vs{23}Καὶ πάντα τὰ περὶ τῆς ἁμαρτίας, ὧν ἐὰν εἰσενεχθῇ ἀπὸ τοῦ αἵματος αὐτῶν εἰς τὴν σκηνὴν τοῦ μαρτυρίου ἐξιλάσασθαι ἐν τῷ ἁγίῳ, οὐ βρωθήσεται· ἐν πυρὶ κατακαυθήσεται.

\ch{7}
Καὶ οὗτος ὁ νόμος τοῦ κριοῦ τοῦ περὶ τῆς πλημμελείας· ἅγια ἁγίων ἐστίν.
\vs{2}Ἐν τόπῳ οὗ σφάζουσι τὸ ὁλοκαύτωμα, σφάξουσι τὸν κριὸν τῆς πλημμελείας ἔναντι Κυρίου· καὶ τὸ αἷμα προσχεεῖ ἐπὶ τὴν βάσιν τοῦ θυσιαστηρίου κύκλῳ·
\vs{3}Καὶ πᾶν τὸ στέαρ αὐτοῦ προσοίσει ἀπʼ αὐτοῦ, καὶ τὴν ὀσφὺν, καὶ πᾶν τὸ στέαρ τὸ κατακαλύπτον τὰ ἐνδόσθια, καὶ πᾶν τὸ στέαρ τὸ ἐπὶ τῶν ἐνδοσθίων,
\vs{4}καὶ τοὺς δύο νεφροὺς, καὶ τὸ στέαρ τὸ ἐπʼ αὐτῶν, τὸ ἐπὶ τῶν μηρίων, καὶ τὸν λοβὸν τὸν ἐπὶ τοῦ ἥπατος σὺν τοῖς νεφροῖς, περιελεῖ αὐτά.
\vs{5}Καὶ ἀνοίσει αὐτὰ ὁ ἱερεὺς ἐπὶ τὸ θυσιαστήριον κάρπωμα τῷ Κυρίῳ· περὶ πλημμελείας ἐστί.
\vs{6}Πᾶς ἄρσην ἐκ τῶν ἱερέων ἔδεται αὐτά· ἐν τόπῳ ἁγίῳ ἔδονται αὐτά ἅγια ἁγίων ἐστίν.
\vs{7}Ὥσπερ τὸ περὶ τῆς ἁμαρτίας, οὕτω καὶ τὸ τῆς πλημμελείας· νόμος εἷς αὐτῶν· ὁ ἱερεὺς ὅστις ἐξιλάσεται ἐν αὐτῷ, αὐτῷ ἔσται.
\vs{8}Καὶ ὁ ἱερεὺς ὁ προσάγων ὁλοκαύτωμα ἀνθρώπου, τὸ δέρμα τῆς ὁλοκαυτώσεως, ἧς προσφέρει αὐτὸς, αὐτῷ ἔσται.
\vs{9}Καὶ πᾶσα θυσία ἥτις ποιηθήσεται ἐν τῷ κλιβάνῳ, καὶ πᾶσα ἥτις ποιηθήσεται ἐπʼ ἐσχάρας, ἢ ἐπὶ τηγάνου, τοῦ ἱερέως τοῦ προσφέροντος αὐτὴν, αὐτῷ ἔσται.
\vs{10}Καὶ πᾶσα θυσία ἀναπεποιημένη ἐν ἐλαίῳ, καὶ μὴ ἀναπεποιημένη, πᾶσι τοῖς υἱοῖς Ἀαρὼν ἔσται, ἑκάστῳ τὸ ἶσον.

\vs{11}Οὗτος ὁ νόμος θυσίας σωτηρίου, ἣν προσοίσουσι Κυρίῳ.
\vs{12}Ἐὰν μὲν περὶ αἰνέσεως προσφέρῃ αὐτήν, καὶ προσοίσει ἐπὶ τῆς θυσίας τῆς αἰνέσεως ἄρτους ἐκ σεμιδάλεως ἀναπεποιημένους ἐν ἐλαίῳ, καὶ λάγανα ἄζυμα διακεχρισμένα ἐν ἐλαίῳ, καὶ σεμίδαλιν πεφυραμένην ἐν ἐλαίῳ.
\vs{13}Ἐπʼ ἄρτοις ζυμίταις προσοίσει τὰ δῶρα αὐτοῦ ἐπὶ θυσίᾳ αἰνέσεως σωτηρίου.
\vs{14}Καὶ προσάξει ἓν ἀπὸ πάντων τῶν δώρων αὐτοῦ, ἀφαίρεμα Κυρίῳ· τῷ ἱερεῖ τῷ προσχέοντι τὸ αἷμα τοῦ σωτηρίου, αὐτῷ ἔσται.
\vs{15}Καὶ τὰ κρέα θυαίας αἰνέσεως σωτηρίου αὐτῷ ἔσται· καὶ ἐν ᾗ ἡμέρᾳ δωρεῖται, βρωθήσεται· οὐ καταλείψουσιν ἀπʼ αὐτοῦ εἰς τὸ πρωΐ.
\vs{16}Καὶ ἐὰν εὐχὴ ᾖ, ἢ ἑκούσιον θυσιάζῃ τὸ δῶρον αὐτοῦ, ᾗ ἂν ἡμέρᾳ προσαγάγῃ τὴν θυσίαν αὐτοῦ, βρωθήσεται, καὶ τῇ αὔριον.
\vs{17}Καὶ τὸ καταλειφθὲν ἀπὸ τῶν κρεῶν τῆς θυσίας ἕως ἡμέρας τρίτης, ἐν πυρὶ κατακαυθήσεται.
\vs{18}Ἐὰν δὲ φαγὼν φάγῃ ἀπὸ τῶν κρεῶν τῇ ἡμέρᾳ τῇ τρίτῃ, οὐ δεχθήσεται αὐτῷ τῷ προσφέροντι αὐτό· οὐ λογισθήσεται αὐτῷ, μίασμά ἐστιν· ἡ δὲ ψυχὴ ἥτις ἐὰν φάγῃ ἀπʼ αὐτοῦ, τὴν ἁμαρτίαν λήψεται.
\vs{19}Καὶ κρέα ὅσα ἐὰν ἅψηται παντὸς ἀκαθάρτου, οὐ βρωθήσεται, ἐν πυρὶ κατακαυθήσεται· πᾶς καθαρὸς φάγεται κρέα.
\vs{20}Ἡ δὲ ψυχὴ ἥτις ἐὰν φάγῃ ἀπὸ τῶν κρεῶν τῆς θυσίας τοῦ σωτηρίου, ὅ ἐστι Κυρίου, καὶ ἡ ἀκαθαρσία αὐτοῦ ἐπʼ αὐτῷ, ἀπολεῖται ἡ ψυχὴ ἐκείνη ἐκ τοῦ λαοῦ αὐτῆς.
\vs{21}Καὶ ἡ ψυχὴ ἣ ἂν ἅψηται παντὸς πράγματος ἀκαθάρτου, ἢ ἀπὸ ἀκαθαρσίας ἀνθρώπου, ἢ τῶν τετραπόδων τῶν ἀκαθάρτω ἢ παντὸς βδελύγματος ἀκαθάρτου, καὶ φάγῃ ἀπὸ τῶν κρεῶν τῆς θυσίας τοῦ σωτηρίου, ὅ ἐστι Κυρίου, ἀπολεῖται ἡ ψυχὴ ἐκείνη ἐκ τοῦ λαοῦ αὐτῆς.

\vs{22}Καὶ ἐλάλησε Κύριος πρὸς Μωυσῆν, λέγων,
\vs{23}λάλησον τοῖς υἱοῖς Ἰσραὴλ, λέγων, πᾶν στέαρ βοῶν, καὶ προβάτων, καὶ αἰγῶν οὐκ ἔδεσθε.
\vs{24}Καὶ στέαρ θνησιμαίων καὶ θηριαλώτων ποιηθήσεται εἰς πᾶν ἔργον, καὶ εἰς βρῶσιν οὐ βρωθήσεται.
\vs{25}Πᾶς ὁ ἔσθων στέαρ ἀπὸ τῶν κτηνῶν, ὧν προσάξει ἀπʼ αὐτῶν κάρπωμα Κυρίῳ, ἀπολεῖται ἡ ψυχὴ ἐκείνη ἀπὸ τοῦ λαοῦ αὐτῆς.
\vs{26}Πᾶν αἷμα οὐκ ἔδεσθε ἐν πάσῃ τῇ κατοικίᾳ ὑμῶν, ἀπό τε τῶν κτηνῶν καὶ ἀπὸ τῶν πετεινῶν.
\vs{27}Πᾶσα ψυχὴ ἣ ἂν φάγῃ αἷμα, ἀπολεῖται ἡ ψυχὴ ἐκείνη ἀπὸ τοῦ λαοῦ αὐτῆς.

\vs{28}Καὶ ἐλάλησε Κύριος πρὸς Μωυσῆν, λέγων,
\vs{29}καὶ τοῖς υἱοῖς Ἰσραὴλ λαλήσεις, λέγων, ὁ προσφέρων θυσίαν σωτηρίου, οἴσει τὸ δῶρον αὐτοῦ Κυρίῳ καὶ ἀπὸ τῆς θυσίας τοῦ σωτηρίου.
\vs{30}Αἱ χεῖρες αὐτοῦ προσοίσουσι τὰ καρπώματα Κυρίῳ· τὸ στέαρ τὸ ἐπὶ τοῦ στηθυνίου, καὶ τὸν λοβὸν τοῦ ἥπατος προσοίσει αὐτὰ, ὥστε ἐπιτιθέναι δόμα ἔναντι Κυρίου.
\vs{31}Καὶ ἀνοίσει ὁ ἱερεὺς τὸ στέαρ ἐπὶ τοῦ θυσιαστηρίου· καὶ ἔσται τὸ στηθύνιον Ἀαρὼν καὶ τοῖς υἱοῖς αὐτοῦ.
\vs{32}Καὶ τὸν βραχίονα τὸν δεξιὸν δώσετε ἀφαίρεμα τῷ ἱερεῖ ἀπὸ τῶν θυσιῶν τοῦ σωτηρίου ὑμῶν.
\vs{33}Ὁ προσφέρων τὸ αἷμα τοῦ σωτηρίου, καὶ τὸ στέαρ τὸ ἀπὸ τῶν υἱῶν Ἀαρὼν, αὐτῷ ἔσται ὁ βραχίων ὁ δεξιὸς ἐν μερίδι.
\vs{34}Τὸ γὰρ στηθύνιον τοῦ ἐπιθέματος καὶ τὸν βραχίονα τοῦ ἀφαιρέματος εἴληφα παρὰ τῶν υἱῶν Ἰσραὴλ ἀπὸ τῶν θυσιῶν τοῦ σωτηρίου ὑμῶν, καὶ ἔδωκα αὐτὰ Ἀαρὼν τῷ ἱερεῖ καὶ τοῖς υἱοῖς αὐτοῦ, νόμιμον αἰώνιον παρὰ τῶν υἱῶν Ἰσραήλ.
\vs{35}Αὕτη ἡ χρίσις Ἀαρὼν, καὶ ἡ χρίσις τῶν υἱῶν αὐτοῦ ἀπὸ τῶν καρπωμάτων Κυρίου, ἐν ᾗ ἡμέρᾳ προσηγάγετο αὐτοὺς τοῦ ἱερατεύειν τῷ Κυρίῳ,
\vs{36}καθὰ ἐνετείλατο Κύριος δοῦναι αὐτοῖς ᾗ ἡμέρᾳ ἔχρισεν αὐτοὺς παρὰ τῶν υἱῶν Ἰσραὴλ, νόμιμον αἰώνιον εἰς τὰς γενεὰς αὐτῶν.
\vs{37}Οὗτος ὁ νόμος τῶν ὁλοκαυτωμάτων, καὶ θυσίας, καὶ περὶ ἁμαρτίας, καὶ τῆς πλημμελείας καὶ τῆς τελειώσεως, καὶ τῆς θυσίας τοῦ σωτηρίου,
\vs{38}ὃν τρόπον ἐνετείλατο Κύριος τῷ Μωυσῇ ἐν τῷ ὄρει Σινᾷ, ᾗ ἡμέρᾳ ἐνετείλατο τοῖς υἱοῖς Ἰσραὴλ προσφέρειν τὰ δῶρα αὐτῶν ἔναντι Κυρίου ἐν τῇ ἐρήμῳ Σινᾷ.

\ch{8}
Καὶ ἐλάλησε Κύριος πρὸς Μωυσῆν, λέγων,
\vs{2}λάβε Ἀαρὼν καὶ τοὺς υἱοὺς αὐτοῦ, καὶ τὰς στολὰς αὐτοῦ, καὶ τὸ ἔλαιον τῆς χρίσεως, καὶ τὸν μόσχον τὸν περὶ τῆς ἁμαρτίας, καὶ τοὺς δύο κριοὺς, καὶ τὸ κανοῦν τῶν ἀζύμων,
\vs{3}καὶ πᾶσαν τὴν συναγωγὴν ἐκκλησίασον ἐπὶ τὴν θύραν τῆς σκηνῆς τοῦ μαρτυρίου.
\vs{4}Καὶ ἐποίησε Μωυσῆς ὃν τρόπον συνέταξεν αὐτῷ Κύριος· καὶ ἐξεκκλησίασε τὴν συναγωγὴν ἐπὶ τὴν θύραν τῆς σκηνῆς τοῦ μαρτυρίου.
\vs{5}Καὶ εἶπε Μωυσῆς τῇ συναγωγῇ, τοῦτό ἐστι τὸ ῥῆμα, ὃ ἐνετείλατο Κύριος ποιῆσαι.
\vs{6}Καὶ προσήνεγκε Μωυσῆς τὸν Ἀαρὼν, καὶ τοὺς υἱοὺς αὐτοῦ, καὶ ἔλουσεν αὐτοὺς ὕδατι.
\vs{7}Καὶ ἐνέδυσεν αὐτὸν τὸν χιτῶνα, καὶ ἔζωσεν αὐτὸν τὴν ζώνην, καὶ ἐνέδυσεν αὐτὸν τὸν ὑποδύτην, καὶ ἐπέθηκεν ἐπʼ αὐτὸν τὴν ἐπωμίδα.
\vs{8}Καὶ συνέζωσεν αὐτὸν κατὰ τὴν ποίησιν τῆς ἐπωμίδος, καὶ συνέσφιγξεν αὐτὸν ἐν αὐτῇ· καὶ ἐπέθηκεν ἐπʼ αὐτὴν τὸ λογεῖον, καὶ ἐπέθηκεν ἐπὶ τὸ λογεῖον τὴν δήλωσιν καὶ τὴν ἀλήθειαν.
\vs{9}Καὶ ἐπέθηκε τὴν μίτραν ἐπὶ τὴν κεφαλὴν αὐτοῦ, καὶ ἐπέθηκεν ἐπὶ τὴν μίτραν κατὰ πρόσωπον αὐτοῦ τὸ πέταλον τὸ χρυσοῦν τὸ καθηγιασμένον ἅγιον, ὃν τρόπον συνέταξε Κύριος τῷ Μωυσῇ.

\vs{10}Καὶ ἔλαβε Μωυσῆς ἀπὸ τοῦ ἐλαίου τῆς χρίσεως,
\vs{11}καὶ ἔῤῥανεν ἀπʼ αὐτοῦ ἐπὶ τὸ θυσιαστήριον ἑπτάκις· καὶ ἔχρισε τὸ θυσιαστήριον, καὶ ἡγίασεν αὐτὸ, καὶ πάντα τὰ ἐν αὐτῷ, καὶ τὸν λουτῆρα, καὶ τὴν βάσιν αὐτοῦ, καὶ ἡγίασεν αὐτά· καὶ ἔχρισε τὴν σκηνὴν, καὶ πάντα τὰ σκεύη αὐτῆς, καὶ ἡγίασεν αὐτήν.
\vs{12}Καὶ ἐπέχεε Μωυσῆς ἀπὸ τοῦ ἐλαίου τῆς χρίσεως ἐπὶ τὴν κεφαλὴν Ἀαρών· καὶ ἔχρισεν αὐτὸν, καὶ ἡγίασεν αὐτόν.
\vs{13}Καὶ προσήγαγε Μωυσῆς τοὺς υἱοὺς Ἀαρὼν, καὶ ἐνέδυσεν αὐτοὺς χιτῶνας, καὶ ἔζωσεν αὐτοὺς ζώνας, καὶ περιέθηκεν αὐτοῖς κιδάρεις, καθάπερ συνέταξε Κύριος τῷ Μωυσῇ.

\vs{14}Καὶ προσήγαγε Μωυσῆς τὸν μόσχον τὸν περὶ τῆς ἁμαρτίας· καὶ ἐπέθηκεν Ἀαρὼν καὶ οἱ υἱοὶ αὐτοῦ τὰς χεῖρας ἐπὶ τὴν κεφαλὴν τοῦ μόσχου τοῦ τῆς ἁμαρτίας.
\vs{15}Καὶ ἔσφαξεν αὐτόν· καὶ ἔλαβε Μωυσῆς ἀπὸ τοῦ αἵματος, καὶ ἐπέθηκεν ἐπὶ τὰ κέρατα τοῦ θυσιαστηρίου κύκλῳ τῷ δακτύλῳ, καὶ ἐκαθάρισε τὸ θυσιαστήριον· καὶ τὸ αἷμα ἐξέχεεν ἐπὶ τὴν βάσιν τοῦ θυσιαστηρίου, καὶ ἡγίασεν αὐτὸ, τοῦ ἐξιλάσασθαι ἐπʼ αὐτοῦ.
\vs{16}Καὶ ἔλαβε Μωυσῆς πᾶν τὸ στέαρ τὸ ἐπὶ τῶν ἐνδοσθίων, καὶ τὸν λοβὸν τὸν ἐπὶ τοῦ ἥπατος, καὶ ἀμφοτέρους τοὺς νεφροὺς, καὶ τὸ στέαρ τὸ ἐπʼ αὐτῶν, καὶ ἀνήνεγκε Μωυσῆς ἐπὶ τὸ θυσιαστήριον.
\vs{17}Καὶ τὸν μόσχον, καὶ τὴν βύρσαν αὐτοῦ, καὶ τὰ κρέα αὐτοῦ, καὶ τὴν κόπρον αὐτοῦ, κατέκαυσεν αὐτὰ πυρὶ ἔξω τῆς παρεμβολῆς, ὃν τρόπον συνέταξε Κύριος τῷ Μωυσῇ.
\vs{18}Καὶ προσήγαγε Μωυσῆς τὸν κριὸν τὸν εἰς ὁλοκαύτωμα· καὶ ἐπέθηκεν Ἀαρὼν καὶ υἱοὶ αὐτοῦ τὰς χεῖρας αὐτῶν ἐπὶ τὴν κεφαλὴν τοῦ κριοῦ. Καὶ ἔσφαξε Μωυσῆς τὸν κριόν· καὶ προσέχεε Μωυσῆς τὸ αἷμα ἐπὶ τὸ θυσιαστήριον κύκλῳ.
\vs{19}Καὶ τὸν κριὸν ἐκρεανόμησε κατὰ μέλη· καὶ ἀνήνεγκε Μωυσῆς τὴν κεφαλὴν, καὶ τὰ μέλη, καὶ τὸ στέαρ· καὶ τὴν κοιλίαν, καὶ τοὺς πόδας ἔπλυνεν ὕδατι.
\vs{20}Καὶ ἀνήνεγκε Μωυσῆς ὅλον τὸν κριὸν ἐπὶ τὸ θυσιαστήριον· ὁλοκαύτωμά ἐστιν εἰς ὀσμὴν εὐωδίας· κάρπωμά ἐστι τῷ Κυρίῳ, καθάπερ ἐνετείλατο Κύριος τῷ Μωυσῇ.

\vs{21}Καὶ προσήγαγε Μωυσῆς τὸν κριὸν τὸν δεύτερον, κριὸν τελειώσεως· καὶ ἐπέθηκεν Ἀαρὼν καὶ οἱ υἱοὶ αὐτοῦ τὰς χεῖρας αὐτῶν ἐπὶ τὴν κεφαλὴν τοῦ κριοῦ.
\vs{22}Καὶ ἔσφαξεν αὐτόν· καὶ ἔλαβε Μωυσῆς ἀπὸ τοῦ αἵματος αὐτοῦ, καὶ ἐπέθηκεν ἐπὶ τὸν λοβὸν τοῦ ὠτὸς Ἀαρὼν τοῦ δεξιοῦ, καὶ ἐπὶ τὸ ἄκρον τῆς χειρὸς τῆς δεξιᾶς, καὶ ἐπὶ τὸ ἄκρον τοῦ ποδὸς τοῦ δεξιοῦ.
\vs{23}Καὶ προσήγαγε Μωυσῆς τοὺς υἱοὺς Ἀαρών· καὶ ἐπέθηκε Μωυσῆς ἀπὸ τοῦ αἵματος ἐπὶ τοὺς λοβοὺς τῶν ὤτων τῶν δεξιῶν, καὶ ἐπὶ τὰ ἄκρα τῶν χειρῶν αὐτῶν τῶν δεξιῶν· καὶ ἐπὶ τὰ ἄκρα τῶν ποδῶν αὐτῶν τῶν δεξιῶν· καὶ προσέχεε Μωυσῆς τὸ αἷμα ἐπὶ τὸ θυσιαστήριον κύκλῳ.
\vs{24}Καὶ ἔλαβε τὸ στέαρ, καὶ τὴν ὀσφὺν, καὶ τὸ στέαρ τὸ ἐπὶ τῆς κοιλίας, καὶ τὸν λοβὸν τοῦ ἥπατος, καὶ τοὺς δύο νεφροὺς, καὶ τὸ στέαρ τὸ ἐπʼ αὐτῶν, καὶ τὸν βραχίονα τὸν δεξιόν.
\vs{25}Καὶ ἀπὸ τοῦ κανοῦ τῆς τελειώσεως, τοῦ ὄντος ἔναντι Κυρίου, καὶ ἔλαβεν ἄρτον ἕνα ἄζυμον, καὶ ἄρτον ἐξ ἐλαίου ἕνα, καὶ λάγανον ἓν, καὶ ἐπέθηκεν ἐπὶ τὸ στέαρ, καὶ τὸν βραχίονα τὸν δεξιόν.
\vs{26}Καὶ ἐπέθηκεν ἅπαντα ἐπὶ τὰς χεῖρας Ἀαρὼν, καὶ ἐπὶ τὰς χεῖρας τῶν υἱῶν αὐτοῦ, καὶ ἀνήνεγκεν αὐτὰ ἀφαίρεμα ἔναντι Κυρίου.
\vs{27}Καὶ ἔλαβε Μωυσῆς ἀπὸ τῶν χειρῶν αὐτῶν, καὶ ἀνήνεγκεν αὐτὰ Μωυσῆς ἐπὶ τὸ θυσιαστήριον, ἐπὶ τὸ ὁλοκαύτωμα τῆς τελειώσεως, ὅ ἐστι ὀσμὴ εὐωδίας· κάρπωμά ἐστιν τῷ Κυρίῳ.
\vs{28}Καὶ λαβὼν Μωυσῆς τὸ στηθύνιον, ἀφεῖλεν αὐτὸ ἐπίθεμα ἔναντι Κυρίου, ἀπὸ τοῦ κριοῦ τῆς τελειώσεως· καὶ ἐγένετο Μωυσῇ ἐν μεριδι, καθὰ ἐνετείλατο Κύριος τῷ Μωυσῇ.

\vs{29}Καὶ ἔλαβε Μωυσῆς ἀπὸ τοῦ ἐλαίου τῆς χρίσεως, καὶ ἀπὸ τοῦ αἵματος τοῦ ἐπὶ τοῦ θυσιαστηρίου, καὶ προσέῤῥνεν ἐπὶ Ἀαρὼν, καὶ τὰς στολὰς αὐτοῦ, καὶ τοὺς υἱοὺς αὐτοῦ, καὶ τὰς στολὰς τῶν υἱῶν αὐτοῦ μετʼ αὐτοῦ.
\vs{30}Καὶ ἡγίασεν Ἀαρὼν, καὶ τὰς στολάς αὐτοῦ, καὶ τοὺς υἱοὺς αὐτοῦ, καὶ τὰς στολὰς τῶν υἱῶν αὐτοῦ μετʼ αὐτοῦ.
\vs{31}Καὶ εἶπε Μωυσῆς πρὸς Ἀαρὼν, καὶ τοὺς υἱοὺς αὐτοῦ, ἑψήσατε τὰ κρέα ἐν τῇ αὐλῇ τῆς σκηνῆς τοῦ μαρτυρίου ἐν τόπῳ ἁγίῳ· καὶ ἐκεῖ φάγεσθε αὐτὰ, καὶ τοὺς ἄρτους τοὺς ἐν τῷ κανῷ τῆς τελειώσεως, ὃν τρόπον συντέτακταί μοι, λέγων, Ἀαρὼν καὶ οἱ υἱοὶ αὐτοῦ φάγονται αὐτά.
\vs{32}Καὶ τὸ καταλειφθὲν τῶν κρεῶν καὶ τῶν ἄρτων ἐν πυρὶ κατακαύσατε.
\vs{33}Καὶ ἀπὸ τῆς θύρας τῆς σκηνῆς τοῦ μαρτυρίου οὐκ ἐξελεύσεσθε ἑπτὰ ἡμέρας, ἕως ἡμέρα πληρωθῇ, ἡμέρα τελειώσεως ὑμῶν· ἑπτὰ γὰρ ἡμέρας τελειώσει τὰς χεῖρας ὑμῶν.
\vs{34}Καθάπερ ἐποίησεν ἐν τῇ ἡμέρᾳ ταύτῃ, ᾗ ἐνετείλατο Κύριος τοῦ ποιῆσαι, ὥστε ἐξιλάσασθαι περὶ ὑμῶν.
\vs{35}Καὶ ἐπὶ τὴν θύραν τῆς σκηνῆς τοῦ μαρτυρίου καθήσεσθε ἑπτὰ ἡμέρας, ἡμέραν καὶ νύκτα· φυλάξεσθε τὰ φυλάγματα Κυρίου, ἵνα μὴ ἀποθάνητε· οὕτω γὰρ ἐνετείλατό μοι Κύριος ὁ Θεός.
\vs{36}Καὶ ἐποίησεν Ἀαρὼν καὶ οἱ υἱοὶ αὐτοῦ πάντας τοὺς λόγους, οὓς συνέταξε Κύριος τῷ Μωυσῇ.

\ch{9}
Καὶ ἐγενήθη τῇ ἡμέρᾳ τῇ ὀγδόῃ, ἐκάλεσε Μωυσῆς Ἀαρὼν, καὶ τοὺς υἱοὺς αὐτοῦ, καὶ τὴν γερουσίαν Ἰσραὴλ,
\vs{2}καὶ εἶπε Μωυσῆς πρὸς Ἀαρών, λάβε σεαυτῷ μοσχάριον ἐκ βοῶν περὶ ἁμαρτίας, καὶ κριὸν εἰς ὁλοκαύτωμα, ἄμωμα, καὶ προσένεγκε αὐτὰ ἔναντι Κυρίου.
\vs{3}Καὶ τῇ γερουσίᾳ Ἰσραὴλ λάλησον, λέγων, λάβετε χίμαρον ἐξ αἰγῶν ἕνα περὶ ἁμαρτίας, καὶ μοσχάριον, καὶ ἀμνὸν ἐνιαύσιον εἰς ὁλοκάρπωσιν, ἄμωμα,
\vs{4}καὶ μόσχον, καὶ κριὸν εἰς θυσίαν σωτηρίου ἔναντι Κυρίου, καὶ σεμίδαλιν πεφυραμένην ἐν ἐλαίῳ· ὅτι σήμερον Κύριος ὀφθήσεται ἐν ὑμῖν.
\vs{5}Καὶ ἔλαβον καθὸ ἐνετείλατο Μωυσῆς ἀπέναντι τῆς σκηνῆς τοῦ μαρτυρίου· καὶ προσῆλθε πᾶσα συναγωγὴ, καὶ ἔστησαν ἔναντι Κυρίου.
\vs{6}Καὶ εἶπε Μωυσῆς, τοῦτο τὸ ῥῆμα, ὃ εἶπε Κύριος, ποιήσατε, καὶ ὀφθήσεται ἐν ὑμῖν ἡ δόξα Κυρίου.
\vs{7}Καὶ εἶπε Μωυσῆς τῷ Ἀαρὼν, πρόσελθε πρὸς τὸ θυσιαστήριον, καὶ ποίησον τὸ περὶ τῆς ἁμαρτίας σου, καὶ τὸ ὁλοκαύτωμά σου, καὶ ἐξίλασαι περὶ σεαυτοῦ, καὶ τοῦ οἴκου σου· καὶ ποίησον τὰ δῶρα τοῦ λαοῦ, καὶ ἐξίλασαι περὶ αὐτῶν, καθάπερ ἐνετείλατο Κύριος τῷ Μωυσῇ.
\vs{8}Καὶ προσῆλθεν Ἀαρὼν πρὸς τὸ θυσιαστήριον, καὶ ἔσφαξε τὸ μοσχάριον τὸ περὶ τῆς ἁμαρτίας αὐτοῦ.
\vs{9}Καὶ προσήνεγκαν οἱ υἱοὶ Ἀαρὼν τὸ αἷμα πρὸς αὐτόν· καὶ ἔβαψε τὸν δάκτυλον εἰς τὸ αἷμα, καὶ ἐπέθηκεν ἐπὶ τὰ κέρατα τοῦ θυσιαστηρίου· καὶ τὸ αἷμα ἐξέχεεν ἐπὶ τὴν βάσιν τοῦ θυσιαστηρίου.
\vs{10}Καὶ τὸ στέαρ καὶ τοὺς νεφροὺς καὶ τὸν λοβὸν τοῦ ἥπατος τοῦ περὶ τῆς ἁμαρτίας ἀνήνεγκεν ἐπὶ τὸ θυσιαστήριον, ὃν τρόπον ἐνετείλατο Κύριος τῷ Μωυσῇ.
\vs{11}Καὶ τὰ κρέα καὶ τὴν βύρσαν κατέκαυσεν αὐτὰ πυρὶ, ἔξω τῆς παρεμβολῆς.
\vs{12}Καὶ ἔσφαξε τὸ ὁλοκαύτωμα· καὶ προσήνεγκαν οἱ υἱοὶ Ἀαρὼν τὸ αἷμα πρὸς αὐτόν· καὶ προσέχεεν ἐπὶ τὸ θυσιαστήριον κύκλῳ.
\vs{13}Καὶ τὸ ὁλοκαύτωμα προσήνεγκαν αὐτὸ κατὰ μέλη· αὐτὰ καὶ τὴν κεφαλὴν ἐπέθηκεν ἐπὶ τὸ θυσιαστήριον.
\vs{14}Καὶ ἔπλυνε τὴν κοιλίαν καὶ τοὺς πόδας ὕδατι· καὶ ἐπέθηκεν ἐπὶ τὸ ὁλοκαύτωμα ἐπὶ τὸ θυσιαστήριον.

\vs{15}Καὶ προσήνεγκε τὸ δῶρον τοῦ λαοῦ, καὶ ἔλαβε τὸν χίμαρον τὸν περὶ τῆς ἁμαρτίας τοῦ λαοῦ, καὶ ἔσφαξεν αὐτὸν, καὶ ἐκαθάρισεν αὐτὸν, καθὰ καὶ τὸν πρῶτον.
\vs{16}Καὶ προσήνεγκε τὸ ὁλοκαύτωμα, καὶ ἐποίησεν αὐτὸ ὡς καθήκει.
\vs{17}Καὶ προσήνεγκε τὴν θυσίαν, καὶ ἔπλησε τὰς χεῖρας ἀπʼ αὐτῆς, καὶ ἐπέθηκεν ἐπὶ τὸ θυσιαστήριον χωρὶς τοῦ ὁλοκαυτώματος τοῦ πρωϊνοῦ.
\vs{18}Καὶ ἔσφαξε τὸν μόσχον, καὶ τὸν κριὸν τῆς θυσίας τοῦ σωτηρίου τῆς τοῦ λαοῦ· καὶ προσήνεγκαν οἱ υἱοὶ Ἀαρὼν τὸ αἷμα πρὸς αὐτόν, καὶ προσέχεε πρὸς τὸ θυσιαστήριον κύκλῳ,
\vs{19}καὶ τὸ στέαρ τὸ ἀπὸ τοῦ μόσχου, καὶ τοῦ κριοῦ τὴν ὀσφὺν, καὶ τὸ στέαρ τὸ κατακαλύπτον ἐπὶ τῆς κοιλίας, καὶ τοὺς δύο νεφροὺς, καὶ τὸ στέαρ τὸ ἐπʼ αὐτῶν, καὶ τὸν λοβὸν τὸν ἐπὶ τοῦ ἥπατος.
\vs{20}Καὶ ἐπέθηκε τὰ στέατα ἐπὶ τὰ στηθύνια καὶ ἀνήνεγκε τὰ στέατα ἐπὶ τὸ θυσιαστήριον.
\vs{21}Καὶ τὸ στηθύνιον, καὶ τὸν βραχίονα τὸν δεξιὸν ἀφεῖλεν Ἀαρὼν ἀφαίρεμα ἔναντι Κυρίου, ὃν τρόπον συνέταξε Κύριος τῷ Μωυσῇ.
\vs{22}Καὶ ἐξάρας Ἀαρὼν τὰς χεῖρας ἐπὶ τὸν λαὸν, εὐλόγησεν αὐτούς· καὶ κατέβη ποιήσας τὸ περὶ τῆς ἁμαρτίας, καὶ τὰ ὁλοκαυτώματα, καὶ τὰ τοῦ σωτηρίου.
\vs{23}Καὶ εἰσῆλθε Μωυσῆς καὶ Ἀαρὼν εἰς τὴν σκηνὴν τοῦ μαρτυρίου· καὶ ἐξελθόντες εὐλόγησαν πάντα τὸν λαὸν· καὶ ὤφθη δόξα Κυρίου παντὶ τῷ λαῷ.
\vs{24}Καὶ ἐξῆλθε πῦρ παρὰ Κυρίου, καὶ κατέφαγε τὰ ἐπὶ τοῦ θυσιαστηρίου, τά τε ὁλοκαυτώματα, καὶ τὰ στέατα· καὶ εἶδε πᾶς ὁ λαὸς, καὶ ἐξέστη, καὶ ἔπεσαν ἐπὶ πρόσωπον.

\ch{10}
Καὶ λαβόντες οἱ δύο υἱοὶ Ἀαρὼν Ναδὰβ καὶ Ἀβιοὺδ, ἕκαστος τὸ πυρεῖον αὐτοῦ, ἐπέθηκαν ἐπʼ αὐτὸ πῦρ, καὶ ἐπέβαλον ἐπʼ αὐτὸ θυμίαμα, καὶ προσήνεγκαν ἔναντι Κυρίου πῦρ ἀλλότριον, ὃ οὐ προσέταξε Κύριος αὐτοῖς.
\vs{2}Καὶ ἐξῆλθε πῦρ παρὰ Κυρίου, καὶ κατέφαγεν αὐτοὺς, καὶ ἀπέθανον ἔναντι Κυρίου.
\vs{3}Καὶ εἶπε Μωυσῆς πρὸς Ἀαρὼν, τοῦτό ἐστιν, ὃ εἶπε Κύριος, λέγων, ἐν τοῖς ἐγγίζουσί μοι ἁγιασθήσομαι, καὶ ἐν πάσῃ τῇ συναγωγῇ δοξασθήοσμαι· καὶ κατενύχθη Ἀαρών.
\vs{4}Καὶ ἐκάλεσε Μωυσῆς τὸν Μισαδάη, καὶ τὸν Ἐλισαφὰν, υἱοὺς Ὀζιὴλ, υἱοὺς τοῦ ἀδελφοῦ τοῦ πατρὸς Ἀαρὼν, καὶ εἶπεν αὐτοῖς, προσέλθατε καὶ ἄρατε τοὺς ἀδελφοὺς ὑμῶν ἐκ προσώπου τῶν ἁγίων ἔξω τῆς παρεμβολῆς.
\vs{5}Καὶ προσῆλθον, καὶ ᾖραν αὐτοὺς ἐν τοῖς χιτῶσιν αὐτῶν ἔξω τῆς παρεμβολῆς, ὃν τρόπον εἶπε Μωυσῆς.
\vs{6}Καὶ εἶπε Μωυσῆς πρὸς Ἀαρὼν καὶ Ἐλεάζαρ καὶ Ἰθάμαρ τοὺς υἱοὺς αὐτοῦ τοὺς καταλελειμμένους, τὴν κεφαλὴν ὑμῶν οὐκ ἀποκιδαρώσετε, καὶ τὰ ἱμάτια ὑμῶν οὐ διαῤῥήξετε, ἵνα μὴ ἀποθάνητε, καὶ ἐπὶ πᾶσαν τὴν συναγωγὴν ἔσται θυμός· οἱ δὲ ἀδελφοὶ ὑμῶν, πᾶς ὁ οἶκος Ἰσραὴλ, κλαύσονται τὸν ἐμπυρισμὸν, ὃν ἐνεπυρίσθησαν ὑπὸ Κυρίου.
\vs{7}Καὶ ἀπὸ τὴς θύρας τῆς σκηνῆς τοῦ μαρτυρίου οὐκ ἐξελεύσεσθε, ἵνα μὴ ἀποθάνητε· τὸ ἔλαιον γὰρ τῆς χρίσεως, τὸ παρὰ Κυρίου, ἐφʼ ὑμῖν, καὶ ἐποίησαν κατὰ τὸ ῥῆμα Μωυσῆ.

\vs{8}Καὶ ἐλάλησε Κύριος τῷ Ἀαρὼν, λέγων,
\vs{9}οἶνον καὶ σίκερα οὐ πίεσθε σὺ καὶ οἱ υἱοί σου μετὰ σοῦ, ἡνίκα ἐὰν εἰσπορεύησθε εἰς τὴν σκηνὴν τοῦ μαρτυρίου, ἢ προσπορευομένων ὑμῶν πρὸς τὸ θυσιαστήριον, καὶ οὐ μὴ ἀποθάνητε· νόμιμον αἰώνιον εἰς τὰς γενεὰς ὑμῶν,
\vs{10}διαστεῖλαι ἀναμέσον τῶν ἁγίων καὶ τῶν βεβήλων, καὶ ἀναμέσον τῶν ἀκαθάρτων καὶ τῶν καθαρῶν,
\vs{11}καὶ συμβιβάξειν τοὺς υἱοὺς Ἰσραὴλ ἅπαντα τὰ νόμιμα, ἃ ἐλάλησε Κύριος πρὸς αὐτοὺς διὰ χειρὸς Μωυσῆ.
\vs{12}Καὶ εἶπε Μωυσῆς πρὸς Ἀαρὼν καὶ πρὸς Ἐλεάζαρ καὶ Ἰθάμαρ τοὺς υἱοὺς Ἀαρὼν τοὺς καταλειφθέντας, λάβετε τὴν θυσίαν τὴν καταλειφθεῖσαν ἀπὸ τῶν καρπωμάτων Κυρίου, καὶ φάγεσθε ἄζυμα παρὰ τὸ θυσιαστήριον· ἅγια ἁγίων ἐστί.
\vs{13}Καὶ φάγεσθε αὐτὴν ἐν τόπῳ ἁγίῳ· νόμιμον γάρ σοι ἐστὶ, καὶ νόμιμον τοῖς υἱοῖς σου τοῦτο ἀπὸ τῶν καρπωμάτων Κυρίου· οὕτω γὰρ ἐντέταλταί μοι.
\vs{14}Καὶ τὸ στηθύνιον τοῦ ἀφορίσματος, καὶ τὸν βραχίονα τοῦ ἀφαιρέματος φάγεσθε ἐν τόπῳ ἁγίῳ, σὺ καὶ οἱ υἱοί σου καὶ ὁ οἶκός σου μετὰ σοῦ· νόμιμον γὰρ σοι, καὶ νόμιμον τοῖς υἱοῖς σου ἐδόθη ἀπὸ τῶν θυσιῶν τοῦ σωτηρίου τῶν υἱῶν Ἰσραήλ.
\vs{15}Τὸν βραχίονα τοῦ ἀφαιρέματος, καὶ τὸ στηθύνιον τοῦ ἀφορίσματος ἐπὶ τῶν καρπωμάτων τῶν στεάτωι· προσοίσουσιν ἀφόρισμα ἀφορίσαι ἔναντι Κυρίου· καὶ ἔσται σοι καὶ τοῖς υἱοῖς σου καὶ ταῖς θυγατράσι σου μετὰ σοῦ νόμιμον αἰώνιον, ὃν τρόπον συνέταξε Κύριος τῷ Μωυσῇ.

\vs{16}Καὶ τὸν χίμαρον τὸν περὶ τῆς ἁμαρτίας ζητῶν ἐξεζήτησε Μωυσῆς· καὶ ὁ δὲ ἐνπεπύριστο· καὶ ἐθυμώθη Μωυσῆς ἐπὶ Ἐλεάζαρ καὶ Ἰθάμαρ τοὺς υἱοὺς Ἀαρὼν τοὺς καταλελειμμένους, λέγων,
\vs{17}διατί οὐκ ἐφάγετε τὸ περὶ τῆς ἁμαρτίας ἐν τόπῳ ἁγίῳ; ὅτι γὰρ ἅγια ἁγίων ἐστι, τοῦτο ἔδωκεν ὑμῖν φαγεῖν, ἵνα ἀφέλητε τὴν ἁμαρτίαν τῆς συναγωγῆς, καὶ ἐξιλάσησθε περὶ αὐτῶν ἔναντι Κυρίου.
\vs{18}Οὐ γὰρ εἰσήχθη τοῦ αἵματος αὐτοῦ εἰς τὸ ἅγιον· κατὰ πρόσωπον ἔσω φάγεσθε αὐτὸ ἐν τόπῳ ἁγίῳ, ὃν τρόπον μοι συνέταξε Κύριος.
\vs{19}Καὶ ἐλάλησεν Ἀαρὼν πρὸς Μωυσῆν, λέγων, εἰ σήμερον προσαγιόχασι τὰ περὶ τῆς ἁμαρτίας αὐτῶν, καὶ τὰ ὁλοκαυτώματα αὐτῶν ἔναντι Κυρίου, καὶ συμβέβηκέ μοι τοιαῦτα, καὶ φάγομαι τὰ περὶ τῆς ἁμαρτίας σήμερον, μὴ ἀρεστὸν ἔται Κυρίῳ;
\vs{20}Καὶ ἤκουσε Μωυσῆς, καὶ ἤρεσεν αὐτῷ.

\ch{11}
Καὶ ἐλάλησε Κύριος πρὸς Μωυσῆν καὶ Ἀαρὼν, λέγων,
\vs{2}λαλήσατε τοῖς υἱοῖς Ἰσραὴλ, λέγοντες, ταῦτα τὰ κτήνη, ἃ φάγεσθε ἀπὸ πάντων τῶν κτηνῶν τῶν ἐπὶ τῆς γῆς.
\vs{3}Πᾶν κτῆνος διχηλοῦν ὁπλὴν καὶ ὀνυχιστῆρας ὀνυχίζον δύο χηλῶν, καὶ ἀνάγον μηρυκισμὸν ἐν τοῖς κτήνεσι, ταῦτα φάγεσθε.
\vs{4}Πλὴν ἀπὸ τούτων οὐ φάγεσθε, ἀπὸ τῶν ἀναγόντων μηρυκισμὸν, καὶ ἀπὸ τῶν διχηλούντων τὰς ὁπλὰς, καὶ ὀνυχιζόντων ὀνυχιστῆρας· τὸν κάμηλον, ὅτι ἀνάγει μηρυκισμὸν τοῦτο, ὁπλὴν δὲ οὐ διχηλεῖ, ἀκάθαρτον τοῦτο ὑμῖν.
\vs{5}Καὶ τὸν δασύποδα, ὅτι ἀνάγει μηρυκισμὸν τοῦτο, καὶ ὁπλὴν οὐ διχηλεῖ, ἀκάθαρτον τοῦτο ὑμῖν.
\vs{6}Καὶ τὸν χοιρογρύλλιον, ὅτι οὐκ ἀνάγει μηρυκισμὸν τοῦτο, καὶ ὁπλὴν οὐ διχηλεῖ, ἀκάθαρτον τοῦτο ὑμῖν.
\vs{7}Καὶ τὸν ὗν, ὅτι διχηλεῖ ὁπλὴν τοῦτο, καὶ ὀνυχίζει ὄνυχας ὁπλῆς, καὶ τοῦτο οὐκ ἀνάγει μηρυκισμὸν, ἀκάθαρτον τοῦτο ὑμῖν.
\vs{8}Ἀπὸ τῶν κρεῶν αὐτῶν οὐ φάγεσθε, καὶ τῶν θνησιμαίων αὐτῶν οὐχ ἅψεσθε· ἀκάθαρτα ταῦτα ὑμῖν.

\vs{9}Καὶ ταῦτα, ἃ φάγεσθε ἀπὸ πάντων τῶν ἐν τοῖς ὕδασι· πάντα ὅσα ἐστὶν αὐτοῖς πτερύγια καὶ λεπίδες ἐν τοῖς ὕδασι, καὶ ἐν ταῖς θαλάσσαις, καὶ ἐν τοῖς χειμάῤῥοις, ταῦτα φάγεσθε.
\vs{10}Καὶ πάντα ὅσα οὐκ ἔστιν αὐτοῖς πτερύγια, οὐδὲ λεπίδες ἐν τῷ ὕδατι, ἢ ἐν ταῖς θαλάσσαις, καὶ ἐν τοῖς χειμάῤῥοις, ἀπὸ πάντων ὧν ἐρεύγεται τὰ ὕδατα, καὶ ἀπὸ πάσης ψυχῆς τῆς ζώσης ἐν τῷ ὕδατι, βδέλυγμά ἐστι, καὶ βδελύγματα ἔσονται ὑμῖν.
\vs{11}Ἀπὸ τῶν κρεῶν αὐτῶν οὐκ ἔδεσθε, καὶ τὰ θνησιμαῖα αὐτῶν βδελύξεσθε.
\vs{12}Καὶ πάντα ὅσα οὐκ ἔστιν αὐτοῖς πτερύγια, οὐδὲ λεπίδες τῶν ἐν τοῖς ὕδασι, βδέλυγμα τοῦτό ἐστιν ὑμῖν.
\vs{13}Καὶ ταῦτα, ἃ βδελύξεσθε ἀπὸ τῶν πετεινῶν, καὶ οὐ βρωθήσεται, βδέλυγμά ἐστι· τὸν ἀετὸν, καὶ τὸν γρύπα, καὶ τὸν ἁλιαίετον,
\vs{14}καὶ τὸν γύπα, καὶ τὸν ἴκτινον καὶ τὰ ὅμοια αὐτῷ.
\vs{15}Καὶ στρουθὸν, καὶ γλαῦκα, καὶ λάρον, καὶ τὰ ὅμοια αὐτῷ·
\vs{16}Καὶ πάντα κόρακα, καὶ τὰ ὅμοια αὐτῷ· καὶ ἱέρακα, καὶ τὰ ὅμοια αὐτῷ·
\vs{17}καὶ νυκτικόρακα, καὶ καταράκτην, καὶ ἴβιν,
\vs{18}καὶ πορφυρίωνα, καὶ πελεκᾶνα, καὶ κύκνον,
\vs{19}καὶ ἐρωδιὸν, καὶ χαράδριον, καὶ τὰ ὅμοια αὐτῷ· καὶ ἔποπα, καὶ νυκτερίδα.
\vs{20}Καὶ πάντα τὰ ἑρπετὰ τῶν πετεινῶν, ἃ πορεύεται ἐπὶ τέσσαρα, βδελύγματά ἐστιν ὑμῖν.
\vs{21}Ἀλλὰ ταῦτα φάγεσθε ἀπὸ τῶν ἑρπετῶν τῶν πετεινῶν, ἃ πορεύεται ἐπὶ τέσσαρα, ἃ ἔχει σκέλη ἀνώτερον τῶν ποδῶν αὐτοῦ, πηδᾷν ἐν αὐτοῖς ἐπὶ τῆς γῆς.
\vs{22}Καὶ ταῦτα φάγεσθε ἀπʼ αὐτῶν· τὸν βροῦχον, καὶ τὰ ὅμοια αὐτῷ· καὶ τὸν ἀττάκην, καὶ τὰ ὅμοια αὐτῷ· καὶ ὀφιομάχην, καὶ τὰ ὅμοια αὐτῷ· καὶ τὴν ἀκρίδα, καὶ τὰ ὅμοια αὐτῇ·
\vs{23}Πᾶν ἑρπετὸν ἀπὸ τῶν πετεινῶν, οἷς εἰσι τέσσαρες πόδες, βδελύγματά ἐστιν ὑμῖν,
\vs{24}καὶ ἐν τούτοις μιανθήσεσθε· πᾶς ὁ ἁπτόμενος τῶν θνησιμαίων αὐτῶν, ἀκάθαρτος ἔσται ἕως ἑσπέρας.
\vs{25}Καὶ πᾶς ὁ αἴρων τῶν θνησιμαίων αὐτῶν, πλυνεῖ τὰ ἱμάτια αὐτοῦ, καὶ ἀκάθαρτος ἔσται ἕως ἑσπέρας.
\vs{26}Καὶ ἐν πᾶσι τοῖς κτήνεσιν ὅ ἐστι διχηλοῦν ὁπλὴν, καὶ ὀνυχιστῆρας ὀνυχίζει, καὶ μηρυκισμὸν οὐ μηρυκᾶται, ἀκάθαρτα ἔσονται ὑμῖν· πᾶς ὁ ἁπτόμενος τῶν θνησιμαίων αὐτῶν, ἀκάθαρτος ἔσται ἕως ἑσπέρας.
\vs{27}Καὶ πᾶς ὃς πορεύεται ἐπὶ χειρῶν ἐν πᾶσι τοῖς θηρίοις, ἃ πορεύεται ἐπὶ τέσσαρα, ἀκάθαρτά ἐστιν ὑμῖν· πᾶς ὁ ἁπτόμενος τῶν θνησιμαίων αὐτῶν, ἀκάθαρτος ἔσται ἕως ἑσπέρας.
\vs{28}Καὶ ὁ αἴρων τῶν θνησιμαίων αὐτῶν, πλυνεῖ τὰ ἱμάτια αὐτοῦ, καὶ ἀκάθαρτος ἔσται ἕως ἑσπέρας· ἀκάθαρτα ταῦτά ἔστιν ὑμῖν.

\vs{29}Καὶ ταῦτα ὑμῖν ἀκάθαρτα ἀπὸ τῶν ἑρπετῶν τῶν ἐπὶ τῆς γῆς· ἡ γαλὴ, καὶ ὁ μῦς, καὶ ὁ κροκόδειλος ὁ χερσαῖος,
\vs{30}μυγάλη, καὶ χαμαιλέων, καὶ χαλαβώτης, καὶ σαῦρα, καὶ ἀσπάλαξ.
\vs{31}Ταῦτα ἀκάθαρτα ὑμῖν ἀπὸ πάντων τῶν ἑρπετῶν τῶν ἐπὶ τῆς γῆς· πᾶς ὁ ἁπτόμενος αὐτῶν τεθνηκότων, ἀκάθαρτος ἔσται ἕως ἑσπέρας.
\vs{32}Καὶ πᾶν ἐφʼ ὃ ἂν ἐπιπέσῃ ἀπʼ αὐτῶν ἐπʼ αὐτὸ τεθνηκότων αὐτῶν, ἀκάθαρτον ἔσται ἀπὸ παντὸς σκεύους ξυλίνου ἢ ἱματίου ἢ δέρματος ἢ σάκκου· πᾶν σκεῦος ὃ ἂν ποιηθῇ ἔργον ἐν αὐτῷ, εἰς ὕδωρ βαφήσεται, καὶ ἀκάθαρτον ἔσται ἕως ἑσπέρας· καὶ καθαρὸν ἔσται.
\vs{33}Καὶ πᾶν σκεῦος ὀστράκινον εἰς ὃ ἐὰν πέσῃ ἀπὸ τούτων ἔνδον, ὅσα ἐὰν ἔνδον ᾖ, ἀκάθαρτα ἔσται, καὶ αὐτὸ συντριβήσεται.
\vs{34}Καὶ πᾶν βρῶμα, ὃ ἔσθεται, εἰς ὃ ἂν ἐπέλθῃ ἐπʼ αὐτὸ ὕδωρ, ἀκάθαρτον ἔσται· καὶ πᾶν ποτὸν, ὃ πίνεται ἐν παντὶ ἀγγείῳ, ἀκάθαρτον ἔσται.
\vs{35}Καὶ πᾶν ὃ ἐὰν ἐπιπέσῃ ἀπὸ τῶν θνησιμαίων αὐτῶν ἐπʼ αὐτό, ἀκάθαρτον ἔσται· κλίβανοι καὶ χυτρόποδες καθαιρεθήσονται· ἀκάθαρτα ταῦτά ἐστι, καὶ ἀκάθαρτα ταῦτα ὑμῖν ἔσονται.
\vs{36}Πλὴν πηγῶν ὑδάτων καὶ λάκκου καὶ συναγωγῆς ὕδατος, ἔσται καθαρόν· ὁ δὲ ἁπτόμενος τῶν θνησιμαίων αὐτῶν, ἀκάθαρτος ἔσται.
\vs{37}Ἐὰν δὲ ἐπιπέσῃ ἀπὸ τῶν θνησιμαίων αὐτῶν ἐπὶ πᾶν σπέρμα σπόριμον, ὃ σπαρήσεται, καθαρὸν ἔσται.
\vs{38}Ἐὰν δὲ ἐπιχυθῇ ὕδωρ ἐπὶ πᾶν σπέρμα, καὶ ἐπιπέσῃ τῶν θνησιμαίων αὐτῶν ἐπʼ αὐτό, ἀκάθαρτόν ἐστιν ὑμῖν.
\vs{39}Ἐὰν δὲ ἀποθάνῃ τῶν κτηνῶν, ὅ ἐστιν ὑμῖν φαγεῖν τοῦτο, ὁ ἁπτόμενος τῶν θνησιμαίων αὐτῶν, ἀκάθαρτος ἔσται ἕως ἑσπέρας·
\vs{40}καὶ ὁ ἐσθίων ἀπὸ τῶν θνησιμαίων τούτων, πλυνεῖ τὰ ἱμάτια, καὶ ἀκάθαρτος ἔσται ἕως ἑσπέρας· καὶ ὁ αἴρων ἀπὸ θνησιμαίων αὐτῶν, πλυνεῖ τὰ ἱμάτια, καὶ λούσεται ὕδατι, καὶ ἀκάθαρτος ἔσται ἕως ἑσπέρας.
\vs{41}Καὶ πᾶν ἑρπετὸν, ὃ ἕρπει ἐπὶ τῆς γῆς, βδέλυγμα ἔσται τοῦτο ὑμῖν· οὐ βρωθήσεται.
\vs{42}Καὶ πᾶς ὁ πορευόμενος ἐπὶ κοιλίας, καὶ πᾶς ὁ πορευόμενος ἐπὶ τέσσαρα διαπαντός, ὃ πολυπληθεῖ ποσὶν ἐν πᾶσι τοῖς ἑρπετοῖς τοῖς ἕρπουσιν ἐπὶ τῆς γῆς, οὐ φάγεσθε αὐτὸ, ὅτι βδέλυγμα ὑμῖν ἐστι.
\vs{43}Καὶ οὐ μὴ βδελύξητε τὰς ψυχὰς ὑμῶν ἐν πᾶσι τοῖς ἑρπετοῖς τοῖς ἕρπουσιν ἐπὶ τῆς γῆς, καὶ οὐ μιανθήσεσθε ἐν τούτοις, καὶ οὐκ ἀκάθαρτοι ἔσεσθε ἐν αὐτοῖς,
\vs{44}ὅτι ἐγώ εἰμι Κύριος ὁ Θεὸς ὑμῶν· καὶ ἁγιασθήσεσθε, καὶ ἅγιοι ἔσεσθε, ὅτι ἅγιός εἰμι ἐγὼ Κύριος ὁ Θεὸς ὑμῶν· καὶ οὐ μιανεῖτε τὰς ψυχὰς ὑμῶν ἐν πᾶσι τοῖς ἑρπετοῖς τοῖς κινουμένοις ἐπὶ τῆς γῆς,
\vs{45}ὅτι ἐγώ εἰμι Κύριος ὁ ἀναγαγὼν ὑμᾶς ἐκ γῆς Αἰγύπτου εἶναι ὑμῶν Θεός· καὶ ἔσεσθε ἅγιοι, ὅτι ἅγιός εἰμι ἐγὼ Κύριος.
\vs{46}Οὗτος ὁ νόμος περὶ τῶν κτηνῶν καὶ τῶν πετεινῶν καὶ πάσης ψυχῆς τῆς κινουμένης ἐν τῷ ὕδατι, καὶ πάσης ψυχῆς ἑρπούσης ἐπὶ τῆς γῆς,
\vs{47}διαστεῖλαι ἀναμέσον τῶν ἀκαθάρτων καὶ ἀναμέσον τῶν καθαρῶν, καὶ ἀναμέσον τῶν ζωογονούντων τὰ ἐσθιόμενα καὶ ἀναμέσον τῶν ζωογονούντων τὰ μὴ ἐσθιόμενα.

\ch{12}
Καὶ ἐλάλησε Κύριος πρὸς Μωυσῆν, λέγων,
\vs{2}λάλησον τοῖς υἱοῖς Ἰσραὴλ, καὶ ἐρεῖς πρὸς αὐτοὺς, γυνὴ ἥτις ἐὰν σπερματισθῇ, καὶ τέκῃ ἄρσεν, καὶ ἀκάθαρτος ἔσται ἑπτὰ ἡμέρας· κατὰ τὰς ἡμέρας τοῦ χωρισμοῦ τῆς ἀφέδρου αὐτῆς, ἀκάθαρτος ἔσται.
\vs{3}Καὶ τῇ ἡμέρᾳ τῇ ὀγδόῃ περιτεμεῖ τὴν σάρκα τῆς ἀκροβυστίας αὐτοῦ.
\vs{4}Καὶ τριάκοντα καὶ τρεῖς ἡμέρας καθήσεται ἐν αἵματι ἀκαθάρτῳ αὐτῆς· παντὸς ἁγίου οὐξ ἅψεται, καὶ εἰς τὸ ἁγιαστήριον οὐκ εἰσελεύσεται, ἕως ἂν πληρωθῶσιν αἱ ἡμέραι καθάρσεως αὐτῆς.
\vs{5}Ἐὰν δὲ θῆλυ τέκῃ, καὶ ἀκάθαρτος ἔσται δὶς ἑπτὰ ἡμέρας, κατὰ τὴν ἄφεδρον αὐτῆς· καὶ ἑξήκοντα ἡμέρας καὶ ἓξ καθεσθήσεται ἐν αἵματι ἀκαθάρτῳ αὐτῆς.

\vs{6}Καὶ ὅταν ἀναπληρωθῶσιν αἱ ἡμέραι καθάρσεως αὐτῆς ἐφʼ υἱῷ ἢ ἐπὶ θυγατρι, προσοίσει ἀμνὸν ἐνιαύσιον ἄμωμον εἰς ὁλοκαύτωμα, καὶ νοσσὸν περιστερᾶς ἢ τρυγόνα περὶ ἁμαρτίας ἐπὶ τὴν θύραν τῆς σκηνῆς τοῦ μαρτυρίου, πρὸς τὸν ἱερέα.
\vs{7}Καὶ προσοίσει αὐτὸν ἔναντι Κυρίου· καὶ ἐξιλάσεται περὶ αὐτῆς ὁ ἱερεὺς, καὶ καθαριεῖ αὐτὴν ἀπὸ τῆς πηγῆς τοῦ αἵματος αὐτῆς· οὗτος ὁ νόμος τῆς τικτούσης ἄρσεν ἢ θῆλυ.
\vs{8}Ἐὰν δὲ μὴ εὑρίσκῃ ἡ χεὶρ αὐτῆς τὸ ἱκανὸν εἰς ἀμνὸν, καὶ λήψεται δύο τρυγόνας ἢ δύο νοσσοὺς περιστερῶν, μίαν εἰς ὁλοκαύτωμα, καὶ μίαν περὶ ἁμαρτίας· καὶ ἐξιλάσεται περὶ αὐτῆς ὁ ἱερεὺς, καὶ καθαρισθήσεται.

\ch{13}
Καὶ ἐλάλησε Κύριος πρὸς Μωυσῆν καὶ Ἀαρὼν, λέγων,
\vs{2}ἀνθρώπῳ ἐάν τινι γένηται ἐν δέρματι χρωτὸς αὐτοῦ οὐλὴ σημασίας τηλαυγὴς, καὶ γένηται ἐν δέρματι χρωτὸς αὐτοῦ ἁφὴ λέπρας, ἀχθήσεται πρὸς Ἀαρὼν τὸν ἱερέα, ἢ ἕνα τῶν υἱῶν αὐτοῦ τῶν ἱερέων.
\vs{3}Καὶ ὄψεται ὁ ἱερεὺς τὴν ἁφὴν ἐν δέρματι τοῦ χρωτὸς αὐτοῦ, καὶ ἡ θρὶξ ἐν τῇ ἁφῇ μεταβάλῃ λευκὴ, καὶ ἡ ὄψις τῆς ἁφῆς ταπεινὴ ἀπὸ τοῦ δέρματος τοῦ χρωτὸς, ἁφὴ λέπρας ἐστί· καὶ ὄψεται ὁ ἱερεὺς, καὶ μιανεῖ αὐτόν.
\vs{4}Ἐὰν δὲ καὶ τηλαυγὴς λευκὴ ἦ ἐν τῷ δέρματι τοῦ χρωτὸς αὐτοῦ, καὶ ταπεινὴ μὴ ἦ ἡ ὄψις αὐτῆς ἀπὸ τοῦ δέρματος, καὶ ἡ θρὶξ αὐτοῦ οὐ μετέβαλε τρίχα λευκὴν, αὐτὴ δέ ἐστιν ἀμαυρὰ, καὶ ἀφοριεῖ ὁ ἱερεὺς τὴν ἁφὴν ἑπτὰ ἡμέρας.
\vs{5}Καὶ ὄψεται ὁ ἱερεὺς τὴν ἁφὴν τῇ ἡμέρᾳ τῇ ἑβδόμῃ· καὶ ἰδοὺ ἡ ἁφὴ μένει ἐναντίον αὐτοῦ, οὐ μετέπεσεν ἡ ἁφὴ ἐν τῷ δέρματι, καὶ ἀφοριεῖ αὐτὸν ὁ ἱερεὺς ἑπτὰ ἡμέρας τοδεύτερον.
\vs{6}Καὶ ὄψεται ὁ ἱερεὺς αὐτὸν τῇ ἡμέρᾳ τῇ ἑβδόμῃ τοδεύτερον· καὶ ἰδοὺ ἀμαυρὰ ἡ ἁφή, οὐ μετέπεσεν ἡ ἁφὴ ἐν τῷ δέρματι· καὶ καθαριεῖ αὐτὸν ὁ ἱερεύς, σημασία γάρ ἐστι· καὶ πλυνάμενος τὰ ἱμάτια αὐτοῦ, καθαρὸς ἔσται.
\vs{7}Ἐὰν δὲ μεταβαλοῦσα μεταπέσῃ ἡ σημασία ἐν τῷ δέρματι, μετὰ τὸ ἰδεῖν αὐτὸν τὸν ἱερέα τοῦ καθαρίσαι αὐτόν, καὶ ὀφθήσεται τοδεύτερον τῷ ἱερεῖ.
\vs{8}Καὶ ὄψεται αὐτὸν ὁ ἱερεύς, καὶ ἰδοὺ μετέπεσεν ἡ σημασία ἐν τῷ δέρματι, καὶ μιανεῖ αὐτὸν ὁ ἱερεύς· λέπρα ἐστί.

\vs{9}Καὶ ἁφὴ λέπρας ἐὰν γένηται ἐν ἀνθρώπῳ, και ἥξει πρὸς τὸν ἱερέα·
\vs{10}Καὶ ὄψεται ὁ ἱερεὺς, καὶ ἰδοὺ οὐλὴ λευκὴ ἐν τῷ δέρματι, καὶ αὕτη μετέβαλε τρίχα λευκὴν, καὶ ἀπὸ τοῦ ὑγιοῦς τῆς σαρκὸς τῆς ζώσης ἐν τῇ οὐλῇ.
\vs{11}Λέπρα παλαιουμένη ἐστὶν ἐν τῷ δέρματι τοῦ χρωτός, καὶ μιανεῖ αὐτὸν ὁ ἱερεὺς, καὶ ἀφοριεῖ αὐτὸν, ὅτι ἀκάθαρτός ἐστιν.

\vs{12}Ἐὰν δὲ ἀνθοῦσα ἐξανθήσῃ λέπρα ἐν τῷ δέρματι, καὶ καλύψῃ ἡ λέπρα πᾶν τὸ δέρμα τῆς ἁφῆς ἀπὸ κεφαλῆς ἕως ποδῶν, καθʼ ὅλην τὴν ὅρασιν τοῦ ἱερέως·
\vs{13}Καὶ ὄψεται ὁ ἱερεὺς, καὶ ἰδοὺ ἐκάλυψεν ἡ λέπρα πᾶν τὸ δέρμα τοῦ χρωτός· καὶ καθαριεῖ αὐτὸν ὁ ἱερεὺς τὴν ἁφήν, ὅτι πᾶν μετέβαλε λευκὸν, καθαρόν ἐστι.
\vs{14}Καὶ ᾗ ἂν ἡμέρᾳ ὀφθῇ ἐν αὐτῷ χρὼς ζῶν, μιανθήσεται.
\vs{15}Καὶ ὅψεται ὁ ἱερεὺς τὸν χρῶτα τὸν ὑγιῆ, καὶ μιανεῖ αὐτὸν ὁ χρὼς ὁ ὑγιὴς, ὅτι ἀκάθαρτός ἐστι· λέπρα ἐστίν.
\vs{16}Ἐὰν δὲ ἀποκαταστῇ ὁ χρὼς ὁ ὑγιὴς, καὶ μεταβάλῃ λευκὴ, καὶ ἐλεύσεται πρὸς τὸν ἱερέα·
\vs{17}Καὶ ὄψεται ὁ ἱερεὺς, καὶ ἰδοὺ μετέβαλεν ἡ ἁφὴ εἰς τὸ λευκόν, καὶ καθαριεῖ ὁ ἱερεὺς τὴν ἁφήν· καθαρός ἐστι.

\vs{18}Καὶ σὰρξ ἐὰν γένηται ἐν τῷ δέρματι αὐτοῦ ἕλκος, καὶ ὑγιασθῇ,
\vs{19}καὶ γένηται ἐν τῷ τόπῳ τοῦ ἕλκους οὐλὴ λευκὴ, ἢ τηλαυγὴς λευκαίνουσα, ἢ πυῤῥίζουσα, καὶ ὀφθήσεται τῷ ἱερεῖ·
\vs{20}Καὶ ὄψεται ὁ ἱερεὺς, καὶ ἰδοὺ ἡ ὄψις ταπεινοτέρα τοῦ δέρματος, καὶ ἡ θρὶξ αὐτῆς μετέβαλεν εἰς λευκὴν, καὶ μιανεῖ αὐτὸν ὁ ἱερεὺς, ὅτι λέπρα ἐστίν· ἐν τῷ ἕλκει ἐξήνθησεν.
\vs{21}Ἐὰν δὲ ἴδῃ ὁ ἱερεὺς, καὶ ἰδοὺ οὐκ ἔστιν ἐν αὐτῷ θρὶξ λευκὴ, καὶ ταπεινὸν μὴ ᾖ ἀπὸ τοῦ δέρματος τοῦ χρωτὸς, καὶ αὐτὴ ᾖ ἀμαυρά, καὶ ἀφοριεῖ αὐτὸν ὁ ἱερεὺς ἑπτὰ ἡμέρας.
\vs{22}Ἐὰν δὲ διαχύσει διαχέηται ἐν τῷ δέρματι, καὶ μιανεῖ αὐτὸν ὁ ἱερεὺς, ἁφὴ λέπρας ἐστίν· ἐν τῷ ἕλκει ἐξήνθησεν·
\vs{23}Ἐὰν δὲ κατὰ χώραν μείνῃ τὸ τηλαύγημα καὶ μὴ διαχέηται, οὐλὴ τοῦ ἕλκους ἐστὶ, καὶ καθαριεῖ αὐτὸν ὁ ἱερεύς.

\vs{24}Καὶ σὰρξ ἐὰν γένηται ἐν τῷ δέρματι αὐτοῦ κατάκαυμα πυρὸς, καὶ γένηται ἐν τῷ δέρματι αὐτοῦ τὸ ὑγιασθὲν τοῦ κατακαύματος αὐγάζον τηλαυγὲς λευκὸν, ὑποπυῤῥίζον, ἢ ἔκλευκον·
\vs{25}Καὶ ὄψεται αὐτὸν ὁ ἱερεὺς, καὶ ἰδοὺ μετέβαλε θρὶξ λευκὴ εἰς τὸ αὐγάζον, καὶ ἡ ὄψις αὐτοῦ ταπεινὴ ἀπὸ τοῦ δέρματος, λέπρα ἐστίν· ἐν τῷ κατακαύματι ἐξήνθησε· καὶ μιανεῖ αὐτὸν ὁ ἱερεὺς, ἁφὴ λέπρας ἐστίν.
\vs{26}Ἐὰν δὲ ἴδῃ ὁ ἱερεὺς, καὶ ἰδοὺ οὐκ ἔστιν ἐν τῷ· αὐγάζοντι θρὶξ λευκή, καὶ ταπεινὸν μὴ ᾖ ἀπὸ τοῦ δέρματος, αὐτὸ δὲ ἀμαυρὸν, καὶ ἀφοριεῖ αὐτὸν ὁ ἱερεὺς ἑπτὰ ἡμέρας.
\vs{27}Καὶ ὄψεται αὐτὸν ὁ ἱερεὺς τῇ ἡμέρᾳ τῇ ἑβδόμῃ· ἐὰν δὲ διαχύσει διαχέηται ἐν τῷ δέρματι, καὶ μιανεῖ αὐτὸν ὁ ἱερεὺς, ἁφὴ λέπρας ἐστίν· ἐν τῷ ἕλκει ἐξήνθησεν.
\vs{28}Ἐὰν δὲ κατὰ χώραν μείνῃ τὸ αὐγάζον, καὶ μὴ διαχυθῇ ἐν τῷ δέρματι, αὐτὴ δὲ ἀμαυρὰ ᾖ, οὐλὴ τοῦ κατακαύματός ἐστι, καὶ καθαριεῖ αὐτὸν ὁ ἱερεύς· ὁ γὰρ χαρακτὴρ τοῦ κατακαύματός ἐστι.

\vs{29}Καὶ ἀνδρὶ ἢ γυναικὶ ἐὰν γένηται ἐν αὐτοῖς ἁφὴ λέπρας ἐν τῇ κεφαλῇ ἢ ἐν τῷ πώγωνι·
\vs{30}Καὶ ὄψεται ὁ ἱερεὺς τὴν ἁφὴν, καὶ ἰδοὺ ἡ ὄψις αὐτῆς ἐγκοιλοτέρα τοῦ δέρματος, ἐν αὐτῇ δὲ θρὶξ ξανθίζουσα λεπτὴ, καὶ μιανεῖ αὐτὸν ὁ ἱερεύς· θραῦσμά ἐστι, λέπρα τῆς κεφαλῆς ἢ λέπρα τοῦ πώγωνός ἐστι.
\vs{31}Καὶ ἐὰν ἴδῃ ὁ ἱερεὺς τὴν ἁφὴν τοῦ θραύσματος, καὶ ἰδοὺ οὐχ ἡ ὄψις ἐγκοιλοτέρα τοῦ δέρματος, καὶ θρὶξ ξανθίζουσα οὐκ ἔστιν ἐν αὐτῇ, καὶ ἀφοριεῖ ὁ ἱερεὺς τὴν ἁφὴν τοῦ θραύσματος ἑπτὰ ἡμέρας.
\vs{32}Καὶ ὄψεται ὁ ἱερεὺς τὴν ἁφὴν τῇ ἡμέρᾳ τῇ ἑβδόμῃ, καὶ ἰδοὺ οὐ διεχύθη τὸ θραῦσμα, καὶ θρὶξ ξανθίζουσα οὐκ ἔστιν ἐν αὐτῇ, καὶ ἡ ὄψις τοῦ θραύσματος οὐκ ἔστι κοίλη ἀπὸ τοῦ δέρματος·
\vs{33}Καὶ ξυρηθήσεται τὸ δέρμα, τὸ δὲ θραῦσμα οὐ ξυρηθήσεται, καὶ ἀφοριεῖ ὁ ἱερεὺς τὸ θραῦσμα ἑπτὰ ἡμέρας τὸ δεύτερον.
\vs{34}Καὶ ὄψεται ὁ ἱερεὺς τὸ θραῦσμα τῇ ἡμέρᾳ τῇ ἑβδόμῃ, καὶ ἰδοὺ οὐ διεχύθη τὸ θραῦσμα ἐν τῷ δέρματι μετὰ τὸ ξυρηθῆναι αὐτόν, καὶ ἡ ὄψις τοῦ θραύσματος οὐκ ἔστιν κοίλη ἀπὸ τοῦ δέρματος, καὶ καθαριεῖ αὐτὸν ὁ ἱερεὺς, καὶ πλυνάμενος τὰ ἱμάτια, καθαρὸς ἔσται.
\vs{35}Ἐὰν δὲ διαχύσει διαχέηται τὸ θραῦσμα ἐν τῷ δέρματι μετὰ τὸ καθαρισθῆναι αὐτόν·
\vs{36}Καὶ ὄψεται ὁ ἱερεὺς, καὶ ἰδοὺ διακέχυται τὸ θραῦσμα ἐν τῷ δέρματι, οὐκ ἐπισκέψεται ὁ ἱερεὺς περὶ τῆς τριχὸς τῆς ξανθῆς, ὅτι ἀκάθαρτός ἐστιν.
\vs{37}Ἐὰν δὲ ἐνώπιον μείνῃ ἐπὶ χώρας τὸ θραῦσμα, καὶ θρὶξ μέλαινα ἀνατείλῃ ἐν αὐτῷ, ὑγίακε τὸ θραῦσμα, καθαρός ἐστι, καὶ καθαριεῖ αὐτὸν ὁ ἱερεύς.
\vs{38}Καὶ ἀνδρὶ ἢ γυναικὶ ἐὰν γένηται ἐν δέρματι τῆς σαρκὸς αὐτοῦ αὐγάματα αὐγάζοντα λευκανθίζοντα·
\vs{39}καὶ ὄψεται ὁ ἱερεὺς, καὶ ἰδοὺ ἐν δέρματι τῆς σαρκὸς αὐτοῦ αὐγάσματα αὐγάζοντα λευκαθίζοντα, ἀλφός ἐστιν· ἐξανθεῖ ἐν τῷ δέρματι τῆς σαρκὸς αὐτοῦ, καθαρός ἐστιν.
\vs{40}Ἐὰν δέ τινι μαδήσῃ ἡ κεφαλὴ αὐτοῦ, φαλακρός ἐστι, καθαρός ἐστιν.
\vs{41}Ἐὰν δὲ κατὰ πρόσωπον μαδήσῃ ἡ κεφαλὴ αὐτοῦ, ἀναφάλαντός ἐστι, καθαρός ἐστιν.
\vs{42}Ἐὰν δὲ γένηται ἐν τῷ φαλακρώματι αὐτοῦ ἢ ἐν τῷ ἀναφαλαντώματι αὐτοῦ ἁφὴ λευκὴ ἢ πυῤῥίζουσα, λέπρα ἐστὶν ἐν τῷ φαλακρώματι αὐτοὺ, ἢ ἐν τῷ ἀναφαλαντώματι αὐτοῦ·
\vs{43}καὶ ὄψεται αὐτὸν ὁ ἱερεύς, καὶ ἰδοὺ ἡ ὄψις τῆς ἁφῆς λευκὴ ἢ πυῤῥίζουσα ἐν τῷ φαλακρώματι αὐτοῦ ἢ ἐν τῷ ἀναφαλαντώματι αὐτοῦ, ὡς ωἶδος λέπρας ἐν δέρματι τῆς σαρκὸς αὐτοῦ·
\vs{44}Ἄνθρωπος λεπρός ἐστι· μιάνσει μιανεῖ αὐτὸν ὁ ἱερεὺς, ἐν τῇ κεφαλῇ αὐτοῦ ἡ ἁφὴ αὐτοῦ.
\vs{45}Καὶ ὁ λεπρὸς ἐν ᾧ ἐστιν ἡ ἁφὴ, τὰ ἱμάτια αὐτοῦ ἔστω παραλελυμένα, καὶ ἡ κεφαλὴ αὐτοῦ ἀκάλυπτος, καὶ περὶ τὸ στόμα αὐτοῦ περιβαλέσθω, καὶ ἀκάθαρτος κεκλήσεται.
\vs{46}Πάσας τὰς ἡμέρας, ὅσας ἐὰν ᾖ ἐπʼ αὐτὸν ἡ ἁφὴ, ἀκάθαρτος ὢν ἀκάθαρτος ἔσται· κεχωρισμένος καθήσεται, ἔξω τῆς παρεμβολῆς αὐτοῦ ἔσται ἡ διατριβή.

\vs{47}Καὶ ἱματίῳ ἐὰν γένηται ἁφὴ ἐν αὐτῷ λέπρας, ἐν ἱματίῳ ἐρέῳ, ἢ ἐν ἱματίῳ στυππυίνῳ,
\vs{48}ἢ ἐν στήμονι, ἢ ἐν κρόκῃ, ἢ ἐν τοῖς λινοῖς, ἢ ἐν τοῖς ἐρέοις, ἢ ἐν δέρματι, ἢ ἐν παντὶ ἐργασίμῳ δέρματι,
\vs{49}καὶ γένηται ἡ ἁφὴ χλωρίζουσα ἢ πυῤῥίζουσα ἐν τῷ δέρματι, ἢ ἐν τῷ ἱματίῳ, ἢ ἐν τῷ στήμονι, ἢ ἐν τῇ κρόκῃ, ἢ ἐν παντὶ σκεύει ἐργασίμῳ δέρματος, ἁφὴ λέπρας ἐστί· καὶ δείξει τῷ ἱερεῖ·
\vs{50}Καὶ ὄψεται ὁ ἱερεὺς τὴν ἁφήν, καὶ ἀφοριεῖ ὁ ἱερεὺς τὴν ἁφὴν ἑπτὰ ἡμέρας.
\vs{51}Καὶ ὄψεται ὁ ἱερεὺς τὴν ἁφὴν τῇ ἡμέρᾳ τῇ ἑβδόμῃ· ἐὰν δὲ διαχέηται ἡ ἁφὴ ἐν τῷ ἱματίῳ, ἢ ἐν τῷ στήμονι, ἢ ἐν τῇ κρόκῃ, ἢ ἐν τῷ δέρματι, κατὰ πάντα ὅσα ἐὰν ποιηθῇ δέρματα ἐν τῇ ἐργασίᾳ, λέπρα ἔμμονός ἐστιν ἡ ἁφὴ, ἀκάθαρτός ἐστι.
\vs{52}Κατακαύσει τὸ ἱμάτιον, ἢ τὸν στήμονα, ἢ τὴν κρόκην ἐν τοῖς ἐρέοις, ἢ ἐν τοῖς λινοῖς, ἢ ἐν παντὶ σκεύει δερματίνῳ, ἐν ᾧ ἂν ᾖ ἐν αὐτῷ ἡ ἁφὴ, ὅτι λέπρα ἔμμονός ἐστιν, ἐν πυρὶ κατακαυθήσεται.

\vs{53}Ἐὰν δὲ ἴδῃ ὁ ἱερεὺς, καὶ μὴ διαχέηται ἡ ἁφὴ ἐν τῷ ἱματίῳ, ἢ ἐν τῷ στήμονι, ἢ ἐν τῇ κρόκῃ, ἢ ἐν παντὶ σκεύει δερματίνῳ·
\vs{54}Καὶ συντάξει ὁ ἱερεύς, καὶ πλυνεῖ ἐφʼ οὗ ἐὰν ᾖ ἐπʼ αὐτοῦ ἡ ἁφὴ, καὶ ἀφοριεῖ ὁ ἱερεὺς τὴν ἁφὴν ἑπτὰ ἡμέρας τοδέυτερον.
\vs{55}Καὶ ὄψεται ὁ ἱερεὺς μετὰ τὸ πλυθῆναι αὐτὸ τὴν ἁφὴν, καὶ ἥδε οὐ μὴ μετέβαλεν ἡ ἁφὴ τὴν ὄψιν, καὶ ἡ ἁφὴ οὐ διαχεῖται, ἀκάθαρτόν ἐστιν, ἐν πυρὶ κατακαυθήσεται· ἐστήρικται ἐν τῷ ἱματίῳ, ἢ ἐν τῷ στήμονι, ἢ ἐν τῇ κρόκη.
\vs{56}Καὶ ἐὰν ἴδῃ ὁ ἱερεὺς, καὶ ᾖ ἀμαυρὰ ἡ ἁφὴ μετὰ τὸ πλυθήναι αὐτὸ, ἀποῤῥήξει αὐτὸ ἀπὸ τοῦ ἱματίου, ἢ ἀπὸ τοῦ στήμονος, ἢ ἀπὸ τῆς κρόκης, ἢ ἀπὸ τοῦ δέρματος.
\vs{57}Ἐὰν δὲ ὀφθῇ ἔτι ἐν τῷ ἱματίῳ, ἢ ἐν τῷ στήμονι, ἢ ἐν τῇ κρόκῃ, ἢ ἐν παντὶ σκεύει δερματίνῳ, λέπρα ἐξανθοῦσά ἐστιν ἐν πυρὶ κατακαυθήσεται ἐν ᾧ ἐστιν ἡ ἁφή.
\vs{58}Καὶ τὸ ἱμάτιον, ἢ ὁ στήμων, ἢ ἡ κρόκη, ἢ πᾶν σκεῦος δερμάτινον, ὃ πλυθήσεται, καὶ ἀποστήσεται ἀπʼ αὐτοῦ ἡ ἁφὴ, καὶ πλυθήσεται τὸ δέυτερον, καὶ καθαρὸν ἔσται.
\vs{59}Οὗτος ὁ νόμος ἁφῆς λέπρας ἱματίου ἐρέου, ἢ στυππυίνου, ἢ στήμονος, ἢ κρόκης, ἢ παντὸς σκεύους δερματίνου, εἰς τὸ καθαρίσαι αὐτὸ, ἢ μιᾶναι αὐτό.

\ch{14}
Καὶ ἐλάλησε Κύριος πρὸς Μωυσῆν, λέγων,
\vs{2}οὗτος ὁ νὁμος τοῦ λεπροῦ· ᾗ ἂν ἡμέρᾳ καθαρισθῇ, καὶ προσαχθήσεται πρὸς τὸν ἱερέα.
\vs{3}Καὶ ἐξελεύσεται ὁ ἱερεὺς ἔξω τῆς παρεμβολῆς, καὶ ὄψεται ὁ ἱερεὺς, καὶ ἰδοὺ ἰᾶται ἡ ἁφὴ τῆς λέπρας ἀπὸ τοῦ λεπροῦ.
\vs{4}Καὶ προστάξει ὁ ἱερεὺς, καὶ λήψονται τῷ κεκαθαρισμένῳ δύο ὀρνίθια ζῶντα καθαρὰ, καὶ ξύλον κέδρινον, καὶ κεκλωσμένον κόκκινον, καὶ ὕσσωπον.
\vs{5}Καὶ προστάξει ὁ ἱερεὺς, καὶ σφάξουσι τὸ ὀρνίθιον τὸ ἓν εἰς ἀγγεῖον ὀστράκινον ἐφʼ ὕδατι ζῶντι.
\vs{6}Καὶ τὸ ὀρνίθιον τὸ ζῶν λήμψεται αὐτὸ, καὶ τὸ ξύλον τὸ κέδρινον, καὶ τὸ κλωστὸν κόκκινον, καὶ τὸν ὕσσωπον, καὶ βάψει αὐτὰ καὶ τὸ ὀρνίθιον τὸ ζῶν εἰς τὸ αἷμα τοῦ ὀρνιθίου τοῦ σφαγέντος ἐφʼ ὕδατι ζῶντι.
\vs{7}Καὶ περιῤῥανεῖ ἐπὶ τὸν καθαρισθέντα ἀπὸ τῆς λέπρας ἑπτάκις, καὶ καθαρὸς ἔσται· καὶ ἐξαποστελεῖ τὸ ὀρνίθιον τὸ ζῶν εἰς τὸ πεδίον.
\vs{8}Καὶ πλυνεῖ ὁ καθαρισθεὶς τὰ ἱμάτια αὐτοῦ, καὶ ξυρηθήσεται αὐτοῦ πᾶσαν τὴν τρίχα, καὶ λούσεται ἐν ὕδατι, καὶ καθαρὸς ἔσται· καὶ μετὰ ταῦτα εἰσελεύσεται εἰς τὴν παρεμβολὴν, καὶ διατρίψει ἔξω τοῦ οἴκου αὐτοῦ ἑπτὰ ἡμέρας.
\vs{9}Καὶ ἔσται τῇ ἡμέρᾳ τῇ ἑβδόμῃ, ξυρηθήσεται πᾶσαν τὴν τρίχα αὐτοῦ, τὴν κεφαλὴν αὐτοῦ, καὶ τὸν πώγωνα, καὶ τὰς ὀφρῦς, καὶ πᾶσαν τὴς τρίχα αὐτοῦ ξυρηθήσεται· καὶ πλυνεῖ τὰ ἱμάτια, καὶ λούσεται τὸ σῶμα αὐτοῦ ὕδατι, καὶ καθαρὸς ἔσται.
\vs{10}Καὶ τῇ ἡμέρᾳ τῇ ὀγδόῃ λήψεται δύο ἀμνοὺς ἀμώμους ἑνιαυσίους, καὶ πρόβατον ἄμωμον ἑνιαύσιον, καὶ τρία δέκατα σεμιδάλεως εἰς θυσίαν πεφυραμένης ἐν ἐλαίῳ, καὶ κοτύλην ἐλαίου μίαν.
\vs{11}Καὶ στήσει ὁ ἱερεὺς ὁ καθαρίζων, τὸν ἄνθρωπον τὸν καθαριζόμενον, καὶ ταῦτα ἔναντι Κυρίου, ἐπὶ τὴν θύραν τῆς σκηνῆς τοῦ μαρτυρίου.
\vs{12}Καὶ λήψεται ὁ ἱερεὺς τὸν ἀμνὸν τὸν ἕνα, καὶ προσάξει αὐτὸν τῆς πλημμελείας, καὶ τὴν κοτύλην τοῦ ἐλαίου, καὶ ἀφοριεῖ αὐτὰ ἀφόρισμα ἔναντι Κυρίου.
\vs{13}Καὶ σφάξουσι τὸν ἀμνὸν ἐν τόπῳ, οὗ σφάζουσι τὰ ὁλοκαυτώματα, καὶ τὰ περὶ ἁμαρτίας, ἐν τόπῳ ἁγίῳ· ἔστι γὰρ τὸ περὶ ἁμαρτίας, ὥσπερ τὸ τῆς πλημμελείας ἐστὶ τῷ ἱερει· ἅγια ἁγίων ἐστί.
\vs{14}Καὶ λήψεται ὁ ἱερεὺς ἀπὸ τοῦ αἵματος τοῦ τῆς πλημμελείας, καὶ ἐπιθήσει ὁ ἱερεὺς ἐπὶ τὸν λοβὸν τοῦ ὠτὸς τοῦ καθαριζομένου τοῦ δεξιοῦ, καὶ ἐπὶ τὸ ἄκρον τῆς χειρὸς τῆς δεξιᾶς, καὶ ἐπὶ τὸ ἄκρον τοῦ ποδὸς τοῦ δεξιοῦ.
\vs{15}Καὶ λαβὼν ὁ ἱερεὺς ἀπὸ τῆς κοτύλης τοῦ ἐλαίου, ἐπιχεεῖ ἐπὶ τὴν χεῖρα τοῦ ἱερέως τὴν ἀριστεράν.
\vs{16}Καὶ βάψει τὸν δάκτυλον τὸν δεξιὸν ἀπὸ τοὺ ἐλαίου τοῦ ὄντος ἐπὶ τῆς χειρὸς αὐτοῦ τῆς ἀριστερᾶς· καὶ ῥανεῖ τῷ δακτύλῳ ἑπτάκις ἔναντι Κυρίου.
\vs{17}Τὸ δὲ καταλειφθὲν ἔλαιον τὸ ὂν ἐν τῇ χειρὶ, ἐπιθήσει ὁ ἱερεὺς ἐπὶ τὸν λοβὸν τοῦ ὠτὸς τοῦ καθαριζομένου τοῦ δεξιοῦ, καὶ ἐπὶ τὸ ἄκρον τῆς χειρὸς τῆς δεξιᾶς, καὶ ἐπὶ τὸ ἄκρον τοῦ ποδὸς τοῦ δεξιοῦ, ἐπὶ τὸν τόπον τοῦ αἵματος τοῦ τῆς πλημμελείας.
\vs{18}Τὸ δὲ καταλειφθὲν ἔλαιον τὸ ἐπὶ τῆς χειρὸς τοῦ ἱερέως, ἐπιθήσει ὁ ἱερεὺς ἐπὶ τὴν κεφαλὴν τοῦ καθαρισθέντος· καὶ ἐξιλάσεται περὶ αὐτοῦ ὁ ἱερεὺς ἔναντι Κυρίου.
\vs{19}Καὶ ποιήσει ὁ ἱερεὺς τὸ περὶ τῆς ἁμαρτίας, καὶ ἐξιλάσεται ὁ ἱερεὺς περὶ τοῦ καθαριζομένου ἀπὸ τῆς ἁμαρτίας αὐτοῦ· καὶ μετὰ τοῦτο σφάξει ὁ ἱερεὺς τὸ ὁλοκαύτωμα.
\vs{20}Καὶ ἀνοίσει ὁ ἱερεὺς τὸ ὁλοκαύτωμα, καὶ τὴν θυσίαν ἐπὶ τὸ θυσιαστήριον ἔναντι κυρίου· καὶ ἐξιλάσεται περὶ αὐτοῦ ὁ ἱερεὺς, καὶ καθαρισθήσεται.
\vs{21}Ἐὰν δὲ πένηται, καὶ ἡ χεὶρ αὐτοῦ μὴ εὑρίσκῃ, λήψεται ἀμνὸν ἕνα εἰς ὃ ἐπλημμέλησεν εἰς ἀφαίρεμα, ὥστε ἐξιλάσασθαι περὶ αὐτοῦ, καὶ δέκατον σεμιδάλεως πεφυραμένης ἐν ἐλαίῳ εἰς θυσίαν, καὶ κοτύλην ἐλαίου μίαν,
\vs{22}καὶ δύο τρυγόνας, ἢ δύο νοσσοὺς περιστερῶν, ὅσα εὗρεν ἡ χεὶρ αὐτοῦ, καὶ ἔσται ἡ μία περὶ ἁμαρτίας, καὶ ἡ μία εἰς ὁλοκαύτωμα.
\vs{23}Καὶ προσοίσει αὐτὰ τῇ ἡμέρᾳ τῇ ὀγδόῃ, εἰς τὸ καθαρίσαι αὐτὸν, πρὸς τὸν ἱερέα, ἐπὶ τὴν θύραν τῆς σκηνῆς τοῦ μαρτυρίου ἔναντι Κυρίου.
\vs{24}Καὶ λαβὼν ὁ ἱερεὺς τὸν ἀμνὸν τῆς πλημμελείας, καὶ τὴν κοτύλην τοῦ ἐλαίου, ἐπιθήσει αὐτὰ ἐπίθεμα ἔναντι Κυρίου.
\vs{25}Καὶ σφάξει τὸν ἀμνὸν τὸν τῆς πλημμελείας, καὶ λήψεται ὁ ἱερεὺς ἀπὸ τοῦ αἵματος τοῦ τῆς πλημμελείας, καὶ ἐπιθήσει ἐπὶ τὸν λοβὸν τοῦ ὠτὸς τοῦ καθαριζομένου τοῦ δεξιοῦ, καὶ ἐπὶ τὸ ἄκρον τῆς χειρὸς τῆς δεξιᾶς, καὶ ἐπὶ τὸ ἄκρον τοῦ ποδὸς τοῦ δεξιοῦ.
\vs{26}Καὶ ἀπὸ τοῦ ἐλαίου ἐπιχεεῖ ὁ ἱερεὺς ἐπὶ τὴν χεῖρα τοῦ ἱερέως τὴν ἀριστεράν.
\vs{27}Καὶ ῥανεῖ ὁ ἱερεὺς τῷ δακτύλῳ τῷ δεξιῷ ἀπὸ τοῦ ἐλαίου τοῦ ἐν τῇ χειρὶ αὐτοῦ τῇ ἀριστερᾷ ἑπτάκις ἔναντι Κυρίου.
\vs{28}Καὶ ἐπιθήσει ὁ ἱερεὺς ἀπὸ τοῦ ἐλαίου τοῦ ἐπὶ τῆς χειρὸς αὐτοῦ ἐπὶ τὸν λοβὸν τοῦ ὠτὸς τοῦ καθαριζομένου τοῦ δεξιοῦ, καὶ ἐπὶ τὸ ἄκρον τῆς χειρὸς αὐτοῦ τῆς δεξιᾶς, καὶ ἐπὶ τὸ ἄκρον τοῦ ποδὸς αὐτοῦ τοῦ δεξιοῦ, ἐπὶ τὸν τόπον τοῦ αἵματος τοῦ τῆς πλημμελείας.
\vs{29}Τὸ δὲ καταλειφθὲν ἀπὸ τοῦ ἐλαίου τὸ ὂν ἐπὶ τῆς χειρὸς τοῦ ἱερέως, ἐπιθήσει ἐπὶ τὴν κεφαλὴν τοῦ καθαρισθέντος· καὶ ἐξιλάσεται περὶ αὐτοῦ ὁ ἱερεὺς ἔναντι Κυρίου.

\vs{30}Καὶ ποιήσει μίαν ἀπὸ τῶν τρυγόνων ἢ ἀπὸ τῶν νοσσῶν τῶν περιστερῶν, καθότι εὗρεν αὐτοῦ ἡ χεὶρ,
\vs{31}τὴν μίαν περὶ ἁμαρτίας, καὶ τὴν μίαν εἰς ὁλοκαύτωμα σὺν τῇ θυσίᾳ· καὶ ἐξιλάσεται ὁ ἱερεὺς περὶ τοῦ καθαριζομένου ἔναντι Κυρίου.
\vs{32}Οὗτος ὁ νόμος ἐν ᾧ ἐστιν ἡ ἁφὴ τῆς λέπρας, καὶ τοῦ μὴ εὑρίσκοντος τῇ χειρὶ εἰς τὸν καθαρισμὸν αὐτοῦ.

\vs{33}Καὶ ἐλάλησε Κύριος πρὸς Μωυσῆν καὶ Ἀαρὼν, λέγων,
\vs{34}ὡς ἂν εἰσέλθητε εἰς τὴν γῆν τῶν Χαναναίων, ἣν ἐγὼ δίδωμι ὑμῖν ἐν κτήσει, καὶ δώσω ἁφὴν λέπρας ἐν ταῖς οἰκίαις τῆς γῆς τῆς ἐγκτήτου ὑμῖν·
\vs{35}καὶ ἥξει τίνος αὐτοῦ ἡ οἰκία, καὶ ἀναγγελεῖ τῷ ἱερεῖ, λέγων, ὥσπερ ἁφὴ ἑώραταί μοι ἐν τῇ οἰκίᾳ.
\vs{36}Καὶ προστάξει ὁ ἱερεὺς ἀποσκευάσαι τὴν οἰκίαν, πρὸ τοῦ εἰσελθόντα τὸν ἱερέα ἰδεῖν τὴν ἁφὴν, καὶ οὐ μὴ ἀκάθαρτα γένηται ὅσα ἂν ᾖ ἐν τῇ οἰκίᾳ· καὶ μετὰ ταῦτα εἰσελεύσεται ὁ ἱερεὺς καταμαθεῖν τὴν οἰκίαν.
\vs{37}Καὶ ὄψεται τὴν ἁφὴν, καὶ ἰδοὺ ἡ ἁφὴ ἐν τοῖς τοίχοις τῆς οἰκίας, κοιλάδας χλωριζούσας, ἢ πυῤῥιζούσας, καὶ ἡ ὄψις αὐτῶν ταπεινοτέρα τῶν τοίχων.
\vs{38}Καὶ ἐξελθὼν ὁ ἱερεὺς ἐκ τῆς οἰκίας ἐπὶ τὴν θύραν τῆς οἰκίας, καὶ ἀφοριεῖ ὁ ἱερεὺς τὴν οἰκίαν ἑπτὰ ἡμέρας.
\vs{39}Καὶ ἐπανήξει ὁ ἱερεὺς τῇ ἡμέρᾳ τῇ ἑβδόμῃ, καὶ ὄψεται τὴν οἰκίαν, καὶ ἰδοὺ διεχύθη ἡ ἁφὴ ἐν τοῖς τοίχοις τῆς οἰκίας.
\vs{40}Καὶ προστάξει ὁ ἱερεὺς, καὶ ἐξελοῦσι τοὺς λίθους ἐν οἷς ἐστιν ἡ ἁφὴ, καὶ ἐκβαλοῦσιν αὐτοὺς ἔξω τῆς πόλεως εἰς τόπον ἀκάθαρτον.
\vs{41}Καὶ τὴν οἰκίαν ἀποξύσουσιν ἔσωθεν κύκλῳ, καὶ ἐκχεοῦσι τὸν χοῦν τὸν ἀπεξυσμένον ἔξω τῆς πόλεως εἰς τόπον ἀκάθαρτον.
\vs{42}Καὶ λήψονται λίθους ἀπεξυσμένους ἑτέρους, καὶ ἀντιθήσουσιν ἀντὶ τῶν λίθων· καὶ χοῦν ἕτερον λήψονται, καὶ ἐξαλείψουσι τὴν οἰκίαν.
\vs{43}Ἐὰν δὲ ἐπέλθῃ πάλιν ἡ ἁφὴ, καὶ ἀνατείλῃ ἐν τῇ οἰκίᾳ μετὰ τὸ ἐξελεῖν τοὺς λίθους, καὶ μετὰ τὸ ἀποξυσθῆναι τὴν οἰκίαν, καὶ μετὰ τὸ ἐξαλειφθῆναι,
\vs{44}καὶ εἰσελεύσεται ὁ ἱερεὺς, καὶ ὄψεται εἰ διακέχυται ἡ ἁφὴ ἐν τῇ οἰκίᾳ, λέπρα ἔμμονός ἐστιν ἐν τῇ οἰκίᾳ, ἀκάθαρτός ἐστι.
\vs{45}Καὶ καθελοῦσι τὴν οἰκίαν, καὶ τὰ ξύλα αὐτῆς, καὶ τοὺς λίθους αὐτῆς, καὶ πάντα τὸν χοῦν ἐξοίσουσιν ἔξω τῆς πόλεως εἰς τόπον ἀκάθαρτον.
\vs{46}Καὶ ὁ εἰσπορευόμενος εἰς τὴν οἰκίαν πάσας τὰς ἡμέρας, ἃς ἀφωρισμένη ἐστὶν, ἀκάθαρτος ἔσται ἕως ἑσπέρας·
\vs{47}Καὶ ὁ κοιμώμενος ἐν τῇ οἰκίᾳ, πλυνεῖ τὰ ἱμάτια αὐτοῦ, καὶ ἀκάθαρτος ἔσται ἕως ἑσπέρας· καὶ ὁ ἔσθων ἐν τῇ οἰκίᾳ, πλυνεῖ τὰ ἱμάτια αὐτοῦ, καὶ ἀκάθαρτος ἔσται ἕως ἑσπέρας.

\vs{48}Ἐὰν δὲ παραγενόμενος εἰσέλθῃ ὁ ἱερεὺς καὶ ἴδῃ, καὶ ἰδοὺ οὐ διαχύσει οὐ διαχεῖται ἡ ἁφὴ ἐν τῇ οἰκίᾳ μετὰ τὸ ἐξαλειφθῆναι τὴν οἰκίαν, καὶ καθαριεῖ ὁ ἱερεὺς τὴν οἰκίαν, ὅτι ἰάθη ἡ ἁφή.
\vs{49}Καὶ λήψεται ἀφαγνίσαι τὴν οἰκίαν, δύο ὀρνίθια ζῶντα καθαρὰ, καὶ ξύλον κέδρινον, καὶ κεκλωσμένον κόκκινον, καὶ ὕσσωπον.
\vs{50}Καὶ σφάξει τὸ ὀρνίθιον τὸ ἓν εἰς σκεῦος ὀστράκινον ἐφʼ ὕδατι ζῶντι·
\vs{51}Καὶ λήψεται τὸ ξύλον τὸ κέδρινον, καὶ τὸ κεκλωσμένον κόκκινον, καὶ τὸν ὕσσωπον, καὶ τὸ ὀρνίθιον τὸ ζῶν· καὶ βάψει αὐτὸ εἰς τὸ αἷμα τοῦ ὀρνιθίου τοῦ ἐσφαγμενου ἐφʼ ὕδατι ζῶντι· καὶ περιῤῥανεῖ ἐν αὐτοῖς ἐπὶ τὴν οἰκίαν ἑπτάκις.
\vs{52}Καὶ ἀφαγνιεῖ τὴν οἰκίαν ἐν τῷ αἵματι τοῦ ὀρνιθίου, καὶ ἐν τῷ ὕδατι τῷ ζῶντι, καὶ ἐν τῷ ὀρνιθίῳ τῷ ζῶντι, καὶ ἐν τῷ ξύλῳ τῷ κεδρίνῳ, καὶ ἐν τῷ ὑσσώπῳ, καὶ ἐν τῷ κεκλωσμένῳ κοκκίνῳ.
\vs{53}Καὶ ἐξαποστελεῖ τὸ ὀρνίθιον τὸ ζῶν ἔξω τῆς πόλεως εἰς τὸ πεδίον, καὶ ἐξιλάσεται περὶ τῆς οἰκίας, καὶ καθαρὰ ἔσται.
\vs{54}Οὗτος ὁ νόμος κατὰ πᾶσαν ἁφὴν λέπρας, καὶ θραύσματος,
\vs{55}καὶ τῆς λέπρας ἱματίου, καὶ οἰκίας,
\vs{56}καὶ οὐλῆς, καὶ σημασίας, καὶ τοῦ αὐγάζοντος,
\vs{57}καὶ τοῦ ἐξηγήσασθαι ᾗ ἡμέρᾳ ἀκάθαρτον, καὶ ᾗ ἡμέρᾳ καθαρισθήσεται· οὗτος ὁ νόμος τῆς λέπρας.

\ch{15}
Καὶ ἐλάλησε Κύριος πρὸς Μωυσῆν καὶ Ἀαρὼν, λέγων,
\vs{2}λάλησον τοῖς υἱοῖς Ἰσραὴλ, καὶ ἐρεῖς αὐτοῖς, ἀνδρὶ ἀνδρὶ ᾧ ἐὰν γένηται ῥύσις ἐκ τοῦ σώματος αὐτοῦ, ἡ ῥύσις αὐτοῦ ἀκάθαρτός ἐστι.
\vs{3}Καὶ οὗτος ὁ νόμος τῆς ἀκαθαρσίας αὐτοῦ· ῥέων γόνον ἐκ σώματος αὐτοῦ, ἐκ τῆς ῥύσεως, ἧς συνέστηκε τὸ σῶμα αὐτοῦ διὰ τῆς ῥύσεως, αὕτη ἡ ἀκαθαρσία αὐτοῦ ἐν αὐτῷ· πᾶσαι αἱ ἡμέραι ῥύσεως σώματος αὐτοῦ, ᾗ συνέστηκε τὸ σῶμα αὐτοῦ διὰ τῆς ῥύσεως, ἀκαθαρσία αὐτοῦ ἐστι.
\vs{4}Πᾶσα κοίτη ἐφʼ ἧς ἂν κοιμηθῇ ἐπʼ αὐτῆς ὁ γονοῤῥυὴς, ἀκάθαρτός ἐστι, καὶ πᾶν σκεῦος ἐφʼ ὃ ἂν καθίσῃ ἐπʼ αὐτὸ ὁ γονοῤῤυὴς, ἀκάθαρτον ἔσται.
\vs{5}Καὶ ἄνθρωπος, ὃς ἐὰν ἅψηται τῆς κοίτης αὐτοῦ, πλυνεῖ τὰ ἱμάτια αὐτοῦ, καὶ λούσεται ὕδατι, καὶ ἀκάθαρτος ἔσται ἕως ἑσπέρας.
\vs{6}Καὶ ὁ καθήμενος ἐπὶ τοῦ σκεύους ἐφʼ ὃ ἂν καθίσῃ ὁ γονοῤῥυὴς, πλυνεῖ τὰ ἱμάτια αὐτοῦ, καὶ λούσεται ὕδατι, καὶ ἀκάθαρτος ἔσται ἕως ἑσπέρας.
\vs{7}Καὶ ὁ ἁπτόμενος τοῦ χρωτὸς τοῦ γονοῤῥυοῦς, πλυνεῖ τὰ ἱμάτια, καὶ λούσεται ὕδατι, καὶ ἀκάθαρτος ἔσται ἕως ἑσπέρας.
\vs{8}Ἐὰν δὲ προσσιελίσῃ ὁ γονοῤῥυὴς ἐπὶ τὸν καθαρὸν, πλυνεῖ τὰ ἱμάτια αὐτοῦ, καὶ λούσεται ὕδατι, καὶ ἀκάθαρτος ἔσται ἕως ἑσπέρας.
\vs{9}Καὶ πᾶν ἐπίσαγμα ὄνου, ἐφʼ ὃ ἂν ἐπιβῇ ἐπʼ αὐτὸ ὁ γονοῤῥυὴς, ἀκάθαρτον ἔσται ἕως ἑσπέρας.
\vs{10}Καὶ πᾶς ὁ ἁπτόμενος ὅσα ἂν ᾖ ὑποκάτω αὐτοῦ, ἀκάθαρτος ἔσται ἕως ἑσπέρας· καὶ ὁ αἴρων αὐτὰ, πλυνεῖ τὰ ἱμάτια αὐτοῦ, καὶ λούσεται ὕδατι, καὶ ἀκάθαρτος ἔσται ἕως ἑσπέρας.
\vs{11}Καὶ ὅσων ἐὰν ἅψηται ὁ γονοῤῥυὴς, καὶ τὰς χεῖρας οὐ νένιπται ὕδατι, πλυνεῖ τὰ ἱμάτια, καὶ λούσεται τὸ σῶμα ὕδατι, καὶ ἀκάθαρτος ἔσται ἕως ἑσπέρας.
\vs{12}Καὶ σκεῦος ὀστράκινον οὗ ἂν ἅψηται ὁ γονοῤῥυὴς, συντριβήσεται· καὶ σκεῦος ξύλινον νιφήσεται ὕδατι, καὶ καθαρὸν ἔσται.
\vs{13}Ἐὰν δὲ καθαρισθῇ ὁ γονοῤῥυὴς ἐκ τῆς ῥύσεως αὐτοῦ, καὶ ἐξαριθμηθήσεται αὐτῷ ἑπτὰ ἡμέρας εἰς τὸν καθαρισμὸν αὐτοῦ, καὶ πλυνεῖ τὰ ἱμάτια αὐτοῦ, καὶ λούσεται τὸ σῶμα ὕδατι, καὶ καθαρὸς ἔσται.
\vs{14}Καὶ τῇ ἡμέρᾳ τῇ ὀγδόῃ λήψεται ἑαυτῷ δύο τρυγόνας, ἢ δύο νοσσοὺς περιστερῶν, καὶ οἴσει αὐτὰ ἔναντι Κυρίου ἐπὶ τὰς θύρας τῆς σκηνῆς τοῦ μαρτυρίου, καὶ δώσει αὐτὰ τῷ ἱερεῖ.
\vs{15}Καὶ ποιήσει αὐτὰ ὁ ἱερεὺς μίαν περὶ ἁμαρτίας, καὶ μίαν εἰς ὁλοκαύτωμα· καὶ ἐξιλάσεται περὶ αὐτοῦ ὁ ἱερεὺς ἔναντι Κυρίου ἀπὸ τῆς ῥύσεως αὐτοῦ.

\vs{16}Καὶ ἄνθρωπος ᾧ ἂν ἐξέλθῃ ἐξ αὐτοῦ κοίτη σπέρματος, καὶ λούσεται ὕδατι πᾶν τὸ σῶμα αὐτοῦ, καὶ ἀκάθαρτος ἔσται ἕως ἑσπέρας.
\vs{17}Καὶ πᾶν ἱμάτιον, καὶ πᾶν δέρμα ἐφʼ ὃ ἂν ᾖ ἐπʼ αὐτὸ κοίτη σπέρματος, καὶ πλυθήσεται ὕδατι, καὶ ἀκάθαρτον ἔσται ἕως ἑσπέρας.
\vs{18}Καὶ γυνὴ ἐὰν κοιμηθῇ ἀνὴρ μετʼ αὐτῆς κοίτην σπέρματος, καὶ λούσονται ὕδατι, καὶ ἀκάθαρτοι ἔσονται ἕως ἑσπέρας.
\vs{19}Καὶ γυνὴ ἥτις ἂν ᾖ ῥέουσα αἵματι, καὶ ἔσται ἡ ῥύσις αὐτῆς ἐν τῷ σώματι αὐτῆς, ἑπτὰ ἡμέρας ἔσται ἐν τῇ ἀφέδρῳ αὐτῆς· πᾶς ὁ ἁπτόμενος αὐτῆς, ἀκάθαρτος ἔσται ἕως ἑσπέρας.
\vs{20}Καὶ πᾶν ἐφʼ ὃ ἂν κοιτάζηται ἐπʼ αὐτὸ ἐν τῇ ἀφέδρῳ αὐτῆς, ἀκάθαρτον ἔσται· καὶ πᾶν ἐφʼ ὃ ἂν ἐπικαθίσῃ ἐπʼ αὐτὸ, ἀκάθαρτον ἔσται.
\vs{21}Καὶ πᾶς ὃς ἂν ἅψηται τῆς κοίτης αὐτῆς, πλυνεῖ τὰ ἱμάτια αὐτοῦ, καὶ λούσεται τὸ σῶμα αὐτοῦ ὕδατι, καὶ ἀκάθαρτος ἔσται ἕως ἑσπέρας.
\vs{22}Καὶ πᾶς ὁ ἁπτόμενος παντὸς σκεύους οὗ ἐὰν καθίσῃ ἐπʼ αὐτὸ, πλυνεῖ τὰ ἱμάτια αὐτοῦ, καὶ λούσεται ὕδατι, καὶ ἀκάθαρτος ἔσται ἕως ἑσπέρας.
\vs{23}Ἐὰν δὲ ἐν τῇ κοίτῃ αὐτῆς οὔσης, ἢ ἐπὶ τοῦ σκεύους οὗ ἐὰν καθίσῃ ἐπʼ αὐτῷ ἐν τῷ ἅπτεσθαι αὐτὸν αὐτῆς, ἀκάθαρτος ἔσται ἕως ἑσπέρας.

\vs{24}Ἐὰν δὲ κοίτῃ κοιμηθῇ τις μετʼ αὐτῆς, καὶ γένηται ἡ ἀκαθαρσία αὐτῆς ἐπʼ αὐτῷ, ἀκάθαρτος ἔσται ἑπτὰ ἡμέρας· καὶ πᾶσα κοίτη ἐφʼ ᾗ ἂν κοιμηθῇ ἐπʼ αὐτῇ, ἀκάθαρτος ἔσται.
\vs{25}Καὶ γυνὴ ἐὰν ῥέῃ ῥύσει αἵματος ἡμέρας πλείους, οὐκ ἐν καιρῷ τῆς ἀφέδρου αὐτῆς, ἐὰν καὶ ῥέῃ μετὰ τὴν ἄφεδρον αὐτῆς, πᾶσαι αἱ ἡμέραι ῥύσεως ἀκαθαρσίας αὐτῆς, καθάπερ αἱ ἡμέραι τῆς ἀφέδρου αὐτῆς, ἔσται ἀκάθαρτος.
\vs{26}Καὶ πᾶσα κοίτη ἐφʼ ἧς ἂν κοιμηθῇ ἐπʼ αὐτῆς πάσας τὰς ἡμέρας τῆς ῥύσεως, κατὰ τὴν κοίτην τῆς ἀφέδρου, ἔσται αὐτῇ· καὶ πᾶν σκεῦος ἐφʼ ὃ ἂν καθίσῃ ἐπʼ αὐτὸ, ἀκάθαρτον ἔσται κατὰ τὴν ἀκαθαρσίαν τῆς ἀφέδρου.
\vs{27}Πᾶς ὁ ἁπτόμενος αὐτῆς ἀκάθαρτος ἔσται, καὶ πλυνεῖ τὰ ἱμάτια καὶ λούσεται τὸ σῶμα ὕδατι, καὶ ἀκάθαρτος ἔσται ἕως ἑσπέρας.
\vs{28}Ἐὰν δὲ καθαρισθῇ ἀπὸ τῆς ῥύσεως, καὶ ἐξαριθμήσεται αὐτῇ ἑπτὰ ἡμέρας, καὶ μετὰ ταῦτα καθαρισθήσεται.
\vs{29}Καὶ τῇ ἡμέρᾳ τῇ ὀγδόῃ λήψεται αὑτῇ δύο τρυγόνας, ἢ δύο νοσσοὺς περιστερῶν, καὶ οἴσει αὐτὰ πρὸς τὸν ἱερέα ἐπὶ τὴν θύραν τῆς σκηνῆς τοῦ μαρτυρίου.
\vs{30}Καὶ ποιήσει ὁ ἱερεὺς τὴν μίαν περὶ ἁμαρτίας, καὶ τὴν μίαν εἰς ὁλοκαύτωμα· καὶ ἐξιλάσεται περὶ αὐτῆς ὁ ἱερεὺς ἔναντι Κυρίου ἀπὸ ῥύσεως ἀκαθαρσίας αὐτῆς.

\vs{31}Καὶ εὐλαβεῖς ποιήσετε τοὺς υἱοὺς Ἰσραὴλ ἀπὸ τῶν ἀκαθαρσιῶν αὐτῶν· καὶ οὐκ ἀποθανοῦνται διὰ τὴν ἀκαθαρσίαν αὐτῶν, ἐν τῷ μιαίνειν αὐτοὺς τὴν σκηνήν μου τὴν ἐν αὐτοῖς.
\vs{32}Οὗτος ὁ νόμος τοῦ γονοῤῥυοῦς· καὶ ἐάν τινι ἐξέλθῃ ἐξ αὐτοῦ κοίτη σπέρματος, ὥστε μιανθῆναι ἐν αὐτῇ,
\vs{33}καὶ τῇ αἱμοῤῥοούσῃ ἐν τῇ ἀφέδρῳ αὐτῆς, καὶ ὁ γονοῤῥυὴς ἐν τῇ ῥύσει αὐτοῦ τῷ ἄρσενι ἢ τῇ θηλείᾳ, καὶ τῷ ἀνδρὶ, ὃς ἂν κοιμηθῇ μετὰ ἀποκαθημένης.

\ch{16}
Καὶ ἐλάλησε Κύριος πρὸς Μωυσῆν, μετὰ τὸ τελευτῆσαι τοὺς δύο υἱοὺς Ἀαρὼν ἐν τῷ προσάγειν αὐτοὺς πῦρ ἀλλότριον ἔναντι Κυρίου, καὶ ἐτελεύτησαν.
\vs{2}Καὶ εἶπε Κύριος πρὸς Μωυσῆν, λάλησον πρὸς Ἀαρὼν τὸν ἀδελφόν σου, καὶ μὴ εἰσπορευέσθω πᾶσαν ὥραν εἰς τὸ ἅγιον ἐσώτερον τοῦ καταπετάσματος εἰς πρόσωπον τοῦ ἱλαστηρίου, ὅ ἐστιν ἐπὶ τῆς κιβωτοῦ τοῦ μαρτυρίου, καὶ οὐκ ἀποθανεῖται· ἐν γὰρ νεφέλῃ ὀφθήσομαι ἐπὶ τοῦ ἱλαστηρίου.
\vs{3}Οὕτως εἰσελεύσεται Ἀαρὼν εἰς τὸ ἅγιον· ἐν μόσχῳ ἐκ βοῶν περὶ ἁμαρτίας, καὶ κριὸν εἰς ὁλοκαύτωμα.
\vs{4}Καὶ χιτῶνα λινοῦν ἡγιασμένον ἐνδύσεται, καὶ περισκελὲς λινοῦν ἔσται ἐπὶ τοῦ χρωτὸς αὐτοῦ, καὶ ζώνῃ λινῇ ζώσεται, καὶ κίδαριν λινῆν περιθήσεται, ἱμάτια ἅγιά ἐστι· καὶ λούσεται ὕδατι πᾶν τὸ σῶμα αὐτοῦ καὶ ἐνδύσεται αὐτά.
\vs{5}Καὶ παρὰ τῆς συναγωγῆς τῶν υἱῶν Ἰσραὴλ λήψεται δύο χιμάρους ἐξ αἰγῶν περὶ ἁμαρτίας, καὶ κριὸν ἕνα εἰς ὁλοκαύτωμα.
\vs{6}Καὶ προσάξει Ἀαρὼν τὸν μόσχον τὸν περὶ τῆς ἁμαρτίας αὐτοῦ, καὶ ἐξιλάσεται περὶ αὐτοῦ, καὶ τοῦ οἴκου αὐτοῦ.
\vs{7}Καὶ λήψεται τοὺς δύο χιμάρους, καὶ στήσει αὐτοὺς ἔναντι Κυρίου παρὰ τὴν θύραν τῆς σκηνῆς τοῦ μαρτυρίου.
\vs{8}Καὶ ἐπιθήσει Ἀαρὼν ἐπὶ τοὺς δύο χιμάρους κλῆρους· κλῆρον ἕνα τῷ Κυρίῳ, καὶ κλῆρον ἕνα τῷ ἀποπομπαίῳ.
\vs{9}Καὶ προσάξει Ἀαρὼν τὸν χίμαρον ἐφʼ ὃν ἐπῆλθεν ἐπʼ αὐτὸν ὁ κλῆρος τῷ Κυρίῳ, καὶ προσοίσει περὶ ἁμαρτίας.
\vs{10}Καὶ τὸν χίμαρον, ἐφʼ ὃν ἐπῆλθεν ἐπʼ αὐτὸν ὁ κλῆρος τοῦ ἀποπομπαίου, στήσει αὐτὸν ζῶντα ἔναντι Κυρίου, τοῦ ἐξιλάσασθαι ἐπʼ αὐτοῦ, ὥστε ἀποστεῖλαι αὐτὸν εἰς τὴν ἀποπομπήν, καὶ ἀφήσει αὐτὸν εἰς τὴν ἔρημον.
\vs{11}Καὶ προσάξει Ἀαρὼν τὸν μόσχον τὸν περὶ τῆς ἁμαρτίας αὑτοῦ, καὶ ἐξιλάσεται περὶ ἑαυτοῦ, καὶ τοῦ οἴκου· καὶ σφάξει τὸν μόσχον περὶ τῆς ἁμαρτίας αὐτοῦ.
\vs{12}Καὶ λήψεται τὸ πυρεῖον πλῆρες ἀνθράκων πυρὸς ἀπὸ τοῦ θυσιαστηρίου, τοῦ ἀπέναντι Κυρίου· καὶ πλήσει τὰς χεῖρας θυμιάματος συνθέσεως λεπτῆς, καὶ εἰσοίσει ἐσώτερον τοῦ καταπετάσματος.
\vs{13}Καὶ ἐπιθήσει τὸ θυμίαμα ἐπὶ τὸ πῦρ ἔναντι Κυρίου· καὶ καλύψει ἡ ἀτμὶς τοῦ θυμιάματος τὸ ἱλαστήριον τὸ ἐπὶ τῶν μαρτυρίων, καὶ οὐκ ἀποθανεῖται.
\vs{14}Καὶ λήψεται ἀπὸ τοῦ αἵματος τοῦ μόσχου, καὶ ῥανεῖ τῷ δακτύλῳ ἐπὶ τὸ ἱλαστήριον κατὰ ἀνατολάς· κατὰ πρόσωπον τοῦ ἱλαστηρίου ῥανεῖ ἑπτάκις ἀπὸ τοῦ αἵματος τῷ δακτύλῳ.

\vs{15}Καὶ σφάξει τὸν χίμαρον τὸν περὶ ἁμαρτίας, τὸν περὶ τοῦ λαοῦ, ἔναντι Κυρίου· καὶ εἰσοίσει τοῦ αἵματος αὐτοῦ ἐσώτερον τοῦ καταπετάσματος, καὶ ποιήσει τὸ αἷμα αὐτοῦ, ὃν τρόπον ἐποίησε τὸ αἷμα τοῦ μόσχου· καὶ ῥανεῖ τὸ αἷμα αὐτοῦ ἐπὶ τὸ ἱλαστήριον, κατὰ πρόσωπον τοῦ ἱλαστηρίου.
\vs{16}Καὶ ἐξιλάσεται τὸ ἅγιον ἀπὸ τῶν ἀκαθαρσιῶν τῶν υἱῶν Ἰσραὴλ, καὶ ἀπὸ τῶν ἀδικημάτων αὐτῶν περὶ πασῶν τῶν ἁμαρτιῶς αὐτῶν· καὶ οὕτω ποιήσει τῇ σκηνῇ τοῦ μαρτυρίου τῇ ἐκτισμένῃ ἐν αὐτοῖς ἐν μέσῳ τῆς ἀκαθαρσίας αὐτῶν.
\vs{17}Καὶ πᾶς ἄνθρώπος οὐκ ἔσται ἐν τῇ σκηνῇ τοῦ μαρτυρίου, εἰσπορευομένου αὐτοῦ ἐξιλάσασθαι ἐν τῷ ἁγίῳ, ἕως ἂν ἐξέλθῃ· καὶ ἐξιλάσεται περὶ ἑαυτοῦ, καὶ τοῦ οἴκου αὐτοῦ, καὶ περὶ πάσης συναγωγῆς υἱῶν Ἰσραήλ.
\vs{18}Καὶ ἐξελεύσεται ἐπὶ τὸ θυσιαστήριον τὸ ὂν ἀπέναντι Κυρίου, καὶ ἐξιλάσεται ἐπʼ αὐτοῦ· καὶ λήψεται ἀπὸ τοῦ αἵματος τοῦ μόσχου, καὶ ἀπὸ τοῦ αἵματος τοῦ χιμάρου, καὶ ἐπιθήσει ἐπὶ τὰ κέρατα τοῦ θυσιαστηρίου κύκλῳ.
\vs{19}Καὶ ῥανεῖ ἐπʼ αὐτὸ ἀπὸ τοῦ αἵματος τῷ δακτύλῳ ἑπτάκις, καὶ καθαριεῖ αὐτό, καὶ ἁγιάσει αὐτὸ ἀπὸ τῶν ἀκαθαρσιῶν τῶν υἱῶν Ἰσραήλ.
\vs{20}Καὶ συντελέσει ἐξιλασκόμενος τὸ ἅγιον, καὶ τὴν σκηνὴν τοῦ μαρτυρίου, καὶ τὸ θυσιαστήριον, καὶ περὶ τῶν ἱερέων καθαριεῖ· καὶ προσάξει τὸν χίμαρον τὸν ζῶντα.
\vs{21}Καὶ ἐπιθήσει Ἀαρὼν τὰς χεῖρας αὐτοῦ ἐπὶ τὴν κεφαλὴν τοῦ χιμάρου τοῦ ζῶντος, καὶ ἐξαγορεύσει ἐπʼ αὐτοῦ πάσας τὰς ἀνομίας τῶν υἱῶν Ἰσραὴλ, καὶ πάσας τὰς ἀδικίας αὐτῶν, καὶ πάσας τὰς ἁμαρτίας αὐτῶν· καὶ ἐπιθήσει αὐτὰς ἐπὶ τὴν κεφαλὴν τοῦ χιμάρου τοῦ ζῶντος· καὶ ἐξαποστελεῖ ἐν χειρὶ ἀνθρώπου ἑτοιμου εἰς τὴν ἔρημον.
\vs{22}Καὶ λήψεται ὁ χίμαρος ἐφʼ ἑαυτῷ τὰς ἀδικίας αὐτῶν εἰς γῆν ἄβατον· καὶ ἐξαποστελεῖ τὸν χίμαρον εἰς τὴν ἔρημον.
\vs{23}Καὶ εἰσελεύσεται Ἀαρὼν εἰς τὴν σκηνὴν τοῦ μαρτυρίου, καὶ ἐκδύσεται τὴν στολὴν τὴν λινῆν, ἣν ἐνδεδύκει, εἰσπορευομένου αὐτοῦ εἰς τὸ ἅγιον, καὶ ἀποθήσει αὐτὴν ἐκεῖ.
\vs{24}Καὶ λούσεται τὸ σῶμα αὐτοῦ ὕδατι ἐν τόπῳ ἁγίῳ, καὶ ἐνδύσεται τὴν στολὴν αὐτοῦ, καὶ ἐξελθὼν ποιήσει τὸ ὁλοκαύτωμα αὐτοῦ καὶ τὸ ὁλοκάρπωμα τοῦ λαοῦ, καὶ ἐξιλάσεται περὶ αὐτοῦ, καὶ περὶ τοῦ οἴκου αὐτοῦ, καὶ περὶ τοῦ λαοῦ, ὡς περὶ τῶν ἱερέων.
\vs{25}Καὶ τὸ στέαρ τὸ περὶ τῶν ἁμαρτιῶν ἀνοίσει ἐπὶ τὸ θυσιαστήριον.

\vs{26}Καὶ ὁ ἐξαποστέλλων τὸν χίμαρον τὸν διεσταλμένον εἰς ἄφεσιν, πλυνεῖ τὰ ἱμάτια, καὶ λούσεται τὸ σῶμα αὐτοῦ ὕδατι, καὶ μετὰ ταῦτα εἰσελεύσεται εἰς τὴν παρεμβολήν.
\vs{27}Καὶ τὸν μόσχον τὸν περὶ τῆς ἁμαρτίας, καὶ τὸν χίμαρον τὸν περὶ τῆς ἁμαρτίας, ὧν τὸ αἷμα εἰσηνέχθη ἐξιλάσασθαι ἐν τῷ ἁγίῳ, ἐξοίσουσιν αὐτὰ ἔξω τῆς παρεμβολῆς, ταὶ κατακαύσουσιν αὐτὰ ἐν πυρὶ, καὶ τὰ δέρματα αὐτῶν καὶ τὰ κρέα αὐτῶν καὶ τὴν κόπρον αὐτῶν.
\vs{28}Ὁ δὲ κατακαίων αὐτὰ, πλυνεῖ τὰ ἱμάτια, καὶ λούσεται τὸ σῶμα αὐτοῦ ὕδατι, καὶ μετὰ ταῦτα εἰσελεύσεται εἰς τὴν παρεμβολήν.

\vs{29}Καὶ ἔσται τοῦτο ὑμῖν νόμιμον αἰώνιον· ἐν τῷ μηνὶ τῷ ἑβδόμῳ, δεκάτῃ τοῦ μηνὸς, ταπεινώσετε τὰς ψυχὰς ὑμῶν, καὶ πᾶν ἔργον οὐ ποιήσετε ὁ αὐτόχθων, καὶ ὁ προσήλυτος ὁ προσκείμενος ἐν ὑμῖν.
\vs{30}Ἐν γὰρ τῇ ἡμέρᾳ ταύτῃ ἐξιλάσεται περὶ ὑμῶν, καθαρίσαι ὑμᾶς ἀπὸ πασῶν τῶν ἁμαρτιῶν ὑμῶν ἔναντι Κυρίου, καὶ καθαρισθήσεσθε.
\vs{31}Σάββατα σαββάτων ἀνάπαυσις αὕτη ἔσται ὑμῖν· καὶ ταπεινώσετε τὰς ψυχὰς ὑμῶν, νόμιμον αἰώνιον.
\vs{32}Ἐξιλάσεται ὁ ἱερεὺς, ὃν ἂν χρίσωσιν αὐτὸν, καὶ ὃν ἂν τελειώσωσι τὰς χεῖρας αὐτοῦ ἱερατεύειν μετὰ τὸν πατέρα αὐτοῦ· καὶ ἐνδύσεται τὴν στολὴν τὴν λινῆν, στολὴν ἁγίαν.
\vs{33}Καὶ ἐξιλάσεται τὸ ἅγιον τοῦ ἁγίου, καὶ τὴν σκηνὴν τοῦ μαρτυρίου, καὶ τὸ θυσιαστήριον ἐξιλάσεται, καὶ περὶ τῶν ἱερέων, καὶ περὶ πάσης συναγωγῆς ἐξιλάσεται.
\vs{34}Καὶ ἔσται τοῦτο ὑμῖν νόμιμον αἰώνιον ἐξιλάσκεσθαι περὶ τῶν νἱῶν Ἰσραὴλ ἀπὸ πασῶν τῶν ἁμαρτιῶν αὐτῶν· ἅπαξ τοῦ ἐνιαυτοῦ ποιηθήσεται, καθὰ συνέταξε Κύριος τῷ Μωυσῇ.

\ch{17}
Καὶ ἐλάλησε Κύριος πρὸς Μωυσῆν, λέγων,
\vs{2}λάλησον πρὸς Ἀαρὼν καὶ πρὸς τοὺς υἱοὺς αὐτοῦ, καὶ πρὸς πάντας υἱοὺς Ἰσραὴλ, καὶ ἐρεῖς πρὸς αὐτοὺς, τοῦτο τὸ ῥῆμα ὃ ἐνετείλατο Κύριος, λέγων,
\vs{3}ἄνθρωπος ἄνθρωπος τῶν υἱῶν Ἰσραὴλ, ἢ τῶν προσηλύτων τῶν προσκειμένων ἐν ὑμῖν, ὃς ἐὰν σφάξῃ μόσχον, ἢ πρόβατον, ἢ αἶγα ἐν τῇ παρεμβολῇ, καὶ ὃς ἂν σφάξῃ ἔξω τῆς παρεμβολῆς,
\vs{4}καὶ ἐπὶ τὴν θύραν τῆς σκηνῆς τοῦ μαρτυρίου μὴ ἐνέγκῃ, ὥστε ποιῆσαι αὐτὸ εἰς ὁλοκαύτωμα ἢ σωτήριον Κυρίῳ δεκτὸν εἰς ὀσμὴν εὐωδίας· καὶ ὃς ἂν σφάξῃ ἔξω, καὶ ἐπὶ τὴν θύραν τῆς σκηνῆς τοῦ μαρτυρίου μὴ ἐνέγκῃ αὐτὸ, ὥστε προσενέγκαι δῶρον τῷ Κυρίῳ ἀπέναντι τῆς σκηνῆς Κυρίου· καὶ λογισθήσεται τῷ ἀνθρώπῳ ἐκείνῳ αἷμα· αὶμα ἐξέχεεν· ἐξολοθρευθήσεται ἡ ψυχὴ ἐκείνη ἐκ τοῦ λαοῦ αὐτῆς.
\vs{5}Ὅπως ἀναφέρωσιν οἱ υἱοὶ Ἰσραὴλ τὰς θυσίας αὐτῶν, ὅσας ἂν αὐτοὶ σφάξουσιν ἐν τοῖς πεδίοις, καὶ οἴσουσι τῷ Κυρίῳ ἐπὶ τὰς θύρας τῆς σκηνῆς τοῦ μαρτυρίου πρὸς τὸν ἱερέα· καὶ θύσουσι θυσίαν σωτηρίου τῷ Κυρίῳ αὐτά.
\vs{6}Καὶ προσχεεῖ ὁ ἱερεὺς τὸ αἷμα ἐπὶ τὸ θυσιαστήριον κύκλῳ ἀπέναντι Κυρίου παρὰ τὰς θύρας τῆς σκηνῆς τοῦ μαρτυρίου· καὶ ἀνοίσει τὸ στέαρ εἰς ὀσμὴν εὐωδίας Κυρίῳ.

\vs{7}Καὶ οὐ θύσουσιν ἔτι τὰς θυσίας αὐτῶν τοῖς ματαίοις, οἷς αὐτοὶ ἐκπορνεύουσιν ὀπίσω αὐτῶν· νόμιμον αἰώνιον ἔσται ὑμῖν εἰς τὰς γενεὰς ὑμῶν.
\vs{8}Καὶ ἐρεῖς πρὸς αὐτοὺς, ἄνθρωπος ἄνθρωπος τῶν υἱῶν Ἰσραὴλ, ἢ ἀπὸ τῶν υἱῶν τῶν προσηλύτων τῶν προσκειμένων ἐν ὑμῖν, ὃς ἂν ποιήσῃ ὁλοκαύτωμα ἢ θυσίαν,
\vs{9}καὶ ἐπὶ τὴν θύραν τῆς σκηνῆς τοῦ μαρτυρίου μὴ ἐνέγκῃ ποιῆσαι αὐτὸ τῷ Κυρίῳ, ἐξολοθρευθήσεται ὁ ἄνθρωπος ἐκεῖνος ἐκ τοῦ λαοῦ αὐτοῦ.
\vs{10}Καὶ ἄνθρωπος ἄνθρωπος τῶν υἱῶν Ἰσραὴλ, ἢ τῶν προσηλύτων τῶν προσκειμένων ἐν ὑμῖν, ὃς ἂν φάγῃ πᾶν αἷμα· καὶ ἐπιστήσω τὸ πρόσωπόν μου ἐπὶ τὴν ψυχὴν τὴν ἔσθουσαν τὸ αἷμα, καὶ ἀπολῶ αὐτὴν ἐκ τοῦ λαοῦ αὐτῆς.
\vs{11}Ἡ γὰρ ψυχὴ πάσης σαρκὸς αἷμα αὐτοῦ ἐστι· καὶ ἐγὼ δέδωκα αὐτὸ ὑμῖν ἐπὶ τοῦ θυσιαστηρίου ἐξιλάσκεσθαι περὶ τῶν ψυχῶν ὑμῶν· τὸ γὰρ αἷμα αὐτοῦ ἀντὶ ψυχῆς ἐξιλάσεται.
\vs{12}Διὰ τοῦτο εἴρηκα τοῖς υἱοῖς Ἰσραὴλ, πᾶσα ψυχὴ ἐξ ὑμῶν οὐ φάγεται αἷμα· καὶ ὁ προσήλυτος ὁ προσκείμενος ἐν ὑμῖν οὐ φάγεται αἷμα.
\vs{13}Καὶ ἄνθρωπος ἄνθρωπος τῶν υἱῶν Ἰσραὴλ, ἢ τῶν προσηλύτων τῶν προσκειμένων ἐν ὑμῖν, ὃς ἂν θηρεύσῃ θήρευμα θηρίον ἢ πετεινὸν, ὃ ἔσθεται, καὶ ἐκχεεῖ τὸ αἷμα, καὶ καλύψει αὐτὸ τῇ γῇ.
\vs{14}Ἡ γὰρ ψυχὴ πάσης σαρκὸς αἷμα αὐτοῦ ἐστι· καὶ εἶπα τοῖς υἱοῖς Ἰσραὴλ, αἷμα πάσης σαρκὸς οὐ φάγεσθε, ὅτι ἡ ψυχὴ πάσης σαρκὸς αἷμα αὐτοῦ ἐστί· πᾶς ὁ ἔσθων αὐτὸ, ἐξολοθρευθήσεται.
\vs{15}Καὶ πᾶσα ψυχὴ, ἥτις φάγεται θνησιμαῖον, ἢ θηριάλωτον ἐν τοῖς αὐτόχθοσιν, ἢ ἐν τοῖς προσηλύτοις, πλυνεῖ τὰ ἱμάτια αὐτοῦ, καὶ λούσεται ὕδατι, καὶ ἀκάθαρτος ἔσται ἕως ἑσπέρας, καὶ καθαρὸς ἔσται.
\vs{16}Ἐὰν δὲ μὴ πλύνῃ τὰ ἱμάτια, καὶ τὸ σῶμα μὴ λούσηται ὕδατι, καὶ λήψεται ἀνόμημα αὐτοῦ.

\ch{18}
Καὶ εἶπε Κύριος πρὸς Μωυσῆν, λέγων,
\vs{2}λάλησον τοῖς υἱοῖς Ἰσραὴλ, καὶ ἐρεῖς πρὸς αὐτοὺς, ἐγὼ Κύριος ὁ Θεὸς ὑμῶν.
\vs{3}Κατὰ τὰ ἐπιτηδεύματα Αἰγύπτου, ἐν ᾗ κατῳκήσατε ἐπʼ αὐτῇ, οὐ ποιήσετε· καὶ κατὰ τὰ ἐπιτηδεύματα γῆς Χαναὰν, εἰς ἣν ἐγὼ εἰσάγω ὑμᾶς ἐκεῖ, οὐ ποιήσετε, καὶ τοῖς νομίμοις αὐτῶν οὐ πορεύσεσθε.
\vs{4}Τὰ κρίματά μου ποιήσετε, καὶ τὰ προστάγματά μου φυλάξεσθε, καὶ πορεύεσθε ἐν αὐτοῖς· ἐγὼ Κύριος ὁ Θεὸς ὑμῶν.
\vs{5}Καὶ φυλάξεσθε πάντα τὰ προστάγματά μου, καὶ πάντα τὰ κρίματά μου, καὶ ποιήσετε αὐτά· ἃ ποιήσας αὐτὰ ἄνθρωπος, ζήσεται ἐν αὐτοῖς· ἐγὼ Κύριος ὁ Θεὸς ὑμῶν.
\vs{6}Ἄνθρωπος ἄνθρωπος πρὸς πάντα οἰκεῖα σαρκὸς αὐτοῦ οὐ προσελεύσεται ἀποκαλύψαι ἀσχημοσύνην· ἐγὼ Κύριος.
\vs{7}Ἀσχημοσύνην πατρός σου καὶ ἀσχημοσύνην μητρός σου οὐκ ἀποκαλύψεις, μήτηρ γάρ σου ἐστὶν, οὐκ ἀποκαλύψεις τὴν ἀσχημοσύνην αὐτῆς.
\vs{8}Ἀσχημοσύνην γυναικὸς πατρός σου οὐκ ἀποκαλύψεις, ἀσχημοσύνη πατρός σου ἐστίν.
\vs{9}Ἀσχημοσύνην τῆς ἀδελφῆς σου ἐκ πατρός σου ἢ ἐκ μητρός σου, ἐνδογενοῦς ἢ γεγεννημένης ἔξω, οὐκ ἀποκαλύψεις ἀσχημοσύνην αὐτῶν.
\vs{10}Ἀσχημοσύνην θυγατρὸς υἱοῦ σου, ἢ θυγατρὸς θυγατρός σου, οὐκ ἀποκαλύψεις τὴν ἀσχημοσύνην αὐτῶν, ὅτι σὴ ἀσχημοσύνη ἐστίν.
\vs{11}Ἀσχημοσύνην θυγατρὸς γυναικὸς πατρός σου οὐκ ἀποκαλύψεις, ὁμοπατρία ἀδελφή σου ἐστὶν, οὐκ ἀποκαλύψεις τὴν ἀσχημοσύνην αὐτῆς.
\vs{12}Ἀσχημοσύνην ἀδελφῆς πατρός σου οὐκ ἀποκαλύψεις, οἰκεία γὰρ πατρός σου ἐστιν.
\vs{13}Ἀσχημοσύνην ἀδελφῆς μητρός σου οὐκ ἀποκαλύψεις, οἰκεία γὰρ μητρός σου ἐστίν.
\vs{14}Ἀσχημοσύνην ἀδελφοῦ τοῦ πατρός σου οὐκ ἀποκαλύψεις, καὶ πρὸς τὴν γυναῖκα αὐτοῦ οὐκ εἰσελεύσῃ, συγγενὴς γάρ σου ἐστίν.
\vs{15}Ἀσχημοσύνην νύμφης σου οὐκ ἀποκαλύψεις, γυνὴ γὰρ υἱοῦ σου ἐστὶν, οὐκ ἀποκαλύψεις τὴν ἀσχημοσύνην αὐτῆς.
\vs{16}Ἀσχημοσύνην γυναικὸς ἀδελφοῦ σου οὐκ ἀποκαλύψεις, ἀσχημοσύνη ἀδελφοῦ σου ἐστίν.
\vs{17}Ἀσχημοσύνην γυναικὸς καὶ θυγατρὸς αὐτῆς οὐκ ἀποκαλύψεις· τὴν θυγατέρα τοῦ υἱοῦ αὐτῆς, καὶ τὴν θυγατέρα τῆς θυγατρὸς αὐτῆς οὐ λήψῃ ἀποκαλύψαι τὴν ἀσχημοσύνην αὐτῶν, οἰκεῖαι γάρ σου εἰσίν· ἀσέβημα ἐστι.
\vs{18}Γυναῖκα ἐπʼ ἀδελφῇ αὐτῆς οὐ λήψῃ ἀντίζηλον ἀποκαλύψαι τὴν ἀσχημοσύνην αὐτῆς ἐπʼ αὐτῇ, ἔτι ζώσης αὐτῆς.

\vs{19}Καὶ πρὸς γυναῖκα ἐν χωρισμῷ ἀκαθαρσίας αὐτῆς οὐκ εἰσελεύσῃ ἀποκαλύψαι τὴν ἀσχημοσύνην αὐτῆς.
\vs{20}Καὶ πρὸς τὴν γυναῖκα τοῦ πλησίον σου οὐ δώσεις κοίτην σπέρματός σου, ἐκμιανθῆναι πρὸς αὐτήν.
\vs{21}Καὶ ἀπὸ τοῦ σπέρματός σου οὐ δώσεις λατρεύειν ἄρχοντι· καὶ οὐ βεβηλώσεις τὸ ὄνομα τὸ ἅγιον· ἐγὼ Κύριος.
\vs{22}Καὶ μετὰ ἄρσενος οὐ κοιμηθήσῃ κοίτην γυναικείαν, βδέλυγμα γάρ ἐστι.
\vs{23}Καὶ πρὸς πᾶν τετράπουν οὐ δώσεις τὴν κοίτην σου εἰς σπερματισμὸν, ἐκμιανθῆναι πρὸς αὐτό· καὶ γυνὴ οὐ στήσεται πρὸς πᾶν τετράπουν βιβασθῆναι· μυσαρὸν γάρ ἐστι.
\vs{24}Μὴ μιαίνεσθε ἐν πᾶσι τούτοις· ἐν πᾶσι γὰρ τούτοις ἐμίανθησαν τὰ ἔθνη, ἃ ἐγὼ ἐξαποστέλλω πρὸ προσώπου ὑμῶν,
\vs{25}καὶ ἐξεμιάνθη ἡ γῆ καὶ ἀνταπέδωκα ἀδικίαν αὐτοῖς διʼ αὐτὴν, καὶ προσώχθισεν ἡ γῆ τοῖς ἐγκαθημένοις ἐπʼ αὐτῆς.
\vs{26}Καὶ φυλάξεσθε πάντα τὰ νόμιμά μου, καὶ πάντα τὰ προστάγματά μου, καὶ οὐ ποιήσετε ἀπὸ πάντων τῶν βδελυγμάτων τούτων ὁ ἐγχώριος, καὶ ὁ προσγενόμενος προσήλυτος ἐν ὑμῖν·
\vs{27}(Πάντα γὰρ τὰ βδελύγματα ταῦτα ἐποίησαν οἱ ἄνθρωποι τῆς γῆς, οἱ ὄντες πρότερον ὑμῶν, καὶ ἐμιάνθη ἡ γῆ·)
\vs{28}καὶ ἵνα μὴ προσοχθίσῃ ὑμῖν ἡ γῆ ἐν τῷ μιαίνειν ὑμᾶς αὐτὴν, ὃν τρόπον προσώχθισε τοῖς ἔθνεσι τοῖς πρὸ ὑμῶν.
\vs{29}Ὅτι πᾶς ὃς ἐὰν ποιήσῃ ἀπὸ πάντων τῶν βδελυγμάτων τούτων, ἐξολοθρευθήσονται αἱ ψυχαὶ αἱ ποιοῦσαι ἐκ τοῦ λαοῦ αὐτῶν.
\vs{30}Καὶ φυλάξετε τὰ προστάγματά μου, ὅπως μὴ ποιήσητε ἀπὸ πάντων τῶν νομίμων τῶν ἐβδελυγμένων, ἃ γέγονε πρὸ τοῦ ὑμᾶς· καὶ οὐ μιανθήσεσθε ἐν αὐτοῖς, ὅτι ἐγὼ Κύριος ὁ Θεὸς ὑμῶν.

\ch{19}
Καὶ ἐλάλησε Κύριος πρὸς Μωυσῆν, λέγων,
\vs{2}λάλησον τῇ συναγωγῇ τῶν υἱῶν Ἰσραὴλ, καὶ ἐρεῖς πρὸς αὐτοῦς, ἅγιοι ἔσεσθε, ὅτι ἅγιος ἐγὼ Κύριος ὁ Θεὸς ὑμῶν.
\vs{3}Ἕκαστος πατέρα αὐτοῦ καὶ μητέρα αὐτοῦ φοβείσθω, καὶ τὰ σάββατά μου φυλάξεσθε· ἐγὼ Κύριος ὁ Θεὸς ὑμῶν.
\vs{4}Οὐκ ἐπακολουθήσετε εἰδώλοις, καὶ θεοὺς χωνευτοὺς οὐ ποιήσετε ὑμῖν· ἐγὼ Κύριος ὁ Θεὸς ὑμῶν.
\vs{5}Καὶ ἐὰν θύσητε θυσίαν σωτηρίου τῷ Κυρίῳ, δεκτὴν ὑμῶν θύσετε.
\vs{6}ᾟ ἂν ἡμέρᾳ θύσετε, βρωθήσεται, καὶ τῇ αὔριον· καὶ ἐὰν καταλειφθῇ ἕως ἡμέρας τρίτης, ἐν πυρὶ κατακαυθήσεται.
\vs{7}Ἐὰν δὲ βρώσει βρωθῇ τῇ ἡμέρᾳ τῇ τρίτῃ, ἄθυτόν ἐστιν, οὐ δεχθήσεται.
\vs{8}Ὁ δὲ ἔσθων αὐτὸ, ἁμαρτίαν λήψεται, ὅτι τὰ ἅγια Κυρίου ἐβεβήλωσε· καὶ ἐξολοθρευθήσονται αἱ ψυχαὶ αἱ ἔσθουσαι ἐκ τοῦ λαοῦ αὐτῶν.

\vs{9}Καὶ ἐκθεριζόντων ὑμῶν τὸν θερισμὸν τῆς γῆς ὑμῶν, οὐ συντελέσετε τὸν θερισμὸν ὑμῶν τοῦ ἀγροῦ σου ἐκθερίσαι· καὶ τὰ ἀποπίπτοντα τοῦ θερισμοῦ σου οὐ συλλέξεις,
\vs{10}καὶ τὸν ἀμπελῶνά σου οὐκ ἐπανατρυγήσεις, οὐδὲ τὰς ῥῶγας τοῦ ἀμπελῶνός σου συλλέξεις· τῷ πτωχῷ καὶ τῷ προσηλύτῳ καταλείψεις αὐτά. ἐγώ εἰμι Κύριος ὁ Θεὸς ὑμῶν.
\vs{11}Οὐ κλέψετε, οὐ ψεύσεσθε, οὐδὲ συκοφαντήσει ἕκαστος τὸν πλησίον.
\vs{12}Καὶ οὐκ ὀμεῖσθε τῷ ὀνόματί μου ἐπʼ ἀδίκῳ, καὶ οὐ βεβηλώσετε τὸ ὄνομα τὸ ἅγιον τοῦ Θεοῦ ὑμῶν· ἐγώ εἰμι Κύριος ὁ Θεὸς ὑμῶν.
\vs{13}Οὐκ ἀδικήσεις τὸν πλησίον, καὶ οὐχ ἁρπᾷ· καὶ οὐ μὴ κοιμηθήσεται ὁ μισθὸς τοῦ μισθωτοῦ σου παρὰ σοὶ ἕως πρωΐ.

\vs{14}Οὐ κακῶς ἐρεῖς κωφόν, καὶ ἀπέναντι τυφλοῦ οὐ προσθήσεις σκάνδαλον· καὶ φοβηθήσῃ Κύριον τὸν Θεόν σου· ἐγώ εἰμι Κύριος ὁ Θεὸς ὑμῶν.
\vs{15}Οὐ ποιήσετε ἄδικον ἐν κρίσει· οὐ λήψῃ πρόσωπον πτωχοῦ, οὐδὲ μὴ θαυμάσῃς πρόσωπον δυνάστου· ἐν δικαιοσύνῃ κρίνεις τὸν πλησίον σου.
\vs{16}Οὐ πορεύσῃ δόλῳ ἐν τῷ ἔθνει σου· οὐκ ἐπιστήσῃ ἐφʼ αἷμα τοῦ πλησίον σου· ἐγώ εἰμι Κύριος ὁ Θεὸς ὑμῶν.
\vs{17}Οὐ μισήσεις τὸν ἀδελφόν σου τῇ διανοίᾳ σου· ἐλεγμῷ ἐλέγξεις τὸν πλησίον σου, καὶ οὐ λήψῃ διʼ αὐτὸν ἁμαρτίαν.
\vs{18}Καὶ οὐκ ἐκδικᾶταί σου ἡ χείρ· καὶ οὐ μηνιεῖς τοῖς υἱοῖς τοῦ λαοῦ σου· καὶ ἀγαπήσεις τὸν πλησίον σου ὡς σεαυτόν· ἐγώ εἰμι Κύριος.

\vs{19}Τὸν νόμον μου φυλάξεσθε· τὰ κτήνη σου οὐ κατοχεύσεις ἑτεροζύγῳ· καὶ τὸν ἀμπελῶνά σου οὐ κατασπερεῖς διάφορον· καὶ ἱμάτιον ἐκ δύο ὑφασμένον κίβδηλον οὐκ ἐπιβαλεῖς σεαυτῷ.
\vs{20}Καὶ ἐάν τις κοιμηθῇ μετὰ γυναικὸς κοίτην σπέρματος, καὶ αὕτη ᾖ οἰκέτις διαπεφυλαγμένη ἀνθρώπῳ, καὶ αὕτη λύτροις οὐ λελύτρωται, ἢ ἐλευθερία οὐκ ἐδόθη αὐτῇ, ἐπισκοπὴ ἔσται αὐτοῖς· οὐκ ἀποθανοῦνται, ὅτι οὐκ ἀπηλευθερώθη.
\vs{21}Καὶ προσάξει τῆς πλημμελείας αὐτοῦ τῷ Κυρίῳ παρὰ τὴν θύραν τῆς σκηνῆς τοῦ μαρτυρίου κριὸν πλημμελείας.
\vs{22}Καὶ ἐξιλάσεται περὶ αὐτοῦ ὁ ἱερεὺς ἐν τῷ κριῷ τῆς πλημμελείας ἔναντι Κυρίου περὶ τῆς ἁμαρτίας ἧς ἥμαρτε, καὶ ἀφεθήσεται αὐτῷ ἡ ἁμαρτία ἣν ἥμαρτεν.
\vs{23}Ὅταν δὲ εἰσέλθητε εἰς τὴν γῆν, ἣν Κύριος ὁ Θεὸς ὑμῶν δίδωσιν ὑμῖν, καὶ καταφυτεύσετε πᾶν ξύλον βρώσιμον, καὶ περικαθαριεῖτε τὴν ἀκαθαρσίαν αὐτοῦ· ὁ καρπὸς αὐτοῦ τρία ἔτη ἔσται ὑμῖν ἀπερικάθαρτος, οὐ βρωθήσεται.
\vs{24}Καὶ τῷ ἔτει τῷ τετάρτῳ ἔσται πᾶς ὁ καρπὸς αὐτοῦ ἅγιος αἰνετὸς τῷ Κυρίῳ.
\vs{25}Ἐν δὲ τῷ ἔτει τῷ πέμπτῳ φάγεσθε τὸν καρπόν, πρόσθεμα ὑμῖν τὰ γεννήματα αὐτοῦ· ἐγώ εἰμι Κύριος ὁ Θεὸς ὑμῶν.

\vs{26}Μὴ ἔσθετε ἐπὶ τῶν ὀρέων, καὶ οὐκ οἰωνιεῖσθε, οὐδὲ ὀρνιθοσκοπήσεσθε.
\vs{27}Οὐ ποιήσετε σισόην ἐκ τῆς κόμης τῆς κεφαλῆς ὑμῶν, οὐδὲ φθερεῖτε τὴν ὄψιν τοῦ πώγωνος ὑμῶν.
\vs{28}Καὶ ἐντομίδας οὐ ποιήσετε ἐπὶ ψυχῇ ἐν τῷ σώματι ὑμῶν· καὶ γράμματα στικτὰ οὐ ποιήσετε ἐν ὑμῖν· ἐγώ εἰμι Κύριος ὁ Θεὸς ὑμῶν.
\vs{29}Οὐ βεβηλώσεις τὴν θυγατέρα σου ἐκπορνεῦσαι αὐτήν· καὶ οὐκ ἐκπορνεύσει ἡ γῆ, καὶ ἡ γῆ πλησθήσεται ἀνομίας.
\vs{30}Τὰ σάββατά μον φυλάξεσθε, καὶ ἀπὸ τῶν ἁγίων μου φοβηθήσεσθε· ἐγώ εἰμι Κύριος.
\vs{31}Οὐκ ἐπακολουθήσετε ἐγγαστριμύθοις, καὶ τοῖς ἐπαοιδοῖς οὐ προσκολληθήσεσθε, ἐκμιανθῆναι ἐν αὐτοῖς· ἐγώ εἰμι Κύριος ὁ Θεὸς ὑμῶν.
\vs{32}Ἀπὸ προσώπου πολιοῦ ἐξαναστήσῃ, καὶ τιμήσεις πρόσωπον πρεσβυτέρου, καὶ φοβηθήσῃ τὸν Θεόν σου· ἐγώ εἰμι Κύριος ὁ Θεὸς ὑμῶν.
\vs{33}Ἐὰν δέ τις προσέλθῃ ὑμῖν προσήλυτος ἐν τῇ γῇ ὑμῶν, οὐ θλίψετε αὐτόν.
\vs{34}Ὡς ὁ αὐτόχθων ἐν ὑμῖν ἔσται ὁ προσήλυτος ὁ προσπορευόμενος πρὸς ὑμᾶς, καὶ ἀγαπήσεις αὐτὸν ὡς σεαυτόν· ὅτι προσήλυτοι ἐγενήθητε ἐν γῇ Αἰγύπτῳ· ἐγώ εἰμι Κύριος ὁ Θεὸς ὑμῶν.
\vs{35}Οὐ ποιήσετε ἄδικον ἐν κρίσει, ἐν μέτροις καὶ ἐν σταθμίοις καὶ ἐν ζυγοῖς.
\vs{36}Ζυγὰ δίκαια καὶ σταθμία δίκαια καὶ χοῦς δίκαιος ἔσται ἐν ὑμιν· ἐγώ εἰμι Κύριος ὁ Θεὸς ὑμῶν, ὁ ἐξαγαγὼν ὑμᾶς ἐκ γῆς Αἰγύπτου.
\vs{37}Καὶ φυλάξεσθε πάντα τὸν νόμον μου, καὶ πάντα τὰ προστάγματά μου, καὶ ποιήσετε αὐτά· ἐγώ εἰμι Κύριος ὁ Θεὸς ὑμῶν.

\ch{20}
Καὶ ἐλάλησε Κύριος πρὸς Μωυσῆν, λέγων, καὶ τοῖς υἱοῖς Ἰσραὴλ λαλήσεις,
\vs{2}ἐάν τις ἀπὸ τῶν υἱῶν Ἰσραὴλ, ἢ ἀπὸ τῶν γεγενημένων προσηλύτων ἐν Ἰσραὴλ, ὃς ἂν δῷ τοῦ σπέρματος αὐτοῦ ἄρχοντι, θανάτῳ θανατούσθω· τὸ ἔθνος τὸ ἐπὶ τῆς γῆς λιθοβολήσουσιν αὐτὸν ἐν λίθοις.
\vs{3}Καὶ ἐγὼ ἐπιστήσω τὸ πρόσωπόν μου ἐπὶ τὸν ἄνθρωπον ἐκεῖνον, καὶ ἀπολῶ αὐτὸν ἐκ τοῦ λαοῦ αὐτοῦ, ὅτι τοῦ σπέρματος αὐτοῦ ἔδωκεν ἄρχοντι, ἵνα μιάνῃ τὰ ἅγιά μου, καὶ βεβηλώσῃ τὸ ὄνομα τῶν ἡγιασμένων μοι.
\vs{4}Ἐὰν δὲ ὑπερόψει ὑπερίδωσιν οἱ αὐτόχθονες τῆς γῆς τοῖς ὀφθαλμοῖς αὐτῶν ἀπὸ τοῦ ἀνθρώπου ἐκείνου, ἐν τῷ δοῦναι αὐτὸν τοῦ σπέρματος αὐτοῦ ἄρχοντι, τοῦ μὴ ἀποκτεῖναι αὐτόν·
\vs{5}καὶ ἐπιστήσω τὸ πρόσωπόν μου ἐπὶ τὸν ἄνθρωπον ἐκεῖνον, καὶ τὴν συγγένειαν αὐτοῦ, καὶ ἀπολῶ αὐτὸν, καὶ πάντας τοὺς ὁμονοοῦντας αὐτῷ, ὥστε ἐκπορνεύειν αὐτὸν εἰς τοὺς ἄρχοντας, ἐκ τοῦ λαοῦ αὐτῶν.

\vs{6}Καὶ ψυχὴ ἣ ἂν ἐπακολουθήσῃ ἐγγαστριμύθοις ἢ ἐπαοιδοῖς, ὥστε ἐκπορνεῦσαι ὀπίσω αὐτῶν, ἐπιστήσω τὸ πρόσωπόν μου ἐπὶ τὴν ψυχὴν ἐκείνην, καὶ ἀπολῶ αὐτῆν ἐκ τοῦ λαοῦ αὐτῆς.
\vs{7}Καὶ ἔσεσθε ἅγιοι, ὅτι ἅγιος ἐγὼ Κύριος ὁ Θεὸς ὑμῶν.
\vs{8}Καὶ φυλάξεσθε τὰ προστάγματά μου, καὶ ποιήσετε αὐτά· ἐγὼ Κύριος ὁ ἁγιάζεν ὑμᾶς.
\vs{9}Ἄνθρωπος ἄνθρωπος, ὃς ἂν κακῶς εἴπῃ τὸν πατέρα αὐτοῦ ἢ τὴν μητέρα αὐτοῦ, θανάτῳ θανατούσθω· πατέρα αὐτοῦ ἢ μητέρα αὐτοῦ κακῶς εἶπεν; ἔνοχος ἔσται.

\vs{10}Ἄνθρωπος ὃς ἂν μοιχεύσηται γυναῖκα ἀνδρός, ἢ ὃς ἂν μοιχεύσηται γυναῖκα τοῦ πλησίον, θανάτῳ θανατούσθωσαν, ὁ μοιχεύων καὶ ἡ μοιχευομένη.
\vs{11}Καὶ ἐάν τις κοιμηθῇ μετὰ γυναικὸς τοῦ πατρὸς αὐτοῦ, ἀσχημοσύνην τοῦ πατρὸς αὐτοῦ ἀπεκάλυσε· θανάτῳ θανατούσθωσαν ἀμφότεροι, ἔνοχοί εἰσι.
\vs{12}Καὶ ἐάν τις κοιμηθῇ μετὰ νύμφης αὐτοῦ, θανάτῳ θανατούσθωσαν ἀμφότεροι· ἠσεβήκασι γάρ, ἔνοχοί εἰσι.
\vs{13}Καὶ ὃς ἂν κοιμηθῇ μετὰ ἄρσενος κοίτην γυναικὸς, βδέλυγμα ἐποίησαν ἀμφότεροι· θανάτῳ θανατούσθωσαν, ἔνοχοι εἰσιν.
\vs{14}Ὃς ἂν λάβῃ γυναῖκα καὶ τὴν μητέρα αὐτῆς, ἀνόμημά ἐστιν· ἐν πυρὶ κατακαύσουσιν αὐτὸν καὶ αὐτὰς, καὶ οὐκ ἔσται ἀνομία ἐν ὑμῖν.
\vs{15}Καὶ ὃς ἂν δῷ κοιτασίαν αὐτοῦ ἐν τετράποδι, θανάτῳ θανάτούσθω, καὶ τὸ τετράπουν ἀποκτενεῖτε.
\vs{16}Καὶ γυνὴ ἥτις προσελεύσεται πρὸς πᾶν κτῆνος βιβασθῆναι αὐτὴν ὑπʼ αὐτοὺ, ἀποκτενεῖτε τὴν γυναῖκα καὶ τὸ κτῆνος· θανάτῳ θανατούσθωσαν, ἔνοχοί εἰσιν.
\vs{17}Ὃς ἂν λάβῃ τὴν ἀδελφὴν αὐτοῦ ἐκ πατρὸς αὐτοῦ ἢ ἐκ μητρὸς αὐτοῦ, καὶ ἴδῃ τὴν ἀσχημοσύνην αὐτῆς, καὶ αὕτη ἴδῃ τὴν ἀσχημοσύνην αὐτοῦ, ὄνειδός ἐστιν, ἐξολοθρευθήσονται ἐνωπιον υἱῶν γένους αὐτῶν· ἀσχημοσύνην ἀδελφῆς αὐτοῦ ἀπεκάλυψεν, ἁμαρτίαν κομιοῦνται.
\vs{18}Καὶ ἀνὴρ ὃς ἂν κοιμηθῇ μετὰ γυναικὸς ἀποκαθημένης, καὶ ἀποκαλύψῃ τὴν ἀσχημοσύνην αὐτῆς, τὴν πηγὴν αὐτῆς ἀπεκάλυψε, καὶ αὕτη ἀπεκάλυψε τὴν ῥύσιν τοῦ αἵματος αὐτῆς· ἐξολοθρευθήσονται ἀμφότεροι ἐκ τῆς γενεᾶς αὐτῶν.
\vs{19}Καὶ ἀσχημοσύνην ἀδελφῆς πατρὸς σου, καὶ ἀδελφῆς μητρός σου οὐκ ἀποκαλύψεις· τὴν γὰρ οἰκειότητα ἀπεκάλυψεν, ἁμαρτίαν ἀποίσονται.
\vs{20}Ὃς ἂν κοιμηθῇ μετὰ τῆς συγγενοῦς αὐτοῦ, ἀσχημοσύνην τῆς συγγενείας αὐτοῦ ἀπεκάλυψεν, ἄτεκνοι ἀποθανοῦνται.
\vs{21}Ὃς ἐὰς λάβῃ γυναῖκα τοῦ ἀδελφοῦ αὐτοῦ, ἀκαθαρσία ἐστίν· ἀσχημοσύνην τοῦ ἀδελφοῦ αὐτοῦ ἀπεκάλυψεν, ἄτεκνοι ἀποθανοῦνται.

\vs{22}Καὶ φυλάξασθε πάντα τὰ προστάγματά μου, καὶ τὰ κρίματά μου, καὶ ποιήσετε αὐτὰ, καὶ οὐ μὴ προσοχθίσῃ ὑμῖν ἡ γῆ, εἰς ἣν ἐγὼ εἰσάγω ὑμᾶς ἐκεῖ κατοικεῖν ἐπʼ αὐτῆς.
\vs{23}Καὶ οὐχὶ πορεύεσθε τοῖς νομίμοις τῶν ἐθνῶν, οὓς ἐξαποστέλλω ἀφʼ ὑμῶν· ὅτι ταῦτα πάντα ἐποίησαν, καὶ ἐβδελυξάμην αὐτούς.
\vs{24}Καὶ εἶπα ὑμῖν, ὑμεῖς κληρονομήσετε τὴν γῆν αὐτῶν, καὶ ἐγὼ δώσω ὑμῖν αὐτὴν ἐν κτήσει, γῆν ῥέουσαν γάλα καὶ μέλι· ἐγὼ Κύριος ὁ Θεὸς ὑμῶν, ὃς διώρισα ὑμᾶς ἀπὸ πάντων τῶν ἐθνῶν.
\vs{25}Καὶ ἀφοριεῖτε αὐτοὺς ἀναμέσον τῶν κτηνῶν τῶν καθαρῶν καὶ ἀναμέσον τῶν κτηνῶν τῶν ἀκαθάρτων, καὶ ἀναμέσον τῶν πετεινῶν τῶν καθαρῶν καὶ τῶν ἀκαθάρτων· καὶ οὐ βδελύξετε τὰς ψυχὰς ὑμῶν ἐν τοῖς κτήνεσι, καὶ ἐν τοῖς πετεινοῖς, καὶ ἐν πᾶσι τοῖς ἑρπετοῖς τῆς γῆς ἃ ἐγὼ ἀφώρισα ὑμῖν ἐν ἀκαθαρσίᾳ.
\vs{26}Καὶ ἔσεσθέ μοι ἅγιοι, ὅτι ἐγὼ ἅγιός εἶμι Κύριος ὁ Θεὸς ὑμῶν, ὁ ἀφορίσας ὑμᾶς ἀπὸ πάντων τῶν ἐθνῶν, εἶναί μοι.

\vs{27}Καὶ ἀνὴρ ἢ γυνὴ ὃς ἂν γένηται αὐτῶν ἐγγαστρίμυθος ἢ ἐπαοιδός, θανάτῳ θανατούσθωσαν ἀμφότεροι· λίθοις λιθοβολήσετε αὐτοὺς, ἔνοχοί εἰσι.

\ch{21}
Καὶ εἶπε Κύριος πρὸς Μωυσῆν, λέγων, εἶπον τοῖς ἱερεύσι τοῖς υἱοῖς Ἀαρὼν, καὶ ἐρεῖς πρὸς αὐτούς, ἐν ταῖς ψυχαῖς οὐ μιανθήσονται ἐν τῷ ἔθνει αὐτῶν,
\vs{2}ἀλλʼ ἢ ἐν τῷ οἰκείῳ τῷ ἔγγιστα αὐτῶν, ἐπὶ πατρὶ καὶ μητρὶ, καὶ υἱοῖς, καὶ θυγατράσιν, ἐπʼ ἀδελφῷ,
\vs{3}καὶ ἐπʼ ἀδελφῇ παρθένῳ τῇ ἐγγιζούσῃ αὐτῷ, τῇ μὴ ἐκδεδομένῃ ἀνδρί, ἐπὶ τούτοις μιανθήσεται.
\vs{4}Οὐ μιανθήσεται ἐξάπινα ἐν τῷ λαῷ αὐτοῦ εἰς βεβήλωσιν αὐτοῦ.
\vs{5}Καὶ φαλάκρωμα οὐ ξυρηθήσεσθε τὴν κεφαλὴν ἐπὶ νεκρῷ· καὶ τὴν ὄψιν τοῦ πώγωνος οὐ ξυρήσονται· καὶ ἐπὶ τὰς σάρκας αὐτῶν οὐ κατατεμοῦσιν ἐντομίδας.
\vs{6}Ἅγιοι ἔσονται τῷ Θεῷ αὐτῶν, καὶ οὐ βεβηλώσουσι τὸ ὄνομα τοῦ Θεοῦ αὐτῶν· τὰς γὰρ θυσίας Κυρίου δῶρα τοῦ Θεοῦ αὐτῶν αὐτοὶ προσφέρουσι, καὶ ἔσονται ἅγιοι.
\vs{7}Γυναῖκα πόρνην καὶ βεβηλωμένην οὐ λήψονται, καὶ γυναῖκα ἐκβεβλημένην ἀπὸ ἀνδρὸς αὐτῆς, ὅτι ἅγιός ἐστι Κυρίῳ τῷ Θεῷ αὐτοῦ.
\vs{8}Καὶ ἁγιάσεις αὐτόν· τὰ δῶρα Κυρίου τοῦ Θεοῦ ὑμῶν οὗτος προσφέρει, ἅγιος ἔσται· ὅτι ἅγιος ἐγὼ Κύριος ὁ ἁγιάζων αὐτούς.
\vs{9}Καὶ θυγάτηρ ἀνθρώπου ἱερέως ἐὰν βεβηλωθῇ τοῦ ἐκπορνεύσαι, τὸ ὄνομα τοῦ πατρὸς αὐτῆς αὐτὴ βεβηλοῖ· ἐπὶ πυρὸς κατακαυθήσεται.

\vs{10}Καὶ ὁ ἱερεὺς ὁ μέγας ἀπὸ τῶν ἀδελφῶν αὐτοῦ, τοῦ ἐπικεχυμένου ἐπὶ τὴν κεφαλὴν τοῦ ἐλαίου τοῦ χριστοῦ, καὶ τετελειωμένου ἐνδύσασθαι τὰ ἱμάτια, τὴν κεφαλὴν οὐκ ἀποκιδαρώσει, καὶ τὰ ἱμάτια οὐ διαῤῥήξει,
\vs{11}καὶ ἐπὶ πάσῃ ψυχῇ τετελευτηκυίᾳ οὐκ εἰσελεύσεται, ἐπὶ πατρὶ αὐτοῦ οὐδὲ ἐπὶ μητρὶ αὐτοῦ οὐ μιανθήσεται.
\vs{12}Καὶ ἐκ τῶν ἁγίων οὐκ ἐξελεύσεται, καὶ οὐ βεβηλώσει τὸ ἡγιασμένον τοῦ Θεοῦ αὐτοῦ, ὅτι τὸ ἅγιον ἔλαιον τὸ χριστὸν τοῦ Θεοῦ ἐπʼ αὐτῷ· ἐγὼ Κύριος.
\vs{13}Οὗτος γυναῖκα παρθένον ἐκ τοῦ γένους αὐτοῦ λήψεται.
\vs{14}Χήραν δὲ καὶ ἐκβεβλημένην καὶ βεβηλωμένην καὶ πόρνην, ταύτας οὐ λήψεται, ἀλλʼ ἢ παρθένον ἐκ τοῦ λαοῦ αὐτοῦ λήψεται γυναῖκα.
\vs{15}Καὶ οὐ βεβηλώσει τὸ σπέρμα αὐτοῦ ἐν τῷ λαῷ αὐτοῦ· ἐγὼ Κύριος ὁ ἁγιάζων αὐτόν.
\vs{16}Καὶ ἐλάλησε Κύριος πρὸς Μωυσῆν, λέγων,
\vs{17}εἶπον Ἀαρὼν, ἄνθρωπος ἐκ τοῦ γένους σου εἰς τὰς γενεὰς ὑμῶν, τινὶ ἐὰν ᾖ ἐν αὐτῷ μῶμος, οὐ προσελεύσεται προσφέρειν τὰ δῶρα τοῦ Θεοῦ αὐτοῦ.
\vs{18}πᾶς ἄνθρωπος ᾧ ἐστιν ἐν αὐτῷ μῶμος, οὐ προσελεύσεται· ἄνθρωπος τυφλὸς.
\vs{19}ἢ χωλὸς, ἢ κολοβόριν, ἢ ὠτότμητος, ἢ ἄνθρωπος ᾧ ἂν ᾖ ἐν αὐτῳ σύντριμμα χειρὸς, ἢ σύντριμμα ποδὸς,
\vs{20}ἢ κυρτὸς, ἢ ἔφηλος, ἢ πτίλλος τοὺς ὀφθαλμούς, ἢ ἄνθρωπος ᾧ ἂν ᾖ ἐν αὐτῷ ψώρα ἀγρία, ἢ λειχὴν, ἢ μονόρχις.
\vs{21}Πᾶς ᾧ ἐστιν ἐν αὐτῷ μῶμος, ἐκ τοῦ σπέρματος Ἀαρὼν τοῦ ἱερέως, οὐκ ἐγγιεῖ τοῦ προσενεγκεῖν τὰς θυσίας τῷ Θεῷ σου, ὅτι μῶμος ἐν αὐτῷ· τὰ δῶρα τοῦ Θεοῦ οὐ προσελεύσεται προσενεγκεῖν.
\vs{22}Τὰ δῶρα τοῦ Θεοῦ τὰ ἅγια τῶν ἁγίων, καὶ ἀπὸ τῶν ἁγίων φάγεται.
\vs{23}Πλὴν πρὸς τὸ καταπέτασμα οὐ προσελεύσεται, καὶ πρὸς τὸ θυσιαστήριον οὐκ ἐγγιεῖ, ὅτι μῶμον ἔχει· καὶ οὐ βεβηλώσει τὸ ἅγιον τοῦ Θεοῦ αὐτοῦ, ὅτι ἐγώ εἰμι Κύριος ὁ ἁγιάζων αὐτούς.
\vs{24}Καὶ ἐλάλησε Μωυσῆς πρὸς Ἀαρὼν καὶ τοὺς υἱοὺς αὐτοῦ, καὶ πρὸς πάντας υἱοὺς Ἱσραήλ.

\ch{22}
Καὶ ἐλάλησε Κύριος πρὸς Μωυσῆν, λέγων,
\vs{2}εἶπον Ἀαρὼν καὶ τοῖς υἱοῖς αὐτοῦ· καὶ προσεχέτωσαν ἀπὸ τῶν ἁγίων τῶν υἱῶν Ἰσραὴλ, καὶ οὐ βεβηλώσουσι τὸ ὄνομα τὸ ἅγιόν μου, ὅσα αὐτοὶ ἁγιάζουσί μοι· ἐγὼ Κύριος.
\vs{3}Εἶπον αὐτοῖς, εἰς τὰς γενεὰς ὑμῶν πᾶς ἄνθρωπος, ὃς ἂν προσέλθῃ ἀπὸ παντὸς τοῦ σπέρματος ὑμῶν πρὸς τὰ ἅγια, ὅσα ἂν ἁγιάζωσιν οἱ υἱοὶ Ἰσραὴλ τῷ Κυρίῳ, καὶ ἡ ἀκαθαρσία αὐτοῦ ἐπʼ αὐτῷ ᾖ, ἐξολοθρευθήσεται ἡ ψυχὴ ἐκείνη ἀπʼ ἐμοῦ· ἐγὼ Κύριος ὁ Θεὸς ὑμῶν.
\vs{4}Καὶ ἄνθρωπος ἐκ τοῦ σπέρματος Ἀαρὼν τοῦ ἱερέως, καὶ οὗτος λεπρᾷ ἢ γονοῤῥυεῖ, τῶν ἁγίων οὐκ ἔδεται, ἕως ἂν καθαρισθῇ· καὶ ὁ ἁπτόμενος πάσης ἀκαθαρσίας ψυχῆς, ἢ ἄνθρωπος ᾧ ἂν ἐξέλθῃ ἐξ αὐτοῦ κοίτη σπέρματος,
\vs{5}ἢ ὅστις ἂν ἅψηται παντὸς ἑρπετοῦ ἀκαθάρτου, ὃ μιανεῖ αὐτὸν, ἢ ἐπʼ ἀνθρώπῳ, ἐν ᾧ μιανεῖ αὐτὸν κατὰ πᾶσαν ἀκαθαρσίαν αὐτοῦ·
\vs{6}Ψυχὴ ἥτις ἐὰν ἅψηται αὐτῶν, ἀκάθαρτος ἔσται ἕως ἑσπέρας· οὐκ ἔδεται ἀπὸ τῶν ἁγίων, ἐὰν μὴ λούσηται τὸ σῶμα αὐτοῦ ὕδατι.
\vs{7}Καὶ δύῃ ὁ ἥλιος, καὶ καθαρὸς ἔσται· καὶ τότε φάγεται τῶν ἁγίων, ὅτι ἄρτος αὐτοῦ ἐστι.
\vs{8}Θνησιμαῖον καὶ θηριάλωτον οὐ φάγεται, μιανθῆναι αὐτὸν ἐν αὐτοῖς· ἐγὼ Κύριος.
\vs{9}Καὶ φυλάξονται τὰ φυλάγματά μου, ἵνα μὴ λάβωσι διʼ αὐτὰ ἁμαρτίαν, καὶ ἀποθάνωσι διʼ αὐτὰ, ἐὰν βεβηλώσουσιν αὐτά· ἐγὼ Κύριος ὁ Θεὸς ὁ ἁγιάζων αὐτούς.
\vs{10}Καὶ πᾶς ἀλλογενὴς οὐ φάγεται ἅγια· πάροικος ἱερέως, ἢ μισθωτὸς, οὐ φάγεται ἅγια.
\vs{11}Ἐὰν δὲ ἱερεὺς κτήσηται ψυχὴν ἔγκτητον ἀργυρίου, οὗτος φάγεται ἐκ τῶν ἄρτων αὐτοῦ· καὶ οἱ οἰκογενεῖς αὐτοῦ, καὶ οὗτοι φάγονται τῶν ἄρτων αὐτοῦ.
\vs{12}Καὶ θυγάτηρ ἀνθρώπου ἱερέως ἐὰν γένηται ἀνδρὶ ἀλλογενεῖ, αὐτὴ τῶν ἀπαρχῶν ἁγίου οὐ φάγεται.
\vs{13}Καὶ θυγάτηρ ἱερέως ἐὰν γένηται χήρα ἢ ἐκβεβλημένη, σπέρμα δὲ μὴ ᾖ αὐτῇ, ἐπαναστρέψει ἐπὶ τὸν οἶκον τὸν πατρικὸν κατὰ τὴν νεότητα αὐτῆς· ἀπὸ τῶν ἄρτων τοῦ πατρὸς αὐτῆς φάγεται· καὶ πᾶς ἀλλογενὴς οὐ φάγεται ἀπʼ αὐτῶν.
\vs{14}Καὶ ἄνθρωπος ὃς ἂν φάγῃ ἅγια κατʼ ἄγνοιαν, καὶ προσθήσει τὸ ἐπίπεμπτον αὐτοῦ ἐπʼ αὐτὸ, καὶ δώσει τῷ ἱερεῖ τὸ ἅγιον.
\vs{15}Καὶ οὐ βεβηλώσουσι τὰ ἅγια τῶν υἱῶν Ἰσραὴλ, ἃ αὐτοὶ ἀφαιροῦσι τῷ Κυρίῳ,
\vs{16}καὶ ἐπάξουσιν ἐφʼ ἑαυτοὺς ἀνομίαν πλημμελείας ἐν τῷ ἐσθίειν αὐτοὺς τὰ ἅγια αὐτῶν, ὅτι ἐγὼ Κύριος ὁ ἁγιάζων αὐτούς.

\vs{17}Καὶ ἐλάλησε Κύριος πρὸς Μωυσῆν, λέγων,
\vs{18}λάλησον Ἀαρὼν καὶ τοῖς υἱοῖς αὐτοῦ, καὶ πάσῃ συναγωγῇ Ἰσραὴλ, καὶ ἐρεῖς πρὸς αὐτοὺς, ἄνθρωπος ἄνθρωπος ἀπὸ τῶν υἱῶν Ἰσραὴλ, ἢ τῶν προσηλύτων τῶν προσκειμένων πρὸς αὐτοὺς ἐν Ἰσραὴλ, ὃς ἂν προσενέγκῃ τὰ δῶρα αὐτοῦ κατὰ πᾶσαν ὁμολογίαν αὐτῶν, ἢ κατὰ πᾶσαν αἵρεσιν αὐτῶν, ὅσα ἂν προσενέγκωσι τῷ Θεῷ εἰς ὁλοκαύτωμα.
\vs{19}Δεκτὰ ὑμῖν ἄμωμα ἄρσενα ἐκ τῶν βουκολίων, ἢ ἐκ τῶν προβάτων, καὶ ἐκ τῶν αἰγῶν.
\vs{20}Πάντα ὅσα ἂν ἔχῃ μῶμον ἐν αὐτῷ οὐ προσάξουσι Κυρίῳ, διότι οὐ δεκτὸν ἔσται ὑμῖν.
\vs{21}Καὶ ἄνθρωπος ὃς ἂν προσενέγκῃ θυσίαν σωτηρίου τῷ Κυρίῳ, διαστείλας εὐχὴν ἢ κατὰ αἵρεσιν ἢ ἐν ταῖς ἑορταῖς ὑμῶν, ἐκ τῶν βουκολίων ἢ ἐκ τῶν προβάτων, ἄμωμον ἔσται εἰσδεκτὸν, πᾶς μῶμος οὐκ ἔσται ἐν αὐτῷ.
\vs{22}Τυφλὸν ἢ συντετριμμένον ἢ γλωσσότμητον ἢ μυρμηκιῶντα ἢ ψωραγριῶντα ἢ λειχῆνας ἔχοντα, οὐ προσάξουσι ταῦτα τῷ Κυρίῳ, καὶ εἰς κάρπωσιν οὐ δώσετε ἀπʼ αὐτῶν ἐπὶ τὸ θυσιαστήριον τῷ Κυρίῳ.
\vs{23}Καὶ μόσχον ἢ πρόβατον ὠτότμητον ἢ κολοβόκερκον, σφάγια ποιήσεις αὐτὰ σεαυτῷ, εἰς δὲ εὐχήν σου οὐ δεχθήσεται.
\vs{24}Θλαδίαν καὶ ἐκτεθλιμμένον καὶ ἐκτομίαν καὶ ἀπεσπασμένον, οὐ προσάξεις αὐτὰ τῷ Κυρίῳ, καὶ ἐπὶ τῆς γῆς ὑμῶν οὐ ποιήσετε.
\vs{25}Καὶ ἐκ χειρὸς ἀλλογενοῦς οὐ προσοίσετε τὰ δῶρα τοῦ Θεοῦ ὑμῶν ἀπὸ πάντων τούτων· ὅτι φθάρματά ἐστιν ἐν αὐτοῖς, μῶμος ἐν αὐτοῖς οὐ δεχθήσεται ταῦτα ὑμῖν.
\vs{26}Καὶ ἐλάλησε Κύριος πρὸς Μωυσῆν, λέγων,
\vs{27}μόσχον ἢ πρόβατον ἢ αἶγα, ὡς ἂν τεχθῇ, καὶ ἔσται ἑπτὰ ἡμέρας ὑπὸ τὴν μητέρα, τῇ δὲ ἡμέρᾳ τῇ ὀγδόῃ καὶ ἐπέκεινα δεχθήσεται εἰς δῶρα, κάρπωμα Κυρίῳ.
\vs{28}Καὶ μόσχον καὶ πρόβατον, αὐτὴν καὶ τὰ παιδία αὐτῆς, οὐ σφάξεις ἐν ἡμέρᾳ μιᾷ.

\vs{29}Ἐὰν δὲ θύσῃς θυσίαν εὐχὴν χαρμοσύνης Κυρίῳ, εἰσδεκτὸν ὑμῖν θύσετε αὐτό.
\vs{30}Αὐτῇ τῇ ἡμέρᾳ ἐκείνῃ βρωθήσεται· οὐκ ἀπολείψετε ἀπὸ τῶν κρεῶν εἰς τοπρωΐ· ἐγώ εἰμι Κύριος.
\vs{31}Καὶ φυλάξετε τὰς ἐντολάς μου, καὶ ποιήσετε αὐτάς.
\vs{32}Καὶ οὐ βεβηλώσετε τὸ ὄνομα τοῦ ἁγίου, καὶ ἁγιασθήσομαι ἐν μέσῳ τῶν υἱῶν Ἰσραήλ· ἐγὼ Κύριος ὁ ἁγιάζων ὑμᾶς,
\vs{33}ὁ ἐξαγαγὼν ὑμᾶς ἐκ γῆς Αἰγύπτου, ὥστε εἶναι ὑμῶν Θεός· ἐγὼ Κύριος.

\ch{23}
Καὶ εἶπε Κύριος πρὸς Μωυσῆν, λέγων,
\vs{2}λάλησον τοῖς υἱοῖς Ἰσραὴλ, καὶ ἐρεῖς πρὸς αὐτούς, αἱ ἑορταὶ Κυρίου ἃς καλέσετε αὐτὰς κλητὰς ἁγίας, αὗταί εἰσιν αἱ ἑορταί μου.
\vs{3}Ἓξ ἡμέρας ποιήσεις ἔργα, τῇ δὲ ἡμέρᾳ τῇ ἑβδόμῃ σάββατα, ἀνάπαυσις, κλητὴ ἁγία τῷ Κυρίῳ· πᾶν ἔργον οὐ ποιήσεις· σάββατά ἐστι τῷ Κυρίῳ ἐν πάσῃ κατοικίᾳ ὑμῶν.

\vs{4}Αὗται αἱ ἑορταὶ τῷ Κυρίῳ κληταὶ ἅγιαι, ἃς καλέσετε αὐτὰς ἐν τοῖς καιροῖς αὐτῶν.
\vs{5}Ἐν τῷ πρώτῳ μηνὶ, ἐν τῇ τεσσαρεσκαιδεκάτῃ ἡμέρᾳ τοῦ μηνὸς, ἀναμέσον τῶν ἑσπερινῶν πάσχα τῷ Κυρίῳ.
\vs{6}Καὶ ἐν τῇ πεντεκαιδεκάτῃ ἡμέρᾳ τοῦ μηνὸς τούτου ἑορτὴ τῶν ἀζύμων τῷ Κυρίῳ· ἑπτὰ ἡμέρας ἄζυμα ἔδεσθε.
\vs{7}Καὶ ἡμέρα ἡ πρώτη κλητὴ ἁγία ἔσται ὑμῖν· πᾶν ἔργον λατρευτὸν οὐ ποιήσετε.
\vs{8}Καὶ προσάξετε ὁλοκαυτώματα τῷ Κυρίῳ ἑπτὰ ἡμέρας· καὶ ἡ ἡμέρα ἡ ἑβδόμη κλητὴ ἁγία ἔσται ὑμῖν· πᾶν ἔργον λατρευτὸν οὐ ποιήσετε.
\vs{9}Καὶ ἐλάλησε Κύριος πρὸς Μωυσῆν, λέγων,
\vs{10}εἶπον τοῖς υἱοῖς Ἰσραὴλ, καὶ ἐρεῖς πρὸς αὐτοὺς, ὅταν εἰσέλθητε εἰς τὴν γῆν, ἣν ἐγὼ δίδωμι ὑμῖν, καὶ θερίζητε τὸν θερισμὸν αὐτῆς, καὶ οἴσετε τὸ δράγμα ἀπαρχὴν τοῦ θερισμοῦ ὑμῶν πρὸς τὸν ἱερέα·
\vs{11}Καὶ ἀνοίσει τὸ δράγμα ἔναντι Κυρίου δεκτὸν ὑμῖν· τῇ ἐπαύριον τῆς πρώτης ἀνοίσει αὐτό ὁ ἱερεύς.
\vs{12}Καὶ ποιήσετε ἐν τῇ ἡμέρᾳ ἐν ᾗ ἂν φέρητε τὸ δράγμα, πρόβατον ἄμωμον ἐνιαύσιον εἰς ὁλοκαύτωμα τῷ Κυρίῳ.
\vs{13}Καὶ τὴν θυσίαν αὐτοῦ δύο δέκατα σεμιδάλεως ἀναπεποιημένης ἐν ἐλαίῳ· θυσία τῷ Κυρίῳ, ὀσμὴ εὐωδίας Κυρίῳ· καὶ σπονδὴν αὐτοῦ τὸ τέταρτον τοῦ ἳν οἴνου.
\vs{14}Καὶ ἄρτον, καὶ πεφρυγμένα χίδρα νέα οὐ φάγεσθε ἕως εἰς αὐτὴν τὴν ἡμέραν ταύτην, ἕως ἂν προσενέγκητε ὑμεῖς τὰ δῶρα τῷ Θεῷ ὑμῶν· νόμιμον αἰώνιον εἰς τὰς γενεὰς ὑμῶν ἐν πάσῃ κατοικίᾳ ὑμῶν.

\vs{15}Καὶ ἀριθμήσετε ὑμῖν ἀπὸ τῆς ἐπαύριον τῶν σαββάτων, ἀπὸ τῆς ἡμέρας ἧς ἂν προσενέγκητε τὸ δράγμα τοῦ ἐπιθέματος, ἑπτὰ ἑβδομάδας ὁλοκλήρους,
\vs{16}ἕως τῆς ἐπαύριον τῆς ἐσχάτης ἑβδομάδος ἀριθμήσετε πεντήκοντα ἡμέρας, καὶ προσοίσετε θυσίαν νέαν τῷ Κυρίῳ.
\vs{17}Ἀπὸ τῆς κατοικίας ὑμῶν προσοίσετε ἄρτους ἐπίθεμα, δύο ἄρτους· ἐκ δύο δεκάτων σεμιδάλεως ἔσονται, ἐζυμωμένοι πεφθήσονται πρωτογεννημάτων τῷ Κυρίῳ.
\vs{18}Καὶ προσάξετε μετὰ τῶν ἄρτων ἑπτὰ ἀμνοὺς ἀμώμους ἐνιαυσίους, καὶ μόσχον ἕνα ἐκ βουκολίου, καὶ κριοὺς δύο ἀμώμους, καὶ ἔσονται ὁλοκαύτωμα τῷ Κυρίῳ· καὶ αἱ θυσίαι αὐτῶν καὶ αἱ σπονδαὶ αὐτῶν θυσία ὀσμὴ εὐωδίας τῷ Κυρίῳ.
\vs{19}Καὶ ποιήσουσι χίμαρον ἐξ αἰγῶν ἕνα περὶ ἁμαρτίας, καὶ δύο ἀμνοὺς ἐνιαυσίους εἰς θυσίαν σωτηρίου μετὰ τῶν ἄρτων τοῦ πρωτογεννήματος.
\vs{20}Καὶ ἐπιθήσει αὐτὰ ὁ ἱερεὺς μετὰ τῶν ἄρτων τοῦ πρωτογεννήματος ἐπίθεμα ἐναντίον Κυρίου μετὰ τῶν δύο ἀμνῶν, ἅγια ἔσονται τῷ Κυρίῳ· τῷ ἱερεῖ τῷ προσφέροντι αὐτὰ αὐτῷ ἔσται.
\vs{21}Καὶ καλέσετε ταύτην τὴν ἡμέραν κλητήν· ἁγία ἔσται ὑμῖν· πᾶν ἔργον λατρευτὸν οὐ ποιήσετε ἐν αὐτῇ· νόμιμον αἰώνιον εἰς τὰς γενεὰς ὑμῶν ἐν πάσῃ τῇ κατοικίᾳ ὑμῶν.
\vs{22}Καὶ ὅταν θερίζητε τὸν θερισμὸν τῆς γῆς ὑμῶν, οὐ συντελέσετε τὸ λοιπὸν τοῦ θερισμοῦ τοῦ ἀγροῦ σου ἐν τῷ θερίζειν σε, καὶ τὰ ἀποπίπτοντα τοῦ θερισμοῦ σου οὐ συλλέξεις· τῷ πτωχῷ καὶ τῷ προσηλύτῳ ὑπολείψεις αὐτά· ἐγὼ Κύριος ὁ Θεὸς ὑμῶν.

\vs{23}Καὶ ἐλάλησε Κύριος πρὸς Μωυσῆν, λέγων,
\vs{24}λάλησον τοῖς υἱοῖς Ἰσραὴλ, λέγων, τοῦ μηνὸς τοῦ ἐβδόμου μιᾷ τοῦ μηνὸς ἔσται ὑμῖν ἀνάπαυσις, μνημόσυνον σαλπίγγων· κλητὴ ἁγία ἔσται ὑμῖν·
\vs{25}Πᾶν ἔργον λατρευτὸν οὐ ποιήσετε· καὶ προσάξετε ὁλοκαύτωμα Κυρίῳ.

\vs{26}Καὶ ἐλάλησε Κύριος πρὸς Μωυσῆν, λέγων,
\vs{27}καὶ τῇ δεκάτῃ τοῦ μηνὸς τοῦ ἐβδόμου τούτου, ἡμέρα ἐξιλασμοῦ, κλητὴ ἁγία ἔσται ὑμῖν· καὶ ταπεινώσετε τὰς ψυχὰς ὑμῶν, καὶ προσάξετε ὁλοκαύτωμα τῷ Κυρίῳ.
\vs{28}Πᾶν ἔργον οὐ ποιήσετε ἐν αὐτῇ τῇ ἡμέρᾳ ταύτῃ· ἔστι γὰρ ἡμέρα ἐξιλασμοῦ αὕτη ὑμῖν, ἐξιλάσασθαι περὶ ὑμῶν ἔναντι Κυρίου τοῦ Θεοῦ ὑμῶν.
\vs{29}Πᾶσα ψυχὴ, ἥτις μὴ ταπεινωθήσεται ἐν αὐτῇ τῇ ἡμέρᾳ ταύτῃ, ἐξολοθρευθήσεται ἐκ τοῦ λαοῦ αὐτῆς.
\vs{30}Καὶ πᾶσα ψυχὴ, ἥτις ποιήσει ἔργον ἐν αὐτῇ τῇ ἡμέρᾳ ταύτῃ, ἀπολεῖται ἡ ψυχὴ ἐκείνη ἐκ τοῦ λαοῦ αὐτῆς.
\vs{31}Πᾶν ἔργον οὐ ποιήσετε· νόμιμον αἰώνιον εἰς τὰς γενεὰς ὑμῶν ἐν πάσαις κατοικίαις ὑμῶν.
\vs{32}Σάββατα σαββάτων ἔσται ὑμῖν· καὶ ταπεινώσετε τὰς ψυχὰς ὑμῶν· ἀπὸ ἐνάτης τοῦ μηνὸς, ἀπὸ ἑσπέρας ἕως ἑσπέρας σαββατιεῖτε τὰ σάββατα ὑμῶν.

\vs{33}Καὶ ἐλάλησε Κύριος πρὸς Μωυσῆν, λέγων,
\vs{34}λάλησον τοῖς υἱοῖς Ἰσραὴλ, λέγων, τῇ πεντεκαιδεκάτῃ τοῦ μηνὸς τοῦ ἐβδόμου τούτου, ἑορτὴ σκηνῶν ἑπτὰ ἡμέρας τῷ Κυρίῳ.
\vs{35}Καὶ ἡ ἡμέρα ἡ πρώτη κλητὴ ἁγία· πᾶν ἔργον λατρευτὸν οὐ ποιήσετε.
\vs{36}Ἑπτὰ ἡμέρας προσάξετε ὁλοκαυτώματα τῷ Κυρίῳ, καὶ ἡ ἡμέρα ἡ ὀγδόη κλητὴ ἁγία ἔσται ὑμῖν· καὶ προσάξετε ὁλοκαυτώματα Κυρίῳ· ἐξόδιόν ἐστι· πᾶν ἔργον λατρευτὸν οὐ ποιήσετε.
\vs{37}Αὗται ἑορταὶ Κυρὶῳ, ἃς καλέσετε κλητὰς ἁγίας, ὥστε προσενέγκαι καρπώματα τῷ Κυρίῳ, ὁλοκαυτώματα καὶ θυσίας αὐτῶν, καὶ σπονδὰς αὐτῶν τὸ καθʼ ἡμέραν εἰς ἡμέραν·
\vs{38}πλὴν τῶν σαββάτων Κυρίου, καὶ πλὴν τῶν δομάτων ὑμῶν, καὶ πλὴν πασῶν τῶν εὐχῶν ὑμῶν, καὶ πλὴν τῶν ἐκουσίων ὑμῶν, ἃ ἂν δώτε τῷ Κυρίῳ.
\vs{39}Καὶ ἐν τῇ πεντεκαιδεκάτῃ ἡμέρᾳ τοῦ μηνὸς τοῦ ἑβδόμου τούτου, ὅταν συντελέσητε τὰ γεννήματα τῆς γῆς, ἑορτάσατε τῷ Κυρίῳ ἑπτὰ ἡμέρας· τῇ ἡμέρᾳ τῇ πρώτῃ ἀνάπαυσις, καὶ τῇ ἡμέρᾳ τῇ ὀγδόῃ ἀνάπαυσις.
\vs{40}Καὶ λήψεσθε τῇ ἡμέρᾳ τῇ πρώτῃ καρπὸν ξύλου ὡραῖου, καὶ κάλλυνθρα φοινίκων, καὶ κλάδους ξύλου δασεῖς, καὶ ἰτέας, καὶ ἄγνου κλάδους ἐκ χειμάῤῥου, εὐφρανθῆναι ἔναντι Κυρίου τοῦ Θεοῦ ὑμῶν ἑπτὰ ἡμέρας τοῦ ἐνιαυτοῦ.
\vs{41}Νόμιμον αἰώνιον εἰς τὰς γενεὰς ὑμῶν· ἐν τῷ μηνὶ τῷ ἑβδόμῳ ἑορτάσετε αὐτήν.
\vs{42}Ἐν σκηναῖς κατοικήσετε ἑπτὰ ἡμέρας· πᾶς ὁ αὐτόχθων ἐν Ἰσραὴλ κατοικήσει ἐν σκηναῖς,
\vs{43}ὅπως ἴδωσιν αἱ γενεαὶ ὑμῶν, ὅτι ἐν σκηναῖς κατῴκισα τοὺς υἱοὺς Ἰσραὴλ, ἐν τῷ ἐξαγαγεῖν με αὐτοὺς ἐκ γῆς Αἰγύπτου· ἐγὼ Κύριος ὁ Θεὸς ὑμῶν.
\vs{44}Καὶ ἐλάλησε Μωυσῆς τὰς ἑορτὰς Κυρίου τοῖς υἱοῖς Ἰσραήλ.

\ch{24}
Καὶ ἐλάλησε Κύριος πρὸς Μωυσῆν, λέγων,
\vs{2}ἔντειλαι τοῖς υἱοῖς Ἰσραὴλ, καὶ λαβέτωσάν σοι ἔλαιον ἐλάϊνον καθαρὸν κεκομμένον εἰς φῶς, καῦσαι λύχνον διαπαντὸς,
\vs{3}ἔξωθεν τοῦ καταπετάσματος ἐν τῇ σκηνῇ τοῦ μαρτυρίου· καὶ καύσουσιν αὐτὸ Ἀαρὼν καὶ οἱ υἱοὶ αὐτοῦ ἀπὸ ἑσπέρας ἕως πρωῒ ἐνώπιον Κυρίου ἐνδελεχῶς, νόμιμον αἰώνιον εἰς τὰς γενεὰς ὑμῶν.
\vs{4}Ἐπὶ τῆς λυχνίας τῆς καθαρᾶς καύσετε τοὺς λύχνους ἐναντίον Κυρίου ἕως εἰς τοπρωΐ.
\vs{5}Καὶ λήμψεσθε σεμίδαλιν, καὶ ποιήσετε αὐτὴν δώδεκα ἄρτους· δύο δεκάτων ἔσται ὁ ἄρτος ὁ εἷς.
\vs{6}Καὶ ἐπιθήσετε αὐτοὺς δύο θέματα, ἓξ ἄρτους τὸ ἓν θέμα ἐπὶ τὴν τράπεζαν τὴν καθαρὰν ἔναντι Κυρίου.
\vs{7}Καὶ ἐπιθήσετε ἐπὶ τὸ θέμα λίβανον καθαρὸν καὶ ἅλα, καὶ ἔσονται εἰς ἄρτους εἰς ἀνάμνησιν προκείμενα τῷ Κυρίῳ.
\vs{8}Τῇ ἡμέρᾳ τῶν σαββάτων προσθήσεται ἔναντι Κυρίου διαπαντὸς ἐνώπιον τῶν υἱῶν Ἰσραὴλ, διαθήκην αἰώνιον.
\vs{9}Καὶ ἔσται Ἀαρὼν καὶ τοῖς υἱοῖς αὐτοῦ· καὶ φάγονται αὐτὰ ἐν τόπῳ ἁγίῳ· ἔστι γὰρ ἅγια τῶν ἁγίῳν τοῦτο αὐτῶν ἀπὸ τῶν θυσιαζομένων τῷ Κυρίῳ, νόμιμον αἰώνιον.

\vs{10}Καὶ ἐξῆλθεν υἱὸς γυναικὸς Ἰσραηλίτιδος, καὶ οὗτος ἦν υἱὸς Αἰγυπτίου ἐν τοῖς υἱοῖς Ἰσραήλ· καὶ ἐμαχέσαντο ἐν τῇ παρεμβολῇ ὁ ἐκ τῆς Ἰσραηλίτιδος, καὶ ὁ ἄνθρωπος ὁ Ἰσραηλίτης.
\vs{11}Καὶ ἐπονομάσας ὁ υἱὸς τῆς γυναικὸς τῆς Ἰσραηλίτιδος τὸ ὄνομα κατηράσατο· καὶ ἤγαγον αὐτὸν πρὸς Μωυσῆν· καὶ τὸ ὄνομα τῆς μητρὸς αὐτοῦ Σαλωμεὶθ θυγάτηρ Δαβρεὶ ἐκ τῆς φυλῆς Δάν.
\vs{12}Καὶ ἀπέθεντο αὐτὸν εἰς φυλακὴν διακρῖναι αὐτὸν διὰ προστάγματος Κυρίου.
\vs{13}Καὶ ἐλάλησε Κύριος πρὸς Μωυσῆν, λέγων,
\vs{14}ἐξάγαγε τὸν καταρασάμενον ἔξω τῆς παρεμβολῆς, καὶ ἐπιθήσουσι πάντες οἱ ἀκούσαντες τὰς χεῖρας αὐτῶν ἐπὶ τὴν κεφαλὴν αὐτοῦ, καὶ λιθοβολήσουσιν αὐτὸν πᾶσα ἡ συναγωγή.
\vs{15}Καὶ τοῖς υἱοῖς Ἰσραὴλ λάλησον, καὶ ἐρεῖς πρὸς αὐτοὺς, ἄνθρωπος ὃς ἐὰν καταράσηται Θεὸν, ἁμαρτίαν λήψεται.
\vs{16}Ὀνομάζων δὲ τὸ ὄνομα Κυρίου, θανάτῳ θανατούσθω· λίθοις λιθοβολείτω αὐτὸν πᾶσα ἡ συναγωγὴ Ἰσραήλ· ἐάν τε προσήλυτος ἐάν τε αὐτόχθων, ἐν τῷ ὀνομάσαι αὐτὸν τὸ ὄνομα Κυρίου, τελευτάτω.
\vs{17}Καὶ ἄνθρωπος ὃς ἂν πατάξῃ ψυχὴν ἀνθρώπου, καὶ ἀποθάνῃ, θανάτῳ θανατούσθω.
\vs{18}Καὶ ὃς ἂν πατάξῃ κτῆνος, καὶ ἀποθάνῃ, ἀποτισάτω ψυχὴν ἀντὶ ψυχῆς.
\vs{19}Καὶ ἐάν τις δῷ μῶμον τῷ πλησίον, ὡς ἐποίησεν αὐτῷ, ὡσαύτως ἀντιποιηθήσεται αὐτῷ·
\vs{20}Σύντριμμα ἀντὶ συντρίμματος, ὀφθαλμὸν ἀντὶ ὀφθαλμοῦ, ὀδόντα ἀντὶ ὀδόντος, καθότι ἂν δῷ μῶμον τῷ ἀνθρώπῳ, οὕτω δοθήσεται αὐτῷ.
\vs{21}Ὃς ἂν πατάξῃ ἄνθρωπον, καὶ ἀποθάνῃ, θανάτῳ θανατούσθω.
\vs{22}Δικαίωσις μία ἔσται τῷ προσηλύτῳ καὶ τῷ ἐγχωρίῳ, ὅτι ἐγώ εἰμι Κύριος ὁ Θεὸς ὑμῶν.
\vs{23}Καὶ ἐλάλησε Μωυσῆς τοῖς υἱοῖς Ἰσραήλ· καὶ ἐξήγαγον τὸν καταρασάμενον ἔξω τῆς παρεμβολῆς, καὶ ἐλιθοβόλησαν αὐτὸν ἐν λίθοις· καὶ οἱ υἱοὶ Ἰσραὴλ ἐποίησαν καθάπερ συνέταξε Κύριος τῷ Μωυσῇ.

\ch{25}
Καὶ ἐλάλησε Κύριος πρὸς Μωυσῆν ἐν τῷ ὄρει Σινᾷ, λέγων,
\vs{2}λάλησον τοῖς υἱοῖς Ἰσραὴλ, καὶ ἐρεῖς πρὸς αὐτοὺς, ὅταν εἰσέλθητε εἰς τὴν γῆν, ἣν ἐγὼ δίδωμι ὑμῖν, καὶ ἀναπαύσεται ἡ γῆ, ἣν ἐγὼ δίδωμι ὑμῖν, σάββατα τῷ Κυρίῳ.
\vs{3}Ἓξ ἔτη σπερεῖς τὸν ἀγρόν σου, καὶ ἓξ ἔτη τεμεῖς τὴν ἄμπελόν σου, καὶ συνάξεις τὸν καρπὸν αὐτῆς.
\vs{4}Τῷ δὲ ἔτει τῷ ἑβδόμῳ σάββατα· ἀνάπαυσις ἔσται τῇ γῇ, σάββατα τῷ Κυρίῳ· τὸν ἀγρόν σου οὐ σπερεῖς, καὶ τὴν ἄμπελόν σου οὐ τεμεῖς,
\vs{5}καὶ τὰ αὐτόματα ἀναβαίνοντα τοῦ ἀγροῦ σου οὐκ ἐκθερίσεις, καἰ τὴν σταφυλὴν τοῦ ἁγιάσματός σου οὐκ ἐκτρυγήσεις· ἐνιαυτὸς ἀναπαύσεως ἔσται τῇ γῇ.
\vs{6}Καὶ ἔσται τὰ σάββατα τῆς γῆς βρώματά σοι, καὶ τῷ παιδί σου, καὶ τῇ παιδίσκῃ σου, καὶ τῷ μισθωτῷ σου, καὶ τῷ παροίκῳ τῷ προσκειμένῳ πρὸς σέ.
\vs{7}Καὶ τοῖς κτήνεσί σου, καὶ τοῖς θηρίοις τοῖς ἐν τῇ γῇ σου ἔσται πᾶν τὸ γέννημα αὐτοῦ εἰς βρῶσιν.

\vs{8}Καὶ ἐξαριθμήσεις σεαυτῷ ἑπτὰ ἀναπαύσεις ἐτῶν, ἑπτὰ ἔτη ἑπτάκις· καὶ ἔσονταί σοι ἑπτὰ ἑβδομάδες ἐτῶν ἐννέα καὶ τεσσαράκοντα ἔτη.
\vs{9}Διαγγελεῖτε σάλπιγγος φωνῇ ἐν πάσῃ τῇ γῇ ὑμῶν ἐν τῷ μηνὶ τῷ ἑβδόμῳ τῇ δεκάτῃ τοῦ μηνός· τῇ ἡμέρᾳ τοῦ ἱλασμοῦ διαγγελεῖτε σάλπιγγι ἐν πάσῃ τῇ γῇ ὑμῶν.
\vs{10}Καὶ ἁγιάσετε τὸ ἔτος τὸν πεντηκοστὸν ἐνιαυτὸν, καὶ διαβοήσετε ἄφεσιν ἐπὶ τῆς γῆς πᾶσι τοῖς κατοικοῦσιν αὐτήν· ἐνιαυτὸς ἀφέσεως σημασία αὕτη ἔσται ὑμῖν· καὶ ἀπελεύσεται εἷς ἕκαστος εἰς τὴν κτῆσιν αὐτοῦ, καὶ ἕκαστος εἰς τὴν πατριὰν αὐτοῦ ἀπελεύσεσθε.
\vs{11}Ἀφέσεως σημασία αὕτη, τὸ ἔτος τὸ πεντηκοστὸν ἐνιαυτὸς ἔσται ὑμῖν· οὐ σπερεῖτε, οὐδὲ ἀμήσετε τὰ αὐτόματα ἀναβαίνοντα αὐτῆς, καὶ οὐ τρυγήσετε τὰ ἡγιασμένα αὐτῆς,
\vs{12}ὅτι ἀφέσεως σημασία ἐστίν· ἅγιον ἔσται ὑμῖν· ἀπὸ τῶν πεδίων φάγεσθε τὰ γεννήματα αὐτῆς.
\vs{13}Ἐν τῷ ἔτει τῆς ἀφέσεως σημασίας αὐτῆς ἐπανελεύσεται εἰς τὴν ἔγκτησιν αὐτοῦ.
\vs{14}Ἐὰν δὲ ἀποδῷ πράσιν τῷ πλησίον σου, ἐὰν δὲ καὶ κτήσῃ παρὰ τοῦ πλησίον σου, μὴ θλιβέτω ἄνθρωπος τὸν πλησίον.
\vs{15}Κατὰ ἀριθμὸν ἐτῶν μετὰ τὴν σημασίαν κτήσῃ παρὰ τοῦ πλησίον, κατὰ ἀριθμὸν ἐνιαυτῶν γεννημάτων ἀποδώσεταί σοι.
\vs{16}Καθότι ἂν πλεῖον τῶν ἐτῶν πληθυνεῖ τὴν ἔγκτησιν αὐτοῦ, καὶ καθότι ἂν ἔλαττον τῶν ἐτῶν ἐλαττονώσει τὴν ἔγκτησιν αὐτοῦ· ὅτι ἀριθμὸν γεννημάτων αὐτοῦ, οὕτως ἀποδώσεταί σοι.
\vs{17}Μὴ θλιβέτω ἄνθρωπος τὸν πλησίον· καὶ φοβηθήσῃ Κύριον τὸν Θεόν σου· ἐγώ εἰμι Κύριος ὁ Θεὸς ὑμῶν.

\vs{18}Καὶ ποιήσετε πάντα τὰ δικαιώματά μου, καὶ πάσας τὰς κρίσεις μου, καὶ φυλάξασθε, καὶ ποιήσετε αὐτὰ, καὶ κατοικήσετε ἐπὶ τῆς γῆς πεποιθότες.
\vs{19}Καὶ δώσει ἡ γῆ τὰ ἐκφόρια αὐτῆς, καὶ φάγεσθε εἰς πλησμονὴν, καὶ κατοικήσετε πεποιθότες ἐπʼ αὐτῆς.
\vs{20}Ἐὰν δὲ λέγητε, τί φαγόμεθα ἐν τῷ ἔτει τῷ ἑβδόμῳ τούτῳ, ἐὰν μὴ σπείρωμεν μηδὲ συναγάγωμεν τὰ γεννήματα ἡμῶν;
\vs{21}Καὶ ἀποστέλλω τὴν εὐλογίαν μου ὑμῖν ἐν τῷ ἔτει τῷ ἕκτῳ, καὶ ποιήσει τὰ γεννήματα αὐτῆς εἰς τὰ τρία ἔτη.
\vs{22}Καὶ σπερεῖτε τὸ ἔτος τὸ ὄγδοον, καὶ φάγεσθε ἀπὸ τῶν γεννημάτων παλαιὰ ἕως τοῦ ἔτους τοῦ ἐνάτου· ἕως ἂν ἔλθῃ τὸ γέννημα αὐτῆς, φάγεσθε παλαιὰ παλαιῶν.
\vs{23}Καὶ ἡ γῆ οὐ πραθήσεται εἰς βεβαίωσιν· ἐμὴ γάρ ἐστιν ἡ γῆ, διότι προσήλυτοι καὶ πάροικοι ὑμεῖς ἐστε ἐναντίον μου.
\vs{24}Καὶ κατὰ πᾶσαν γῆν κατασχέσεως ὑμῶν, λύτρα δώσετε τῆς γῆς.
\vs{25}Ἐὰν δὲ πένηται ὁ ἀδελφός σου ὁ μετὰ σοῦ, καὶ ἀποδῶται ἀπὸ τῆς κατασχέσεως αὐτοῦ, καὶ ἔλθῃ ὁ ἀγχιστεύων ὁ ἐγγίζων αὐτῷ, καὶ λυτρώσεται τὴν πρᾶσιν τοῦ ἀδελφοῦ αὐτοῦ.
\vs{26}Ἐὰν δὲ μὴ ᾖ τινι ὁ ἀγχιστεύων, καὶ εὐπορηθῇ τῇ χειρὶ, καὶ εὑρεθῇ αὐτῷ τὸ ἱκανὸν, λύτρα αὐτοῦ·
\vs{27}καὶ συλλογιεῖται τὰ ἔτη τῆς πράσεως αὐτοῦ, καὶ ἀποδώσει ὅ ὑπερέχει τῷ ἀνθρώπῳ, ᾧ ἀπέδοτο αὐτὸ αὐτῷ, καὶ ἀπελεύσεται εἰς τὴν κατάσχεσιν αὐτοῦ.
\vs{28}Ἐὰν δὲ μὴ εὑπορηθῇ αὐτοῦ ἡ χεὶρ τὸ ἱκανὸν, ὥστε ἀποδοῦναι αὐτῷ, καὶ ἔσται ἡ πράσις τῷ κτησαμένῳ αὐτὰ ἕως τοῦ ἕκτου ἔτους τῆς ἀφέσεως, καὶ ἐξελεύσεται ἐν τῇ ἀφέσει, καὶ ἀπελεύσεται εἰς τὴν κατάσχεσιν αὐτοῦ.

\vs{29}Ἐὰν δέ τις ἀποδῶται οἰκίαν οἰκητὴν ἐν πόλει τετειχισμένῃ, καὶ ἔσται ἡ λύτρωσις αὐτῆς, ἕως πληρωθῇ· ἐνιαυτὸς ἡμερῶν ἔσται ἡ λύτρωσις αὐτῆς.
\vs{30}Ἐὰν δὲ μὴ λυτρωθῇ ἕως ἂν πληρωθῇ αὐτῆς ἐνιαυτὸς ὅλος, κυρωθήσεται ἡ οἰκία ἡ οὖσα ἐν πόλει τῇ ἐχούσῃ τεῖχος, βεβαίως τῷ κτησαμένῳ αὐτὴν εἰς τὰς γενεὰς αὐτοῦ, καὶ οὐκ ἐξελεύσεται ἐν τῇ ἀφέσει.
\vs{31}Αἱ δὲ οἰκίαι αἱ ἐν ἐπαύλεσιν, αἷς οὐκ ἔστιν ἐν αὐταῖς τεῖχος κύκλῳ, πρὸς τὸν ἀγρὸν τῆς γῆς λογισθήσονται· λυτρωταὶ διαπαντὸς ἔσονται, καὶ ἐν τῇ ἀφέσει ἐξελεύσονται.
\vs{32}Καὶ αἱ πόλεις τῶν Λευιτῶν, οἰκίαι τῶν πόλεων κατασχέσεως αὐτῶν, λυτρωταὶ διαπαντὸς ἔσονται τοῖς Λευίταις.
\vs{33}Καὶ ὃς ἂν λυτρώσηται παρὰ τῶν Λευιτῶν, καὶ ἐξελεύσεται ἡ διάπρασις αὐτῶν οἰκιῶν πόλεως κατασχέσεως αὐτῶν ἐν τῇ ἀφέσει, ὅτι οἰκίαι τῶν πόλεων τῶν Λευιτῶν κατάσχεσις αὐτῶν ἐν μέσῳ υἱῶν Ἰσραήλ.
\vs{34}Καὶ οἱ ἀγροὶ ἀφωρισμένοι ταῖς πόλεσιν αὐτῶν οὐ πραθήσονται, ὅτι κατάσχεσις αἰωνία τοῦτο αὐτῶν ἐστον.

\vs{35}Ἐὰν δὲ πένηται ὁ ἀδελφός σοῦ ὁ μετὰ σοῦ, καὶ ἀδυνατήσῃ ταῖς χερσὶ παρὰ σοὺ, ἀντιλήψῃ αὐτοῦ ὡς προσηλύτου καὶ παροίκου, καὶ ζήσεται ὁ ἀδελφός σου μετὰ σοῦ.
\vs{36}Οὐ λήψῃ παρʼ αὐτοῦ τόκον, οὐδὲ ἐπὶ πλήθει· καὶ φοβηθήσῃ τὸν Θεόν σου· ἐγὼ Κύριος· καὶ ζήσεται ὁ ἀδελφός σου μετὰ σοῦ.
\vs{37}Τὸ ἀργύριόν σου οὐ δώσεις αὐτῷ ἐπὶ τὸκῳ, καὶ ἐπὶ πλεονασμῷ οὐ δώσεις αὐτῷ τὰ βρώματά σου.
\vs{38}Ἐγὼ Κύριος ὁ Θεὸς ὑμῶν, ὁ ἐξαγαγὼν ὑμᾶς ἐκ γῆς Αἰγύπτου, δοῦναι ὑμῖν τὴν γῆν Χαναὰν, ὥστε εἶναι ὑμῶν Θεός.

\vs{39}Ἐὰν δὲ ταπεινωθῇ ὁ ἀδελφός σου παρὰ σοὶ, καὶ πραθῇ σοι, οὐ δουλεύσει σοι δουλείαν οἰκέτου.
\vs{40}Ὡς μισθωτὸς ἢ πάροικος ἔσται σοι· ἕως τοῦ ἔτους τῆς ἀφέσεως ἐργᾶται παρὰ σοί,
\vs{41}καὶ ἐξελεύσεται τῇ ἀφέσει, καὶ τὰ τέκνα αὐτοῦ μετʼ αὐτοῦ, καὶ ἀπελεύσεται εἰς τὴν γενεὰν αὐτοῦ, εἰς τὴν κατάσχεσιν τὴν πατρικὴν ἀποδραμεῖται.
\vs{42}Διότι οἰκέται μου εἰσὶν οὗτοι, οὓς ἐξήγαγον ἐκ γῆς Αἰγύπτου· οὐ πραθήσεται ἐν πράσει οἰκέτου.
\vs{43}Οὐ κατατενεῖς αὐτὸν ἐν τῷ μόχθῳ, καὶ φοβηθήσῃ Κύριον τὸν Θεόν σου,
\vs{44}καὶ παῖς καὶ παιδίσκη ὅσοι ἂν γένωνταί σοι, ἀπὸ τῶν ἐθνῶν ὅσοι κύκλῳ σου εἰσὶν, ἀπʼ αὐτῶν κτήσεσθε δοῦλον καὶ δούλην,
\vs{45}καὶ ἀπὸ τῶν υἱῶν τῶν παροίκων τῶν ὄντων ἐν ὑμῖν, ἀπὸ τούτων κτήσεσθε καὶ ἀπὸ τῶν συγγενῶν αὐτῶν, ὅσοι ἂν γένωνται ἐν τῇ γῇ ὑμῶν, ἔστωσαν ὑμῖν εἰς κατάσχεσιν.
\vs{46}Καὶ καταμεριεῖτε αὐτοὺς τοῖς τέκνοις ὑμῶν μεθʼ ὑμᾶς· καὶ ἔσονται ὑμῖν κατόχιμοι εἰς τὸν αἰῶνα· τῶν δὲ ἀδελφῶν ὑμῶν τῶν υἱῶν Ἰσραὴλ, ἕκαστος τὸν ἀδελφὸν αὐτοῦ οὐ κατατενεῖ αὐτὸν ἐν τοῖς μόχθοις.

\vs{47}Ἐὰν δὲ εὕρῃ ἡ χεὶρ τοὺ προσηλύτου ἢ τοῦ παροίκου τοῦ παρὰ σοὶ, καὶ ἀπορηθεὶς ὁ ἀδελφός σου πραθῇ τῷ προσηλύτῳ ἢ τῷ παροίκῳ τῷ παρὰ σοὶ, ἢ ἐκ γενετῆς προσηλύτῳ,
\vs{48}μετὰ τὸ πραθῆναι αὐτῷ, λύτρωσις ἔσται αὐτοῦ· εἷς τῶν ἀδελφῶν αὐτοῦ λυτρώσεται αὐτόν.
\vs{49}Ἀδελφὸς πατρὸς αὐτοῦ, ἢ υἱὸς ἀδελφοῦ πατρὸς λυτρώσεται αὐτόν, ἢ ἀπὸ τῶν οἰκείων τῶν σαρκῶν αὐτοῦ ἐκ τῆς φυλῆς αὐτοῦ λυτρῶται αὐτόν· ἐὰν δὲ εὐπορηθεὶς ταῖς χερσὶ λυτρῶται ἑαυτὸν,
\vs{50}καὶ συλλογιεῖται πρὸς τὸν κεκτημένον αὐτὸν ἀπὸ τοῦ ἔτους οὗ ἀπέδοτο ἑαυτὸν αὐτῷ ἕως τοῦ ἐνιαυτοῦ τῆς ἀφέσεως· καὶ ἔσται τὸ ἀργύριον τῆς πράσεως αὐτοῦ ὡς μισθίου· ἔτος ἐξ ἔτους ἔσται μετʼ αὐτοῦ.
\vs{51}Ἐὰν δέ τινι πλεῖον τῶν ἐτῶν ᾖ, πρὸς ταῦτα ἀποδώσει τὰ λύτρα αὐτοῦ ἀπὸ τοῦ ἀργυρίου τῆς πράσεως αὐτοῦ.
\vs{52}Ἐὰν δὲ ὀλίγον καταλειφθῇ ἀπὸ τῶν ἐτῶν εἰς τὸν ἐνιαυτὸν τῆς ἀφέσεως, καὶ συλλογιεῖται αὐτῷ κατὰ τὰ ἔτη αὐτοῦ, καὶ ἀποδώσει τὰ λύτρα αὐτοῦ ὡς μισθωτός·
\vs{53}ἐνιαυτὸς ἐξ ἐνιαυτοῦ ἔσται μετʼ αὐτοῦ· οὐ κατατενεῖς αὐτὸν ἐν τῷ μόχθῳ ἐνώπιόν σου.
\vs{54}Ἐὰν δὲ μὴ λυτρῶται κατὰ ταῦτα, ἐξελεύσεται ἐν τῷ ἔτει τῆς ἀφέσεως αὐτὸς καὶ τὰ παιδία αὐτοῦ μετʼ αὐτοῦ.
\vs{55}Ὅτι ἐμοὶ οἱ υἱοὶ Ἰσραὴλ οἰκέται εἰσὶ, παῖδές μου οὗτοί εἰσιν, οὓς ἐξήγαγον ἐκ γῆς Αἰγύπτου.

\ch{26}
Ἐγὼ Κύριος ὁ Θεὸς ὑμῶν· οὐ ποιήσετε ὑμῖν αὐτοῖς χειροποίητα, οὐδὲ γλυπτὰ, οὐδὲ στήλην ἀναστήσετε ὑμῖν, οὐδὲ λίθον σκοπὸν θήσετε ἐν τῇ γῇ ὑμῶν προσκυνῆσαι αὐτῷ· ἐγώ εἰμι Κύριος ὁ Θεὸς ὑμῶν.
\vs{2}Τὰ σάββατά μου φυλάξεσθε, καὶ ἀπὸ τῶν ἁγίων μου φοβηθήσεσθε· ἐγώ εἰμι Κύριος.
\vs{3}Ἐὰν τοῖς προστάγμασί μου πορεύησθε, καὶ τὰς ἐντολάς μου φυλάσσησθε, καὶ ποιήσητε αὐτὰς,
\vs{4}καὶ δώσω τὸν ὑετὸν ὑμῖν ἐν καιρῷ αὐτοῦ, καὶ ἡ γῆ δώσει τὰ γεννήματα αὐτῆς, καὶ τὰ ξύλα τῶν πεδίων ἀποδώσει τὸν καρπὸν αὐτῶν·
\vs{5}Καὶ καταλήψεται ὑμῖν ὁ ἁλοητὸς τὸν τρυγητὸν, καὶ ὁ τρυγητὸς καταλήψεται τὸν σπόρον· καὶ φάγεσθε τὸν ἄρτον ὑμῶν εἰς πλησμονήν· καὶ κατοικήσετε μετὰ ἀσφαλείας ἐπὶ τῆς γῆς ὑμῶν, καὶ πόλεμος οὐ διελεύσεται διὰ τῆς γῆς ὑμῶν·
\vs{6}Καὶ δώσω εἰρήνην ἐν τῇ γῇ ὑμῶν· καὶ κοιμηθήσεσθε, καὶ οὐκ ἔσται ὑμᾶς ὁ ἐκφοβῶν· καὶ ἀπολῶ θηρία πονηρὰ ἐκ τῆς γῆς ὑμῶν.
\vs{7}Καὶ διώξεσθε τοὺς ἐχθροὺς ὑμῶν, καὶ πεσοῦνται ἐναντίον ὑμῶν φόνῳ.
\vs{8}Καὶ διώξονται ἐξ ὑμῶν πέντε ἑκατὸν, καὶ ἑκατὸν ὑμῶν διώξονται μυριάδας· καὶ πεσοῦνται οἱ ἐχθροὶ ὑμῶν ἐναντίον ὑμῶν μαχαίρᾳ.
\vs{9}Καὶ ἐπιβλέψω ἐφʼ ὑμᾶς, καὶ αὐξανῶ ὑμᾶς, καὶ πληθυνῶ ὑμᾶς, καὶ στήσω τὴν διαθήκην μου μεθʼ ὑμῶν·
\vs{10}Καὶ φάγεσθε παλαιὰ καὶ παλαιὰ παλαιῶν, καὶ παλαιὰ ἐκ προσώπου νέων ἐξοίσετε.
\vs{11}Καὶ θήσω τὴν σκηνήν μου ἐν ὑμῖν, καὶ οὐ βδελύξεται ἡ ψυχή μου ὑμᾶς,
\vs{12}καὶ ἐμπεριπατήσω ἐν ὑμῖν· καὶ ἔσομαι ὑμῶν Θεὸς, καὶ ὑμεῖς ἔσεσθέ μοι λαός.
\vs{13}Ἐγώ εἰμι Κύριος ὁ Θεὸς ὑμῶν, ὁ ἐξαγαγὼν ὑμᾶς ἐκ γῆς Αἰγύπτου, ὄντων ὑμῶν δούλων· καὶ συνέτριψα τὸν δεσμὸν τοῦ ζυγοῦ ὑμῶν, καὶ ἤγαγον ὑμᾶς μετὰ παῤῥησίας.

\vs{14}Ἐὰν δὲ μὴ ὑπακούσητέ μου, μηδὲ ποιήσητε τὰ προστάγματά μου ταῦτα,
\vs{15}ἀλλὰ ἀπειθήσητε αὐτοῖς, καὶ τοῖς κρίμασί μου προσοχθίσῃ ἡ ψυχὴ ὑμῶν, ὥστε ὑμᾶς μὴ ποιεῖν πάσας τὰς ἐντολάς μου, ὥστε διασκεδάσαι τὴν διαθήκην μου,
\vs{16}καὶ ἐγὼ ποιήσω οὕτως ὑμῖν· καὶ ἐπιστήσω ἐφʼ ὑμᾶς τὴν ἀπορίαν, τήν τε ψώραν, καὶ τὸν ἴκτερα σφακελίζοντα τοὺς ὀφθαλμοὺς ὑμῶν, καὶ τὴν ψυχὴν ὑμῶν ἐκτήκουσαν· καὶ σπερεῖτε διακενῆς τὰ σπέρματα ὑμῶν, καὶ ἔδονται οἱ ὑπεναντίοι ὑμῶν.
\vs{17}Καὶ ἐπιστήσω τὸ πρόσωπόν μου ἐφʼ ὑμᾶς, καὶ πεσεῖσθε ἐναντίον τῶν ἐχθρῶν ὑμῶν, καὶ διώξονται ὑμᾶς οἱ μισοῦντες ὑμᾶς, καὶ φεύξεσθε οὐδενὸς διώκοντος ὑμᾶς.
\vs{18}Καὶ ἐὰν ἕως τούτου μὴ ὑπακούσητέ μου, καὶ προσθήσω τοῦ παιδεῦσαι ὑμᾶς ἑπτάκις ἐπὶ ταῖς ἁμαρτίαις ὑμῶν.
\vs{19}Καὶ συντρίψω τὴν ὕβριν τῆς ὑπερηφανίας ὑμῶν· καὶ θήσω τὸν οὐρανὸν ὑμῖν σιδηροῦν, καὶ τὴν γῆν ὑμῶν ὡσεὶ χαλκῆν.
\vs{20}Καὶ ἔσται εἰς κενὸν ἡ ἰσχὺς ὑμῶν· καὶ οὐ δώσει ἡ γῆ ὑμῶν τὸν σπόρον αὐτῆς, καὶ τὸ ξύλον τοῦ ἀγρου ὑμῶν οὐ δώσει τὸν καρπὸν αὐτοῦ.

\vs{21}Καὶ ἐὰν μετὰ ταῦτα πορεύησθε πλάγιοι, καὶ μὴ βούλησθε ὑπακούειν μου, προσθήσω ὑμῖν πληγὰς ἑπτὰ κατὰ τὰς ἁμαρτίας ὑμῶν.
\vs{22}Καὶ ἀποστέλλω ἐφʼ ὑμᾶς τὰ θηρία τὰ ἄγρια τῆς γῆς, καὶ κατέδεται ὑμᾶς, καὶ ἐξαναλώσει τὰ κτήνη ὑμῶν, καὶ ὀλιγοστοὺς ποιήσω ὑμᾶς, καὶ ἐρημωθήσονται αἱ ὁδοὶ ὑμῶν.
\vs{23}Καὶ ἐπὶ τούτοις ἐὰν μὴ παιδευθῆτε, ἀλλὰ πορεύησθε πρός με πλάγιοι,
\vs{24}πορεύσομαι κᾀγὼ μεθʼ ὑμῶν θυμῷ πλαγίῳ, καὶ πατάξω ὑμᾶς κᾀγὼ ἑπτάκις ἀντὶ τῶν ἁμαρτιῶν ὑμῶν.
\vs{25}Καὶ ἐπάξω ἐφʼ ὑμᾶς μάχαιραν ἐκδικοῦσαν δίκην διαθήκης, καὶ καταφεύξεσθε εἰς τὰς πόλεις ὑμῶν· καὶ ἐξαποστελῶ θάνατον εἰς ὑμᾶς, καὶ παραδοθήσεσθε εἰς χεῖρας τῶν ἐχθρῶν.
\vs{26}Ἐν τῷ θλίψαι ὑμᾶς σιτοδείᾳ ἄρτων, καὶ πέψουσι δέκα γυναῖκες τοὺς ἄρτους ὑμῶν ἐν κλιβάνῳ ἑνὶ, καὶ ἀποδώσουσι τοὺς ἄρτους ὑμῶν ἐν σταθμῷ, καὶ φάγεσθε, καὶ οὐ μὴ ἐμπλησθῆτε.

\vs{27}Ἐὰν δὲ ἐπὶ τούτοις μὴ ὑπακούσητέ μου, καὶ πορεύησθε πρός με πλάγιοι,
\vs{28}καὶ αὐτὸς πορεύσομαι μεθʼ ὑμῶν ἐν θυμῷ πλαγίῳ, καὶ παιδεύσω ὑμᾶς ἐγὼ ἑπτάκις κατὰ τὰς ἁμαρτίας ὑμῶν.
\vs{29}Καὶ φάγεσθε τὰς σάρκας τῶν υἱῶν ὑμῶν, καὶ τὰς σάρκας τῶν θυγατέρων ὑμῶν φάγεσθε.
\vs{30}Καὶ ἐρημώσω τὰς στήλας ὑμῶν, καὶ ἐξολοθρεύσω τὰ ξύλινα χειροποίητα ὑμῶν, καὶ θήσω τὰ κῶλα ὑμῶν ἐπὶ τὰ κῶλα τῶν εἰδώλων ὑμῶν, καὶ προσοχθιεῖ ἡ ψυχή μου ὑμῖν.
\vs{31}Καὶ θήσω τὰς πόλεις ὑμῶν ἐρήμους, καὶ ἐξερημώσω τὰ ἅγια ὑμῶν, καὶ οὐ μὴ ὀσφρανθῶ τῆς ὀσμῆς τῶν θυσιῶν ὑμῶν.
\vs{32}Καὶ ἐξερημώσω ἐγὼ τὴν γῆν ὑμῶν, καὶ θαυμάσονται ἐπʼ αὐτῇ οἱ ἐχθροὶ ὑμῶν, οἱ ἐνοικοῦντες ἐν αὐτῇ.
\vs{33}Καὶ διασπερῶ ὑμᾶς εἰς τὰ ἔθνη, καὶ ἐξαναλώσει ὑμᾶς ἐπιπορευομένη ἡ μάχαιρα, καὶ ἔσται ἡ γῆ ὑμῶν ἔρημος, καὶ αἱ πόλεις ὑμῶν ἔσονται ἔρημοι.
\vs{34}Τότε εὐδοκήσει ἡ γῆ τὰ σάββατα αὐτῆς πάσας τὰς ἡμέρας τῆς ἐρημώσεως αὐτῆς,
\vs{35}καὶ ὑμεῖς ἔσεσθε ἐν τῇ γῇ τῶν ἐχθρῶν ὑμῶν· τότε σαββατιεῖ ἡ γη, καὶ εὐδοκήσει ἡ γῆ τὰ σάββατα αὐτῆς πάσας τὰς ἡμέρας τῆς ἐρημώσεως αὐτῆς· σαββατιεῖ ἃ οὐκ ἐσαββάτισεν ἐν τοῖς σαββάτοις ὑμῶν, ἡνίκα κατῳκεῖτε αὐτήν.
\vs{36}Καὶ τοῖς καταλειφθεῖσιν ἐξ ὑμῶν ἐπάξω δουλείαν εἰς τὴν καρδίαν αὐτῶν ἐν τῇ γῇ τῶν ἐχθρῶν αὐτῶν· καὶ διώξεται αὐτοὺς φωνὴ φύλλου φερομένου, καὶ φεύξονται ὡς φεύγοντες ἀπὸ πολέμου, καὶ πεσοῦνται οὐθενὸς διώκοντος.
\vs{37}Καὶ ὑπερόψεται ὁ ἀδελφὸς τὸν ἀδελφὸν ὡσεὶ ἐν πολέμῳ, οὐθενὸς κατατρέχοντος· καὶ οὐ δυνήσεσθε ἀντιστῆναι τοῖς ἐχθροῖς ὑμῶν.
\vs{38}Καὶ ἀπολεῖσθε ἐν τοῖς ἔθνεσι, καὶ κατέδεται ὑμᾶς ἡ γῆ τῶν ἐχθρῶν ὑμῶν.
\vs{39}Καὶ οἱ καταλειφθέντες ἀφʼ ὑμῶν, καταφθαρήσονται διὰ τὰς ἁμαρτίας αὐτῶν, καὶ διὰ τὰς ἁμαρτίας τῶν πατέρων αὐτῶν· ἐν τῇ γῇ τῶν ἐχθρῶν αὐτῶν τακήσονται.

\vs{40}Καὶ ἐξαγορεύσουσι τὰς ἁμαρτίας αὐτῶν, καὶ τὰς ἁμαρτίας τῶν πατέρων αὐτῶν, ὅτι παρέβησαν καὶ ὑπερεῖδόν με, καὶ ὅτι ἐπορεύθησαν ἐναντίον μου πλάγιοι,
\vs{41}καὶ ἐγὼ ἐπορεύθην μετʼ αὐτῶν ἐν θυμῷ πλαγίῳ· καὶ ἀπολῶ αὐτοὺς ἐν τῇ γῇ τῶν ἐχθρῶν αὐτῶν· τότε ἐντραπήσεται ἡ καρδία αὐτῶν ἡ ἀπερίτμητος, καὶ τότε εὐδοκήσουσι τὰς ἁμαρτίας αὐτῶν.
\vs{42}Καὶ μνησθήσομαι τῆς διαθήκης Ἰακὼβ, καὶ τῆς διαθήκης Ἰσαὰκ, καὶ τῆς διαθήκης Ἁβραὰμ μνησθήσομαι.

\vs{43}Καὶ τῆς γῆς μνησθήσομαι, καὶ ἡ γῆ ἐγκαταλειφθήσεται ἀπʼ αὐτῶν· τότε προσδέξεται ἡ γῆ τὰ σάββατα αὐτῆς, ἐν τῷ ἐρημωθῆναι αὐτὴν διʼ αὐτούς· καὶ αὐτοὶ προσδέξονται τὰς αὐτῶν ἀνομίας, ἀνθʼ ὧν τὰ κρίματά μου ὑπερεῖδον, καὶ τοῖς προστάγμασί μου προσώχθισαν τῇ ψυχῇ αὐτῶν.
\vs{44}Καὶ οὐδʼ ὡς ὄντων αὐτῶν ἐν τῇ γῇ τῶν ἐχθρῶν αὐτῶν, οὐχ ὑπερεῖδον αὐτούς, οὐδὲ προσώχθισα αὐτοῖς ὥστε ἐξαναλῶσαι αὐτούς τοῦ διασκεδάσαι τὴν διαθήκην μου τὴν πρὸς αὐτούς· ἐγὼ γάρ εἰμι Κύριος ὁ Θεὸς αὐτῶν.
\vs{45}Καὶ μνησθήσομαι διαθήκης αὐτῶν τῆς προτέρας, ὅτε ἐξήγαγον αὐτοὺς ἐκ γῆς Αἰγύπτου, ἐξ οἴκου δουλείας ἔναντι τῶν ἐθνῶν, τοῦ εἶναι αὐτῶν Θεός· ἐγώ εἰμι Κύριος.
\vs{46}Ταῦτα τὰ κρίματά μου, καὶ τὰ προστάγματά μου, καὶ ὁ νόμος ὃν ἔδωκε Κύριος ἀναμέσον αὐτοῦ καὶ ἀναμέσον τῶν υἱῶν Ἰσραὴλ, ἐν τῷ ὄρει Σινᾷ ἐν χειρὶ Μωυσῆ.

\ch{27}
Καὶ ἐλάλησε Κύριος πρὸς Μωυσῆν, λέγων,
\vs{2}λάλησον τοῖς υἱοῖς Ἰσραὴλ, καὶ ἐρεῖς αὐτοῖς, ὃς ἂν εὔξηται εὐχὴν ὥστε τιμὴν τῆς ψυχῆς αὐτοῦ τῷ Κυρίῳ,
\vs{3}ἔσται ἡ τιμὴ τοῦ ἄρσενος ἀπὸ εἰκοσαετοῦς, ἕως ἑξηκονταετοῦς, ἔσται αὐτοῦ ἡ τιμὴ πεντήκοντα δίδραχμα ἀργυρίου τῷ σταθμῷ τῷ ἁγίῳ·
\vs{4}Τῆς δὲ θηλείας ἔσται ἡ συντίμησις τριάκοντα δίδραχμα.
\vs{5}Ἐὰν δὲ ἀπὸ πενταετοῦς ἕως εἴκοσι ἐτῶν, ἔσται ἡ τιμὴ τοῦ ἄρσενος εἴκοσι δίδραχμα· τῆς δὲ θηλείας, δέκα δίδραχμα.
\vs{6}Ἀπὸ δὲ μηνιαίου ἕως πενταετοῦς, ἔσται ἡ τιμὴ τοῦ ἄρσενος πέντε δίδραχμα· τῆς δὲ θηλείας, τρία δίδραχμα ἀργυρίου.
\vs{7}Ἐὰν δὲ ἀπὸ ἑξήκοντα ἐτῶν καὶ ἐπάνω, ἐὰν μὲν ἄρσεν ᾖ, ἔσται ἡ τιμὴ αὐτοῦ πεντεκαίδεκα δίδραχμα ἀργυρίου· ἐὰν δὲ θήλεια, δέκα δίδραχμα.
\vs{8}Ἐὰν δὲ ταπεινὸς ᾖ τῇ τιμῇ, στήσεται ἐναντίον τοῦ ἱερέως· καὶ τιμήσεται αὐτὸν ὁ ἱερεύς· καθάπερ ἰσχύει ἡ χεὶρ τοῦ εὐξαμένου, τιμήσεται αὐτὸν ὁ ἱερεύς.

\vs{9}Ἐὰν δὲ ἀπὸ τῶν κτηνῶν τῶν προσφερομένων ἀπʼ αὐτῶν δῶρον τῷ Κυρίῳ, ὃς ἂν δῷ ἀπὸ τούτων τῷ Κυρίῳ, ἔσται ἅγιον.
\vs{10}Οὐκ ἀλλάξει αὐτὸ καλὸν πονηρῷ, οὐδὲ πονηρὸν καλῷ· ἐὰν δὲ ἀλλάσσων ἀλλάξῃ αὐτὸ κτῆνος κτήνει, ἔσται αὐτὸ καὶ τὸ ἄλλαγμα ἅγια.
\vs{11}Ἐὰν δὲ πᾶν κτῆνος ἀκάθαρτον, ἀφʼ ὧν οὐ προσφέρεται ἀπʼ αὐτῶν δῶρον τῷ Κυρίῳ, στήσει τὸ κτῆνος ἔναντι τοῦ ἱερέως,
\vs{12}καὶ τιμήσεται αὐτὸ ὁ ἱερεὺς ἀναμέσον καλοῦ καὶ ἀναμέσον πονηροῦ· καὶ καθότι ἂν τιμήσηται αὐτὸ ὁ ἱερεὺς, οὕτω στήσεται.
\vs{13}Ἐὰν δὲ λυτρούμενος λυτρώσηται αὐτὸ, προσθήσει τὸ ἐπίπεμπτον πρὸς τὴν τιμὴν αὐτοῦ.
\vs{14}Καὶ ἄνθρωπος ὃς ἂν ἁγιάσῃ τὴν οἰκίαν αὐτοῦ ἁγίαν τῷ Κυρίῳ, καὶ τιμήσεται αὐτὴν ὁ ἱερεὺς ἀναμέσον καλῆς καὶ ἀναμέσον πονηρᾶς· ὡς ἂν τιμήσηται αὐτὴν ὁ ἱερεὺς, οὕτω σταθήσεται.
\vs{15}Ἐὰν δὲ ὁ ἁγιάσας αὐτὴν λυτρῶται τὴν οἰκίαν αὐτοῦ, προσθήσει ἐπʼ αὐτὸ τὸ ἐπίπεμπτον τοῦ ἀργυρίου τῆς τιμῆς, καὶ ἔσται αὐτῷ.

\vs{16}Ἐὰν δὲ ἀπὸ τοῦ ἀγροῦ τῆς κατασχέσεως αὐτοῦ ἁγιάσῃ ἄνθρωπος τῷ Κυρίῳ, καὶ ἔσται ἡ τιμὴ κατὰ τὸν σπόρον αὐτοῦ, κόρου κριθῶν πεντήκοντα δίδραχμα ἀργυρίου.
\vs{17}Ἐὰν δὲ ἀπὸ τοῦ ἐνιαυτοῦ τῆς ἀφέσεως ἁγιάσῃ τὸν ἀγρὸν αὐτοῦ, κατὰ τὴν τιμὴν αὐτοῦ στήσεται.
\vs{18}Ἐὰν δὲ ἔσχατον μετὰ τὴν ἄφεσιν ἁγιάσῃ τὸν ἀγρὸν αὐτοῦ, προσλογιεῖται αὐτῷ ὁ ἱερεὺς τὸ ἀργύριον ἐπὶ τὰ ἔτη τὰ ἐπίλοιπα, ἕως εἰς τὸν ἐνιαυτὸν τῆς ἀφέσεως, καὶ ἀνθυφαιρεθήσεται ἀπὸ τῆς συντιμήσεως αὐτοῦ.
\vs{19}Ἐὰν δὲ λυτρῶται τὸν ἀγρὸν ὁ ἁγιάσας αὐτὸν, προσθήσει τὸ ἐπίπεμπτον τοῦ ἀργυρίου πρὸς τὴν τιμὴν αὐτοῦ, καὶ ἔσται αὐτῷ.
\vs{20}Ἐὰν δὲ μὴ λυτρῶται τὸν ἀγρὸν, καὶ ἀποδῶται τὸν ἀγρὸν ἀνθρώπῳ ἑτέρῳ, οὐκέτι μὴ λυτρώσηται αὐτόν.
\vs{21}Ἀλλʼ ἔσται ὁ ἀγρὸς ἐξεληλυθυίας τῆς ἀφέσεως ἅγιος τῷ Κυρίῳ, ὥσπερ ἡ γῆ ἡ ἀφωρισμένη τῷ ἱερεῖ ἔσται κατάσχεσις αὐτοῦ.
\vs{22}Ἐὰν δὲ ἀπὸ τοῦ ἀγροῦ οὗ κέκτηται, ὃς οὐκ ἔστιν ἀπὸ τοῦ ἀγροῦ τῆς κατασχέσεως αὐτοῦ, ἁγιάσῃ τῷ Κυρίῳ,
\vs{23}λογιεῖται πρὸς αὐτὸν ὁ ἱερεὺς τὸ τέλος τῆς τιμῆς ἐκ τοῦ ἐνιαυτοῦ τῆς ἀφέσεως, καὶ ἀποδώσει τὴν τιμὴν ἐν τῇ ἡμέρᾳ ἐκείνῃ ἁγίαν τῷ Κυρίῳ·
\vs{24}Καὶ ἐν τῷ ἐνιαυτῷ τῆς ἀφέσεως ἀποδοθήσεται ὁ ἀγρὸς τῷ ἀνθρώπῳ παρʼ οὗ κέκτηται αὐτὸν, οὗ ἦν ἡ κατάσχεσις τῆς γῆς.
\vs{25}Καὶ πᾶσα τιμὴ ἔσται σταθμίοις ἁγίοις· εἴκοσι ὀβολοὶ ἔσται τὸ δίδραχμον.
\vs{26}Καὶ πᾶν πρωτότοκον ὃ ἐὰν γένηται ἐν τοῖς κτήνεσί σου, ἔσται τῷ Κυρίῳ, καὶ οὐ καθαγιάσει αὐτὸ οὐδείς· ἐάν τε μόσχον, ἐάν τε πρόβατον, τῷ Κυρίῳ ἐστίν.
\vs{27}Ἐὰν δὲ τῶν τετραπόδων τῶν ἀκαθάρτων ἀλλάξῃ κατὰ τὴν τιμὴν αὐτοῦ, καὶ προσθήσει τὸ ἐπίπεμπτον πρὸς αὐτὸ, καὶ ἔσται αὐτῷ· ἐὰν δὲ μὴ λυτρῶται, πραθήσεται κατὰ τὸ τίμημα αὐτοῦ.

\vs{28}Πᾶν δὲ ἀνάθεμα, ὃ ἂν ἀναθῇ ἄνθρωπος τῷ Κυρίῳ ἀπὸ πάντων, ὅσα αὐτῷ ἐστιν, ἀπὸ ἀνθρώπου ἕως κτήνους, καὶ ἀπὸ ἀγροῦ κατασχέσεως αὐτοῦ, οὐκ ἀποδώσεται οὐδὲ λυτρώσεται· πᾶν ἀνάθεμα ἅγιον ἁγίων ἔσται τῷ Κυρίῳ.
\vs{29}Καὶ πᾶν ὃ ἐὰν ἀνατεθῇ ἀπὸ τῶν ἀνθρώπων, οὐ λυτρωθήσεται, ἀλλὰ θανάτῳ θανατωθήσεται.
\vs{30}Πᾶσα δεκάτη τῆς γῆς, ἀπὸ τοῦ σπέρματος τῆς γῆς, καὶ τοῦ καρποῦ τοῦ ξυλίνου, τῷ. Κυρίῳ ἐστίν, ἅγιον τῷ Κυρίῳ.
\vs{31}Ἐὰν δὲ λυτρῶται λύτρῳ ἄνθρωπος τὴν δεκάτην αὐτοῦ, τὸ ἐπίπεμπτον προσθήσει πρὸς αὐτὸν, καὶ ἔσται αὐτῷ.
\vs{32}Καὶ πᾶσα δεκάτη βοῶν, καὶ προβάτων, καὶ πᾶν ὃ ἂν ἔλθῃ ἐν τῷ ἀριθμῷ ὑπὸ τὴν ῥάβδον, τὸ δέκατον ἔσται ἅγιον τῷ Κυρίῳ.
\vs{33}Οὐκ ἀλλάξεις καλὸν πονηρῷ, οὐδὲ πονηρὸν καλῷ· ἐὰν δὲ ἀλλάσσων ἀλλάξῃς αὐτό, καὶ τὸ ἄλλαγμα αὐτοῦ ἔσται ἅγιον, οὐ λυτρωθήσεται.

\vs{34}Αὗταί εἰσιν αἱ ἐντολαὶ ἃς ἐνετείλατο Κύριος τῷ Μωυσῇ πρὸς τοὺς υἱοὺς Ἰσραὴλ ἐν τῷ ὄρει Σινᾷ.


\def\book{ΑΡΙΘΜΟΙ}
\biblebook{ΑΡΙΘΜΟΙ}


\lettrine[lines=2, loversize=0.2, nindent=0em, findent=.25em]{\textcolor{bookheadingcolor}{Κ}}{ΑΙ} ἐλάλησε Κύριος πρὸς Μωυσῆν ἐν τῇ ἐρήμῳ τῇ Σινᾷ, ἐν τῇ σκηνῇ τοῦ μαρτυρίου, ἐν μιᾷ τοῦ μηνὸς τοῦ δευτέρου, ἔτους δευτέρου ἐξελθόντων αὐτῶν ἐκ γῆς Αἰγύπτου, λέγων,
\vs{2}λάβετε ἀρχὴν πάσης συναγωγῆς Ἰσραὴλ κατὰ συγγενείας, κατʼ οἴκους πατριῶν αὐτῶν, κατὰ ἀριθμὸν ἐξ ὀνόματος αὐτῶν, κατὰ κεφαλὴν αὐτῶν·
\vs{3}πᾶς ἄρσην ἀπὸ εἰκοσαετοῦς καὶ ἐπάνω, πᾶς ὁ ἐκπορευόμενος ἐν δυνάμει Ἰσραήλ, ἐπισκέψασθε αὐτοὺς σὺν δυνάμει αὐτῶν· σὺ καὶ Ἀαρὼν ἐπισκέψασθε αὐτούς.
\vs{4}Καὶ μεθʼ ὑμῶν ἔσονται ἕκαστος κατὰ φυλὴν ἑκάστου ἀρχόντων, κατʼ οἴκους πατριῶν ἔσονται.

\vs{5}Καὶ ταῦτα τὰ ὀνόματα τῶν ἀνδρῶν, οἵτινες παραστήσονται μεθʼ ὑμῶν· τῶν Ῥουβὴν, Ἐλισοὺρ υἱὸς Σεδιούρ·
\vs{6}Τῶν Συμεὼν, Σαλαμιὴλ υἱὸς Σουρισαδαί.
\vs{7}Τῶν Ἰούδα, Ναασσὼν υἱὸς Ἀμιναδάβ.
\vs{8}Τῶν Ἰσσάχαρ, Ναθαναὴλ υἱὸς Σωγάρ·
\vs{9}Τῶν Ζαβουλὼν, Ἐλιὰβ υἱὸς Χαιλών· Τῶν υἱῶν Ἰωσὴφ τῶν Ἐφραὶμ, Ἐλισαμὰ υἱὸς Ἐμιούδ·
\vs{10}τῶν Μανασσῆ, Γαμαλιὴλ υἱὸς Φαδασούρ.
\vs{11}Τῶν Βενιαμὶν, Ἀβιδὰν υἱὸς Γαδεωνί.
\vs{12}Τῶν Δὰν, Ἀχιέζερ υἱὸς Ἀμισαδαΐ.
\vs{13}Τῶν Ἀσὴρ, Φαγαϊὴλ, υἱὸς Ἐχράν.
\vs{14}Τῶν Γὰδ, Ἐλισὰφ υἱὸς Ῥαγουήλ.
\vs{15}Τῶν Νεφθαλὶ, Ἀχιρὲ υἱὸς Αἰνάν.
\vs{16}Οὗτοι ἐπίκλητοι τῆς συναγωγῆς, ἄρχοντες τῶν φυλῶν κατὰ πατριὰς αὐτῶν, χιλίαρχοι Ἰσραήλ εἰσι.

\vs{17}Καὶ ἔλαβε Μωυσῆς καὶ Ἀαρὼν τοὺς ἄνδρας τούτους τοὺς ἀνακληθέντας ἐξ ὀνόματος.
\vs{18}Καὶ πᾶσαν τὴν συναγωγὴν συνήγαγον ἐν μιᾷ τοῦ μηνὸς τοῦ δευτέρου ἔτους· καὶ ἐπηξονοῦσαν κατὰ γενέσεις αὐτῶν, κατὰ πατριὰς αὐτῶν, κατὰ ἀριθμὸν ὀνομάτων αὐτῶν, ἀπὸ εἰκοσαετοῦς καὶ ἐπάνω, πᾶν ἀρσενικὸν κατὰ κεφαλὴν αὐτῶν,
\vs{19}ὃν τρόπον συνέταξε Κύριος τῷ Μωυσῇ· καὶ ἐπεσκέπησαν ἐν τῇ ἐρήμῳ τοῦ Σινά.

\vs{20}Καὶ ἐγένοντο οἱ υἱοὶ Ῥουβὴν πρωτοτόκου Ἰσραὴλ κατὰ συγγενείας αὐτῶν, κατὰ δήμους αὐτῶν, κατʼ οἴκους πατριῶν αὐτῶν, κατὰ ἀριθμὸν ὀνομάτων αὐτῶν, κατὰ κεφαλὴν αὐτῶν, πάντα ἀρσενικὰ ἀπὸ εἰκοσαετοῦς καὶ ἐπάνω, πᾶς ὁ ἐκπορευόμενος ἐν τῇ δυνάμει,
\vs{21}ἡ ἐπίσκεψις αὐτῶν ἐκ τῆς φυλῆς Ῥουβὴν, ἓξ καὶ τεσσαράκοντα χιλιάδες καὶ πεντακόσιοι.
\vs{22}Τοῖς υἱοῖς Συμεὼν κατὰ συγγενείας αὐτῶν, κατὰ δήμους αὐτῶν, κατʼ οἴκους πατριῶν αὐτῶν, κατὰ ἀριθμὸν ὀνομάτων αὐτῶν, κατὰ κεφαλὴν αὐτῶν, πάντα ἀρσενικὰ ἀπὸ εἰκοσαετοῦς καὶ ἐπάνω, πᾶς ὁ ἐκπορευόμενος ἐν τῇ δυνάμει,
\vs{23}ἡ ἐπίσκεψις αὐτῶν ἐκ τῆς φυλῆς Συμεὼν, ἐννέα καὶ πεντήκοντα χιλιάδες καὶ τριακόσιοι.

\vs{24}Τοῖς υἱοῖς Ἰούδα κατὰ συγγενείας αὐτῶν, κατὰ δήμους αὐτῶν, κατʼ οἴκους πατριῶν αὐτῶν, κατὰ ἀριθμὸν ὀνομάτων αὐτῶν, κατὰ κεφαλὴν αὐτῶν, πάντα ἀρσενικὰ ἀπὸ εἰκοσαετοῦς καὶ ἐπάνω, πᾶς ὁ ἐκπορευόμενος ἐν τῇ δυνάμει,
\vs{25}ἡ ἐπίσκεψις αὐτῶν ἐκ τῆς φυλῆς Ἰούδα, τέσσαρες καὶ ἑβδομήκοντα χιλιάδες καὶ ἑξακόσιοι.

\vs{26}Τοῖς υἱοῖς Ἰσσάχαρ κατὰ συγγενίας αὐτῶν, κατὰ δήμους αὐτῶν, κατʼ οἴκους πατριῶν αὐτῶν, κατὰ ἀριθμὸν ὀνομάτων αὐτῶν, κατὰ κεφαλὴν αὐτῶν, πάντα ἀρσενικὰ ἀπὸ εἰκοσαετοῦς καὶ ἐπάνω, πᾶς ὁ ἐκπορευόμενος ἐν τῇ δυνάμει,
\vs{27}ἡ ἐπίσκεψις αὐτῶν ἐκ τῆς φυλῆς Ἰσσάχαρ, τέσσαρες καὶ πεντήκοντα χιλιάδες καὶ τετρακόσιοι.
\vs{28}Τοῖς υἱοῖς Ζαβουλὼν κατὰ συγγενείας αὐτῶν, κατὰ δήμους αὐτῶν, κατʼ οἴκους πατριῶν αὐτῶν, κατὰ ἀριθμὸν ὀνομάτων αὐτῶν, κατὰ κεφαλὴν αὐτῶν, πάντα ἀρσενικὰ ἀπὸ εἰκοσαετοῦς καὶ ἐπάνω, πᾶς ὁ ἐκπορευόμενος ἐν τῇ δυνάμει,
\vs{29}ἡ ἐπίσκεψις αὐτῶν ἐκ τῆς φυλῆς Ζαβουλὼν, ἑπτὰ καὶ πεντήκοντα χιλιάδες καὶ τετρακόσιοι.

\vs{30}Τοῖς υἱοῖς Ἰωσὴφ υἱοῖς Ἐφραὶμ κατὰ συγγενείας αὐτῶν, κατὰ δήμους αὐτῶν, κατʼ οἴκους πατριῶν αὐτῶν, κατὰ ἀριθμὸν ὀνομάτων αὐτῶν, κατὰ κεφαλὴν αὐτῶν, πάντα ἀρσενικὰ ἀπὸ εἰκοσαετοῦς καὶ ἐπάνω, πᾶς ὁ ἐκπορευόμενος ἐν τῇ δυνάμει,
\vs{31}ἡ ἐπίσκέψις αὐτῶν ἐκ τῆς φυλῆς Ἐφραὶμ, τεσσαράκοντα χιλιάδες καὶ πεντακόσιοι.
\vs{32}Τοῖς υἱοῖς Μανασσὴ κατὰ συγγενείας αὐτῶν, κατὰ δήμους αὐτῶν, κατʼ οἴκους πατριῶν αὐτῶν, κατὰ ἀριθμὸν ὀνομάτων αὐτῶν, κατὰ κεφαλὴν αὐτῶν, πάντα ἀρσενικὰ, ἀπὸ εἰκοσαετοῦς καὶ ἐπάνω, πᾶς ὁ ἐκπορευόμενος ἐν τῇ δυνάμει,
\vs{33}ἡ ἐπίσκεψις αὐτῶν ἐκ τῆς φυλῆς Μανασσὴ, δύο καὶ τριάκοντα χιλιάδες καὶ διακόσιοι.
\vs{34}Τοῖς υἱοῖς Βενιαμὶν κατὰ συγγενείας αὐτῶν, κατὰ δήμους αὐτῶν, κατʼ οἴκους πατριῶν αὐτῶν, κατὰ ἀριθμὸν ὀνομάτων αὐτῶν, κατὰ κεφαλὴν αὐτῶν, πάντα ἀρσενικὰ ἀπὸ εἰκοσαετοῦς καὶ ἐπάνω, πᾶς ὁ ἐκπορευόμενος ἐν τῇ δυνάμει,
\vs{35}ἡ ἐπίσκεψις αὐτῶν ἐκ τῆς φυλῆς Βενιαμὶν, πέντε καὶ τριάκοντα χιλιάδες καὶ τετρακόσιοι.
\vs{36}Τοῖς υἱοῖς Γὰδ κατὰ συγγενείας αὐτῶν, κατὰ δήμους αὐτῶν, κατʼ οἴκους πατριῶν αὐτῶν, κατὰ ἀριθμὸν ὀνομάτων αὐτῶν, κατὰ κεφαλὴν αὐτῶν, πάντα ἀρσενικὰ ἀπὸ εἰκοσαετοῦς καὶ ἐπάνω, πᾶς ὁ ἐκπορευόμενος ἐν τῇ δυνάμει,
\vs{37}ἡ ἐπίσκεψις αὐτῶν· ἐκ τῆς φυλῆς Γὰδ, πέντε καὶ τεσσαράκοντα χιλιάδες καὶ ἑξακόσιοι καὶ πεντήκοντα.

\vs{38}Τοῖς υἱοῖς Δὰν κατὰ συγγενείας αὐτῶν, κατὰ δήμους αὐτῶν, κατʼ οἴκους πατριῶν αὐτῶν, κατὰ ἀριθμὸν ὀνομάτων αὐτῶν, κατὰ κεφαλὴν αὐτῶν, πάντα ἀρσενικὰ ἀπὸ εἰκοσαετοῦς καὶ ἐπάνω, πᾶς ὁ ἐκπορευόμενος ἐν τῇ δυνάμει,
\vs{39}ἡ ἐπίσκεψις αὐτῶν ἐκ τῆς φυλῆς Δὰν, δύο καὶ ἑξήκοντα χιλιάδες καὶ ἑπτακόσιοι.
\vs{40}Τοῖς υἱοῖς Ἀσὴρ κατὰ συγγενείας αὐτῶν, κατὰ δήμους αὐτῶν, κατʼ οἴκους πατριῶν αὐτῶν, κατὰ ἀριθμὸν ὀνομάτων αὐτῶν, κατὰ κεφαλὴν αὐτῶν, πάντα ἀρσενικὰ ἀπὸ εἰκοσαετοῦς καὶ ἐπάνω, πᾶς ὁ ἐκπορευόμενος ἐν τῇ δυνάμει,
\vs{41}ἡ ἐπίσκεψις αὐτῶν ἐκ τῆς φυλῆς Ἀσὴρ, μία καὶ τεσσαράκοντα χιλιάδες καὶ πεντακόσιοι.

\vs{42}Τοῖς υἱοῖς Νεφθαλὶ κατὰ συγγενείας αὐτῶν, κατὰ δήμους αὐτῶν, κατʼ οἴκους πατριῶν αὐτῶν, κατὰ ἀριθμὸν ὀνομάτων αὐτῶν, κατὰ κεφαλὴν αὐτῶν, πάντα ἀρσενικὰ ἀπὸ εἰκοσαετοῦς καὶ ἐπάνω, πᾶς ὁ ἐκπορευόμενος ἐν τῇ δυνάμει,
\vs{43}ἡ ἐπίσκεψις αὐτῶν ἐκ τῆς φυλῆς Νεφθαλὶ, τρεῖς καὶ πεντήκοντα χιλιάδες καὶ τετρακόσιοι.

\vs{44}Αὕτη ἡ ἐπίσκεψις, ἣν ἐπεσκέψαντο Μωυσῆς καὶ Ἀαρὼν καὶ οἱ ἄρχοντες Ἰσραὴλ δώδεκα ἄνδρες· ἀνὴρ εἷς κατὰ φυλὴν μίαν, κατὰ φυλὴν οἴκων πατριᾶς ἦσαν.
\vs{45}Καὶ ἐγένετο πᾶσα ἡ ἐπίσκεψις υἱῶν Ἰσραὴλ σὺν δυνάμει αὐτῶν ἀπὸ εἰκοσαετοῦς καὶ ἐπάνω, πᾶς ὁ ἐκπορευόμενος παρατάξασθαι ἐν Ἰσραὴλ
\vs{46}ἑξακόσιαι χιλιάδες καὶ τρισχίλιοι καὶ πεντακόσιοι καὶ πεντήκοντα.

\vs{47}Οἱ δὲ Λευῖται ἐκ τῆς φυλῆς πατριᾶς αὐτῶν οὐκ ἐπεσκέπησαν ἐν τοῖς υἱοῖς Ἰσραήλ.
\vs{48}καὶ ἐλάλησε Κύριος πρὸς Μωυσῆν, λέγων,
\vs{49}ὅρα, τὴν φυλὴν Λευὶ οὐ συνεπισκέψῃ, καὶ τὸν ἀριθμὸν αὐτῶν οὐ λήμῃ, ἐν μέσῳ υἱῶν Ἰσραήλ.
\vs{50}Καὶ σὺ ἐπίστησον τοὺς Λευίτας ἐπὶ τὴν σκηνὴν τοῦ μαρτυρίου, καὶ ἐπὶ πάντα τὰ σκεύη αὐτῆς, καὶ ἐπὶ πάντα ὅσα ἐστὶν ἐν αὐτῇ· ἀροῦσιν αὐτοὶ τὴν σκηνὴν, καὶ πάντα τὰ σκεύη αὐτῆς· καὶ αὐτοὶ λειτουργήσουσιν ἐν αὐτῇ, καὶ κύκλῳ τῆς σκηνῆς παρεμβαλοῦσι.
\vs{51}Καὶ ἐν τῷ ἐξαίρειν τὴν σκηνὴν, καθελοῦσιν αὐτὴν οἱ Λευῖται, καὶ ἐν τῷ παρεμβάλλειν τὴν σκηνὴν, ἀναστήσουσι· καὶ ὁ ἀλλογενῆς ὁ προσπορευόμενος ἀποθανέτω.
\vs{52}Καὶ παρεμβαλοῦσιν οἱ υἱοὶ Ἰσραὴλ, ἀνὴρ ἐν τῇ ἑαυτοῦ τάξει, καὶ ἀνὴρ κατὰ τὴν ἑαυτοῦ ἡγεμονίαν, σὺν δυνάμει αὐτῶν.
\vs{53}Οἱ δὲ Λευῖται παρεμβαλλέτωσαν ἐναντίοι κύκλῳ τῆς σκηνῆς τοῦ μαρτυρίου, καὶ οὐκ ἔσται ἁμάρτημα ἐν υἱοῖς Ἰσραήλ. καὶ φυλάξουσιν οἱ Λευῖται αὐτοὶ τὴν φυλακὴν τῆς σκηνῆς τοῦ μαρτυρίου.
\vs{54}Καὶ ἐποίησαν οἱ υἱοὶ Ἰσραὴλ, κατὰ πάντα ἃ ἐνετείλατο Κύριος τῷ Μωυσῇ καὶ Ἀαρὼν, οὕτως ἐποίησαν.

\ch{2}
Καὶ ἐλάλησε Κύριος πρὸς Μωυσῆν καὶ Ἀαρὼν, λέγων,
\vs{2}ἄνθρωπος ἐχόμενος αὐτοῦ κατὰ τάγμα, κατὰ σημαίας, κατʼ οἴκους πατριῶν αὐτῶν, παρεμβαλλέτωσαν οἱ υἱοὶ Ἰσραήλ ἐναντίοι· κύκλῳ τῆς σκηνῆς τοῦ μαρτυρίου παρεμβαλοῦσιν οἱ υἱοὶ Ἰσραήλ.
\vs{3}Καὶ οἱ παρεμβάλλοντες πρῶτοι κατὰ ἀνατολὰς, τάγμα παρεμβολῆς Ἰούδα σὺν δυνάμει αὐτῶν, καὶ ὁ ἄρχων τῶν υἱῶν Ἰούδα, Ναασσὼν υἱὸς Ἀμιναδάβ.
\vs{4}Δύναμις αὐτοῦ οἱ ἐπεσκεμμένοι, τέσσαρες καὶ ἑβδομήκοντα χιλιάδες καὶ ἑξακόσιοι.
\vs{5}Καὶ οἱ παρεμβάλλοντες ἐχόμενοι φυλῆς Ἰσσάχαρ, καὶ ὁ ἄρχων τῶν υἱῶν Ἰσσάχαρ, Ναθαναὴλ υἱὸς Σωγάρ.
\vs{6}Δύναμις αὐτοῦ οἱ ἐπεσκεμμένοι, τέσσαρες καὶ πεντήκοντα χιλιάδες καὶ τετρακόσιοι.
\vs{7}Καὶ οἱ παρεμβάλλοντες ἐχόμενοι φυλῆς Ζαβουλὼν, καὶ ὁ ἄρχων τῶν υἱῶν Ζαβουλὼν, Ἑλιὰβ υἱὸς Χαιλών.
\vs{8}Δύναμις αὐτοῦ οἱ ἐπεσκεμμένοι, ἑπτὰ καὶ πεντήκοντα χιλιάδες καὶ τετρακόσιοι.
\vs{9}Πάντες οἱ ἐπεσκεμμένοι ἐκ τῆς παρεμβολῆς Ἰούδα, ἑκατὸν ὀγδοήκοντα χιλιάδες καὶ ἑξακισχίλιοι καὶ τετρακόσιοι, σὺν δυνάμει αὐτῶν πρῶτον ἐξαροῦσι.
\vs{10}Τάγματα παρεμβολῆς Ῥουβὴν, πρὸς λίβα δύναμις αὐτῶν, καὶ ὁ ἄρχων τῶν υἱῶν ʼΡουβὴν, Ἐλισοὺρ υἱὸς Σεδιούρ.
\vs{11}Δύναμις αὐτοῦ οἱ ἐπεσκεμμένοι, ἓξ καὶ τεσσαράκοντα χιλιάδες καὶ πεντακόσιοι.
\vs{12}Καὶ οἱ παρεμβάλλοντες ἐχόμενοι αὐτοῦ φυλῆς Συμεὼν, καὶ ὁ ἄρχων τῶν υἱῶν Συμεὼν, Σαλαμιὴλ υἱὸς Σουρισαδαί.
\vs{13}Δύναμις αὐτοῦ οἱ ἐπεσκεμμένοι, ἐννέα καὶ πεντήκοντα χιλιάδες καὶ τριακόσιοι.
\vs{14}Καὶ οἱ παρεμβάλοντες ἐχόμενοι αὐτοῦ φυλὴ Γὰδ, καὶ ὁ ἄρχων τῶν υἱῶν Γὰδ, Ἑλισὰφ υἱὸς Ῥαγουήλ.
\vs{15}Δύναμις αὐτοῦ οἱ ἐπεσκεμμένοι, πέντε καὶ τεσσαράκοντα χιλιάδες καὶ ἑξακόσιοι καὶ πεντήκοντα.
\vs{16}Πάντες οἱ ἐπεσκεμμένοι τῆς παρεμβολῆς Ῥουβὴν, ἑκατὸν πεντήκοντα μία χιλιάδες καὶ τετρακόσιοι καὶ πεντήκοντα, σὺν δυνάμει αὐτῶν δεύτεροι ἐξαροῦσι.

\vs{17}Καὶ ἀρθήσεται ἡ σκηνὴ τοῦ μαρτυρίου, καὶ ἡ παρεμβολὴ τῶν Λευιτῶν μέσον τῶν παρεμβολῶν· ὡς καὶ παρεμβαλοῦσιν, οὕτω καὶ ἐξαροῦσιν ἕκαστος ἐχόμενος καθʼ ἡγεμονίας.
\vs{18}Τάγμα παρεμβολῆς Ἐφραὶμ παρὰ θάλασσαν σὺν δυνάμει αὐτῶν, καὶ ὁ ἄρχων τῶν υἱῶν Ἐφραὶμ, Ἐλισαμὰ υἱὸς Ἐμιούδ.
\vs{19}Δύναμις αὐτοῦ οἱ ἐπεσκεμμένοι, τεσσαράκοντα χιλιάδες καὶ πεντακόσιοι.

\vs{20}Καὶ οἱ παρεμβάλλοντες ἐχόμενοι φυλῆς Μανασσῆ, καὶ ὁ ἄρχων τῶν υἱῶν Μανασσῆ, Γαμαλιὴλ υἱὸς Φαδασσούρ.
\vs{21}Δύναμις αὐτοῦ οἱ ἐπεσκεμμένοι, δύο καὶ τριάκοντα χιλιάδες καὶ διακόσιοι.
\vs{22}Καὶ οἱ παρεμβάλλοντες ἐχόμενοι φυλῆς Βενιαμὶν, καὶ ὁ ἄρχων τῶν υἱῶν Βενιαμὶν, Ἀβιδὰν υἱὸς Γαδεωνί.
\vs{23}Δύναμις αὐτοῦ οἱ ἐπεσκεμμένοι, πέντε καὶ τριάκοντα χιλιάδες καὶ τετρακόσιοι.
\vs{24}Πάντες οἱ ἐπεσκεμμένοι τῆς παρεμβολῆς Ἐφραὶμ, ἑκατὸν χιλιάδες καὶ ὀκτακισχίλιοι καὶ ἑκατὸν· σὺν δυνάμει αὐτῶν τρίτοι ἐξαροῦσι.

\vs{25}Τάγμα παρεμβολῆς Δὰν πρὸς βοῤῥᾶν σὺν δυνάμει αὐτῶν, καὶ ὁ ἄρχων τῶν υἱῶν Δὰν, Ἀχιέζερ υἱὸς Ἀμισαδαί.
\vs{26}Δύναμις αὐτοῦ οἱ ἐπεσκεμμένοι, δύο καὶ ἑξήκοντα χιλιάδες καὶ ἑπτακόσιοι.
\vs{27}Καὶ οἱ παρεμβάλλοντες ἐχόμενοι αὐτοῦ φυλὴ Ἀσὴρ, καὶ ὁ ἄρχων τῶν υἱῶν Ἀσὴρ, Φαγεὴλ υἱὸς Ἐχράν.
\vs{28}Δύναμις αὐτοῦ οἱ ἐπεσκεμμένοι, μία καὶ τεσσαράκοντα χιλιάδες καὶ πεντακόσιοι.
\vs{29}Καὶ οἱ παρεμβάλλοντες ἐχόμενοι φυλῆς Νεφθαλί, καὶ ὁ ἄρχων τῶν υἱῶν Νεφθαλὶ, Ἀχιρὲ υἱὸς Αἰνάν.
\vs{30}Δύναμις αὐτοῦ οἱ ἐπεσκεμμένοι, τρεῖς καὶ πεντήκοντα χιλιάδες καὶ τετρακόσιοι.
\vs{31}Πάντες οἱ ἐπεσκεμμένοι τῆς παρεμβολῆς Δὰν, ἑκατὸν καὶ πεντηκονταεπτὰ χιλιάδες καὶ ἑξακόσιοι· ἔσχατοι ἐξαροῦσι κατὰ τάγμα αὐτῶν.

\vs{32}Αὕτη ἡ ἐπίσκεψις τῶν υἱῶν Ἰσραὴλ κατʼ οἴκους πατριῶν αὐτῶν· πᾶσα ἡ ἐπίσκεψις τῶν παρεμβολῶν σὺν ταῖς δυνάμεσιν αὐτῶν, ἑξακόσιαι χιλιάδες καὶ τρισχίλιοι πεντακόσιοι πεντήκοντα.
\vs{33}Οἱ δὲ Λευῖται οὐ συνεπεσκέπησαν ἐν αὐτοῖς, καθὰ ἐνετείλατο Κύριος τῷ Μωυσῇ.
\vs{34}Καὶ ἐποίησαν οἱ υἱοὶ Ἰσραὴλ πάντα ὅσα συνέταξε Κύριος τῷ Μωυσῇ· οὕτω παρενέβαλον κατὰ τάγμα αὐτῶν, καὶ οὕτως ἐξῇρον ἕκαστος ἐχόμενοι κατὰ δήμους αὐτῶν, κατʼ οἴκους πατριῶν αὐτῶν.

\ch{3}
Καὶ αὗται αἱ γενέσεις Ἀαρὼν καὶ Μωυσῆ, ἐν ᾗ ἡμέρᾳ ἐλάλησε Κύριος τῷ Μωυσῇ ἐν ὄρει Σινᾷ.
\vs{2}Καὶ ταῦτα τὰ ὀνόματα τῶν υἱῶν Ἀαρών· πρωτότοκος Ναδάβ, καὶ Ἀβιοὺδ, Ἐλεάζαρ, καὶ Ἰθάμαρ.
\vs{3}Ταῦτα τὰ ὀνόματα τῶν υἱῶν Ἀαρὼν, οἱ ἱερεῖς οἱ ἠλειμμένοι, οὓς ἐτελείωσαν τὰς χεῖρας αὐτῶν ἱερατεύειν.
\vs{4}Καὶ ἐτελεύτησε Ναδὰβ καὶ Ἀβιοὺδ ἔναντι Κυρίου, προσφερόντων αὐτῶν πῦρ ἀλλότριον ἔναντι Κυρίου, ἐν τῇ ἐρήμῳ Σινᾷ, καὶ παιδία οὐκ ἦν αὐτοῖς· καὶ ἱεράτευσεν Ἐλεάζαρ καὶ Ἰθάμαρ μετὰ Ἀαρὼν τοῦ πατρὸς αὐτῶν.

\vs{5}Καὶ ἐλάλησε Κύριος πρὸς Μωυσῆν, λέγων,
\vs{6}λάβε τὴν φυλὴν Λευὶ, καὶ στήσεις αὐτοὺς ἐναντίον Ἀαρὼν τοῦ ἱερέως, καὶ λειτουργήσουσιν αὐτῷ,
\vs{7}καὶ φυλάξουσι τὰς φυλακὰς αὐτοῦ, καὶ τὰς φυλακὰς τῶν υἱῶν Ἰσραὴλ ἔναντι τῆς σκηνῆς τοῦ μαρτυρίου, ἐργάζεσθαι τὰ ἔργα τῆς σκηνῆς.
\vs{8}Καὶ φυλάξουσι πάντα τὰ σκεύη τῆς σκηνῆς τοῦ μαρτυρίου, καὶ τὰς φυλακὰς τῶν υἱῶν Ἰσραὴλ κατὰ πάντα τὰ ἔργα τῆς σκηνῆς.
\vs{9}Καὶ δώσεις τοὺς Λευίτας Ἀαρὼν, καὶ τοῖς υἱοῖς αὐτοῦ τοῖς ἱερεῦσι· δεδομένοι δόμα οὗτοί μοι εἰσὶν ἀπὸ τῶν υἱῶν Ἰσραήλ.
\vs{10}Καὶ Ἀαρὼν καὶ τοὺς υἱοὺς αὐτοῦ καταστήσεις ἐπὶ τῆς σκηνῆς τοῦ μαρτυρίου· καὶ φυλάξουσι τὴν ἱερατείαν αὐτῶν, καὶ πάντα τὰ κατὰ τὸν βωμὸν, καὶ ἔσω τοῦ καταπετάσματος· καὶ ὁ ἀλλογενὴς ὁ ἁπτόμενος ἀποθανεῖται.
\vs{11}Καὶ ἐλάλησε Κύριος πρὸς Μωυσῆν, λέγων,
\vs{12}καὶ ἰδοὺ ἐγὼ εἴληφα τοὺς Λευίτας ἐκ μέσου τῶν υἱῶν Ἰσραὴλ ἀντὶ παντὸς πρωτοτόκου διανοίγοντος μήτραν παρὰ τῶν υἱῶν Ἰσραήλ· λύτρα αὐτῶν ἔσονται, καὶ ἔσονται ἐμοὶ οἱ Λευῖται.
\vs{13}Ἐμοὶ γὰρ πᾶν προτότοκον· ἐν ᾗ ἡμέρᾳ ἐπάταξα πᾶν πρωτότοκον ἐν γῇ Αἰγύπτου, ἡγίασα ἐμοὶ πᾶν πρωτότοκον ἐν Ἰσραήλ· ἀπὸ ἀνθρώπου ἕως κτήνους ἐμοὶ ἔσονται· ἐγὼ Κύριος.

\vs{14}Καὶ ἐλάλησε Κύριος πρὸς Μωυσῆν ἐν τῇ ἐρήμῳ Σινᾷ, λέγων,
\vs{15}ἐπίσκεψαι τοὺς υἱοὺς Λευὶ κατʼ οἴκους πατριῶν αὐτῶν, κατὰ δήμους αὐτῶν· πᾶν ἀρσενικὸν ἀπὸ μηνιαίου καὶ ἐπάνω, ἐπισκέψασθε αὐοτύς.
\vs{16}Καὶ ἐπεσκέψαντο αὐτοὺς Μωυσῆς καὶ Ἀαρὼν διὰ φωνῆς Κυρίου, ὃν τρόπον συνέταξεν αὐτοῖς Κύριος.

\vs{17}Καὶ ἦσαν οὗτοι οἱ υἱοὶ Λευὶ ἐξ ὀνομάτων αὐτῶν· Γεδσὼν, Καὰθ, καὶ Μεραρί.
\vs{18}Καὶ ταῦτα τὰ ὀνόματα τῶν υἱῶν Γεδσὼν κατὰ δήμους αὐτῶν· Λοβενὶ καὶ Σεμεΐ.
\vs{19}Καὶ υἱοὶ Καὰθ κατὰ δήμους αὐτῶν· Ἀμρὰμ καὶ Ἰσσαὰρ, Χεβρὼν καὶ Ὀζιήλ.
\vs{20}Καὶ υἱοὶ Μεραρὶ κατὰ δήμους αὐτῶν· Μοολὶ καὶ Μουσί· οὗτοί εἰσι δῆμοι τῶν Λευιτῶν κατʼ οἴκους πατριῶν αὐτῶν.
\vs{21}Τῷ Γεδσὼν δῆμος τοῦ Λοβενὶ, καὶ δῆμος τοῦ Σεμεΐ· οὗτοι δῆμοι τοῦ Γεδσών.
\vs{22}Ἡ ἐπίσκεψις αὐτῶν κατὰ ἀριθμὸν παντὸς ἀρσενικοῦ ἀπὸ μηνιαίου καὶ ἐπάνω, ἡ ἐπίσκεψις αὐτῶν, ἑπτακισχίλιοι καὶ πεντακόσιοι.
\vs{23}Καὶ οἱ υἱοὶ Γεδσὼν ὀπίσω τῆς σκηνῆς παρεμβαλοῦσι παρὰ θάλασσαν.
\vs{24}Καὶ ὁ ἄρχων οἴκου πατριᾶς τοῦ δήμου τοῦ Γεδσὼν, Ἑλισὰφ υἱὸς Δαήλ.
\vs{25}Καὶ ἡ φυλακὴ υἱῶν Γεδσὼν ἐν τῇ σκηνῇ τοῦ μαρτυρίου, ἡ σκηνὴ καὶ τὸ κάλυμμα, καὶ τὸ κατακάλυμμα τῆς θύρας τῆς σκηνῆς τοῦ μαρτυρίου,
\vs{26}καὶ τὰ ἱστία τῆς αὐλῆς, καὶ τὸ καταπέτασμα τῆς πύλης τῆς αὐλῆς τῆς οὔσης ἐπὶ τῆς σκηνῆς, καὶ τὰ κατάλοιπα πάντων τῶν ἔργων αὐτοῦ.

\vs{27}Τῷ Καὰθ δῆμος ὁ Ἀμρὰμ εἷς, καὶ δῆμος ὁ Ἰσσαὰρ εἷς, καὶ δῆμος ὁ Χεβρων εἷς, καὶ δῆμος ὁ Ὀζιὴλ εἷς· οὗτοί εἰσιν οἱ δῆμοι τοῦ Καὰθ, κατὰ ἀριθμόν.
\vs{28}Πᾶν ἀρσενικὸν ἀπὸ μηνιαίου καὶ ἐπάνω, ὀκτακισχίλιοι καὶ ἑξακόσιοι, φυλάσσοντες τὰς φυλακὰς τῶν ἁγίων.
\vs{29}Οἱ δῆμοι τῶν υἱῶν Καὰθ παρεμβαλοῦσιν ἐκ πλαγίων τῆς σκηνῆς κατὰ Λίβα.
\vs{30}Καὶ ὁ ἄρχων οἴκου πατριῶν τῶν δήμων τοῦ Καὰθ, Ἑλισαφὰν υἱὸς Ὀζιήλ.

\vs{31}Καὶ ἡ φυλακὴ αὐτῶν ἡ κιβωτὸς, καὶ ἡ τράπεζα, καὶ ἡ λυχνία, καὶ τὰ θυσιαστήρια, καὶ τὰ σκεύη τοῦ ἁγίου ὅσα λειτουργοῦσιν ἐν αὐτοῖς, καὶ τὸ κατακάλυμμα, καὶ πάντα τὰ ἔργα αὐτῶν.
\vs{32}Καὶ ὁ ἄρχων ἐπὶ τῶν ἀρχόντων τῶν Λευιτῶν, Ἐλεάζαρ ὁ υἱὸς Ἀαρὼν τοῦ ἱερέως, καθεσταμένος φυλάσσειν τὰς φυλακὰς τῶν ἁγίων.
\vs{33}Τῷ Μεραρὶ δῆμος ὁ Μοολὶ, καὶ δῆμος ὁ Μουσί· οὗτοί εἰσι δῆμοι τοῦ Μεραρί.
\vs{34}Ἡ ἐπίσκεψις αὐτῶν κατὰ ἀριθμὸν, πᾶν ἀρσενικὸν ἀπὸ μηνιαίου καὶ ἐπάνω, ἑξακισχίλιοι καὶ πεντήκοντα.
\vs{35}Καὶ ὁ ἄρχων οἴκου πατριῶν τοῦ δήμου τοῦ Μεραρὶ, Σουριὴλ υἱὸς Ἀβιχαίλ· ἐκ πλαγίων τῆς σκηνῆς παρεμβαλοῦσι πρὸς βοῤῥᾶν.
\vs{36}Ἡ ἐπίσκεψις τῆς φυλακῆς υἱῶν Μεραρὶ, τὰς κεφαλίδας τῆς σκηνῆς, καὶ τοὺς μοχλοὺς αὐτῆς, καὶ τοὺς στύλους αὐτῆς, καὶ τὰς βάσεις αὐτῆς, καὶ πάντα τὰ σκεύη αὐτῶν, καὶ τὰ ἔργα αὐτῶν,
\vs{37}καὶ τοὺς στύλους τῆς αὐλῆς κύκλῳ, καὶ τὰς βάσεις αὐτῶν, καὶ τοὺς πασσάλους, καὶ τοὺς κάλους αὐτῶν.

\vs{38}Οἱ παρεμβάλλοντες κατὰ πρόσωπον τῆς σκηνῆς τοῦ μαρτυρίου ἀπὸ ἀνατολῆς, Μωυσῆς καὶ Ἀαρὼν καὶ οἱ υἱοὶ αὐτοῦ, φυλάσσοντες τὰς φυλακὰς τοῦ ἁγίου εἰς τὰς φυλακὰς τῶν υἱῶν Ἰηραήλ· καὶ ὁ ἀλλογενὴς ὁ ἁπτόμενος, ἀποθανεῖται.
\vs{39}Πᾶσα ἡ ἐπίσκεψις τῶν Λευιτῶν, οὓς ἐπεσκέψατο Μωυσῆς καὶ Ἀαρὼν διὰ φωνῆς Κυρίου κατὰ δήμους αὐτῶν, πᾶν ἀρσενικὸν ἀπὸ μηνιαίου καὶ ἐπάνω, δύο καὶ εἴκοσι χιλιάδες.

\vs{40}Καὶ εἶπε Κύριος πρὸς Μωυσῆν, λέγων, ἐπίσκεψαι πᾶν πρωτότοκον ἄρσεν τῶν υἱῶν Ἰσραὴλ ἀπὸ μηνιαίου καὶ ἐπάνω· καὶ λάβετε τὸν ἀριθμὸν ἐξ ὀνόματος.
\vs{41}Καὶ λήψῃ τοὺς Λευίτας ἐμοί, ἐγὼ Κύριος, ἀντὶ πάντων τῶν πρωτοτόκων τῶν υἱῶν Ἰσραήλ, καὶ τὰ κτήνη τῶν Λευιτῶν ἀντὶ πάντων τῶν πρωτοτόκων ἐν τοῖς κτήνεσι τῶν υἱῶν Ἰσραήλ.
\vs{42}Καὶ ἐπεσκέψατο Μωυσῆς ὃν τρόπον ἐνετείλατο Κύριος πᾶν πρωτότοκον ἐν τοῖς υἱοῖς Ἰσραήλ.
\vs{43}Καὶ ἐγένοντο πάντα τὰ πρωτότοκα τὰ ἀρσενικὰ κατὰ ἀριθμὸν ἐξ ὀνόματος ἀπὸ μηνιαίου καὶ ἐπάνω ἐκ τῆς ἐπισκέψεως αὐτῶν, δύο καὶ εἴκοσι χιλιάδες καὶ τρεῖς καὶ ἑβδομήκοντα καὶ διακόσιοι.
\vs{44}Καὶ ἐλάλησε Κύριος πρὸς Μωυσῆν, λέγων,
\vs{45}λάβε τοὺς Λευίτας ἀντὶ πάντων τῶν πρωτοτόκων υἱῶν Ἰσραὴλ, καὶ τὰ κτήνη τῶν Λευιτῶν ἀντὶ τῶν κτηνῶν αὐτῶν, καὶ ἔσονται ἐμοὶ οἱ Λευῖται· ἐγὼ Κύριος.
\vs{46}Καὶ τὰ λύτρα τριῶν καὶ ἑβδομήκοντα καὶ διακοσίων οἱ πλεονάζοντες παρὰ τοὺς Λευίτας ἀπὸ τῶν πρωτοτόκων τῶν υἱῶν Ἰσραήλ·
\vs{47}Καὶ λήψῃ πέντε σίκλους κατὰ κεφαλὴν, κατὰ τὸ δίδραχμον τὸ ἅγιον λήψῃ, εἴκοσι ὀβολοὺς τοῦ σίκλου.
\vs{48}Καὶ δώσεις τὸ ἀργύριον Ἀαρὼν καὶ τοῖς υἱοῖς αὐτοῦ, λύτρα τῶν πλεοναζόντων ἐν αὐτοῖς.
\vs{49}Καὶ ἔλαβε Μωυσῆς τὸ ἀργύριον τὰ λύτρα τῶν πλεοναζόντων εἰς τὴν ἐκλύτρωσιν τῶν Λευιτῶν.
\vs{50}Παρὰ τῶν πρωτοτόκων τῶν υἱῶν Ἰσραὴλ ἔλαβε τὸ ἀργύριον, χιλίους τριακοσίους ἑξηκονταπέντε σίκλους, κατὰ τὸν σίκλον τὸν ἅγιον.
\vs{51}Καὶ ἔδωκε Μωυσῆς τὰ λύτρα τῶν πλεοναζόντων Ἀαρὼν καὶ τοῖς υἱοῖς αὐτοῦ, διὰ φωνῆς Κυρίου, ὃν τρόπον συνέταξε Κύριος τῷ Μωυσῇ.

\ch{4}
Καὶ ἐλάλησε Κύριος πρὸς Μωυσῆν καὶ Ἀαρὼν, λέγεν,
\vs{2}λάβε τὸ κεφάλαιον τῶν υἱῶν Καὰθ ἐκ μέσου υἱῶν Λευὶ, κατὰ δήμους αὐτῶν, κατʼ οἴκους πατριῶν αὐτῶν,
\vs{3}ἀπὸ εἴκοσι καὶ πέντε ἐτῶν καὶ ἐπάνω ἕως πεντήκοντα ἐτῶν, πᾶς ὁ εἰσπορευόμενος λειτουργεῖν, ποιῆσαι πάντα τὰ ἔργα ἐν τῇ σκηνῇ τοῦ μαρτυρίου.

\vs{4}Καὶ ταῦτα τὰ ἔργα τῶν υἱῶν Καὰθ ἐν τῇ σκηνῇ τοῦ μαρτυρίου· ἅγιον τῶν ἁγίων.
\vs{5}Καὶ εἰσελεύσεται Ἀαρὼν καὶ υἱοὶ αὐτοῦ, ὅταν ἐξαίρῃ ἡ παρεμβολὴ, καὶ καθελοῦσι τὸ καταπέτασμα τὸ συσκιάζον, καὶ κατακαλύψουσιν ἐν αὐτῷ τὴν κιβωτὸν τοῦ μαρτυρίου,
\vs{6}καὶ ἐπιθήσουσιν ἐπʼ αὐτὸ κατακάλυμμα δέρμα ὑακίνθινον, καὶ ἐπιβαλοῦσιν ἐπʼ αὐτὴν ἱμάτιον ὅλον ὑακίνθινον ἄνωθεν, καὶ διεμβαλοῦσι τοὺς ἀναφορεῖς.

\vs{7}Καὶ ἐπί τὴν τράπεζαν τὴν προκειμένην ἐπιβαλοῦσιν ἐπʼ αὐτὴν ἱμάτιον ὁλοπόρφυρον, καὶ τὰ τρυβλία, καὶ τὰς θυΐσκας, καὶ τοὺς κυάθους, καὶ τὰ σπονδεῖα ἐν οἷς σπένδει, καὶ οἱ ἄρτοι οἳ διαπαντὸς ἐπʼ αὐτῆς ἔσονται.
\vs{8}Καὶ ἐπιβαλοῦσιν ἐπʼ αὐτὴν ἱμάτιον κόκκινον, καὶ καλύψουσιν αὐτὴν καλύμματι δερματίνῳ ὑακινθίνῳ, καὶ διεμβαλοῦσι διʼ αὐτῆς τοὺς ἀναφορεῖς.
\vs{9}Καὶ λήψονται ἱμάτιον ὑακίνθινον, καὶ καλύψουσι τὴν λυχνίαν τὴν φωτίζουσαν, καὶ τοὺς λύχνους αὐτῆς, καὶ τὰς λαβίδας αὐτῆς, καὶ τὰς ἐπαρυστρίδας αὐτῆς, καὶ πάντα τὰ ἀγγεῖα τοῦ ἐλαίου οἷς λειτουργοῦσιν ἐν αὐτοῖς.
\vs{10}Καὶ ἐμβαλοῦσιν αὐτὴν, καὶ πάντα τὰ σκεύη αὐτῆς, εἰς κάλυμμα δερμάτινον ὑακίνθινον, καὶ ἐπιθήσουσιν αὐτὴν ἐπʼ ἀναφορέων.
\vs{11}Καὶ ἐπὶ τὸ θυσιαστήριον τὸ χρυσοῦν ἐπικαλύψουσιν ἱμάτιον ὑακίνθινον, καὶ καλύψουσιν αὐτὸ καλύμματι δερματίνῳ ὑακινθίνῳ, καὶ διεμβαλοῦσι τοὺς ἀναφορεῖς αὐτοῦ.

\vs{12}καὶ λήψονται πάντα τὰ σκεύη τὰ λειτουργικὰ ὅσα λειτουργοῦσιν ἐν αὐτοῖς ἐν τοῖς ἁγίοις· καὶ ἐμβαλοῦσιν εἰς ἱμάτιον ὑακίνθινον, καὶ καλύψουσιν αὐτὰ καλύμματι δερματίνῳ ὑακινθίνῳ, καὶ ἐπιθήσουσιν ἐπὶ ἀναφορεῖς.
\vs{13}Καὶ τὸν καλυπτῆρα ἐπιθήσει ἐπὶ τὸ θυσιαστήριον, καὶ ἐπικαλύψουσιν ἐπʼ αὐτὸ ἱμάτιον ὁλοπόρφυρον.
\vs{14}Καὶ ἐπιθήσουσιν ἐπʼ αὐτὸ πάντα τὰ σκεύη ὅσοις λειτουργοῦσιν ἐπʼ αὐτῷ ἐν αὐτοῖς, καὶ τὰ πυρεῖα, καὶ τὰς κρεάγρας, καὶ τὰς φιάλας, καὶ τὸν καλυπτῆρα, καὶ πάντα τὰ σκεύη τοῦ θυσιαστηρίου· καὶ ἐπιβαλοῦσιν ἐπʼ αὐτὸ κάλυμμα δερμάτινον ὑακίνθινον, καὶ διεμβαλοῦσι τοὺς ἀναφορεῖς αὐτοῦ· καὶ λήψονται ἱμάτιον πορφυροῦν, καὶ συγκαλύψουσι τὸν λουτῆρα καὶ τὴν βάσιν αὐτοῦ, καὶ ἐμβαλοῦσιν αὐτὸ εἰς κάλυμμα δερμάτινον ὑακίνθινον, καὶ ἐπιθήσουσιν ἐπὶ ἀναφορεῖς,
\vs{15}καὶ συντελέσουσιν Ἀαρὼν καὶ οἱ υἱοὶ αὐτοῦ, καλύπτοντες τὰ ἅγια, καὶ πάντα τὰ σκεύη τὰ ἅγια, ἐν τῷ ἐξαίρειν τὴν παρεμβολήν· καὶ μετὰ ταῦτα εἰσελεύσονται υἱοὶ Καὰθ αἴρειν, καὶ οὐχ ἅψονται τῶν ἁγίων, ἵνα μὴ ἀποθάνωσι· ταῦτα ἀροῦσιν οἱ υἱοὶ Καὰθ ἐν τῇ σκηνῇ τοῦ μαρτυρίου.

\vs{16}Ἐπίσκοπος Ἐλεάζαρ υἱὸς Ἀαρὼν τοῦ ἱερέως, τὸ ἔλαιον τοῦ φωτὸς, καὶ τὸ θυμίαμα τῆς συνθέσεως, καὶ ἡ θυσία ἡ καθʼ ἡμέραν, καὶ τὸ ἔλαιον τῆς χρίσεως, ἡ ἐπισκοπὴ ὅλης τῆς σκηνῆς, καὶ ὅσα ἐστὶν ἐν αὐτῇ ἐν τῷ ἁγίῳ, ἐν πᾶσι τοῖς ἔργοις.

\vs{17}Καὶ ἐλάλησε Κύριος πρὸς Μωυσῆν καὶ Ἀαρὼν, λέγων,
\vs{18}μὴ ὀλοθρεύσητε τῆς φυλῆς τὸν δῆμον τὸν Καὰθ ἐκ μέσου τῶν Λευιτῶν.
\vs{19}Τοῦτο ποιήσατε αὐτοῖς, καὶ ζήσονται καὶ οὐ μὴ ἀποθάνωσι, προσπορευομένων αὐτῶν πρὸς τὰ ἅγια τῶν ἁγίων· Ἀαρὼν καὶ οἱ υἱοὶ αὐτοῦ προσπορεύεσθωσαν, καὶ καταστήσουσιν αὐτοὺς ἕκαστον κατὰ τὴν ἀναφορὰν αὐτοῦ,
\vs{20}καὶ οὐ μὴ εἰσέλθωσιν ἰδεῖν ἐξάπινα τὰ ἅγια, καὶ ἀποθανοῦνται.

\vs{21}Καὶ ἐλάλησε Κύριος πρὸς Μωυσῆν, λέγων,
\vs{22}λάβε τὴν ἀρχὴν τῶν υἱῶν Γεδσὼν, καὶ τούτους κατʼ οἴκους πατριῶν αὐτῶν, κατὰ δήμους αὐτῶν,
\vs{23}ἀπὸ πέντε καὶ εἰκοσαετοῦς καὶ ἐπάνω ἕως πεντηκονταετοῦς ἐπίσκεψαι αὐτοὺς, πᾶς ὁ εἰσπορευόμενος λειτουργεῖν, ποιεῖν τὰ ἔργα αὐτοῦ ἐν τῇ σκηνῇ τοῦ μαρτυρίου.
\vs{24}Αὕτη ἡ λειτουργία τοῦ δήμου τοῦ Γεδσὼν, λειτουργεῖν καὶ αἴρειν.
\vs{25}Καὶ ἀρεῖ τὰς δέῤῥεις τῆς σκηνῆς, καὶ τὴν σκηνὴν τοῦ μαρτυρίου, καὶ τὸ κάλυμμα αὐτῆς, καὶ τὸ κατακάλυμμα τὸ ὑακίνθινον τὸ ὂν ἐπʼ αὐτῆς ἄνωθεν, καὶ τὸ κάλυμμα τῆς θύρας τῆς σκηνῆς τοῦ μαρτυρίου,
\vs{26}καὶ τὰ ἱστία τῆς αὐλῆς, ὅσα ἐπὶ τῆς σκηνῆς τοῦ μαρτυρίου, καὶ τὰ περισσὰ, καὶ πάντα τὰ σκεύη τὰ λειτουργικὰ ὅσα λειτουργοῦσιν ἐν αὐτοῖς ποιήσουσι.
\vs{27}Κατὰ στόμα Ἀαρὼν καὶ τῶν υἱῶν αὐτοῦ ἔσται ἡ λειτουργία τῶν υἱῶν Γεδσὼν κατὰ πάσας τὰς λειτουργίας αὐτῶν, καὶ κατὰ πάντα τὰ ἔργα αὐτῶν· καὶ ἐπισκέψῃ αὐτοὺς ἐξ ὀνόματος πάντα τὰ ἀρτὰ ὑπʼ αὐτῶν.
\vs{28}Αὕτη ἡ λειτουργία τῶν υἱῶν Γεδσὼν ἐν τῇ σκηνῇ τοῦ μαρτυρίου, καὶ ἡ φυλακὴ αὐτῶν ἐν χειρὶ Ἰθάμαρ τοῦ υἱοῦ Ἀαρὼν τοῦ ἱερέως.

\vs{29}Οἱ υἱοὶ Μεραρὶ κατὰ δήμους αὐτῶν, κατʼ οἴκους πατριῶν αὐτῶν, ἐπισκέψασθε αὐτοὺς,
\vs{30}ἀπὸ πέντε καὶ εἰκοσαετοῦς καὶ ἐπάνω ἕως πεντηκονταετοῦς ἐπισκέψασθε αὐτοὺς, πᾶς ὁ εἰσπορευόμενος λειτουργεῖν τὰ ἔργα τῆς σκηνῆς τοῦ μαρτυρίου.
\vs{31}Καὶ ταῦτα τὰ φυλάγματα τῶν αἰρομένων ὑπʼ αὐτῶν κατὰ πάντα τὰ ἔργα αὐτῶν ἐν τῇ σκηνῇ τοῦ μαρτυρίου· τὰς κεφαλίδας τῆς σκηνῆς, καὶ τοὺς μοχλοὺς, καὶ τοὺς στύλους αὐτῆς, καὶ τὰς βάσεις αὐτῆς, καὶ τὸ κατακάλυμμα, καὶ αἱ βάσεις αὐτῶν, καὶ οἱ στύλοι αὐτῶν, καὶ τὸ κατακάλυμμα τῆς θύρας τῆς σκηνῆς,
\vs{32}καὶ τοὺς στύλους τῆς αὐλῆς κύκλῳ, καὶ αἱ βάσεις αὐτῶν, καὶ τοὺς στύλους τοῦ καταπετάσματος τῆς πύλης τῆς αὐλῆς, καὶ τὰς βάσεις αὐτῶν, καὶ τοὺς πασσάλους αὐτῶν, καὶ τοὺς κάλους αὐτῶν, καὶ πάντα τὰ σκεύη αὐτῶν, καὶ πάντα τὰ λειτουργήματα αὐτῶν· ἐξ ὀνομάτων ἐπισκέψασθε αὐτοὺς, καὶ πάντα τὰ σκεύη τῆς φυλακῆς τῶν αἰρομένων ὑπʼ αὐτῶν.
\vs{33}Αὕτη ἡ λειτουργία δήμου υἱῶν Μεραρὶ ἐν πᾶσι τοῖς ἔργοις αὐτῶν ἐν τῇ σκηνῇ τοῦ μαρτυρίου ἐν χειρὶ Ἰθάμαρ τοῦ υἱοῦ Ἀαρὼν τοῦ ἱερέως.

\vs{34}Καὶ ἐπεσκέψατο Μωυσῆς καὶ Ἀαρὼν καὶ οἱ ἄρχοντες Ἰσραὴλ τοὺς υἱοὺς Καὰθ κατὰ δήμους αὐτῶν, κατʼ οἴκους πατριῶν αὐτῶν,
\vs{35}ἀπὸ πέντε καὶ εἰκοσαετοῦς καὶ ἐπάνω ἕως πεντηκονταετοῦς, πᾶς ὁ εἰσπορευόμενος λειτουργεῖν καὶ ποιεῖν ἐν τῇ σκηνῇ τοῦ μαρτυρίου.
\vs{36}Καὶ ἐγένετο ἡ ἐπίσκεψις αὐτῶν κατὰ δήμους αὐτῶν, δισχίλιοι ἑπτακόσιοι πεντήκοντα.
\vs{37}Αὕτη ἡ ἐπίσκεψις δήμου Καάθ, πᾶς ὁ λειτουργῶν ἐν τῇ σκηνῇ τοῦ μαρτυρίου, καθὰ ἐπεσκέψατο Μωυσῆς καὶ Ἀαρὼν διὰ φωνῆς Κυρίου, ἐν χειρὶ Μωυσῆ.

\vs{38}Καὶ ἐπεσκέπησαν υἱοὶ Γεδσὼν κατὰ δήμους αὐτῶν, κατʼ οἴκους πατριῶν αὐτῶν,
\vs{39}ἀπὸ πέντε καὶ εἰκοσαετοῦς καὶ ἐπάνω ἕως πεντηκονταετοῦς, πᾶς ὁ εἰσπορευόμενος λειτουργεῖν καὶ ποιεῖν τὰ ἔργα ἐν τῇ σκηνῇ τοῦ μαρτυρίου.
\vs{40}Καὶ ἐγένετο ἡ ἐπίσκεψις αὐτῶν, κατὰ δήμους αὐτῶν, κατʼ οἴκους πατριῶν αὐτῶν, δισχίλιοι ἑξακόσιοι τριάκοντα.
\vs{41}Αὕτη ἡ ἐπίσκεψις δήμου υἱῶν Γεδσὼν, πᾶς ὁ λειτουργῶν ἐν τῇ σκηνῇ τοῦ μαρτυρίου, οὓς ἐπεσκέψατο Μωυσῆς καὶ Ἀαρὼν διὰ φωνῆς Κυρίου, ἐν χειρὶ Μωυσῆ.

\vs{42}Ἐπεσκέπησαν δὲ καὶ δῆμος υἱῶν Μεραρὶ κατὰ δήμους αὐτῶν, κατʼ οἴκους πατριῶν αὐτῶν,
\vs{43}ἀπὸ πέντε καὶ εἰκοσαετοῦς καὶ ἐπάνω ἕως πεντηκονταετοῦς, πᾶς ὁ εἰσπορευόμενος λειτουργεῖν πρὸς τὰ ἔργα τῆς σκηνῆς τοῦ μαρτυρίου.
\vs{44}Καὶ ἐγενήθη ἡ ἐπίσκεψις αὐτῶν κατὰ δήμους αὐτῶν, κατʼ οἴκους πατριῶν αὐτῶν, τρισχίλιοι καὶ διακόσιοι.
\vs{45}Αὕτη ἡ ἐπίσκεψις δήμου υἱῶν Μεραρὶ, οὓς ἐπεσκέψατο Μωυσῆς καὶ Ἀαρὼν διὰ φωνῆς Κυρίου, ἐν χειρὶ Μωυσῆ.
\vs{46}Πάντες οἱ ἐπεσκεμμένοι, οὓς ἐπεσκέψατο Μωυσῆς καὶ Ἀαρὼν καὶ οἱ ἄρχοντες Ἰσραὴλ τοὺς Λευίτας, κατὰ δήμους καὶ κατʼ οἴκους πατριῶν αὐτῶν,
\vs{47}ἀπὸ πέντε καὶ εἰκοσαετοῦς καὶ ἐπάνω ἕως πεντηκονταετοῦς, πᾶς ὁ εἰσπορευόμενος πρὸς τὸ ἔργον τῶν ἔργων, καὶ τὰ ἔργα τὰ αἰρόμενα ἐν τῇ σκηνῇ τοῦ μαρτυρίου.
\vs{48}Καὶ ἐγενήθησαν οἱ ἐπισκεπέντες, ὀκτακισχίλιοι πεντακόσιοι ὀγδοήκοντα.
\vs{49}Διὰ φωνῆς Κυρίου ἐπεσκέψατο αὐτοὺς ἐν χειρὶ Μωυσῆ, ἄνδρα κατὰ ἄνδρα ἐπὶ τῶν ἔργων αὐτῶν, καὶ ἐπὶ ὧν αἴρουσιν αὐτοί· καὶ ἐπεσκέπησαν, ὃν τρόπον συνέταξε Κύριος τῷ Μωυσῇ.

\ch{5}
Καὶ ἐλάλησε Κύριος πρὸς Μωυσῆν, λέγων,
\vs{2}πρόσταξον τοῖς υἱοῖς Ἰσραὴλ, καὶ ἐξαποστειλάτωσαν ἐκ τῆς παρεμβολῆς πάντα λεπρὸν, καὶ πάντα γονοῤῥυῆ, καὶ πάντα ἀκάθαρτον ἐπὶ ψυχῇ.
\vs{3}Ἀπὸ ἀρσενικοῦ ἕως θηλυκοῦ, ἐξαποστείλατε ἔξω τῆς παρεμβολῆς, καὶ οὐ μὴ μιανοῦσι τὰς παρεμβολὰς αὐτῶν, ἐν οἷς ἐγὼ καταγίνομαι ἐν αὐτοῖς.
\vs{4}Καὶ ἐποίησαν οὕτως οἱ υἱοὶ Ἰσραὴλ, καὶ ἐξαπέστειλαν αὐτοὺς ἔξω τῆς παρεμβολῆς· καθὰ ἐλάλησε Κύριος Μωυσῇ, οὕτως ἐποίησαν οἱ υἱοὶ Ἰσραήλ.

\vs{5}Καὶ ἐλάλησε Κύριος πρὸς Μωυσῆν, λέγων,
\vs{6}λάλησον τοῖς υἱοῖς Ἰσραὴλ, λέγων, ἀνὴρ ἢ γυνὴ, ὅστις ἂν ποιήσῃ ἀπὸ πασῶν τῶν ἁμαρτιῶν τῶν ἀνθρωπίνων, καὶ παριδὼν παρίδῃ καὶ πλημμελήσῃ ἡ ψυχὴ ἐκείνη,
\vs{7}ἐξαγορεύσει τὴν ἁμαρτίαν, ἣν ἐποίησε, καὶ ἀποδώσει τὴν πλημμέλειαν· τὸ κεφάλαιον, καὶ τὸ ἐπίπεμπτον αὐτοῦ προσθήσει ἐπʼ αὐτὸ, καὶ ἀποδώσει τίνι ἐπλημμέλησεν αὐτῷ.
\vs{8}Ἐὰν δὲ μὴ ᾖ τῷ ἀνθρώπῳ ὁ ἀγχιστεύων, ὥστε ἀποδοῦναι αὐτῷ τὸ πλημμέλημα πρὸς αὐτὸν, τὸ πλημμέλημα τὸ ἀποδιδόμενον Κυρίῳ, τῷ ἱερεῖ ἔσται, πλὴν τοῦ κριοῦ τοῦ ἱλασμοῦ, διʼ οὗ ἐξιλάσεται ἐν αὐτῷ περὶ αὐτοῦ.

\vs{9}Καὶ πᾶσα ἀπαρχὴ κατὰ πάντα τὰ ἁγιαζόμενα ἐν υἱοῖς Ἰσραὴλ, ὅσα ἐὰν προσφέρωσι κυρίῳ, τῷ ἱερεῖ αὐτῷ ἔσται·
\vs{10}Καὶ ἑκάστου τὰ ἡγιασμένα, αὐτοῦ ἔσται· καὶ ἀνὴρ, ὃς ἂν δῷ τῷ ἱερεῖ, αὐτῷ ἔσται.

\vs{11}Καὶ ἐλάλησε Κύριος πρὸς Μωυσῆν, λέγων,
\vs{12}λάλησον τοῖς υἱοῖς Ἰσραὴλ, καὶ ἐρεῖς πρὸς αὐτοὺς, ἀνδρὸς ἀνδρὸς ἐὰν παραβῇ ἡ γυνὴ αὐτοῦ, καὶ ὑπεριδοῦσα παρίδῃ αὐτὸν,
\vs{13}καὶ κοιμηθῇ τις μετʼ αὐτῆς κοίτην σπέρματος, καὶ λάθῃ ἐξ ὀφθαλμῶν τοῦ ἀνδρὸς αὐτῆς, καὶ κρύψῃ, αὐτὴ δὲ ᾖ μεμιασμένη, καὶ μάρτυς μὴ ἦν μετʼ αὐτῆς, καὶ αὐτὴ μὴ ᾖ συνειλημμένη,
\vs{14}καὶ ἐπέλθῃ αὐτῷ πνεῦμα ζηλώσεως, καὶ ζηλώσῃ τὴν γυναῖκα αὐτοῦ, αὐτὴ δὲ μεμίανται, ἢ ἐπέλθῃ αὐτῷ πνεῦμα ζηλώσεως, καὶ ζηλώσῃ τὴν γυναῖκα αὐτοῦ, αὐτὴ δὲ μὴ ᾖ μεμιασμένη,
\vs{15}καὶ ἄξει ὁ ἄνθρωπος τὴν γυναῖκα αὐτοῦ πρὸς τὸν ἱερέα, καὶ προσοίσει τὸ δῶρον περὶ αὐτῆς, τὸ δέκατον τοῦ οἰφὶ ἄλευρον κρίθινον· οὐκ ἐπιχεεῖ ἐπʼ αὐτὸ ἔλαιον, οὐδὲ ἐπιθήσει ἐπʼ αὐτὸ λίβανον· ἔστι γὰρ θυσία ζηλοτυπίας, θυσία μνημοσύνου, ἀναμιμνήσκουσα ἁμαρτίαν.

\vs{16}Καὶ προσάξει αὐτὴν ὁ ἱερεὺς, καὶ στήσει αὐτὴν ἔναντι Κυρίου.
\vs{17}Καὶ λήψεται ὁ ἱερεὺς ὕδωρ καθαρὸν ζῶν ἐν ἀγγείῳ ὀστρακίνῳ, καὶ τῆς γῆς τῆς οὔσης ἐπὶ τοῦ ἐδάφους τῆς σκηνῆς τοῦ μαρτυρίου, καὶ λαβὼν ὁ ἱερεὺς ἐμβαλεῖ εἰς τὸ ὕδωρ.
\vs{18}Καὶ στήσει ὁ ἱερεὺς τὴν γυναῖκα ἔναντι Κυρίου, καὶ ἀποκαλύψει τὴν κεφαλὴν τῆς γυναικὸς, καὶ δώσει ἐπὶ τὰς χεῖρας αὐτῆς τὴν θυσίαν τοῦ μνημοσύνου, τὴν θυσίαν τῆς ζηλοτυπίας· ἐν δὲ τῇ χειρὶ τοῦ ἱερέως ἔσται τὸ ὕδωρ τοῦ ἐλεγμοῦ τοῦ ἐπικαταρωμένου τούτου.
\vs{19}Καὶ ὁρκιεῖ αὐτὴν ὁ ἱερεὺς, καὶ ἐρεῖ τῇ γυναικὶ, εἰ μὴ κεκοίμηταί τις μετὰ σοῦ, εἰ μὴ παραβέβηκας μιανθῆναι ὑπὸ τὸν ἄνδρα τὸν σεαυτῆς, ἀθῶα ἴσθι ἀπὸ τοῦ ὕδατος τοῦ ἐλεγμοῦ τοῦ ἐπικαταρωμένου τούτου.
\vs{20}Εἰ δὲ σὺ παραβέβηκας ὕπανδρος οὖσα, ἢ μεμίανσαι, καὶ ἔδωκέ τις τὴν κοίτην αὐτοῦ ἐν σοὶ, πλὴν τοῦ ἀνδρός σου·
\vs{21}Καὶ ὁρκιεῖ ὁ ἱερεὺς τὴν γυναῖκα ἐν τοῖς ὅρκοις τῆς ἀρᾶς ταύτης, καὶ ἐρεῖ ὁ ἱερεὺς τῇ γυναικὶ, δῴη σε Κύριος ἐν ἀρᾷ καὶ ἐνόρκιον ἐν μέσῳ τοῦ λαοῦ σου, ἐν τῷ δοῦναι Κύριον τὸν μηρόν σου διαπεπτωκότα, καὶ τὴν κοιλίαν σου πεπρησμένην.
\vs{22}Καὶ εἰσελεύσεται τὸ ὕδωρ τὸ ἐπικαταρώμενον τοῦτο εἰς τὴν κοιλίαν σου πρῆσαι γαστέρα, καὶ διαπεσεῖν μηρόν σου· καὶ ἐρεῖ ἡ γυνὴ, γένοιτο, γένοιτο.

\vs{23}Καὶ γράψει ὁ ἱερεὺς τὰς ἀρὰς ταύτας εἰς βιβλίον, καὶ ἐξαλείψει εἰς τὸ ὕδωρ τοῦ ἐλεγμοῦ τοῦ ἐπικαταρωμένου.
\vs{24}Καὶ ποτιεῖ τὴν γυναῖκα τὸ ὕδωρ τοῦ ἐλεγμοῦ τοῦ ἐπικαταρωμένου· καὶ εἰσελεύσεται εἰς αὐτὴν τὸ ὕδωρ τὸ ἐπικαταρώμενον τοῦ ἐλεγμοῦ.

\vs{25}Καὶ λήψεται ὁ ἱερεὺς ἐκ χειρὸς τῆς γυναικὸς τὴν θυσίαν τῆς ζηλοτυπίας, καὶ ἐπιθήσει τὴν θυσίαν ἔναντι Κυρίου, καὶ προσοίσει αὐτὴν πρὸς τὸ θυσιαστήριον.
\vs{26}Καὶ δράξεται ὁ ἱερεὺς ἀπὸ τῆς θυσίας τὸ μνημόσυνον αὐτῆς, καὶ ἀνοίσεται αὐτὸ ἐπὶ τὸ θυσιαστήριον, καὶ μετὰ. ταῦτα ποτιεῖ τὴν γυναῖκα τὸ ὕδωρ.
\vs{27}Καὶ ἔσται ἐὰν ᾖ μεμιασμένη καὶ λήθῃ λάθῃ τὸν ἄνδρα αὐτῆς, καὶ εἰσελεύσεται εἰς αὐτὴν τὸ ὕδωρ τοῦ ἐλεγμοῦ τὸ ἐπικαταρώμενον, καὶ πρησθήσεται τὴν κοιλίαν, καὶ διαπεσεῖται ὁ μηρὸς αὐτῆς, καὶ ἔσται ἡ γυνὴ εἰς ἀρὰν τῷ λαῷ αὐτῆς.
\vs{28}Ἐὰν δὲ μὴ μιανθῇ ἡ γυνὴ, καὶ καθαρὰ ᾖ, καὶ ἀθῶα ἔσται καὶ ἐκσπερματιεῖ σπέρμα.
\vs{29}Οὗτος ὁ νόμος τῆς ζηλοτυπίας, ᾧ ἂν παραβῇ ἡ γυνὴ ὕπανδρος οὖσα, καὶ μιανθῇ.
\vs{30}Ἢ ἄνθρωπος ὃς ἐὰν ἐπέλθῃ ἐπʼ αὐτὸν πνεῦμα ζηλώσεως, καὶ ζηλώσῃ τὴν γυναῖκα αὐτοῦ, καὶ στήσῃ τὴν γυναῖκα αὐτοῦ ἔναντι Κυρίου, καὶ ποιήσει αὐτῇ ὁ ἱερεὺς πάντα τὸν νόμον τοῦτον,
\vs{31}καὶ ἀθῶος ἔσται ὁ ἄνθρωπος ἀπὸ ἁμαρτίας· καὶ γυνὴ ἐκείνη λήψεται τὴν ἁμαρτίαν αὐτῆς.

\ch{6}
Καὶ ἐλάλησε Κύριος πρὸς Μωυσῆν, λέγων, λάλησον τοῖς υἱοῖς Ἰσραὴλ,
\vs{2}καὶ ἐρεῖς πρὸς αὐτοὺς, ἀνὴρ ἢ γυνὴ, ὃς ἂν μεγάλως εὔξηται εὐχὴν ἀφαγνίσασθαι ἁγνείαν Κυρίῳ,
\vs{3}ἀπὸ οἴνου καὶ σίκερα ἁγνισθήσεται· καὶ ὄξος ἐξ οἴνου καὶ ὄξος ἐκ σίκερα οὐ πίεται· καὶ ὅσα κατεργάζεται ἐκ σταφυλῆς οὐ πίεται· καὶ σταφυλὴν πρόσφατον καὶ σταφίδα οὐ φάγεται πάσας τὰς ἡμέρας τῆς εὐχῆς αὐτοῦ·
\vs{4}ἀπὸ πάντων ὅσα γίνεται ἐξ ἀμπέλου, οἶνον ἀπὸ στεμφύλων ἕως γιγάρτου οὐ φάγεται πάσας τὰς ἡμέρας τοῦ ἁγνισμοῦ·
\vs{5}ξυρὸν οὐκ ἐπελεύσεται ἐπὶ τὴν κεφαλὴν αὐτοῦ, ἓως ἄν πληρωθῶσιν αἱ ἡμέραι, ὅσας ηὐξατο Κυρίῳ· ἅγιος ἔσται τρέφων κόμην τρίχα κεφαλῆς πάσας τὰς ἡμέρας τῆς εὐχῆς Κυρίῳ·
\vs{6}ἐπὶ πάσῃ ψυχῇ τετελευτηκυίᾳ οὐκ εἰσελεύσεται ἐπὶ πατρὶ καὶ μητρὶ,
\vs{7}καὶ ἐπʼ ἀδελφῷ καὶ ἐπʼ ἀδελφῇ, οὐ μιανφήσεται ἐπʼ αὐτοὶς ἀποφανόντων αὐτῶν, ὅτι εὐχὴ Θεοῦ αὐτοῦ ἐπʼ αὐτῷ ἐπὶ κεφαλῆς αὐτοῦ·

\vs{8}Πάσας τὰς ἡμέρας τῆς εὐχῆς αὐτοῦ ἅγιος ἔσται Κυρίῳ.
\vs{9}Ἐὰν δέ τις ἀποθάνῃ ἐπʼ αὐτῷ ἐξάπινα, παραχρῆμα μιανθήσεται ἡ κεφαλὴ εὐχῆς αὐτοῦ· καὶ ξυρήσεται τὴν κεφαλὴν αὐτοῦ ᾗ ἂν ἡμέρᾳ καθαρισθῇ· τῇ ἡμέρᾳ τῇ ἑβδόμῃ ξυρηθήσεται.
\vs{10}Καὶ τῇ ἡμέρᾳ τῇ ὀγδόῃ οἴσει δύο τρυγόνας, ἢ δύο νοσσοὺς περιστερῶν πρὸς τὸν ἱερέα, ἐπὶ τὰς θύρας τῆς σκηνῆς τοῦ μαρτυρίου.

\vs{11}Καὶ ποιήσει ὁ ἱερεὺς μίαν περὶ ἁμαρτίας, καὶ μίαν εἰς ὁλοκαύτωμα· καὶ ἐξιλάσεται περὶ αὐτοῦ ὁ ἱερεὺς περὶ ὧν ἥμαρτε περὶ τῆς ψυχῆς· καὶ ἁγιάσει τὴν κεφαλὴν αὐτοῦ ἐν ἐκείνῃ τῇ ἡμέρᾳ, ᾗ ἡγιάσθη Κυρίῳ, τὰς ἡμέρας τῆς εὐχῆς·
\vs{12}καὶ προσάξει ἀμνὸν ἐνιαύσιον εἰς πλημμέλειαν· καὶ αἱ ἡμέραι αἱ πρότεραι ἄλογοι ἔσονται, ὅτι ἐμιάνθη ἡ κεφαλὴ εὐχῆς αὐτοῦ.

\vs{13}Καὶ οὗτος ὁ νόμος τοῦ εὐξαμένου· ᾗ ἂν ἡμέρᾳ πληρώσῃ ἡμέρας εὐχῆς αὐτοῦ, προσοίσει αὐτὸς παρὰ τὰς θύρας τῆς σκηνῆς τοῦ μαρτυρίου.
\vs{14}Καὶ προσάξει τὸ δῶρον αὐτοῦ Κυρίῳ ἀμνὸν ἐνιαύσιον ἄμωμον ἕνα εἰς ὁλοκαύτωσιν, καὶ ἀμνάδα ἐνιαυσίαν μίαν ἄμωμον εἰς ἁμαρτίαν, καὶ κριὸν ἕνα ἄμωμον εἰς σωτήριον,
\vs{15}καὶ κανοῦν ἀζύμων σεμιδάλεως ἄρτους ἀναπεποιημένους ἐν ἐλαίῳ, καὶ λάγανα ἄζυμα κεχρισμένα ἐν ἐλαίῳ, καὶ θυσίαν αὐτῶν, καὶ σπονδὴν αὐτῶν.
\vs{16}Καὶ προσοίσει ὁ ἱερεὺς ἔναντι Κυρίου, καὶ ποιήσει τὸ περὶ ἁμαρτίας αὐτοῦ, καὶ τὸ ὁλοκαύτωμα αὐτοῦ.
\vs{17}Καὶ τὸν κριὸν ποιήσει θυσίαν σωτηρίου τῷ Κυρίῳ ἐπὶ τῷ κανῷ τῶν ἀζύμων· καὶ ποιήσει ὁ ἱερεὺς τὴν θυσίαν αὐτοῦ, καὶ τὴν σπονδὴν αὐτοῦ.
\vs{18}Καὶ ξυρήσεται ὁ ηὐγμένος παρὰ τὰς θύρας τῆς σκηνῆς τοῦ μαρτυρίου τὴν κεφαλὴν τῆς εὐχῆς αὐτοῦ, καὶ ἐπιθήσει τὰς τρίχας ἐπὶ τὸ πῦρ, ὅ ἐστιν ὑπὸ τὴν θυσίαν τοῦ σωτηρίου.

\vs{19}Καὶ λήψεται ὁ ἱερεὺς τὸν βραχίονα ἐφθὸν ἀπὸ τοῦ κριοῦ, καὶ ἄρτον ἕνα ἄζυμον ἀπὸ τοῦ κανοῦ, καὶ λάγανον ἄζυμον ἕν, καὶ ἐπιθήσει ἐπὶ τὰς χεῖρας τοῦ ηὐγμένου μετὰ τὸ ξυρήσασθαι αὐτὸν τὴν εὐχὴν αὐτοῦ,
\vs{20}καὶ προσοίσει αὐτὰ ὁ ἱερεὺς ἐπίθεμα ἔναντι Κυρίου· ἅγιον ἔσται τῷ ἱερεῖ ἐπὶ τοῦ στηθηνίου τοῦ ἐπιθέματος, καὶ ἐπὶ τοῦ βραχίονος τοῦ ἀφαιρέματος· καὶ μετὰ ταῦτα πίεται ὁ ηὐγμένος οἶνον.
\vs{21}Οὗτος ὁ νόμος τοῦ εὐξαμένου, ὃς ἂν εὔξηται Κυρίῳ δῶρον αὐτοῦ Κυρίῳ περὶ τῆς εὐχῆς, χωρὶς ὧν ἂν εὕρῃ ἡ χεὶρ αὐτοῦ, κατὰ δύναμιν τῆς εὐχῆς αὐτοῦ, ἣν ἂν εὔξηται κατὰ νόμον ἁγνείας.

\vs{22}Καὶ ἐλάλησε Κύριος πρὸς Μωυσῆν, λέγων,
\vs{23}λάλησον Ἀαρὼν καὶ τοῖς υἱοῖς αὐτοῦ, λέγων, οὕτως εὐλογήσετε τοὺς υἱοὺς Ἰσραὴλ, λέγοντες αὐτοῖς,
\vs{24}εὐλογήσαι σε Κύριος, καὶ φυλάξαι, σε.
\vs{25}Ἐπιφάναι Κύριος τὸ πρόσωπον αὐτοῦ ἐπὶ σὲ, καὶ ἐλεήσαι σε.
\vs{26}Ἐπάραι Κύριος τὸ πρόσωπον αὐτοῦ ἐπὶ σὲ, καὶ δῴη σοι εἰρήνην.
\vs{27}Καὶ ἐπιθήσουσι τὸ ὄνομά μου ἐπὶ τοὺς υἱοὺς Ἰσραὴλ, καὶ ἐγὼ Κύριος εὐλογήσω αὐτούς.

\ch{7}
Καὶ ἐγένετο ᾗ ἡμέρᾳ συνετέλεσε Μωυσῆς, ὥστε ἀναστῆσαι τὴν σκηνήν, καὶ ἔχρισεν αὐτὴν, καὶ ἡγίασεν αὐτὴν, καὶ πάντα τὰ σκεύη αὐτῆς, καὶ τὸ θυσιαστήριον, καὶ πάντα τὰ σκεύη αὐτοῦ, καὶ ἔχρισεν αὐτὰ, καὶ ἡγίασεν αὐτά.
\vs{2}Καὶ προσήνεγκαν οἱ ἄρχοντες Ἰσραὴλ, δώδεκα ἄρχοντες οἴκων πατριῶν αὐτῶν· οὗτοι οἱ ἄρχοντες φυλῶν, οὗτοι οἱ παρεστηκότες ἐπὶ τῆς ἐπισκοπῆς.
\vs{3}Καὶ ἤνεγκαν τὸ δῶρον αὐτῶν ἔναντι Κυρίου, ἓξ ἁμάξας λαμπηνίκας, καὶ δώδεκα βόας· ἅμαξαν παρὰ δύο ἀρχόντων, καὶ μόσχον παρὰ ἑκάστου· καὶ προσήγαγον ἐναντίον τῆς σκηνῆς.
\vs{4}Καὶ εἶπε Κύριος πρὸς Μωυσῆν, λέγων,
\vs{5}λάβε παρʼ αὐτῶν, καὶ ἔσονται πρὸς τὰ ἔργα τὰ λειτουργικὰ τῆς σκηνῆς τοῦ μαρτυρίου· καὶ δώσεις αὐτὰ τοῖς Λευίταις, ἑκάστῳ κατὰ τὴν αὐτοῦ λειτουργίαν.
\vs{6}Καὶ λαβὼν Μωυσῆς τὰς ἁμάξας καὶ τοὺς βόας, ἔδωκεν αὐτὰ τοῖς Λευίταις.
\vs{7}Καὶ τὰς δύο ἁμάξας καὶ τοὺς τέσσαρας βόας ἔδωκε τοῖς υἱοῖς Γεδσὼν κατὰ τὰς λειτουργίας αὐτῶν.
\vs{8}Καὶ τὰς τέσσαρας ἁμάξας καὶ τοὺς ὀκτὼ βόας ἔδωκε τοῖς υἱοῖς Μεραρὶ κατὰ τὰς λειτουργίας αὐτῶν, διὰ Ἰθάμαρ υἱοῦ Ἀαρὼν τοῦ ἱερέως.
\vs{9}Καὶ τοῖς υἱοῖς Καὰθ οὐ δέδωκεν, ὅτι τὰ λειτουργήματα τοῦ ἁγίου ἔχουσιν· ἐπʼ ὤμων ἀροῦσι.

\vs{10}Καὶ προσήνεγκαν οἱ ἄρχοντες εἰς τὸν ἐγκαινισμὸν τοῦ θυσιαστηρίου, ἐν τῇ ἡμέρᾳ ᾗ ἔχρισεν αὐτὸ, καὶ προσήνεγκαν οἱ ἄρχοντες τὰ δῶρα αὐτῶν ἀπέναντι τοῦ θυσιαστηρίου.
\vs{11}Καὶ εἶπε Κύριος πρὸς Μωυσῆν, ἄρχων εἷς καθʼ ἡμέραν, ἄρχων καθʼ ἡμέραν προσοίσουσι τὰ δῶρα αὐτῶν εἰς τὸν ἐγκαινισμὸν τοῦ θυσιαστηρίου.

\vs{12}Καὶ ἦν ὁ προσφέρων ἐν τῇ ἡμέρᾳ τῇ πρώτῃ τὸ δῶρον αὐτοῦ, Ναασσὼν υἱὸς Ἀμιναδὰβ, ἄρχων τῆς φυλῆς Ἰούδα.
\vs{13}Καὶ προσήνεγκε τὸ δῶρον αὐτοῦ, τρυβλίον ἀργυροῦν ἕν, τριάκοντα καὶ ἑκατὸν ὁλκὴ αὐτοῦ· φιάλην μίαν ἀργυρᾶν, ἑβδομήκοντα σίκλων κατὰ τὸν σίκλον τὸν ἅγιον· ἀμφοτέρα πλήρη σεμιδάλεως ἀναπεποιημένης ἐν ἐλαίῳ εἰς θυσίαν.
\vs{14}Θυΐσκην μίαν δέκα χρυσῶν, πλήρη θυμιάματος.
\vs{15}Μόσχον ἕνα ἐκ βοῶν, κριὸν ἕνα, ἀμνὸν ἕνα ἐνιαύσιον εἰς ὁλοκαύτωμα,
\vs{16}καὶ χίμαρον ἐξ αἰγῶν ἕνα περὶ ἁμαρτίας.
\vs{17}Καὶ εἰς θυσίαν σωτηρίου δαμάλεις δύο, κριοὺς πέντε, τράγους πέντε, ἀμνάδας ἐνιαυσίας πέντε· τοῦτο δῶρον Ναασσὼν υἱοῦ Ἀμιναδάβ.

\vs{18}Τῇ ἡμέρᾳ τῇ δευτέρᾳ προσήνεγκε Ναθαναὴλ υἱὸς Σωγὰρ, ὅ ἄρχων τῆς φυλῆς Ἰσσάχαρ.
\vs{19}Καὶ προσήνεγκε τὸ δῶρον αὐτοῦ, τρυβλίον ἀργυροῦν ἓν, τριάκοντα καὶ ἑκατὸν ὁλκὴ αὐτοῦ· φιάλην μίαν ἀργυρᾶν, ἑβδομήκοντα σίκλων κατὰ τὸν σίκλον τὸν ἅγιον· ἀμφότερα πλήρη σεμιδάλεως ἀναπεποιημένης ἐν ἐλαίῳ εἰς θυσίαν.
\vs{20}Θυΐσκην μίαν δέκα χρυσῶν, πλήρη θυμιάματος.
\vs{21}Μόσχον ἔνα ἐκ βοῶν, κριὸν ἕνα, ἀμνὸν ἕνα ἐνιαύσιον εἰς ὁλοκαύτωμα,
\vs{22}καὶ χίμαρον ἐξ αἰγῶν ἕνα περὶ ἁμαρτίας.
\vs{23}Καὶ εἰς θυσίαν σωτηρίου δαμάλεις δύο, κριοὺς πέντε, τράγους πέντε, ἀμνάδας ἐνιαυσίας πέντε· τοῦτο τὸ δῶρον Ναθαναὴλ υἱοῦ Σωγάρ.

\vs{24}Τῇ ἡμέρᾳ τῇ τρίτῃ ἄρχων τῶν υἱῶν Σαβουλὼν, Ἑλιὰβ υἱὸς Χαιλών.
\vs{25}Τὸ δῶρον αὐτοῦ, τρυβλίον ἀργυροῦν ἕν, τριάκοντα καὶ ἑκατὸν ὁλκὴ αὐτοῦ· φιάλην μίαν ἀργυρᾶν, ἑβδομήκοντα σίκλων κατὰ τὸν σίκλον τὸν ἅγιον· ἀμφότερα πλήρη σεμιδάλεως ἀναπεποιημένης ἐν ἐλαίῳ εἰς θυσίαν·
\vs{26}Θυΐσκην μίαν δέκα χρυσῶν, πλήρη θυμιάματος.
\vs{27}Μόσχον ἕνα ἐκ βοῶν, κριὸν ἕνα, ἀμνὸν ἕνα ἐνιαύσιον εἰς ὁλοκαύτωμα,
\vs{28}καὶ χίμαρον ἐξ αἰγῶν ἕνα περὶ ἁμαρτίας.
\vs{29}Καὶ εἰς θυσίαν σωτηρίου δαμάλεις δύο, κριοὺς πέντε, τράγους πέντε, ἀμνάδας ἐνιαυσίας πέντε· τοῦτο τὸ δῶρον Ἑλιὰβ υἱοῦ Χαιλών.

\vs{30}Τῇ ἡμέρᾳ τῇ τετάρτῃ ἄρχων τῶν υἱῶν Ῥουβὴν, Ἑλισοῦρ υἱὸς Σεδιούρ.
\vs{31}Τὸ δῶρον αὐτοῦ, τρυβλίον ἀργυροῦν ἓν, τριάκοντα καὶ ἑκατὸν ὁλκὴ αὐτοῦ· φιάλην μίαν ἀργυρᾶν, ἑβδομήκοντα σίκλων κατὰ τὸν σίκλον τὸν ἅγιον· ἀμφότερα πλήρη σεμιδάλεως ἀναπεποιημένης ἐν ἐλαίῳ εἰς θυσίαν·
\vs{32}Θυΐσκην μίαν δέκα χρυσῶν, πλήρη θυμιάματος.
\vs{33}Μόσχον ἕνα ἐκ βοῶν, κριὸν ἕνα, ἀμνὸν ἕνα ἐνιαύσιον εἰς ὁλοκαύτωμα,
\vs{34}καὶ χίμαρον ἐξ αἰγῶν ἕνα περὶ ἁμαρτίας.
\vs{35}Καὶ εἰς θυσίαν σωτηρίου δαμάλεις δύο, κριοὺς πέντε, τράγους πέντε, ἀμνάδας ἐνιαυσίας πέντε· τοῦτο τὸ δῶρον Ἑλισοὺρ υἱοῦ Σεδιούρ.

\vs{36}Τῇ ἡμέρᾳ τῇ πέμπτῃ ἄρχων τῶν υἱῶν Συμεὼν, Σαλαμιὴλ υἱὸς Σουρισαδαί.
\vs{37}Τὸ δῶρον αὐτοῦ, τρυβλίον ἀργυροῦν ἓν, τριάκοντα καὶ ἑκατὸν ὁλκὴ αὐτοῦ· φιάλην μίαν ἀργυρᾶν, ἑβδομήκοντα σίκλων κατὰ τὸν σίκλον τὸν ἅγιον· ἀμφότερα πλήρη σεμιδάλεως ἐναπεποιημένης ἐν ἐλαίῳ εἰς θυσίαν.
\vs{38}Θυϊσκην μίαν δέκα χρυσῶν, πλήρη θυμιάματος.
\vs{39}Μόσχον ἕνα ἐκ βοῶν, κριὸν ἕνα, ἀμνὸν ἕνα ἐνιαύσιον εἰς ὁλοκαύτωμα,
\vs{40}καὶ χίμαρον ἐξ αἰγῶν ἕνα περὶ ἁμαρτίας.
\vs{41}Καὶ εἰς θυσίαν σωτηρίου δαμάλεις δύο, κριοὺς πέντε, τράγους πέντε, ἀμνάδας ἐνιαυσίας πέντε· τοῦτο τὸ δῶρον Σαλαμιὴλ υἱοῦ Σουρισαδαί.

\vs{42}Τῇ ἡμέρᾳ τῇ ἕκτῃ ἄρχων τῶν υἱῶν Γὰδ, Ἐλεισὰφ υἱὸς Ῥαγουήλ.
\vs{43}Τὸ δῶρον αὐτοῦ, τρυβλίον ἀργυροῦν ἕν, τριάκοντα καὶ ἑκατὸν ὁλκὴ αὐτοῦ· φιάλην μίαν ἀργυρᾶν, ἑβδομήκοντα σίκλων κατὰ τὸν σίκλον τὸν ἅγιον· ἀμφότερα πλήρη σεμιδάλεως ἀναπεποιημένης ἐν ἐλαίῳ εἰς θυσίαν.
\vs{44}Θυΐσκην μίαν δέκα χρυσῶν, πλήρη θυμιάματος.
\vs{45}Μόσχον ἕνα ἐκ βοῶν, κριὸν ἕνα, ἀμνὸν ἕνα ἐνιαύσιον εἰς ὁλοκαύτωμα,
\vs{46}καὶ χίμαρον ἐξ αἰγῶν ἕνα περὶ ἁμαρτίας.
\vs{47}Καὶ εἰς θυσίαν σωτηρίου δαμάλεις δύο, κριοὺς πέντε, τράγους πέντε, ἀμνάδας ἐνιαυσίας πέντε· τοῦτο τὸ δῶρον Ἐλισὰφ υἱοῦ Ῥαγουήλ.

\vs{48}Τῇ ἡμέρᾳ τῇ ἑβδόμῃ ἄρχων τῶν υἱῶν Ἐφραὶμ, Ἐλισαμὰ υἱὸς Ἐμιούδ.
\vs{49}Τὸ δῶρον αὐτοῦ, τρυβλίον ἀργυροῦν ἓν, τριάκοντα καὶ ἑκατὸν ὁλκὴ αὐτοῦ· φιάλην μίαν ἀργυρᾶν, ἑβδομήκοντα σίκλων κατὰ τὸν σίκλον τὸν ἅγιον· ἀμφότερα πλήρη σεμιδάλεως ἀναπεποιημένης ἐν ἐλαίῳ εἰς θυσίαν.
\vs{50}Θυΐσκην μίαν δέκα χρυσῶν, πλήρη θυμιάματος.
\vs{51}Μόσχον ἕνα ἐκ βοῶν, κριὸν ἕνα, ἀμνὸν ἕνα ἐνιαύσιον εἰς ὁλοκαύτωμα,
\vs{52}καὶ χίμαρον ἐξ αἰγῶν ἕνα περὶ ἁμαρτίας.
\vs{53}Καὶ εἰς θυσίαν σωτηρίου δαμάλεις δύο, κριοὺς πέντε, τράγους πέντε, ἀμνάδας ἐνιαυσίας πέντε· τοῦτο τὸ δῶρον Ἐλισαμὰ υἱοῦ Ἐμιούδ.

\vs{54}Τῇ ἡμέρᾳ τῇ ὀγδόῃ ἄρχων τῶν υἱῶν Μανασσῆ, Γαμαλιὴλ υἱὸς Φαδασσούρ.
\vs{55}Τὸ δῶρον αὐτοῦ, τρυβλίον ἀργυροῦν ἓν, τριάκοντα καὶ ἑκατὸν ὁλκὴ αὐτοῦ· φιάλην μίαν ἀργυρᾶν, ἑβδομήκοντα σίκλων κατὰ τὸν σίκλον τὸν ἅγιον· ἀμφότερα πλήρη σεμιδάλεως ἀναπεποιημένης ἐν ἐλαίῳ εἰς θυσίαν.
\vs{56}Θυΐσκην μίαν δέκα χρυσῶν, πλήρη θυμιάματος.
\vs{57}Μόσχον ἕνα ἐκ βοῶν, κριὸν ἕνα, ἀμνὸν ἕνα ἐνιαύσιον εἰς ὁλοκαύτωμα,
\vs{58}καὶ χίμαρον ἐξ αἰγῶν ἕνα περὶ ἁμαρτίας·
\vs{59}Καὶ εἰς θυσίαν σωτηρίου δαμάλεις δύο, κριοὺς πέντε, τράγους πέντε, ἀμνάδας ἐνιαυσίας πέντε· τοῦτο τὸ δῶρον Γαμαλιὴλ υἱοὺ Φαδασσούρ.

\vs{60}Τῇ ἡμέρᾳ τῇ ἐνάτῃ ἄρχων τῶν υἱῶν Βενιαμὶν, Ἀβιδὰν υἱὸς Γαδεωνί.
\vs{61}Τὸ δῶρον αὐτοῦ, τρυβλίον ἀργυροῦν ἓν, τριάκοντα καὶ ἑκατὸν ὁλκὴ αὐτοῦ· φιάλην, μίαν ἀργυρᾶν, ἑβδομήκοντα σίκλων κατὰ τὸν σίκλον τὸν ἅγιον· ἀμφότερα πλήρη σεμιδάλεως ἀναπεποιημένης ἐν ἐλαίῳ εἰς θυσίαν·
\vs{62}Θυΐσκην μίαν δέκα χρυσῶν, πλήρη θυμιάματος.
\vs{63}Μόσχον ἕνα ἐκ βοῶν, κριὸν ἕνα, ἀμνὸν ἕνα ἐνιαύσιον εἰς ὁλοκαύτωμα,
\vs{64}καὶ χίμαρον ἐξ αἰγῶν ἕνα περὶ ἁμαρτίας.
\vs{65}Καὶ εἰς θυσίαν σωτηρίου δαμάλεις δύο, κριοὺς πέντε, τράγους πέντε, ἀμνάδας ἐνιαυσίας πέντε· τοῦτο τὸ δῶρον Ἀβιδὰν υἱοῦ Γαδεωνί.

\vs{66}Τῇ ἡμέρᾳ τῇ δεκάτῃ ἄρχων τῶν υἱῶν Δὰν, Ἀχιέζερ υἱὸς Ἀμισαδαί.
\vs{67}Τὸ δῶρον αὐτοῦ, τρυβλίον ἀργυροῦν ἕν, τριάκοντα καὶ ἑκατὸν ὁλκὴ αὐτοῦ· φιάλην μίαν ἀργυρᾶν, ἑβδομήκοντα σίκλων κατὰ τὸν σίκλον τὸν ἅγιον· ἀμφότερα πλήρη σεμιδάλεως ἀναπεποιημένης ἐν ἐλαίῳ εἰς θυσίαν.
\vs{68}Θυΐσκην μίαν δέκα χρυσῶν, πλήρη θυμιάματος.
\vs{69}Μόσχον ἕνα ἐκ βοῶν, κριὸν ἕνα, ἀμνὸν ἕνα ἐνιαύσιον εἰς ὁλοκαύτωμα,
\vs{70}καὶ χίμαρον ἐξ αἰγῶν ἕνα περὶ ἁμαρτίας.
\vs{71}Καὶ εἰς θυσίαν σωτηρίου δαμάλεις δύο, κριοὺς πέντε, τράγους πέντε, ἀμνάδας ἐνιαυσίας πέντε· τοῦτο τὸ δῶρον Ἀχιέζερ υἱοῦ Ἀμισαδαί.

\vs{72}Τῇ ἡμέρᾳ τῇ ἑνδέκατῃ ἄρχων τῶν υἱῶν Ἀσὴρ, Φαγεὴλ υἱὸς Ἐχράν.
\vs{73}Τὸ δῶρον αὐτοῦ, τρυβλίον ἀργυροῦν ἓν, τριάκοντα καὶ ἑκατὸν ὁλκὴ αὐτοῦ· φιάλην μίαν ἀργυρᾶν, ἑβδομήκοντα σίκλων κατὰ τὸν σίκλον τὸν ἅγιον· ἀμφότερα πλήρη σεμιδάλεως ἀναπεποιημένης ἐν ἐλαίῳ εἰς θυσίαν.
\vs{74}Θυΐσκην μίαν δέκα χρυσῶν, πλήρη θυμιάματος.
\vs{75}Μόσχον ἕνα ἐκ βοῶν, κριὸν ἕνα, ἀμνὸν ἐνιαύσιον ἕνα εἰς ὁλοκαύτωμα,
\vs{76}καὶ χίμαρον ἐξ αἰγῶν ἕνα περὶ ἁμαρτίας.
\vs{77}Καὶ εἰς θυσίαν σωτηρίου δαμάλεις δύο, κριοὺς πέντε, τράγους πέντε, ἀμνάδας ἐνιαυσίας πέντε· τοῦτο τὸ δῶρον Φαγεὴλ υἱοῦ Ἐχράν.

\vs{78}Τῇ ἡμέρᾳ τῇ δωδεκάτῃ ἄρχων τῶν υἱῶν Νεφθαλί, Ἀχιρὲ υἱὸς Αἰνάν.
\vs{79}Τὸ δῶρον αὐτοῦ, τρυβλίον ἀργυροῦν ἓν, τριάκοντα καὶ ἑκατὸν ὁλκὴ αὐτοῦ· φιάλην μίαν ἀργυρᾶν, ἑβδομήκοντα σίκλων κατὰ τὸν σίκλον τὸν ἅγιον· ἀμφότερα πλήρη σεμιδάλεως ἀναπεποιημένης ἐν ἐλαίῳ εἰς θυσίαν.
\vs{80}Θυΐσκην μίαν δέκα χρυσῶν, πλήρη θυμιάματος.
\vs{81}Μόσχον ἕνα ἐκ βοῶν, κριὸν ἕνα, ἀμνὸν ἕνα ἐνιαύσιον εἰς ὁλόκαύτωμα,
\vs{82}καὶ χίμαρον ἐξ αἰγῶν ἕνα περὶ ἁμαρτίας.
\vs{83}Καὶ εἰς θυσίαν σωτηρίου δαμάλεις δύο, κριοὺς πέντε, τράγους πέντε, ἀμνάδας ἐνιαυσίας πέντε· τοῦτο τὸ δῶρον Ἀχιρὲ υἱοῦ Αἰνάν.

\vs{84}Οὗτος ὁ ἐγκαινισμὸς τοῦ θυσιαστηρίου ᾗ ἡμέρᾳ ἔχρισεν αὐτὸ, παρὰ τῶν ἀρχόντων τῶν υἱῶν Ἰσραὴλ· τρυβλία ἀργυρᾶ δώδεκα, φιάλαι ἀργυραῖ δώδεκα, θυΐσκαι χρυσαῖ δώδεκα.
\vs{85}Τριάκοντα καὶ ἑκατὸν σίκλων, τὸ τρυβλίον τὸ ἕν, καὶ ἑβδομήκοντα σίκλων ἡ φιάλη ἡ μία· πᾶν τὸ ἀργύριον τῶν σκευῶν, δισχίλιοι καὶ τετρακόσιοι σίκλοι· σίκλοι, ἐν τῷ σίκλῳ τῷ ἁγίῳ.
\vs{86}Θυΐσκαι χρυσαῖ δώδεκα πλήρεις θυμιάματος· πᾶν τὸ χρυσίον τῶν θυϊσκῶν, εἴκοσι καὶ ἑκατὸν χρυσοῖ.
\vs{87}Πᾶσαι αἱ βόες αἱ εἰς ὁλοκαύτωσιν, μόσχοι δώδεκα, κριοὶ δώδεκα, ἀμνοὶ ἐνιαύσιοι δώδεκα, καὶ αἱ θυσίαι αὐτῶν, καὶ αἱ σπονδαὶ αὐτῶν· καὶ χίμαροι ἐξ αἰγῶν δώδεκα περὶ ἁμαρτίας.
\vs{88}Πᾶσαι αἱ βόες εἰς θυσίαν σωτηρίου, δαμάλεις εἰκοσιτέσσαρες, κριοὶ ἑξήκοντα, τράγοι ἑξήκοντα ἐνιαύσιαι, ἀμνάδες ἑξήκοντα ἐνιαύσιοι ἄμωμοι· αὕτη ἡ ἐγκαίνωσις τοῦ θυσιαστηρίου, μετὰ τὸ πληρῶσαι τὰς χεῖρας αὐτοῦ, καὶ μετὰ τὸ χρίσαι αὐτόν.

\vs{89}Ἐν τῷ εἰσπορεύεσθαι Μωυσῆν εἰς τὴν σκηνὴν τοῦ μαρτυρίου λαλῆσαι αὐτῷ, καὶ ἤκουσε τὴν φωνὴν Κυρίου λαλοῦντος πρὸς αὐτὸν ἄνωθεν τοῦ ἱλαστηρίου, ὅ ἐστιν ἐπὶ τῆς κιβωτοῦ τοῦ μαρτυρίου, ἀναμέσον τῶν δύο χερουβίμ· καὶ ἐλάλει πρὸς αὐτόν.

\ch{8}
Καὶ ἐλάλησε Κύριος πρὸς Μωυσῆν, λέγων, λάλησον τῷ Ἀαρὼν,
\vs{2}καὶ ἐρεῖς πρὸς αὐτὸν, ὅταν ἐπιτιθῇς τοὺς λύχνους ἐκ μέρους, κατὰ πρόσωπον τῆς λυχνίας φωτιοῦσιν οἱ ἑπτὰ λύχνοι.
\vs{3}Καὶ ἐποίησεν οὕτως Ἀαρών· ἐκ τοῦ ἑνὸς μέρους κατὰ πρόσωπον τῆς λυχνίας ἐξῆψε τοὺς λύχνους αὐτῆς, καθὰ συνέταξε Κύριος τῷ Μωυσῇ·
\vs{4}Καὶ αὕτη ἡ κατασκευὴ τῆς λυχνίας· στερεὰ, χρυσῆ, ὁ καυλὸς αὐτῆς, καὶ τὰ κρίνα αὐτῆς, στερεὰ ὅλη· κατὰ τὸ εἶδος ὃ ἔδειξε Κύριος τῷ Μωυσῇ, οὕτως ἐποίησε τὴν λυχνίαν.

\vs{5}Καὶ ἐλάλησε Κύριος πρὸς Μωυσῆν λέγων,
\vs{6}λάβε τοὺς Λευίτας ἐκ μέσου υἱῶν Ἰσραὴλ, καὶ ἀφαγνιεῖς αὐτούς.
\vs{7}Καὶ οὕτω ποιήσεις αὐτοῖς τὸν ἁγνισμὸν αὐτῶν· περιῤῥανεῖς αὐτοὺς ὕδωρ ἁγνισμοῦ· καὶ ἐπελεύσεται ξυρὸν ἐπὶ πᾶν τὸ σῶμα αὐτῶν, καὶ πλυνοῦσι τὰ ἱμάτια αὐτῶν, καὶ καθαροὶ ἔσονται.

\vs{8}Καὶ λήμψονται μόσχον ἕνα ἐκ βοῶν, καὶ τούτου θυσίαν σεμίδαλιν ἀναπεποιημένην ἐν ἐλαίῳ· καὶ μόσχον ἐνιαύσιον ἐκ βοῶν λήψῃ περὶ ἁμαρτίας.
\vs{9}Καὶ προσάξεις τοὺς Λευίτας ἔναντι τῆς σκηνῆς τοῦ μαρτυρίου· καὶ συνάξεις πᾶσαν συναγωγὴν υἱῶν Ἰσραήλ·
\vs{10}Καὶ προσάξεις τοὺς Λευίτας ἔναντι Κυρίου, καὶ ἐπιθήσουσιν οἱ υἱοὶ Ἰσραὴλ τὰς χεῖρας αὐτῶν ἐπὶ τοὺς Λευίτας·
\vs{11}Καὶ ἀφοριεῖ Ἀαρὼν τοὺς Λευίτας ἀπόδομα ἔναντι Κυρίου παρὰ τῶν υἱῶν Ἰσραήλ· καὶ ἔσονται ὥστε ἐργάζεσθαι τὰ ἔργα Κυρίου.
\vs{12}Οἱ δὲ Λευῖται ἐπιθήσονσι τὰς χεῖρας ἐπὶ τὰς κεφαλὰς τῶν μόσχων· καὶ ποιήσεις τὸν ἕνα περὶ ἁμαρτίας, καὶ τὸν ἕνα εἰς ὁλοκαύτωμα Κυρίῳ, ἐξιλάσασθαι περὶ αὐτῶν.

\vs{13}Καὶ στήσεις τοὺς Λευίτας ἔναντι Κυρίου, καὶ ἔναντι Ἀαρὼν, καὶ ἔναντι τῶν υἱῶν αὐτοῦ, καὶ ἀποδώσεις αὐτοὺς ἀπόδομα ἔναντι Κυρίου·
\vs{14}Καὶ διαστελεῖς τοὺς Λευίτας ἐκ μέσου υἱῶν Ἰσραήλ· καὶ ἔσονταί μοι.
\vs{15}Καὶ μετὰ ταῦτα εἰσελεύσονται οἱ Λευῖται ἐργάζεσθαι τὰ ἔργα τῆς σκηνῆς τοῦ μαρτυρίον· καὶ καθαριεῖς αὐτοὺς, καὶ ἀποδώσεις αὐτοὺς ἔναντι Κυρίου.
\vs{16}ὅτι ἀπόδομα ἀποδεδομένοι οὗτοί μοι εἰσὶν ἐκ μέσου υἱῶν Ἰσραήλ· ἀντὶ τῶν διανοιγόντων πᾶσαν μήτραν πρωτοτόκων πάντων ἐκ τῶν υἱῶν Ἰσραὴλ εἴληφα αὐτοὺς ἐμοί.
\vs{17}Ὅτι ἐμοὶ πᾶν πρωτότοκον ἐν υἱοῖς Ἰσραὴλ ἀπὸ ἀνθρώπων ἕως κτήνους· ᾗ ἡμέρᾳ ἐπάταξα πᾶν πρωτότοκον ἐν γῇ Αἰγύπτου, ἡγίασα αὐτοὺς ἐμοὶ,
\vs{18}καὶ ἔλαβον τοὺς Λευίτας ἀντὶ παντὸς πρωτοτόκου ἐν υἱοῖς Ἰσραήλ.
\vs{19}Καὶ ἀπέδωκα τοὺς Λευίτας ἀπόδομα δεδομένους Ἀαρὼν καὶ τοῖς υἱοῖς αὐτοῦ ἐκ μέσου υἱῶν Ἰσραὴλ, ἐργάζεσθαι τὰ ἔργα τῶν υἱῶν Ἰσραὴλ ἐν τῇ σκηνῇ τοῦ μαρτυρίου, καὶ ἐξιλάσκεσθαι περὶ τῶν υἱῶν Ἰσραήλ· καὶ οὐκ ἔσται ἐν τοῖς υἱοῖς Ἰσραὴλ προσεγγίζων πρὸς τὰ ἅγια.

\vs{20}Καὶ ἐποίησε Μωυσῆς καὶ Ἀαρὼν καὶ πᾶσα ἡ συναγωγὴ υἱῶν Ἰσραὴλ τοῖς Λευίταις καθὰ ἐνετείλατο Κύριος τῷ Μωυσῇ περὶ τῶν Λευιτῶν, οὕτως ἐποίησαν αὐτοῖς οἱ υἱοὶ Ἰσραήλ.
\vs{21}Καὶ ἡγνίσαντο οἱ Λευῖται, καὶ ἐπλύναντο τὰ ἱμάτια· καὶ ἀπέδωκεν αὐτοὺς Ἀαρὼν ἀπόδομα ἔναντι Κυρίου, καὶ ἐξιλάσατο περὶ αὐτῶν Ἀαρὼν ἀφαγνίσασθαι αὐτούς.
\vs{22}Καὶ μετὰ ταῦτα εἰσῆλθον οἱ Λευῖται λειτουργεῖν τὴν λειτουργίαν αὐτῶν ἐν τῇ σκηνῇ τοῦ μαρτυρίου ἔναντι Ἀαρὼν, καὶ ἔναντι τῶν υἱῶν αὐτοῦ· καθὰ συνέταξε Κύριος τῷ Μωυσῇ περὶ τῶν Λευιτῶν, οὕτως ἐποίησαν αὐτοῖς.

\vs{23}Καὶ ἐλάλησε Κύριος πρὸς Μωυσῆν, λέγων,
\vs{24}τοῦτό ἐστι τὸ περὶ τῶν Λευιτῶν· ἀπὸ πέντε καὶ εἰκοσαετοῦς καὶ ἐπάνω, εἰσελεύσονται ἐνεργεῖν ἐν τῇ σκηνῇ τοῦ μαρτυρίου·
\vs{25}Καὶ ἀπὸ πεντηκονταετοῦς ἀποστήσεται ἀπὸ τῆς λειτουργίας, καὶ οὐκ ἐργᾶται ἔτι.
\vs{26}Καὶ λειτουργήσει ὁ ἀδελφὸς αὑτοῦ ἐν τῇ σκηνῇ τοῦ μαρτυρίου φυλάσσειν φυλακὰς, ἔργα δὲ οὐκ ἐργᾶται· οὕτως ποιήσεις τοῖς Λευίταις ἐν ταῖς φυλακαῖς αὐτῶν.

\ch{9}
Καὶ ἐλάλησε Κύριος πρὸς Μωυσῆν ἐν τῇ ἐρήμῳ Σινᾷ ἐν τῷ ἔτει τῷ δευτέρῳ, ἐξελθόντων αὐτῶν ἐκ γῆς Αἰγύπτου, ἐν τῷ μηνὶ τῷ πρώτῳ, λέγων,
\vs{2}εἶπον, καὶ ποιείτωσαν οἱ υἱοὶ Ἰσραὴλ τὸ πάσχα καθʼ ὥραν αὐτοῦ,
\vs{3}τῇ τεσσαρεσκαιδεκάτῃ ἡμέρᾳ τοῦ μηνὸς τοῦ πρώτου πρὸς ἑσπέραν, ποιήσεις αὐτὸ κατὰ καιρούς· κατὰ τὸν νόμον αὐτοῦ, καὶ κατὰ τὴν σύγκρισιν αὐτοῦ ποιήσεις αὐτό.
\vs{4}Καὶ ἐλάλησε Μωυσῆς τοῖς υἱοῖς Ἰσραὴλ ποιῆσαι τὸ πάσχα ἐναρχομένου τῇ τεσσαρεσκαιδεκάτῃ ἡμέρᾳ τοῦ μηνὸς ἐν τῇ ἐρήμῳ τοῦ Σινᾶ.
\vs{5}καθὰ συνέταξε Κύριος τῷ Μωυσῇ, οὕτως ἐποίησαν οἱ υἱοὶ Ἰσραήλ.

\vs{6}Καὶ παρεγένοντο οἱ ἄνδρες οἳ ἦσαν ἀκάθαρτοι ἐπὶ ψυχῇ ἀνθρώπου, καὶ οὐκ ἠδύναντο ποιῆσαι τὸ πάσχα ἐν τῇ ἡμέρᾳ ἐκείνῃ· καὶ προσῆλθον ἐναντίον Μωυσῆ καὶ Ἀαρὼν ἐν ἐκείνῃ τῇ ἡμέρᾳ·
\vs{7}Καὶ εἶπαν οἱ ἄνδρες ἐκεῖνοι πρὸς αὐτὸν, ἡμεῖς ἀκάθαρτοι ἐπὶ ψυχῇ ἀνθρώπου· μὴ οὖν ὑστερήσωμεν προσενέγκαι τὸ δῶρον Κυρίῳ κατὰ καιρὸν αὐτοῦ ἐν μέσῳ υἱῶν Ἰσραήλ;
\vs{8}Καὶ εἶπε πρὸς αὐτοὺς Μωυσῆς, στῆτε αὐτοῦ, καὶ ἀκούσομαι τί ἐντελεῖται Κύριος περὶ ὑμῶν.
\vs{9}Καὶ ἐλάλησε Κύριος πρὸς Μωυσῆν, λέγων,
\vs{10}λάλησον τοῖς υἱοῖς Ἰσραὴλ, λέγων, ἄνθρωπος ἄνθρωπος, ὃν ἐὰν γένηται ἀκάθαρτος ἐπὶ ψυχῇ, ἀνθρώπου, ἢ ἐν ὁδῷ μακρὰν ὑμῖν, ἢ ἐν ταῖς γενεαῖς ὑμῶν, καὶ ποιήσει τὸ πάσχα Κυρίῳ ἐν τῷ μηνὶ τῷ δευτέρῳ ἐν τῇ τεσσαρεσκαιδεκάτῃ ἡμέρᾳ·
\vs{11}τὸ πρὸς ἑσπέραν ποιήσουσιν αὐτὸ, ἐπʼ ἀζύμων καὶ πικρίδων φάγονται αὐτό.
\vs{12}Οὐ καταλείψουσιν ἀπʼ αὐτοῦ εἰς τὸ πρωῒ, καὶ ὀστοῦν οὐ συντρίψουσιν ἀπʼ αὐτοῦ· κατὰ τὸν νόμον τοῦ πάσχα ποιήσουσιν αὐτό.
\vs{13}Καὶ ἄνθρωπος ὃς ἐὰν καθαρὸς ᾖ, καὶ ἐν ὁδῷ μακρὰν οὐκ ἔστι, καὶ ὑστερήσῃ ποιῆσαι τὸ πάσχα, ἐξολοθρευθήσεται ἡ ψυχὴ ἐκείνη ἐκ τοῦ λαοῦ αὐτῆς, ὅτι τὸ δῶρον Κυρίῳ οὐ προσήνεγκε κατὰ τὸν καιρὸν αὐτοῦ· ἁμαρτίαν αὐτοῦ λήψεται ὁ ἄνθρωπος ἐκεῖνος.
\vs{14}Ἐὰν δὲ προσέλθῃ πρὸς ὑμᾶς προσήλυτος ἐν τῇ γῇ ὑμῶν, καὶ ποιήσῃ τὸ πάσχα Κυρίῳ, κατὰ τὸν νόμον τοῦ πάσχα, καὶ κατὰ τὴν σύνταξιν αὐτοῦ ποιήσει αὐτό· νόμος εἷς ἔσται ὑμῖν καὶ τῷ προσηλύτῳ καὶ τῷ αὐτόχθονι τῆς γῆς.

\vs{15}Καὶ τῇ ἡμέρᾳ ᾗ ἐστάθη ἡ σκηνὴ, ἐκάλυψεν ἡ νεφελη τὴν σκηνὴν, τὸν οἶκον τοῦ μαρτυρίου· καὶ τὸ ἑσπέρας ἦν ἐπὶ τῆς σκηνῆς ὡς εἶδος πυρὸς ἕως πρωΐ.
\vs{16}Οὕτως ἐγίνετο διαπαντός· ἡ νεφέλη ἐκάλυπτεν αὐτὴν ἡμέρας, καὶ εἶδος πυρὸς τὴν νύκτα.
\vs{17}Καὶ ἡνίκα ἀνέβη ἡ νεφέλη ἀπὸ τῆς σκηνῆς, καὶ μετὰ ταῦτα ἀπῇραν οἱ υἱοὶ Ἰσραήλ· καὶ ἐν τῷ τόπῳ οὗ ἂν ἔστη ἡ νεφέλη, ἐκεῖ παρενέβαλον οἱ υἱοὶ Ἰσραήλ.
\vs{18}Διὰ προστάγματος Κυρίου παρεμβαλοῦσιν οἱ υἱοὶ Ἰσραὴλ, καὶ διὰ προστάγματος Κυρίου ἀπαροῦσι· πάσας τὰς ἡμέρας ἐν αἷς σκιάζει ἡ νεφέλη ἐπὶ τῆς σκηνῆς, παρεμβαλοῦσιν οἱ υἱοὶ Ἰσραήλ.
\vs{19}Καὶ ὅταν ἐφέλκηται ἡ νεφέλη ἐπὶ τῆς σκηνῆς ἡμέρας πλείους, καὶ φυλάξονται οἱ υἱοὶ Ἰσραὴλ τὴν φυλακὴν τοῦ Θεοῦ, καὶ οὐ μὴ ἐξάρωσι.
\vs{20}Καὶ ἔσται ὅταν σκεπάζῃ ἡ νεφέλη ἡμέρας ἀριθμῷ ἐπὶ τῆς σκηνῆς, διὰ φωνῆς Κυρίου παρεμβαλοῦσι, καὶ διὰ προστάγματος Κυρίου ἀπαροῦσι.
\vs{21}Καὶ ἔσται ὅταν γένηται ἡ νεφέλη ἀφʼ ἑσπέρας ἕως πρωῒ, καὶ ἀναβῇ ἡ νεφέλη τοπρωῒ, καὶ ἀπαροῦσιν ἡμέρας ἢ νυκτός.
\vs{22}Μηνὸς ἡμέρας πλεοναζούσης τῆς νεφέλης σκιαζούσης ἐπʼ αὐτῆς, παρεμβαλοῦσιν οἱ υἱοὶ Ἰσραὴλ, καὶ οὐ μὴ ἀπάρωσιν.
\vs{23}Ὅτι διὰ προστάγματος Κυρίου ἀπαροῦσι· τὴν φυλακὴν Κυρίου ἐφυλάξαντο διὰ προστάγματος Κυρίου ἐν χειρὶ Μωυσῆ.

\ch{10}
Καὶ ἐλάλησε Κύριος πρὸς Μωυσῆν, λέγων, ποίησον σεαυτῷ δύο σάλπιγγας ἀγυρᾶς·
\vs{2}ἐλατὰς ποιήσεις αὐτάς· καὶ ἔσονταί σοι ἀνακαλεῖν τὴν συναγωγὴν, καὶ ἐξαίρειν τὰς παρεμβολάς.
\vs{3}Καὶ σαλπιεῖς ἐν αὐταῖς, καὶ συναχθήσεται πᾶσα ἡ συναγωγὴ ἐπὶ τὴν θύραν τῆς σκηνῆς τοῦ μαρτυρίου·
\vs{4}Ἐὰν δὲ ἐν μιᾷ σαλπίσωσι, προσελεύσονται πρὸς σὲ πάντες οἱ ἄρχοντες ἀρχηγοὶ Ἰσραήλ.
\vs{5}καὶ σαλπιεῖτε σημασίαν, καὶ ἐξαροῦσιν αἱ παρεμβάλλουσαι αἱ παρεμβαλοῦσαι ἀνατολάς·
\vs{6}καὶ σαλπιεῖτε σημασίαν δευτέραν, καὶ ἐξαροῦσιν αἱ παρεμβολαὶ αἱ παρεμβάλλουσαι Λίβα· καὶ σαλπιεῖτε σημασίαν τρίτην, καὶ ἐξαροῦσιν αἱ παρεμβολαὶ αἱ παρεμβάλλουσαι παρὰ θάλασσαν· καὶ σαλπιεῖτε σημασίαν τετάρτην, καὶ ἐξαροῦσιν αἱ παρεμβολαὶ αἱ παρεμβάλλουσαι πρὸς Βοῤῥᾶν· σημασίᾳ σαλπιοῦσιν ἐν τῇ ἐξάρσει αὐτῶν.
\vs{7}καὶ ὅταν συναγάγητε τὴν συναγωγὴν, σαλπιεῖτε, καὶ οὐ σημασίᾳ.
\vs{8}Καὶ οἱ υἱοὶ Ἀαρὼν οἱ ἱερεῖς σαλπιοῦσι ταῖς σάλπιγξι· καὶ ἔσται ὑμῖν νόμιμον αἰώνιον εἰς τὰς γενεὰς ὑμῶν.
\vs{9}Ἐὰν δὲ ἐξέλθητε εἰς πόλεμον ἐν τῇ γῇ ὑμῶν πρὸς τοὺς ὑπεναντίους τοὺς ἀνθεστηκότας ὑμῖ, καὶ σημανεῖτε ταῖς σάλπιγξιν, καὶ ἀναμνησθήσεσθε ἔναντι Κυρίου, καὶ διασωθήσεσθε ἀπὸ τῶν ἐχθρῶν ὑμῶν.
\vs{10}Καὶ ἐν ταῖς ἡμέραις τῆς εὐφροσύνης ὑμῶν, καὶ ἐν ταῖς ἑορταῖς ὑμῶν, καὶ ἐν ταῖς νουμηνίαις ὑμῶν, σαλπιεῖτε ταῖς σάλπιγξιν ἐπὶ τοῖς ὁλοκαυτώμασι, καὶ ἐπὶ ταῖς θυσίαις τῶν σωτηρίων ὑμῶν· καὶ ἔσται ὑμῖν ἀνάμνησις ἔναντι τοῦ Θεοῦ ὑμῶν· ἐγὼ Κύριος ὁ Θεὸς ὑμῶν.

\vs{11}Καὶ ἐγένετο ἐν τῷ ἐνιαυτῷ τῷ δευτέρῳ ἐν τῷ μηνὶ τῷ δευτέρῳ εἰκάδι τοῦ μηνὸς, ἀνέβη ἡ νεφέλη ἀπὸ τῆς σκηνῆς τοῦ μαρτυρίου.
\vs{12}καὶ ἐξῇραν οἱ υἱοὶ Ἰσραὴλ σὺν ἀπαρτίαις αὐτῶν ἐν τῇ ἐρήμῳ Σινᾶ· καὶ ἔστη ἡ νεφέλη ἐν τῇ ἐρήμῳ τοῦ Φαράν.
\vs{13}Καὶ ἐξῆραν πρῶτοι διὰ φωνῆς Κυρίου ἐν χειρὶ Μωυσῆ.

\vs{14}Καὶ ἐξῇραν τάγμα παρεμβολῆς υἱῶν Ἰούδα πρῶτοι σὺν δυνάμει αὐτῶν· καὶ ἐπὶ τῆς δυνάμεως αὐτῶν, Ναασσὼν υἱὸς Ἀμιναδάβ.
\vs{15}καὶ ἐπὶ τῆς δυνάμεως φυλῆς υἱῶν Ἰσσάχαρ, Ναθαναὴλ υἱὸς Σωγάρ.
\vs{16}καὶ ἐπὶ τῆς δυνάμεως φυλῆς υἱῶν Ζαβουλὼν, Ἐλιὰβ, υἱὸς Χαιλών.
\vs{17}Καὶ καθελοῦσι τὴν σκηνὴν, καὶ ἐξαροῦσιν οἱ υἱοὶ Γεδσὼν, καὶ οἱ υἱοὶ Μεραρὶ, οἱ αἴροντες τὴν σκηνήν.

\vs{18}Καὶ ἐξῇραν τάγμα παρεμβολῆς Ῥουβὴν σὺν δυνάμει αὐτῶν· καὶ ἐπὶ τῆς δυνάμεως αὐτῶν, Ἐλισοὺρ υἱὸς Σεδιούρ·
\vs{19}Καὶ ἐπὶ τῆς δυνάμεως φυλῆς υἱῶν Συμεὼν, Σαλαμιὴλ υἱὸς Σουρισαδαΐ.
\vs{20}Καὶ ἐπὶ τῆς δυνάμεως φυλῆς υἱῶν Γὰδ, Ἐλισὰφ ὁ τοῦ Ῥαγουήλ.
\vs{21}Καὶ ἐξαροῦσιν οἱ υἱοὶ Καὰθ αἴροντες τὰ ἅγια· καὶ στήσουσι τὴν σκηνὴν ἕως παραγένωνται.
\vs{22}Καὶ ἐξαροῦσι τάγμα παρεμβολῆς Ἐφραὶμ σὺν δυνάμει αὐτῶν· καὶ ἐπὶ τῆς δυνάμεως αὐτῶν, Ἐλισαμὰ υἱὸς Σμιούδ.

\vs{23}Καὶ ἐπὶ τῆς δυνάμεως φυλῆς υἱῶν Μανασσῆ, Γαμαλιὴλ ὁ τοῦ Φαδασσούρ.
\vs{24}Καὶ ἐπὶ τῆς δυνάμεως φυλῆς υἱῶν Βενιαμὶν, Ἀβιδὰν ὁ τοῦ Γαδεωνί.
\vs{25}Καὶ ἐξαροῦσι τάγμα παρεμβολῆς υἱῶν Δὰν, ἔσχατοι πασῶν τῶν παρεμβολῶν, σὺν δυνάμει αὐτῶν· καὶ ἐπὶ τῆς δυνάμεως αὐτῶν, Ἀχιέζερ ὁ τοῦ Ἀμισαδαΐ.
\vs{26}Καὶ ἐπὶ τῆς δυνάμεως φυλῆς υἱῶν Ἀσὴρ, Φαγεὴλ υἱὸς Ἐχράν.
\vs{27}Καὶ ἐπὶ τῆς δυνάμεως φυλῆς υἱῶν Νεφθαλὶ, Ἀχιρὲ υἱὸς Αἰνάν.
\vs{28}Αὗται αἱ στρατιαὶ υἱῶν Ἰσραήλ· καὶ ἐξῇραν σὺν δυνάμει αὐτῶν.

\vs{29}Καὶ εἶπε Μωυσῆς τῷ Ὀβὰβ υἱῷ Ῥαγουὴλ τῷ Μαδιανίτῃ τῷ γαμβρῷ Μωυσῆ, ἐξαίρομεν ἡμεῖς εἰς τὸν τόπον ὃν εἶπε Κύριος, τοῦτον δώσω ὑμῖν· δεῦρο μεθʼ ἡμῶν, καὶ εὖ σε ποιήσομεν, ὅτι Κύριος ἐλάλησε καλὰ περὶ Ἰσραήλ.
\vs{30}Καὶ εἶπε πρὸς αὐτόν, οὐ πορεύσομαι, ἀλλὰ εἰς τὴν γῆν μου, καὶ εἰς τὴν γενεάν μου.
\vs{31}Καὶ εἶπε, μὴ ἐγκαταλίπῃς ἡμᾶς, οὗ ἕνεκεν ἦσθα μεθʼ ἡμῶν ἐν τῇ ἐρήμῳ, καὶ ἔσῃ ἐν ἡμῖν πρεσβύτης.
\vs{32}Καὶ ἔσται ἐὰν πορευθῇς μεθʼ ἡμῶν, καὶ ἔσται τὰ ἀγαθὰ ἐκεῖνα ὅσα ἂν ἀγαθοποιήσῃ Κύριος ἡμᾶς, καὶ εὖ σε ποιήσομεν.

\vs{33}Καὶ ἐξῇραν ἐκ τοῦ ὄρους Κυρίου ὁδὸν τριῶν ἡμερῶν· καὶ ἡ κιβωτὸς τῆς διαθήκης Κυρίου προεπορεύετο προτέρα αὐτῶν ὁδὸν τριῶν ἡμερῶν κατασκέψασθαι αὐτοῖς ἀνάπαυσιν.
\vs{34}Καὶ ἐγένετο ἐν τῷ ἐξαίρειν τὴν κιβωτὸν, καὶ εἶπε Μωυσῆς, ἐξεγέρθητι Κύριε, καὶ διασκορπισθήτωσαν οἱ ἐχθροί σου, φυγέτωσαν πάντες οἱ μισοῦντές σε.
\vs{35}Καὶ ἐν τῇ καταπαύσει εἶπεν, ἐπίστρεφε Κύριε χιλιάδας μυριάδας ἐν τῷ Ἰσραήλ.
\vs{36}Καὶ ἡ νεφέλη ἐγένετο σκιάζουσα ἐπʼ αὐτοῖς ἡμέρας, ἐν τῷ ἐξαίρειν αὐτοὺς ἐκ τῆς παρεμβολῆς.

\ch{11}
Καὶ ἦν ὁ λαὸς γογγύζων πονηρὰ ἔναντι Κυρίου· καὶ ἤκουσε Κύριος, καὶ ἐθυμώθη ὀργῇ· καὶ ἐξεκαύθη ἐν αὐτοῖς πῦρ παρὰ Κυρίου, καὶ κατέφαγε μέρος τι τῆς παρεμβολῆς.
\vs{2}Καὶ ἐκέκραξεν ὁ λαὸς πρὸς Μωυσῆν· καὶ ηὔξατο Μωυσῆς πρὸς Κύριον, καὶ ἐκόπασε τὸ πῦρ.
\vs{3}Καὶ ἐκλήθη τὸ ὄνομα τοῦ τόπου ἐκείνου, Ἐμπυρισμός· ὅτι ἐξεκαύθη ἐν αὐτοῖς παρὰ Κυρίου.
\vs{4}Καὶ ὁ ἐπίμικτος ὁ ἐν αὐτοῖς ἐπεθύμησεν ἐπιθυμίαν· καὶ καθίσαντες ἔκλαιον καὶ οἱ υἱοὶ Ἰσραὴλ, καὶ εἶπαν, τίς ἡμᾶς ψωμιεῖ κρέα;
\vs{5}Ἐμνήσθημεν τοὺς ἰχθύας, οὓς ἠσθίομεν ἐν Αἰγύπτῳ δωρεὰν. καὶ τοὺς σικύας, καὶ τοὺς πέπονας, καὶ τὰ πράσα, καὶ τὰ κρόμμυα, καὶ τὰ σκόρδα.
\vs{6}Νυνὶ δὲ ἡ ψυχὴ ἡμῶν κατάξηρος· οὐδὲν πλὴν εἰς τὸ μάννα οἱ ὀφθαλμοὶ ἡμῶν.
\vs{7}Τὸ δὲ μάννα ὡσεὶ σπέρμα κορίου ἐστὶ, καὶ τὸ εἶδος αὐτοῦ εἶδος κρυστάλλου.
\vs{8}Καὶ διεπορεύετο ὁ λαὸς, καὶ συνέλεγον, καὶ ἤληθον αὐτὸ ἐν τῷ μύλῳ, καὶ ἔτριβον ἐν τῇ θυΐᾳ, καὶ ἥψουν αὐτὸ ἐν τῇ χύτρᾳ, καὶ ἐποίουν αὐτὸ ἐνκρυφίας· καὶ ἦν ἡ ἡδονὴ αὐτοῦ ὡσεὶ γεῦμα ἐγκρὶς ἐξ ἐλαίου.
\vs{9}Καὶ ὅταν κατέβη ἡ δρόσος ἐπὶ τὴν παρεμβολὴν νυκτὸς, κατέβαινε τὸ μάννα ἐπʼ αὐτῆς.

\vs{10}Καὶ ἤκουσε Μωυσῆς κλαίοντων αὐτῶν κατά δήμους αὐτῶν, ἕκαστον ἐπὶ τῆς θύρας αὐτοῦ· καὶ ἐθυμώθη ὀργῇ Κύριος σφόδρα· καὶ ἔναντι Μωυσῆ ἦν πονηρόν.
\vs{11}Καὶ εἶπε Μωυσῆς πρὸς Κύριον, ἱνατί ἐκάκωσας τὸν θεράποντά σου, καὶ διατί οὐχ εὕρηκα χάριν ἐναντίον σου, ἐπιθεῖναι τὴν ὁρμὴν τοῦ λαοῦ τούτου ἐπʼ ἐμέ;
\vs{12}Μὴ ἐγὼ ἐν γαστρὶ ἔλαβον πάντα τὸν λαὸν τοῦτον, ἢ ἐγὼ ἔτεκον αὐτούς; ὅτι λέγεις μοι, λάβε αὐτὸν εἰς τὸν κόλπον σου, ὡσεὶ ἄραι τιθηνὸς τὸν θηλάζοντα, εἰς τὴν γῆν ἣν ὤμοσας τοῖς πατράσιν αὐτῶν;
\vs{13}Πόθεν μοι κρέα, δοῦναι παντὶ τῷ λαῷ τούτῳ; ὅτι κλαίουσιν ἐπʼ ἐμοὶ, λέγοντες, δὸς ἡμῖν κρέα, ἵνα φάγωμεν.
\vs{14}Οὐ δυνήσομαι ἐγὼ μόνος φέρειν τὸν λαὸν τοῦτον, ὅτι βαρύτερόν μοι ἐστὶ τὸ ῥῆμα τοῦτο.
\vs{15}Εἰ δʼ οὕτω σὺ ποιεῖς μοι, ἀπόκτεινόν με ἀναιρέσει, εἰ εὕρηκα ἔλεος παρὰ σοὶ, ἵνα μὴ ἴδω τὴν κάκωσίν μου.

\vs{16}Καὶ εἶπε Κύριος πρὸς Μωυσῆς, συνάγαγέ μοι ἑβδομήκοντα ἄνδρας ἀπὸ τῶν πρεσβυτέρων Ἰσραήλ, οὓς αὐτὸς σὺ οἶδας, ὅτι οὗτοί εἰσι πρεσβύτεροι τοῦ λαοῦ καὶ γραμματεῖς αὐτῶς· καὶ ἄξεις αὐτοὺς πρὸς τὴν σκηνὴν τοῦ μαρτυρίου, καὶ στήσονται ἐκεῖ μετὰ σοῦ.
\vs{17}Καὶ καταβήσομαι, καὶ λαλήσω ἐκεῖ μετὰ σοῦ· καὶ ἀφελῶ ἀπὸ τοῦ πνεύματος τοῦ ἐπὶ σοὶ, καὶ ἐπιθήσω ἐπʼ αὐτούς· καὶ συναντιλήψονται μετὰ σοῦ τὴν ὁρμὴν τοῦ λαοῦ, καὶ οὐκ οἴσεις αὐτὸς σὺ μόνος.
\vs{18}Καὶ τῷ λαῷ ἐρεῖς, ἁγνίσασθε εἰς αὔριον, καὶ φάγεσθε κρέα· ὅτι ἐκλαύσατε ἔναντι Κυρίου, λέγοντες, τίς ἡμᾶς ψωμιεῖ κρέα; ὅτι καλὸν ἡμῖν ἐστιν ἐν Αἰγυπτῳ· καὶ δώσει Κύριος ὑμῖν φαγεῖν κρέα, καὶ φάγεσθε κρέα.
\vs{19}Οὐχ ἡμέραν μίαν φάγεσθε, οὐ δὲ δύο, οὐ δὲ πέντε ἡμέρας, οὐ δὲ δέκα ἡμέρας, οὐ δὲ εἴκοσι ἡμέρας,
\vs{20}ἕως μηνὸς ἡμερῶν φάγεσθε, ἕως ἂν ἐξέλθῃ ἐκ τῶν μυκτήρων ὑμῶν· καὶ ἔσται ὑμῖν εἰς χολέραν, ὅτι ἠπειθήσατε Κυρίῳ, ὅς ἐστιν ἐν ὑμῖν, καὶ ἐκλαύσατε ἐναντίον αὐτοῦ, λέγοντες, ἱνατί ἡμῖν ἐξελθεῖν ἐξ Αἰγύπτου;
\vs{21}Καὶ εἶπε Μωυσῆς, ἑξακόσιαι χιλιάδες πεζῶν ὁ λαὸς, ἐν οἷς εἰμι ἐν αὐτοῖς· καὶ σὺ εἶπας, κρέα δώσω αὐτοῖς φαγεῖν, καὶ φάγονται μῆνα ἡμερῶν·
\vs{22}Μὴ πρόβατα καὶ βόες σφαγήσονται αὐτοῖς, καὶ ἀρκέσει αὐτοῖς; ἢ πᾶν τὸ ὄψος τῆς θαλάσσης συναχθήσεται αὐτοῖς, καὶ ἀρκέσει αὐτοῖς;
\vs{23}Καὶ εἶπε Κύριος πρὸς Μωυσῆν, μὴ χεὶρ Κυρίου οὐκ ἐξαρκέσει; ἤδη γνώσῃ εἰ ἐπικαταλήμψεταί σε ὁ λόγος μου ἢ οὔ.

\vs{24}Καὶ ἐξῆλθε Μωυσῆς, καὶ ἐλάλησε πρὸς τὸν λαὸν τὰ ῥήματα Κυρίου· καὶ συνήγαγεν ἑβδομήκοντα ἄνδρας ἀπὸ τῶν πρεσβυτέρων τοῦ λαοῦ, καὶ ἔστησεν αὐτὸς κύκλῳ τῆς σκηνῆς.
\vs{25}Καὶ κατέβη Κύριος ἐν νεφέλῃ, καὶ ἐλάλησε πρὸς αὐτόν· καὶ παρείλατο ἀπὸ τοῦ πνεύματος τοῦ ἐπʼ αὐτῷ, καὶ ἐπέθηκεν ἐπὶ τοὺς ἑβδομήκοντα ἄνδρας τοὺς πρεσβυτέρους· ὡς δὲ ἐπανεπαύσατο πνεῦμα ἐπʼ αὐτοὺς, καὶ ἐπροφήτευσαν, καὶ οὐκ ἔτι προσέθεντο.
\vs{26}Καὶ κατελείφθησαν δύο ἄνδρες ἐν τῇ παρεμβολῇ, ὄνομα τῷ ἑνὶ Ἑλδὰδ, καὶ ὄνομα τῷ δευτέρῳ Μωδάδ· καὶ ἐπανεπαύσατο ἐπʼ αὐτοὺς πνεῦμα· καὶ οὗτοι ἦσαν τῶν καταγεγραμμένων, καὶ οὐκ ἦλθον πρὸς τὴν σκηνήν· καὶ ἐπροφήτευσαν ἐν τῇ παρεμβολῇ.
\vs{27}Καὶ προσδραμὼν ὁ νεανίσκος, ἀπήγγειλε Μωυσῇ· καὶ εἶπε, λέγων, Ἑλδὰδ καὶ Μωδὰδ προφητεύουσιν ἐν τῇ παρεμβολῇ.
\vs{28}Καὶ ἀποκριθεὶς Ἰησοῦς ὁ τοῦ Ναυὴ, ὁ παρεστηκὼς Μωυσῇ, ὁ ἐκλεκτὸς, εἶπε, κύριε Μωυσῆ, κώλυσον αὐτούς.
\vs{29}Καὶ εἶπε Μωυσῆς αὐτῷ, μὴ ζηλοῖς ἐμέ; καὶ τίς δῴη πάντα τὸν λαὸν Κυρίου προφήτας, ὅταν δῷ Κύριος τὸ πνεῦμα αὐτοῦ ἐπʼ αὐτούς;
\vs{30}Καὶ ἀπῆλθε Μωυσῆς εἰς τὴν παρεμβολὴν αὐτὸς καὶ οἱ πρεσβύτεροι Ἰσραήλ.

\vs{31}Καὶ πνεῦμα ἐξῆλθε παρὰ Κυρίου, καὶ ἐξεπέρασεν ὀρτυγομήτραν ἀπὸ τῆς θαλάσσης· καὶ ἐπέβαλεν ἐπὶ τὴν παρεμβολὴν ὁδὸν ἡμέρας ἐντεῦθεν, καὶ ὁδὸν ἡμέρας ἐντεῦθεν, κύκλῳ τῆς παρεμβολῆς, ὡσεὶ δίπηχυ ἀπὸ τῆς γῆς.
\vs{32}Καὶ ἀναστὰς ὁ λαὸς ὅλην τὴν ἡμέραν, καὶ ὅλην τὴν νύκτα, καὶ ὅλην τὴν ἡμέραν τὴν ἐπαύριον, καὶ συνήγαγον τὴν ὀρτυγομήτραν· ὁ τὸ ὀλίγον, συνήγαγε δέκα κόρους· καὶ ἔψυξαν ἑαυτοῖς ψυγμοὺς κύκλῳ τῆς παρεμβολῆς.
\vs{33}Τὰ κρέα ἔτι ἦν ἐν τοῖς ὀδοῦσιν αὐτῶν πρινὴ ἐκλείπειν, καὶ Κύριος ἐθυμώθη εἰς τὸν λαὸν, καὶ ἐπάταξε Κύριος τὸν λαὸν πληγὴν μεγάλην σφόδρα.
\vs{34}Καὶ ἐκλήθη τὸ ὄνομα τοῦ τόπου ἐκείνου, Μνήματα τῆς ἐπιθυμίας· ὅτι ἐκεῖ ἔθαψαν τὸν λαὸν τὸν ἐπιθυμητήν.
\vs{35}Ἀπὸ Μνημάτων ἐπιθυμίας ἐξῇρεν ὁ λαὸς εἰς Ἀσηρώθ· καὶ ἐγένετο ὁ λαὸς ἐν Ἀσηρώθ.

\ch{12}
Καὶ ἐλάλησε Μαριὰμ καὶ Ἀαρὼν κατὰ Μωυσῆ, ἕνεκεν τῆς γυναικὸς τῆς Αἰθιοπίσσης ἣν ἔλαβε Μωυσῆς, ὅτι γυναῖκα Αἰθιόπισσαν ἔλαβε,
\vs{2}καὶ εἶπαν, μὴ Μωυσῇ μόνῳ λελάληκε Κύριος; οὐχὶ καὶ ἡμῖν ἐλάλησε; καὶ ἤκουσε Κύριος.
\vs{3}Καὶ ὁ ἄνθρωπος Μωυσῆς πραῢς σφόδρα παρὰ πάντας τοὺς ἀνθρώπους τοὺς ὄντας ἐπὶ τῆς γῆς.
\vs{4}Καὶ εἶπε Κύριος παραχρῆμα πρὸς Μωυσῆν καὶ Ἀαρὼν καὶ Μαριὰμ, ἐξέλθετε ὑμεῖς οἱ τρεῖς εἰς τὴν σκηνὴν τοῦ μαρτυρίου.
\vs{5}Καὶ ἐξῆλθον οἱ τρεῖν εἰν τὴν σκηνὴν τοῦ μαρτυρίου· καὶ κατέβη Κύριος ἐν στύλῳ νεφέλης, καὶ ἔστη ἐπὶ τῆς θύρας τῆς σκηνῆς τοῦ μαρτυρίου· καὶ ἐκλήθησαν Ἀαρὼν καὶ Μαριάμ· καὶ ἐξήλθοσαν ἀμφότεροι.
\vs{6}Καὶ εἶπε πρὸς αὐτοὺς, ἀκούσατε τῶν λόγων μου· ἐὰν γένηται προφήτης ὑμῶν Κυρίῳ, ἐν ὁράματι αὐτῷ γυωσθήσομαι, καὶ ἐν ὕπνῳ λαλήσω αὐτῷ.
\vs{7}Οὐχ οὕτως ὁ θεράπων μου Μωυσῆς, ἐν ὅλῳ τῷ οἴκῳ μου πιστός ἐστι·
\vs{8}Στόμα κατὰ στόμα λαλήσω αὐτῶ, ἐν εἴδει, καὶ οὐ διʼ αἰνιγμάτων, καὶ τὴν δόξαν Κυρίου εἶδε· καὶ διατί οὐκ ἐφοβήθητε καταλαλῆσαι κατὰ τοῦ θεράποντός μου Μωυσῆ;
\vs{9}Καὶ ὀργὴ θυμοῦ Κυρίου ἐπʼ αὐτοῖς, καὶ ἀπῆλθε.
\vs{10}Καὶ ἡ νεφέλη ἀπέστη ἀπὸ τῆς σκηνῆς· καὶ ἰδοὺ Μαριὰμ λεπρῶσα ὡσεὶ χιών· καὶ ἐπέβλεψεν Ἀαρὼν ἐπὶ Μαριὰμ, καὶ ἰδοὺ λεπρῶσα.
\vs{11}Καὶ εἶπεν Ἀαρὼν πρὸς Μωυσῆν, δέομαι κύριε, μὴ συνεπιθῇ ἡμῖν ἁμαρτίαν, διότι ἠγνοήσαμεν καθʼ ὅτι ἡμάρτομεν·
\vs{12}Μὴ γένηται ὡσεὶ ἶσον θανάτῳ, ὡσεὶ ἔκτρωμα ἐκπορευόμενον ἐκ μήτρας μητρὸς, καὶ κατεσθίει τὸ ἥμισυ τῶν σαρκῶν αὐτῆς.
\vs{13}Καὶ ἐβόησε Μωυσῆς πρὸς Κύριον, λέγων, ὁ Θεὸς δέομαί σου, ἴασαι αὐτήν.
\vs{14}Καὶ εἶπε Κύριος πρὸς Μωυσῆν, εἰ ὁ πατὴρ αὐτῆς πτύων ἐνέπτυσεν εἰς τὸ πρόσωπον αὐτῆς οὐκ ἐντραπήσεται ἑπτὰ ἡμέρας; ἀφορισθήτω ἑπτὰ ἡμέρας ἔξω τῆς παρεμβολῆς, καὶ μετὰ ταῦτα εἰσελεύσεται.

\vs{15}Καὶ ἀφωρίσθη Μαριὰμ ἔξω τῆς παρεμβολῆς ἑπτὰ ἡμέρας· καὶ ὁ λαὸς οὐκ ἐξῇρεν, ἕως ἐκαθαρίσθη Μαριάμ.

\vs{16}Καὶ μετὰ ταῦτα ἐξῇρεν ὁ λαὸς ἐξ Ἀσηρὼθ, καὶ παρενέβαλον ἐν τῇ ἐρήμῳ τοῦ Φαράν.

\ch{13}Καὶ ἐλάλησε Κύριος πρὸς Μωυσῆν, λέγων, ἀπόστειλον σεαυτῷ ἄνδρας,
\vs{2}καὶ κατασκεψάσθωσαν τὴν γῆν τῶν Χαναναίων, ἣν ἐγὼ δίδωμι τοῖς υἱοῖς Ἰσραὴλ εἰς κατάσχεσιν· ἄνδρα ἕνα κατὰ φυλήν, κατὰ δήμους πατριῶν αὐτῶν ἀποστελεῖς αὐτοὺς, πάντα ἀρχηγὸν ἐξ αὐτῶν.

\vs{3}Καὶ ἐξαπέστειλεν αὐτοὺς Μωυσῆς ἐκ τῆς ἐρήμου Φαρὰν διὰ φωνῆς Κυρίου· πάντες ἄνδρες ἀρχηγοὶ υἱῶν Ἰσραὴλ οὗτοι.
\vs{4}Καὶ ταῦτα τὰ ὀνόματα αὐτῶν· τῆς φυλῆς Ῥουβὴν, Σαμουὴλ υἱὸς Ζαχούρ.
\vs{5}Τῆς φυλῆς Συμεὼν, Σαφὰτ υἱὸς Σουρί.
\vs{6}Τῆς φυλῆς Ἰούδα, Χάλεβ υἱὸς Ἰεφοννή.
\vs{7}Τῆς φυλῆς Ἰσσάχαρ, Ἰλαὰλ υἱὸς Ἰωσήφ.
\vs{8}Τῆς φυλῆς Ἐφραὶμ, Αὐσὴ υἱὸς Ναυή.
\vs{9}Τῆς φυλῆς Βενιαμὶν, Φαλτὶ υἱὸς Ῥαφοῦ.
\vs{10}Τῆς φυλῆς Ζαβουλὼν, Γουδιὴλ υἱὸς Σουδί.
\vs{11}Τῆς φυλῆς Ἰωσὴφ τῶν υἱῶν Μανασσῆ, Γαδδὶ υἱὸς Σουσί.
\vs{12}Τῆς φυλῆς Δὰν, Ἀμιὴλ υἱὸς Γαμαλί.
\vs{13}Τῆς φυλῆς Ἀσὴρ, Σαθοὺρ υἱὸς Μιχαήλ.
\vs{14}Τῆς φυλῆς Νεφθαλὶ, Ναβὶ υἱὸς Σαβί.
\vs{15}Τῆς φυλῆς Γὰδ, Γουδιὴλ υἱὸς Μακχί.
\vs{16}Ταῦτα τὰ ὀνόματα τῶν ἀνδρῶν, οὓς ἀπέστειλε Μωυσῆς κατασκέψασθαι τὴν γῆν· καὶ ἐπωνόμασε Μωυσῆς τὸν Αὐσὴ υἱὸν Ναυὴ, Ἰησοῦν.

\vs{17}Καὶ ἀπέστειλεν αὐτοὺς Μωυσῆς κατασκέψασθαι τὴν γῆν Χαναὰν, καὶ εἶπε πρὸς αὐτοὺς, ἀνάβητε ταύτῃ τῇ ἐρήμῳ, καὶ ἀναβήσεσθε εἰς τὸ ὄρος,
\vs{18}καὶ ὄψεσθε τὴν γῆν τίς ἐστι, καὶ τὸν λαὸν τὸν ἐγκαθήμενον ἐπʼ αὐτῆς, εἰ ἰσχυρός ἐστιν ἢ ἀσθενὴς, ἢ ὀλίγοι εἰσὶν ἢ πολλοί.
\vs{19}Καὶ τίς ἡ γῆ εἰς ἣν οὗτοι ἐγκάθηνται ἐπʼ αὐτῆς, ἢ καλή ἐστιν ἢ πονηρά· καὶ τίνες αἱ πόλεις ἃς οὗτοι κατοικοῦσιν ἐν αὐταῖς, εἰ ἐν τειχήρεσιν ἢ ἐν ἀτειχίστοις.
\vs{20}Καὶ τίς ἡ γῆ, ἢ πίων ἢ παρειμένη· εἰ ἔστιν ἐν αὐτῇ δένδρα, ἢ οὔ· καὶ προσκαρτερήσαντες λήψεσθε ἀπὸ τῶν καρπῶν τῆς γῆς. καὶ αἱ ἡμέραι, ἡμέραι ἔαρος, πρόδρομοι σταφυλῆς.

\vs{21}Καὶ ἀναβάντες κατεσκέψαντο τὴν γῆν ἀπὸ τῆς ἐρήμου Σὶν ἕως Ῥοὸβ, εἰσπορευομένων Αἰμάθ.
\vs{22}Καὶ ἀνέβησαν κατὰ τὴν ἔρημον, καὶ ἀπῆλθον ἕως Χεβρὼν, καὶ ἐκεῖ Ἀχιμὰν, καὶ Σεσσὶ, καὶ Θελαμὶ, γενεαὶ Ἐνάχ· καὶ Χεβρὼν ἐπτὰ ἔτεσιν ᾠκοδομήθη πρὸ τοῦ Τανὶν Αἰγύπτου.
\vs{23}Καὶ ἤλθοσαν ἕως φάραγγος βότρυος, καὶ κατεσκέψαντο αὐτήν· καὶ ἔκοψαν ἐκεῖθεν κλῆμα καὶ βότρυν σταφυλῆς ἕνα ἐπʼ αὐτοῦ, καὶ ᾖραν αὐτὸν ἐπʼ ἀναφορεύσι, καὶ ἀπὸ τῶν ῥοῶν, καὶ ἀπὸ τῶν συκῶν.
\vs{24}Καὶ τὸν τόπον ἐκεῖνον ἐπωνόμασαν Φάραγξ βότρυος, διὰ τὸν βότρυν, ὃν ἔκοψαν ἐκεῖθεν οἱ υἱοὶ Ἰσραήλ.
\vs{25}Καὶ ἀπέστρεψαν ἐκεῖθεν κατασκεψάμενοι τὴν γῆν μετὰ τεσσαράκοντα ἡμέρας.

\vs{26}Καὶ πορευθέντες ἦλθον πρὸς Μωυσῆν καὶ Ἀαρὼν καὶ πρὸς πᾶσαν συναγωγὴν υἱῶν Ἰσραὴλ, εἰς τὴν ἔρημον Φαρὰν Κάδης· καὶ ἀπεκρίθησαν αὐτοῖς ῥῆμα καὶ πάσῃ συναγωγῇ, καὶ ἔδειξαν τὸν καρπὸν τῆς γῆς,
\vs{27}καὶ διηγήσαντο αὐτῷ, καὶ εἶπαν, ἤλθαμεν εἰς τὴν γῆν εἰς ἣν ἀπέστειλας ἡμᾶς, γῆν ῥέουσαν γάλα καὶ μέλι· καὶ οὗτος ὁ καρπὸς αὐτῆς.
\vs{28}Ἀλλʼ ἢ ὅτι θρασὺ τὸ ἔθνος τὸ κατοικοῦν ἐπʼ αὐτῆς, καὶ πόλεις ὀχυραὶ τετειχισμέναι μεγάλαι σφόδρα· καὶ τὴν γενεὰν Ἐνὰχ ἑωράκαμεν ἐκεῖ.
\vs{29}Καὶ Ἀμαλὴκ κατοικεῖ ἐν τῇ γῇ τῇ πρὸς Νότον· καὶ ὁ Χετταῖος, καὶ ὁ Εὐαῖος, καὶ ὁ Ἰεβουσαῖος, καὶ ὁ Ἀμοῤῥαῖος κατοικεῖ ἐν τῇ ὀρεινῇ· καὶ ὁ Χαναναῖος κατοικεῖ παρὰ θάλασσαν, καὶ παρὰ τὸν Ἰορδάνην ποταμόν.
\vs{30}Καὶ κατεσιώπησε Χάλεβ τὸν λαὸν πρὸς Μωυσῆν, καὶ εἶπεν αὐτῷ, οὐχὶ, ἀλλὰ ἀναβάντες ἀναβησόμεθα, καὶ κατακληρονομήσομεν αὐτὴν, ὅτι δυνατοὶ δυνησόμεθα πρὸς αὐτούς.
\vs{31}Καὶ οἱ ἄνθρωποι οἱ συναναβάντες μετʼ αὐτοῦ, εἶπαν, οὐκ ἀναβαίνομεν, ὅτι οὐ μὴ δυνώμεθα ἀναβῆναι πρὸς τὸ ἔθνος, ὅτι ἰσχυρότερον ἡμῶν ἐστι μᾶλλον.
\vs{32}Καὶ ἐξήνεγκαν ἔκστασιν τῆς γῆς ἣν κατεσκέψαντο αὐτὴν πρὸς τοὺς υἱοὺς Ἰσραὴλ, λέγοντες, τὴν γῆς ἣν παρήλθομεν αὐτὴν κατασκέψασθαι, γῆ κατέσθουσα τοὺς κατοικοῦντας ἐπʼ αὐτῆς ἐστι· καὶ πᾶς ὁ λαὸς ὃν ἑωράκαμεν ἐν αὐτῇ, ἄνδρες ὑπερμήκεις.
\vs{33}Καὶ ἐκεῖ ἑωράκαμεν τοὺς γίγαντας, καὶ ἦμεν ἐνώπιον αὐτῶν ὡσεὶ ἀκρίδες· ἀλλὰ καὶ οὕτως ἦμεν ἐνώπιον αὐτῶν.

\ch{14}
Καὶ ἀναλαβοῦσα πᾶσα ἡ συναγωγὴ, ἐνέδωκε φωνήν· καὶ ἔκλαιεν ὁ λαὸς ὅλην τὴν νύκτα ἐκείνην.
\vs{2}Καὶ διεγόγγυζον ἐπὶ Μωυσῆν καὶ Ἀαρὼν πάντες οἱ υἱοὶ Ἰσραήλ· καὶ εἶπαν πρὸς αὐτοὺς πᾶσα ἡ συναγωγὴ,
\vs{3}Ὄφελον ἀπεθάνομεν ἐν γῇ Αἰγύπτῳ, ἢ ἐν τῇ ἐρήμῳ ταύτῃ, εἰ ἀπεθάνομεν· καὶ ἱνατί Κύριος εἰσάγει ἡμᾶς εἰς τὴν γῆν ταύτην πεσεῖν ἐν πολέμῳ; αἱ γυναῖκες ἡμῶν καὶ τὰ παιδία ἔσονται εἰς διαρπαγήν· νῦν οὖν βέλτιόν ἐστιν ἀποστραφῆναι εἰς Αἴγυπτον.
\vs{4}Καὶ εἶπαν ἕτερος τῷ ἑτέρῳ, δῶμεν ἀρχηγὸν, καὶ ἀποστρέψωμεν εἰς Αἴγυπτον.
\vs{5}Καὶ ἔπεσε Μωυσῆς καὶ Ἀαρὼν ἐπὶ πρόσωπον ἐναντίον πάσης συναγωγῆς υἱῶν Ἰσραήλ.

\vs{6}Ἰησοῦς δὲ ὁ τοῦ Ναυὴ, καὶ Χάλεβ ὁ τοῦ Ἰεφοννὴ τῶν κατασκεψαμένων τὴν γῆν, διέῤῥηξαν τὰ ἱμάτια αὐτῶν,
\vs{7}καὶ εἶπαν πρὸς πᾶσαν συναγωγὴν υἱῶν Ἰσραὴλ, λέγοντες, ἡ γῆ ἣν κατεσκεψάμεθα αὐτὴν, ἀγαθή ἐστι σφόδρα σφόδρα.
\vs{8}Εἰ αἱρετίζει ἡμᾶς Κύριος, εἰσάξει ἡμᾶς εἰς τὴν γῆν ταύτην, καὶ δώσει αὐτὴν ἡμῖν· γῆ ἥτις ἐστὶ ῥέουσα γάλα καὶ μέλι.
\vs{9}Ἀλλὰ ἀπὸ τοῦ Κυρίου μὴ ἀποστάται γίνεσθε· ὑμεῖς δὲ μὴ φοβηθῆτε τὸν λαὸν τῆς γῆς, ὅτι κατάβρωμα ὑμῖν ἐστιν· ἀφέστηκε γὰρ ὁ καιρὸς ἀπʼ αὐτῶν· ὁ δὲ Κύριος ἐν ἡμῖν· μὴ φοβηθήτε αὐτούς.

\vs{10}Καὶ εἶπε πᾶσα ἡ συναγωγὴ καταλιθοβολῆσαι αὐτοὺς ἐν λίθοις· καὶ ἡ δόξα Κυρίου ὤφθη ἐν τῇ νεφέλῃ ἐπὶ τῆς σκηνῆς τοῦ μαρτυρίου πᾶσι τοῖς υἱοῖς Ἰσραήλ.
\vs{11}Καὶ εἶπε Κύριος πρὸς Μωυσῆν, ἕως τίνος παροξύνει με ὁ λαὸς οὗτος; καὶ ἕως τίνος οὐ πιστεύουσί μοι ἐπὶ πᾶσι τοῖς σημείοις, οἷς ἐποίησα ἐν αὐτοῖς;
\vs{12}Πατάξω αὐτοὺς θανάτῳ, καὶ ἀπολῶ αὐτούς· καὶ ποιήσω σε καὶ τὸν οἶκον τοῦ πατρός σου εἰς ἔθνος μέγα, καὶ πολὺ μᾶλλον ἢ τοῦτο.
\vs{13}Καὶ εἶπε Μωυσῆς πρὸς Κύριον, καὶ ἀκούσεται Αἴγυπτος, ὅτι ἀνήγαγες τῇ ἰσχύϊ σου τὸν λαὸν τοῦτον ἐξ αὐτῶν.
\vs{14}Ἀλλὰ καὶ πάντες οἱ κατοικοῦντες ἐπὶ τῆς γῆς ταύτης ἀκηκόασιν, ὅτι σὺ εἶ Κύριος ἐν τῷ λαῷ τούτῷ, ὅστις ὀφθαλμοῖς κατʼ ὀφθαλμοὺς ὀπτάζῃ Κύριε, καὶ ἡ νεφέλη σου ἐφέστηκεν ἐπʼ αὐτῶν, καὶ ἐν στύλῳ νεφέλης σὺ πορεύῃ πρότερος αὐτῶν τὴν ἡμέραν, καὶ ἐν στύλῳ πυρὸς τὴν νύκτα.
\vs{15}Καὶ ἐκτρίψεις τὸν λαὸν τοῦτον ὡσεὶ ἄνθρωπον ἕνα· καὶ ἐροῦσι τὰ ἔθνη ὅσοι ἀκηκόασι τὸ ὄνομά σου, λέγοντες,
\vs{16}παρὰ τὸ μὴ δύνασθαι Κύριον εἰσαγαγεῖν τὸν λαὸν τοῦτον εἰς τὴν γῆν ἣν ὤμοσεν αὐτοῖς, κατέστρωσεν αὐτοὺς ἐν τῇ ἐρήμῳ.
\vs{17}Καὶ νῦν ὑψωθήτω ἡ ἰσχύς σου Κύριε, ὃν τρόπον εἶπας, λέγων,
\vs{18}Κύριος μακρόθυμος, καὶ πολυέλεος, καὶ ἀληθινὸς, ἀφαιρῶν ἀνομίας καὶ ἀδικίας καὶ ἁμαρτίας, καὶ καθαρισμῷ οὐ καθαριεῖ τὸν ἔνοχον, ἀποδιδοὺς ἁμαρτίας πατέρων ἐπὶ τέκνα ἕως τρίτης καὶ τετάρτης γενεᾶς.
\vs{19}Ἄφες τὴν ἁμαρτίαν τῷ λαῷ τούτῳ κατὰ τὸ μέγα ἔλεός σου, καθάπερ ἵλεως ἐγένου αὐτοῖς ἀπʼ Αἰγύπτου ἕως τοῦ νῦν.

\vs{20}Καὶ εἶπε Κύριος πρὸς Μωυσῆν, ἵλεως αὐτοῖς εἰμι κατὰ τὸ ῥῆμά σου.
\vs{21}Ἀλλὰ ζῶ ἐγὼ καὶ ζῶν τὸ ὄνομά μου, καὶ ἐμπλήσει ἡ δόξα Κυρίου πᾶσαν τὴν γῆν.
\vs{22}Ὅτι πάντες οἱ ἄνδρες οἱ ὁρώντες τὴν δόξαν μου, καὶ τὰ σημεῖα ἃ ἐποίησα ἐν Αἰγύπτῳ, καὶ ἐν τῇ ἐρήμῳ, καὶ ἐπείρασάν με τοῦτο δέκατον, καὶ οὐκ εἰσήκουσαν τῆς φωνῆς μου,
\vs{23}ἦ μὴν οὐκ ὄψονται τὴν γῆν, ἣν ὤμοσα τοῖς πατράσιν αὐτῶν· ἀλλʼ ἢ τὰ τέκνα αὐτῶν ἅ ἐστι μετʼ ἐμοῦ ὧδε, ὅσοι οὐκ οἴδασιν ἀγαθὸν οὐδὲ κακὸν, πᾶς νεώτερος ἄπειρος, τούτοις δώσω τὴν γῆν· πάντες δὲ οἱ παροξύναντές με, οὐκ ὄψονται αὐτήν.
\vs{24}Ὁ δὲ παῖς μου Χάλεβ, ὅτι πνεῦμα ἕτερον ἐν αὐτῷ, καὶ ἐπηκολούθησέ μοι, εἰσάξω αὐτὸν εἰς τὴν γῆν εἰς ἣν εἰσῆλθεν ἐκεῖ, καὶ τὸ σπέρμα αὐτοῦ κληρονομήσει αὐτήν.
\vs{25}Ὁ δὲ Ἀμαλὴκ καὶ ὁ Χαναναῖος κατοικοῦσιν ἐν τῇ κοιλάδι· αὔριον ἐπιστράφητε καὶ ἀπάρατε ὑμεῖς εἰς τὴν ἔρημον, ὁδὸν θάλασσαν ἐρυθρᾶν.

\vs{26}Καὶ εἶπε Κύριος πρὸς Μωυσῆν καὶ Ἀαρὼν, λέγων,
\vs{27}ἕως τίνος τὴν συναγωγὴν τὴν πονηρὰν ταύτην; ἅ αὐτοὶ γογγύζουσιν ἐναντίον μου, τὴν γόγγυσιν τῶν υἱῶν Ἰσραὴλ, ἣν ἐγόγγυσαν περὶ ὑμῶν, ἀκήκοα.
\vs{28}Εἶπον αὐτοῖς, ζῶ ἐγὼ, λέγει Κύριος· ἦ μὴν ὃν τρόπον λελαλήκατε εἰς τὰ ὦτά μου, οὕτω ποιήσω ὑμῖν.
\vs{29}Ἐν τῇ ἐρήμῳ ταύτῃ πεσεῖται τὰ κῶλα ὑμῶν· καὶ πᾶσα ἡ ἐπισκοπὴ ὑμῶν, καὶ οἱ κατηριθμημένοι ὑμῶν ἀπὸ εἰκοσαετοῦς καὶ ἐπάνω, ὅσοι ἐγόγγυσαν ἐπʼ ἐμοί·
\vs{30}εἰ ὑμεῖς εἰσελεύσεσθε εἰς τὴν γῆν ἐφʼ ἣν ἐξέτεινα τὴν χεῖρά μου κατασκηνῶσαι ὑμᾶς ἐπʼ αὐτῆς· ἀλλʼ ἢ Χάλεβ υἱὸς Ἰεφοννὴ, καὶ Ἰησοῦς ὁ τοῦ Ναυὴ.
\vs{31}Καὶ τὰ παιδία, ἃ εἴπατε ἐν διαρπαγῇ ἔσεσθαι, εἰσάξω αὐτοὺς εἰς τὴν γῆν· καὶ κληρονομήσουσι τὴν γῆν, ἣν ὑμεῖς ἁπέστητε ἀπʼ αὐτῆς.
\vs{32}Καὶ τὰ κῶλα ὑμῶν πεσεῖται ἐν τῇ ἐρήμῳ ταύτῃ.
\vs{33}Οἱ δὲ υἱοὶ ὑμῶν ἔσονται νεμόμενοι ἐν τῇ ἐρήμῳ τεσσαράκοντα ἔτη· καὶ ἀνοίσουσι τὴν πορνείαν ὑμῶν, ἕως ἂν ἀναλωθῇ τὰ κῶλα ὑμῶν ἐν τῇ ἐρήμῳ,
\vs{34}κατὰ τὸν ἀριθμὸν τῶν ἡμερῶν ὅσας κατεσκέψασθε τὴν γῆν, τεσσαράκοντα ἡμέρας, ἡμέραν τοῦ ἐνιαυτοῦ, λήμψεσθε τὰς ἁμαρτίας ὑμῶν τεσσαράκοντα ἔτη· καὶ γνώσεσθε τὸν θυμὸν τῆς ὀργῆς μου.
\vs{35}Ἐγὼ Κύριος ἐλάλησα, ἦ μὴν οὕτω ποιήσω τῇ συναγωγῇ τῇ πονηρᾷ ταύτῃ, τῇ ἐπισυνισταμένῃ ἐπʼ ἐμέ· ἐν τῇ ἐρήμῳ ταύτη ἐξαναλωθήσονται, καὶ ἐκεῖ ἀποθανοῦνται.

\vs{36}Καὶ οἱ ἄνθρωποι, οὓς ἀπέστειλεν Μωυσῆς κατασκέψασθαι τὴν γῆν, καὶ παραγενηθέντες διεγόγγυσαν κατʼ αὐτῆς πρὸς τὴν συναγωγὴν ἐξενέγκαι ῥήματα πονηρὰ περὶ τῆς γῆς,
\vs{37}καὶ ἀπέθανον οἱ ἄνθρωποι οἱ κατείπαντες πονηρὰ κατὰ τῆς γῆς ἐν τῇ πληγῇ ἔναντι Κυρίου.
\vs{38}Καὶ Ἰησοῦς υἱὸς Ναυὴ καὶ Χάλεβ υἱὸς Ἰεφοννὴ ἔζησαν ἀπὸ τῶν ἀνθρώπων ἐκείνων τῶν πεπορευμένων κατασκέψασθαι τὴν γῆν.
\vs{39}Καὶ ἐλάλησε Μωυσῆς τὰ ῥήματα ταῦτα πρὸς πάντας υἱοὺς Ἰσραήλ· καὶ ἐπενθησεν ὁ λαὸς σφόδρα.

\vs{40}Καὶ ὀρθρίσαντες τοπρωῒ ἀνέβησαν εἰς τὴν κορυφὴν τοῦ ὄρους, λέγοντες, ἰδοὺ, οἵδε ἡμεῖς ἀναβησόμεθα εἰς τὸν τόπον ὃν εἶπε Κύριος, ὅτι ἡμάρτομεν.
\vs{41}Καὶ εἶπε Μωυσῆς, ἱνατί ὑμεῖς παραβαίνετε τὸ ῥῆμα Κυρίου; οὐκ εὔοδα ἔσται ὑμῖν.
\vs{42}Μὴ ἀναβαίνετε, οὐ γάρ ἐστι Κύριος μεθʼ ὑμῶν· καὶ πεσεῖσθε πρὸ προσώπου τῶν ἐχθρῶν ὑμῶν.
\vs{43}Ὅτι ὁ Ἀμαλὴκ καὶ ὁ Χαναναῖος ἐκεῖ ἔμπροσθεν ὑμῶν, καὶ πεσεῖσθε μαχαίρᾳ, οὗ εἵνεκεν ἀπεστράφητε ἀπειθοῦντες Κυρίῳ, καὶ οὐκ ἔσται Κύριος ἐν ὑμῖν.
\vs{44}Καὶ διαβιασάμενοι, ἀνέβησαν ἐπὶ τὴν κορυφὴν τοῦ ὄρους· ἡ δὲ κιβωτὸς τῆς διαθήκης Κυρίου καὶ Μωυσῆς οὐκ ἐκινήθησαν ἐκ τῆς παρεμβολῆς.
\vs{45}Καὶ κατέβη ὁ Ἀμαλὴκ καὶ ὁ Χαναναῖος ὁ ἐνκαθήμενος ἐν τῷ ὄρει ἐκείνῳ, καὶ ἐτρέψαντο αὐτοὺς, καὶ κατέκοψαν αὐτοὺς ἕως Ἑρμάν· καὶ ἀπεστράφησαν εἰς τὴν παρεμβολήν.

\ch{15}
Καὶ εἶπε Κύριος πρὸς Μωυσῆν, λέγων, λάλησον τοῖς ὑοῖς Ἰσραὴλ,
\vs{2}καὶ ἐρεῖς πρὸς αὐτούς, ὅταν εἰσέλθητε εἰς τὴν γῆν τῆς κατοικήσεως ὑμῶν, ἣν ἐγὼ δίδωμι ὑμῖν,
\vs{3}καὶ ποιήσεις ὁλοκαυτώματα Κυρίῳ, ὁλοκάρπωμα ἢ θυσίαν, μεγαλῦναι εὐχὴν, ἢ καθʼ ἑκούσιον, ἢ ἐν ταῖς ἑορταῖς ὑμῶν ποιῆσαι ὀσμὴν εὐωδίας τῷ Κυρίῳ, εἰ μὲν ἀπὸ τῶν βοῶν ἢ ἀπὸ τῶν προβάτων.
\vs{4}Καὶ προσοίσει ὁ προσφέρων τὸ δῶρον αὐτοῦ Κυρίῳ, θυσίαν σεμιδάλεως δέκατον τοῦ οἰφί ἀναπεποιημένης ἐν ἐλαίῳ ἐν τετάρτῳ τοῦ ἴν.
\vs{5}Καὶ οἶνον εἰς σπονδὴν τὸ τέταρτον τοῦ ἴν ποιήσετε ἐπὶ τῆς ὁλοκαυτώσεως, ἢ ἐπὶ τῆς θυσίας· τῷ ἀμνῷ τῷ ἑνὶ ποιήσεις τοσοῦτο, κάρπωμα ὀσμὴν εὐωδίας τῷ Κυρίῳ.
\vs{6}Καὶ τῷ κριῷ, ὅταν ποιῆτε αὐτὸν εἰς ὁλοκαύτωμα ἢ εἰς θυσίαν, ποιήσεις θυσίαν σεμιδάλεως δύο δέκατα ἀναπεποιημένης ἐν ἐλαίῳ τὸ τρίτον τοῦ ἴν·
\vs{7}Καὶ οἶνον εἰς σπονδὴν τὸ τρίτον τοῦ ἴν προσοίσετε εἰς ὀσμὴν εὐωδίς Κυρίῳ.

\vs{8}Ἐὰν δὲ ποιῆτε ἀπὸ τῶν βοῶν εἰς ὁλοκαύτωσιν ἤ εἰς θυσίαν μεγαλῦναι εὐχὴν, ἢ εἰς σωτήριον Κυρίῳ,
\vs{9}καὶ προσοίσει ἐπὶ τοῦ μόσχου θυσίαν σεμιδάλεως τρία δέκατα ἀναπεποιημένης ἐν ἐλαίῳ ἥμισυ τοῦ ἴν.
\vs{10}Καὶ οἶνον εἰς σπονδὴν τὸ ἥμιου τοῦ ἴν, κάρπωμα ὀσμὴν εὐωδίας Κυρίῳ.

\vs{11}Οὕτω ποιήσεις τῷ μόσχῳ τῷ ἑνὶ, ἢ τῷ κριῷ τῷ ἑνὶ, ἢ τῷ ἀμνῷ τῷ ἑνὶ ἐκ τῶν προβάτων ἢ ἐκ τῶν αἰγῶν·
\vs{12}Κατὰ τὸν ἀριθμὸν ὧν ἐὰν ποιήσντε, οὕτως ποιήσετε τῷ ἑνὶ, κατὰ τὸν ἀριθμὸν αὐτῶν.

\vs{13}Πᾶς ὁ αὐτόχθων ποιήσει οὕτως τοιαῦτα προσενέγκαι καρπώματα εἰς ὀσμὴν εὐωδίας Κυρίῳ.
\vs{14}Ἐὰν δὲ προσήλυτος ἐν ὑμῖν προσγένηται ἐν τῇ γῇ ὑμῶν, ἢ ὂς ἂν γένηται ἐν ὑμῖν ἐν ταῖς γενεαῖς ὑμῶν, καὶ ποιήσει κάρπωμα ὀσμὴν εὐωδίας Κυρίῳ, ὃν τρόπον ποιεῖτε ὑμεῖς, οὕτω ποιήσει ἡ συναγωγὴ Κυρίῳ.

\vs{15}Νόμος εἷς ἔσται ὑμῖν καὶ τοῖς προσηλύτοις τοῖς προσκειμένοις ἐν ὑμῖν, νὀμος αἰώνιος εἰς τὰς γενεὰς ὑμῶν· ὡς ὑμεῖς, καὶ ὁ προσήλυτος ἔσται ἔναντι Κυρίου.
\vs{16}Νόμος εἷς ἔσται καὶ δικαίωμα ἓν ἔσται ὑμῖν καὶ τῷ προσηλύτῳ τῷ προσκειμένῳ ἐν ὑμῖν.

\vs{17}Καὶ ἐλάλησε Κύριος πρὸς Μωυσῆν, λέγων,
\vs{18}λάλησον τοῖς υἱοῖς Ἰσραὴλ, καὶ ἐρεῖς πρὸς αὐτοὺς, ἐν τῷ εἰσπορεύεσθαι ὑμᾶς εἰς τὴν γῆν, εἰς ἣν ἐγὼ εἰσάγω ὑμᾶς ἐκεῖ,
\vs{19}καὶ ἔσται ὅταν ἔσθητε ὑμεῖς ἀπὸ τῶν ἄρτων τῆς γῆς, ἀφελεῖτε ἀφαίρεμα ἀφόρισμα Κυρίῳ, ἀπαρχὴν φυράματον ὑμῶν.
\vs{20}Ἄρτον ἀφοριεῖτε ἀφαίρεμα αὐτό· ὡς ἀφαίρεμα ἀπὸ ἅλω, οὕτως ἀφελεῖτε αὐτὸν,
\vs{21}ἀπαρχὴν φυράματος ὑμῶν, καὶ δώσετε Κυρίῳ ἀφαίρεμα εἰς τὰς γενεὰς ὑμῶν.

\vs{22}Ὅταν δὲ διαμάρτητε καὶ μὴ ποιήσητε πάσας τὰς ἐντολὰς ταύτας, ἅς ἐλάλησε Κύριος πρὸς Μωυσῆν,
\vs{23}καθὰ συνέταξε Κύριος πρὸς ὑμᾶς ἐν χειρὶ Μωυσῆ, ἀπὸ τῆς ἡμέρας ᾗ συνέταξε Κύριος πρὸς ὑμᾶς καὶ ἐπέκεινα εἰς τὰς γενεὰς ὑμῶν,
\vs{24}καὶ ἔσται ἐὰν ἐξ ὀφθαλμῶν τῆς συναγωγῆς γενηθῇ ἀκουσίως, καὶ ποιήσει πᾶσα ἡ συναγωγὴ μόσχον ἕνα ἐκ βοῶν ἄμωμον εἰς ὁλοκαύτωμα εἰς ὀσμὴν εὐωδίας Κυρίῳ, καὶ θυσίαν τούτου καὶ σπονδὴν αὐτοῦ κατὰ τὴν σύνταξιν, καὶ χίμαρον ἐξ αἰγῶν ἕνα περὶ ἁμαρτίας.
\vs{25}Καὶ ἐξιλάσεται ὁ ἱερεὺς περὶ πάσης συναγωγῆς υἱῶν Ἰσραὴλ, καὶ ἀφεθήσεται αὐτοῖς, ὅτι ἀκούσιόν ἐστι· καὶ αὐτοὶ ἤνεγκαν τὸ δῶρον αὐτῶν κάρπωμα Κυρίῳ περὶ τῆς ἁμαρτίας αὐτῶν ἔνατι Κυρίου, περὶ τῶν ἀκουσίων αὐτῶν.
\vs{26}Καὶ ἀφεθήσεται κατὰ πᾶσαν συναγωγην υἱῶν Ἰσραὴλ, καὶ τῷ προσηλύτῳ τῷ προσκειμένῳ πρὸς ὑμᾶς, ὅτι παντὶ τῷ λαῷ ἀκούσιον.

\vs{27}Ἐάν τε ψυχὴ μία ἁμάρτῃ ἀκουσίως, προσάξει αἶγα μίαν ἐνιαυσίαν περὶ ἁμαρτίας.
\vs{28}Καὶ ἐξιλάσεται ὁ ἱερεὺς περὶ τῆς ψυχῆς τῆς ἀκουσιασθείσης, καὶ ἁμαρτούσης ἀκουσίως ἔναντι Κυρίου, ἐξιλάσασθαι περὶ αὐτοῦ.
\vs{29}Τῷ ἐγχωρίῳ ἐν υἱοῖς Ἰσραὴλ, καὶ τῷ προσηλύτῳ τῷ προσκειμένῳ ἐν αὐτοῖς νόμος εἷς ἔσται αὐτοῖς, ὃς ἐὰν ποιήσῃ ἀκουσίως.

\vs{30}Καὶ ψυχὴ ἥτις ποιήσῃ ἐν χειρὶ ὑπερηφανίας ἀπὸ τῶν αὐτοχθόνων ἢ ἀπὸ τῶν προσηλύτων, τὸν Θεὸν οὗτος παροξυνεῖ, ἐξόλοθρευθήσεται ἡ ψυχὴ ἐκείνη ἐκ τοῦ λαοῦ αὐτῆς,
\vs{31}ὅτι τὸ ῥῆμα Κυρίου ἐφαύλισε, καὶ τὰς ἐντολὰς αὐτοῦ διεσκέδασεν· ἐκτρίψει ἐκτριβήσεται ἡ ψυχὴ ἐκείνη, ἡ ἁμαρτία αὐτῆς ἐν αὐτῇ.

\vs{32}Καὶ ἦσαν οἱ υἱοὶ Ἰσραὴλ ἑν τῇ ἐρήμῳ, καὶ εὗρον ἄνδρα συλλέγοντα ξύλα τῇ ἡμέρᾳ τῶν σαββάτων.
\vs{33}Καὶ προσήγαγον αὐτὸν οἱ εὐρόντες συλλέγοντα ξύλα τῇ ἡμέρᾳ τῶν σαββάτων πρὸς Μωυσῆν καὶ Ἀαρὼν, καὶ πρὸς πᾶσαν συναγωγὴν υἱῶν Ἰσραήλ.
\vs{34}Καὶ ἀπέθεντο αὐτὸν εἰς φυλακὴν, οὐ γὰρ συνέκριναν τί ποιήσωσιν αὐτόν.
\vs{35}Καὶ ἐλάλησε Κύριος πρὸς Μωυσῆν, λέγων, θανάτῳ θανατούσθω ὁ ἄνθρωπος· λιθοβολήσατε αὐτὸν λίθοις πᾶσα ἡ συναγωγή.
\vs{36}Καὶ ἐξήγαγον αὐτὸν πᾶσα ἡ συναγωγὴ ἔξω τῆς παρεμβολῆς· καὶ ἐλιθοβόλησεν αὐτὸν πᾶσα ἡ συναγωγὴ λίθοις ἔξω τῆς παρεμβολῆς, καθὰ συνέταξε Κύπιος τῷ Μωυσῇ.

\vs{37}Καὶ εἶπε Κύριος πρὸς Μωυσῆν, λέγων,
\vs{38}λάλησον τοῖς υἱοῖς Ἰσραὴλ, καὶ ἐρεῖς πρὸς αὐτοὺς, καὶ ποιησάτωσαν ἐαυτοῖς κράσπεδα ἐπὶ τὰ πτερύγια τῶν ἱματίων αὐτῶν εἰς τὰς γενεὰς αὐτῶν· καὶ ἐπιθήσετε ἐπὶ τὰ κράσπεδα τῶν πτερυγίων κλῶσμα ὑακίνθινον.
\vs{39}Καὶ ἔσται ὑμῖν ἐν τοῖς κρασπέδοις, καὶ ὄψεσθε αὐτά· καὶ μνησθήσεσθε πασῶν τῶν ἐντολῶν Κυρίου, καὶ ποιήσετε αὐτάς· καὶ οὐ διαστραφήσεσθε ὀπίσω τῶν διανοιῶν ὑμῶν, καὶ τῶν ὀφθαλμῶν ἐν οἷς ὑμεῖς ἐκπορνεύετε ὀπίσω αὐτῶν, ὅπως ἂν μνησθῆτε καὶ ποιήσητε πάσας τὰς ἐντολάς μου,
\vs{40}καὶ ἔσεσθε ἅγιοι τῷ Θεῷ ὑμῶν.
\vs{41}Ἐγὼ Κύριος ὁ Θεὸς ὑμῶν ὁ ἐξαγαγὼν ὑμᾶς ἐκ γῆς Αἰγύπτου, εἶναι ὑμῶν Θεός· ἐγὼ Κύριος ὁ Θεὸς ὑμῶν.

\ch{16}
Καὶ ἐλάλησε Κορὲ υἱὸς Ἰσαὰρ υἱοῦ Καὰθ υἱοῦ Λευί, καὶ Δαθὰν καὶ Ἀβειρὼν υἱοὶ Ἐλιάβ, καὶ Αὒν υἱὸς Φαλὲθ υἱοῦ Ῥουβήν·
\vs{2}καὶ ἀνέστησαν ἔναντι Μωυσῆ, καὶ ἄνδρες τῶν υἱῶν Ἰσραὴλ πεντήκοντα καὶ διακόσιοι, ἀρχηγοὶ συναγωγῆς, σύγκλητοι βουλῆς, καὶ ἄνδρες ὀνομαστοί.
\vs{3}Συνέστησαν ἐπὶ Μωυσῆν καὶ Ἀαρὼν, καὶ εἶπαν, ἐχέτω ὑμῖν ὅτι πᾶσα ἡ συναγωγὴ πάντες ἅγιοι, καὶ ἐν αὐτοῖς Κύριος· καὶ διατί κατανίστασθε ἐπὶ τὴν συναγωγὴν Κυρίου;
\vs{4}Καὶ ἀκούσας Μωυσῆς, ἔπεσεν ἐπὶ πρόσωπον.
\vs{5}Καὶ ἐλάλησε πρὸς Κόρε καὶ πρὸς πᾶσαν αὐτοῦ τὴν συναγωγὴν, λέγων, ἐπέσκεπται καὶ ἔγνω ὁ Θεὸς τοὺς ὄντας αὐτοῦ καὶ τοὺς ἁγίους, καὶ προσηγάγετο πρὸς ἑαυτόν· καὶ οὓς ἐξελέξατο ἑαυτῷ, προσηγάγετο πρὸς ἑαυτόν.
\vs{6}Τοῦτο ποιήσατε· λάβετε ὑμῖν αὐτοῖς πυρεῖα Κορὲ, καὶ πᾶσα ἡ συναγωγὴ αὐτοῦ,
\vs{7}καὶ ἐπίθετε ἐπʼ αὐτὰ πῦρ, καὶ ἐπίθετε ἐπʼ αὐτὰ θυμίαμα ἔναντι Κυρίου αὔριον· καὶ ἔσται ὁ ἀνὴρ ὃν ἐκλέλεκται Κύριος, οὗτος ἅγιος· ἱκανούσθω ὑμῖν υἱοὶ Λευί.
\vs{8}Καὶ εἶπε Μωυσῆς πρὸς Κορὲ, εἰσακούσατέ μου υἱοὶ Λευί.
\vs{9}Μὴ μικρόν ἐστι τοῦτο ὑμῖν, ὅτι διέστειλεν ὁ Θεὸς Ἰσραὴλ ὑμᾶς ἐκ συναγωγῆς Ἰσραὴλ, καὶ προσηγάγετο ὑμᾶς πρὸς ἑαυτὸν λειτουργεῖν τᾶς λειτουργίας τῆς σκηνῆς Κυρίου, καὶ παρίστασθαι ἔναντι τῆς σκηνῆς λατρεύειν αὐτοῖς;
\vs{10}καὶ προσηγάγετό σε καὶ πάντας τοὺς ἀδελφούς σου υἱοὺς Λευὶ μετὰ σοῦ, καὶ ζητεῖτε καὶ ἱερατεύειν;
\vs{11}Οὕτως σὺ καὶ πᾶσα ἡ συναγωγή σου ἡ συνηθροισμένη πρὸς τὸν Θεόν· καὶ Ἀαρὼν τίς ἐστιν, ὅτι διαγογγύζετε κατʼ αὐτοῦ;

\vs{12}Καὶ ἀπέστειλε Μωυσῆς καλέσαι Δαθὰν καὶ Ἀβειρὼν υἱοὺς Ἑλιάβ· καὶ εἶπαν, οὐκ ἀναβαίνομεν.
\vs{13}Μὴ μικρὸν τοῦτο, ὅτι ἀνήγαγες ἡμᾶς εἰς γῆν ῥέουσαν γάλα καὶ μέλι, ἀποκτεῖναι ἡμᾶς ἐν τῇ ἐρήμῳ, ὅτι κατάρχεις ἡμῶν;
\vs{14}Ἀρχων εἶ· καὶ σὺ εἰς γῆν ῥέουσαν γάλα καὶ μέλι εἰσήγαγες ἡμᾶς, καὶ ἔδωκας ἡμῖν κλῆρον ἀγροῦ καὶ ἀμπελῶνας; τοὺς ὀφθαλμοὺς τῶν ἀνθρώπων ἐκείνων ἄν ἐξέκοψας; οὐκ ἀναβαίνομεν.
\vs{15}Καὶ ἐβαρυθύμησε Μωυσῆς σφόδρα, καὶ εἶπε πρὸς Κύριον, μὴ πρόσχῃς εἰς τὴν θυσίαν αὐτῶν· οὐκ ἐπιθύμημα οὐδενὸς αὐτῶν εἴληφα, οὐδὲ ἐκάκωσα οὐδένα αὐτῶν.
\vs{16}Καὶ εἶπε Μωυσῆς πρὸς Κορὲ, ἁγίασον τὴν συναγωγήν σου, καὶ γίνεσθε ἕτοιμοι ἔναντι Κυρίου σὺ καὶ Ἀαρὼν καὶ αὐτοὶ αὔριον·
\vs{17}Καὶ λάβετε ἕκαστος τὸ πυρεῖον αὐτοῦ, καὶ ἐπιθήσετε ἐπʼ αὐτὰ θυμίαμα, καὶ προσάξετε ἔναντι Κυρίου ἕκαστος τὸ πυρεῖον αὐτοῦ, πεντήκοντα καὶ διακόσια πυρεῖα, καὶ σὺ καὶ Ἀαρὼν ἕκαστος τὸ πυρεῖον αὐτοῦ.

\vs{18}Καὶ ἔλαβεν ἕκαστος τὸ πυρεῖον αὐτοῦ, καὶ ἐπέθηκαν ἐπʼ αὐτὰ πῦρ, καὶ ἐπέβαλον ἐπʼ αὐτὰ θυμίαμα· καὶ ἔστησαν παρὰ τὰς θύρας τῆς σκηνῆς τοῦ μαρτυρίου Μωυσῆς καὶ Ἀαρών.
\vs{19}Καὶ ἐπισυνέστησεν ἐπʼ αὐτοὺς Κορὲ τὴν πᾶσαν αὐτοῦ συναγωγὴν παρὰ τὴν θύραν τῆς σκηνῆς τοῦ μαρτυρίου· καὶ ὤφθη ἡ δόξα Κυρίου πάσῃ τῇ συναγωγῇ.
\vs{20}Καὶ ἐλάλησε Κύριος πρὸς Μωυσῆν καὶ Ἀαρὼν, λέγων,
\vs{21}ἀποσχίσθητε ἐκ μέσου τῆς συναγωγῆς ταύτης, καὶ ἐξαναλώσω αὐτοὺς εἰσάπαξ.
\vs{22}Καὶ ἔπεσαν ἐπὶ πρόσωπον αὐτῶν, καὶ εἶπαν, Θεὸς, Θεὸς τῶν πνευμάτων καὶ πάσης σαρκὸς, εἰ ἄνθρωπος εἷς ἥμαρτεν, ἐπὶ πᾶσαν τὴν συναγωγὴν ὀργὴ Κυρίου;
\vs{23}Καὶ ἐλάλησε Κύριος πρὸς Μωυσῆν, λέγων,
\vs{24}λάλησον τῇ συναγωγῇ, λέγων, ἀναχωρήσατε κύκγῳ ἀπὸ τῆς συναγωγῆς Κορὲ.

\vs{25}Καὶ ἀνέστη Μωυσῆς, καὶ ἐπορεύθη πρὸς Δαθὰν καὶ Ἀβειρὼν, καὶ συνεπορεύθησαν μετʼ αὐτοῦ πάντες οἱ πρεσβύτεροι Ἰσραήλ.
\vs{26}Καὶ ἐλάλησε πρὸς τὴν συναγωγὴν, λέγων, ἀποσχίσθητε ἀπὸ τῶν σκηνῶν τῶν ἀνθρώπων τῶν σκληρῶν τούτων, καὶ μὴ ἅπτεσθε ἀπὸ πάντων ὧν ἐστιν αὐτοῖς, μὴ συναπόλησθε ἐν πάσῃ τῇ ἁμαρτίᾳ αὐτῶν.
\vs{27}Καὶ ἀπέστησαν ἀπὸ τῆς σκηνῆς Κορὲ κύκλῳ· καὶ Δαθὰν καὶ Ἀβειρὼν ἐξῆλθον, καὶ εἱστήκεισαν παρὰ τὰς θύρας τῶν σκηνῶν αὐτῶν, καὶ αἱ γυναῖκες αὐτῶν, καὶ τὰ τέκνα αὐτῶν, καὶ ἡ ἀποσκευὴ αὐτῶν.

\vs{28}Καὶ εἶπε Μωυσῆς, ἐν τούτῳ γνώσεσθε ὅτι Κύριος ἀπέστειλέ με ποιῆσαι πάντα τὰ ἔργα ταῦτα, ὅτι οὐκ ἀπʼ ἐμαυτοῦ.
\vs{29}Εἰ κατὰ θάνατον πάντων ἀνθρώπων ἀποθανοῦνται οὗτοι, εἰ καὶ κατʼ ἐπίσκεψιν πάντων ἀνθρώπων ἐπισκοπὴ ἔσται αὐτῶν, οὐχὶ Κύριος ἀπέσταλκέ με.
\vs{30}Ἀλλʼ ἢ ἐν φάσματι δείξει Κύριος, καὶ ἀνοίξασα ἡ γῆ τὸ στόμα αὐτῆς καταπίεται αὐτοὺς, καὶ τοὺς οἴκους αὐτῶν, καὶ τὰς σκηνὰς αὐτῶν, καὶ πάντα ὅσα ἐστὶν αὐτοῖς, καὶ καταβήσονται ζῶντες εἰς ᾅδου, καὶ γνώσεσθε, ὅτι παρώξυναν οἱ ἄνθρωποι οὗτοι τὸν Κύριον.

\vs{31}Ὡς δὲ ἐπαύσατο λαλῶν πάντας τοὺς λόγους τούτους, ἐῤῥάγη ἡ γῆ ὑποκάτω αὐτῶν.
\vs{32}Καὶ ἠνοίχθη ἡ γῆ, καὶ κατέπιεν αὐτοὺς, καὶ τοὺς οἴκους αὐτῶν, καὶ πάντας τοὺς ἀνθρώπους τοὺς ὄντας μετὰ Κορὲ, καὶ τὰ κτήνη αὐτῶν.
\vs{33}Καὶ κατέβησαν αὐτοὶ, καὶ ὅσα ἐστὶν αὐτῶν ζῶντα εἰς ᾅδου, καὶ ἐκάλυψεν αὐτοὺς ἡ γῆ, καὶ ἀπώλοντο ἐκ μέσου τῆς συναγωγῆς.
\vs{34}Καὶ πᾶς Ἰσραὴλ οἱ κύκλῳ αὐτῶν ἔφυγον ἀπὸ τῆς φωνῆς αὐτῶν, ὅτι λέγοντες, μή ποτε καταπίῃ ἡμᾶς ἡ γῆ.
\vs{35}Καὶ πῦρ ἐξῆλθε παρὰ Κυρίου, καὶ κατέφαγε τοὺς πεντήκοντα καὶ διακοσίους ἄνδρας τοὺς προσφέροντας τὸ θυμίαμα.

\ch{17}
Καὶ εἶπε Κύριος πρὸς Μωυσῆν,
\vs{2}καὶ πρὸς Ἐλεάζαρ τὸν υἱὸν Ἀαρὼν τὸν ἱερέα, ἀνέλεσθε τὰ πυρεῖα τὰ χαλκᾶ ἐκ μέσου τῶν κατακεκαυμένων, καὶ τὸ πῦρ τὸ ἀλλότριον τοῦτο σπεῖρον ἐκεῖ, ὅτι ἡγίασαν τὰ πυρεῖα τῶν ἁμαρτωλῶν τούτων ἐν ταῖς ψυχαῖς αὐτῶν,
\vs{3}καὶ ποίησον αὐτὰ λεπίδας ἐλατὰς περίθεμα τῷ θυσιαστηρίῳ, ὅτι προσηνέχθησαν ἔναντι Κυρίου καὶ ἡγιάσθησαν· καὶ ἐγένοντο εἰς σημεῖον τοῖς υἱοῖς Ἰσραήλ.
\vs{4}Καὶ ἔλαβεν Ἐλεάζαρ υἱοὶς Ἀαρὼν τοῦ ἱερέως τὰ πυρεῖα τὰ χαλκᾶ, ὅσα προσήνεγκαν οἱ κατακεκαυμένοι, καὶ προσέθηκαν αὐτὰ περίθεμα τῷ θυσιαστηρίῳ,
\vs{5}μνημόσυνον τοῖς υἱοῖς Ἰσραήλ, ὅπως ἂν μὴ προσέλθῃ μηδὶς ἀλλογενής, ὃς οὐκ ἔστιν ἐκ τοῦ σπέρματος Ἀαρὼν, ἐπιθεῖναι θυμίαμα ἔναντι Κυρίου· καὶ οὐκ ἔσται ὥσπερ Κορὲ, καὶ ἡ ἐπισύστασις αὐτοῦ, καθὰ ἐλάλησε Κύριος ἐν χειρὶ Μωυσῆ αὐτῷ.

\vs{6}Καὶ ἐγόγγυσαν οἱ υἱοὶ Ἰσραὴλ τῇ ἐπαύριον ἐπὶ Μωυσῆν καὶ Ἀαρὼν, λέγοντες, ὑμεῖς ἀπεκτάγκατε τὸν λαὸν Κυρίου.
\vs{7}Καὶ ἐγένετο ἐν τῷ ἐπισυστρέφεσθαι τὴν συναγωγὴν ἐπὶ Μωυσῆν καὶ Ἀαρὼν, καὶ ὥρμησαν ἐπὶ τὴν σκηνὴν τοῦ μαρτυρίου· καὶ τήνδε ἐκάλυψεν αὐτὴν ἡ νεφέλη, καὶ ὤφθη ἡ δόξα Κυρίου.
\vs{8}Καὶ εἰσῆλθε Μωυσῆς καὶ Ἀαρὼν κατὰ πρόσωπον τῆς σκηνῆς τοῦ μαρτυρίου.
\vs{9}Καὶ ἐλάλησε Κύριος πρὸς Μωυσῆν καὶ Ἀαρὼν, λέγων,
\vs{10}ἐκχωρήσατε ἐκ μέσου τῆς συναγωγῆς ταύτης, καὶ ἐξαναλώσω αὐτοὺς εἰσάπαξ· καὶ ἔπεσον ἐπὶ πρόσωπον αὐτῶν.
\vs{11}Καὶ εἶπε Μωυσῆς πρὸς Ἀαρὼν, λάβε τὸ πυρεῖον, καὶ ἐπίθες ἐπʼ αὐτὸ πῦρ ἀπὸ τοῦ θυσιαστηρίου, καὶ ἐπίβαλε ἐπʼ αὐτὸ θυμίαμα, καὶ ἀπένεγκε τοτάχος εἰς τὴν παρεμβολὴν, καὶ ἐξίλασαι περὶ αὐτῶν· ἐξῆλθε γὰρ ὀργὴ ἀπὸ προσώπου Κυρίου, ἦρκται θραύειν τὸν λαόν.
\vs{12}Καὶ ἔλαβεν Ἀαρὼν καθάπερ ἐλάλησεν αὐτῷ Μωυσῆς, καὶ ἔδραμεν εἰς τὴν συναγωγήν· καὶ ἤδη ἐνῆρκτο ἡ θραῦσις ἐν τῷ λαῷ· καὶ ἐπέβαλε τὸ θυμίαμα καὶ ἐξιλάσατο περὶ τοῦ λαοῦ.
\vs{13}Καὶ ἔστη ἀναμέσον τῶν τεθνηκότων καὶ τῶν ζώντων, καὶ ἐκόπασεν ἡ θραῦσις.
\vs{14}Καὶ ἐγένοντο οἱ τεθνηκότες ἐν τῇ θραύσει τεσσαρεσκαίδεκα χιλιάδες καὶ ἑπτακόσιοι, χωρὶς τῶν τεθνηκότων ἕνεκεν Κορέ.
\vs{15}Καὶ ἐπέστρεψεν Ἀαρὼν πρὸς Μωυσῆν ἐπὶ τὴν θύραν τῆς σκηνῆς τοῦ μαρτυρίου, καὶ ἐκόπασεν ἡ θραῦσις.

\vs{16}Καὶ ἐλάλησε Κύριος πρὸς Μωυσῆν, λέγων,
\vs{17}λάλησον τοῖς υἱοῖς Ἰσραὴλ, καὶ λάβε παρʼ αὐτῶν ῥάβδον, ῥάβδον κατʼ οἴκους πατριῶν παρὰ πάντων τῶν ἀρχόντων αὐτῶν, κατʼ οἴκους πατριῶν αὐτῶν, δώδεκα ῥάβδους, καὶ ἑκάστου τὸ ὄνομα αὐτοῦ ἐπὶγραψον ἐπὶ τῆς ῥάβδου.
\vs{18}Καὶ τὸ ὄνομα Ἀαρὼν ἐπίγραψον ἐπὶ τῆς ῥάβδου Λευί· ἔστι γὰρ ῥάβδος μία· κατὰ φυλὴν οἴκου πατριῶν αὐτῶν δώσουσι.
\vs{19}Καὶ θήσεις αὐτὰς ἐν τῇ σκηνῇ τοῦ μαρτυρίου, κατέναντι τοῦ μαρτυρίου, ἐν οἷς γνωσθήσομαί σοι ἐκεῖ.
\vs{20}Καὶ ἔσται ὁ ἄνθρωπος ὃν ἂν ἐκλέξωμαι αὐτὸν, ἡ ῥάβδος αὐτοῦ ἐκβλαστήσει· καὶ περιελῶ ἀπʼ ἐμοῦ τὸν γογγυσμὸν υἱῶν Ἰσραήλ, ἃ αὐτοὶ γογγύζουσιν ἐφʼ ὑμῖν.

\vs{21}Καὶ ἐλάλησε Μωυσῆς τοῖς υἱοῖς Ἰσραήλ· καὶ ἔδωκαν αὐτῷ πάντες οἱ ἄρχοντες αὐτῶν ῥάβδον· τῷ ἄρχοντι τῷ ἑνὶ ῥάβδον κατʼ ἄρχοντα, κατʼ οἴκους πατριῶν αὐτῶν, δώδεκα ῥάβδους· καὶ ἡ ῥάβδος Ἀαρὼν ἀναμέσον τῶν ῥάβδων αὐτῶν.
\vs{22}Καὶ ἀπέθηκε Μωυσῆς τὰς ῥάβδους ἔναντι Κυρίου ἐν τῇ σκηνῇ τοῦ μαρτυρίου.
\vs{23}Καὶ ἐγενετο τῇ ἐπαύριον, καὶ εἰσῆλθε Μωυσῆς καὶ Ἀαρὼν ἐν τῇ σκηνῇ τοῦ μαρτυρίου· καὶ ἰδοὺ ἐβλάστησεν ἡ ῥάβδος Ἀαρὼν εἰς οἶκον Λευί, καὶ ἐξήνεγκε βλαστὸν, καὶ ἐξήνθησεν ἄνθη, καὶ ἐβλάστησε κάρυα.
\vs{24}Καὶ ἐξήνεγκε Μωυσῆς πάσας τὰς ῥάβδους ἀπὸ προσώπου Κυρίου πρὸς πάντας υἱοὺς Ἰσραήλ· καὶ εἶδον, καὶ ἔλαβον ἕκαστος τὴν ῥάβδον αὐτοῦ.

\vs{25}Καὶ εἶπε Κύριος πρὸς Μωυσῆν, ἀπόθες τὴν ῥάβδον Ἀαρὼν ἐνώπιον τῶν μαρτυρίων εἰς διατήρησιν, σημεῖον τοῖς υἱοῖς τῶν ἀνηκόων· καὶ παυσάσθω ὁ γογγυσμὸς αὐτῶν ἀπʼ ἐμοῦ, καὶ οὐ μὴ ἀποθάνωσι.
\vs{26}Καὶ ἐποίησε Μωυσῆς καὶ Ἀαρὼν, καθὰ συνέταξε κύριος τῷ Μωυσῇ, οὕτως ἐποίησαν.
\vs{27}Καὶ εἶπαν οἱ υἱοὶ Ἰσραὴλ πρὸς Μωυσῆν, λέγοντες, ἰδοὺ ἐξανηλώμεθα, ἀπολώλαμεν, παρανηλώμεθα.
\vs{28}Πᾶς ὁ ἁπτόμενος τῆς σκηυῆς Κυρίου, ἀποθνήσκει· ἕως εἰς τέλος ἀποθάνωμεν;

\ch{18}
Καὶ εἶπε Κύριος πρὸς Ἀαρὼν, λέγων, σὺ καὶ οἱ υἱοί σου καὶ ὁ οἶκος τοῦ πατρός σου λήψεσθε τὰς ἀμαρτίας τῶν ἁγίων, καὶ σὺ καὶ οἱ υἱοί σου λήψεσθε τὰς ἁμαρτίας τῆς ἱερατείας ὑμῶν.
\vs{2}Καὶ τοὺς ἀδελφούς σου φυλὴν Λευὶ, δῆμον τοῦ πατρός σου προσαγάγου πρὸς σεαυτὸν, καὶ προστεθήτωσάν σοι, καὶ λειτουργείτωσάν σοι· καὶ σὺ καὶ οἱ υἱοί σου μετὰ σοῦ ἀπέναντι τῆς σκηνῆς τοῦ μαρτυρίου.
\vs{3}Καὶ φυλάξονται τὰς φυλακάς σου, καὶ τὰς φυλακὰς τῆς σκηνῆς. πλὴν πρὸς τὰ σκεύη τὰ ἅγια, καὶ πρὸς τὸ θυσιαστήριον οὐ προσελεύσονται, καὶ οὐκ ἀποθανοῦνται καὶ οὗτοι καὶ ὑμεῖς.
\vs{4}Καὶ προστεθήσονται πρὸς σέ, καὶ φυλάξονται τὰς φυλακὰς τῆς σκηνῆς τοῦ μαρτυρίου, κατὰ πάσας τὰς λειτουργίας τῆς σκηνῆς· καὶ ὁ ἀλλογενὴς οὐ προσελεύσεται πρὸς σέ.
\vs{5}Καὶ φυλάξεσθε τὰς φυλακὰς τῶν ἁγίων, καὶ τὰς φυλακὰς τοῦ θυσιαστηρίου, καὶ οὐκ ἔσται θυμὸς ἐν τοῖς υἱοῖς Ἰσραήλ.
\vs{6}Καὶ ἐγὼ εἴληφα τοὺς ἀδελφοὺς ὑμῶν τοὺς Λευίτας ἐκ μέσου τῶν υἱῶν Ἰσραὴλ δόμα δεδομένον Κυρίῳ, λειτουργεῖν τὰς λειτουργίας τῆς σκηνῆς τοῦ μαρτυρίου.
\vs{7}Καὶ σὺ καὶ οἱ υἱοί σου μετὰ σοῦ διατηρήσετε τὴν ἱερατείαν ὑμῶν, κατὰ πάντα τρόπον τοῦ θυσιαστηρίου, καὶ τὸ ἔνδοθεν τοῦ καταπετάσματος· καὶ λειτουργήσετε τὰς λειτουργίας δόμα τῆς ἱερατείας ὑμῶν· καὶ ὁ ἀλλογενὴς ὁ προσπορευόμενος ἀποθανεῖται.

\vs{8}Καὶ ἐλάλησε Κύριος πρὸς Ἀαρὼν, καὶ ἰδοὺ ἐγὼ δέδωκα ὑμῖν τὴν διατήρησιν τῶν ἀπαρχῶν ἀπὸ πάντων τῶν ἡγιασμένων μοι παρὰ τῶν υἱῶν Ἰσραήλ· σοὶ δέδωκα αὐτὰ εἰς γέρας, καὶ τοῖς υἱοῖς σου μετὰ σὲ νόμιμον αἰώνιον.
\vs{9}Καὶ τοῦτο ἔστω ὑμῖν ἀπὸ τῶν ἡγιασμένων ἁγίων τῶν καρπωμάτων, ἀπὸ πάντων τῶν δώρων αὐτῶν, καὶ ἀπὸ πάντων τῶν θυσιασμάτων αὐτῶν, καὶ ἀπὸ πάσης πλημμελείας αὐτῶν, καὶ ἀπὸ πασῶν τῶν ἁμαρτιῶν αὐτῶν, ὅσα ἀποδιδόασί μοι ἀπὸ πάντων τῶν ἁγίων, σοὶ ἔσται καὶ τοῖς υἱοῖς σου.
\vs{10}Ἐν τῷ ἁγίῳ τῶν ἁγίων φάγεσθε αὐτά· πᾶν ἀρσενικὸν φάγεται αὐτά· σὺ καὶ οἱ υἱοί σου· ἅγια ἔσται σοι.

\vs{11}Καὶ τοῦτο ἔσται ὑμῖν ἀπαρχῶν δομάτων αὐτῶν, ἀπὸ πάντων τῶν ἐπιθεμάτων τῶν υἱῶν Ἰσραήλ· σοὶ δέδωκα αὐτὰ καὶ τοῖς υἱοῖς σου καὶ ταῖς θυγατράσι σου μετὰ σοῦ, νόμιμον αἰώνιον· πᾶς καθαρὸς ἐν τῷ οἴκῳ σου ἔδεται αὐτά.

\vs{12}Πᾶσα ἀπαρχὴ ἐλαίου, καὶ πᾶσα ἀπαρχὴ οἴνου, σίτου ἀπαρχὴ αὐτῶν ὅσα ἂν δῶσι τῷ Κυρίῳ, σοὶ δέδωκα αὐτά.
\vs{13}Τὰ πρωτογεννήματα πάντα ὅσα ἐν τῇ γῇ αὐτῶν, ὅσα ἂν ἐνέγκωσι Κυρίῳ, σοὶ ἔσται· πᾶς καθαρὸς ἐν τῷ οἴκῳ σου ἔδεται αὐτά.

\vs{14}Πᾶν ἀνατεθεματισμένον ἐν υἱοῖς Ἰσραὴλ, σοὶ ἔσται.
\vs{15}Καὶ πᾶν διανοῖγον μήτραν ἀπὸ πάσης σαρκὸς, ὅσα προσφέρουσι Κυρίῳ ἀπὸ ἀνθρώπου ἕως κτήνους, σοὶ ἔσται· ἀλλʼ ἢ λύτροις λυτρωθήσεται τὰ πρωτότοκα τῶν ἀνθρώπων, καὶ τὰ πρωτότοκα τῶν κτηνῶν τῶν ἀκαθάρτων λυτρώσῃ.
\vs{16}Καὶ ἡ λύτρωσις αὐτοῦ, ἀπὸ μηνιαίου· ἡ συντίμησις πέντε σίκλων, κατὰ τὸν σίκλον τὸν ἅγιον εἴκοσι ὀβολοί εἰσι.
\vs{17}Πλὴν πρωτότοκα μόσχων καὶ πρωτότοκα προβάτων, καὶ πρωτότοκα αἰγῶν οὐ λυτρώσῃ· ἅγιά ἐστι· καὶ τὸ αἷμα αὐτῶν προσχεεῖς πρὸς τὸ θυσιαστήριον, καὶ τὸ στέαρ ἀνοίσεις κάρπωμα εἰς ὀσμὴν εὐωδίας Κυρίῳ.

\vs{18}Καὶ τὰ κρέα ἔσται σοι, καθὰ καὶ τὸ στηθύνιον τοῦ ἐπιθέματος· καὶ κατὰ τὸν βραχίονα τὸν δεξιὸν, σοὶ ἔσται.
\vs{19}Πᾶν ἀφαίρεμα τῶν ἁγίων, ὅσα ἐὰν ἀφέλωσιν οἱ υἱοὶ Ἰσραὴλ Κυρίῳ, δέδωκά σοι καὶ τοῖς υἱοῖς σου καὶ ταῖς θυγατράσι σου μετὰ σοῦ, νόμιμον αἰώνιον· διαθήκη ἁλὸς αἰωνίου ἔστιν ἔναντι Κυρίου, σοὶ καὶ τῷ σπέρματί σου μετὰ σέ.

\vs{20}Καὶ ἐλάλησε Κύριος πρὸς Ἀαρὼν, ἐν τῇ γῇ αὐτῶν οὐ κληρονομήσεις, καὶ μερὶς οὐκ ἔσται σοι ἐν αὐτοῖς, ὅτι ἐγὼ μερίς σου καὶ κληρονομία σου ἐν μέσῳ τῶν υἱῶν Ἰσραήλ.

\vs{21}Καὶ τοῖς υἱοῖς Λευὶ ἰδοὺ δέδωκα πᾶν ἐπιδέκατον ἐν Ἰσραὴλ ἐν κλήρῳ ἀντὶ τῶν λειτουργιῶν αὐτῶν, ὅσα αὐτοὶ λειτουργοῦσι λειτουργίαν ἐν τῇ σκηνῇ τοῦ μαρτυρίου.
\vs{22}Καὶ οὐ προσελεύσονται ἔτι οἱ υἱοὶ Ἰσραὴλ εἰς τὴν σκηνὴν τοῦ μαρτυρίου λαβεῖν ἁμαρτίαν θανατηφόρον.
\vs{23}Καὶ λειτουργήσει ὁ Λενίτης αὐτὸς τὴν λειτουργίαν τῆς σκηνῆς τοῦ μαρτυρίου· καὶ αὐτοὶ λήψονται τὰ ἁμαρτήματα αὐτῶν, νόμιμον αἰώνιον εἰς τὰς γενεὰς αὐτῶν· καὶ ἐν μέσῳ υἱῶν Ἰσραὴλ οὐ κληρονομήσουσι κληρονομίαν.
\vs{24}Ὅτι τὰ ἐπιδέκατα τῶν υἱῶν Ἰσραὴλ ὅσα ἐὰν ἀφορίσωσι Κυρίῳ, ἀφαίρεμα δέδωκα τοῖς Λευίταις ἐν κλήρῳ· διὰ τοῦτο εἴρηκα αὐτοῖς, ἐν μέσῳ υἱῶν Ἰσραὴλ οὐ κληρονομήσουσι κλῆρον.

\vs{25}Καὶ ἐλάλησε Κύριος πρὸς Μωυσῆν, λέγων,
\vs{26}καὶ τοῖς Λενίταις λαλήσεις, καὶ ἐρεῖς πρὸς αὐτοὺς, ἐὰν λάβητε παρὰ τῶν υἱῶν Ἰσραὴλ τὸ ἐπιδέκατον, ὃ δέδωκα ὑμῖν παρʼ αὐτῶν ἐν κλήρῳ, καὶ ἀφελεῖτε ὑμεῖς ἀπʼ αὐτοῦ ἀφαίρεμα Κυρίῳ, ἐπιδέκατον ἀπὸ τοῦ ἐπιδεκάτου.
\vs{27}Καὶ λογισθήσεται ὑμῖν τὰ ἀφαιρέματα ὑμῶν ὡς σῖτος ἀπὸ ἅλω, καὶ ἀφαίρεμα ἀπὸ ληνοῦ.
\vs{28}Οὕτως ἀφελεῖτε αὐτοὺς καὶ ὑμεῖς ἀπὸ πάντων τῶν ἀφαιρεμάτων Κυρίου ἀπὸ πάντων τῶν ἐπιδεκάτων ὑμῶν, ὅσα ἐὰν λάβητε παρὰ τῶν υἱῶν Ἰσραήλ· καὶ δώσετε ἀπʼ αὐτῶν ἀφαίρεμα Κυρίῳ Ἀαρὼν τῷ ἱερεῖ.
\vs{29}Ἀπὸ πάντων τῶν δομάτων ὑμῶν ἀφελεῖτε ἀφαιρεμα Κυρίῳ, ἢ ἀπὸ πάντων τῶν ἀπαρχῶν τὸ ἡγιασμένον ἀπʼ αὐτοῦ.
\vs{30}Καὶ ἐρεῖς πρὸς αὐτοὺς, ὅταν ἀφαιρῆτε τὴν ἀπαρχὴν ἀπʼ αὐτοῦ, καὶ λογισθήσεται τοῖς Λευίταις ὡς γέννημα ἀπὸ ἅλω, καὶ ὡς γέννημα ἀπὸ ληνοῦ.
\vs{31}Καὶ ἔδεσθε αὐτὸ ἐν παντὶ τόπῳ ὑμεῖς καὶ οἱ οἶκοι ὑμῶν, ὅτι μισθὸς οὗτος ὑμῖν ἐστιν ἀντὶ τῶν λειτουργιῶν ὑμῶν τῶν ἐν τῇ σκηνῇ τοῦ μαρτυρίου.
\vs{32}Καὶ οὐ λήψεσθε διʼ αὐτὸ ἁμαρτίαν, ὅτι ἂν ἀφαιρῆτε τὴν ἀπαρχὴν ἀπʼ αὐτοῦ· καὶ τὰ ἅγια τῶν υἱῶν Ἰσραὴλ οὐ βεβηλώσετε, ἵνα μὴ ἀποθάνητε.

\ch{19}
Καὶ ἐλάλησε Κύριος πρὸς Μωυσῆν καὶ Ἀαρὼν, λέγων,
\vs{2}αὕτη ἡ διαστολὴ τοῦ νόμου, ὅσα συνέταξε Κύριος, λέγων, λάλησον τοῖς υἱοῖς Ἰσραήλ· καὶ λαβέτωσαν πρὸς σὲ δάμαλιν πυῤῥὰν ἄμωμου, ἥτις οὐκ ἔχει ἐν αὐτῇ μῶμον, καὶ ᾗ οὐκ ἐπεβλήθη ἐπʼ αὐτὴν ζυγός.
\vs{3}Καὶ δώσεις αὐτὴν πρὸς Ἐλεάζαρ τὸν ἱερέα· καὶ ἐξάξουσιν αὐτὴν ἔξω τῆς παρεμβολῆς εἰς τόπον καθαρὸν, καὶ σφάξουσιν αὐτὴν ἐνώπιον αὐτοῦ.
\vs{4}Καὶ λήψεται Ἐλεάζαρ ἀπὸ τοῦ αἵματος αὐτῆς, καὶ ῥανεῖ ἀπέναντι τοῦ προσώπου τῆς σκηνῆς τοῦ μαρτυρίου ἀπὸ τοῦ αἵματος αὐτῆς ἑπτάκις.
\vs{5}Καὶ κατακαύσουσιν αὐτὴν ἐναντίον αὐτοῦ· καὶ τὸ δέρμα καὶ τὰ κρέα αὐτῆς καὶ τὸ αἷμα αὐτῆς σὺν τῇ κόπρῳ αὐτῆς κατακαυθήσεται.
\vs{6}Καὶ λήψεται ὁ ἱερεὺς ξύλον κέδρινον καὶ ὕσσωπον καὶ κόκκινον, καὶ ἐμβαλοῦσιν εἰς μέσον τοῦ κατακαύματος τῆς δαμάλεως.

\vs{7}Καὶ πλυνεῖ τὰ ἱμάτια αὐτοῦ ὁ ἱερεὺς, καὶ λούσεται τὸ σῶμα αὐτοῦ ὕδατι, καὶ μετὰ ταῦτα εἰσελεύσεται εἰς τὴν παρεμβολὴν, καὶ ἀκάθαρτος ἔσται ὁ ἱερεὺς ἕως ἑσπέρας.
\vs{8}Καὶ ὁ κατακαίων αὐτὴν πλυνεῖ τὰ ἱμάτια αὐτοῦ, καὶ λούσεται τὸ σῶμα αὐτοῦ, καὶ ἀκάθαρτος ἔσται ἕως ἑσπέρας.
\vs{9}Καὶ συνάξει ἄνθρωπος καθαρὸς τὴν σποδὸν τῆς δαμάλεως, καὶ ἀποθήσει ἔξω τῆς παρεμβολῆς εἰς τόπον καθαρόν· καὶ ἔσται τῇ συναγωγῇ υἱῶν Ἰσραὴλ εἰς διατήρησιν· ὕδωρ ῥαντισμοῦ ἅγνισμά ἐστι.
\vs{10}Καὶ ὁ συνάγων τὴν σποδιὰν τῆς δαμάλεως, πλυνεῖ τὰ ἱμάτια αὐτου, καὶ ἀκάθαρτος ἔσται ἕως ἑσπέρας· καὶ ἔσται τοῖς υἱοῖς Ἰσραὴλ καὶ τοῖς προσηλύτοις προσκειμένοις νόμιμον αἰώνιον.

\vs{11}Ὁ ἁπτόμενος τοῦ τεθνηκότος πάσης ψυχῆς ἀνθρώπου, ἀκάθαρτος ἔσται ἑπτὰ ἡμέρας.
\vs{12}Οὗτος ἁγνισθήσεται τῇ ἡμέρᾳ τῇ τρίτῃ καὶ τῇ ἡμέρᾳ τῇ ἑβδόμῃ, καὶ καθαρὸς ἔσται· ἐὰν δὲ μὴ ἀφαγνισθῇ τῇ ἡμέρᾳ τῇ τρίτῃ καὶ τῇ ἡμέρᾳ τῇ ἑβδόμῃ, οὐ καθαρὸς ἔσται.
\vs{13}Πᾶς ὁ ἁπτόμενος τοῦ τεθνηκότος ἀπὸ ψυχῆς ἀνθρώπου, ἐὰν ἀποθάνῃ, καὶ μὴ ἀφαγνισθῇ, τὴν σκηνὴν Κυρίου ἐμίανεν· ἐκτριβήσεται ἡ ψυχὴ ἐκείνη ἐξ Ἰσραὴλ, ὅτι ὕδωρ ῥαντισμοῦ οὐ περιεῤῥαντίσθη ἐπʼ αὐτόν· ἀκάθαρτός ἐστιν· ἔτι ἡ ἀκαθαρσία αὐτοῦ ἐν αὐτῷ ἐστι.
\vs{14}Καὶ οὗτος ὁ νόμος· ἄνθρωπος ἐὰν ἀποθάνῃ ἐν οἰκίᾳ, πᾶς ὁ εἰσπορευόμενος εἰς τὴν οἰκίαν, καὶ ὅσα ἐστὶν ἐν τῇ οἰκίᾳ, ἀκάθαρτα ἔσται ἑπτὰ ἡμέρας.
\vs{15}Καὶ πᾶν σκεῦος ἀνεῳγμένον ὅσα οὐχὶ δεσμὸν καταδέδεται ἐπʼ αὐτῷ, ἀκάθαρτά ἐστι.
\vs{16}Καὶ πᾶς ὃς ἂν ἅψηται ἐπὶ προσώπου τοῦ πεδίου τραυματίου ἢ νεκροῦ ἢ ὀστέου ἀνθρωπίνου ἢ μνήματος, ἑπτὰ ἡμέρας ἀκάθαρτος ἔσται.

\vs{17}Καὶ λήψονται τῷ ἀκαθάρτῳ ἀπὸ τῆς σποδιᾶς τῆς κατακεκαυμένης τοῦ ἁγνισμοῦ, καὶ κέχεοῦσιν ἐπʼ αὐτὴν ὕδωρ ζῶν εἰς σκεῦος.
\vs{18}Καὶ λήψεται ὕσσωπον, καὶ βάψει εἰς τὸ ὕδωρ ἀνὴρ καθαρὸς, καὶ περιῤῥανεῖ ἐπὶ τὸν οἶκον, καὶ ἐπὶ τὰ σκεύη, καὶ ἐπὶ τὰς ψυχὰς, ὅσαι ἂν ὦσιν ἐκεῖ, καὶ ἐπὶ τὸν ἡμμένον τοῦ ὀστέου τοῦ ἀνθρωπίνου, ἢ τοῦ τραυματίου, ἢ τοῦ τεθνηκότος, ἢ τοῦ μνήματος.
\vs{19}Καὶ περιῤῥανεῖ ὁ καθαρὸς ἐπὶ τὸν ἀκάθαρτον ἐν τῇ ἡμέρᾳ τῇ τρίτῃ καὶ ἐν τῇ ἡμέρᾳ τῇ ἑβδόμῃ, καὶ ἀφαγνισθήσεται τῇ ἡμέρᾳ τῇ ἑβδόμῃ· καὶ πλυνεῖ τὰ ἱμάτια αὐτοῦ, καὶ λούσεται ὕδατι, καὶ ἀκάθαρτος ἔσται ἕως ἑσπέρας.
\vs{20}Καὶ ἄνθρωπος ὃς ἂν μιανθῇ, καὶ μὴ ἀφαγνισθῇ, ἐξολοθρευθήσεται ἡ ψυχὴ ἐκείνη ἐκ μέσου τῆς συναγωγῆς, ὅτι τὰ ἅγια Κυρίου ἐμίανεν, ὅτι ὕδωρ ῥαντισμοῦ οὐ περιεῤῥαντίσθη ἐπʼ αὐτόν· ἀκάθαρτός ἐστι.
\vs{21}Καὶ ἔσται ὑμῖν νόμιμον αἰώνιον· καὶ ὁ περιῤῥαίνων ὕδωρ ῥαντισμοῦ, πλυνεῖ τὰ ἱμάτια αὐτοῦ· καὶ ὁ ἁπτόμενος τοῦ ὕδατος τοῦ ῥαντισμοῦ, ἀκάθαρτος ἔσται ἕως ἑσπέρας.
\vs{22}Καὶ παντὸς οὗ ἐὰν ἅψηται αὐτοῦ ὁ ἀκάθαρτος, ἀκάθαρτον ἔσται· καὶ ψυχὴ ἡ ἁπτομένη, ἀκάθαρτος ἔσται ἕως ἑσπέρας.

\ch{20}
Καὶ ἦλθον οἱ υἱοὶ Ἰσραὴλ, πᾶσα ἡ συναγωγὴ, εἰς τὴν ἔρημον Σὶν, ἐν τῷ μηνὶ τῷ πρώτῳ, καὶ κατέμεινεν ὁ λαὸς ἐν Κάδης· καὶ ἐτελεύτησεν ἐκεῖ Μαριὰμ, καὶ ἐτάφη ἐκεῖ.
\vs{2}Καὶ οὐκ ἦν ὕδωρ τῇ συναγωγῇ· καὶ ἠθροίσθησαν ἐπὶ Μωυσῆν καὶ Ἀαρών.
\vs{3}Καὶ ἐλοιδορεῖτο ὁ λαὸς πρὸς Μωυσῆν, λέγοντες, ὄφελον ἀπεθάνομεν ἐν τῇ ἀπωλείᾳ τῶν ἀδελφῶν ἡμῶν ἔναντι Κυρίου.
\vs{4}Καὶ ἱνατί ἀνηγάγετε τὴν συναγωγὴν Κυρίου εἰς τὴν ἔρημον ταύτην ἀποκτεῖναι ἡμᾶς, καὶ τα κτήνη ἡμῶν;
\vs{5}Καὶ ἱνατί τοῦτο; ἀνηγάγετε ἡμᾶς ἐξ Αἰγύπτου, παραγενέσθαι εἰς τὸν τόπον τὸν πονηρὸν τοῦτον· τόπος οὗ οὐ σπείρεται, οὐδὲ συκαῖ, οὐδὲ ἄμπελοι, οὔτε ῥοαὶ, οὔτε ὕδωρ ἐστὶ πιεῖν.

\vs{6}Καὶ ἦλθε Μωυσῆς καὶ Ἀαρὼν ἀπὸ προσώπου τῆς συναγωγῆς ἐπὶ τὴν θύραν τῆς σκηνῆς τοῦ μαρτυρίου, καὶ ἔπεσον ἐπὶ πρόσωπον· καὶ ὤφθη ἡ δόξα Κυρίου πρὸς αὐτοὺς.
\vs{7}Καὶ ἐλάλησε Κύριος πρὸς Μωυσῆν, λέγων,
\vs{8}λάβε τὴν ῥάβδον σου, καὶ ἐκκλησίασον τὴν συναγωγὴν σὺ καὶ Ἀαρὼν ὁ ἀδελφός σου, καὶ λαλήσατε πρὸς τὴν πέτραν ἐναντίον αὐτῶν, καὶ δώσει τὰ ὕδατα αὐτῆς, καὶ ἐξοίσετε αὐτοῖς ὕδωρ ἐκ τῆς πέτρας, καὶ ποτιεῖτε τὴν συναγωγὴν, καὶ τὰ κτήνη αὐτῶν.
\vs{9}Καὶ ἔλαβε Μωυσῆς τὴν ῥάβδον τὴν ἀπέναντι Κυρίου, καθὰ συνέταξε Κύριος.
\vs{10}Καὶ ἐξεκκλησίασε Μωυσῆς καὶ Ἀαρὼν τὴν συναγωγὴν ἀπέναντι τῆς πέτρας, καὶ εἶπε πρὸς αὐτοὺς, ἀκούσατέ μου οἱ ἀπειθεῖς· μὴ ἐκ τῆς πέτρας ταύτης ἐξάξομεν ὑμῖν ὕδωρ;
\vs{11}Καὶ ἐπάρας Μωυσῆς τὴν χεῖρα αὐτοῦ, ἐπάταξε τὴν πέτραν τῇ ῥάβδῳ δίς· καὶ ἐξῆλθεν ὕδωρ πολὺ, καὶ ἔπιεν ἡ συναγωγὴ, καὶ τὰ κτήνη αὐτῶν.
\vs{12}Καὶ εἶπε Κύριος πρὸς Μωυσῆν καὶ Ἀαρὼν, ὅτι οὐκ ἐπιστεύσατε ἁγιάσαι με ἐναντίον τῶν υἱῶν Ἰσραὴλ, διὰ τοῦτο οὐκ εἰσάξετε ὑμεῖς τὴν συναγωγὴν ταύτην εἰς τὴν γῆν ἣν δέδωκα αὐτοῖς.
\vs{13}Τοῦτο τὸ ὕδωρ Ἀντιλογίας, ὅτι ἐλοιδορήθησαν οἱ υἱοὶ Ἰσραὴλ ἔναντι Κυρίου, καὶ ἡγιάσθη ἐν αὐτοῖς.

\vs{14}Καὶ ἀπέστειλε Μωυσῆς ἀγγέλους ἐκ Κάδης πρὸς βασιλέα Ἐδὼμ, λέγων, τάδε λέγει ὁ ἀδελφός σου Ἰσραήλ· σὺ ἐπίστῃ πάντα τὸν μόχθον τὸν εὑρόντα ἡμᾶς.
\vs{15}Καὶ κατέβησαν οἱ πατέρες ἡμῶν εἰς Αἴγυπτον, καὶ παρῳκήσαμεν ἐν Αἰγύπτῳ ἡμέρας πλείους, καὶ ἐκάκωσαν ἡμᾶς οἱ Αἰγύπτιοι καὶ τοὺς πατέρας ἡμῶν.
\vs{16}Καὶ ἀνεβοήσαμεν πρὸς Κύριον, καὶ εἰσήκουσε Κύριος τῆς φωνῆς ἡμῶν, καὶ ἀποστείλας ἄγγελον, ἐξήγαγεν ἡμᾶς ἐξ Αἰγύπτου· καὶ νῦν ἐσμὲν ἐν Κάδης πόλει, ἐκ μέρους τῶν ὁρίων σου.
\vs{17}Παρελευσόμεθα διὰ τῆς γῆς σου· οὐ διελευσόμεθα διʼ ἀγρῶν, οὐδὲ διʼ ἀμπελώνων, οὐδὲ πιόμεθα ὕδωρ ἐκ λάκκου σου· ὁδῷ βασιλικῇ πορευσόμεθα· οὐκ ἐκκλινοῦμεν δεξιὰ οὐδὲ εὐώνυμα, ἕως ἂν παρέλθωμεν τὰ ὅριά σου.
\vs{18}Καὶ εἶπε πρὸς αὐτὸν Ἐδὼμ, Οὐ διελεύσῃ διʼ ἐμοῦ· εἰ δὲ μὴ, ἐν πολέμῳ ἐξελεύσομαι εἰς συνάντησίν σοι.
\vs{19}Καὶ λέγουσιν αὐτῷ οἱ υἱοὶ Ἰσραὴλ, παρὰ τὸ ὄρος παρελευσόμεθα· ἐὰν δὲ τοῦ ὕδατός σου πίωμεν ἐγώ τε καὶ τὰ κτήνη μου, δώσω τιμήν σοι· ἀλλὰ τὸ πρᾶγμα οὐδέν ἐστι· παρὰ τὸ ὄρος παρελευσόμεθα.
\vs{20}Ὁ δὲ εἶπεν, οὐ διελεύσῃ διʼ ἐμοῦ· καὶ ἐξῆλθεν Ἐδὼμ εἰς συνάντησιν αὐτῷ ἐν ὄχλῳ βαρεῖ, καὶ ἐν χειρὶ ἰσχυρᾷ.
\vs{21}Καὶ οὐκ ἠθέλησεν Ἐδὼμ δοῦναι τῷ Ἰσραὴλ παρελθεῖν διὰ τῶν ὁρίων αὐτοῦ· καὶ ἐξέκλινεν Ἰσραὴλ ἀπʼ αὐτοῦ.
\vs{22}Καὶ ἀπῇραν ἐκ Καδης· καὶ παρεγένοντο οἱ υἱοὶ Ἰσραὴλ πᾶσα ἡ συναγωγὴ εἰς Ὢρ τὸ ὄρος.

\vs{23}Καὶ εἶπε Κύριος πρὸς Μωυσῆν καὶ Ἀαρὼν ἐν Ὢρ τῷ ὄρει ἐπὶ τῶν ὁρίων τῆς γῆς Ἐδὼμ, λέγων,
\vs{24}προστεθήτω Ἀαρὼν πρὸς τὸν λαὸν αὐτοῦ, ὅτι οὐ μὴ εἰσέλθητε εἰς τὴν γῆν ἣν δέδωκα τοῖς υἱοῖς Ἰσραήλ, διότι παρωξύνατέ με ἐπὶ τοῦ ὕδατος τῆς λοιδορίας.
\vs{25}Λάβε τὸν Ἀαρὼν, καὶ Ἐλεάζαρ τὸν υἱὸν αὐτοῦ, καὶ ἀναβίβασον αὐτοὺς εἰς Ὢρ τὸ ὄρος, ἔναντι πάσης τῆς συναγωγῆς,
\vs{26}καὶ ἔκδυσον Ἀαρὼν τὴν στολὴν αὐτοῦ, καὶ ἔνδυσον Ἐλεάζαρ τὸν υἱὸν αὐτοῦ· καὶ Ἀαρὼν προστεθεὶς ἀποθανέτω ἐκεῖ.
\vs{27}Καὶ ἐποίησε Μωυσῆς καθὰ συνέταξε Κύριος αὐτῷ, καὶ ἀνεβίβασεν αὐτὸν εἰς Ὢρ τὸ ὄρος, ἐναντίον πάσης τῆς συναγωγῆς,
\vs{28}καὶ ἐξέδυσε τὸν Ἀαρὼν τὰ ἱμάτια αὐτοῦ, καὶ ἐνέδυσεν αὐτὰ Ἐλεάζαρ τὸν υἱὸν αὐτοῦ· καὶ ἀπέθανεν Ἀαρὼν ἐπὶ τῆς κορυφῆς τοῦ ὄρους· καὶ κατέβη Μωυσῆς καὶ Ἐλεάζαρ ἐκ τοῦ ὄρους.
\vs{29}Καὶ εἶδε πᾶσα ἡ συναγωγὴ ὅτι ἀπελύθη Ἀαρὼν, καὶ ἔκλαυσαν τὸν Ἀαρὼν τριάκοντα ἡμέρας πᾶς οἶκος Ἰσραήλ.

\ch{21}
Καὶ ἤκουσεν ὁ Χανανεὶς βασιλεὺς Ἀρὰδ ὁ κατοικῶν κατὰ τὴν ἔρημον, ὅτι ἦλθεν Ἰσραὴλ ὁδὸν Ἀθαρεὶν, καὶ ἐπολέμησε πρὸς Ἰσραὴλ, καὶ κατεπροενόμευσεν ἐξ αὐτῶν αἰχμαλωσίαν.
\vs{2}Καὶ ηὔξατο Ἰσραὴλ εὐχὴν Κυρίῳ, καὶ εἶπεν, ἐάν μοι παραδῷς τὸν λαὸν τοῦτον ὑποχείριον, ἀναθεματιῶ αὐτὸν καὶ τὰς πόλεις αὐτοῦ.
\vs{3}Καὶ εἰσήκουσε Κύριος τῆς φωνῆς Ἰσραὴλ, καὶ παρέδωκε τὸν Χανανεὶν ὑποχείριον αὐτοῦ· καὶ ἀνεθεμάτισεν αὐτὸν, καὶ τὰς πόλεις αὐτοῦ· καὶ ἐπεκάλεσαν τὸ ὄνομα τοῦ τόπου ἐκείνου, Ἀνάθεμα.

\vs{4}Καὶ ἀπάραντες ἐξ Ὢρ τοῦ ὄρους ὁδὸν ἐπὶ θάλασσαν ἐρυθρᾶν, περιεκύκλωσαν γῆν Ἐδώμ· καὶ ὠλιγοψύχησεν ὁ λαὸς ἐν τῇ ὁδῷ.
\vs{5}Καὶ κατελάλει ὁ λαὸς πρὸς τὸν Θεὸν καὶ κατὰ Μωυσῆ, λέγοντες, ἱνατί τοῦτο; ἐξήγαγες ἡμᾶς ἐξ Αἰγύπτου ἀποκτεῖναι ἐν τῇ ἐρήμῳ; ὅτι οὐκ ἔστιν ἄρτος, οὐδὲ ὕδωρ· ἡ δὲ ψυχὴ ἡμῶν προσώχθισεν ἐν τῷ ἄρτῳ τῷ διακένῳ τούτῳ.
\vs{6}Καὶ ἀπέστειλε Κύριος εἰς τὸν λαὸν τοὺς ὄφεις τοὺς θανατοῦντας, καὶ ἔδακνον τὸν λαόν, καὶ ἀπέθανε λαὸς πολὺς τῶν υἱῶν Ἰσραήλ.
\vs{7}Καὶ παραγενόμενος ὁ λαὸς πρὸς Μωυσῆν, ἔλεγον, ὅτι ἡμάρτομεν, ὅτι κατελαλήσαμεν κατὰ τοῦ Κυρίου, καὶ κατὰ σοῦ· εὔξαι οὖν πρὸς Κύριον, καὶ ἀφελέτω ἀφʼ ἡμῶν τὸν ὄφιν.
\vs{8}Καὶ ηὔξατο Μωυσῆς πρὸς Κύριον περὶ τοῦ λαοῦ· καὶ εἰπε Κύριος πρὸς Μωυσῆν, ποίησον σεαυτῷ ὄφιν, καὶ θὲς αὐτὸν ἐπὶ σημείου, καὶ ἔσται ἐὰν δάκῃ ὄφις ἄνθρωπον, πᾶς ὁ δεδηγμένος ἰδὼν αὐτὸν ζήσεται.
\vs{9}Καὶ ἐποίησε Μωυσῆς ὄφιν χαλκοῦν, καὶ ἔστησεν αὐτὸν ἐπὶ σημείου· καὶ ἐγένετο ὅταν ἔδακνεν ὄφις ἄνθρωπον, καὶ ἐπέβλεψεν ἐπὶ τὸν ὄφιν τὸν χαλκοῦν, καὶ ἔζη.

\vs{10}Καὶ ἀπῇραν οἱ υἱοὶ Ἰσραὴλ, καὶ παρενέβαλον ἐν Ὠβώθ.
\vs{11}Καὶ ἐξάραντες ἐξ Ὠβὼθ, καὶ παρενέβαλον ἐν Ἀχαλγαὶ ἐκ τοῦ πέραν ἐν τῇ ἐρήμῳ, ἥ ἐστι κατὰ πρόσωπον Μωὰβ, κατʼ ἀνατολὰς ἡλίου.
\vs{12}Καὶ ἐκεῖθεν ἀπῇραν, καὶ παρενέβαλον εἰς φάραγγα Ζαρέδ.
\vs{13}Καὶ ἐκεῖθεν ἀπάραντες παρενέβαλον εἰς τὸ πέραν Ἀρνῶν ἐν τῇ ἐρήμῳ, τὸ ἐξέχον ἀπὸ τῶν ὁρίων τῶν Ἀμοῤῥαίων· ἔστι γὰρ Ἀρνῶν ὅρια Μωὰβ, ἀναμέσον Μωὰβ καὶ ἀναμέσον τοῦ Ἀμοῤῥαίου.
\vs{14}Διὰ τοῦτο λέγεται ἐν βιβλίῳ, πόλεμος τοῦ Κυρίου τὴν Ζωὸβ ἐφλόγισε, καὶ τοὺς χιμάῤῥους Ἀρνῶν.
\vs{15}Καὶ τοὺς χιμάῤῥους κατέστησε κατοικίσαι Ἤρ· καὶ πρόσκειται τοῖς ὁρίοις Μωάβ.

\vs{16}Καὶ ἐκεῖθεν τὸ φρέαρ· τοῦτο φρέαρ, ὃ εἶπε Κύριος πρὸς Μωυσῆν, συνάγαγε τὸν λαὸν, καὶ δώσω αὐτοῖς ὕδωρ πιεῖν.
\vs{17}Τότε ᾖσεν Ἰσραὴλ τὸ ἆσμα τοῦτο ἐπὶ τοῦ φρέατος, ἐξάρχετε αὐτῷ φρέαρ,
\vs{18}ὤρυξαν αὐτὸ ἄρχοντες, ἐξελατόμησαν αὐτὸ βασιλεῖς ἐθνῶν ἐν τῇ βασιλείᾳ αὐτῶν, ἐν τῷ κυριεῦσαι αὐτῶν·
\vs{19}καὶ ἀπὸ φρέατος εἰς Μανθαναεὶν, καὶ ἀπὸ Μανθαναεὶν εἰς Νααλιὴλ, καὶ ἀπὸ Νααλιὴλ εἰς Βαμὼθ, καὶ ἀπὸ Βαμὼθ εἰς Ἰανὴν, ἥ ἐστιν ἐν τῷ πεδίῳ Μωὰβ, ἀπὸ κορυφῆς τοῦ λελαξευμένου, τὸ βλέπον κατὰ πρόσωπον τῆς ἐρήμου.

\vs{20}Καὶ ἀπέστειλε Μωυσῆς πρέαβεις πρὸς Σηὼν βασιλέα Ἀμοῤῥαίων, λόγοις εἰρηνικοῖς, λέγων,
\vs{21}παρελευσόμεθα διὰ τῆς γῆς σου, τῇ ὁδῷ πορευσόμεθα· οὐκ ἐκκλινοῦμεν οὔτε εἰς ἀγρὸν, οὔτε εἰς ἀμπελῶνα·
\vs{22}Οὐ πιόμεθα ὕδωρ ἐκ φρέατός σου· ὁδῷ βασιλικῇ παρευσόμεθα, ἕως παρέλθωμεν τὰ ὅριά σου.
\vs{23}Καὶ οὐκ ἔδωκε Σηὼν τῷ Ἰσραὴλ παρελθεῖν διὰ τῶν ὁρίων αὐτοῦ· καὶ συνήγαγε Σηὼν πάντα τὸν λαὸν αὐτοῦ, καὶ ἐξῆλθε παρατάξασθαι τῷ Ἰσραὴλ εἰς τὴν ἔρημον· καὶ ἦλθεν εἰς Ἰασσὰ, καὶ παρετάξατο τῷ Ἰσραήλ.
\vs{24}Καὶ ἐπάταξεν αὐτὸν Ἰσραὴλ φόνῳ μαχαίρης, καὶ κατεκυρίευσαν τῆς γῆς αὐτοῦ, ἀπὸ Ἀρνῶν ἕως Ἰαβὸκ, ἕως υἱῶν Ἀμμὰν, ὅτι Ἰαζὴρ ὅρια υἱῶν Ἀμμάν ἐστι.
\vs{25}Καὶ ἔλαβεν Ἰσραὴλ πάσας τὰς πόλεις ταύτας, καὶ κατῷκησεν Ἰσραὴλ ἐν πάσαις ταῖς πόλεσι τῶν Ἀμοῤῥαίων, ἐν Ἑσεβὼν, καὶ ἐν πάσαις ταῖς συγκυρούσαις αὐτῇ.
\vs{26}Ἔστι γὰρ Ἐσεβὼν, πόλις Σηὼν τοῦ βασιλέως τῶν Ἀμοῤῥαίων ἐστίν· καὶ οὗτος ἐπολέμησε βασιλέα Μωὰβ τὸ πρότερον· καὶ ἔλαβον πᾶσαν τὴν γῆν αὐτοῦ, ἀπὸ Ἀροὴρ ἕως Ἀρνῶν.
\vs{27}Διὰ τοῦτο ἐροῦσιν οἱ αἰνιγματισταὶ, ἔλθετε εἰς Ἐσεβὼν, ἵνα οἰκοδομηθῇ καὶ κατασκευασθῇ πόλις Σηών·
\vs{28}ὅτι πῦρ ἐξῆλθεν ἐξ Ἑσεβὼν, φλὸξ ἐκ πόλεως Σηὼν, καὶ κατέφαγεν ἕως Μωὰβ, καὶ κατέπιε στήλας Ἀρνῶν.
\vs{29}Οὐαί σοι Μωὰβ, ἀπώλου λαὸς Χαμώς· ἀπεδόθησαν οἱ υἱοὶ αὐτῶν διασώζεσθαι, καὶ αἱ θυγατέρες αὐτῶν αἰχμάλωτοι τῷ βασιλεῖ τῶν Ἀμοῤῥαίων Σηὼν,
\vs{30}καὶ τὸ σπέρμα αὐτῶν ἀπολεῖται, Ἑσεβὼν ἕως Δαιβών· καὶ αἱ γυναῖκες ἔτι προσεξέκαυσαν πῦρ ἐπὶ Μωάβ.

\vs{31}Κατῴκησε δὲ Ἰσραὴλ ἐν πάσαις ταῖς πόλεσι τῶν Ἀμοῤῥαίων.
\vs{32}Καὶ ἀπέστειλε Μωυσῆς κατασκέψασθαι τὴν Ἰαζήρ· καὶ κατελάβοντο αὐτὴν, καὶ τὰς κώμας αὐτῆς, καὶ ἐξέβαλον τὸν Ἀμοῤῥαῖον τὸν κατοικοῦντα ἐκεῖ.
\vs{33}Καὶ ἐπιστρέψαντες, ἀνέβησαν ὁδὸν τὴν εἰς Βασάν· καὶ ἐξῆλθεν Ὢγ βασιλεὺς τῆς Βασὰν εἰς συνάντησιν αὐτοῖς, καὶ πᾶς ὁ λαὸς αὐτοῦ εἰς πόλεμον εἰς Ἐδραείν.
\vs{34}Καὶ εἶπε Κύριος πρὸς Μωυσῆν, μὴ φοβηθῇς αὐτὸν, ὅτι εἰς τὰς χεῖράς σου παραδέδωκα αὐτὸν, καὶ πάντα τὸν λαὸν αὐτοῦ, καὶ πᾶσαν τὴν γῆν αὐτοῦ· καὶ ποιήσεις αὐτῷ καθὼς ἐποίησας τῷ Σηὼν βασιλεῖ τῶν Ἀμοῤῥαίων, ὃς κατῴκει ἐν Ἑσεβών.
\vs{35}Καὶ ἐπάταξεν αὐτὸν καὶ τοὺς υἱοὺς αὐτοῦ, καὶ πάντα τὸν λαὸν αὐτοῦ, ἕως τοῦ μὴ καταλιπεῖν αὐτοῦ ζωγρείαν· καὶ ἐκληρονόμησαν τὴν γῆν αὐτον.

\ch{22}
Καὶ ἀπάραντες οἱ υἱοὶ Ἰσραὴλ παρενέβαλον ἐπὶ δυσμῶν Μωὰβ παρὰ τὸν Ἰορδάνην κατὰ Ἰεριχώ.
\vs{2}Καὶ ἰδὼν Βαλὰκ υἱὸς Σεπφὼρ πάντα ὅσα ἐποίησεν Ἰσραὴλ τῷ Ἀμοῤῥαίῳ,
\vs{3}καὶ ἐφοβήθη Μωὰβ τὸν λαὸν σφόδρα ὅτι πολλοὶ ἦσαν· καὶ προσώχθισε Μωὰβ ἀπὸ προσώπου υἱῶν Ἰσραήλ.
\vs{4}Καὶ εἶπε Μωὰβ τῇ γερουσίᾳ Μαδιὰμ, νῦν ἐκλείξει ἡ συναγωγὴ αὕτη πάντας τοὺς κύκλῳ ἡμῶν, ὡσεὶ ἐκλείξαι ὁ μόσχος τὰ χλωρὰ ἐκ τοῦ πεδίου· καὶ Βαλὰκ υἱὸς Σεπφὼρ βασιλεὺς Μωὰβ ἦν κατὰ τὸν καιρὸν ἐκεῖνον.
\vs{5}Καὶ ἀπέστειλε πρέσβεις πρὸς Βαλαὰμ υἱὸν Βεὼρ Φαθουρὰ, ὅ ἐστιν ἐπὶ τοῦ ποταμοῦ γῆς υἱῶν λαοῦ αὐτοῦ, καλέσαι αὐτὸν, λέγων, ἰδοὺ λαὸς ἐξελήλυθεν ἐξ Αἰγύπτου, καὶ ἰδοὺ κατεκάλυψε τὴν ὄψιν τῆς γῆς, καὶ οὗτος ἐγκάθηται ἐχόμενός μου.
\vs{6}Καὶ νῦν δεῦρο ἄρασαί μοι τὸν λαὸν τοῦτον, ὅτι ἰσχύει οὗτος ἢ ἡμεῖς, ἐὰν δυνώμεθα πατάξαι ἐξ αὐτῶν, καὶ ἐκβαλῶ αὐτοὺς ἐκ τῆς γῆς· ὅτι οἶδα οὓς ἐὰν εὐλογήσῃς σὺ, εὐλόγηνται, καὶ οὓς ἂν καταράσῃ σὺ, κεκατήρανται.
\vs{7}Καὶ ἐπορεύθη ἡ γερουσία Μωὰβ, καὶ ἡ γερουσία Μαδιὰμ, καὶ τὰ μαντεῖα ἐν ταῖς χερσὶν αὐτῶν· καὶ ἦλθον πρὸς Βαλαὰμ, καὶ εἶπαν αὐτῷ τὰ ῥήματα Βαλάκ.
\vs{8}Καὶ εἶπε πρὸς αὐτοὺς, καταλύσατε αὐτοῦ τὴν νύκτα, καὶ ἀποκριθήσομαι ὑμῖν πράγματα ἃ ἂν λαλήσῃ Κύριος πρὸς μέ· καὶ κατέμειναν οἱ ἄρχοντες Μωὰβ παρὰ Βαλαάμ.

\vs{9}Καὶ ἦλθεν ὁ Θεὸς πρὸς Βαλαὰμ, καὶ εἶπεν αὐτῷ, τί οἱ ἄνθρωποι οὗτοι παρὰ σοί;
\vs{10}Καὶ εἶπε Βαλαὰμ πρὸς τὸν Θεὸν, Βαλὰκ υἱὸς Σεπφὼρ, βασιλεὺς Μωὰβ, ἀπέστειλεν αὐτοὺς πρὸς μὲ, λέγων,
\vs{11}ἰδοὺ λαὸς ἐξελήλυθεν ἐξ Αἰγύπτου, καὶ κεκάλυφεν τὴν ὄψιν τῆς γῆς, καὶ οὗτος ἐγκάθηται ἐχόμενός μου, καὶ νῦν δεῦρο ἄρασαί μοι αὐτὸν, εἰ ἄρα δυνήσομαι πατάξαι αὐτὸν, καὶ ἐκβαλῶ αὐτὸν ἀπὸ τῆς γῆς.
\vs{12}Καὶ εἶπεν ὁ Θεὸς πρὸς Βαλαὰμ, οὐ πορεύσῃ μετʼ αὐτῶν, οὐδὲ καταράσῃ τὸν λαόν· ἔστι γὰρ εὐλογημένος.
\vs{13}Καὶ ἀναστὰς Βαλαὰμ τοπρωῒ, εἶπε τοῖς ἄρχουσι Βαλὰκ, ἀποτρέχετε πρὸς τὸν κύριον ὑμῶν, οὐκ ἀφίησί με ὁ Θεὸς πορεύεσθαι μεθʼ ὑμῶν.
\vs{14}Καὶ ἀναστάντες οἱ ἄρχοντες Μωὰβ, ἦλθον πρὸς Βαλὰκ, καὶ εἶπαν, οὐ θέλει Βαλαὰμ πορευθῆναι μεθʼ ἡμῶν.

\vs{15}Καὶ προσέθετο Βαλὰκ ἔτι ἀποστεῖλαι ἄρχοντας πλείους, καὶ ἐντιμοτέρους τούτων.
\vs{16}Καὶ ἦλθον πρὸς Βαλαὰμ, καὶ λέγουσιν αὐτῷ, τάδε λέγει Βαλὰκ ὁ τοῦ Σεπφώρ· ἀξιῶ σε μὴ ὀκνήσῃς ἐλθεῖν πρὸς μέ.
\vs{17}ἐντίμως γὰρ τιμήσω σε, καὶ ὅσα ἐὰν εἴπῃς ποιήσω σοι· καὶ δεῦρο ἐπικατάρασαί μοι τὸν λαὸν τοῦτον.
\vs{18}Καὶ ἀπεκρίθη Βαλαὰμ, καὶ εἶπε τοῖς ἄρχουσι Βαλὰκ, ἐὰν δῷ μοι Βαλὰκ πλήρη τὸν οἶκον αὐτοῦ ἀργυρίου καὶ χρυσίου, οὐ δυνήσομαι παραβῆναι τὸ ῥῆμα Κυρίου τοῦ Θεοῦ, ποιῆσαι αὐτὸ μικρὸν ἢ μέγα ἐν τῇ διανοίᾳ μου.
\vs{19}Καὶ νῦν ὑπομείνατε αὐτοῦ καὶ ὑμεῖς τὴν νύκτα ταύτην, καὶ γνώσομαι τί προσθήσει Κύριος λαλῆσαι πρὸς μέ.
\vs{20}Καὶ ἦλθεν ὁ Θεὸς πρὸς Βαλαὰμ νυκτὸς, καὶ εἶπεν αὐτῷ, εἰ καλέσαι σε πάρεισιν οἱ ἄνθρωποι οὗτοι, ἀναστὰς ἀκολούθησον αὐτοῖς· ἀλλὰ τὸ ῥῆμα ὃ ἐὰν λαλήσω πρὸς σὲ, τοῦτο ποιήσεις.

\vs{21}Καὶ ἀναστὰς Βαλαὰμ τοπρωῒ, ἐπέσαξε τὴν ὄνον αὐτοῦ, καὶ ἐπορεύθη μετὰ τῶν ἀρχόντων Μωάβ.
\vs{22}Καὶ ὠργίσθη θυμῷ ὁ Θεὸς ὅτι ἐπορεύθη αὐτός· καὶ ἀνέστη ὁ ἄγγελος τοῦ Θεοῦ διαβαλεῖν αὐτόν· καὶ αὐτὸς ἐπιβεβήκει ἐπὶ τῆς ὄνου αὐτοῦ, καὶ οἱ δύο παῖδες αὐτοῦ μετʼ αὐτοῦ.
\vs{23}Καὶ ἰδοῦσα ἡ ὄνος τὸν ἄγγελον τοῦ Θεοῦ ἀνθεστηκότα ἐν τῇ ὁδῷ, καὶ τὴν ῥομφαίαν ἐσπασμένην ἐν τῇ χειρὶ αὐτοῦ, καὶ ἐξέκλινεν ἡ ὄνος ἐκ τῆς ὁδοῦ, καὶ ἐπορεύετο εἰς τὸ πεδίον· καὶ ἐπάταξε τὴν ὄνον ἐν τῇ ῥάβδῳ αὐτοῦ τοῦ εὐθῦναι αὐτὴν ἐν τῇ ὁδῷ.

\vs{24}Καὶ ἔστη ὁ ἄγγελος τοῦ Θεοῦ ἐν ταῖς αὔλαξι τῶν ἀμπέλων, φραγμὸς ἐντεῦθεν καὶ φραγμὸς ἐντεῦθεν·
\vs{25}Καὶ ἰδοῦσα ἡ ὄνος τὸν ἄγγελον τοῦ Θεοῦ, προσέθλιψεν ἑαυτὴν πρὸς τὸν τοῖχον, καὶ ἀπέθλιψεν τὸν πόδα Βαλαὰμ πρὸς τὸν τοῖχον· καὶ προσέθετο ἔτι μαστίξαι αὐτήν.

\vs{26}Καὶ προσέθετο ὁ ἄγγελος τοῦ Θεοῦ, καὶ ἀπελθὼν ὑπέστη ἐν τόπῳ στενῷ, εἰς ὃν οὐκ ἦν ἐκκλῖναι δεξιὰν ἢ ἀριστεράν.
\vs{27}Καὶ ἰδοῦσα ἡ ὄνος τὸν ἄγγελον τοῦ Θεοῦ, συνεκάθισεν ὑποκάτω Βαλαάμ· καὶ ἐθυμώθη Βαλαὰμ, καὶ ἔτυπτε τὴν ὄνον τῇ ῥάβδῳ.
\vs{28}Καὶ ἤνοιξεν ὁ Θεὸς τὸ στόμα τῆς ὄνου, καὶ λέγει τῷ Βαλαὰμ, τί ἐποίησά σοι, ὅτι πέπαικάς με τρίτον τοῦτο;
\vs{29}Καὶ εἶπε Βαλαὰμ τῇ ὄνῳ, ὅτι ἐμπέπαιχάς μοι, καὶ εἰ εἶχον μάχαιραν ἐν τῇ χειρὶ ἤδη ἂν ἐξεκέντησά σε.
\vs{30}Καὶ λέγει ἡ ὄνος τῷ Βαλαὰμ, οὐκ ἐγὼ ἡ ὄνος σου ἐφʼ ἧς ἐπέβαινες ἀπὸ νεότητός σου, ἕως τῆς σήμερον ἡμέρας; μὴ ὑπεροράσει ὑπεριδούσα ἐποίησά σοι οὕτως; ὁ δὲ εἶπεν, οὐχί.
\vs{31}Ἀπεκάλυψε δὲ ὁ Θεὸς τοὺς ὀφθαλμοὺς Βαλαὰμ, καὶ ὁρᾷ τὸν ἄγγελον Κυρίου ἀνθεστηκότα ἐν τῇ ὁδῷ, καὶ τὴν μάχαιραν ἐσπασμένην ἐν τῇ χειρὶ αὐτοῦ, καὶ κύψας προσεκύνησε τῷ προσώπῳ αὐτοῦ.
\vs{32}Καὶ εἶπεν αὐτῷ ὁ ἄγγελος τοῦ Θεοῦ, διατί ἐπάταξας τὴν ὄνον σου τοῦτο τρίτον; καὶ ἰδοὺ ἐγὼ ἐξῆλθον εἰς διαβολήν σου, ὅτι οὐκ ἀστεία ἡ ὁδός σου ἐναντίον μου, καὶ ἰδοῦσά με ἡ ὄνος, ἐξέκλινεν ἀπʼ ἐμοῦ τρίτον τοῦτο.
\vs{33}Καὶ εἰ μὴ ἐξέκλινεν, νῦν οὖν σὲ μὲν ἀπέκτεινα, ἐκείνην δʼ ἂν περιεποιησάμην.
\vs{34}Καὶ εἶπε Βαλαὰμ τῷ ἀγγέλῳ Κυρίου, ἡμάρτηκα, οὐ γὰρ ἠπιστάμην ὅτι σύ μοι ἀνθέστηκας ἐν τῇ ὁδῷ εἰς συνάντησιν· καὶ νῦν εἰ μή σοι ἀρκέσει, ἀποστραφήσομαι.
\vs{35}Καὶ εἶπεν ὁ ἄγγελος τοῦ Θεοῦ πρὸς Βαλαὰμ, συμπορεύθητι μετὰ τῶν ἀνθρώπων· πλὴν τὸ ῥῆμα ὃ ἐὰν εἴπω πρὸς σὲ, τοῦτο φυλάξῃ λαλῆσαι. Καὶ ἐπορεύθη Βαλαὰμ μετὰ τῶν ἀρχόντων Βαλάκ.

\vs{36}Καὶ ἀκούσας Βαλὰκ ὅτι ἥκει Βαλαὰμ, ἐξῆλθεν εἰς συνάντησιν αὐτῷ, εἰς πόλιν Μωάβ, ἥ ἐστιν ἐπὶ τῶν ὁρίων Ἀρνῶν, ἥ ἐστιν ἐκ μέρους τῶν ὁρίων.
\vs{37}Καὶ εἶπε Βαλὰκ πρὸς Βαλαὰμ, οὐχὶ ἀπέστειλα πρὸς σὲ καλέσαι σε; διατί οὐκ ἤρχου πρὸς μέ; ὄντως οὐ δυνήσομαι τιμῆσαί σε;
\vs{38}Καὶ εἶπε Βαλαὰμ πρὸς Βαλὰκ, ἰδοὺ ἥκω πρὸς σὲ νῦν· δυνατὸς ἔσομαι λαλῆσαί τι; τὸ ῥῆμα ὃ ἐὰν ἐμβάλῃ ὁ Θεὸς εἰς τὸ στόμα μου, τοῦτο λαλήσω.
\vs{39}Καὶ ἐπορεύθη Βαλαὰμ μετὰ Βαλὰκ, καὶ ἦλθον εἰς πόλεις ἐπαύλεων.
\vs{40}Καὶ ἔθυσε Βαλὰκ πρόβατα καὶ μόσχους, καὶ ἀπέστειλε τῷ Βαλαὰμ καὶ τοῖς ἄρχουσι τοῖς μετʼ αὐτοῦ.
\vs{41}Καὶ ἐγενήθη πρωΐ· καὶ παραλαβὼν Βαλὰκ τὸν Βαλαάμ, ἀνεβίβασεν αὐτὸν ἐπὶ τὴν στήλην τοῦ Βαὰλ, καὶ ἔδειξεν αὐτῷ ἐκεῖθεν μέρος τι τοῦ λαοῦ.

\ch{23}
Καὶ εἶπε Βαλαὰμ τῷ Βαλὰκ οἰκοδόμησόν μοι ἐνταῦθα ἑπτὰ βωμοὺς, καὶ ἐτοίμασόν μοι ἐνταῦθα ἑπτὰ μόσχους, καὶ ἑπτὰ κριούς.
\vs{2}Καὶ ἐποίησε Βαλὰκ ὃν τρόπον εἶπεν αὐτῷ Βαλαάμ· καὶ ἀνήνεγκε μόσχον καὶ κριὸν ἐπὶ τὸν βωμόν.
\vs{3}Καὶ εἶπε Βαλαὰμ πρὸς Βαλὰκ, παράστηθι ἐπὶ τῆς θυσίας σου, καὶ πορεύσομαι εἴ μοι φανεῖται ὁ Θεὸς ἐν συναντήσει, καὶ ῥῆμα ὃ ἐάν μοι δείξῃ, ἀναγγελῶ σοι· καὶ παρέστη Βαλὰκ ἐπὶ τῆς θυσίας αὐτοῦ.
\vs{4}Καὶ Βαλαὰμ ἐπορεύθη ἐπερωτῆσαι τὸν Θεόν· καὶ ἐπορεύθη εὐθεῖαν· καὶ ἐφάνη ὁ Θεὸς τῷ Βαλαὰμ· καὶ εἶπε πρὸς αὐτὸν Βαλαάμ, τοὺς ἑπτὰ βωμοὺς ἡτοίμασα, καὶ ἀνεβίβασα μόσχον καὶ κριὸν ἐπὶ τὸν βωμόν.
\vs{5}Καὶ ἐνέβαλεν ὁ Θεὸς ῥῆμα εἰς τὸ στόμα Βαλαὰμ, καὶ εἶπεν, ἐπιστραφεὶς πρὸς Βαλὰκ, οὕτω λαλήσεις.
\vs{6}Καὶ ἀπεστράφη πρὸς αὐτόν· καὶ ὁ δὲ ἐφιστήκει ἐπὶ τῶν ὁλοκαυτωμάτων αὐτοῦ, καὶ πάντες οἱ ἄρχοντες Μωὰβ μετʼ αὐτοῦ· καὶ ἐγενήθη πνεῦμα Θεοῦ ἐπʼ αὐτῷ.
\vs{7}Καὶ ἀναλαβὼν τὴν παραβολὴν αὐτοῦ, εἶπεν ἐκ Μεσοποταμίας μετεπέμψατό με Βαλὰκ βασιλεὺς Μωὰβ ἐξ ὀρέων ἀπʼ ἀνατολῶν, λέγων, δεῦρο ἄρασαί μοι τὸν Ἰακώβ, καὶ δεῦρο ἐπικατάρασαί μοι τὸν Ἰσραήλ.
\vs{8}Τί ἀράσωμαι ὃν μὴ ἀρᾶται Κύριος; ἢ τί καταράσωμαι ὃν μὴ καταρᾶται ὁ Θεός;
\vs{9}Ὅτι ἀπὸ κορυφῆς ὀρέων ὄψομαι αὐτὸν, καὶ ἀπὸ βουνῶν προσνοήσω αὐτόν· ἰδοὺ λαὸς μόνος κατοικήσει, καὶ ἐν ἔθνεσιν οὐ συλλογισθήσεται.
\vs{10}Τίς ἐξηκριβάσατο τὸ σπέρμα Ἰακὼβ, καὶ τίς ἐξαριθμήσεται δήμους Ἰσραήλ; ἀποθάνοι ἡ ψυχή μου ἐν ψυχαῖς δικαίων, καὶ γένοιτο τὸ σπέρμα μου ὡς τὸ σπέρμα τούτων.

\vs{11}Καὶ εἶπε Βαλὰκ πρὸς Βαλαὰμ, τί πεποίηκάς μοι; εἰς κατάρασιν ἐχθρῶν μου κέκληκά σε, καὶ ἰδοὺ εὐλόγηκας εὐλογίαν.
\vs{12}Καὶ εἶπε Βαλαὰμ πρὸς Βαλὰκ, οὐχὶ ὅσα ἂν ἐμβάλῃ ὁ Θεὸς εἰς τὸ στόμα μου, τοῦτο φυλάξω λαλῆσαι;
\vs{13}Καὶ εἶπε πρὸς αὐτὸν Βαλὰκ, δεῦρο ἔτι μετʼ ἐμοῦ εἰς τόπον ἄλλον ἐξ οὗ οὐκ ὄψει αὐτὸν ἐκεῖθεν, ἀλλʼ ἢ μέρος τι αὐτοῦ ὄψει, πάντας δὲ οὐ μὴ ἴδῃς, καὶ κατάρασαί μοι αὐτὸν ἐκεῖθεν.

\vs{14}Καὶ παρέλαβεν αὐτὸν εἰς ἀγροῦ σκοπιὰν ἐπὶ κορυφὴν λελαξευμένου· καὶ ᾠκοδόμησεν ἐκεῖ ἑπτὰ βωμοὺς, καὶ ἀνεβίβασε μόσχον καὶ κριὸν ἐπὶ τὸν βωμόν.
\vs{15}Καὶ εἶπε Βαλαὰμ πρὸς Βαλὰκ, παράστηθι ἐπὶ τῆς θυσίας σου, ἐγὼ δὲ πορεύσομαι ἐπερωτῆσαι τὸν Θεόν.
\vs{16}Καὶ συνήντησεν ὁ Θεὸς τῷ Βαλαὰμ, καὶ ἐνέβαλε ῥῆμα εἰς τὸ στόμα αὐτοῦ, καὶ εἶπεν, ἀποστράφηθι πρὸς Βαλὰκ, καὶ τάδε λαλήσεις.
\vs{17}Καὶ ἀπεστράφη πρὸς αὐτόν· καὶ ὁ δὲ ἐφειστήκει ἐπὶ τῆς ὁλοκαυτώσεως αὐτοῦ, καὶ πάντες οἱ ἄρχοντες Μωὰβ μετʼ αὐτοῦ· καὶ εἶπεν αὐτῷ Βαλὰκ, τί ἐλάλησε Κύριος;
\vs{18}Καὶ ἀναλαβὼν τὴν παραβολὴν αὐτοῦ, εἶπεν, ἀνάστηθι Βαλὰκ, καὶ ἄκουε, ἐνώτισαι μάρτυς υἱὸς Σεπφώρ.
\vs{19}Οὐχ ὡς ἄνθρωπος ὁ Θεὸς διαρτηθῆναι, οὐδʼ ὡς υἱὸς ἀνθρώπου ἀπειληθῆναι, αὐτὸς εἴπας, οὐχὶ ποιήσει; λαλήσει, καὶ οὐχὶ ἐμμενεῖ;
\vs{20}Ἰδοὺ εὐλογεῖν παρείλημμαι· εὐλογήσω, καὶ οὐ μὴ ἀποστρέψω.
\vs{21}Οὐκ ἔσται μόχθος ἐν Ἰακὼβ, οὐδὲ ὀφθήσεται πόνος ἐν Ἰσραήλ· Κύριος ὁ θεὸς αὐτοῦ μετʼ αὐτοῦ, τὰ ἔνδοξα ἀρχόντων ἐν αὐτῷ.
\vs{22}Θεὸς ὁ ἐξαγαγὼν αὐτὸν ἐξ Αἰγύπτου, ὡς δόξα μονοκέρωτος αὐτῷ.
\vs{23}Οὐ γάρ ἐστιν οἰωνισμὸς ἐν Ἰακὼβ, οὐδὲ μαντεία ἐν Ἰσραήλ· κατὰ καιρὸν ῥηθήσεται Ἰακὼβ, καὶ τῷ Ἰσραὴλ, τί ἐπιτελέσει ὁ Θεός;
\vs{24}Ἰδοὺ λαὸς ὡς σκύμνος ἀναστήσεται, καὶ ὡς λέων γαυρωθήσεται· οὐ κοιμηθήσεται ἕως φάγῃ θήραν, καὶ αἷμα τραυματιῶν πίεται.

\vs{25}Καὶ εἶπε Βαλὰκ πρὸς Βαλαὰμ, οὔτε κατάραις καταράσῃ μοι αὐτὸν, οὔτε εὐλογῶν μὴ εὐλογήσῃς αὐτόν.
\vs{26}Καὶ ἀποκριθεὶς Βαλαὰμ, εἶπε τῷ Βαλὰκ, οὐκ ἐλάλησά σοι, λέγων, τὸ ῥῆμα ὃ ἐὰν λαλήσῃ ὁ Θεὸς, τοῦτο ποιήσω;
\vs{27}Καὶ εἶπε Βαλὰκ πρὸς Βαλαὰμ, δεῦρο παραλάβω σε εἰς τόπον ἄλλον, εἰ ἀρέσει τῷ Θεῷ, καὶ κατάρασαί μοι αὐτὸν ἐκεῖθεν.
\vs{28}Καὶ παρέλαβε Βαλὰκ τὸν Βαλαὰμ ἐπὶ κορυφὴν τοῦ Φογὼρ, τὸ παρατεῖνον εἰς τὴν ἔρημον.
\vs{29}Καὶ εἶπε Βαλαὰμ πρὸς Βαλὰκ, οἰκοδόμησόν μοι ὧδε ἑπτὰ βωμοὺς, καὶ ἑτοίμασόν μοι ὧδε ἑπτὰ μόσχους, καὶ ἑπτὰ κριούς.
\vs{30}Καὶ ἐποίησε Βαλὰκ καθάπερ εἶπεν αὐτῷ Βαλαάμ, καί ἀνήνεγκε μόσχον καὶ κριὸν ἐπὶ τὸν βωμόν.

\ch{24}
Καὶ ἰδὼν Βαλαὰμ ὅτι καλόν ἐστιν ἐναντίον Κυρίου εὐλογεῖν τὸν Ἰσραὴλ, οὐκ ἐπορεύθη κατὰ τὸ εἰωθὸς αὐτῷ εἰς συνάντησιν τοῖς οἰωνοῖς, καὶ ἀπέστρεψε τὸ πρόσωπον αὐτοῦ εἰς τὴν ἔρημον.
\vs{2}Καὶ ἐξάρας Βαλαὰμ τοὺς ὀφθαλμοὺς αὐτοῦ, καθορᾷ τὸν Ἰσραὴλ ἐστρατοπεδευκότα κατὰ φυλάς· καὶ ἐγένετο ἐπʼ αὐτῷ πνεῦμα Θεου.
\vs{3}Καὶ ἀναλαβὼν τὴν παραβολὴν αὐτοῦ, εἶπε, φησὶ Βαλαὰμ υἱοῖς Βεὼρ, φησὶν ὁ ἄνθρωπος ὁ ἀληθινῶς ὁρῶν,
\vs{4}φησὶν ἀκούων λόγια ἰσχυροῦ, ὅστις ὅρασιν Θεοῦ εἶδεν ἐν ὕπνῳ· ἀποκεκαλυμμένοι οἱ ὀφθαλμοὶ αὐτοῦ.
\vs{5}Ὡς καλοὶ οἱ οἶκοί σου Ἰακὼβ, αἱ σκηναί σου Ἰσραήλ·
\vs{6}Ὡσεὶ νάπαι σκιάζουσαι, καὶ ὡσεὶ παράδεισοι ἐπὶ ποταμῷ, καὶ ὡσεὶ, σκηναὶ, ἃς ἔπηξε Κύριος, καί ὡσεὶ κέδροι παρʼ ὕδατα.
\vs{7}Ἐξελεύσεται ἄνθρωπος ἐκ τοῦ σπέρματος αὐτοῦ, καὶ κυριεύσει ἐθνῶν πολλῶν· καὶ ὑψωθήσεται ἡ Γὼγ βασιλεία, καὶ αὐξηθήσεται βασιλεία αὐτοῦ.
\vs{8}Θεὸς ὡδήγησεν αὐτὸν ἐξ Αἰγύπτου· ὡς δόξα μονοκέρωτος αὐτῷ· ἔδεται ἔθνη ἐχθρῶν αὐτοῦ, καὶ τὰ πάχη αὐτῶν ἐκμυελιεῖ, καὶ ταῖς βολίσιν αὐτοῦ κατατοξεύσει ἐχθρόν.
\vs{9}Κατακλιθεὶς ἀνεπαύσατο ὡς λέων, καὶ ὡς σκύμνος· τίς ἀναστήσει αὐτόν· οἱ εὐλογοῦντές σε, εὐλόγηνται· καὶ οἱ καταρώμενοί σε, κεκατήρανται.

\vs{10}Καὶ ἐθυμώθη Βαλὰκ ἐπὶ Βαλαὰμ, καὶ συνεκρότησε ταῖς χερσὶν αὐτοῦ, καὶ εἶπε Βαλὰκ πρὸς Βαλαὰμ, καταρᾶσθαι τὸν ἐχθρόν μου κέκληκά σε, καὶ ἰδοὺ εὐλογῶν εὐλόγησας τρίτον τοῦτο.
\vs{11}Νῦν οὖν φεῦγε εἰς τὸν τόπον σου· εἶπα, τιμήσω σε, καὶ νῦν ἐστέρησέ σε Κύριος τῆς δόξης.
\vs{12}Καὶ εἶπε Βαλαὰμ πρὸς Βαλὰκ, οὐχὶ καὶ τοῖς ἀγγέλοις σου οὓς ἀπέστειλας πρὸς με ἐλάλησα, λέγων,
\vs{13}ἐάν μοι δῷ Βαλὰκ πλήρη τὸν οἶκον αὐτοῦ ἀργυρίου καὶ χρυσίου, οὐ δυνήσομαι παραβῆναι τὸ ῥῆμα Κυρίου ποιῆσαι αὐτὸ καλὸν ἤ πονηρὸν παρʼ ἐμαυτοῦ· ὅσα ἂν εἴπῃ ὁ Θεός, ταῦτα ἐρῶ.
\vs{14}Καὶ νῦν ἰδοὺ ἀποτρέχω εἰς τὸν τόπον μου· δεῦρο, συμβουλεύσω σοι, τί ποιήσει ὁ λαὸς οὗτος τὸν λαόν σου ἐπʼ ἐσχάτου τῶν ἡμερῶν.

\vs{15}Καὶ ἀναλαβὼν τὴν παραβολὴν αὐτοῦ, εἶπε,

\vs{16}Φησὶ Βαλαὰμ υἱὸς Βεὼρ, φησὶν ὁ ἄνθρωπος ὁ ἀληθινῶς ὁρῶν, ἀκούων λόγια Θεοῦ, ἐπιστάμενος ἐπιστήμην παρὰ ὑψίστου, καὶ ὅρασιν Θεοῦ ἰδὼν ἐν ὕπνῳ· ἀποκεκαλυμμένοι οἱ ὀφθαλμοὶ αὐτοῦ.
\vs{17}Δείξω αὐτῷ, καὶ οὐχὶ νῦν· μακαρίζω, καὶ οὐκ ἐγγίζει· ἀνατελεῖ ἄστρον ἐξ Ἰακὼβ, ἀναστήσεται ἄνθρωπος ἐξ Ἰσραήλ· καὶ θραύσει τοὺς ἀρχηγοὺς Μωὰβ, καῒ προνομεύσει πάντας υἱοὺς Σήθ.
\vs{18}Καὶ ἔσται Ἐδὼμ κληρονομία, καὶ ἔσται κληρονομία Ἡσαῦ ὁ ἐχθρὸς αὐτοῦ· καὶ Ἰσραὴλ ἐποίησεν ἐν ἰσχύϊ.
\vs{19}Καὶ ἐξεγερθήσεται ἐξ Ἰακὼβ, καὶ ἀπολεῖ σωζόμενον ἐκ πόλεως.
\vs{20}Καὶ ἰδὼν τὸν Ἀμαλὴκ, καὶ ἀναλαβὼν τὴν παραβολὴν αὐτοῦ, εἶπεν, ἀρχὴ ἐθνῶν Ἀμαλὴκ, καὶ τὸ σπέρμα αὐτῶν ἀπολεῖται.
\vs{21}Καὶ ἰδὼν τὸν Κεναῖον, καὶ ἀναλαβὼν τὴν παραβολὴν αὐτοῦ, εἶπεν, ἰσχυρὰ ἡ κατοικία σου· καὶ ἐὰν θῇς ἐν πέτρᾳ τὴν νοσσίαν σου,
\vs{22}καὶ ἐὰν γένηται τῷ Βεὼρ νοσσιὰ πανουργίας, Ἀσσύριοι αἰχμαλωτεύσουσί σε.
\vs{23}Καὶ ἰδὼν τὸν Ὢγ, καὶ ἀναλαβὼν τὴν παραβολὴν αὐτοῦ, εἶπεν, ὢ ὢ, τίς ζήσεται, ὅταν θῇ ταῦτα ὁ Θεός;
\vs{24}Καὶ ἐξελεύσεται ἐκ χειρῶς Κιτιαίων, καὶ κακώσουσιν Ἀσσοὺρ, καὶ κακώσουσιν Ἐβραίους, καὶ αὐτοὶ ὁμοθυμαδὸν ἀπολοῦνται.
\vs{25}Καὶ ἀναστὰς Βαλαὰμ ἀπῆλθεν, ἀποστραφεὶς εἰς τὸν τόπον αὐτοῦ· καὶ Βαλὰκ ἀπῆλθε πρὸς ἑαυτόν.

\ch{25}
Καὶ κατέλυσεν Ἰσραὴλ ἐν Σαττεὶν, καὶ ἐβεβηλώθη ὁ λαὸς ἐκπορνεῦσαι εἰς τὰς θυγατέρας Μωάβ.
\vs{2}Καὶ ἐκάλεσαν αὐτοὺς εἰς τὰς θυσίας τῶν εἰδώλων αὐτῶν· καὶ ἔφαγεν ὁ λαὸς τῶν θυσιῶν αὐτῶν, καὶ προσεκύνησαν τοῖς εἰδώλοις αὐτῶν·
\vs{3}Καὶ ἐτελέσθη Ἰσραὴλ τῷ Βεελφεγώρ· καὶ ὠργίσθη θυμῷ Κύριος τῷ Ἰσραήλ.
\vs{4}Καὶ εἶπε Κύριος τῷ Μωυσῇ, λάβε πάντας τοὺς ἀρχηγοὺς τοῦ λαοῦ, καὶ παραδειγμάτισον αὐτοὺς Κυρίῳ κατέναντι τοῦ ἡλίου, καὶ ἀποστραφήσεται ὀργὴ θυμοῦ Κυρίου ἀπὸ Ἰσραήλ.
\vs{5}Καὶ εἶπε Μωυσῆς ταῖς φυλαῖς Ἰσραήλ, ἀποκτείνατε ἕκαστος τὸν οἰκεῖον αὐτοῦ τὸν τετελεσμένον τῷ Βεελφεγώρ.
\vs{6}Καὶ ἰδοὺ ἄνθρωπος τῶν υἱῶν Ἰσραὴλ ἐλθὼν προσήγαγε τὸν ἀδελφὸν αὐτοῦ πρὸς τὴν Μαδιανίτιν ἐναντίον Μωυσῆ, καὶ ἐναντίον πάσης συναγωγῆς υἱῶν Ἰσραήλ· αὐτοὶ δὲ ἔκλαιον παρὰ τὴν θύραν τῆς σκηνῆς τοῦ μαρτυρίου.
\vs{7}Καὶ ἰδὼν Φινεὲς υἱὸς Ἐλεάζαρ υἱοῦ Ἀαρὼν τοῦ ἱερέως, ἐξανέστη ἐκ μέσου τῆς συναγωγῆς, καὶ λαβὼν σειρομάστην ἐν τῇ χειρὶ,
\vs{8}εἰσῆλθεν ὀπίσω τοῦ ἀνθρώπου τοῦ Ἰσραηλίτου εἰς τὴν κάμινον, καὶ ἀπεκέντησεν ἀμφοτέρους, τόν τε ἄνθρωπον τὸν Ἰσραηλίτην, καὶ τὴν γυναῖκα διὰ τῆς μήτρας αὐτῆς· καὶ ἐπαύσατο ἡ πληγὴ ἀπὸ υἱῶν Ἰσραήλ.
\vs{9}Καὶ ἐγένοντο οἱ τεθνηκότες ἐν τῇ πληγῇ, τέσσαρες καὶ εἴκοσι χιλιάδες.

\vs{10}Καὶ ἐλάλησε Κύριος πρὸς Μωυσῆν, λέγων,
\vs{11}Φινεὲς υἱὸς Ἐλεάζαρ υἱοῦ Ἀαρὼν τοῦ ἱερέως κατέπαυσε τὸν θυμόν μου ἀπὸ υἱῶν Ἰσραὴλ, ἐν τῷ ζηλῶσαί μου τὸν ζῆλον ἐν αὐτοῖς, καὶ οὐκ ἐξανήλωσα τοὺς υἱοὺς Ἰσραὴλ ἐν τῷ ζήλῳ μου.
\vs{12}Οὕτως εἰπον, ἰδοὺ ἐγὼ δίδωμι αὐτῷ διαθήκην εἰρήνης,
\vs{13}καὶ ἔσται αὐτῷ καὶ τῷ σπέρματι αὐτοῦ μετʼ αὐτὸν διαθήκη ἱερατείας αἰωνία, ἀνθʼ ὧν ἐζήλωσεν τῷ Θεῷ αὐτοῦ, καὶ ἐξιλάσατο περὶ τῶν υἱῶν Ἰσραήλ.
\vs{14}Τὸ δὲ ὄνομα τοῦ ἀνθρώπου τοῦ Ἰσραηλείτου τοῦ πεπληγότος, ὃς ἐπλήγη μετὰ τῆς Μαδιανίτιδος, Ζαμβρὶ, υἱὸς Σαλμὼν, ἄρχων οἴκου πατριᾶς τῶν Συμεών.
\vs{15}Καὶ ὄνομα τῇ γυναικὶ τῇ Μαδιανίτιδι τῇ πεπληγυίᾳ, Χασβὶ, θυγάτηρ Σοὺρ, ἄρχοντος ἔθνους Ὀμμὼθ· οἴκου πατριᾶς ἐστι τῶν Μαδιάμ.

\vs{16}Καὶ ἐλάλησε Κύριος πρὸς Μωυσῆν, λέγων, λάλησον τοῖς υἱοῖς Ἰσραὴλ, λέγων,
\vs{17}ἐχθραίνετε τοῖς Μαδιηναίοις καὶ πατάξατε αὐτοὺς,
\vs{18}ὅτι ἐχθραίνουσιν αὐτοὶ ὑμῖν ἐν δολιότητι, ὅσα δολιοῦσιν ὑμᾶς διὰ Φογὼρ, καὶ διὰ Χασβὶ θυγατέρα ἄρχοντος Μαδιὰμ ἀδελφὴν αὐτῶν, τὴν πεπληγυῖαν ἐν τῇ ἡμέρᾳ τῆς πληγῆς διὰ Φογώρ.

\ch{26}
Καὶ ἐγένετο μετὰ τὴν πληγὴν, καὶ ἐλάλησε Κύριος πρὸς Μωυσῆν καὶ Ἐλεάζαρ τὸν ἱερέα, λέγων,
\vs{2}λάβε τὴν ἀρχὴν πάσης συναγωγῆς υἱῶν Ἰσραὴλ ἀπὸ εἰκοσαετοῦς καὶ ἐπάνω κατʼ οἴκους πατριῶν αὐτῶν, πᾶς ὁ ἐκπορευόμενος παρατάξασθαι ἐν Ἰσραήλ.

\vs{3}Καὶ ἐλάλησε Μωυσῆς καὶ Ἐλεάζαρ ὁ ἱερεὺς ἐν Ἀραβὼθ Μωἀβ ἐπὶ τοῦ Ἰορδάνου κατὰ Ἱεριχὼ, λέγων,
\vs{4}ἀπὸ εἰκοσαετοῦς καὶ ἐπάνω, ὃν τρόπον συνέταξε Κύριος τῷ Μωυσῇ· καὶ οἱ υἱοὶ Ἰσραὴλ οἱ ἐξελθόντες ἐξ Αἰγύπτου,
\vs{5}Ῥουβὴν πρωτότοκος Ἰσραήλ· υἱοὶ δὲ Ῥουβὴν Ἐνὼχ, καὶ δῆμος τοῦ Ἐνώχ· τῷ Φαλλοῦ, δῆμος τοῦ Φαλλουί.
\vs{6}Τῷ Ἀσρὼν, δῆμος τοῦ Ἀσρωνεί· τῷ Χαρμὶ, δῆμος τοῦ Χαρμί.
\vs{7}Οὗτοι δῆμοι Ῥουβήν· καὶ ἐγένετο ἡ ἐπίσκεψις αὐτῶν, τρεῖς καὶ τεσσαράκοντα χιλιάδες καὶ ἑπτακόσιοι καὶ τριάκοντα.

\vs{8}Καὶ υἱοῖ Φαλλοὺ, Ἑλιάβ.
\vs{9}Καὶ υἱοὶ Ἑλιὰβ, Ναμουὴλ, καὶ Δαθὰν, καὶ Ἀβειρών· οὗτοι ἐπίκλητοι τῆς συναγωγῆς· οὗτοί εἰσιν οἱ ἐπισυστάντες ἐπὶ Μωυσῆν καὶ Ἀαρὼν ἐν τῇ συναγωγῇ Κορὲ, ἐν τῇ ἐπισυστάσει Κυρίου.
\vs{10}Καὶ ἀνοίξασα ἡ γῆ τὸ στόμα αὐτῆς, κατέπιεν αὐτοὺς καὶ Κορὲ, ἐν τῷ θανάτῳ τῆς συναγωγῆ αὐτοῦ, ὅτε κατέφαγε τὸ πῦρ τοὺς πεντήκοντα καὶ διακοσίους, καὶ ἐγενήθησαν ἐν σημείῳ·
\vs{11}οἱ δὲ υἱοὶ Κορὲ οὐκ ἀπέθανον.

\vs{12}Καὶ οἱ υἱοὶ Συμεὼν, ὁ δῆμος τῶν υἱῶν Συμεών· τῷ Ναμουὴλ, δῆμος ὁ Ναμουηλί· τῷ Ἰαμὶν, δῆμος ὁ Ἰαμινί· τῷ Ἰαχὶν, δῆμος Ἰαχινί·
\vs{13}Τῷ Ζαρὰ δῆμος ὁ Ζαραΐ· τῷ Σαοὺλ, δῆμος ὁ Σαουλί.
\vs{14}Οὗτοι δῆμοι Συμεὼν ἐκ τῆς ἐπισκέψεως αὐτῶν, δύο καὶ εἴκοσι χιλιάδες καὶ διακόσιοι.

\vs{15}Υἱοὶ δὲ Ἰούδα, Ἢρ καὶ Αὐνάν· καὶ ἀπέθανον Ἤρ καὶ Αὐνὰν ἐν γῇ Χαναάν.
\vs{16}Καὶ ἐγένοντο οἱ υἱοὶ Ἰούδα κατὰ δήμους αὐτῶν· τῷ Σηλὼμ, δῆμος ὁ Σηλωνί· τῷ Φαρὲς, δῆμος ὁ Φαρεσί· τῷ Ζαρὰ, δῆμος ὁ Ζαραΐ.
\vs{17}Καὶ ἐγένοντο οἱ υἱοὶ Φαρὲς, τῷ Ἀσρὼν, δῆμος ὁ Ἀσρωνί· τῷ Ἰαμοῦν, δῆμος ὁ Ἰαμουνί.
\vs{18}Οὗτοι δῆμοι τοῦ Ἰούδα κατὰ τὴν ἐπίσκεψιν αὐτῶν, ἓξ καὶ ἑβδομήκοντα χιλιάδες καὶ πεντακόσιοι.

\vs{19}Καὶ υἱοὶ Ἰσσάχαρ κατὰ δήμους αὐτῶν· τῷ Θωλᾷ, δῆμος ὁ Θωλαΐ· τῷ Φουᾷ, δῆμος ὁ Φουαΐ·
\vs{20}Τῷ Ἰασοὺβ, δῆμος ὁ Ἰασουβί· τῷ Σαμρὰμ, δῆμος ὁ Σαμραμί.
\vs{21}Οὗτοι δῆμοι Ἰσσάχαρ ἐξ ἐπισκέψεως αὐτῶν, τέσσαρες καὶ ἑξήκοντα χιλιάδες καὶ τετρακόσιοι.

\vs{22}Υἱοὶ Ζαβουλὼν κατὰ δήμους αὐτῶν· τῷ Σαρὲδ, δῆμος ὁ Σαρεδί· τῷ Ἀλλὼν, δῆμος ὁ Ἀλλωνί· τῷ Ἀλλὴλ, δῆμος ὁ Ἀλληλί.
\vs{23}Οὗτοι δῆμοι Ζαβουλὼν ἐξ ἐπισκέψεως αὐτῶν, ἐξήκοντα χιλιάδες καὶ πεντακόσιοι.

\vs{24}Υἱοὶ Γὰδ κατὰ δήμους αὐτῶν· τῷ Σαφὼν, δῆμος ὁ Σαφωνί· τῷ Ἀγγὶ, δῆμος ὁ Ἀγγί· τῷ Σουνὶ, δῆμος ὁ Σουνί·
\vs{25}τῷ Ἀζενὶ, δῆμος ὁ Ἀζενί· τῷ Ἀδδὶ, δῆμος ὁ Ἀδδί·
\vs{26}τῷ Ἀροδὶ, δῆμος ὁ Αροαδί· τῷ Ἀριὴλ, δῆμος ὁ Ἀριηλί.
\vs{27}Οὗτοι δῆμοι υἱῶν Γὰδ ἐξ ἑπισκέψεως αὐτῶν, τέσσαρες καὶ τεσσαράκοντα χιλιάδες καὶ πεντακόσιοι.

\vs{28}Υἱοὶ Ἀσὴρ κατὰ δήμους αὐτῶν· τῷ Ἰαμὶν, δῆμος ὁ Ἰαμινί· τῷ Ἰεσοὺ, δῆμος ὁ Ἰεσουΐ· τῷ Βαριὰ, δῆμος ὁ Βαριαΐ.
\vs{29}Τῷ Χοβὲρ, δῆμος ὁ Χοβερί· τῷ Μελχιὴλ, δῆμος ὁ Μελχιηλί.
\vs{30}Καὶ τὸ ὄνομα θυγατρὸς Ἀσὴρ, Σάρα.
\vs{31}Οὗτοι δῆμοι Ἀσὴρ ἐξ ἐπισκέψεως αὐτῶν, τρεῖς καὶ τεσσαράκοντα χιλιάδες καὶ τετρακόσιοι.

\vs{32}Υἱοὶ Ἰωσὴφ κατὰ δήμους αὐτῶν, Μανασσῆ καὶ Ἐφραίμ.

\vs{33}Υἱοὶ Μανασσῆ. Τῷ Μαχὶρ, δῆμος ὁ Μαχιρί· καὶ Μαχὶρ ἐγέννησε τὸν Γαλαάδ· τῷ Γαλαὰδ, δῆμος ὁ Γαλααδί.
\vs{34}Καὶ οὗτοι υἱοὶ Γαλαάδ· Ἀχιέζερ, δῆμος ὁ Ἀχιεζερί· τῷ Χελὲγ, δῆμος ὁ Χελεγί.
\vs{35}Τῷ Ἐσριὴλ, δῆμος ὁ Ἐσριηλί· τῷ Συχὲμ, δῆμος ὁ Συχεμί·
\vs{36}Τῷ Συμαὲρ, δῆμος ὁ Συμαερί· καὶ τῷ Ὀφὲρ, δῆμος ὁ Ὀφερὶ.
\vs{37}Καὶ τῷ Σαλπαὰδ, υἱῷ Ὀφὲρ, οὐκ ἐγένοντο αὐτῷ υἱοὶ, ἀλλʼ ἢ θυγατέρες· καὶ ταῦτα τὰ ὀνόματα τῶν θυγατέρων Σαλπαάδ· Μαλὰ, καὶ Νουὰ, καὶ Ἐγλὰ, καὶ Μελχὰ, καὶ Θερσά.
\vs{38}Οὗτοι δῆμοι Μανασσὴ ἐξ ἐπισκέψεως αὐτῶν, δύο καὶ πεντήκοντα χιλιαδες καὶ ἑπτακόσιοι.

\vs{39}Καὶ οὗτοι υἱοὶ Ἐφραίμ· τῷ Σουθαλὰ, δῆμος ὁ Σουθαλάν· τῷ Τανὰχ, δῆμος ὁ Ταναχί.
\vs{40}Οὗτοι υἱοὶ Σουθαλά· τῷ Ἐδὲν, δῆμος ὁ Ἐδενί.
\vs{41}Οὗτοι δῆμοι Ἐφραὶμ ἐξ ἐπισκέψεως αὐτῶν, δύο καὶ τριάκοντα χιλιάδες καὶ πεντακόσιοι· οὗτοι δῆμοι υἱῶν Ἰωσὴφ κατὰ δήμους αὐτῶν.

\vs{42}Υἱοὶ Βενιαμὶν κατὰ δήμους αὐτῶν· τῷ Βαλὲ, δῆμος ὁ Βαλί· τῷ Ἀσυβὴρ, δῆμος ὁ Ἀσυβηρί· τῷ Ἰαχιρὰν, δῆμος ὁ Ἰαχιρανί.

\vs{43}Τῷ Σωφὰν, δῆμος ὁ Σωφανι.
\vs{44}Καὶ ἐγένοντο οἱ υἱοὶ Βαλὲ, Ἀδὰρ, καὶ Νοεμάν· τῷ Ἀδὰρ, δῆμος ὁ Αδαρί· καὶ τῷ Νοεμὰν, δῆμος ὁ Νοεμανί.
\vs{45}Οὗτοι υἱοὶ Βενιαμὶν κατὰ δήμους αὐτῶν ἐξ ἐπισκέψεως αὐτῶν, πέντε καὶ τριάκοντα χιλιάδες καὶ πεντακόσιοι.

\vs{46}Καὶ υἱοὶ Δὰν κατὰ δήμους αὐτῶν· τῷ Σαμὲ, δῆμος ὁ Σαμεΐ, οὗτοι δῆμοι Δὰν κατὰ δήμους αὐτῶν.
\vs{47}Πάντες οἱ δῆμοι Σαμεῒ κατʼ ἐπισκοπὴν αὐτῶν, τέσσαρες καὶ ἑξήκοντα χιλιάδες καὶ τετρακόσιοι.

\vs{48}Υἱοὶ Νεφθαλὶ κατὰ δήμους αὐτῶν· τῷ Ἀσιὴλ· δῆμος ὁ Ἀσιηλί· τῷ Γαυνὶ, δῆμος ὁ Γαυνί.
\vs{49}Τῷ Ἰεσὲρ, δῆμος ὁ Ἰεσερί· τῷ Σελλὴμ, δῆμος ὁ Σελλημί.
\vs{50}Οὗτοι δῆμοι Νεφθαλὶ ἐξ ἐπισκέψεως αὐτῶν, τεσσαράκοντα χιλιάδες καὶ τριακόσιοι.

\vs{51}Αὕτη ἡ ἐπίσκεψις υἱῶν Ἰσραὴλ, ἑξακόσιαι χιλιάδες καὶ χίλιοι καὶ ἑπτακόσιοι καὶ τριάκοντα.

\vs{52}Καὶ ἐλάλησε Κύριος πρὸς Μωυσῆν, λέγων,
\vs{53}τούτοις μερισθήσεται ἡ γῆ, κληρονομεῖν ἐξ ἀριθμοῦ ὀνομάτων.
\vs{54}Τοῖς πλείοσι πλεονάσεις τὴν κληρονομίαν, καὶ τοῖς ἐλάττοσιν ἐλαττώσεις τὴν κληρονομίαν αὐτῶν· ἑκάστῳ, καθὼς ἐπεσκέπησαν, δοθήσεται ἡ κληρονομία αὐτῶν.
\vs{55}Διὰ κλήρων μερισθήσεται ἡ γῆ τοῖς ὀνόμασι· κατὰ φυλὰς πατριῶν αὐτῶν κληρονομήσουσιν.
\vs{56}Ἐκ τοῦ κλήρου μεριεῖς τὴν κληρονομίαν αὐτῶν ἀναμέσον πολλῶν καὶ ὀλίγων.

\vs{57}Καὶ υἱοὶ Λευὶ κατὰ δήμους αὐτῶν· τῷ Γεδσὼν, δῆμος ὁ Γεδσωνί· τῷ Καὰθ, δῆμος ὁ Κααθί· τῷ Μεραρὶ, δῆμος ὁ Μεραρί.
\vs{58}Οὗτοι δῆμοι υἱῶν Λευί· δῆμος ὁ Λοβενί, δῆμος ὁ Χεβρωνὶ, δῆμος ὁ Κορὲ, καὶ δῆμος ὁ Μουσί· καὶ Καὰθ ἐγέννησε τὸν Ἁμράμ.
\vs{59}Τὸ δὲ ὄνομα τῆς γυναικὸς αὐτοῦ Ἰωχαβὲδ, θυγάτηρ Λευὶ, ἣ ἔτεκε τούτους τῷ Λευὶ ἐν Αἰγύπτῳ, καὶ ἔτεκε τῷ Ἀμρὰμ τὸν Ἀαρὼν καὶ Μωυσῆν, καὶ Μαριὰμ τὴν ἀδελφὴν αὐτῶν.
\vs{60}Καὶ ἐγενήθησαν τῷ Ἀαρὼν, ὅ, τε Ναδὰβ, καὶ Ἀβιοὺδ, καὶ Ἐλεάζαρ, καὶ Ἰθάμαρ.
\vs{61}Καὶ ἀπέθανε Ναδὰβ καὶ Ἀβιοὺδ ἐν τῷ προσφέρειν αὐτοὺς πῦρ ἀλλότριον ἔναντι Κυρίου ἐν τῇ ἐρήμῳ Σινᾷ.
\vs{62}Καὶ ἐγενήθησαν ἐξ ἐπισκέψεως αὐτῶν τρεῖς καὶ εἴκοσι χιλιάδες, πᾶν ἀρσενικὸν ἀπὸ μηνιαίου καὶ ἐπάνω· οὐ γὰρ συνεπεσκέπησαν ἐν μέσῳ υἱῶν Ἰσραὴλ, ὅτι οὐ δίδοται αὐτοῖς κλῆρος ἐν μέσῳ υἱῶν Ἰσραήλ.

\vs{63}Καὶ αὕτη ἡ ἐπίσκεψις Μωυσῆ καὶ Ἐλεάζαρ τοῦ ἱερέως, οἱ ἐπεσκέψαντο τοὺς υἱοὺς Ἰσραὴλ ἐν Ἀραβὼθ Μωὰβ, ἐπὶ τοῦ Ἰορδάνου κατὰ Ἱεριχώ.
\vs{64}Καὶ ἐν τούτοις οὐκ ἦν ἄνθρωπος τῶν ἐπεσκεμμένων ὑπὸ Μωυσῆ καὶ Ἀαρὼν, οὓς ἐπεσκέψαντο τοὺς υἱοὺς Ἰσραὴλ ἐν τῇ ἐρήμῳ Σινᾷ.
\vs{65}Ὅτι εἶπε Κύριος αὐτοῖς, θανάτῳ ἀποθανοῦνται ἐν τῇ ἐρήμῳ· καὶ οὐ κατελείφθη ἐξ αὐτῶν οὐδὲ εἷς, πλὴν Χάλεβ υἱὸς Ἰεφοννὴ, καὶ Ἰησοῦς ὁ τοῦ Ναυή.

\ch{27}
Καὶ προσελθοῦσαι αἱ θυγατέρες Σαλπαὰδ υἱοῦ Ὀφὲρ, υἱοῦ Γαλαὰδ, υἱοῦ Μαχιρ, τοῦ δήμου Μανασσῆ, τῶν υἱῶν Ἰωσὴφ, καὶ ταῦτα τὰ ὀνόματα αὐτῶν, Μααλὰ, καὶ Νουὰ, καὶ Ἐγλὰ, καὶ Μελχὰ, καὶ Θερσὰ,
\vs{2}καὶ στᾶσαι ἔναντι Μωυσῆ, καὶ ἔναντι Ἐλεάζαρ τοῦ ἱερέως, καὶ ἔναντι τῶν ἀρχόντων, καὶ ἔναντι πάσης συναγωγῆς ἐπὶ τῆς θύρας τῆς σκηνῆς τοῦ μαρτυρίου, λέγουσιν,
\vs{3}ὁ πατὴρ ἡμῶν ἀπέθανεν ἐν τῇ ἐρήμῳ, καὶ αὐτὸς οὐκ ἦν ἐν μέσῳ τῆς συναγωγῆς τῆς ἐπισυστάσης ἔναντι Κυρίου ἐν τῇ συναγωγῇ Κορὲ, ὅτι διʼ ἁμαρτίαν αὐτοῦ ἀπέθανε, καὶ υἱοὶ οὐκ ἐγένοντο αὐτῷ· μὴ ἐξαλειφθήτω τὸ ὄνομα τοῦ πατρὸς ἡμῶν ἐκ μέσου τοῦ δήμου αὐτοῦ, ὅτι οὐκ ἔστιν αὐτῷ υἱός· δότε ἡμῖν κατάσχεσιν ἐν μέσῳ ἀδελφῶν πατρὸς ἡμῶν.
\vs{4}Καὶ προσήγαγε Μωυσῆν τὴν κρίσιν αὐτῶν ἔναντι Κυρίου.

\vs{5}Καὶ ἐλάλησε Κύριος πρὸς Μωυσῆν,
\vs{6}λέγεν, ὀρθῶς θυγατέρες Σαλπαὰδ λελαλήκασι· δόμα δώσεις αὐταῖς κατάσχεσιν κληρονομίας ἐν μέσῳ ἀδελφῶν πατρὸς αὐτῶν, καὶ περιθήσεις τὸν κλῆρον τοῦ πατρὸς αὐτῶν αὐταῖς.
\vs{7}Καὶ τοῖς υἱοῖς Ἰσραὴλ λαλήσεις,
\vs{8}λέγων, ἄνθρωπος ἐὰν ἀποθάνῃ, καὶ υἱὸς μὴ ᾖ αὐτῷ, περιθήσετε τὴν κληρονομίαν αὐτοῦ τῇ θυγατρὶ αὐτοῦ·
\vs{9}Ἐὰν δὲ μὴ ᾖ θυγάτηρ αὐτῷ, δώσετε τὴν κληρονομίαν τῷ ἀδελφῷ αὐτοῦ.
\vs{10}Ἐὰν δὲ μὴ ὦσιν αὐτῷ ἀδελφοὶ, δώσετε τὴν κληρονομίαν τῷ ἀδελφῷ τοῦ πατρὸς αὐτοῦ.
\vs{11}Ἐὰν δὲ μὴ ὦσιν ἀδελφοὶ τοῦ πατρὸς αὐτοῦ, δώσετε τὴν κληρονομίαν τῷ οἰκείῳ τῷ ἔγγιστα αὐτοῦ ἐκ τῆς φυλῆς αὐτοῦ, κληρονομῆσαι τὰ αὐτοῦ· καὶ ἔσται τοῦτο τοῖς υἱοῖς Ἰσραὴλ δικαίωμα κρίσεως, καθὰ συνέταξε Κύριος τῷ Μωυσῇ.

\vs{12}Καὶ εἶπε Κύριος πρὸς Μωυσῆν, ἀνάβηθι εἰς τὸ ὄρος τὸ ἐν τῷ πέραν τοῦ Ἰορδάνου, τοῦτο τὸ ὄρος Ναβαὺ, καὶ ἴδε τὴν γῆν Χαναὰν, ἣν ἐγὼ δίδωμι τοῖς υἱοῖς Ἰσραὴλ ἐν κατασχέσει.
\vs{13}Καὶ ὄψῃ αὐτὴν, καὶ προστεθήσῃ πρὸς τὸν λαόν σου καὶ σὺ, καθὰ προσετέθη Ἀαρὼν ὁ ἀδελφός σου ἐν Ὢρ τῷ ὄρει.
\vs{14}Διότι παρέβητε τὸ ῥῆμά μου ἐν τῇ ἐρήμῳ Σὶν, ἐν τῷ ἀντιπίπτειν τὴν συναγωγὴν ἁγιάσαι με, οὐχ ἡγιάσατέ με ἐπὶ τῷ ὕδατι ἔναντι αὐτῶν· τοῦτʼ ἔστι τὸ ὕδωρ ἀντιλογίας ἐν Κάδης ἐν τῇ ἐρήμῳ Σίν.
\vs{15}Καὶ εἶπε Μωυσῆς πρὸς Κύριον,
\vs{16}ἐπισκεψάσθω Κύριος ὁ Θεὸς τῶν πνευμάτων καὶ πάσης σαρκὸς ἄνθρωπον ἐπὶ τῆς συναγωγῆς ταύτης,
\vs{17}ὅστις ἐξελεύσεται πρὸ προσώπου αὐτῶν, καὶ ὅστις εἰσελεύσεται πρὸ προσώπου αὐτῶν, καὶ ὅστις ἐξάξει αὐτοὺς, καὶ ὃστις εἰσάξει αὐτοὺς, καὶ οὐκ ἔσται ἡ συναγωγὴ Κυρίου ὡσεὶ πρόβατα οἷς οὐκ ἔστι ποιμήν.
\vs{18}Καὶ ἐλάλησε Κύριος πρὸς Μωυσῆν, λέγων, λάβε πρὸς σεαυτὸν Ἰησοῦν υἱὸν Ναυὴ, ἄνθρωπον ὃς ἔχει πνεῦμα ἐν ἑαυτῷ, καὶ ἐπιθήσεις τὰς χεῖράς σου ἐπʼ αὐτόν·
\vs{19}Καὶ στήσεις αὐτὸν ἔναντι Ἐλεάζαρ τοῦ ἱερέως, καὶ ἐντελῇ αὐτῷ ἔναντι πάσης συναγωγῆς, καὶ ἐντελῇ περὶ αὐτοῦ ἐναντίον αὐτῶν.
\vs{20}Καὶ δώσεις τῆς δόξης σου ἐπʼ αὐτὸν, ὅπως ἂν εἰσακούσωσιν αὐτοῦ οἱ υἱοὶ Ἰσραήλ.
\vs{21}Καὶ ἔναντι Ἐλεάζαρ τοῦ ἱερέως στήσεται, καὶ ἐπερωτήσουσιν αὐτὸν τὴν κρίσιν τῶν δήλων ἔναντι Κυρίου· ἐπὶ τῷ στόματι αὐτοῦ ἐξελεύσονται, καὶ ἐπὶ τῷ στόματι αὐτοῦ εἰσελεύσονται αὐτὸς καὶ οἱ υἱοὶ Ἰσραὴλ ὁμοθυμαδὸν, καὶ πᾶσα ἡ συναγωγή.

\vs{22}Καὶ ἐποίησε Μωυσῆς καθὰ ἐνετείλατο αὐτῷ Κύριος· καὶ λαβὼν τὸν Ἰησοῦν, ἔστησεν αὐτὸν ἐναντίον Ἐλεάζαρ τοῦ ἱερέως, καὶ ἐναντίον πάσης συναγωγῆς,
\vs{23}καὶ ἐπέθηκε τὰς χεῖρας αὐτοῦ ἐπʼ αὐτὸν, καὶ συνέστησεν αὐτὸν καθάπερ συνέταξε Κύριος τῷ Μωυσῇ.

\ch{28}
Καὶ ἐλάλησε Κύριος πρὸς Μωυσῆν, λέγων,
\vs{2}ἔντειλαι τοῖς υἱοῖς Ἰσραὴλ, καὶ ἐρεῖς πρὸς αὐτοὺς, λέγων, τὰ δῶρά μου δόματά μου καρπώματά μου εἰς ὀσμὴν εὐωδίας διατηρήσετε προσφέρειν ἐμοὶ ἐν ταῖς ἑορταῖς μου.
\vs{3}Καὶ ἐρεῖς πρὸς αὐτοὺς, ταῦτα τὰ καρπώματα ὅσα προσάξετε Κυρίῳ, ἀμνοὺς ἐνιαυσίους ἀμώμους δύο τὴν ἡμέραν εἰς ὁλοκαύτωσιν ἐνδελεχῶς.
\vs{4}Τὸν ἀμνὸν τὸν ἕνα ποιήσεις τὸ τοπρωῒ, καὶ τὸν ἀμνὸν τὸν δεύτερον ποιήσεις τὸ πρὸς ἑσπέραν.

\vs{5}Καὶ ποιήσεις τὸ δέκατον τοῦ οἰφὶ σεμίδαλιν εἰς θυσίαν ἀναπεποιημένην ἐν ἐλαίῳ ἐν τετάρτῳ τοῦ ἴν.
\vs{6}Ὁλοκαύτωμα ἐνδελεχισμοῦ, ἡ γενομένη ἐν τῷ ὄρει Σινᾷ εἰς ὀσμὴν εὐωδίας Κυρίῳ.
\vs{7}Καὶ σπονδὴν αὐτοῦ τὸ τέταρτον τοῦ ἴν τῷ ἀμνῷ τῷ ἑνί· ἐν τῷ ἁγίῳ σπείσεις σπονδὴν σίκερα Κυρίῳ·
\vs{8}καὶ τὸν ἀμνὸν τὸν δεύτερον ποιήσεις τὸ πρὸς ἑσπέραν· κατὰ τὴν θυσίαν αὐτοῦ καὶ κατὰ τὴν σπονδὴν αὐτοῦ ποιήσετε εἰς ὀσμὴν εὐωδίας Κυρίῳ.
\vs{9}Καὶ τῇ ἡμέρᾳ τῶν σαββάτων προσάξετε δύο ἀμνοὺς ἐνιαυσίους ἀμώμους, καὶ δύο δέκατα σεμιδάλεως ἀναπεποιημένης ἐν ἐλαίῳ εἰς θυσίαν καὶ σπονδὴν,
\vs{10}ὁλοκαύτωμα σαββάτων ἐν τοῖς σαββάτοις ἐπὶ τὴς ὁλοκαυτώσεως τῆς διαπαντὸς, καὶ τὴν σπονδὴν αὐτοῦ.

\vs{11}Καὶ ἐν ταῖς νεομηνίαις προσάξετε ὁλοκαύτωμα τῷ Κυρίῳ, μόσχους ἐκ βοῶν δύο, καὶ κριὸν ἕνα, ἀμνοὺς ἐνιαυσίους ἑπτὰ ἀμώμους·
\vs{12}Τρία δέκατα σεμιδάλεως ἀναπεποιημένης ἐν ἐλαίῳ τῷ μόσχῳ τῷ ἑνὶ, καὶ δύο δέκατα σεμιδάλεως ἀναπεποιημένης ἐν ἐλαίῳ τῷ κριῷ τῷ ἑνί·
\vs{13}Δέκατον δέκατον σεμιδάλεως ἀναπεποιημένης ἐν ἐλαίῳ τῷ ἀμνῷ τῷ ἑνὶ, θυσίαν ὀσμὴν εὐωδίας κάρπωμα Κυρίῳ.
\vs{14}Ἡ σπονδὴ αὐτῶν τὸ ἥμισυ τοῦ ἴν ἔσται τῷ μόσχῳ τῷ ἑνί· καὶ τὸ τρίτον τοῦ ἴν ἔσται τῷ κριῷ τῷ ἑνί. Καὶ τὸ τέταρτον τοῦ ἴν ἔσται τῷ ἀμνῷ τῷ ἑνὶ οἴνου· τοῦτο τὸ ὁλοκαύτωμα μῆνα ἐκ μηνὸς εἰς τοὺς μῆνας τοῦ ἐνιαυτοῦ.

\vs{15}Καὶ χίμαρον ἐξ αἰγῶν ἕνα περὶ ἁμαρτίας Κυρίῳ, ἐπὶ τῆς ὁλοκαυτώσεως τῆς διαπαντὸς ποιηθήσεται, καὶ ἡ σπονδὴ αὐτοῦ.

\vs{16}Καὶ ἐν τῷ μηνὶ τῷ πρώτῳ τεσσαρεσκαιδεκάτῃ ἡμέρᾳ τοῦ μηνὸς πάσχα Κυρίῳ.
\vs{17}Καὶ τῇ πεντεκαιδεκάτῃ ἡμέρᾳ τοῦ μηνὸς τούτου ἑορτή· ἑπτὰ ἡμέρας ἄζυμα ἔδεσθε.
\vs{18}Καὶ ἡ ἡμέρα ἡ πρώτη ἐπίκλητος ἁγία ἔσται ὑμῖν· πᾶν ἔργον λατρευτὸν οὐ ποιήσετε.
\vs{19}Καὶ προσάξετε ὁλοκαυτώματα κάρπωμα Κυρίῳ, μόσχους ἐκ βοῶν δύο, κριὸν ἕνα, ἀμνοὺς ἐνιαυσίους ἑπτά· ἄμωμοι ἔσονται ὑμῖν.
\vs{20}Καὶ θυσία αὐτῶν σεμίδαλις ἀναπεποιημένη ἐν ἐλαίῳ· τρία δέκατα τῷ μόσχῳ τῷ ἑνὶ, καὶ δύο δέκατα τῷ κριῷ τῷ ἑνί.
\vs{21}Δέκατον δέκατον ποιήσεις τῷ ἀμνῷ τῷ ἑνὶ, τοῖς ἑπτὰ ἀμνοῖς.
\vs{22}Καὶ χίμαρον ἐξ αἰγῶν ἕνα περὶ ἁμαρτίας, ἐξιλάσασθαι περὶ ὑμῶν·
\vs{23}Πλὴν τῆς ὁλοκαυτώσεως τῆς διαπαντὸς τῆς πρωϊνῆς, ὅ ἐστιν ὁλοκαύτωμα ἐνδελεχισμοῦ.
\vs{24}Ταῦτα κατὰ ταῦτα ποιήσετε τὴν ἡμέραν εἰς τὰς ἑπτὰ ἡμέρας, δῶρον κάρπωμα εἰς ὀσμὴν εὐωδίας Κυρίῳ, ἐπὶ τοῦ ὁλοκαυτώματος τοῦ διαπαντὸς ποιήσεις τὴν σπονδὴν αὐτοῦ.
\vs{25}Καὶ ἡμέρα ἡ ἑβδόμη κλητὴ ἁγία ἔσται ὑμῖν· πᾶν ἔργον λατρευτὸν οὐ ποιήσετε ἐν αὐτῇ.

\vs{26}Καὶ τῇ ἡμέρᾳ τῶν νέων, ὅταν προσφέρητε θυσίαν νέαν Κυρίῳ τῶν ἑβδομάδων, ἐπίκλητος ἁγία ἔσται ὑμῖν· πᾶν ἔργον λατρευτὸν οὐ ποιήσετε.
\vs{27}Καὶ προσάξετε ὁλοκαυτώματα εἰς ὀσμὴν εὐωδίας Κυρίῳ, μόσχους ἐκ βοῶν δύο, κριὸν ἕνα, ἀμνοὺς ἐνιαυσίους ἑπτὰ ἀμώμους.
\vs{28}Ἡ θυσία αὐτῶν σεμίδαλις ἀναπεποιημένη ἐν ἐλαίῳ· τρία δέκατα τῷ μόσχῳ τῷ ἑνὶ, καὶ δύο δέκατα τῷ κριῷ τῷ ἑνί.
\vs{29}Δέκατον δέκατον τῷ ἀμνῷ τῷ ἑνὶ, τοῖς ἑπτὰ ἀμνοῖς·
\vs{30}καὶ χίμαρον ἐξ αἰγῶν ἕνα περὶ ἁμαρτίας, ἐξιλάσασθαι περὶ ὑμῶν· πλὴν τοῦ ὁλοκαυτώματος τοῦ διαπαντός·
\vs{31}καὶ τὴν θυσίαν αὐτῶν ποιήσετέ μοι, ἄμωμοι ἔσονται ὑμῖν, καὶ τὰς σπονδὰς αὐτῶν.

\ch{29}
Καὶ τῷ μηνὶ τῷ ἑβδόμῳ, μιᾷ τοῦ μηνὸς, ἐπίκλητος ἁγία ἔσται ὑμῖν· πᾶν ἔργον λατρευτὸν οὐ ποιήσετε· ἡμέρα σημασίας ἔσται ὑμῖν.
\vs{2}Καὶ ποιήσετε ὁλοκαυτώματα εἰς ὀσμὴν εὐωδίας Κυρίῳ, μόσχον ἕνα ἐκ βοῶν, κριὸν ἕνα, ἀμνοὺς ἐνιαυσίους ἑπτὰ ἀμώμους.
\vs{3}Ἡ θυσία αὐτῶν σεμίδαλις ἀναπεποιημένη ἐν ἐλαίῳ· τρία δέκατα τῷ μόσχῳ τῷ ἑνὶ, καὶ δύο δέκατα τῷ κρίῳ τῷ ἑνί·
\vs{4}Δέκατον δέκατον τῷ ἀμνῷ τῷ ἑνὶ, τοῖς ἑπτὰ ἀμνοῖς·
\vs{5}Καὶ χίμαρον ἐξ αἰγῶν ἕνα περὶ ἁμαρτίας, ἐξιλάσασθαι περὶ ὑμῶν·
\vs{6}Πλὴν τῶν ὁλοκαυτωμάτων τῆς νουμηνίας· καὶ αἱ θυσίαι αὐτῶν, καὶ αἱ σπονδαὶ αὐτῶν, καὶ τὸ ὁλοκαύτωμα τὸ διαπαντός· καί αἱ θυσίαι αὐτῶν καὶ αἱ σπονδαὶ αὐτῶν κατὰ τὴν σύγκρισιν αὐτῶν εἰς ὀσμὴν εὐωδίας Κυρίῳ.

\vs{7}Καὶ τῇ δεκάτῃ τοῦ μηνὸς τούτου ἐπίκλητος ἁγία ἔσται ὑμῖν· καὶ κακώσετε τὰς ψυχὰς ὑμῶν, καὶ πᾶν ἔργον οὐ ποιήσετε.
\vs{8}Καὶ προσοίσετε ὁλοκαυτώματα εἰς ὀσμὴν εὐωδίας Κυρίῳ, καρπώματα Κυρίῳ, μόσχον ἐκ βοῶν ἕνα, κριὸν ἕνα, ἀμνους ἐνιαυσίους ἑπτά· ἄμωμοι ἔσονται ὑμῖν.
\vs{9}Ἡ θυσία αὐτῶν σεμίδαλις ἀναπεποιημένη ἐν ἐλαίῳ· τρία δέκατα τῷ μόσχῳ τῷ ἑνὶ, καὶ δύο δέκατα τῷ κριῷ τῷ ἑνί·
\vs{10}Δέκατον δέκατον τῷ ἀμνῷ τῷ ἑνὶ, εἰς τοὺς ἑπτὰ ἀμνούς·
\vs{11}Καὶ χίμαρον ἐξ αἰγῶν ἕνα περὶ ἁμαρτίας, ἐξιλάσασθαι περὶ ὑμῶν· πλὴν τὸ περὶ τῆς ἁμαρτίας τῆς ἐξιλάσεως, καὶ ἡ ὁλοκαύτωσις ἡ διαπαντός· ἡ θυσία αὐτῆς, καὶ ἡ σπονδὴ αὐτῆς κατὰ τὴν σύγκρισιν εἰς ὀσμὴν εὐωδίας κάρπωμα Κυρίῳ.

\vs{12}Καὶ τῇ πεντεκαιδεκάτῃ ἡμέρᾳ τοῦ μηνὸς τοῦ ἑβδόμου τούτου ἐπίκλητος ἁγία ἔσται ὑμῖν· πᾶν ἔργον λατρευτὸν οὐ ποιήσετε· καὶ ἑορτάσατε αὐτὴν ἑορτὴν Κυρίῳ ἑπτὰ ἡμέρας.
\vs{13}Καὶ προσάξετε ὁλοκαυτώματα κάρπωμα εἰς ὀσμὴν εὐωδίας Κυρίῳ, τῇ ἡμέρᾳ τῇ πρώτῃ μόσχους ἐκ βοῶν τρεῖς καὶ δέκα, κριοὺς δυο, ἀμνοὺς ἐνιαυσίους δεκατέσσαρας· ἄμωμοι ἔσονται.
\vs{14}Αἱ θυσίαι αὐτῶν σεμίδαλις ἀναπεποιημένη ἐν ἐλαίῳ· τρία δέκατα τῷ μόσχῳ τῷ ἑνὶ, τοῖς τρισκαίδεκα μόσχοις· καὶ δύο δέκατα τῷ κριῷ τῷ ἑνὶ, ἐπὶ τοὺς δύο κριούς·
\vs{15}Δέκατον δέκατον τῷ ἀμνῷ τῷ ἑνὶ, ἐπὶ τοὺς τέσσαρας καὶ δέκα ἀμνούς·
\vs{16}Καὶ χίμαρον ἐξ αἰγῶν ἕνα περὶ ἁμαρτίας· πλὴν τῆς ὁλοκαυτώσεως τῆς διαπαντός· αἱ θυσίαι αὐτῶν καὶ αἱ σπονδαὶ αὐτῶν.

\vs{17}Καὶ τῇ ἡμέρᾳ τῇ δευτέρᾳ μόσχους δώδεκα, κριοὺς δύο, ἀμνοὺς ἐνιαυσίους τέσσαρας καὶ δέκα ἀμώμους.
\vs{18}Ἡ θυσία αὐτῶν καὶ ἡ σπονδὴ αὐτῶν τοῖς μόσχοις καὶ τοῖς κριοῖς καὶ τοῖς ἀμνοῖς κατὰ ἀριθμὸν αὐτῶν, κατὰ τὴν σύγκρισιν αὐτῶν.
\vs{19}καὶ χίμαρον ἐξ αἰγῶν ἕνα περὶ ἁμαρτίας· πλὴν τῆς ὁλοκαυτώσεως τῆς διαπαντός· αἱ θυσίαι αὐτῶν καὶ αἱ σπονδαὶ αὐτῶν.

\vs{20}Τῇ ἡμέρᾳ τῇ τρίτῃ μόσχους ἕνδεκα, κριοὺς δύο, ἀμνοὺς ἐνιαυσίους τέσσαρας καὶ δέκα ἀμώμους.
\vs{21}Ἡ θυσία αὐτῶν καὶ ἡ σπονδὴ αὐτῶν τοῖς μόσχοις καὶ τοῖς κριοῖς καὶ τοῖς ἀμνοῖς κατὰ ἀριθμὸν αὐτῶν, κατὰ τὴν σύγκρισιν αὐτῶν.
\vs{22}Καὶ χίμαρον ἐξ αἰγῶν ἕνα περὶ ἁμαρτίας· πλὴν τῆς ὁλοκαυτώσεως τῆς διαπαντός· αἱ θυσίαι αὐτῶν καὶ αἱ σπονδαὶ αὐτῶν.

\vs{23}Τῇ ἡμέρᾳ τῇ τετάρτῃ μόσχους δέκα, κριοὺς δύο, ἀμνοὺς ἐνιαυσίους τέσσαρας καὶ δέκα ἀμώμους.
\vs{24}Αἱ θυσίαι αὐτῶν καὶ αἱ σπονδαὶ αὐτῶν τοῖς μόσχοις καὶ τοῖς κριοῖς καὶ τοῖς ἀμνοῖς κατὰ ἀριθμὸν αὐτῶν, κατὰ τὴν σύγκρισιν αὐτῶν.
\vs{25}Καὶ χίμαρον ἐξ αἰγῶν ἕνα περὶ ἁμαρτίας· πλὴν τῆς ὁλοκαυτώσεως τῆς διαπαντός· αἱ θυσίαι αὐτῶν καὶ αἱ σπονδαὶ αὐτῶν.

\vs{26}Τῇ ἡμέρᾳ τῇ πέμπτῃ μόσχους ἐννέα, κριοὺς δύο, ἀμνοὺς ἐνιαυσίους τέσσαρας καὶ δέκα ἀμώμους.
\vs{27}Αἱ θυσίαι αὐτῶν καὶ αἱ σπονδαὶ αὐτῶν τοῖς μόσχοις καὶ τοῖς κριοῖς καὶ τοῖς ἀμνοῖς κατὰ ἀριθμὸν αὐτῶν, κατὰ τὴν σύγκρισιν αὐτῶν.
\vs{28}Καὶ χίμαρον ἐξ αἰγῶν ἕνα περὶ ἁμαρτίας πλὴν τῆς ὁλοκαυτώσεως τῆς διαπαντός· αἱ θυσίαι αὐτῶν καὶ αἱ σπονδαὶ αὐτῶν.

\vs{29}Τῇ ἡμέρᾳ τῇ ἕκτῃ μόσχους ὀκτὼ, κριοὺς δύο, ἀμνοὺς ἐνιαυσίους δεκατέσσαρας ἀμώμους.
\vs{30}Αἱ θυσίαι αὐτῶν καὶ αἱ σπονδαὶ αὐτῶν τοῖς μόσχοις καὶ τοῖς κριοῖς καὶ τοῖς ἀμνοῖς κατὰ ἀριθμὸν αὐτῶν, κατὰ τὴν σύγκρισιν αὐτῶν.
\vs{31}Καὶ χίμαρον ἐξ αἰγῶν ἕνα περὶ ἁμαρτίας· πλὴν τῆς ὁλοκαυτώσεως τῆς διαπαντός· αἱ θυσίαι αὐτῶν καὶ αἱ σπονδαὶ αὐτῶν.

\vs{32}Τῇ ἡμέρᾳ τῇ ἑβδομῃ μόσχους ἑπτὰ, κριοὺς δύο, ἀμνοὺς ἐνιαυσίους δεκατέσσαρας ἀμώμους.
\vs{33}Αἱ θυσίαι αὐτῶν καὶ αἱ σπονδαὶ αὐτῶν τοῖς μόσχοις καὶ τοῖς κριοῖς καὶ τοῖς ἀμνοῖς κατὰ ἀριθμὸν αὐτῶν, κατὰ τὴν σύγκρισιν αὐτῶν.
\vs{34}Καὶ χίμαρον ἐξ αἰγῶν ἕνα περὶ ἁμαρτίας· πλὴν τῆς ὁλοκαυτώσεως τῆς διαπαντός· αἱ θυσίαι αὐτῶν καὶ αἱ σπονδαὶ αὐτῶν.
\vs{35}Καὶ τῇ ἡμέρᾳ τῇ ὀγδόῃ ἐξόδιον ἔσται ὑμῖν· πᾶν ἔργον λατρευτὸν οὐ ποιησετε ἐν αὐτῇ.
\vs{36}Καὶ προσάξετε ὁλοκαυτώματα εἰς ὀσμὴν εὐωδίας καρπώματα τῷ Κυρίῳ, μόσχον ἕνα, κριὸν ἕνα, ἀμνοὺς ἐνιαυσίους ἑπτὰ ἀμώμους.
\vs{37}Αἱ θυσίαι αὐτῶν καὶ αἱ σπονδαὶ αὐτῶν τῷ μόσχῳ καὶ τῷ κριῷ καὶ τοῖς ἀμνοῖς κατὰ ἀριθμὸν αὐτῶν, κατὰ τὴν σύνκρισιν αὐτῶν.
\vs{38}Καὶ χίμαρον ἐξ αἰγῶν ἕνα περὶ ἁμαρτίας· πλὴν τῆς ὁλοκαυτώσεως τῆς διαπαντός· αἱ θυσίαι αὐτῶν καὶ αἱ σπονδαὶ αὐτῶν.

\vs{39}Ταῦτα ποιήσετε Κυρίῳ ἐν ταῖς ἑορταῖς ὑμῶν, πλὴν τῶν εὐχῶν ὑμῶν, καὶ τὰ ἑκούσια ὑμῶν, καὶ τὰ ὁλοκαυτώματα ὑμῶν, καὶ τὰς θυσίας ὑμῶν, καὶ τὰς σπονδὰς ὑμῶν, καὶ τὰ σωτήρια ὑμῶν.

\ch{30}
Καὶ ἐλάλησε Μωυσῆς τοῖς υἱοῖς Ἰσραὴλ κατὰ πάντα ὅσα ἐνετείλατο Κύριος τῷ Μωυσῇ.
\vs{2}Καὶ ἐλάλησε Μωυσῆς πρὸς τοὺς ἄρχοντας τῶν φυλῶν υἱῶν Ἰσραὴλ, λέγων, τοῦτο τὸ ῥῆμα ὃ συνέταξε Κύριος.
\vs{3}Ἄνθρωπος ἄνθρωπος ὃς ἂν εὔξηται εὐχὴν Κυρίῳ, ἢ ὀμόσῃ ὅρκον, ἢ ὁρίσηται ὁρισμῷ περὶ τῆς ψυχῆς αὐτοῦ, οὐ βεβηλώσει τὸ ῥῆμα αὐτοῦ· πάντα ὅσα ἂν ἐξέλθῃ ἐκ τοῦ στόματος αὐτοῦ, ποιήσει.
\vs{4}Ἐὰν δὲ εὔξηται γυνὴ εὐχὴν Κυρίῳ, ἢ ὁρίσηται ὁρισμὸν ἐν τῷ οἴκῳ τοῦ πατρὸς αὐτῆς ἐν τῇ νεότητι αὐτῆς,
\vs{5}καὶ ἀκούσῃ ὁ πατὴρ αὐτῆς τὰς εὐχὰς αὐτῆς, καὶ τοὺς ὁρισμοὺς αὐτῆς, οὓς ὡρίσατο κατὰ τῆς ψυχῆς αὐτῆς, καὶ παρασιωπήσῃ αὐτῆς ὁ πατὴρ, καὶ στήσονται πᾶσαι αἱ εὐχαὶ αὐτῆς, καὶ πάντες οἱ ὁρισμοὶ οὓς ὡρίσατο κατὰ τῆς ψυχῆς αὐτῆς, μενοῦσιν αὐτῇ·
\vs{6}Ἐὰν δὲ ἀνανεύων ἀνανεύσῃ ὁ πατὴρ αὐτῆς, ᾗ ἂν ἡμέρᾳ ἀκούσῃ πάσας τὰς εὐχὰς αὐτῆς καὶ τοὺς ὁρισμοὺς, οὓς ὡρίσατο κατὰ τῆς ψυχῆς αὐτῆς, οὐ στήσονται· καὶ Κύριος καθαριεῖ αὐτὴν, ὅτι ἀνένευσεν ὁ πατὴρ αὐτῆς.

\vs{7}Ἐὰν δὲ γενομένη γένηται ἀνδρὶ, καὶ αἱ εὐχαὶ αὐτῆς ἐπʼ αὐτῇ κατὰ τὴν διαστολὴν τῶν χειλέων αὐτῆς, οὓς ὡρίσατο κατὰ τῆς ψυχῆς αὐτῆς,
\vs{8}καὶ ἀκούσῃ ὁ ἀνὴρ αὐτῆς, καὶ παρασιωπήσῃ αὐτῇ ᾗ ἂν ἡμέρᾳ ἀκούσῃ, καὶ οὕτω στήσονται πᾶσαι αἱ εὐχαὶ αὐτῆς, καὶ οἱ ὁρισμοὶ αὐτῆς, οὓς ὡρίσατο κατὰ τῆς ψυχῆς αὐτῆς, στήσονται.
\vs{9}Ἐὰν δὲ ἀνανεύων ἀνανεύσῃ ὁ ἀνὴρ αὐτῆς ᾗ ἐὰν ἡμέρᾳ ἀκούσῃ, πᾶσαι αἱ εὐχαὶ αὐτῆς, καὶ οἱ ὁρισμοὶ αὐτῆς οὓς ὡρίσατο κατὰ τῆς ψυχῆς αὐτῆς, οὐ μενοῦσιν, ὅτι ὁ ἀνὴρ ἀνένευσεν ἀπʼ αὐτῆς· καὶ Κύριος καθαριεῖ αὐτήν.

\vs{10}Καὶ εὐχὴ χήρας καὶ ἐκβεβλημένης ὅσα ἐὰν εὔξηται κατὰ τῆς ψυχῆς αὐτῆς, μενοῦσιν αὐτῇ.
\vs{11}Ἐὰν δὲ ἐν τῷ οἴκῳ τοῦ ἀνδρὸς αὐτῆς ἡ εὐχὴ αὐτῆς, ἢ ὁ ὁρισμὸς κατὰ τῆς ψυχῆς αὐτῆς μεθʼ ὅρκου,
\vs{12}καὶ ἀκούσῃ ὁ ἀνὴρ αὐτῆς, καὶ παρασιωπήσῃ αὐτῇ, καὶ μὴ ἀνανεύσῃ αὐτῇ, καὶ στήσονται πᾶσαι αἱ εὐχαὶ αὐτῆς, καὶ πάντες οἱ ὁρισμοὶ αὐτῆς οὓς ὡρίσατο κατὰ τῆς ψυχῆς αὐτῆς, στήσονται κατʼ αὐτῆς.
\vs{13}Ἐὰν δὲ περιελὼν περιέλῃ ὁ ἀνὴρ αὐτῆς ᾗ ἂν ἡμέρᾳ ἀκούσῃ, πάντα ὅσα ἐὰν ἐξέλθῃ ἐκ τῶν χειλέων αὐτῆς κατὰ τὰς εὐχὰς αὐτῆς, καὶ κατὰ τοὺς ὁρισμοὺς τοὺς κατὰ τῆς ψυχῆς αὐτῆς, οὐ μενεῖ αὐτῇ· ὁ ἀνὴρ αὐτῆς περιεῖλε, καὶ Κύριος καθαριεῖ αὐτήν.
\vs{14}Πᾶσα εὐχὴ καὶ πᾶς ὅρκος δεσμοῦ κακῶσαι ψυχὴν, ὁ ἀνὴρ αὐτῆς στήσει αὐτῇ, καὶ ὁ ἀνὴρ αὐτῆς περιελεῖ
\vs{15}Ἐὰν δὲ σιωπῶν παρασιωπήσῃ αὐτῇ ἡμέραν ἐξ ἡμέρας, καὶ στήσει αὐτῇ πάσας τὰς εὐχὰς αὐτῆς, καὶ τοὺς ὁρισμοὺς τοὺς ἐπʼ αὐτῆς στήσει αὐτῇ, ὅτι ἐσιώπησεν αὐτῇ τῇ ἡμέρᾳ ᾗ ἤκουσεν.
\vs{16}Ἐὰν δὲ περιελὼν περιέλῃ ὁ ἀνὴρ αὐτῆς μετὰ τὴν ἡμέραν ἣν ἤκουσε, καὶ λήψεται τὴν ἁμαρτίαν αὐτοῦ.
\vs{17}Ταῦτα τὰ δικαιώματα ὅσα ἐνετείλατο Κύριος τῷ Μωυσῇ, ἀναμέσον ἀνδρὸς καὶ γυναικὸς αὐτοῦ, καὶ ἀναμέσον πατρὸς καὶ θυγατρὸς ἐν νεότητι ἐ οἴκῳ πατρός.

\ch{31}
Καὶ ἐλάλησε Κύριος πρὸς Μωυσῆν, λέγων,
\vs{2}ἐκδίκει τὴν ἐκδίκησιν υἱῶν Ἰσραὴλ ἐκ τῶν Μαδιανιτῶν, καὶ ἔσχατον προστεθήσῃ πρὸς τὸν λαόν σου.
\vs{3}Καὶ ἐλάλησε Μωυσῆς πρὸς τὸν λαὸν, λέγων, ἐξοπλίσατε ἐξ ὑμῶν ἄνδρας, καὶ παρατάξασθε ἔναντι Κυρίου ἐπὶ Μαδιὰν, ἀποδοῦναι ἐκδίκησιν παρὰ τοῦ Κυρίου τῇ Μαδιάν.
\vs{4}Χιλίους ἐκ φυλῆς, χιλίους ἐκ φυλῆς, ἐκ πασῶν φυλῶν υἱῶν Ἰσραὴλ, ἀποστείλατε παρατάξασθαι.
\vs{5}Καὶ ἐξηρίθμησαν ἐκ τῶν χιλιάδων Ἰσραὴλ χιλίους ἐκ φυλῆς, δώδεκα χιλιάδας ἐνωπλισμένοι εἰς παράταξιν.
\vs{6}Καὶ ἀπέστειλεν αὐτοὺς Μωυσῆς χιλίους ἐκ φυλῆς, χιλίους ἐκ φυλῆς σὺν δυνάμει αὐτῶν, καὶ Φινεὲς υἱὸν Ἐλεάζαρ υἱοῦ Ἀαρὼν τοῦ ἱερέως· καὶ τὰ σκεύη τὰ ἅγια, καὶ αἱ σάλπιγγες τῶν σημασιῶν ἐν ταῖς χερσὶν αὐτῶν.

\vs{7}Καὶ παρετάξαντο ἐπὶ Μαδιὰν, καθὰ ἐνετείλατο Κύριος Μωυσῇ· καὶ ἀπέκτειναν πᾶν ἀρσενικόν.
\vs{8}Καὶ τοὺς βασιλεῖς Μαδιὰν ἀπέκτειναν ἅμα τοῖς τραυματίαις αὐτῶν· καὶ τὸν Εὐὶν, καὶ τὸν Ῥοκὸν, καὶ τὸν Σοὺρ, καὶ τὸν Οὒρ, καὶ τὸν Ῥοβὸκ, πέντε βασιλεῖς Μαδιάν· καὶ τὸν Βαλαὰμ υἱὸν Βεὼρ ἀπέκτειναν ἐν ῥομφαίᾳ σὺν τοῖς τραυματίαις αὐτῶν·
\vs{9}Καὶ ἐπρονόμευσαν τὰς γυναῖκας Μαδιὰν, καὶ τὴν ἀποσκευὴν αὐτῶν, καὶ τὰ κτήνη αὐτῶν, καὶ πάντα τὰ ἔγκτητα αὐτῶν· καὶ τὴν δύναμιν αὐτῶν ἐπρονόμευσαν·
\vs{10}Καὶ πάσας τὰς πόλεις αὐτῶν τὰς ἐν ταῖς κατοικίαις αὐτῶν, καὶ τὰς ἐπαύλεις αὐτῶν ἐνέπρησαν ἐν πυρί.
\vs{11}Καὶ ἔλαβον πᾶσαν τὴν προνομὴν αὐτῶν, καὶ πάντα τὰ σκῦλα αὐτῶν ἀπὸ ἀνθρώπου ἕως κτήνους.
\vs{12}Καὶ ἤγαγον πρὸς Μωυσῆν καὶ πρὸς Ἐλεάζαρ τὸν ἱερέα, καὶ πρὸς πάντας υἱοὺς Ἰσραὴλ, τὴν αἰχμαλωσίαν, καὶ τὰ σκύλα, καὶ τὴν προνομὴν εἰς τὴν παρεμβολὴν εἰς Ἀραβὼθ Μωὰβ, ἥ ἐστιν ἐπὶ τοῦ Ἰορδάνου κατὰ Ἰεριχώ.
\vs{13}Καὶ ἐξῆλθε Μωυσῆς καὶ Ἐλεάζαρ ὁ ἱερεὺς καὶ πάντες οἱ ἄρχοντες τῆς συναγωγῆς εἰς συνάντησιν αὐτοῖς ἔξω τῆς παρεμβολῆς.
\vs{14}Καὶ ὠργίσθη Μωυσῆς ἐπὶ τοῖς ἐπισκόποις τῆς δυνάμεως, χιλιάρχοις καὶ ἑκατοντάρχοις τοῖς ἐρχομένοις ἐκ τῆς παρατάξεως τοῦ πολέμου.
\vs{15}Καὶ εἶπεν αὐτοῖς Μωυσῆς, ἱνατί ἐζωγρήσατε πᾶν θῆλυ;
\vs{16}Αὗται γὰρ ἦσαν τοῖς υἱοῖς Ἰσραὴλ κατὰ τὸ ῥῆμα Βαλαὰμ τοῦ ἀποστῆσαι καὶ ὑπεριδεῖν τὸ ῥῆμα Κυρίου, ἕνεκεν Φογώρ· καὶ ἐγένετο ἡ πληγὴ ἐν τῇ συναγωγῇ Κυρίου.
\vs{17}Καὶ νῦν ἀποκτείνατε πᾶν ἀρσενικὸν ἐν πάσῃ τῇ ἀπαρτίᾳ, πᾶσαν γυναῖκα, ἥτις ἔγυω κοίτην ἄρσενος, ἀποκτείνατε.
\vs{18}Καὶ πᾶσαν τὴν ἀπαρτίαν τῶν γυναικῶν, ἥτις οὐκ οἶδε κοίτην ἄρσενος, ζωγρήσατε αὐτάς.
\vs{19}Καὶ ὑμεῖς παρεμβάλετε ἔξω τῆς παρεμβολῆς ἐπτὰ ἡμέρας· πᾶς ὁ ἀνελὼν καὶ ὁ ἁπτόμενος τοῦ τετρωμένου ἁγνισθήσεται τῇ ἡμέρᾳ τῇ τρίτῃ, καὶ τῇ ἡμέρᾳ τῇ ἑβδόμῃ ὑμεῖς καὶ ἡ αἰχμαλωσία ὑμῶν.
\vs{20}Καὶ πᾶν περίβλημα καὶ πᾶν σκεῦος δερμάτινον, καὶ πᾶσαν ἐργασίαν ἐξ αἰγείας, καὶ πᾶν σκεῦος ξύλινον ἀφαγνιεῖτε.

\vs{21}Καὶ εἶπεν Ἐλεάζαρ ὁ ἱερεὺς πρὸς τοὺς ἄνδρας τῆς δυνάμεως τοὺς ἐρχομένους ἐκ τῆς παρατάξεως τοῦ πολέμου, τοῦτο τὸ δικαίωμα τοῦ νόμου ὃ συνέταξε Κύριος τῷ Μωυσῇ.
\vs{22}Πλὴν τοῦ χρυσίου καὶ τοῦ ἀργυρίου καὶ χαλκοῦ καὶ σιδήρου καὶ μολίβου καὶ κασσιτέρου,
\vs{23}πᾶν πρᾶγμα ὃ διελεύσεται ἐν πυρὶ, καὶ καθαρισθήσεται, ἀλλʼ ἢ τῷ ὕδατι τοῦ ἁγνισμοῦ ἁγνισθήσεται· καὶ πάντα ὅσα ἐὰν μὴ διαπορεύηται διὰ πυρὸς, διελεύσεται διʼ ὕδατος.
\vs{24}Καὶ πλυνεῖσθε τὰ ἱμάτια τῇ ἡμέρᾳ τῇ ἑβδόμῃ, καὶ καθαρισθήσεσθε· καὶ μετὰ ταῦτα εἰσελεύσεσθε εἰς τὴν παρεμβολήν.

\vs{25}Καὶ ἐλάλησε Κύριος πρὸς Μωυσῆν, λέγων,
\vs{26}λάβε τὸ κεφάλαιον τῶν σκύλων τῆς αἰχμαλωσίας ἀπὸ ἀνθρώπου ἕως κτήνους σὺ καὶ Ἐλεάζαρ ὁ ἱερεὺς καὶ οἱ ἄρχοντες τῶν πατριῶν τῆς συναγωγῆς.
\vs{27}Καὶ διελεῖτε τὰ σκῦλα ἀναμέσον τῶν πολεμιστῶν τῶν ἐκπεπορευμένων εἰς τὴν παράταξιν, καὶ ἀναμέσον πάσης συναγωγῆς.
\vs{28}Καὶ ἀφελεῖτε τέλος Κυρίῳ παρὰ τῶν ἀνθρώπων τῶν πολεμιστῶν τῶν ἐκπεπορευμένων εἰς τὴν παράταξιν, μίαν ψυχὴν ἀπὸ πεντακοσίων, ἀπὸ τῶν ἀνθρώπων, καὶ ἀπὸ τῶν κτηνῶν, καὶ ἀπὸ τῶν βοῶν, καὶ ἀπὸ τῶν προβάτων, καὶ ἀπὸ τῶν ὄνων·
\vs{29}καὶ ἀπὸ τοῦ ἡμίσους αὐτῶν λήψεσθε. Καὶ δώσεις Ἐλεάζαρ τῷ ἱερεῖ τὰς ἀπαρχὰς Κυρίου.
\vs{30}Καὶ ἀπὸ τοῦ ἡμίσους τοῦ τῶν υἱῶν Ἰσραὴλ λήψῃ ἕνα ἀπὸ πεντήκοντα ἀπὸ τῶν ἀνθρώπων, καὶ ἀπὸ τῶν βοῶν, καὶ ἀπὸ τῶν προβάτων, καὶ ἀπὸ τῶν ὄνων, καὶ ἀπὸ πάντων τῶν κτηνῶν· καὶ δώσεις αὐτὰ τοῖς Λευίταις τοῖς φυλάσσουσι τὰς φυλακὰς ἐν τῇ σκηνῇ Κυρίου.

\vs{31}Καὶ ἐποίησε Μωυσῆς καὶ Ἐλεάζαρ ὁ ἱερεὺς, καθὰ συνέταξε Κύριος τῷ Μωυσῇ.
\vs{32}Καὶ ἐγενήθη τὸ πλεόνασμα τῆς προνομῆς ὁ προενόμευσαν οἱ ἄνδρες οἱ πολεμισταὶ, ἀπὸ τῶν προβάτων, ἑξακόσιαι χιλιάδες καὶ ἑβδομήκοντα καὶ πέντε χιλιάδες·
\vs{33}Καὶ βόες, δύο καὶ ἑβδομήκοντα χιλιάδες·
\vs{34}Καὶ ὄνοι, μία καὶ ἑξήκοντα χιλιάδες·
\vs{35}Καὶ ψυχαὶ ἀνθρώπων ἀπὸ τῶν γυναικῶν αἳ οὐκ ἔγνωσαν κοίτην ἀνδρὸς, πᾶσαι ψυχαὶ, δύο καὶ τριάκοντα χιλιάδες.
\vs{36}Καὶ ἐγενήθη τὸ ἡμίσευμα ἡ μερὶς τῶν ἐκπεπορευμένων εἰς τὸν πόλεμον ἐκ τοῦ ἀριθμοῦ τῶν προβάτων, τριακόσιαι καὶ τριάκοντα χιλιάδες καὶ ἑπτακισχίλια καὶ πεντακόσια.
\vs{37}Καὶ ἐγένετο τὸ τέλος Κυρίῳ ἀπὸ τῶν προβάτων, ἑξακόσιαι ἑβδομήκοντα πέντε·
\vs{38}Καὶ βόες, ἓξ καὶ τριάκοντα χιλιάδες, καὶ τὸ τέλος Κυρίῳ, δύο καὶ ἑβδομήκοντα·
\vs{39}Καὶ ὄνοι, τριάκοντα χιλιάδες καὶ πεντακόσιοι, καὶ τὸ τέλος Κυρίῳ, εἷς καὶ ἑξήκοντα·
\vs{40}Καὶ ψυχαὶ ἀνθρώπων, ἑκκαίδεκα χιλιάδες, καὶ τὸ τέλος αὐτῶν Κυρίῳ, δύο καὶ τριάκοντα ψυχαί.

\vs{41}Καὶ ἔδωκε Μωυσῆς τὸ τέλος Κυρίῳ τὸ ἀφαίρεμα τοῦ Θεοῦ Ἐλεάζαρ τῷ ἱερεῖ, καθὰ συνέταξε Κύριος τῷ Μωυσῇ·
\vs{42}ἀπὸ τοῦ ἡμισεύματος τῶν υἱῶν Ἰσραὴλ, οὓς διεῖλε Μωυσῆς ἀπὸ τῶν ἀνδρῶν τῶν πολεμιστῶν.
\vs{43}Καὶ ἐγένετο τὸ ἡμίσευμα ἀπὸ τῆς συναγωγῆς ἀπὸ τῶν προβάων, τριακόσιαι καὶ τριάκοντα χιλιάδες καὶ ἑπτακισχίλια καὶ πεντακόσια·
\vs{44}Καὶ βόες, ἓξ καὶ τριάκοντα χιλιάδες·
\vs{45}Ὄνοι, τριάκοντα χιλιάδες καὶ πεντακόσιοι·
\vs{46}Καὶ ψυχαὶ ἀνθρώπων, ἓξ καὶ δέκα χιλιάδες.
\vs{47}Καὶ ἔλαβε Μωυσῆς ἀπὸ τοῦ ἡμισεύματος τῶν υἱῶν Ἰσραὴλ τὸ ἓν ἀπὸ τῶν πεντήκοντα, ἀπὸ τῶν ἀνθρώπων καὶ ἀπὸ τῶν κτηνῶν, καὶ ἔδωκεν αὐτὰ τοῖς Λευίταις τοῖς φυλάσσουσι τὰς φυλακὰς τῆς σκηνῆς Κυρίου, ὃν τρόπον συνέταξε Κύριος τῷ Μωυσῇ.

\vs{48}Καὶ προσῆλθον πρὸς Μωυσῆν πάντες οἱ καθεσταμένοι εἰς τὰς χιλιαρχίας τῆς δυνάμεως, χιλίαρχοι καὶ ἑκατόνταρχοι,
\vs{49}καὶ εἶπαν πρὸς Μωυσῆν, Οἱ παῖδές σου εἰλήφασι τὸ κεφάλαιον τῶν ἀνδρῶν τῶν πολεμιστῶν τῶν παρʼ ἡμῖν, καὶ οὐ διαπεφώνηκεν ἀπʼ αὐτῶν οὐδὲ εἷς.
\vs{50}Καὶ προσενηνόχαμεν τὸ δῶρον Κυρίῳ, ἀνὴρ ὃ εὗρε σκεῦος χρυσοῦν καὶ χλιδῶνα καὶ ψέλλιον καὶ δακτύλιον καὶ περιδέξιον καὶ ἐμπλοκιον, ἐξιλάσασθαι περὶ ἡμῶν ἔναντι Κυρίου.
\vs{51}Καὶ ἔλαβε Μωυσῆς καὶ Ἐλεάζαρ ὁ ἱερεὺς τὸ χρυσίον παρʼ αὐτῶν πᾶν σκεῦος εἰργασμένον.
\vs{52}Καὶ ἐγένετο πᾶν τὸ χρυσίον τὸ ἀφαίρεμα ὃ ἀφεῖλον Κυρίῳ, ἑκκίδεκα χιλιάδες καὶ ἑπτακόσιοι καὶ πεντήκοντα σίκλοι παρὰ τῶν χιλιάρχων καὶ παρὰ τῶν ἑκατοντάρχων.
\vs{53}Καὶ οἱ ἄνδρες οἱ πολεμισταὶ ἐπρονόμευσαν ἕκαστος ἑαυτῷ.
\vs{54}Καὶ ἔλαβε Μωυσῆς καὶ Ἐλεάζαρ ὁ ἱερεὺς τὸ χρυσίον παρὰ τῶν χιλιάρχων καὶ παρὰ τῶν ἑκατοντάρχων, καὶ εἰσήνεγκεν αὐτὰ εἰς τὴν σκηνὴν τοῦ μαρτυρίου, μνημόσυνον τῶν υἱῶν Ἰσραὴλ ἔναντι Κυρίου.

\ch{32}
Καὶ κτήνη πλῆθος ἦν τοῖς υἱοῖς Ῥουβὴν καὶ τοῖς υἱοῖς Γὰδ, πλῆθος σφόδρα· καὶ εἶδον τὴν χώραν Ἰαζὴρ, καὶ τὴν χώραν Γαλαάδ· καὶ ἦν ὁ τόπος, τόπος κτήνεσι·
\vs{2}Καὶ προσελθόντες οἱ υἱοὶ Ῥουβὴν καὶ οἱ υἱοὶ Γὰδ, εἶπαν πρὸς Μωυσῆν καὶ πρὸς Ἐλεάζαρ τὸν ἱερέα καὶ πρὸς τοὺς ἄρχοντας τῆς συναγωγῆς, λέγοντες,
\vs{3}Ἀταρὼθ, καὶ Δαιβὼν, καὶ Ἰαζὴρ, καὶ Ναμρὰ, καὶ Ἐσεβὼν, καὶ Ἐλεαλὴ, καὶ Σεβαμὰ, καὶ Ναβαὺ, καὶ Βαιάν,
\vs{4}τὴν γῆν ἣν παραδέδωκε Κύριος ἐνώπιον τῶν υἱῶν Ἰσραήλ, γῆ κτηνοτρόφος ἐστὶ, καὶ τοῖς παισί σου κτήνη ὑπάρχει.
\vs{5}Καὶ ἔλεγον, εἰ εὕρομεν χάριν ἐνώπιόν σου, δοθήτω ἡ γῆ αὕτη τοῖς οἰκέταις σου ἐν κατασχέσει, καὶ μὴ διαβιβάσῃς ἡμᾶς τὸν Ἰορδάνην.

\vs{6}Καὶ εἶπε Μωυσῆς τοῖς υἱοῖς Γὰδ καὶ τοῖς υἱοῖς Ῥουβὴν, οἱ ἀδελφοὶ ὑμῶν πορεύονται εἰς τὸν πόλεμον, καὶ ὑμεῖς καθήσεσθε αὐτοῦ;
\vs{7}Καὶ ἱνατί διαστρέφετε τὰς διανοίας τῶν υἱῶν Ἰσραὴλ μὴ διαβῆναι εἰς τὴν γῆν, ἣν Κύριος δίδωσιν αὐτοῖς;
\vs{8}Οὐχ οὕτως ἐποίησαν οἱ πατέρες ὑμῶν, ὅτε ἀπέστειλα αὐτοὺς ἐκ Κάδης Βαρνὴ κατανοῆσαι τὴν γῆν;
\vs{9}καὶ ἀνέβησαν φάραγγα βότρυος, καὶ κατενόησαν τὴν γῆν, καὶ ἀπέστησαν τὴν καρδίαν τῶν υἱῶν Ἰσραὴλ, ὅπως μὴ εἰσέλθωσιν εἰς τὴν γῆν, ἣν ἔδωκε Κύριος αὐτοῖς.
\vs{10}Καὶ ὠργίσθη θυμῷ Κύριος ἐν τῇ ἡμέρᾳ ἐκείνῃ, καὶ ὤμοσε, λέγων,
\vs{11}εἰ ὄψονται οἱ ἄνθρωποι οὗτοι οἱ ἀναβάντες ἐξ Αἰγύπτου ἀπὸ εἰκοσαετοῦς καὶ ἐπάνω, οἱ ἐπιστάμενοι τὸ ἀγαθὸν καὶ τὸ κακὸν, τὴν γῆν ἣν ὤμοσα τῷ Ἁβραὰμ καὶ Ἰσαὰκ καὶ Ἰακὼβ, οὐ γὰρ συνεπηκολούθησαν ὀπίσω μου·

\vs{12}πλὴν Χάλεβ υἱὸς Ἰεφοννὴ ὁ διακεχωρισμένος, καὶ Ἰησοῦς ὁ τοῦ Ναυὴ, ὅτι συνεπηκολούθησεν ὀπίσω Κυρίου.
\vs{13}Καὶ ὠργίσθη θυμῷ Κύριος ἐπὶ τὸν Ἰσραήλ, καὶ κατερόμβευσεν αὐτοὺς ἐν τῇ ἐρήμῳ τεσσεράκοντα ἔτη, ἕως ἐξανηλώθη πᾶσα ἡ γενεὰ, οἱ ποιοῦντες τὰ πονηρὰ ἔναντι Κυρίου.
\vs{14}Ἰδοὺ ἀνέστητε ἀντὶ τῶν πατέρων ὑμῶν, σύντριμμα ἀνθρώπων ἁμαρτωλῶν, προσθεῖναι ἔτι ἐπὶ τὸν θυμὸν τῆς ὀργῆς Κυρίου ἐπὶ Ἰσραήλ.
\vs{15}Ὅτι ἀποστραφήσεσθε ἀπʼ αὐτοῦ προσθεῖναι ἔτι καταλιπεῖν αὐτὸν ἐν τῇ ἐρήμῳ, καὶ ἀνομήσετε εἰς ὅλην τὴν συναγωγὴν ταύτην.

\vs{16}Καὶ προσῆλθον αὐτῷ, καὶ ἔλεγον, ἐπαύλεις προβάτων οἰκοδομήσομεν ὧδε τοῖς κτήνεσιν ἡμῶν, καὶ πόλεις ταῖς ἀποσκευαῖς ἡμῶν.
\vs{17}Καὶ ἡμεῖς ἐνοπλισάμενοι προφυλακὴν πρότεροι τῶν υἱῶν Ἰσραὴλ, ἕως ἂν ἀγάγωμεν αὐτοὺς εἰς τὸν ἑαυτῶν τόπον· καὶ κατοικήσει ἡ ἀποσκευὴ ἡμῶν ἐν πόλεσι τετειχισμέναις διὰ τοὺς κατοικοῦντας τὴν γῆν.
\vs{18}Οὐ μὴ ἀποστραφῶμεν εἰς τὰς οἰκίας ἡμῶν ἕως ἂν καταμερισθῶσιν οἱ υἱοὶ Ἰσραὴλ, ἕκαστος εἰς τὴν κληρονομίαν αὐτοῦ.
\vs{19}Καὶ οὐκέτι κληρονομήσομεν ἐν αὐτοῖς ἀπὸ τοῦ πέραν τοῦ Ἰορδάνου καὶ ἐπέκεινα, ὅτι ἀπέχομεν τοὺς κλήρους ἡμῶν ἐν τῷ πέραν τοῦ Ἰορδάνου ἐν ἀνατολαῖς.

\vs{20}Καὶ εἶπε πρὸς αὐτοὺς Μωυσῆς, ἐὰν ποιήσητε κατὰ τὸ ῥῆμα τοῦτο, ἐὰν ἐξοπλίσησθε ἔναντι Κυρίου εἰς πόλεμον,
\vs{21}καὶ παρελεύσεται ὑμῶν πᾶς ὁπλίτης τὸν Ἰορδάνην ἔναντι Κυρίου, ἕως ἂν ἐκτριβῇ ὁ ἐχθρὸς αὐτοῦ ἀπὸ προσώπου αὐτοῦ,
\vs{22}καὶ κατακυριευθῇ ἡ γῆ ἔναντι Κυρίου, καὶ μετὰ ταῦτα ἀποστραφήσεσθε, καὶ ἔσεσθε ἀθῶοι ἔναντι Κυρίου, καὶ ἀπὸ Ἰσραήλ· καὶ ἔσται ἡ γῆ αὕτη ὑμῖν ἐν κατασχέσει ἔναντι Κυρίου.
\vs{23}Ἐὰν δὲ μὴ ποιήσητε οὕτως, ἁμαρτήσεσθε ἔναντι Κυρίου· καὶ γνώσεσθε τὴν ἁμαρτίαν ὑμῶν, ὅταν ὑμᾶς καταλάβῃ τὰ κακά.
\vs{24}Καὶ οἰκοδομήσετε ὑμῖν ἑαυτοῖς πόλεις τῇ ἀποσκευῇ ὑμῶν, καὶ ἐπαύλεις τοῖς κτήνεσιν ὑμῶν· καὶ τὸ ἐκπορευόμενον ἐκ τοῦ στόματος ὑμῶν ποιήσετε.

\vs{25}Καὶ εἶπαν υἱοὶ Ῥουβὴν καὶ υἱοὶ Γὰδ πρὸς Μωυσῆν, λέγοντες, οἱ παῖδές σου ποιήσουσι καθὰ ὁ Κύριος ἡμῶν ἐντέλλεται.
\vs{26}Ἡ ἀποσκευὴ ἡμῶν, καὶ αἱ γυναῖκες ἡμῶν, καὶ πάντα τὰ κτήνη ἡμῶν ἔσονται ἐν ταῖς πόλεσιν Γαλαάδ.
\vs{27}Οἱ δὲ παῖδές σου παρελεύσονται πάντες ἐνωπλισμένοι καὶ ἐκτεταγμένοι ἔναντι Κυρίου εἰς τὸν πόλεμον, ὃν τρόπον ὁ κύριος λέγει.

\vs{28}Καὶ συνέστησεν αὐτοῖς Μωυσῆς Ἐλεάζαρ τὸν ἱερέα, καὶ Ἰησοῦν υἱὸν Ναυὴ, καὶ τοὺς ἄρχοντας πατριῶν τῶν φυλῶν Ἰσραήλ.
\vs{29}Καὶ εἶπε πρὸς αὐτοὺς Μωυσῆς, ἐὰν διαβῶσιν οἱ υἱοὶ Ῥουβὴν καὶ οἱ υἱοὶ Γὰδ μεθʼ ὑμῶν τὸν Ἰορδάνην, πᾶς ἐνωπλισμένος εἰς πόλεμον ἔναντι Κυρίου, καὶ κατακυριεύσητε τῆς γῆς ἀπέναντι ὑμῶν, καὶ δώσετε αὐτοῖς τὴν γῆν Γαλαὰδ ἐν κατασχέσει.
\vs{30}Ἐὰν δὲ μὴ διαβῶσιν ἐνωπλισμένοι μεθʼ ὑμῶν εἰς τὸν πόλεμον ἔναντι Κυρίου, καὶ διαβιβάσετε τὴν ἀποσκευὴν αὐτῶν, καὶ τὰς γυναῖκας αὐτῶν, καὶ τὰ κτήνη αὐτῶν πρότερα ὑμῶν εἰς γῆν Χαναὰν, καὶ συγκατακληρονομηθήσονται ἐν ὑμῖν ἐν τῇ γῇ Χαναάν.
\vs{31}Καὶ ἀπεκρίθησαν οἱ υἱοὶ Ῥουβὴν καὶ οἱ υἱοὶ Γὰδ λέγοντες, ὅσα ὁ Κύριος λέγει τοῖς θεράπουσιν, οὕτω ποιήσομεν ἡμεῖς.
\vs{32}Διαβησόμεθα ἐνωπλισμένοι ἔναντι Κυρίου εἰς γῆν Χαναὰν, καὶ δώσετε τὴν κατάσχεσιν ἡμῖν ἐν τῷ πέραν τοῦ Ἰορδάνου.

\vs{33}Καὶ ἔδωκεν αὐτοῖς Μωυσῆς τοῖς υἱοῖς Γὰδ, καὶ τοῖς υἱοῖς Ῥουβὴν, καὶ τῷ ἡμίσει φυλῆς Μανασσῆ υἱῶν Ἰωσὴφ, τὴν βασιλείαν Σηὼν βασιλέως Ἀμοῤῥαίων, καὶ τὴν βασιλείαν Ὢγ βασιλέως τῆς Βασὰν, τὴν γῆν καὶ τὰς πόλεις σὺν τοῖς ὁρίοις αὐτῆς, πόλεις τῆς γῆς κύκλῳ.
\vs{34}Καὶ ᾠκοδόμησαν οἱ υἱοὶ Γὰδ τὴν Δαιβὼν, καὶ τὴν Ἀταρὼθ, καὶ τὴν Ἀροὴρ,
\vs{35}καὶ τὴν Σοφὰρ, καὶ τὴν Ἰαζὴρ, καὶ ὕψωσαν αὐτὰς,
\vs{36}καὶ τὴν Ναμρὰμ, καὶ τὴν Βαιθαρὰν, πόλεις ὀχυρὰς, καὶ ἐπαύλεις προβάτων.
\vs{37}Καὶ οἱ υἱοὶ Ῥουβὴν ᾠκοδόμησαν τὴν Ἐσεβὼν, καὶ Ἐλεάλην, καὶ Καριαθὰμ,
\vs{38}καὶ τὴν Βεελμεὼν, περικεκυκλωμένας, καὶ τὴν Σεβαμά· καὶ ἐπωνόμασαν κατὰ τὰ ὀνόματα αὐτῶν τὰ ὀνόματα τῶν πόλεων, ἃς ᾠκοδόμησαν.
\vs{39}Καὶ ἐπορεύθη υἱὸς Μαχὶρ υἱοῦ Μανασσὴ Γαλαὰδ, καὶ ἔλαβεν αὐτὴν, καὶ ἀπώλεσε τὸν Ἀμοῤῥαῖον τὸν κατοικοῦντα ἐν αὐτῇ.
\vs{40}Καὶ ἔδωκε Μωυσῆς τὴν Γαλαὰδ τῷ Μαχὶρ υἱῷ Μανασσῆ, καὶ κατῴκησεν ἐκεῖ.
\vs{41}Καὶ Ἰαῒρ ὁ τοῦ Μανασσῆ ἐπορεύθη, καὶ ἔλαβε τὰς ἐπαύλεις αὐτῶν, καὶ ἐπωνόμασεν αὐτὰς ἐπαύλεις Ἰαΐρ.
\vs{42}Καὶ Ναβαὺ ἐπορεύθη, καὶ ἔλαβε τὴν Καὰθ καὶ τὰς κώμας αὐτῆς, καὶ ἐπωνόμασεν αὐτὰς Ναβὼθ ἐκ τοῦ ὀνόματος αὐτοῦ.

\ch{33}
Καὶ οὗτοι οἱ σταθμοὶ τῶν υἱῶν Ἰσραὴλ, ὡς ἐξῆλθον ἐκ γῆς Αἰγύπτου σὺν δυνάμει αὐτῶν ἐν χειρὶ Μωυσῆ καὶ Ἀαρών.
\vs{2}Καὶ ἔγραψε Μωυσῆς τὰς ἀπάρσεις αὐτῶν, καὶ τοὺς σταθμοὺς αὐτῶν, διὰ ῥήματος Κυρίου· καὶ οὗτοι σταθμοὶ τῆς πορείας αὐτῶν.
\vs{3}Ἀπῆραν ἐκ Ῥαμεσσὴ τῷ μηνὶ τῷ πρώτῳ, τῇ πεντεκαιδεκάτῃ ἡμέρᾳ τοῦ μηνὸς τοῦ πρώτου· τῇ ἐπαύριον τοῦ πάσχα ἐξῆλθον οἱ υἱοὶ Ἰσραὴλ ἐν χειρὶ ὑψηλῇ ἐναντίον πάντων τῶν Αἰγυπτίων.
\vs{4}Καὶ οἱ Αἰγύπτιοι ἔθαπτον ἐξ αὐτῶν τοὺς τεθνηκότας πάντας οὓς ἐπάταξε Κύριος, πᾶν πρωτότοκον ἐν γῇ Αἰγύπτῳ· καὶ ἐν τοῖς θεοῖς αὐτῶν ἐποίησε τὴν ἐκδίκησιν Κύριος.
\vs{5}Καὶ ἀπάραντες οἱ υἱοὶ Ἰσραὴλ ἐκ Ῥαμεσσῆ, παρενέβαλον εἰς Σοκχώθ.
\vs{6}Καὶ ἀπάραντες ἐκ Σοκχὼθ, παρενέβαλον εἰς Βουθὰν, ὅ ἐστι μέρος τι τῆς ἐρήμου·
\vs{7}Καὶ ἀπῇραν ἐκ Βουθὰν, καὶ παρενέβαλον ἐπὶ τὸ στόμα Εἰρὼθ, ὅ ἐστιν ἀπέναντι Βεελσεπφὼν, καὶ παρενέβαλον ἀπέναντι Μαγδώλου.
\vs{8}Καὶ ἀπῇραν ἀπέναντι Εἰρὼθ, καὶ διέβησαν μέσον τῆς θαλάσσης εἰς τὴν ἔρημον· καὶ ἐπορεύθησαν ὁδὸν τριῶν ἡμερῶν διὰ τῆς ἐρήμου αὐτοὶ, καὶ παρενέβαλον ἐν Πικρίαις.
\vs{9}Καὶ ἀπῇραν ἐκ Πικριῶν, καὶ ἦλθον εἰς Αἰλίμ· καὶ ἐν Αἰλὶμ δώδεκα πηγαὶ ὑδάτων, καὶ ἑβδομήκοντα στελέχη φοινίκων, καὶ παρενέβαλον ἐκεῖ παρὰ τὸ ὕδωρ.
\vs{10}Καὶ ἀπῇραν ἐξ Αἰλὶμ, καὶ παρενέβαλον ἐπὶ θάλασσαν ἐρυθράν.
\vs{11}Καὶ ἀπῇραν ἀπὸ θαλάσσης ἐρυθρᾶς, καὶ παρενέβαλον εἰς τὴν ἔρημον Σίν.

\vs{12}Καὶ ἀπῇραν ἐκ τῆς ἐρήμου Σὶν, καὶ παρενέβαλον εἰς Ῥαφακά.
\vs{13}Καὶ ἀπῇραν ἐκ Ῥαφακὰ, καὶ παρενέβαλον ἐν Αἰλούς.
\vs{14}Καὶ ἀπῇραν ἐξ Αἰλούς, καὶ παρενέβαλον ἐν Ῥαφιδίν· καὶ οὐκ ἦν ἐκεῖ ὕδωρ τῷ λαῷ πιεῖν.
\vs{15}Καὶ ἀπῇραν ἐκ Ῥαφιδὶν, καὶ παρενέβαλον ἐν τῇ ἐρήμῳ Σινᾶ.
\vs{16}Καὶ ἀπῇραν ἐκ τῆς ἐρήμου Σινᾶ, καὶ παρενέβαλον ἐν μνήμασι τῆς ἐπιθυμίας.
\vs{17}Καὶ ἀπῇραν ἐκ μνημάτων τῆς ἐπιθυμίας, καὶ παρενέβαλον ἐν Ἀσηρώθ.
\vs{18}Καὶ ἀπῇραν ἐξ Ἀσηρὼθ, καὶ παρενέβαλον ἐν Ῥαθαμᾷ.

\vs{19}Καὶ ἀπῇραν ἐκ Ῥαθαμᾶ, καὶ παρενέβαλον ἐν Ῥεμμὼν Φαρές.
\vs{20}Καὶ ἀπῇραν ἐκ Ῥεμμὼν Φαρὲς, καὶ παρενέβαλον εἰς Λεβωνᾶ.
\vs{21}Καὶ ἀπῇραν ἐκ Λεβῶνα, καὶ παρενέβαλον εἰς Ῥεσσάν.
\vs{22}Καὶ ἀπῇραν ἐκ Ῥεσσὰν, καὶ παρενέβαλον εἰς Μακελλάθ.
\vs{23}Καὶ ἀπῇραν ἐκ Μακελλὰθ, καὶ παρενέβαλον εἰς Σαφάρ.
\vs{24}Καὶ ἀπῇραν ἑκ Σαφὰρ, καὶ παρενέβαλον εἰς Χαραδάθ.
\vs{25}Καὶ ἀπῇραν ἐκ Χαραδὰθ, καὶ παρενέβαλον εἰς Μακηλώθ.
\vs{26}Καὶ ἀπῇραν ἐκ Μακηλὼθ, καὶ παρενέβαλον εἰς Καταάθ.
\vs{27}Καὶ ἀπῇραν ἐκ Καταὰθ, καὶ παρενέβαλον εἰς Ταράθ.
\vs{28}Καὶ ἀπῇραν ἐκ Ταρὰθ, καὶ παρενέβαλον εἰς Μαθεκκά.
\vs{29}Καὶ ἀπῇραν ἐκ Μαθεκκὰ, καὶ παρενέβαλον εἰς Σελμωνᾶ.
\vs{30}Καὶ ἀπῇραν ἐκ Σελμωνᾶ, καὶ παρενέβαλον εἰς Μασουρούθ.
\vs{31}Καὶ ἀπῇραν ἐκ Μασουροὺθ, καὶ παρενέβαλον εἰς Βαναία.
\vs{32}Καὶ ἀπῇραν ἐκ Βαναία, καὶ παρενέβαλον εἰς τὸ ὄρος Γαδγάδ.

\vs{33}Καὶ ἀπῇραν ἐκ τοῦ ὄρους Γαδγὰδ, καὶ παρενέβαλον εἰς Ἐτεβαθά.
\vs{34}Καὶ ἀπῇραν ἐξ Ἐτεβαθὰ, καὶ παρενέβαλον εἰς Ἐβρωνά.
\vs{35}Καὶ ἀπῇραν ἐξ Ἐβρωνὰ, καὶ παρενέβαλον εἰς Γεσιὼν Γάβερ.
\vs{36}Καὶ ἀπῇραν ἐκ Γεσιὼν Γάβερ, καὶ παρενέβαλον ἐν τῇ ἐρήμῳ Σίν· καὶ ἀπῇραν ἐκ τῆς ἐρήμου Σὶν, καὶ παρενέβαλον εἰς τὴν ἔρημον Φαράν· αὕτη ἐστὶ Κάδης.
\vs{37}Καὶ ἀπῇραν ἐκ Κάδης, καὶ παρενέβαλον εἰς Ὢρ τὸ ὄρος πλησίον γῆς Ἐδώμ.

\vs{38}Καὶ ἀνέβη Ἀαρὼν ὁ ἱερεὺς διὰ προστάγματος Κυρίου, καὶ ἀπέθανεν ἐκεῖ ἐν τῷ τεσσαρακοστῷ ἔτει τῆς ἐξόδου τῶν υἱῶν Ἰσραὴλ ἐκ γῆς Αἰγύπτου, τῷ μηνὶ τῷ πέμπτῳ μιᾷ τοῦ μηνός.
\vs{39}Καὶ Ἀαρὼν ἦν τριῶν καὶ εἴκοσι καὶ ἑκατὸν ἐτῶν, ὅτε ἀπέθνησκεν ἐν Ὢρ τῷ ὄρει.
\vs{40}Καὶ ἀκούσας ὁ Χανανὶς βασιλεὺς Ἀρὰδ, καὶ οὗτος κατῴκει ἐν γῇ Χαναὰν, ὅτε εἰσεπορεύοντο οἱ υἱοὶ Ἰσραήλ·
\vs{41}καὶ ἀπῇραν ἐξ Ὢρ τοῦ ὄρους, καὶ παρενέβαλον εἰς Σελμωνᾶ.
\vs{42}Καὶ ἀπῇραν ἐκ Σελμωνᾶ, καὶ παρενέβαλον εἰς Φινώ.
\vs{43}Καὶ ἀπῇραν ἐκ Φινὼ, καὶ παρενέβαλον ἐν Ὠβώθ.

\vs{44}Καὶ ἀπῇραν ἐξ Ὠβὼθ, καὶ παρενέβαλον ἐν Γαῒ, ἐν τῷ πέραν ἐπὶ τῶν ὁρίων Μωάβ.
\vs{45}Καὶ ἀπῇραν ἐκ Γαῒ, καὶ παρενέβαλον εἰς Δαιβὼν Γάδ.
\vs{46}Καὶ ἀπῇραν ἐκ Δαιβὼν Γὰδ, καὶ παρενέβαλον ἐν Γελμὼν Δεβλαθαίμ.
\vs{47}Καὶ ἀπῇραν ἐκ Γελμὼν Δεβλαθαὶμ, καὶ παρενέβαλον ἐπὶ τὰ ὄρη τὰ Ἀβαρὶμ, ἀπέναντι Ναβαῦ.
\vs{48}Καὶ ἀπῇραν ἀπὸ ὀρέων Ἀβαρὶμ, καὶ παρενέβαλον ἐπὶ δυσμῶν Μωὰβ, ἐπὶ τοῦ Ἰορδάνου κατὰ Ἱεριχώ.
\vs{49}Καὶ παρενέβαλον παρὰ τὸν Ἰορδάνην ἀναμέσον Αἰσιμώθ, ἕως Βελσᾶ τὸ κατὰ δυσμὰς Μωάβ.

\vs{50}Καὶ ἐλάλησε Κύριος πρὸ; Μωυσῆν ἐπὶ δυσμῶν Μωὰβ παρὰ τὸν Ἰορδάνην κατὰ Ἱεριχὼ, λέγων,
\vs{51}λάλησον τοῖς υἱοῖς Ἰσραὴλ, καὶ ἐρεῖς πρὸς αὐτοὺς, ὑμεῖς διαβαίνετε τὸν Ἰορδάνην εἰς γῆν Χαναάν.
\vs{52}Καὶ ἀπολεῖτε πάντας τοὺς κατοικοῦντας ἐν τῇ γῇ πρὸ προσώπου ὑμῶν, καὶ ἐξαρεῖτε τὰς σκοπιὰς αὐτῶν, καὶ πάντα τὰ εἴδωλα τὰ χωνευτὰ αὐτῶν ἀπολεῖτε αὐτὰ, καὶ πάσας τὰς στήλας αὐτῶν ἐξαρεῖτε.
\vs{53}Καὶ ἀπολεῖτε πάντας τοὺς κατοικοῦντας τὴν γῆν, καὶ κατοικήσετε ἐν αὐτῇ, ὑμῖν γὰρ δέδωκα τὴν γῆν αὐτῶν ἐν κλήρῳ.
\vs{54}Καὶ κατακληρονομήσετε τὴν γῆν αὐτῶν ἐν κλήρῳ κατὰ φυλὰς ὑμῶν· τοῖς πλείοσι πληθυνεῖτε τὴν κατάσχεσιν αὐτῶν, καὶ τοῖς ἐλάττοσιν ἐλαττώσετε τὴν κατάσχεσιν αὐτῶν· εἰς ὃ ἂν ἐξέλθῃ τὸ ὄνομα αὐτοῦ, ἐκεῖ αὐτοῦ ἔσται· κατὰ φυλὰς πατριῶν ὑμῶν κληρονομήσετε.
\vs{55}Ἐὰν δὲ μὴ ἀπολέσητε τοὺς κατοικοῦντας ἐπὶ τῆς γῆς ἀπὸ προσώπου ὑμῶν, καὶ ἔσται οὓς ἐὰν καταλίπητε ἐξ αὐτῶν, σκόλοπες ἐν τοῖς ὀφθαλμοῖς ὑμῶν, καὶ βολίδες ἐν ταῖς πλευραῖς ὑμῶν, καὶ ἐχθρεύσουσιν ὑμῖν ἐπὶ τῆς γῆς, ἐφʼ ἣν ὑμεῖς κατοικήσετε.
\vs{56}Καὶ ἔσται καθότι διεγνώκειν ποιῆσαι αὐτοὺς, ποιήσω ὑμᾶς.

\ch{34}
Καὶ ἐλάλησε Κύριος πρὸς Μωυσῆν, λέγων,
\vs{2}ἔντειλαι τοῖς υἱοῖς Ἰσραὴλ, καὶ ἐρεῖς πρὸς αὐτοὺς, ὑμεῖς εἰσπορεύεσθε εἰς τὴν γῆν Χαναὰν· αὕτη ἔσται ὑμῖν εἰς κληρονομίαν, γῆ Χαναὰν σὺν τοῖς ὁρίοις αὐτῆς.
\vs{3}Καὶ ἔσται ὑμῖν τὸ κλίτος τὸ πρὸς Λίβα ἀπὸ ἐρήμου Σὶν ἕως ἐχόμενον Ἐδὼμ, καὶ ἔσται ὑμῖν τὰ ὅρια πρὸς λίβα ἀπὸ μέρους τῆς θαλάσσης τῆς ἁλυκῆς ἀπὸ ἀνατολῶν.
\vs{4}Καὶ κυκλώσει ὑμᾶς τὰ ὅρια ἀπὸ Λιβὸς πρὸς ἀνάβασιν Ἀκραβὶν, καὶ παρελεύσεται Ἐννὰκ, καὶ ἔσται ἡ διέξοδος αὐτοῦ πρὸς Λίβα Κάδης τοῦ Βαρνὴ, καὶ ἐξελεύσεται εἰς ἔπαυλιν Ἀρὰδ, καὶ παρελεύσεται Ἀσεμωνᾶ.
\vs{5}Καὶ κυκλώσει τὰ ὅρια ἀπὸ Ἀσεμωνᾶ χειμάῤῥουν Αἰγύπτου, καὶ ἔσται ἡ διέξοδος ἡ θάλασσα.
\vs{6}Καὶ τὰ ὅρια τῆς θαλάσσης ἔσται ὑμῖν, ἡ θάλασσα ἡ μεγάλη ὁριεῖ, τοῦτο ἔσται ὑμῖν τὰ ὅρια τῆς θαλάσσης.

\vs{7}Καὶ τοῦτο ἔσται ὑμῖν τὰ ὅρια πρὸς βοῤῥᾶν· ἀπὸ τῆς θαλάσσης τῆς μεγάλης καταμετρήσετε ὑμῖν αὐτοῖς παρὰ τὸ ὄρος τὸ ὄρος.
\vs{8}Καὶ ἀπὸ τοῦ ὄρους τὸ ὄρος καταμετρήσετε αὐτοῖς, εἰσπορευομένων εἰς Ἐμὰθ, καὶ ἔσται ἡ διέξοδος αὐτοῦ τὰ ὅρια Σαραδάκ.
\vs{9}Καὶ ἐξελεύσεται τὰ ὅρια Δεφρωνὰ, καὶ ἔσται ἡ διέξοδος αὐτοῦ Ἀρσεναΐν· τοῦτο ἔσται ὑμῖν ὅρια ἀπὸ Βοῤῥᾶ.
\vs{10}Καὶ καταβήσεται τὰ ὅρια ἀπὸ Σεπεφαμὰρ Βηλὰ ἀπὸ ἀνατολῶν ἐπὶ πηγὰς, και καταβήσεται τὰ ὅρια Βηλὰ ἐπὶ νώτου θαλάσσης Χενερὲθ ἀπὸ ἀνατολῶν.
\vs{11}Καὶ καταβήσεται τὰ ὅρια ἀπὸ Σεπφαμὰρ Βηλὰ ἀπὸ ἀνατολῶν ἐπὶ πηγάς, καὶ καταβήσεται τὰ ὅρια Βηλὰ ἐπὶ νώτου θαλάσσης Χενερὲθ ἀπὸ ἀνατολῶν.
\vs{12}Καὶ καταβήσεται τὰ ὅρια ἐπὶ τὸν Ἰορδάνην, καὶ ἔσται ἡ διέξοδος θάλασσα ἡ ἁλυκή· αὕτη ἔσται ὑμῖν ἡ γῆ καὶ τὰ ὅρια αὐτῆς κύκλῳ.

\vs{13}Καὶ ἐνετείλατο Μωυσῆς τοῖς υἱοῖς Ἰσραὴλ, λέγων, αὕτη ἡ γῆ ἣν κατακληρονομήσετε αὐτὴν μετὰ κλήρου, ὃν τρόπον συνέταξε Κύριος δοῦναι αὐτὴν ταῖς ἐννέα φυλαῖς καὶ τῷ ἡμίσει φυλῆς Μανασσῆ.
\vs{14}Ὅτι ἔλαβε φυλὴ υἱῶν Ῥουβὴν, καὶ φυλὴ υἱῶν Γὰδ κατʼ οἴκους πατριῶν αὐτῶν· καὶ τὸ ἥμισυ φυλῆς Μανασσῆ ἀπέλαβον τοὺς κλήρους αὐτῶν.
\vs{15}Δύο φυλαὶ καὶ ἥμισυ φυλῆς ἔλαβον τοὺς κλήρους αὐτῶν πέραν τοῦ Ἰορδάνου κατὰ Ἱεριχὼ ἀπὸ Νότου κατʼ ἀνατολάς.

\vs{16}Καὶ ἐλάλησε Κύριος πρὸς Μωυσῆν, λέγων,
\vs{17}ταῦτα τὰ ὀνόματα τῶν ἀνδρῶν, οἳ κληρονομήσουσιν ὑμῖν τὴν γῆν· Ἐλεάζαρ ὁ ἱερεὺς καὶ Ἰησοῦς ὁ τοῦ Ναυή.
\vs{18}Καὶ ἄρχοντα ἕνα ἐκ φυλῆς λήψεσθε κατακληρονομῆσαι ὑμῖν τὴν γῆν.

\vs{19}Καὶ ταῦτα τὰ ὀνόματα τῶν ἀνδρῶν· τῆς φυλῆς Ἰούδα, Χάλεβ υἱὸς Ἰεφοννή.
\vs{20}Τῆς φυλῆς Συμεὼν, Σαλαμιὴλ υἱὸς Σεμιούδ.
\vs{21}Τῆς φυλῆς Βενιαμὶν, Ἐλδὰδ υἱὸς Χασλών·
\vs{22}Τῆς φυλῆς Δὰν, ἄρχων Βακχὶρ υἱὸς Ἐγλί.
\vs{23}Τῶν υἱῶν Ἰωσὴφ φυλῆς υἱῶν Μανασσῆ, ἄρχων Ἀνιὴλ υἱὸς Σουφί.
\vs{24}Τῆς φυλῆς υἱῶν Ἐφραὶμ, ἄρχων Καμουὴλ υἱὸς Σαβαθᾶν.
\vs{25}Τῆς φυλῆς Ζαβουλὼν, ἄρχων Ἐλισαφὰν υἱὸς Φαρνάχ.
\vs{26}Τῆς φυλῆς υἱῶν Ἰσσάχαρ, ἄρχων Φαλτιὴλ υἱὸς Ὀζᾶ.
\vs{27}Τῆς φυλῆς υἱῶν Ἀσὴρ, ἄρχων Ἀχιὼρ υἱὸς Σελεμί.
\vs{28}Τῆς φυλῆς Νεφθαλὶ, ἄρχων Φαδαὴλ υἱὸς Ἰαμιούδ.

\vs{29}Τούτοις ἐνετείλατο Κύριος καταμερίσαι τοῖς υἱοῖς Ἰσραὴλ ἐν γῇ Χαναάν.

\ch{35}
Καὶ ἐλάλησε Κύριος πρὸς Μωυσῆν ἐπὶ δυσμῶν Μωὰβ παρὰ τὸν Ἰορδάνην κατὰ Ἱεριχὼ, λέγων,
\vs{2}σύνταξον τοῖς υἱοῖς Ἰσραὴλ, καὶ δώσουσι τοῖς Λευίταις ἀπὸ τῶν κλήρων κατασχέσεως αὐτῶν πόλεις κατοικεῖν· καὶ τὰ προάστεια τῶν πόλεων κύκλῳ αὐτῶν δώσουσι τοῖς Λευίταις.
\vs{3}Καὶ ἔσονται αὐτοῖς αἱ πόλεις κατοικεῖν, καὶ τὰ ἀφορίσματα αὐτῶν ἔσται τοῖς κτήνεσιν αὐτῶν, καὶ πᾶσι τοῖς τετράποσιν αὐτῶν.
\vs{4}Καὶ τὰ συνκυροῦντα τῶν πόλεων, ἃς δώσετε τοῖς Λευίταις, ἀπὸ τείχους τῆς πόλεως καὶ ἔξω δισχιλίους πήχεις κύκλῳ.
\vs{5}Καὶ μετρήσεις ἔξω τῆς πόλεως τὸ κλίτος τὸ πρὸς ἀνατολὰς δισχιλίους πήχεις, καὶ τὸ κλίτος τὸ πρὸς Λίβα δισχιλίους πήχεις, καὶ τὸ κλίτος τὸ πρὸς θάλασσαν διαχιλίους πήχεις, καὶ τὸ κλίτος τὸ πρὸς Βοῤῥᾶν δισχιλίους πήχεις· καὶ ἡ πόλις μέσον τούτου ἔσται ὑμῖν, καὶ τὰ ὅμορα τῶν πόλεων.
\vs{6}Καὶ τὰς πόλεις δώσετε τοῖς Λευίταις, τὰς ἓξ πόλεις τῶν φυγαδευτηρίων ἃς δώσετε φυγεῖν ἐκεῖ τῷ φονεύσαντι, καὶ πρὸς ταύταις τεσσαράκοντα καὶ δύο πόλεις.
\vs{7}Πάσας τὰς πόλεις δώσετε τοῖς Λευίταις τεσσαράκοντα καὶ ὀκτὼ πόλεις· ταύτας, καὶ τὰ προάστεια αὐτῶν.
\vs{8}Καὶ τὰς πόλεις ἃς δώσετε ἀπὸ τῆς κατασχέσεως υἱῶν Ἰσραὴλ, ἀπὸ τῶν τὰ πολλὰ, πολλά· καὶ ἀπὸ τῶν ἐλαττόνων, ἐλάττω· ἕκαστος κατὰ τὴν κληρονομίαν αὐτοῦ ἣν κατακληρονομήσουσι, δώσουσιν ἀπὸ τῶν πόλεων τοῖς Λευίταις.

\vs{9}Καὶ ἐλάλησε Κύριος πρὸς Μωυσῆν, λέγων,
\vs{10}λάλησον τοῖς υἱοῖς Ἰσραὴλ, καὶ ἐρεῖς πρὸς αὐτοὺς, ὑμεῖς διαβαίνετε τὸν Ἰορδάνην εἰς γῆν Χαναάν.
\vs{11}Καὶ διαστελεῖτε ὑμῖν αὐτοῖς πόλεις· φυγαδευτήρια ἔσται ὑμῖν φυγεῖν ἐκεῖ τὸν φονευτὴν, πᾶς ὁ πατάξας ψυχὴν ἀκουσίως.
\vs{12}Καὶ ἔσονται αἱ πόλεις ὑμῖν φυγαδευτήρια ἀπὸ τοῦ ἀγχιστεύοντος τὸ αἷμα, καὶ οὐ μὴ ἀποθάνῃ ὁ φονεύων ἕως ἂν στῇ ἔναντι τῆς συναγωγῆς εἰς κρίσιν.
\vs{13}Καὶ αἱ πόλεις ἃς δώσετε τὰς ἓξ πόλεις, φυγαδευτήρια ἔσονται ὑμῖν.
\vs{14}Τὰς τρεῖς πόλεις δώσετε πέραν τοῦ Ἰορδάνου, καὶ τὰς τρεῖς πόλεις δώσετε ἐν γῇ Χαναάν.

\vs{15}Φυγαδεῖον ἔσται τοῖς υἱοῖς Ἰσραήλ, καὶ τῷ προσηλύτῳ, καὶ τῷ παροίκῳ τῷ ἐν ὑμῖν· ἔσονται αἱ πόλεις αὗται εἰς φυγαδευτήριον, φυγεῖν ἐκεῖ παντὶ πατάξαντι ψυχὴν ἀκουσίως.

\vs{16}Ἐὰν δὲ ἐν σκεύει σιδήρου πατάξῃ αὐτὸν, καὶ τελευτήσῃ, φονευτής ἐστι· θανάτῳ θανατούσθω ὁ φονευτής.
\vs{17}Ἐὰν δὲ ἐν λίθῳ ἐκ χειρὸς ἐν ᾧ ἀποθανεῖται ἐν αὐτῷ, πατάξῃ αὐτὸν, καὶ ἀποθάνῃ, φονευτής ἐστι· θανάτῳ θανατούσθω ὁ φονευτής.
\vs{18}Ἐὰν δὲ ἐν σκεύει ξυλίνῳ ἐκ χειρὸς ἐξ οὗ ἀποθανεῖται ἐν αὐτῷ, πατάξῃ αὐτόν, καὶ ἀποθάνῃ, φονευτής ἐστι· θανάτῳ θανατούσθω ὁ φονευτής.

\vs{19}Ὁ ἀγχιστεύων τὸ αἷμα, οὗτος ἀποκτενεῖ τὸν φονεύσαντα· ὅταν συναντήσῃ αὐτῷ οὗτος, ἀποκτενεῖ αὐτόν.
\vs{20}Ἐὰν δὲ διʼ ἔχθραν ὤσῃ αὐτὸν, καὶ ἐπιῤῥίψῃ ἐπʼ αὐτὸν πᾶν σκεῦος ἐξ ἐνέδρου, καὶ ἀποθάνῃ,
\vs{21}ἢ διὰ μῆνιν ἐπάταξεν αὐτὸν τῇ χειρί, καὶ ἀποθάνῃ, θανάτῳ θανατούσθω ὁ πατάξας, φονευτής ἐστι· θανάτῳ θανατούσθω ὁ φονεύων· ὁ ἀγχιστεύων τὸ αἷμα ἀποκτενεῖ τὸν φονεύσαντα ἐν τῷ συναντῆσαι αὐτῷ.

\vs{22}Ἐὰν δὲ ἐξάπινα, οὐ διʼ ἔχθραν ὤσῃ αὐτὸν, ἢ ἐπιῤῥίψῃ ἐπʼ αὐτὸν πᾶν σκεῦος, οὐκ ἐξ ἐνέδρου,
\vs{23}ἢ παντὶ λίθῳ, ἐν ᾧ ἀποθανεῖται ἐν αὐτῷ, οὐκ εἰδὼς, καὶ ἐπιπέσῃ ἐπʼ αὐτὸν, καὶ ἀποθάνῃ, αὐτὸς δὲ οὐκ ἐχθρὸς αὐτοῦ ἦν, οὐδὲ ζητῶν κακοποιῆσαι αὐτόν·
\vs{24}καὶ κρινεῖ ἡ συναγωγὴ ἀναμέσον τοῦ πατάξαντος καὶ ἀναμέσον τοῦ ἀγχιστεύοντος τὸ αἷμα, κατὰ τὰ κρίματα ταῦτα.
\vs{25}Καὶ ἐξελεῖται ἡ συναγωγὴ τὸν φονεύσαντα ἀπὸ τοῦ ἀγχιστεύοντος τὸ αἷμα, καὶ ἀποκαταστήσουσιν αὐτὸν ἡ συναγωγὴ εἰς τὴν πόλιν τοῦ φυγαδευτηρίου αὐτοῦ, οὗ κατέφυγε, καὶ κατοικήσει ἐκεῖ ἕως ἂν ἀποθάνῃ ὁ ἱερεὺς ὁ μέγας, ὃν ἔχρισαν αὐτὸν τῷ ἐλαίῳ τῷ ἁγίῳ.

\vs{26}Ἐὰν δὲ ἐξόδῳ ἐξέλθῃ ὁ φονεύασς τὰ ὅρια τῆς πόλεως εἰς ἣν κατέφυγεν ἐκεῖ,
\vs{27}καὶ εὕρῃ αὐτὸν ὁ ἀγχιστεύων τὸ αἷμα ἔξω τῶν ὁρίων τῆς πόλεως καταφυγῆς αὐτοῦ, καὶ φονεύσῃ ὁ ἀγχιστεύων τὸ αἷμα τὸν φονεύσαντα, οὐκ ἔνοχός ἐστιν.
\vs{28}Ἐν γὰρ τῇ πόλει τῆς καταφυγῆς κατοικείτω ἕως ἂν ἀποθάνῃ ὁ ἱερεὺς ὁ μέγας· καὶ μετὰ τὸ ἀποθανεῖν τὸν ἱερέα τὸν μέγαν, ἐπαναστραφήσεται ὁ φονεύσας εἰς τὴν γῆν τῆς κατασχέσεως αὐτοῦ.

\vs{29}Καὶ ἔσται ταῦτα ὑμῖν εἰς δικαίωμα κρίματος εἰς τὰς γενεὰς ὑμῶν ἐν πάσαις ταῖς κατοικίαις ὑμῶν.
\vs{30}Πᾶς πατάξας ψυχήν, διὰ μαρτύρων φονεύσεις τὸν φονεύσαντα· καὶ μάρτυς εἷς οὐ μαρτυρήσει ἐπὶ ψυχὴν ἀποθανεῖν.
\vs{31}Καὶ οὐ λήψεσθε λύτρα περὶ ψυχῆς παρὰ τοῦ φονεύσαντος τοῦ ἐνόχου ὄντος ἀναιρεθῆναι· θανάτῳ γὰρ θανατωθήσεται.
\vs{32}Οὐ λήψεσθε λύτρα τοῦ φυγεῖν εἰς πόλιν τῶν φυγαδευτηρίων, τοῦ πάλιν κατοικεῖν ἐπὶ τῆς γῆς, ἕως ἂν ἀποθάνῃ ὁ ἱερεὺς ὁ μέγας.
\vs{33}Καὶ οὐ μὴ φονοκτονήσητε τὴν γῆν εἰς ἣν ὑμεῖς κατοικεῖτε· τὸ γὰρ αἷμα τοῦτο φονοκτονεῖ τὴν γῆν, καὶ οὐκ ἐξιλασθήσεται ἡ γῆ ἀπὸ τοῦ αἵματος τοῦ ἐκχυθέντος ἐπʼ αὐτῆς, ἀλλʼ ἐπὶ τοῦ αἵματος τοῦ ἐκχέοντος.
\vs{34}Καὶ οὐ μιανεῖτε τὴν γῆν ἐφʼ ἧς κατοικεῖτε ἐπʼ αὐτῆς, ἐφʼ ἧς ἐγὼ κατασκηνώ ἐν ὑμῖν· ἐγὼ γάρ εἰμι Κύριος κατασκηνῶν ἐν μέσῳ τῶν υἱῶν Ἰσραήλ.

\ch{36}
Καὶ προσῆλθον οἱ ἄρχοντες φυλῆς υἱῶν Γαλαὰδ υἱοῦ Μαχὶρ υἱοῦ Μανασσῆ ἐκ τῆς φυλῆς υἱῶν Ἰωσὴφ, καὶ ἐλάλησαν ἔναντι Μωυσῆ, καὶ ἔναντι Ἐλεάζαρ τοῦ ἱερέως, καὶ ἔναντι τῶν ἀρχόντων οἴκων πατριῶν τῶν υἱῶν Ἰσραὴλ,
\vs{2}καὶ εἶπαν, τῷ κυρῖῳ ἡμῶν ἐνετείλατο Κύριος ἀποδοῦναι τὴν γῆν τῆς κληρονομίας ἐν κλήρῳ τοῖς υἱοῖς Ἰσραήλ· καὶ τῷ κυρίῳ συνέταξε Κύριος δοῦναι τὴν κληρονομίαν Σαλπαὰδ τοῦ ἀδελφοῦ ἡμῶν ταῖς θυγατράσιν αὐτοῦ.
\vs{3}Καὶ ἔσονται ἑνὶ τῶν φυλῶν υἱῶν Ἰσραὴλ γυναῖκες· καὶ ἀφαιρεθήσεται ὁ κλῆρος αὐτῶν ἐκ τῆς κατασχέσεως τῶν πατέρεν ἡμῶν, καὶ προστεθήσεται εἰς κληρονομίαν τῆς φυλῆς, οἷς ἂν γένωνται γυναῖκες, καὶ ἐκ τοῦ κλήρου τῆς κληρονομίας ἡμῶν ἀφαιρεθήσεται.
\vs{4}Ἐὰν δὲ γένηται ἡ ἄφεσις τῶν υἱῶν Ἰσραὴλ, καὶ προστεθήσεται ἡ κληρονομία αὐτῶν ἐπὶ τὴν κληρονομίαν τῆς φυλῆς, οἷς ἂν γένωνται γυναῖκες, καὶ ἀπὸ τῆς κληρονομίας φυλῆς πατριᾶς ἡμῶν ἀφαιρεθήσεται ἡ κληρονομία αὐτῶν.

\vs{5}Καὶ ἐνετείλατο Μωυσῆς τοῖς υἱοῖς Ἰσραὴλ διὰ προστάγματος Κυρίου, λέγων, οὕτως φυλὴ υἱῶν Ἰωσὴφ λέγουσι.
\vs{6}Τοῦτο τὸ ῥῆμα ὃ συνέταξε Κύριος ταῖς θυγατράσι Σαλπαὰδ, λέγων, οὗ ἀρέσκῃ ἐναντίον αὐτῶν, ἔστωσαν γυναῖκες, πλὴν ἐκ τοῦ δήμου τοῦ πατρὸς αὐτῶν ἔστωσαν γυναῖκες.
\vs{7}Καὶ οὐχὶ περιστραφήσεται κληρονομία τοῖς υἱοῖς Ἰσραὴλ ἀπὸ φυλῆς ἐπὶ φύλην, ὅτι ἕκαστος ἐν τῇ κληρονομίᾳ τῆς φυλῆς τῆς πατριᾶς αὐτοῦ προσκολληθήσονται οἱ υἱοὶ Ἰσραήλ.
\vs{8}Καὶ πᾶσα θυγάτηρ ἀγχιατεύουσα κληρονομίαν ἐκ τῶν φυλῶν υἱῶν Ἰσραὴλ, ἑνὶ τῶν ἐκ τοῦ δήμου τοῦ πατρὸς αὐτῆς ἔσονται γυναῖκες, ἵνα ἀγχιστεύσωσιν οἱ υἱοὶ Ἰσραὴλ ἕκαστος τὴν κληρονομίαν τὴν πατρικὴν αὐτοῦ.
\vs{9}Καὶ οὐ περιστραφήσεται ὁ κλῆρος ἐκ φυλῆς ἐπὶ φυλὴν ἑτέραν, ἀλλʼ ἕκαστος ἐν τῇ κληρονομίᾳ αὐτοῦ προσκολληθήσονται οἱ υἱοὶ Ἰσραήλ.

\vs{10}Ὃν τρόπον συνέταξε Κύριος Μωυσῇ, οὕτως ἐποίησαν θυγατράσι Σαλπαάδ.
\vs{11}Καὶ ἐγένοντο Θερσὰ καὶ Ἐγλὰ καὶ Μελχὰ καὶ Νούα καὶ Μαλαὰ θυγατέρες Σαλπαὰδ, τοῖς ἀνεψιοῖς αὐτῶν,
\vs{12}ἐκ τοῦ δήμου τοῦ Μανασσῆ υἱῶν Ἰωσὴφ ἐγενήθησαν γυναῖκες· καὶ ἐγενήθη ἡ κληρονομία αὐτῶν ἐπὶ τὴν φυλὴν δήμου τοῦ πατρὸς αὐτῶν.
\vs{13}Αὗται αἱ ἐντολαὶ καὶ τὰ δικαιώματα καὶ τὰ κρίματα, ἃ ἐνετείλατο Κύριος ἐν χειρὶ Μωυσῆ ἐπὶ δυσμῶν Μωὰβ ἐπὶ τοῦ Ἰορδάνου κατὰ Ἰεριχώ.


\def\book{ΔΕΥΤΕΡΟΝΟΜΙΟΝ}
\biblebook{ΔΕΥΤΕΡΟΝΟΜΙΟΝ}


\lettrine[lines=2, loversize=0.2, nindent=0em, findent=.25em]{\textcolor{bookheadingcolor}{Ο}}{ΥΤΟΙ} οἱ λόγοι οὓς ἐλάλησε Μωυσῆς παντὶ Ἰσραὴλ πέραν τοῦ Ἰορδάνου ἐν τῇ ἐρήμῳ πρὸς δυσμαῖς πλησίον τῆς ἐρυθρᾶς θαλάσσης ἀναμέσον Φαρὰν Τοφὸλ, καὶ Λοβὸν, καὶ Αὐλῶν, καὶ καταχρύσεα.
\vs{2}Ἕνδεκα ἡμερῶν ἐκ Χωρὴβ ὁδὸς ἐπʼ ὄρος
\vs{3}Σηεὶρ ἕως Κάδης Βαρνή. Καὶ ἐγενήθη ἐν τῷ τεσσαρακοστῷ ἔτει ἐν τῷ ἑνδεκάτῳ μηνὶ μιᾷ τοῦ μηνὸς, ἐλάλησε Μωυσῆς πρὸς πάντας υἱοὺς Ἰσραὴλ, κατὰ πάντα ὅσα ἐνετείλατο Κύριος αὐτῷ πρὸς αὐτούς· μετὰ τὸ πατάξαι
\vs{4}Σηὼν βασιλέα Ἀμοῤῥαίων τὸν κατοικήσαντα ἐν Ἐσεβὼν, καὶ τὸν Ὢγ βασιλέα τῆς Βασὰν τὸν κατοικήσαντα ἐν Ἀσταρὼθ καὶ ἐν
\vs{5}Ἐδραῒν, ἐν τῷ πέραν τοῦ Ἰορδάνου ἐν γῇ Μωὰβ, ἤρξατο Μωυσῆς διασαφῆσαι τὸν νόμον τοῦτον, λέγων,
\vs{6}Κύριος ὁ Θεὸς ἡμῶν ἐλάλησεν ἡμῖν ἐν Χωρὴβ, λέγων, ἰκανούσθω ὑμῖν κατοικεῖν ἐν τῷ ὄρει τούτῳ.
\vs{7}Ἐπιστράφητε καὶ ἀπάρατε ὑμεῖς καὶ εἰσπορεύεσθε εἰς ὄρος Ἀμοῤῥαίων, καὶ πρὸς πάντας τοὺς περιοίκους Ἄραβα, εἰς ὄρος καὶ πεδίον, καὶ πρὸς Λίβα, καὶ παραλίαν γῆν Χαναναίων, καὶ Ἀντιλίβανον ἕως τοῦ ποταμοῦ τοῦ μεγάλου, ποταμοῦ Εὐφράτου.
\vs{8}Ἴδετε, παραδέδωκεν ἐνώπιον ὑμῶν τὴν γῆν· εἰσπορευθέντες κληρονομήσατε τὴν γῆν, ἣν ὤμοσα τοῖς πατράσιν ὑμῶν τῷ Ἁβραὰμ, καὶ Ἰσαὰκ, καὶ Ἰσκὼβ, δοῦναι αὐτοῖς καὶ τῷ σπέρματι αὐτῶν μετʼ αὐτούς.

\vs{9}Καὶ εἶπα πρὸς ὑμᾶς ἐν τῷ καιρῷ ἐκείνῳ, λέγων, οὐ δυνήσομαι μόνος φέρειν ὑμᾶς.
\vs{10}Κύριος ὁ Θεὸς ὑμῶν ἐπλήθυνεν ὑμᾶς, καὶ ἰδού ἐστε σήμερον ὡσεὶ τὰ ἄστρα τοῦ οὐρανοῦ τῷ πλήθει.
\vs{11}Κύριος ὁ Θεὸς τῶν πατέρων ὑμῶν προσθείῃ ὑμῖν ὡς ἐστὲ χιλιοπλασίως, καὶ εὐλογήσαι ὑμᾶς καθότι ἐλάλησεν ὑμῖν.
\vs{12}Πῶς δυνήσομαι μόνος φέρειν τὸν κόπον ὑμῶν καὶ τὴν ὑπόστασιν ὑμῶν καὶ τὰς ἀντιλογίας ὑμῶν;
\vs{13}Δότε ἑαυτοῖς ἄνδρας σοφοὺς καὶ ἐπιστήμονας καὶ συνετοὺς εἰς τὰς φυλὰς ὑμῶν, καὶ καταστήσω ἐφʼ ὑμῶν, ἡγουμένους ὑμῶν.
\vs{14}Καὶ ἀπεκρίθητέ μοι, καὶ εἴπατε, καλὸν τὸ ῥῆμα ὃ ἐλάλησας ποιῆσαι.
\vs{15}Καὶ ἔλαβον ἐξ ὑμῶν ἄνδρας σοφοὺς καὶ ἐπιστήμονας καὶ συνετούς, καὶ κατέστησα αὐτοὺς ἡγεῖσθαι ἐφʼ ὑμῶν χιλιάρχους, καὶ ἑκατοντάρχους, καὶ πεντηκοντάρχους, καὶ δεκάρχους, καὶ γραμματοεισαγωγεῖς τοῖς κριταῖς ὑμῶν·
\vs{16}Καὶ ἐνετειλάμην τοῖς κριταῖς ὑμῶν ἐν τῷ καιρῷ ἐκείνῳ, λέγων, διακούετε ἀναμέσον τῶν ἀδελφῶν ὑμῶν, καὶ κρίνατε δικαίως ἀναμέσον ἀνδρὸς, καὶ ἀναμέσον ἀδελφοῦ, καὶ ἀναμέσον προσηλύτου αὐτοῦ.
\vs{17}Οὐκ ἐπιγνώσῃ πρόσωπον ἐν κρίσει, κατὰ τὸν μικρὸν καὶ κατὰ τὸν μέγαν κρινεῖς, οὐ μὴ ὑποστείλῃ πρόσωπον ἀνθρώπου, ὅτι ἡ κρίσις τοῦ Θεοῦ ἐστι· καὶ τὸ ῥῆμα ὃ ἐὰν σκληρὸν ἠ· ἀφʼ ὑμῶν, ἀνοίσετε αὐτὸ ἐπʼ ἐμὲ, καὶ ἀκούσομαι αὐτό.
\vs{18}Καὶ ἐνετειλάμην ὑμῖν ἐν τῷ καιρῷ ἐκείνῳ πάντας τοὺς λόγους, οὓς ποιήσετε.

\vs{19}Καὶ ἀπάραντες ἐκ Χωρὴβ ἐπορεύθημεν πᾶσαν τὴν ἔρημον τὴν μεγάλην καὶ τὴν φοβερὰν ἐκείνην, ἣν εἴδετε, ὁδὸν ὄρους τοῦ Ἀμοῤῥαίου, καθότι ἐνετείλατο Κύριος ὁ Θεὸς ἡμῶν ἡμῖν, καὶ ἤλθομεν ἕως Κάδης Βαρνή.
\vs{20}Καὶ εἶπα πρὸς ὑμᾶς, ἤλθατε ἕως τοῦ ὄρους τοῦ Ἀμοῤῥαίου, ὃ Κύριος ὁ Θεὸς ἡμῶν δίδωσιν ὑμῖν.
\vs{21}Ἴδετε, παραδέδωκεν ὑμῖν Κύριος ὁ Θεὸς ὑμῶν πρὸ προσώπου ὑμῶν τὴν γῆν· ἀναβάντες κληρονομήσατε ὃν τρόπον εἶπε Κύριος ὁ Θεὸς τῶν πατέρων ὑμῶν ὑμῖν· μὴ φοβεῖσθε, μηδὲ δειλιάσητε.
\vs{22}Καὶ προσήλθατέ μοι πάντες, καὶ εἴπατε, Ἀποστείλωμεν ἄνδρας προτέρους ἡμῶν, καὶ ἐφοδευσάτωσαν ἡμῖν τὴν γῆν, καὶ ἀναγγειλάτωσαν ἡμῖν ἀπόκρισιν τὴν ὁδὸν διʼ ἧς ἀναβησόμεθα ἐν αὐτῇ, καὶ τὰς πόλεις εἰς ἃς εἰσπορευσόμεθα εἰς αὐτάς.
\vs{23}Καὶ ἤρεσεν ἐναντίον μου τὸ ῥῆμα· καὶ ἔλαβον ἐξ ὑμῶν δώδεκα ἄνδρας, ἄνδρα ἕνα κατά φυλήν.
\vs{24}Καὶ ἐπιστραφέντες ἀνέβησαν εἰς τὸ ὄρος, καὶ ἤλθοσαν ἕως φάραγγος βότρυος, καὶ κατεσκόπευσαν αὐτήν.
\vs{25}Καὶ ἐλάβοσαν ἐν ταῖς χερσὶν αὐτῶν ἀπὸ τοῦ καρποῦ τῆς γῆς, καὶ κατήνεγκαν πρὸς ὑμᾶς, καὶ ἔλεγον, Ἀγαθὴ ἡ γῆ, ἣν Κύριος ὁ Θεὸς ἡμῶν δίδωσιν ἡμῖν.

\vs{26}Καὶ οὐκ ἠθελήσατε ἀναβῆναι, ἀλλʼ ἠπειθήσατε τῷ ῥήματι Κυρίου τοῦ Θεοῦ ἡμῶν.
\vs{27}Καὶ διεγογγύζετε ἐν ταῖς σκηναῖς ὑμῶν, καὶ εἴπατε, διὰ τὸ μισεῖν Κύριον ἡμᾶς, ἐξήγαγεν ἡμᾶς ἐκ γῆς Αἰγύπτου παραδοῦναι ἡμᾶς εἰς χεῖρας Ἀμοῤῥαίων, ἐξολοθρεῦσαι ἡμᾶς.
\vs{28}Ποῦ ἡμεῖς ἀναβαίνομεν; οἱ δὲ ἀδελφοὶ ὑμῶν ἀπέστησαν τὴν καρδίαν ὑμῶν, λέγοντες, ἔθνος μέγα καὶ πολὺ καὶ δυνατώτερον ἡμῶν, καὶ πόλεις μεγάλαι καὶ τετειχισμέναι ἕως τοῦ οὐρανοῦ· ἀλλὰ καὶ υἱοὺς γιγάντων ἑωράκαμεν ἐκεῖ.
\vs{29}Καὶ εἶπα πρὸς ὑμᾶς, μὴ πτήξητε, μηδὲ φοβηθῆτε ἀπʼ αὐτῶν.
\vs{30}Κύριος ὁ Θεὸς ὑμῶν ὁ προπορευόμενος πρὸ προσώπου ὑμῶν, αὐτὸς συνεκπολεμήσει αὐτοὺς μεθʼ ὑμῶν κατὰ πάντα ὅσα ἐποίησεν ὑμῖν ἐν γῇ Αἰγύπτῳ, καὶ ἐν τῇ ἐρήμῳ ταύτῃ ἣν εἴδετε, ὁδὸν ὄρους τοῦ Ἀμοῤῥαίου· ὡς τροφοφορήσαι σε Κύριος ὁ
\vs{31}Θεός σου, ὡς εἴτις τροφοφορήσαι ἄνθρωπος τὸν υἱὸν αὐτοῦ, κατὰ πᾶσαν τὴν ὁδὸν εἰς ἣν ἐπορεύθητε ἕως ῆλθετε εἰς τὸν τόπον τοῦτον.

\vs{32}Καὶ ἐν τῷ λόγῳ τούτῳ οὐκ ἐνεπιστεύσατε Κυρίῳ τῷ Θεῷ ἡμῶν,
\vs{33}ὃς προπορεύεται πρότερος ὑμῶν ἐν τῇ ὁδῷ ἐκλέγεσθαι ὑμῖν τόπον, ὁδηγῶν ὑμᾶς ἐν πυρὶ νυκτὸς, δεικνύων ὑμῖν τὴν ὁδὸν καθʼ ἣν πορεύεσθε ἐπʼ αὐτῆς, καὶ ἐν νεφέλῃ ἡμέρας.

\vs{34}Καὶ ἤκουσε Κύριος τὴν φωνὴν τῶν λόγων ὑμῶν, καὶ παροξυνθεὶς ὤμοσε, λέγων,
\vs{35}εἰ ὄψεταί τις τῶν ἀνδρῶν τούτων τὴν γῆν ἀγαθὴν ταύτην, ἣν ὤμοσα τοῖς πατράσιν αὐτῶν, πλὴν
\vs{36}Χάλεβ υἱὸς Ἰεφοννὴ, οὗτος ὄψεται αὐτὴν, καὶ τούτῳ δώσω τὴν γῆν ἐφʼ ἣν ἐπέβη, καὶ τοῖς υἱοῖς αὐτοῦ, διὰ τὸ προσκεῖσθαι αὐτὸν τὰ πρὸς Κύριον.
\vs{37}Καὶ ἐμοὶ ἐθυμώθη Κύριος διʼ ὑμᾶς, λέγων, οὐδὲ σὺ οὐ μὴ εἰσέλθῃς ἐκεῖ.
\vs{38}Ἰησοῦς υἱὸς Ναυὴ ὁ παρεστηκώς σοι, οὗτος εἰσελεύσεται ἐκεῖ· αὐτὸν κατίσχυσον, ὅτι αὐτὸς κατακληρονομήσει αὐτὴν τῷ Ἰσραήλ.
\vs{39}Καὶ πᾶν παιδίον νέον ὅστις οὐκ οἶδε σήμερον ἀγαθὸν ἢ κακόν, οὗτοι εἰσελεύσονται ἐκεῖ, καὶ τούτοις δώσω αὐτήν, καὶ αὐτοὶ κληρονομήσουσιν αὐτήν.
\vs{40}Καὶ ὑμεῖς ἐπιστράφεντες ἐστρατοπεδεύσατε εἰς τὴν ἔρημον, ὁδὸν τὴν ἐπὶ τῆς ἐρυθρᾶς θαλάσσης.

\vs{41}Καὶ ἀπεκρίθητε, καὶ εἴπατε, ἡμάρτομεν ἔναντι Κυρίου τοῦ Θεοῦ ἡμῶν· ἡμεῖς ἀναβάντες πολεμήσομεν κατὰ πάντα ὅσα ἐνετείλατο Κύριος ὁ Θεὸς ἡμῶν ἡμῖν· καὶ ἀναλαβόντες ἕκαστος τὰ σκεύη τὰ πολεμικὰ αὐτοῦ, καὶ συναθροισθέντες ἀναβαίνετε εἰς τὸ ὄρος.
\vs{42}Καὶ εἶπε Κύριος πρὸς μὲ, εἶπον αὐτοῖς, οὐκ ἀναβήσεσθε οὐδὲ μὴ πολεμήσετε, οὐ γάρ εἰμι μεθʼ ὑμῶν, καὶ οὐ μὴ συντριβῆτε ἐνώπιον τῶν ἐχθρῶν ὑμῶν.
\vs{43}Καὶ ἐλάλησα ὑμῖν, καὶ οὐκ εἰσηκούσατέ μου· καὶ παρέβητε τὸ ῥῆμα Κυρίου· καὶ παραβιασάμενοι ἀνέβητε εἰς τὸ ὄρος.
\vs{44}Καὶ ἐξῆλθεν ὁ Ἀμοῤῥαῖος ὁ κατοικῶν ἐν τῷ ὄρει ἐκείνῳ εἰς συνάντησιν ὑμῖν, καὶ κατεδίωξεν ὑμᾶς ὡσεὶ ποιήσαισαν αἱ μέλισσαι, καὶ ἐτίτρωσκον ὑμᾶς ἀπὸ Σηεὶρ ἕως Ἑρμᾶ.
\vs{45}Καὶ καθίσαντες ἐκλαίετε ἐναντίον Κυρίου τοῦ Θεοῦ ἡμῶν, καὶ οὐκ εἰσήκουσε Κύριος τῆς φωνῆς ὑμῶν, οὐδὲ προσέσχεν ὑμῖν.

\vs{46}Καὶ ἐνεκάθησθε ἐν Κάδης ἡμέρας πολλάς, ὅσας ποτὲ ἡμέρας ἐνεκάθησθε.

\ch{2}
Καὶ ἐπιστραφέντες ἀπῄραμεν εἰς τὴν ἔρημον, ὁδὸν θάλασσαν ἐρυθράν, ὃν τρόπον ἐλάλησε Κύριος πρὸς μὲ, καὶ ἐκυκλώσαμεν τὸ ὄρος τὸ Σηεὶρ ἡμέρας πολλάς.
\vs{2}Καὶ εἶπε Κύριος πρὸς μέ,
\vs{3}ἱκανούσθω ὑμῖν κυκλούν τὸ ὄρος τοῦτο· ἐπιστράφητε οὖν ἐπὶ Βοῤῥᾶν·
\vs{4}Καὶ τῷ λαῷ ἔντειλαι, λέγων, ὑμεῖς παραπορεύεσθε διὰ τῶν ὁρίων τῶν ἀδελφῶν ὑμῶν υἱῶν Ἡσαύ, οἳ κατοικοῦσιν ἐν Σηεὶρ, καὶ φοβηθήσονται ὑμᾶς, καὶ εὐλαβηθήσονται ὑμᾶς σφόδρα.
\vs{5}Μὴ συνάψητε πρὸς αὐτοὺς πόλεμον, οὐ γὰρ δῶ ὑμῖν ἀπὸ τῆς γῆς αὐτῶν οὐδὲ βῆμα ποδός, ὅτι ἐν κλήρῳ δέδωκα τοῖς υἱοῖς Ἡσαὺ τὸ ὄρος τὸ Σηείρ.
\vs{6}Ἀργυρίου βρώματα ἀγοράσατε παρʼ αὐτῶν καὶ φάγεσθε, καὶ ὕδωρ μέτρῳ λήψεσθε παρʼ αὐτῶν ἀργυρίου καὶ πίεσθε.
\vs{7}Ὁ γὰρ Κύριος ὁ Θεὸς ἡμῶν εὐλόγησέ σε ἐν παντὶ ἔργῳ τῶν χειρῶν σου· διάγνωθι πῶς διῆλθες τὴν ἔρημον τὴν μεγάλην καὶ τὴν φοβερὰν ἐκείνην· ἰδοὺ τεσσαράκοντα ἔτη Κύριος ὁ Θεός σου μετὰ σοῦ· οὐκ ἐπεδεήθης ῥήματος.

\vs{8}Καὶ παρήλθομεν τοὺς ἀδελφοὺς ἡμῶν υἱοὺς Ἡσαῦ, τοὺς κατοικοῦντας ἐν Σηεὶρ, παρὰ τὴν ὁδὸν τὴν Ἄραβα ἀπὸ Αἰλὼν καὶ ἀπὸ Γεσιὼν Γάβερ· καὶ ἐπιστρέψαντες παρήλθομεν ὁδὸν ἔρημον Μωάβ.
\vs{9}Καὶ εἶπε Κύριος πρὸς μέ, μὴ ἐχθραίνετε τοῖς Μωαβιταῖς, καὶ μὴ συνάψητε πρὸς αὐτοὺς πόλεμον· οὐ γὰρ μὴ δῶ ἀπὸ τῆς γῆς αὐτῶν ὑμῖν ἐν κλήρῳ, τοῖς γὰρ υἱοῖς Λὼτ δέδωκα τὴν Ἀροὴρ κληρονομεῖν.
\vs{10}Οἱ Ὀμμὶν πρότεροι ἐνεκάθηντο ἐπʼ αὐτῆς, ἔθνος μέγα καὶ πολὺ καὶ ἰσχύοντες, ὥσπερ οἱ Ἐνακίμ.
\vs{11}Ῥαφαῒν λογισθήσονται καὶ οὗτοι ὥσπερ καὶ οἱ Ἐνακίμ· καὶ οἱ Μωαβῖται ἐπονομάζουσιν αὐτοὺς Ὀμμείν.
\vs{12}Καὶ ἐν Σηεὶρ ἐνεκάθητο ὁ Χοῤῥαῖος τὸ πρότερον, καὶ υἱοὶ Ἡσαὺ ἀπώλεσαν αὐτοὺς, καὶ ἐξέτριψαν αὐτοὺς ἀπὸ προσώπου αὐτῶν. καὶ κατῳκίσθησαν ἀντʼ αὐτῶν, ὃν τρόπον ἐποίησεν Ἰσραὴλ τὴν γῆν τῆς κληρονομίας αὐτοῦ, ἣν δέδωκε Κύριος αὐτοῖς.
\vs{13}Νῦν οὖν ἀνάστητε καὶ ἀπάρατε ὑμεῖς, καὶ παραπορεύεσθε τὴν φάραγγα Ζάρετ.

\vs{14}Καὶ αἱ ἡμέραι ἃς παρεπορεύθημεν ἀπὸ Κάδης Βαρνὴ ἕως οὗ παρήλθομεν τὴν φάραγγα Ζαρὲτ, τριάκοντα καὶ ὀκτὼ ἔτη, ἕως οὗ διέπεσε πᾶσα γενεὰ ἀνδρῶν πολεμιστῶν ἀποθνήσκοντες ἐκ τῆς παρεμβολῆς, καθότι ὤμοσεν Κύριος ὁ
\vs{15}Θεός αὐτοὺς. Καὶ ἡ χεὶρ τοῦ Θεοῦ ἦν ἐπʼ αὐτοῖς ἐξαναλῶσαι αὐτοὺς ἐκ μέσου τῆς παρεμβολῆς ἕως οὗ διέπεσαν.

\vs{16}Καὶ ἐγενήθη ἐπειδὰν ἔπεσαν πάντες οἱ ἄνδρες οἱ πολεμισταὶ ἀποθνήσκοντες ἐκ μέσου τοῦ λαοῦ, καὶ ἐλάλησε
\vs{17}Κύριος πρὸς μὲ, λέγων, σὺ παραπορεύσῃ σήμερον τὰ ὅρια
\vs{18}Μωὰβ τὴν Ἀροὴρ,
\vs{19}καὶ προσάξετε ἐγγὺς υἱῶν Ἀμμάν· μὴ ἐχθραίνετε αὐτοῖς, μηδὲ συνάψετε αὐτοῖς εἰς πόλεμον· οὐ γὰρ μὴ δῶ ἀπὸ τῆς γῆς υἱῶν Ἀμμάν σοι ἐν κλήρῳ, ὅτι τοῖς υἱοῖς Λὼτ δέδωκα αὐτὴν ἐν κλήρῳ.
\vs{20}Γὴ Ῥαφαῒν λογισθήσεται, καὶ γὰρ ἐπʼ αὐτῆς κατῴκουν οἱ Ῥαφαῒν τοπρότερον καὶ οἱ Ἀμμανῖται ἐπονομάζουσιν αὐτοὺς Ζοχομμίν.
\vs{21}Ἔθνος μέγα καὶ πολὺ καὶ δυνατώτερον ὑμῶν, ὥσπερ καὶ οἱ Ἐνακείμ· καὶ ἀπώλεσεν αὐτοὺς Κύριος πρὸ πρόσωπου αὐτῶν, καὶ κατεκληρονόμησαν καὶ κατῳκίσθησαν, ἀντʼ αὐτῶν ἕως τῆς ἡμέρας ταύτης.
\vs{22}Ὥσπερ ἐποίησαν τοῖς υἱοῖς Ἡσαὺ τοῖς κατοικοῦσιν ἐν Σηείρ, ὃν τρόπον ἐξέτριψαν τὸν Χοῤῥαῖον ἀπὸ προσώπου αὐτῶν, καὶ κατεκληρονόμησαν αὐτοὺς, καὶ κατῳκίσθησαν ἀντʼ αὐτῶν ἕως τῆς ἡμέρας ταύτης.
\vs{23}Καὶ οἱ Εὐαῖοι οἱ κατοικοῦντες ἐν Ἀσηδὼθ ἕως Γάζης, καὶ οἱ Καππάδοκες οἱ ἐξελθόντες ἐκ Καππαδοκίας, ἐξέτριψαν αὐτοὺς, καὶ κατῳκίσθησαν ἀντʼ αὐτῶν.

\vs{24}Νῦν οὖν ἀνάστητε καὶ ἀπάρατε, καὶ παρέλθετε ὑμεῖς τὴν φάραγγα Ἀρνών· ἰδοὺ παραδέδωκα εἰς χεῖράς σου τὸν Σηὼν βασιλέα Ἐσεβὼν τὸν Ἀμοῤῥαῖον καὶ τὴν γῆν αὐτοῦ· ἐνάρχου κληρονομεῖν· σύναπτε πρὸς αὐτὸν πόλεμον ἐν τῇ ἡμέρᾳ ταύτῃ.
\vs{25}Ἐνάρχου δοῦναι τὸν τρόμον σου καὶ τὸν φόβον σου ἐπὶ προσώπου πάντων τῶν ἐθνῶν τῶν ὑποκάτω τοῦ οὐρανοῦ, οἵτινες ἀκούσαντες τὸ ὄνομά σου ταραχθήσονται, καὶ ὠδῖνας ἕξουσιν ἀπὸ προσώπου σου.

\vs{26}Καὶ ἀπέστειλα πρέσβεις ἐκ τῆς ἐρήμου Κεδαμὼθ πρὸς Σηὼν βασιλέα Ἐσεβὼν λόγοις εἰρηνικοῖς, λέγων,
\vs{27}παρελεύσομαι διὰ τῆς γῆς σου· ἐν τῇ ὁδῷ παρεύσομαι, οὐκ ἐκκλινῶ δεξιὰ οὐδʼ ἀριστερά.
\vs{28}Βρώματα ἀργυρίου ἀποδώσῃ μοι, καὶ φάγομαι· καὶ ὕδωρ ἀργυρίου ἀποδώσῃ μοι, καὶ πίομαι· πλὴν ὅτι παρελεύσομαι τοῖς ποσί·
\vs{29}Καθὼς ἐποίησάν μοι οἱ υἱοὶ Ἠσαῦ οἱ κατοικοῦντες ἐν Σηεὶρ, καὶ οἱ Μωαβῖται οἱ κατοικοῦντες ἐν Ἀροήρ· ἕως παρέλθω τὸν Ἰορδάνην εἰς τὴν γῆν, ἣν Κύριος ὁ Θεὸς ἡμῶν δίδωσιν ἡμῖν.
\vs{30}Καὶ οὐκ ἠθέλησεν Σηὼν βασιλεὺς Ἐσεβὼν παρελθεῖν ἡμᾶς διʼ αὐτοῦ, ὅτι ἐσκλήρυνε Κύριος ὁ Θεὸς ἡμῶν τὸ πνεῦμα αὐτοῦ, καὶ κατίσχυσε τὴν καρδίαν αὐτοῦ, ἵνα παραδοθῇ εἰς τὰς χεῖράς σου ὡς ἐν τῇ ἡμέρᾳ ταύτῃ.

\vs{31}Καὶ εἶπε Κύριος πρὸς μέ ἰδοὺ ἦργμαι παραδοῦναι πρὸ προσώπου σου τὸν Σηὼν βασιλέα Ἐσεβὼν τὸν Ἀμοῤῥαῖον, καὶ τὴν γῆν αὐτοῦ, καὶ ἔναρξαι κληρονομῆσαι τὴν γῆν αὐτοῦ.
\vs{32}Καὶ ἐξῆλθε Σηὼν βασιλεὺς Ἐσεβὼν εἰς συνάντησιν ἡμῖν, αὐτὸς καὶ πᾶς ὁ λαὸς αὐτοῦ, εἰς πόλεμον εἰς Ἰασσά.
\vs{33}Καὶ παρέδωκεν αὐτὸν Κύριος ὁ Θεὸς ἡμῶν πρὸ προσώπου ἡμῶν· καὶ ἐπατάξαμεν αὐτὸν καὶ τοὺς υἱοὺς αὐτοῦ καὶ πάντα τὸν λαὸν αὐτοῦ.
\vs{34}Καὶ ἐκρατήσαμεν πασῶν τῶν πόλεων αὐτοῦ ἐν τῷ καιρῷ ἐκείνῳ, καὶ ἐξωλοθρεύσαμεν πᾶσαν πόλιν ἑξῆς, καὶ τὰς γυναῖκας αὐτῶν καὶ τὰ τέκνα αὐτῶν· οὐ κατελίπομεν ζωγρίαν.
\vs{35}Πλὴν τὰ κτήνη ἐπρονομεύσαμεν, καὶ τὰ σκῦλα τῶν πόλεων ἐλάβομεν
\vs{36}ἐξ Ἀροὴρ, ἥ ἐστι παρὰ τὸ χεῖλος χειμάῤῥου Ἀρνών, καὶ τὴν πόλιν τὴν οὖσαν ἐν τῇ φάραγγι, καὶ ἕως ὄρους τοῦ Γαλαάδ· οὐκ ἐγενήθη πόλις ἥτις διέφυγεν ἡμᾶς. τὰς πάσας παρέδωκε Κύριος ὁ Θεὸς ἡμῶν εἰς τὰς χεῖρας ἡμῶν.
\vs{37}Πλὴν ἐγγὺς υἱῶν Ἀμμὰν οὐ προσήλθομεν πάντα τὰ συγκυροῦντα χειμάῤῥου Ἰαβὸκ, καὶ τὰς πόλεις τὰς ἐν τῇ ὀρεινῇ, καθότι ἐνετείλατο Κύριος ὁ Θεὸς ἡμῶν ἡμῖν.

\ch{3}
Καὶ ἐπιστραφέντες, ἀνέβημεν ὁδὸν τὴν εἰς Βασάν· καὶ ἐξῆλθεν Ὤγ βασιλεὺς τῆς Βασὰν εἰς συνάντησιν ἡμῖν, αὐτὸς καὶ πᾶς ὁ λαὸς αὐτου εἰς πόλεμον εἰς Ἑδραΐμ.
\vs{2}Καὶ εἶπε Κύριος πρὸς μέ, μὴ φοβηθῇς αὐτόν, ὅτι εἰς τὰς χεῖράς σου παραδέδωκα αὐτὸν, καὶ πάντα τὸν λαὸν αὐτοῦ, καὶ πᾶσαν τὴν γῆν αὐτοῦ· καὶ ποιήσεις αὐτῷ, ὥσπερ ἐποίησας Σηὼν βασιλεῖ τῶν Ἀμοῤῥαίων, ὃς κατῴκει ἐν Ἐσεβών.
\vs{3}Καὶ παρέδωκεν αὐτὸν Κύριος ὁ Θεὸς ἡμῶν εἰς τὰς χεῖρας ἡμῶν, καὶ τὸν Ὢγ βασιλέα τῆς Βασὰν, καὶ πάντα τὸν λαὸν αὐτοῦ· καὶ ἐπατάξαμεν αὐτὸν, ἕως τοῦ μὴ καταλιπεῖν αὐτοῦ σπέρμα.

\vs{4}Καὶ ἐκρατήσαμεν πασῶν τῶν πόλεων αὐτοῦ ἐν τῷ καιρῷ ἐκείνῷ· οὐκ ἦν πόλις, ἣν οὐκ ἐλάβομεν παρʼ αὐτῶν· ἑξήκοντα πόλεις, πάντα τὰ περίχωρα Ἀργὸβ βασιλέως Ὢγ ἐν Βασάν·
\vs{5}Πᾶσαι πόλεις ὀχυραί, τείχη ὑψηλά, πύλαι καὶ μοχλοί· πλὴν τῶν πόλεων τῶν Φερεζαίων τῶν πολλῶν σφόδρα.
\vs{6}Ἐξωλοθρεύσαμεν, ὥσπερ ἐποιήσαμεν τὸν Σηὼν βασιλέα Ἐσεβών, καὶ ἐξωλοθρεύσαμεν πᾶσαν πόλιν ἑξῆς, καὶ τὰς γυναῖκας, καὶ τὰ παιδία,
\vs{7}καὶ πάντα τὰ κτήνη· καὶ τὰ σκῦλα τῶν πόλεων ἐπρονομεύσαμεν ἑαυτοῖς.

\vs{8}Καὶ ἐλάβομεν ἐν τῷ καιρῷ ἐκείνῳ τὴν γῆν ἐκ χειρῶν δύο βασιλέων τῶν Ἀμοῤῥαίων, οἳ ἦσαν πέραν τοῦ Ἰορδάνου ἀπὸ τοῦ χειμάῤῥου Ἀρνὼν καὶ ἕως Ἀερμών·
\vs{9}Οἱ Φοίνικες ἐπονομάζουσιν τὸ Ἀερμὼν Σανιώρ, καὶ ὁ Ἀμοῤῥαῖος ἐπωνόμασεν αὐτὸ Σανίρ·
\vs{10}Πᾶσαι πόλεις Μισὼρ, καὶ πᾶσα Γαλαάδ, καὶ πᾶσα Βασὰν ἕως Ἑλχᾶ καὶ Ἑδραῒμ, πόλεις βασιλείας τοῦ Ὢγ ἐν τῇ Βασάν·
\vs{11}Ὅτι πλὴν Ὢγ βασιλεὺς Βασὰν κατελείφθη ἀπὸ τῶν Ῥαφαΐν· ἰδοὺ ἡ κλὶνη αὐτοῦ κλὶνη σιδηρᾶ, ἰδοὺ αὕτη ἐν τῇ ἄκρᾳ τῶν υἱῶν Ἀμμών· ἐννέα πήχεων τὸ μηκος αὐτῆς, καὶ τεσσάρων πήχεων τὸ εὖρος αὐτῆς ἐν πήχει ἀνδρός.
\vs{12}Καὶ τὴν γῆν ἐκείνην ἐκληρονομήσαμεν ἐν τῷ καιρῷ ἐκείνῳ ἀπὸ Ἀροήρ, ἥ ἐστι παρὰ τὸ χείλος χειμάῤρου Ἀρνών, καὶ τὸ ἥμισυ τοῦ ὄρους Γαλαάδ· καὶ τὰς πόλεις αὐτοῦ ἔδωκα τῷ Ῥουβὴν καὶ τῷ Γάδ.
\vs{13}Καὶ τὸ κατάλοιπον τοῦ Γαλαὰδ, καὶ πᾶσαν τὴν Βασάν βασιλείαν Ὤγ ἔδωκα τῷ ἡμίσει φυλῆς Μανασσή, καὶ πᾶσαν περίχωρον Ἀργόβ, πᾶσαν Βασὰν ἐκείνην· γῆ Ῥαφαῒν λογισθήσεται.
\vs{14}Καὶ Ἰαῒρ υἱὸς Μανασσὴ ἔλαβε πᾶσαν τὴν περίχωρον Ἀργὸβ ἕως τῶν ὁρίων Γαργασὶ καὶ Μαχαθί· ἐπωνόμασεν αὐτὰς ἐπὶ τῷ ὀνόματι αὐτοῦ τὴν Βασὰν Θαυὼθ Ἰαεῒρ ἕως τῆς ἡμέρας ταύτης.
\vs{15}Καὶ τῷ Μαχὶρ ἔδωκα τὴν Γαλαάδ.
\vs{16}Καὶ τῷ Ῥουβὴν καὶ τῷ Γάδ δέδωκα ὑπὸ τῆς Γαλαὰδ ἕως χειμάῤῥου Ἀρνών μέσον τοῦ χειμάῤῥου ὅριον καὶ ἕως τοῦ Ἰαβόκ· ὁ χειμάῤῥους ὅριον τοῖς υἱοῖς Ἀμμάν·
\vs{17}Καὶ ἡ Ἄραβα καὶ ὁ Ἰορδάνης ὅριον Μαχαναρὲθ, καὶ ἕως θαλάσσης Ἄραβά, θαλάσσης ἁλυκῆς ὑπὸ Ἀσηδὼθ τὴν Φασγὰ ἀνατολῶν.

\vs{18}Καὶ ἐνετειλάμην ὑμῖν ἐν τῷ καιρῷ ἐκείνῳ, λέγων, Κύριος ὁ Θεὸς ὑμῶν ἔδωκεν ὑμῖν τὴν γῆν ταύτην ἐν κλήρῳ· ἐνοπλισάμενοι προπορεύεσθε πρὸ προσώπου τῶν ἀδελφῶν ὑμῶν υἱῶν Ἰσραήλ πᾶς δυνατός.
\vs{19}Πλὴν αἱ γυναῖκες ὑμῶν καὶ τὰ τέκνα ὑμῶν καὶ τὰ κτήνη ὑμῶν, οἶδα ὅτι πολλὰ κτήνη ὑμῖν, κατοικείτωσαν ἐν ταῖς πόλεσιν ὑμῶν, αἷς ἔδωκα ὑμῖν, ἕως ἂν καταπαύσῃ Κύριος ὁ
\vs{20}Θεὸς ὑμῶν τοὺς ἀδελφοὺς ὑμῶν, ὥσπερ καὶ ὑμᾶς, καὶ κατακληρονομήσωσι καὶ οὗτοι τὴν γῆν, ἣν Κύριος ὁ Θεὸς ἡμῶν δίδωσιν αὐτοῖς ἐν τῷ πέραν τοῦ Ἰορδάνου· καὶ ἐπαναστραφήσεσθε ἕκαστος εἰς τὴν κληρονομίαν αὐτοῦ, ἣν ἔδωκα ὑμῖν.

\vs{21}Καὶ τῷ Ἰησοῖ ἐνετειλάμην ἐν τῷ καιρῷ ἐκείνῳ, λέγων, οἱ ὀφθαλμοὶ ὑμῶν ἑωράκασιν πάντα, ὅσα ἐποίησε Κύριος ὁ Θεὸς ἡμῶν τοῖς δυσὶ βασιλεῦσι τούτοις· οὕτως ποιήσει Κύριος ὁ Θεὸς ἡμῶν πάσας τὰς βασιλείας ἐφʼ ἃς σὺ διαβαίνεις ἐκεῖ.
\vs{22}Οὐ φοβηθήσεσθε ἀπʼ αὐτῶν, ὅτι Κύριος ὁ Θεὸς ἡμῶν αὐτὸς πολεμήσει περὶ ὑμῶν.

\vs{23}Καὶ ἐδεήθην Κυρίου ἐν τῷ καιρῷ ἐκείνῳ, λέγων,
\vs{24}Κύριε Θεὲ, σὺ ἤρξω δεῖξαι τῷ σῷ θεράποντι τὴν ἰσχύν σου, καὶ τὴν δύναμίν σου, καὶ τὴν χεῖρα τὴν κραταιὰν, καὶ τὸν βραχίονα τὸν ὑψηλόν· τίς γάρ ἐστι Θεὸς ἐν τῷ οὐρανῷ ἢ ἐπὶ τῆς γῆς, ὅστις ποιήσει καθὰ ἐποίησας σὺ, καὶ κατὰ τὴν ἰσχύν σου;
\vs{25}Διαβὰς οὖν ὄψομαι τὴν γῆν τὴν ἀγαθὴν ταύην τὴν οὖσαν πέραν τοῦ Ἰορδάνου, τὸ ὄρος τοῦτο τὸ ἀγαθὸν καὶ τὸν Ἀντιλίβανον.

\vs{26}Καὶ ὑπερεῖδε Κύριος ἐμὲ ἕνεκεν ὑμῶν, καὶ οὐκ εἰσήκουσέ μου· καὶ εἶπε Κύριος πρὸς μέ, ἱκανούσθω σοι, μὴ προσθῇς ἔτι λαλῆσαι τὸν λόγον τοῦτον.
\vs{27}Ἀνάβηθι ἐπὶ τὴν κορυφὴν τοῦ λελαξευμένου, καὶ ἀναβλέψας τοὶς ἀφθαλμοῖς σου κατὰ θάλασσαν καὶ Βοῤῥᾶν καὶ Λίβα καὶ ἀνατολάς, καὶ ἴδε τοῖς ὀφθαλμοῖς σου, ὅτι οὐ διαβήσῃ τὸν Ἰορδάνην τοῦτον.
\vs{28}Καὶ ἔντειλαι Ἰησοῖ καὶ κατίσχυσον αὐτὸν καὶ παρακάλεσον αὐτὸν, ὅτι οὗτος διαβήσεται πρὸ προσώπου τοῦ λαοῦ τούτου, καὶ οὗτος κατακληρονομήσει αὐτοῖς πᾶσαν τὴν γῆν ἣν ἑώρακας.
\vs{29}Καὶ ἐνεκαθήμεθα ἐν νάπῃ σύνεγγυς οἴκου Φογώρ.

\ch{4}
Καὶ νῦν Ἰσραὴλ ἄκουε τῶν δικαιωμάτων καὶ τῶν κριμάτων, ὅσα ἐγὼ διδάσκω ὑμᾶς σήμερον ποιεῖν, ἵνα ζῆτε, καὶ πολυπλασιασθῆτε, καὶ εἰσελθόντες κληρονομήσητε τὴν γῆν, ἣν Κύριος ὁ Θεὸς τῶν πατέρων ὑμῶν δίδωσιν ὑμῖν.
\vs{2}Οὐ προσθήσετε πρὸς τὸ ῥῆμα ὃ ἐγὼ ἐντέλλομαι ὑμῖν, καὶ οὐκ ἀφελεῖτε ἀπʼ αὐτοῦ· φυλάσσεσθε τὰς ἐντολὰς Κυρίου τοῦ Θεοῦ ἡμῶν, ὅσα ἐγὼ ἐντέλλομαι ὑμῖν σήμερον.
\vs{3}Οἱ ὀφθαλμοὶ ὑμῶν ἑωράκασι πάντα ὅσα ἐποίησε Κύριος ὁ Θεὸς ἡμῶν τῷ Βεελφεγὼρ, ὅτι πᾶς ἄνθρωπος ὅστις ἐπορεύθη ὀπίσω Βεελφεγὼρ, ἐξέτριψεν αὐτὸν Κύριος ὁ Θεὸς ὑμῶν ἐξ ὑμῶν.
\vs{4}Ὑμεῖς δὲ οἱ προσκείμενοι Κυρίῳ τῷ Θεῷ ὑμῶν, ζῆτε πάντες ἐν τῇ σήμερον.

\vs{5}Ἴδετε, δέδειχα ὑμῖν δικαιώματα καὶ κρίσεις καθὰ ἐνετείλατό μοι Κύριος, ποιῆσαι οὕτως ἐν τῇ γῇ εἰς ἣν ὑμεῖς εἰσπορεύεσθε ἐκεῖ κληρονομεῖν αὐτήν.
\vs{6}Καὶ φυλάξεσθε καὶ ποιήσετε· ὅτι αὕτη ἡ σοφία ὑμῶν καὶ ἡ σύνεσις ἐναντίον πάντων τῶν ἐθνῶν, ὅσοι ἂν ἀκούσωσι πάντα τὰ δικαιώματα ταῦτα· καὶ ἐροῦσιν, ἰδοὺ λαὸς σοφὸς καὶ ἐπιστήμων τὸ ἔθνος τὸ μέγα τοῦτο.
\vs{7}Ὅτι ποῖον ἔθνος μέγα, ᾧ ἐστιν αὐτῷ Θεὸς ἐγγίζων αὐτοῖς ὡς Κύριος ὁ Θεὸς ἡμῶν ἐν πᾶσιν οἷς ἐὰν αὐτὸν ἐπικαλεσώμεθα;
\vs{8}Καὶ ποῖον ἔθνος μέγα, ᾧ ἐστιν αὐτῷ δικαιώματα καὶ κρίματα δίκαια κατὰ πάντα τὸν νόμον τοῦτον, ὃν ἐγὼ δίδωμι ἐνώπιον ὑμῶν σήμερον;

\vs{9}Πρόσεχε σεαυτῷ, καὶ φύλαξον τὴν ψυχήν σου σφόδρα· μὴ ἐπιλάθῃ πάντας τοὺς λόγους, οὓς ἑωράκασιν οἱ ὀφθαλμοί σου, καὶ μὴ ἀποστήτωσαν ἀπὸ τῆς καρδίας σου πάσας τὰς ἡμέρας τῆς ζωῆς σου· καὶ συμβιβάσεις τοὺς υἱούς σου καὶ τοὺς υἱοὺς τῶν υἱῶν σου, ἡμέραν ἣν ἔστητε ἐνώπιον Κυρίου τοῦ Θεοῦ ἡμῶν ἐν Χωρὴβ τῇ ἡμέρᾳ τῆς ἐκκλησίας· ὅτε εἶπε
\vs{10}Κύριος πρὸς μὲ, ἐκκλησίασον πρὸς μὲ τὸν λαὸν, καὶ ἀκουσάτωσαν τὰ ῥήματά μου, ὅπως μάθωσι φοβεῖσθαί με πάσας τὰς ἡμέρας ἃς αὐτοὶ ζῶσιν ἐπὶ τῆς γῆς, καὶ τοὺς υἱοὺς αὐτῶν διδάξουσι.
\vs{11}Καὶ προσήλθετε καὶ ἔστητε ὑπὸ τὸ ὄρος· καὶ τὸ ὄρος ἐκαίετο πυρὶ ἕως τοῦ οὐρανοῦ· σκότος, γνόφος, θύελλα.
\vs{12}Καὶ ἐλάλησε Κύριος πρὸς ὑμᾶς ἐκ μέσου τοῦ πυρὸς φωνὴν ῥημάτων, ἣν ὑμεῖς ἠκούσατε· καὶ ὁμοίωμα οὐκ εἴδετε, ἀλλʼ ἢ φωνήν·
\vs{13}Καὶ ἀνήγγειλεν ὑμῖν τὴν διαθήκην αὐτοῦ, ἣν ἑνετείλατο ὑμῖν ποιεῖν, τὰ δέκα ῥήματα, καὶ ἔγραψεν αὐτὰ ἐπὶ δύο πλάκας λιθίνας.

\vs{14}Καὶ ἐμοὶ ἐνετείλατο Κύριος ἐν τῷ καιρῷ ἐκείνῳ, διδάξαι ὑμᾶς δικαιώματα καὶ κρίσεις, ποιεῖν ὑμᾶς αὐτὰ ἐπὶ τῆς γῆς, εἰς ἣν ὑμεῖς εἰσπορεύεσθε ἐκεῖ κληρονομῆσαι αὐτήν.
\vs{15}Καὶ φυλάξεσθε σφόδρα τὰς ψυχὰς ὑμῶν, ὅτι οὐκ εἴδετε ὁμοίωμα ἐν τῇ ἡμέρᾳ ᾗ ἐλάλησε Κύριος πρὸς ὑμᾶς ἐν Χωρὴβ ἐν τῷ ὄρει ἐκ μέσου τοῦ πυρός.
\vs{16}Μὴ ἀνομήσητε καὶ ποιήσητε ὑμῖν ἑαυτοῖς γλυπτὸν ὁμοίωμα, πᾶσαν εἰκόνα ὁμοίωμα ἀρσενικοῦ ἢ θηλυκοῦ,
\vs{17}ὁμοίωμα παντὸς κτήνους τῶν ὄντων ἐπὶ τῆς γῆς, ὁμοίωμα παντὸς ὀρνέου πτερωτοῦ ὃ πέταται ὑπὸ τὸν οὐρανὸν,
\vs{18}ὁμοίωμα παντὸς ἑρπετοῦ ὃ ἕρπει ἐπὶ τῆς γῆς, ὁμοίωμα παντὸς ἰχθύος, ὅσα ἐστὶν ἐν τοῖς ὕδασιν ὑποκάτω τῆς γῆς.
\vs{19}Καὶ μὴ ἀναβλέψας εἰς τὸν οὐρανὸν, καὶ ἰδὼν τὸν ἥλιον καὶ τὴν σελήνην καὶ τοὺς ἀστέρας, καὶ πάντα τὸν κόσμον τοῦ οὐρανοῦ, πλανηθεὶς προσκυνήσῃς αὐτοῖς, καὶ λατρεύσῃς αὐτοῖς, ἃ ἀπένειμε Κύριος ὁ Θεός σου αὐτὰ πᾶσι τοῖς ἔθνεσι τοῖς ὑποκάτω τοῦ οὐρανοῦ.
\vs{20}Ὑμᾶς δὲ ἔλαβεν ὁ Θεὸς, καὶ ἐξήγαγεν ὑμᾶς ἐκ γῆς Αἰγύπτου, ἐκ τῆς καμίνου τῆς σιδηρᾶς, ἐξ Αἰγύπτου, εἶναι αὐτῷ λαὸν ἔγκληρον, ὡς ἐν τῇ ἡμέρᾳ ταύτῃ.

\vs{21}Καὶ Κύριος ὁ Θεὸς ἐθυμώθη μοι περὶ τῶν λεγομένων ὑφʼ ὑμῶν, καὶ ὤμοσεν ἵνα μὴ διαβῶ τὸν Ἰορδάνην τοῦτον, καὶ ἵνα μὴ εἰσέλθω εἰς τὴν γῆν, ἣν Κύριος ὁ Θεός σου δίδωσί σοι ἐν κλήρῳ.
\vs{22}Ἐγὼ γὰρ ἀποθνήσκω ἐν τῇ γῇ ταύτῃ, καὶ οὐ διαβαίνω τὸν Ἰορδάνην τοῦτον· ὑμεῖς δὲ διαβαίνετε, καὶ κληρονομήσετε τὴν γῆν τὴν ἀγαθὴν ταύτην.
\vs{23}Προσέχετε ὑμῖν, μὴ ἐπιλάθησθε τὴν διαθήκην Κύριου τοῦ Θεοῦ ἡμῶν, ἣν διέθετο πρὸς ὑμᾶς, καὶ ἀνομήσητε, καὶ ποιήσητε ὑμῖν ἑαυτοῖς γλυπτὸν ὁμοίωμα πάντων ὧν συνέταξέ σοι Κύριος ὁ Θεός σου.
\vs{24}Ὅτι Κύριος ὁ Θεός σου πῦρ καταναλίσκον ἐστί, Θεὸς ζηλωτής.

\vs{25}Ἐὰν δὲ γεννήσῃς υἱοὺς καὶ υἱοὺς τῶν υἱῶν σου, καὶ χρονίσητε ἐπὶ τῆς γῆς, καὶ ἀνομήσητε, καὶ ποιήσετε γλυπτὸν ὁμοίωμα παντός, καὶ ποιήσητε τὸ πονηρὸν ἐνώπιου Κυρίου τοῦ
\vs{26}Θεοῦ ὑμῶν παροργίσαι αὐτόν, διαμαρτύρομαι ὑμῖν σήμερον τὸν τε οὐρανὸν καὶ τὴν γῆν, ὅτι ἀπωλίᾳ ἀπολεῖσθε ἀπὸ τῆς γῆς, εἰς ἣν ὑμεῖς διαβαίνετε τὸν Ἰορδάνην ἐκεῖ κληρονομῆσαι· οὐχὶ πολυχρονιεῖτε ἡμέρας ἐπʼ αὐτῆς, ἀλλʼ ἢ ἐκτριβῇ ἐκτριβήσεσθε.
\vs{27}Καὶ διασπερεῖ Κύριος ὑμᾶς ἐν πᾶσι τοῖς ἔθνεσι, καὶ καταλειφθήσεσθε ὀλίγοι ἀριθμῷ ἐν πᾶσι τοῖς ἔθνεσιν, εἰς οὓς εἰσάξει Κύριος ὑμᾶς ἐκεῖ.
\vs{28}Καὶ λατρεύσετε ἐκεῖ θεοῖς ἑτέροις ἔργοις χειρῶν ἀνθρώπων, ξύλοις καὶ λίθοις, οἳ οὐκ ὄψονται, οὔτε μὴ ἀκούσωσιν, οὔτε μὴ φάγωσιν, οὔτε μὴ ὀσφρανθῶσι.
\vs{29}Καὶ ζητήσετε ἐκεῖ Κύριον τὸν Θεὸν ὑμῶν, καὶ εὑρήσετε αὐτὸν ὅταν ἐκζητήσητε αὐτὸν ἐξ ὅλης τῆς καρδίας σου, καὶ ἐξ ὅλης τῆς ψυχῆς σου ἐν τῇ θλίψει σου·
\vs{30}Καὶ εὑρήσουσί σε πάντες οἱ λόγοι οὗτοι ἐπʼ ἐσχάτῳ τῶν ἡμερῶν, καὶ ἐπιστραφήσῃ πρὸς Κύριον τὸν Θεόν σου, καὶ εἰσακούσῃ τῆς φωνῆς σὐτοῦ·
\vs{31}Ὅτι Θεὸς οἰκτίρμων Κύριος ὁ Θεός σου· οὐκ ἐγκαταλείψει σε, οὐδὲ μὴ ἐκτρίψει σε· οὐκ ἐπιλήσεται τὴν διαθήκην τῶν πατέρων σου, ἣν ὤμοσεν αὐτοῖς Κύριος.

\vs{32}Ἐπερωτήσατε ἡμέρας προτέρας τὰς γενομένας προτέρας σου ἀπὸ τῆς ἡμέρας ἧς ἔκτισεν ὁ Θεὸς ἄνθρωπον ἐπὶ τῆς γῆς, καὶ ἐπὶ τὸ ἄκρον τοῦ οὐρανοῦ ἕως τοῦ ἄκρου τοῦ οὐρανοῦ, εἰ γέγονε κατὰ τὸ ῥῆμα τὸ μέγα τοῦτο, εἰ ἤκουσται τοιοῦτο· εἰ ἀκήκοεν ἔθνος φωνὴν
\vs{33}Θεοῦ ζῶντος λαλοῦντος ἐκ μέσου τοῦ πυρός, ὃν τρόπον ἀκήκοας σὺ καὶ ἔζησας·
\vs{34}εἰ ἐπείρασεν ὁ Θεὸς εἰσελθὼν λαβεῖν ἑαυτῷ ἔθνος ἐκ μέσου ἔθνους ἐν πειρασμῷ, καὶ ἐν σημείοις, καὶ ἐν τέρασι, καὶ ἐν πολέμῳ, καὶ ἐν χειρὶ κραταιᾷ, καὶ ἐν βραχίονι ὑψηλῷ, καὶ ἐν ὁράμασιν μεγάλοις, κατὰ πάντα ὅσα ἐποίησε Κύριος ὁ Θεὸς ἡμῶν ἐν Αἰγύπτῳ ἐνώπιόν σου βλέποντος·
\vs{35}ὥστε εἰδῆσαί σε ὅτι Κύριος ὁ Θεός σου οὗτος Θεός ἐστι, καὶ οὐκ ἔστιν ἔτι πλὴν αὐτοῦ.
\vs{36}Ἐκ τοῦ οὐρανοῦ ἀκουστὴ ἐγένετο ἡ φωνὴ αὐτοῦ παιδεῦσαί σε, καὶ ἐπὶ τῆς γῆς ἔδειξέ σοι τὸ πῦρ αὐτοῦ τὸ μέγα, καὶ τὰ ῥήματα αὐτοῦ ἤκουσας ἐκ μέσου τοῦ πυρός.

\vs{37}Διὰ τὸ ἀγαπῆσαι αὐτὸν τοὺς πατέρας σου, καὶ ἐξελέξατο τὸ σπέρμα αὐτῶν μετʼ αὐτοὺς ὑμᾶς, καὶ ἐξήγαγέ σε αὐτὸς ἐν τῇ ἰσχύϊ αὐτοῦ τῇ μεγάλῃ ἐξ Αἰγύπτου,
\vs{38}ἐξολοθρεῦσαι ἔθνη μεγάλα καὶ ἰσχυρότερά σου πρὸ προσώπου σου, εἰσαγαγεῖν σε δοῦναί σοι τὴν γῆν αὐτῶν κληρονομεῖν, καθὼς ἔχεις σήμερον.

\vs{39}Καὶ γνώσῃ σήμερον, καὶ ἐπιστραφήσῃ τῇ διανοίᾳ, ὅτι Κύριος ὁ Θεός σου οὗτος Θεὸς ἐν τῷ οὐρανῷ ἄνω καὶ ἐπὶ τῆς γῆς κάτω, καὶ οὐκ ἔστιν ἔτι πλὴν αὐτοῦ.
\vs{40}Καὶ φυλάξασθε τὰς ἐντολὰς αὐτοῦ, καὶ τὰ δικαιώματα αὐτοῦ, ὅσα ἐγὼ ἀντέλλομαί σοι σήμερον, ἵνα εὖ σοι γένηται καὶ τοῖς υἱοῖς σου μετὰ σὲ, ὅπως μακροήμεροι γένησθε ἐπὶ τῆς γῆς, ἧς Κύριος ὁ Θεός σου δίδωσί σοι πάσας τὰς ἡμέρας.
\vs{41}Τότε ἀφώρισε Μωυσῆς τρεῖς πόλεις πέραν τοῦ Ἰορδάνου ἀπὸ ἀνατολῶν ἡλίου,
\vs{42}φυγεῖν ἐκεῖ τὸν φονευτὴν ὃς ἂν φονεύσῃ τὸν πλησίον οὐκ εἰδὼς, καὶ οὗτος οὐ μισῶν αὐτὸν πρὸ τῆς χθὲς καὶ τῆς τρίτης, καὶ καταφεύξεται εἰς μίαν τῶν πόλεων τούτων, καὶ ζήσεται·
\vs{43}τὴν Βοσὸρ ἐν τῇ ἐρήμῳ ἐν τῇ γῇ τῇ πεδινῇ τῷ Ῥουβήν, καὶ τὴν Ῥαμὼθ ἐν Γαλαὰδ τῷ Γαδδί, καὶ τὴν Γαυλὼν ἐν Βασὰν τῷ Μανασσῄ.

\vs{44}Οὗτος ὁ νόμος, ὃν παρέθετο Μωυσῆς ἐνώπιον υἱῶν Ἰσραήλ.
\vs{45}Ταῦτα τὰ μαρτύρια, καὶ τὰ δικαιώματα, καὶ τὰ κρίματα, ὅσα ἐλάλησε Μωυσῆς τοῖς υἱοῖς Ἰσραήλ, ἐξελθόντων αὐτῶν ἐκ γῆς Αἰγύπτου,
\vs{46}ἐν τῷ πέραν τοῦ Ἰορδάνου, ἐν φάραγγι, ἐγγὺς οἴκου Φογώρ, ἐν γῇ Σηὼν βασιλέως τῶν Ἀμοῤῥαίων, ὃς κατῴκει ἐν Ἐσεβὼν, ὅν ἐπάταξε Μωυσῆς, καὶ οἱ υἱοὶ Ἰσραήλ, ἐξελθόντων αὐτῶν ἐκ γῆς Αἰγύπτου.
\vs{47}Καὶ ἐκληρονόμησαν τὴν γῆν αὐτοῦ, καὶ τὴν γῆν Ὢγ βασιλέως τῆς Βασὰν, δύο βασιλέων τῶν Ἀμοῤῥαίων, οἳ ἦσαν πέραν τοῦ Ἰορδάνου κατὰ ἀνατολὰς ἡλίου,
\vs{48}ἀπὸ Ἀροὴρ, ἥ ἐστιν ἐπὶ τοῦ χείλους χειμάῤῥου Ἀρνών, καὶ ἐπὶ τοῦ ὄρους τοῦ Σηὼν, ὅ ἐστιν Ἀερμὼν,
\vs{49}πᾶσαν τὴν Ἄραβα πέραν τοῦ Ἰορδάνου κατὰ ἀνατολὰς ἡλίου ὑπὸ Ἀσηδὼθ τὴν λαξευτήν.

\ch{5}
Καὶ ἐκάλεσε Μωυσῆς πάντα Ἰσραὴλ, καὶ εἶπε πρὸς αὐτούς, ἄκουε Ἰσραὴλ τὰ δικαιώματα καὶ τὰ κρίματα, ὅσα ἐγὼ λαλῶ ἐν τοῖς ὠσὶν ὑμῶν ἐν τῇ ἡμέρᾳ ταύτῃ, καὶ μαθήσεσθε αὐτὰ, καὶ φυλάξεσθε ποιεῖν αὐτά.
\vs{2}Κύριος ὁ Θεὸς ὑμῶν διέθετο πρὸς ὑμᾶς διαθήκην ἐν Χωρήβ.
\vs{3}Οὐχὶ τοῖς πατράσιν ὑμῶν διέθετο Κύριος τὴν διαθήκην ταύτην, ἀλλʼ ἢ πρὸς ὑμᾶς· ὑμεῖς ὧδε πάντες ζῶντες σήμερον.
\vs{4}Πρόσωπον κατὰ πρόσωπον ἐλάλησε Κύριος πρὸς ὑμᾶς ἐν τῷ ὄρει ἐκ μέσου τοῦ πυρός.
\vs{5}Κᾀγὼ εἱστήκειν ἀναμέσον Κυρίου καὶ ὑμῶν ἐν τῷ καιρῷ ἐκείνῳ ἀναγγεῖλαι ὑμῖν τὰ ῥήματα Κυρίου, ὅτι ἐφοβήθητε ἀπὸ προσώπου τοῦ πυρὸς, καὶ οὐκ ἀνέβητε εἰς τὸ ὄρος, λέγων
\vs{6}ἐγώ εἰμι Κύριος ὁ Θεός σου ὁ ἐξαγαγών σε ἐκ γῆς Αἰγύπτοῦ, ἐξ οἴκου δουλείας.

\vs{7}Οὐκ ἔσονταί σοι θεοὶ ἕτεροι πρὸ προσώπου μου.

\vs{8}Οὐ ποιήσεις σεαυτῷ εἴδωλον, οὐδὲ παντὸς ὁμοίωμα, ὅσα ἐν τῷ οὐρανῷ ἄνω, καὶ ὅσα ἐν τῇ γῇ κάτω, καὶ ὅσα ἐν τοῖς ὕδασιν ὑποκάτω τῆς γῆς.
\vs{9}Οὐ πρόσκυνήσεις αὐτοῖς, οὐδὲ μὴ λατρεύσῃς αὐτοῖς· ὅτι ἐγώ εἰμι Κύριος ὁ Θεός σου, Θεὸς ζηλωτὴς, ἀποδιδοὺς ἁμαρτίας πατέρων ἐπὶ τέκνα ἐπὶ τρίτην καὶ τετάρτην γενεὰν τοῖς μισοῦσί με,
\vs{10}καὶ ποιῶν ἔλεος εἰς χιλιάδας τοῖς ἀγαπῶσί με, καὶ τοῖς φυλάσσουσι τὰ προστάγματά μου.
\vs{11}Οὐ λήψῃ τὸ ὄνομα Κυρίου τοῦ Θεοῦ σου ἐπὶ ματαίῳ· οὐ γὰρ μὴ καθαρίσῃ Κύριος ὁ Θεός σου τὸν λαμβάνοντα τὸ ὄνομα αὐτοῦ ἐπὶ ματαίῳ.

\vs{12}Φύλαξαι τὴν ἡμέραν τῶν σαββάτων ἁγιάζειν αὐτὴν, ὃν τρόπον ἐνετείλατό σοι Κύριος ὁ Θεός σου.
\vs{13}Ἓξ ἡμέρας ἐργᾷ καὶ ποιήσεις πάντα τὰ ἔργα σου·
\vs{14}τῇ δὲ ἡμέρᾳ τῇ ἑβδόμῃ σάββατα Κυρίῳ τῷ Θεῷ σου· οὐ ποιήσεις ἐν αὐτῇ πᾶν ἔργον σὺ καὶ ὁ υἱός σου καὶ ἡ θυγάτηρ σου, ὁ παῖς σου καὶ ἡ παιδίσκη σου, ὁ βοῦς σου καὶ τὸ ὑποζύγιόνσου, καὶ πᾶν κτῆνός σου, καὶ προσήλυτος ὁ παροικῶν ἐν σοὶ· ἵνα ἀναπαύσηται ὁ παῖς σου, καὶ ἡ παιδίσκη σου, καὶ τὸ ὑποζύγίον σου, ὥσπερ καὶ σύ.
\vs{15}Καὶ μνησθήσῃ ὅτι οἰκέτης ἦσθα ἐν γῇ Αἰγύπτῳ, καὶ ἐξήγαγέ σε Κύριος ὁ Θεός σου ἐκεῖθεν ἐν χειρὶ κραταιᾷ, καὶ ἐν βραχίονι ὑψηλῷ· διὰ τοῦτο συνέταξέ σοι Κύριος ὁ Θεός σου ὥστε φυλάσσεσθαι τὴν ἡμέραν τῶν σαββάτων καὶ ἁγιάζειν αὐτήν.
\vs{16}Τίμα τὸν πατέρα σου καὶ τὴν μητέρα σου, ὃν τρόπον ἐνετείλατό σοι Κύριος ὁ Θεός σου, ἵνα εὖ σοι γένηται, καὶ ἵνα μακροχρόνιος γένῃ ἐπὶ τῆς γῆς, ἧς Κύριος ὁ Θεός σου δίδωσί σοι.
\vs{17}Οὐ φονεύσεις.
\vs{18}Οὐ μοιχεύσεις.
\vs{19}Οὐ κλέψεις.
\vs{20}Οὐ ψευδομαρτυρήσεις κατὰ τοῦ πλησίον σου μαρτυρίαν ψευδῆ.
\vs{21}Οὐκ ἐπιθυμήσεις τὴν γυναῖκα τοῦ πλησίον σου· οὐκ ἐπιθυμήσεις τὴν οἰκίαν τοῦ πλησίον σου, οὔτε τὸν ἀγρὸν αὐτοῦ, οὔτε τὸν παῖδα αὐτοῦ, οὔτε τὴν παιδίσκην αὐτοῦ, οὔτε τοῦ βοὸς αὐτοῦ, οὔτε τοῦ ὑποζυγίου αὐτοῦ, οὔτε παντὸς κτήνους αὐτοῦ, οὔτε πάντα ὅσα τῷ πλησίον σου ἐστί.

\vs{22}Ταῦτα τὰ ῥήματα ἐλάλησε Κύριος πρὸς πᾶσαν συναγωγὴν ὑμῶν ἐν τῷ ὄρει ἐκ μέσου τοῦ πυρός· σκότος, γνόφος, θύελλα, φωνὴ μεγάλη· καὶ οὐ προσέθηκε· καὶ ἔγραψεν αὐτὰ ἐπὶ δύο πλάκας λιθίνας, καὶ ἔδωκέ μοι.
\vs{23}Καὶ ἐγένετο ὡς ἠκούσατε τὴν φωνὴν ἐκ μέσου τοῦ πυρός, καὶ τὸ ὄρος ἐκαίετο πυρί, καὶ προσήλθετε πρὸς μὲ πάντες οἱ ἡγούμενοι τῶν φυλῶν ὑμῶν, καὶ ἡ γερουσία ὑμῶν,
\vs{24}καὶ ἐλέγετε, ἰδοὺ ἔδειξεν ἡμῖν Κύριος ὁ Θεὸς ἡμῶν τὴν δόξαν αὐτοῦ, καὶ τὴν φωνὴν αὐτοῦ ἠκούσαμεν ἐκ μέσου τοῦ πυρός· ἐν τῇ ἡμέρᾳ ταύτῃ εἴδομεν ὅτι λαλήσει ὁ Θεὸς πρὸς ἄνθρωπον, καὶ ζήσεται.
\vs{25}Καὶ νῦν μὴ ἀποθάνωμεν, ὅτι ἐξαναλώσει ἡμᾶς τὸ πῦρ τὸ μέγα τοῦτο, ἐὰν προσθώμεθα ἡμεῖς ἀκοῦσαι τὴν φωνὴν Κυρίου τοῦ Θεοῦ ἡμῶν ἔτι, καὶ ἀποθανούμεθα.
\vs{26}Τίς γὰρ σὰρξ ἥτις ἤκουσε φωνὴν Θεοῦ ζῶντος, λαλοῦντος ἐκ μέσου τοῦ πυρὸς, ὡς ἡμεῖς, καὶ ζήσεται;
\vs{27}Πρόσελθε σὺ, καὶ ἄκουσον πάντα ὅσα ἂν εἴπῃ Κύριος ὁ Θεὸς ἡμῶν, καὶ σὺ λαλήσεις πρὸς ἡμᾶς πάντα ὅσα ἂν λαλήσει Κύριος ὁ Θεὸς ἡμῶν πρὸς σὲ, καὶ ἀκουσόμεθα, καὶ ποιήσομεν.

\vs{28}Καὶ ἤκουσε Κύριος τὴν φωνὴν τῶν λόγων ὑμῶν λαλούντων πρὸς μέ· καὶ εἶπε Κύριος πρὸς μέ, ἤκουσα τὴν φωνὴν τῶν λόγων τοῦ λαοῦ τούτου ὅσα ἐλάλησαν πρὸς σέ· ὀρθῶς πάντα ὅσα ἐλάλησαν.
\vs{29}Τίς δώσει εἶναι οὕτω τὴν καρδίαν αὐτῶν ἐν αὐτοῖς, ὥστε φοβεῖσθαί με καὶ φυλάσσεσθαι τὰς ἐντολάς μου πάσας τὰς ἡμέρας, ἵνα εὖ ἠ· αὐτοῖς, καὶ τοῖς υἱοῖς αὐτῶν διʼ αἰῶνος;
\vs{30}Βάδισον, εἶπον αὐτοῖς, ἀποστράφητε ὑμεῖς εἰς τοὺς οἴκους ὑμῶν·
\vs{31}σὺ δὲ αὐτοῦ στῆθι μετʼ ἐμοῦ, καὶ λαλήσω πρὸς σὲ τὰς ἐντολὰς καὶ τὰ δικαιώματα καὶ τὰ κρίματα ὅσα διδάξεις αὐτοὺς, καὶ ποιείτωσαν οὕτως ἐν τῇ γῇ ἣν ἐγὼ δίδωμι αὐτοῖς ἐν κλήρῳ.
\vs{32}Καὶ φυλάξεσθε ποιεῖν ὃν τρόπον ἐνετειλατό σοι Κύριος ὁ Θεός σου· οὐκ ἐκκλινεῖτε εἰς δεξιὰ οὐδὲ εἰς ἀριστερά,
\vs{33}κατὰ πᾶσαν τὴν ὁδὸν, ἣν ἐνετείλατό σοι Κύριος ὁ Θεός σου πορεύεσθαι ἐν αὐτῇ, ὅπως καταπαύσῃ σε, καὶ εὖ σοι ἠ·, καὶ μακροημερεύσητε ἐπὶ τῆς γῆς ἣν κληρονομήσετε.

\ch{6}
Καὶ αὗται αἱ ἐντολαὶ καὶ τὰ δικαιώματα καὶ τὰ κρίματα ὅσα ἐνετείλατο Κύριος ὁ Θεὸς ἡμῶν διδάξαι ὑμᾶς ποιεῖν οὕτως ἐν τῇ γῇ, εἰς ἣν ὑμεῖς εἰσπορεύεσθε ἐκεῖ κληρονομῆσαι αὐτήν.
\vs{2}Ἵνα φοβῆσθε Κύριον τὸν Θεὸν ὑμῶν, φυλάσσεσθε πάντα τὰ δικαιώματα αὐτοῦ, καὶ τὰς ἐντολὰς αὐτοῦ, ἃς ἐγὼ ἐντέλλομαί σοι σήμερον, σὺ καὶ οἱ υἱοί σου, καὶ οἱ υἱοὶ τῶν υἱῶν σου πάσας τὰς ἡμέρας τῆς ζωῆς σου, ἵνα μακροημερεύσητε.

\vs{3}Καὶ ἄκουσον Ἰσραὴλ, καὶ φύλαξον ποιεῖν, ὅπως εὖ σοι ἠ·, καὶ ἵνα πληθυνθῆτε σφόδρα, καθάπερ ἐλάλησε Κύριος ὁ Θεὸς τῶν πατέρων σου δοῦναί σοι γῆν ῥέουσαν γάλα καὶ μέλι·
\vs{4}καὶ ταῦτα τὰ δικαιώματα καὶ τὰ κρίματα, ὅσα ἐνετείλατο Κύριος τοῖς υἱοῖς Ἰσραὴλ ἐν τῇ ἐρήμῳ, ἐξελθόντων αὐτῶν ἐκ γῆν Αἰγύπτου. Ἄκουε Ἰσραὴλ, Κύριος ὁ Θεὸς ἡμῶν, Κύριος εἷς ἐστι.
\vs{5}Καὶ ἀγαπήσεις Κύριον τὸν Θεόν σου ἐξ ὅλης τῆς διανοίας σου, καὶ ἐξ ὅλης τῆς ψυχῆς σου, καὶ ἐξ ὅλης τῆς δυνάμεώς σου.
\vs{6}Καὶ ἔσται τὰ ῥήματα ταῦτα, ὅσα ἐγὼ ἐντέλλομαί σοι σήμερον, ἐν τῇ καρδίᾳ σου, καὶ ἐν τῇ ψυχῇ σου.
\vs{7}Καὶ προβιβάσεις αὐτὰ τοὺς υἱούς σου, καὶ λαλήσεις ἐν αὐτοῖς καθήμενος ἐν οἴκῳ, καὶ πορευόμενος ἐν ὁδῷ, καὶ κοιταζόμενος, καὶ διανιστάμενος.
\vs{8}Καὶ ἀφάψεις αὐτὰ εἰς σημεῖον ἐπὶ τῆς χειρός σου, καὶ ἔσται ἀσάλευτον πρὸ ὀφθαλμῶν σου.
\vs{9}Καὶ γράψετε αὐτὰ ἐπὶ τὰς φλιὰς τῶν οἰκιῶν ὑμῶν, καὶ τῶν πυλῶν ὑμῶν.

\vs{10}Καὶ ἔσται ὅταν εἰσαγάγῃ σε Κύριος ὁ Θεός σου εἰς τὴν γῆν ἣν ὤμοσε τοῖς πατράσι σου, τῷ Ἁβραὰμ, καὶ τῷ Ἰσαακ, καὶ τῷ Ἰακώβ, δοῦναί σοι πόλεις μεγάλας καὶ καλὰς ἃς οὐκ ᾠκοδόμησας,
\vs{11}οἰκίας πλήρεις πάντων ἀγαθῶν ἃς οὐκ ἐνέπλησας, λάκκους λελατομημένους οὓς οὐκ ἐξελατόμησας, ἀμπελῶνας καὶ ἐλαιῶνας οὓς οὐ κατεφύτευσας, καὶ φαγὼν καὶ ἐμπλησθεὶς, πρόσεχε σεαυτῷ μὴ ἐπιλάθῃ
\vs{12}Κυρίου τοῦ Θεοῦ σου τοῦ ἐξαγαγόντος σε ἐκ γῆς Αἰγύπτου, ἐξ οἴκου δουλείας.
\vs{13}Κύριον τὸν Θεόν σου φοβήθήσῃ, καὶ αὐτῷ μόνῳ λατρεύσεις, καὶ πρὸς αὐτὸν κολληθήσῃ, καὶ ἐπὶ τῷ ὀνόματι αὐτοῦ ὀμῇ.

\vs{14}Οὐ πορεύεσθε ὀπίσω θεῶν ἑτέρων ἀπὸ τῶν θεῶν τῶν ἐθνῶν τῶν περικύκλῳ ὑμῶν,
\vs{15}ὅτι ὁ Θεὸς ζηλωτὴς Κύριος ὁ Θεός σου ἐν σοί· μὴ ὀργισθεὶς θυμῷ Κύριος ὁ Θεός σου σοὶ, ἐξολοθρεύσῃ σε ἀπὸ προσώπου τῆς γῆς.

\vs{16}Οὐκ ἐκπειράσεις Κύριον τὸν Θεόν σου, ὃν τρόπον ἐξεπειράσατε ἐν τῷ πειρασμῷ.
\vs{17}Φυλάσσων φυλάξῃ τὰς ἐντολὰς Κυρίου τοῦ Θεοῦ σου, τὰ μαρτύρια, καὶ τὰ δικαιώματα, ὅσα ἐνετείλατό σοι.
\vs{18}Καὶ ποιήσεις τὸ ἀρεστὸν καὶ τὸ καλὸν ἔναντι Κυρίου τοῦ Θεοῦ σου, ἵνα εὖ σοι γένηται, καὶ εἰσέλθῃς καὶ κληρονομήσῃς τὴν γῆν τὴν ἀγαθὴν, ἣν ὤμοσε Κύριος τοῖς πατράσιν ὑμῶν,
\vs{19}ἐκδιῶξαι πάντας τοὺς ἐχθρούς σου πρὸ προσώπου σου, καθὰ ἐλάλησε Κύριος.

\vs{20}Καὶ ἔσται ὅταν ἐρωτήσῃ σε ὁ υἱός σου αὔριον, λέγων, τί ἐστι τὰ μαρτύρια, καὶ τὰ δικαιώματα καὶ τὰ κρίματα, ὅσα ἐνετείλατο Κύριος ὁ Θεὸς ἡμῶν ἡμῖν;
\vs{21}Καὶ ἐρεῖς τῷ υἱῷ σου, οἰκέται ἦμεν τῷ Φαραὼ ἐν γῇ Αἰγύπτῳ, καὶ ἐξήγαγεν ἡμᾶς Κύριος ἐκεῖθεν ἐν χειρὶ κραταιᾷ, καὶ ἐν βραχίονι ὑψηλῷ.
\vs{22}Καὶ ἔδωκε Κύριος σημεῖα καὶ τέρατα μεγάλα καὶ πονηρὰ ἐν Αἰγύπτῳ ἐν Φαραὼ καὶ ἐν τῷ οἴκῳ αὐτοῦ ἐνώπιον ἡμῶν,
\vs{23}καὶ ἡμᾶς ἐξήγαγεν ἐκεῖθεν δαῦναι ἡμῖν τὴν γῆν ταύτην, ἣν ὤμοσε δοῦναι τοῖς πατράσιν ἡμῶν.
\vs{24}Καὶ ἐνετείλατο ἡμῖν Κύριος ποιεῖν πάντα τὰ δικαιώματα ταῦτα· φοβεῖσθαι Κύριον τὸν Θεὸν ἡμῶν, ἵνα εὖ ἠ· ἡμῖν πάσας τὰς ἡμέρας, ἵνα ζῶμεν ὥσπερ καὶ σήμερον.
\vs{25}Καὶ ἐλεημοσύνη ἔσται ἡμῖν, ἐὰν φυλασσώμεθα ποιεῖν πάσας τὰς ἐντολὰς ταύτας ἐναντίον Κυρίου τοῦ Θεοῦ ἡμῶν, καθὰ ἐνετείλατο ἡμῖν.

\ch{7}
Ἐὰν δὲ εἰσάγῃ σε Κύριος ὁ Θεός σου εἰς τὴν γῆν, εἰς ἣν εἰσπορεύῃ ἐκεῖ κληρονομῆσαι αὐτὴν, καὶ ἐξάρῃ ἔθνη μεγάλα ἀπὸ προσώπου σου, τὸν Χετταῖον καὶ Γεργεσαῖον καὶ Ἀμοῤῥαῖον καὶ Χαναναῖον καὶ Φερεζαῖον καὶ Εὐαῖον καὶ Ἰεβουσαῖον, ἐπτὰ ἔθην πολλὰ καὶ ἰσχυρότερα ὑμῶν·
\vs{2}Καὶ παραδώσει αὐτοὺς Κύριος ὁ Θεός σου εἰς τὰς χεῖράς σου, καὶ πατάξεις αὐτούς· ἀφανισμῷ ἀφανιεῖς αὐτούς· οὐ διαθήσῃ πρὸς αὐτοὺς διαθήκην, οὐδὲ μὴ ἐλεήσητε αὐτούς,
\vs{3}οὐδὲ μὴ γαμβρεύσητε πρὸς αὐτούς· τὴν θυγατέρα σου οὐ δώσεις τῷ υἱῷ αὐτοῦ, καὶ τὴν θυγατέρα αὐτοῦ οὐ λήψῃ τῷ υἱῷ σου.
\vs{4}Ἀποστήσει γὰρ τὸν υἱόν σου ἀπʼ ἐμοῦ, καὶ λατρεύσει θεοῖς ἑτέροις· καὶ ὀργισθήσεται θυμῷ Κύριος εἰς ὑμᾶς, καὶ ἐξολοθρεύσει σε τοτάχος.
\vs{5}Ἀλλʼ οὕτω ποιήσετε αὐτοῖς· τοὺς βωμοὺς αὐτῶν καθελεῖτε, καὶ τὰς στήλας αὐτῶν συντρίψετε, καὶ τὰ ἄλση αὐτῶν ἐκκόψετε, καὶ τὰ γλυπτὰ τῶν θεῶν αὐτῶν κατακαύσετε πυρί.
\vs{6}Ὅτι λαὸς ἅγιος εἶ Κυρίῳ τῷ Θεῷ σου· καὶ σὲ προείλετο Κύριος ὁ Θεός σου εἶναι αὐτῷ λαὸν περιούσιον παρὰ πάντα τὰ ἔθνη, ὅσα ἐπὶ προσώπου τῆς γῆς.

\vs{7}Οὐχ ὅτι πολυπληθεῖτε παρὰ πάντα τὰ ἔθνη, προείλετο Κύριος ὑμᾶς, καὶ ἐξελέξατο Κύριος ὑμᾶς· ὑμεῖς γάρ ἐστε ὀλιγοστοὶ παρὰ πάντα τὰ ἔθνη.
\vs{8}Ἀλλὰ παρὰ τὸ ἀγαπᾷν Κύριον ὑμᾶς, καὶ διατηρῶν τὸν ὅρκον ὃν ὤμοσε τοῖς πατράσιν ὑμῶν, ἐξήγαγεν ὑμᾶς Κύριος ἐν χειρὶ κραταιᾷ, καὶ ἐλυτρώσατό σε Κύριος ἐξ οἴκου δουελίας, ἐκ χειρὸς Φαραὼ βασιλέως Αἰγύπτου.
\vs{9}Καὶ γνώσῃ, ὅτι Κύριος ὁ Θεός σου, οὗτος Θεός· Θεὸς πιστός, ὁ φυλάσσων διαθήκην καὶ ἔλεος τοῖς ἀγαπῶσιν αὐτὸν καὶ τοῖς φυλάσσουσι τὰς ἐντολὰς αὐτοῦ εἰς χιλίας γενεάς,
\vs{10}καὶ ἀποδιδοὺς τοῖς μισοῦσι κατὰ πρόσωπον ἐξολοθρεῦσαι αὐτούς· καὶ οὐχὶ βραδυνεῖ τοίς μισοῦσι· κατὰ πρόσωπον ἀποδώσει αὐτοῖς.

\vs{11}Καὶ φυλάξῃ τὰς ἐντολὰς, καὶ τὰ δικαιώματα, καὶ τὰ κρίματα ταῦτα, ὅσα ἐγὼ ἐντέλλομαί σοι σήμερον ποιεῖν.
\vs{12}Καὶ ἔσται ἡνίκα ἂν ἀκούσητε τὰ δικαιώματα ταῦτα, καὶ φυλάξητε καὶ ποιήσητε αὐτὰ, καὶ διαφυλάξει Κύριος ὁ Θεός σου σοὶ τὴν διαθήκην καὶ τὸ ἔλεος, ὃ ὤμοσε τοῖς πατράσιν ὑμῶν.
\vs{13}Καὶ ἀγαπήσει σε, καὶ εὐλογήσει σε, καὶ πληθυνεῖ σε, καὶ εὐλογήσει τὰ ἔγγονα τῆς κοιλίας σου, καὶ τὸν καρπὸν τῆς γῆς σου, τὸν σῖτόν σου, καὶ τὸν οἶνόν σου, καὶ τὸ ἔλαιόν σου, τὰ βουκόλια τῶν βοῶν σου, καὶ τὰ ποίμνια τῶν προβάτων σου ἐπὶ τῆς γῆς, ἧς ὤμοσε Κύριος τοῖς πατράσι σου δοῦναί σοι.
\vs{14}Εὐλογητὸς ἔσῃ παρὰ πάντα τὰ ἔθνη· οὐκ ἔσται ἐν ὑμῖν ἄγονος, οὐδὲ στεῖρα, καὶ ἐν τοῖς κτήνεσί σου.
\vs{15}Καὶ περιελεῖ Κύριος ὁ Θεός σου ἀπὸ σοῦ πᾶσαν μαλακίαν, καὶ πάσας νόσους Αἰγύπτου τὰς πονηρὰς, ἃς ἑώρακας, καὶ ὅσα ἔγνως, οὐκ ἐπιθήσει ἐπὶ σὲ· καὶ ἐπιθήσει αὐτὰ ἐπὶ πάντας τοὺς μισοῦντὰς σε.

\vs{16}Καὶ φαγῇ πάντα τὰ σκῦλα τῶν ἐθνῶν, ἃ Κύριος ὁ Θεός σου δίδωσί σοι· οὐ φείσεται ὁ ὀφθαλμός σου ἐπʼ αὐτοῖς, καὶ οὐ μὴ λατρεύσῃς τοῖς θεοῖς αὐτῶν· ὅτι σκῶλον τοῦτό ἐστί σοί.

\vs{17}Ἐὰν δὲ λέγῃς ἐν τῇ διανοίᾳ σου, ὅτι πολὺ τὸ ἔθνος τοῦτο ἢ ἐγώ, πῶς δυνήσομαι ἐξολοθρεῦσαι αὐτούς;
\vs{18}Οὐ φοβηθήσῃ αὐτούς· μνείᾳ μνησθήσῃ, ὅσα ἐποίησε Κύριος ὁ Θεός σου τῷ Φαραὼ καὶ πᾶσι τοῖς Αἰγυπτίοις·
\vs{19}Τοὺς πειρασμοὺς τοὺς μεγάλους, οὓς ἴδοσαν οἱ ὀφθαλμοί σου, τὰ σημεῖα καὶ τὰ τέρατα τὰ μεγάλα ἐκεῖνα, τὴν χεῖρα τὴν κραταιὰν, καὶ τὸν βραχίονα τὸν ὑψηλόν· ὡς ἐξήγαγέ σε Κύριος ὁ Θεός σου, οὕτως ποιήσει Κύριος ὁ Θεὸς ὑμῶν πᾶσιν τοῖς ἔθνεσιν, οὓς σὺ φοβῇ ἀπὸ προσώπου αὐτῶν.
\vs{20}Καὶ τὰς σφηκίας ἀποστελεῖ Κύριος ὁ Θεός σου εἰς αὐτοὺς, ἕως ἂν ἐκτριβῶσιν οἱ καταλελειμμένοι καὶ οἱ κεκρυμμένοι ἀπὸ σοῦ·
\vs{21}Οὐ τρωθήσῃ ἀπὸ προσώπου αὐτῶν, ὅτι Κύριος ὁ Θεός σου ἐν σοί, Θεὸς μέγας καὶ κραταιός.
\vs{22}Καὶ καταναλώσει Κύριος ὁ Θεός σου τὰ ἔθνη ταῦτα ἀπὸ προσώπου σου κατὰ μικρὸν μικρόν· οὐ δυνήσῃ ἐξαναλῶσαι αὐτοὺς τοτάχος, ἵνα μὴ γένηται ἡ γῆ ἔρημος, καὶ πληθυνθῇ ἐπὶ σὲ τὰ θηρία τὰ ἄγρια.
\vs{23}Καὶ παραδώσει αὐτοὺς Κύριος ὁ Θεός σου εἰς τὰς χεῖράς σου, καὶ ἀπολεῖς αὐτοὺς ἀπωλείᾳ μεγάλῃ, ἕως ἂν ἐξολοθρεύσητε αὐτούς·
\vs{24}Καὶ παραδώσει τοὺς βασιλεῖς αὐτῶν εἰς τὰς χεῖρας ὑμῶν, καὶ ἀπολεῖτε τὸ ὄνομα αὐτῶν ἐκ τοῦ τόπου ἐκείνου· οὐκ ἀντιστήσεται οὐθεὶς κατὰ πρόσωπόν σου, ἕως ἂν ἐξολοθρεύσῃς αὐτούς.

\vs{25}Τὰ γλυπτὰ τῶν Θεῶν αὐτῶν καύσετε πυρί· οὐκ ἐπιθυμήσεις ἀργύριον, οὐδὲ χρυσίον ἀπʼ αὐτῶν οὐ λήψῃ σεαυτῷ, μὴ πταίσῃς διʼ αὐτὸ, ὅτι βδέλυγμα Κυρίῳ τῷ Θεῷ σού ἐστί.
\vs{26}Καὶ οὐκ εἰσοίσεις βδέλυγμα εἰς τὸν οἶκόν σου, καὶ ἀνάθεμα ἔσῃ ὥσπερ τοῦτο· προσοχθίσματι προσοχθιεῖς, καὶ βδελύγματι βδελύξῃ, ὅτι ἀνάθεμά ἐστι.

\ch{8}
Πάσας τὰς ἐντολὰς, ἃς ἐγὼ ἐντέλλομαι ὑμῖν σήμερον, φυλάξεσθε ποιεῖν, ἵνα ζῆτε καὶ πολυπλασιασθῆτε, καὶ εἰσέλθητε καὶ κληρονομήσητε τὴν γῆν, ἣν ὤμοσε Κύριος ὁ Θεὸς ὑμῶν τοῖς πατράσιν ὑμῶν.
\vs{2}Καὶ μνησθήσῃ πᾶσαν τὴν ὁδὸν, ἣν ἤγαγέ σε Κύριος ὁ Θεός σου ἐν τῇ ἐρήμῳ, ὅπως ἂν κακώσῃ σε καὶ πειράσῃ σε, καὶ διαγνωσθῇ τὰ ἐν τῇ καρδίᾳ σου, εἰ φυλάξῃ τὰς ἐντολὰς αὐτοῦ ἢ οὔ.
\vs{3}Καὶ ἐκάκωσέ σε, καὶ ἐλιμαγχόνησέ σε, καὶ ἐψώμισέ σε τὸ μάννα, ὃ οὐκ ᾔδεισαν οἱ πατέρες σου· ἵνα ἀναγγείλῃ σοι, ὅτι οὐκ ἐπʼ ἄρτῳ μόνῳ ζήσεται ὁ ἄνθρωπος, ἀλλʼ ἐπὶ παντὶ ῥήματι τῷ ἐκπορευομένῳ διὰ στόματος Θεοῦ ζήσεται ὁ ἄνθρωπος.
\vs{4}Τὰ ἱμάτιά σου οὐκ ἐπαλαιώθη ἀπὸ σοῦ, τὰ ὑποδήματά σου οὐ κατετρίβη ἀπὸ σοῦ· οἱ πόδες σου οὐκ ἐτυλώθησαν, ἰδοὺ τεσσαράκοντα ἔτη.

\vs{5}Καὶ γνώσῃ τῇ καρδίᾳ σου, ὅτι ὡς εἴτις ἄνθρωπος παιδεύσῃ τὸν υἱὸν αὐτοῦ, οὕτως Κύριος ὁ Θεός σου παιδεύσει σε.
\vs{6}Καὶ φυλάξῃ τὰς ἐντολὰς Κυρίου τοῦ Θεοῦ σου πορεύεσθαι ἐν ταῖς ὁδοῖς αὐτοῦ, καὶ φοβεῖσθαι αὐτόν.

\vs{7}Ὁ γὰρ Κύριος ὁ Θεός σου εἰσάξει σε εἰς γῆν ἀγαθὴν καὶ πολλὴν, οὗ χείμαῤῥοι ὑδάτων, καὶ πηγαὶ ἀβύσσων ἐκπορευόμεναι διὰ τῶν πεδίων καὶ διὰ τῶν ὀρέων·
\vs{8}Γῆ πυροῦ καὶ κριθῆς, ἄμπελοι, συκαῖ, ῥοαί· γῆ ἐλαίας ἐλαίου καὶ μέλιτος·
\vs{9}γῆ ἐφʼ ἧς οὐ μετὰ πτωχείας φαγῇ τὸν ἄρτον σου, καὶ οὐκ ἐνδεηθήσῃ ἐπʼ αὐτῆς οὐδέν· γῆ ἧς οἱ λίθοι σίδηρος, καὶ ἐκ τῶν ὀρέων αὐτῆς μεταλλεύσεις χαλκόν.

\vs{10}Καὶ φαγῇ καὶ ἐμπλησθήσῃ, καὶ εὐλογήσεις Κύριον τὸν Θεόν σου ἐπὶ τῆς γῆς τῆς ἀγαθῆς, ἧς δέδωκέ σοι.
\vs{11}Πρόσεχε σεαυτῷ μὴ ἐπιλάθῃ Κυρίου τοῦ Θεοῦ σου, τοῦ μὴ φυλάξαι τὰς ἐντολὰς αὐτοῦ, καὶ τὰ κρίματα καὶ τὰ δικαιώματα αὐτοῦ, ὅσα ἐγὼ ἐντέλλομαί σοι σήμερον·
\vs{12}Μὴ φαγὼν καὶ ἐμπλησθεὶς, καὶ οἰκίας καλὰς οἰκοδομήσας καὶ κατοικήσας ἐν αὐταῖς,
\vs{13}καὶ τῶν βοῶν σου καὶ τῶν προβάτων σου πληθυνθέντων σοι, ἀργυρίου καὶ χρυσίου πληθυνθέντος σοι, καὶ πάντων ὅσων σοι ἔσται πληθυνθέντων σοι,
\vs{14}ὑψωθῇς τῇ καρδίᾳ, καὶ ἐπιλάθῃ Κυρίου τοῦ Θεοῦ σου, τοῦ ἐξαγαγόντος σε ἐκ γῆς Αἰγύπτου, ἐξ οἴκου δουλείας·
\vs{15}τοῦ ἀγαγόντος σε διὰ τῆς ἐρήμου τῆς μεγάλης καὶ τῆς φοβερᾶς ἐκείνης, οὗ ὄφις δάκνων, καὶ σκορπίος, καὶ δίψα, οὗ οὐκ ἦν ὕδωρ· τοῦ ἐξαγαγόντος σοι ἐκ πέτρας ἀκροτόμου πηγὴν ὕδατος·
\vs{16}τοῦ ψωμίσαντός σε τὸ μάννα ἐν τῇ ἐρήμῳ ὃ οὐκ ᾔδεις σὺ, καὶ οὐκ ᾔδεισαν οἱ πατέρες σου, ἵνα κακώσῃ σε, καὶ ἐκπειράσῃ σε, καὶ εὖ σε ποιήσῃ ἐπʼ ἐσχάτων τῶν ἡμερῶν σου.
\vs{17}Μὴ εἴπῃς ἐν τῇ καρδίᾳ σου, ἡ ἰσχύς μου, καὶ τὸ κράτος τῆς χειρός μου ἐποίησέ μοι τὴν δύναμιν τὴν μεγάλην ταύτην.
\vs{18}Καὶ μνησθήσῃ Κυρίου τοῦ Θεοῦ σου, ὅτι αὐτός σοι δίδωσιν ἰσχὺν τοῦ ποιῆσαι δύναμιν, καὶ ἵνα στήσῃ τὴν διαθήκην αὐτοῦ ἣν ὤμοσε Κύριος τοῖς πατράσι σου, ὡς σήμερον.

\vs{19}Καὶ ἔσται ἐὰν λήθῃ ἐπιλάθῃ Κυρίου τοῦ Θεοῦ σου, καὶ πορευθῇς, ὀπίσω θεῶν ἑτέρων, καὶ λατρεύσῃς αὐτοῖς, καὶ προσκυνήσῃς αὐτοῖς, διαμαρτύρομαι ὑμῖν σήμερον τόν τε οὐρανὸν καὶ τὴν γῆν, ὅτι ἀπωλείᾳ ἀπολεῖσθε.
\vs{20}Καθὰ καὶ τὰ λοιπὰ ἔθνη ὅσα Κύριος ὁ Θεὸς ἀπολλύει πρὸ προσώπου ὑμῶν, οὕτως ἀπολεῖσθε, ἀνθʼ ὧν οὐκ ἠκούσατε τῆς φωνῆς Κυρίου τοῦ Θεοῦ ὑμῶν.

\ch{9}
Ἄκουε Ἰσραήλ· σὺ διαβαίνεις σήμερον τὸν Ἰορδάνην εἰσελθεῖν κληρονομῆσαι ἔθνη μεγάλα καὶ ἰσχυρότερα μᾶλλον ἢ ὑμεῖς, πόλεις μεγάλας καὶ τειχήρεις ἕως τοῦ οὐρανοῦ,
\vs{2}λαὸν μέγαν καὶ πολὺν καὶ εὐμήκη, υἱοὺς Ἐνάκ, οὓς σὺ οἶσθα, καὶ σὺ ἀκήκοας, τίς ἀντιστήσεται κατὰ πρόσωπον υἱῶν Ἐνάκ;
\vs{3}Καὶ γνώσῃ σήμερον, ὅτι Κύριος ὁ Θεός σου οὗτος προπορεύσεται πρὸ προσώπου σου· πῦρ καταναλίσκον ἐστίν· οὗτος ἐξολοθρεύσει αὐτούς, καὶ οὗτος ἀποστρέψει αὐτοὺς ἀπὸ προσώπου σου, καὶ ἀπολεῖ αὐτοὺς ἐν τάχει, καθάπερ εἶπέ σοι Κύριος.
\vs{4}Μὴ εἴπῃς ἐν τῇ καρδίᾳ σου ἐν τῷ ἐξαναλῶσαι Κύριον τὸν Θεόν σου τὰ ἔθνη ταῦτα πρὸ προσώπου σου, λέγων, διὰ τὴν δικαιοσύνην μου εἰσήγαγέ με Κύριος κληρονομῆσαι τὴν γὴν τὴν ἀγαθὴν ταύτην.
\vs{5}Οὐχὶ διὰ τὴν δικαιοσύνην σου, οὐδὲ διὰ τὴν ὁσιότητα τῆς καρδίας σου σὺ εἰσπορεύῃ κληρονομῆσαι τὴν γῆν αὐτῶν, ἀλλὰ διὰ τὴν ἀσέβειαν τῶν ἐθνῶν τούτων Κύριος ἐξολοθρεύσει αὐτοὺς ἀπὸ προσώπου σου, καὶ ἵνα στήσῃ τὴν διαθήκην, ἥ ὤμοσε Κύριος τοῖς πατράσιν ἡμῶν τῷ Ἁβραὰμ καὶ τῷ Ἰσαὰκ καὶ τῷ Ἰακώβ.

\vs{6}Καὶ γνώσῃ σήμερον, ὅτι οὐχὶ διὰ τὰς δικαιοσύνας σου Κύριος ὁ Θεός σου δίδωσί σοι τὴν γῆν τὴν ἀγαθὴν ταύτην κληρονομῆσαι, ὅτι λαὸς σκληροτράχηλος εἶ.
\vs{7}Μνήσθητι, μὴ ἐπιλάθῃ ὅσα παρώξυνας Κύριον τὸν Θεόν σου ἐν τῇ ἐρήμῳ· ἀφʼ ἧς ἡμέρας ἐξήλθετε ἐξ Αἰγύπτου, καὶ ἤλθετε εἰς τὸν τόπον τοῦτον, ἀπειθοῦντες διετελεῖτε τὰ πρὸς Κύριον.

\vs{8}Καὶ ἐν Χωρὴβ παρωξύνατε Κύριον, καὶ ἐθυμώθη Κύριος ἐφʼ ὑμῖν ἐξολεθρεῦσαι ὑμᾶς,
\vs{9}ἀναβαίνοντός μου εἰς τὸ ὄρος λαβεῖν τὰς πλάκας τὰς λιθίνας, πλάκας διαθήκης ἃς διέθετο Κύριος πρὸς ὑμᾶς, καὶ κατεγενόμην ἐν τῷ ὄρει τεσσαράκοντα ἡμέρας καὶ τεσσεράκοντα νύκτας, ἄρτον οὐκ ἔφαγον καὶ ὕδωρ οὐκ ἔπιον.
\vs{10}Καὶ ἔδωκέ μοι Κύριος τὰς δύο πλάκας τὰς λιθίνας γεγραμμένας ἐν τῷ δακτύλῳ τοῦ Θεοῦ, καὶ ἐπʼ αὐταῖς ἐγέγραπτο πάντες οἱ λόγοι οὓς ἐλάλησε Κύριος πρὸς ὑμᾶς ἐν τῷ ὄρει ἡμέρᾳ ἐκκλησίας·
\vs{11}Καὶ ἐγένετο διὰ τεσσεράκοντα ἡμερῶν καὶ διὰ τεσσαράκοντα νυκτῶν ἔδωκε Κύριος ἐμοὶ τὰς δύο πλάκας τὰς λιθίνας, πλάκας διαθήκης.
\vs{12}Καὶ εἶπε Κύριος πρὸς μέ, ἀνάστηθι, κατάβηθι τοτάχος ἐντεῦθεν, ὅτι ἠνόμησεν ὁ λαός σου, οὓς ἐξήγαγες ἐκ γῆς Αἰγύπτου· παρέβησαν ταχὺ ἐκ τῆς ὁδοῦ ἧς ἐνετείλω αὐτοῖς, καὶ ἐποίησαν ἑαυτοῖς χώνευμα.

\vs{13}Καὶ εἶπε Κύριος πρὸς μὲ, λέγων, λελάληκα πρὸς σὲ ἅπαξ καὶ δὶς, λέγων, ἑώρακα τὸν λαὸν τοὗτον, καὶ ἰδοὺ λαὸς σκληροτράχηλός ἐστι·
\vs{14}Καὶ νῦν ἔασόν με ἐξολοθρεῦσαι αὐτούς, καὶ ἐξαλείψω τὸ ὄνομα αὐτῶν ὑποκάτωθεν τοῦ οὐρανοῦ, καὶ ποιήσω σε εἰς ἔθνος μέγα, καὶ ἰσχυρὸν, καὶ πολὺ μᾶλλον ἢ τοῦτο.
\vs{15}Καὶ ἐπιστρέψας, κατέβην ἐκ τοῦ ὄρους· καὶ τὸ ὄρος ἐκαίετο πυρὶ ἕως τοῦ οὐρανοῦ· καὶ αἱ δύο πλάκες τῶν μαρτυρίων ἐπὶ ταῖς δυσὶ χερσί μου.
\vs{16}Καὶ ἰδὼν ὅτι ἡμάρτετε ἐναντίον Κυρίου τοῦ Θεοῦ ὑμῶν, καὶ ἐποιήσατε ὑμῖν αὐτοῖς χωνευτόν, καὶ παρέβητε ἀπὸ τῆς ὁδοῦ, ἧς ἐνετείλατο Κύριος ὑμῖν ποιεῖν·
\vs{17}καὶ ἐπιλαβόμενος τῶν δύο πλακῶν, ἔῤῥιψα αὐτὰς ἀπὸ τῶν δύο χειρῶν μου, καὶ συνέτριψα ἐναντίον ὑμῶν.
\vs{18}Καὶ ἐδεήθην ἐναντίον Κυρίου δεύτερον καθάπερ καὶ τὸ πρότερον τεσσαράκοντα ἡμέρας καὶ τεσσαράκοντα νύκτας, ἄρτον οὐκ ἔφαγον καὶ ὕδωρ οὐκ ἔπιον, περὶ πασῶν τῶν ἁμαρτιῶν ὑμῶν ὧν ἡμάρτετε ποιῆσαι τὸ πονηρὸν ἐναντίον Κυρίου τοῦ Θεοῦ παροξῦναι αὐτόν.
\vs{19}Καὶ ἔκφοβός εἰμι διὰ τὸν θυμόν καὶ τὴν ὀργὴν, ὅτι παρωξύνθη Κύριος ἐφʼ ὑμῖν τοῦ ἐξολοθρεῦσαι ὑμᾶς· καὶ εἰσήκουσε Κύριος ἐμοῦ καὶ ἐν τῷ καιρῷ τούτῳ.
\vs{20}Καὶ ἐπὶ Ἀαρὼν ἐθυμώθη ἐξολοθρεῦσαι αὐτόν, καὶ ηὐξάμην καὶ περὶ Ἀαρὼν ἐν τῷ καιρῷ ἐκείνῳ.
\vs{21}Καὶ τὴν ἁμαρτίαν ὑμῶν, ἣν ἐποιήσατε, τὸν μόσχον ἔλαβον αὐτὸν, καὶ κατέκαυσα αὐτὸν ἐν πυρί, καὶ συνέκοψα αὐτὸν καταλέσας σφόδρα ἕως ἐγένετο λεπτόν, καὶ ἐγένετο ὡσεὶ κονιορτός· καὶ ἔῤῥιψα τὸν κονιορτὸν εἰς τὸν χειμάῤῥουν τὸν καταβαίνοντα ἐκ τοῦ ὄρους.

\vs{22}Καὶ ἐν τῷ ἐμπυρισμῷ, καὶ ἐν τῷ πειρασμῷ, καὶ ἐν τοῖς μνήμασι τῆς ἐπιθυμίας παροξύναντες ἦτε Κύριον.
\vs{23}Καὶ ὅτε ἐξαπέστειλεν ὑμᾶς Κύριος ἐκ Κάδης Βαρνὴ, λέγων, ἀνάβητε καὶ κληρονομήσατε τὴν γῆν, ἣν δίδωμι ὑμῖν, καὶ ἠπειθήσατε τῷ ῥήματι Κυρίου τοῦ Θεοῦ ὑμῶν, καὶ οὐκ ἐπιστεύσατε αὐτῷ, καὶ οὐκ εἰσηκούσατε τῆς φωνῆς αὐτοῦ.
\vs{24}Ἀπειθοῦντες ἦτε τὰ πρὸς Κύριον ἀπὸ τῆς ἡμέρας ἧς ἐγνώσθη ὑμῖν.
\vs{25}Καὶ ἐδεήθην ἔναντι Κυρίου τεσσαράκοντα ἡμέρας καὶ τεσσαράκοντα νύκτας, ὅσας ἐδεήθην· εἶπε γὰρ Κύριος ἐξολοθρεῦσαι ὑμᾶς.
\vs{26}Καὶ ηὐξάμην πρὸς τὸν Θεὸν, καὶ εἶπα, Κύριε βασιλεῦ τῶν θεῶν, μὴ ἐξολοθρεύσῃς τὸν λαόν σου καὶ τὴν μερίδα σου, ἣν ἐλυτρώσω, οὓς ἐξήγαγες ἐκ γῆς Αἰγύπτου ἐν τῇ ἰσχύϊ σου τῇ μεγάλῃ, καὶ ἐν τῇ χειρί σου τῇ κραταιᾷ, καὶ ἐν τῷ βραχίονί σου τῷ ὑψηλῷ.
\vs{27}Μνήσθητι Ἁβραὰμ καὶ Ἰσαὰκ καὶ Ἰακὼβ τῶν θεραπόντων σου, οἷς ὤμοσας κατὰ σεαυτοῦ· μὴ ἐπιβλέψῃς ἐπὶ τὴν σκληρότητα τοῦ λαοῦ τούτου, καὶ τὰ ἀσεβήματα, καὶ ἐπὶ τὰ ἁμαρτήματα αὐτῶν.
\vs{28}Μὴ εἴπωσιν οἱ κατοικοῦντες τὴν γῆν ὅθεν ἐξήγαγες ἡμᾶς ἐκεῖθεν, λέγοντες, παρὰ τὸ μὴ δύνασθαι Κύριον εἰσαγαγεῖν αὐτοὺς εἰς τὴν γῆν ἣν εἶπεν αὐτοῖς, καὶ παρὰ τὸ μισῆσαι αὐτοὺς, ἐξήγαγεν αὐτοὺς ἐν τῇ ἐρήμῳ ἀποκτεῖναι αὐτούς.
\vs{29}Καὶ οὗτοι λαός σου καὶ κλῆρός σου, οὓς ἐξήγαγες ἐκ γῆς Αἰγύπτου ἐν τῇ ἰσχύϊ σου τῇ μεγάλῃ, καὶ ἐν τῇ χειρί σου τῇ κραταιᾷ, καὶ ἐν τῷ βραχίονί σου τῷ ὑψηλῷ.

\ch{10}
Ἐν ἐκείνῳ τῷ καιρῷ εἶπε Κύριος πρὸς μὲ, λάξευσον σεαυτῷ δύο πλάκας λιθίνας ὥσπερ τὰς πρώτας, καὶ ἀνάβηθι πρὸς μὲ εἰς τὸ ὄρος, καὶ ποιήσεις σεαυτῷ κιβωτὸν ξυλίνην.
\vs{2}Καὶ γράψεις ἐπὶ τὰς πλάκας τὰ ῥήματα, ἃ ἦν ἐν ταῖς πλαξὶν ταῖς πρώταις ἃς συνέτριψας, καὶ ἐμβαλεῖς αὐτὰ εἰς τὴν κιβωτόν.
\vs{3}Καὶ ἐποίησα κιβωτὸν ἐκ ξύλων ἀσήπτων, καὶ ἐλάξευσα τὰς πλάκας λιθίνας ὡς αἱ πρῶται, καὶ ἀνέβην εἰς τὸ ὄρος καὶ αἱ δύο πλάκες ἐπὶ ταῖς χερσί μου.
\vs{4}Καὶ ἔγραψεν ἐπὶ τὰς πλάκας κατὰ τὴν γραφὴν τὴν πρώτην τοὺς δέκα λόγους, οὓς ἐλάλησε Κύριος πρὸς ὑμᾶς ἐν τῷ ὄρει ἐκ μέσου τοῦ πυρός, καὶ ἔδωκεν αὐτὰς Κύριος ἐμοί.
\vs{5}Καὶ ἐπιστρέψας κατέβην ἐκ τοῦ ὄρους, καὶ ἐνέβαλον τὰς πλάκας εἰς τὴν κιβωτὸν ἣν ἐποίησα· καὶ ἦσαν ἐκεῖ, καθὰ ἐνετείλατό μοι Κύριος.
\vs{6}Καὶ οἱ υἱοὶ Ἰσραὴλ ἀπῇραν ἐκ Βηρὼθ υἱῶν Ἰακεὶμ Μισαδαΐ· ἐκεῖ ἀπέθανεν Ἀαρὼν, καὶ ἐτάφη ἐκεῖ, καὶ ἱεράτευσεν Ἐλεάζὰρ υἱὸς αὐτοῦ ἀντʼ αὐτοῦ.
\vs{7}Ἐκεῖθεν ἀπῇραν εἰς Γαδγάδ· καὶ ἀπὸ Γαδγὰδ εἰς Ἐτεβαθᾶ, γῆ χείμαῤῥοι ὑδάτων.

\vs{8}Ἐν ἐκείνῳ τῷ καιρῷ διέστειλε Κύριος τὴν φυλὴν τὴν Λευὶ, αἴρειν τὴν κιβωτὸν τῆς διαθήκης Κυρίου, παρεστάναι ἔναντι Κυρίου, λειτουργεῖν καὶ ἐπεύχεσθαι ἐπὶ τῷ ὀνόματι αὐτοῦ ἕως τῆς ἡμέρας ταύτης.
\vs{9}Διὰ τοῦτο οὐκ ἔστιν τοῖς Λευίταις μερὶς καὶ κλῆρος ἐν τοῖς ἀδελφοῖς αὐτῶν· Κύριος αὐτὸς κλῆρος αὐτοῦ, καθότι εἶπεν αὐτῷ.
\vs{10}Κᾀγὼ εἱστήκειν ἐν τῷ ὄρει τεσσαράκοντα ἡμέρας καὶ τεσσαράκοντα νύκτας. καὶ εἰσήκουσε Κύριος ἐμοῦ καὶ ἐν τῷ καιρῷ τούτῳ, καὶ οὐκ ἠθέλησε Κύριος ἐξολοθρεῦσαι ὑμᾶς.
\vs{11}Καὶ εἶπε Κύριος πρὸς μέ, Βάδιζε, ἄπαρον ἐναντίον τοῦ λαοῦ τούτου, καὶ εἰσπορευέσθωσαν καὶ κληρονομείτωσαν τὴν γῆν, ἣν ὤμοσα τοῖς πατράσιν αὐτῶν δοῦναι αὐτοῖς.

\vs{12}Καὶ νῦν, Ἰσραήλ, τί Κύριος ὁ Θεός σου αἰτεῖται παρὰ σοῦ, ἀλλʼ ἢ φοβεῖσθαι Κύριον τὸν Θεόν σου, καὶ πορεύεσθαι ἐν πάσαις ταῖς ὁδοῖς αὐτοῦ, καὶ ἀγαπᾷν αὐτόν, καὶ λατρεύειν Κυρίῳ τῷ Θεῷ σου ἐξ ὅλης τῆς καρδίας σου, καὶ ἐξ ὅλης τῆς ψυχῆς σου,
\vs{13}φυλάσσεσθαι τὰς ἐντολὰς Κυρίου τοῦ Θεοῦ σου, καὶ τὰ δικαιώματα αὐτοῦ, ὅσα ἐγὼ ἐντέλλομαί σοι σήμερον, ἵνα εὖ σοι ἠ·;
\vs{14}Ἰδοὺ Κυρίου τοῦ Θεοῦ σου ὁ οὐρανὸς καὶ ὁ οὐρανὸς τοῦ οὐρανοῦ, ἡ γῆ καὶ πάντα ὅσα ἐστὶν ἐν αὐτῇ.
\vs{15}Πλὴν τοὺς πατέρας ὑμῶν προείλετο Κύριος ἀγαπᾷν αὐτούς, καὶ ἐξελέξατο τὸ σπέρμα αὐτῶν μετʼ αὐτοὺς, ὑμᾶς, παρὰ πάντα τὰ ἔθνη, κατὰ τὴν ἡμέραν ταύτην.
\vs{16}Καὶ περιτεμεῖσθε τὴν σκληροκαρδίαν ὑμῶν, καὶ τὸν τράχηλον ὑμῶν οὐ σκληρυνεῖτε.
\vs{17}Ὁ γὰρ Κύριος ὁ Θεὸς ὑμῶν, οὗτος Θεὸς τῶν θεῶν, καὶ κύριος τῶν κυρίων, ὁ Θεὸς ὁ μέγας, καὶ ἰσχυρὸς, καὶ φοβερὸς, ὅστις οὐ θαυμάζει πρόσωπον, οὐδὲ οὐ μὴ λάβῃ δῶρον·
\vs{18}ποιῶν κρίσιν προσηλύτῳ καὶ ὀρφανῷ καὶ χήρᾳ, καὶ ἀγαπᾷ τὸν προσήλυτον δοῦναι αὐτῷ ἄρτον καὶ ἱμάτιον.
\vs{19}Καὶ ἀγαπήσετε τὸν προσήλυτον· προσήλυτοι γὰρ ἦτε ἐν γῇ Αἰγύπτῳ.

\vs{20}Κύριον τὸν Θεόν σου φοβηθήσῃ, καὶ αὐτῷ λατρεύσεις, καὶ πρὸς αὐτὸν κολληθήσῃ, καὶ ἐπὶ τῷ ὀνόματι αὐτοῦ ὀμῇ·
\vs{21}Οὗτος καύχημά σου, καὶ οὗτος Θεός σου, ὅστις ἐποίησεν ἐν σοὶ τὰ μεγάλα καὶ τὰ ἔνδοξα ταῦτα, ἃ ἴδοσαν οἱ ὀφθαλμοί σου.
\vs{22}Ἐν ἑβδομήκοντα ψυχαῖς κατέβησαν οἱ πατέρες σου εἰς Αἴγυπτον· νυνὶ δὲ ἐποίησέ σε Κύριος ὁ Θεός σου ὡσεὶ τὰ ἄστρα τοῦ οὐρανοῦ τῷ πλήθει.

\ch{11}
Καὶ ἀγαπήσεις Κύριον τὸν Θεόν σου, καὶ φυλάξῃ τὰ φυλάγματα αὐτοῦ, καὶ τὰ δικαιώματα αὐτοῦ, καὶ τὰς ἐντολὰς αὐτοῦ, καὶ τὰς κρίσεις αὐτοῦ πάσας τὰς ἡμέρας.
\vs{2}Καὶ γνώσεσθε σήμερον, ὅτι οὐχὶ τὰ παιδία ὑμῶν, ὅσοι οὐκ οἴδασιν οὐδὲ ἴδοσαν τὴν παιδείαν Κυρίου τοῦ Θεοῦ σου, καὶ τὰ μεγαλεῖα αὐτοῦ, καὶ τὴν χεῖρα τὴν κραταιὰν, καὶ τὸν βραχίονα τὸν ὑψηλόν,
\vs{3}καὶ τὰ σημεῖα αὐτοῦ, καὶ τὰ τέρατα αὐτοῦ, ὅσα ἐποίησεν ἐν μέσῳ Αἰγύπτου Φαραὼ βασιλεῖ Αἰγύπτου, καὶ πάσῃ τῇ γῇ αὐτοῦ,
\vs{4}καὶ ὅσα ἐποίησε τὴν δύναμιν τῶν Αἰγυπτίων, καὶ τὰ ἅρματα αὐτῶν, καὶ τὴν ἵππον αὐτῶν, καὶ τὴν δύναμιν αὐτῶν, ὡς ἐπέκλυσε τὸ ὕδωρ τῆς θαλάσσης τῆς ἐρυθρᾶς ἐπὶ προσώπου αὐτῶν καταδιωκόντων αὐτῶν ἐκ τῶν ὀπίσω ὑμῶν, καὶ ἀπώλεσεν αὐτοὺς Κύριος ἕως τῆς σήμερον ἡμέρας,
\vs{5}καὶ ὅσα ἐποίησεν ἡμῖν ἐν τῇ ἐρήμῳ ἕως ἤλθετε εἰς τὸν τόπον τοῦτον,
\vs{6}καὶ ὅσα ἐποίησε τῷ Δαθὰν καὶ Ἀβειρὼν υἱοῖς Ἑλιὰβ υἱοῦ Ῥουβὴν, οὓς ἀνοίξασα ἡ γῆ τὸ στόμα αὐτῆς κατέπιεν αὐτοὺς, καὶ τοὺς οἴκους αὐτῶν, καὶ τὰς σκηνὰς αὐτῶν, καὶ πᾶσαν αὐτῶν τὴν ὑπόστασιν τὴν μετʼ αὐτῶν ἐν μέσῳ παντὸς Ἰσραήλ·
\vs{7}Ὅτι οἱ ὀφθαλμοὶ ὑμῶν ἑώρακαν πάντα τὰ ἔργα Κυρίου τὰ μεγάλα, ὅσα ἐποίησεν ἐν ὑμῖν σήμερον.

\vs{8}Καὶ φυλάξεσθε πάσας τὰς ἐντολὰς αὐτοῦ ὅσας ἐγὼ ἐντέλλομαί σοι σήμερον, ἵνα ζῆτε, καὶ πολυπλασιασθῆτε, καὶ εἰσελθόντες κληρονομήσητε τὴν γῆν, εἰς ἣν ὑμεῖς διαβαίνετε τὸν Ἰορδάνην ἐκεῖ κληρονομῆσαι αὐτήν·
\vs{9}Ἵνα μακροημερεύσητε ἐπὶ τῆς γῆς, ἧς ὤμοσε Κύριος τοῖς πατράσιν ὑμῶν δοῦναι αὐτοῖς καὶ τῷ σπέρματι αὐτῶν μετʼ αὐτούς, γῆν ῥέουσαν γάλα καὶ μέλι.
\vs{10}Ἔστι γὰρ ἡ γῆ εἰς ἣν εἰσπορεύῃ ἐκεῖ κληρονομῆσαι αὐτήν, οὐχ ὥσπερ γῆ Αἰγύπτου ἐστίν, ὅθεν ἐκπεπόρευσθε ἐκεῖθεν, ὅταν σπείρωσι τὸν σπόρον, καὶ ποτίζωσι τοῖς ποσὶν αὐτῶν, ὡσεὶ κῆπον λαχανείας·
\vs{11}Ἡ δὲ γῆ εἰς ἣν εἰσπορεύῃ ἐκεῖ κληρονομῆσαι αὐτὴν, γῆ ὀρεινὴ καὶ πεδινὴ· ἐκ τοῦ ὑετοῦ τοῦ οὐρανοῦ πίεται ὕδωρ.
\vs{12}Γῆ, ἣν Κύριος ὁ Θεός σου ἐπισκοπεῖται αὐτὴν διαπαντός, οἱ ὀφθαλμοὶ Κυρίου τοῦ Θεοῦ σου ἐπʼ αὐτῆς ἀπʼ ἀρχῆς τοῦ ἐνιαυτοῦ καὶ ἕως συντελείας τοῦ ἐνιαυτοῦ.

\vs{13}Ἐὰν δὲ ἀκοῇ ἀκούσητε πάσας τὰς ἐντολὰς, ἃς ἐγὼ ἐντελλομαί σοι σήμερον, ἀγαπᾷν Κύριον τὸν Θεόν σου, καὶ λατρεύειν αὐτῷ ἐξ ὅλης τῆς καρδίας σου, καὶ ἐξ ὅλης τῆς ψυχῆς σου,
\vs{14}καὶ δώσει τὸν ὑετὸν τῇ γῇ σου καθʼ ὥραν πρώϊμον καὶ ὄψιμον, καὶ εἰσοίσεις τὸν σῖτόν σου, καὶ τὸν οἶνόν σου, καὶ τὸ ἔλαιόν σου,
\vs{15}καὶ δώσει χορτάσματα ἐν τοῖς ἀγροῖς σου τοῖς κτήνεσί σου· καὶ φαγὼν, καὶ ἐμπλησθεὶς,
\vs{16}πρόσεχε σεαυτῷ μὴ πλατυνθῇ ἡ καρδία σου, καὶ παραβῆτε, καὶ λατρεύσητε θεοῖς ἑτέροις, καὶ προσκυνήσητε αὐτοῖς,
\vs{17}καὶ θυμωθεὶς ὀργῇ Κύριος ἐφʼ ὑμῖν, καὶ συσχῇ τὸν οὐρανὸν, καὶ οὐκ ἔσται ὑετὸς, καὶ ἡ γῆ οὐ δώσει τὸν καρπὸν αὐτῆς, καὶ ἀπολεῖσθε ἐν τάχει ἀπὸ τῆς γῆς τῆς ἀγαθῆς, ἧς Κύριος ἔδωκεν ὑμῖν.

\vs{18}Καὶ ἐμβαλεῖτε τὰ ῥήματα ταῦτα εἰς τὴν καρδίαν ὑμῶν καὶ εἰς τὴν ψυχὴν ὑμῶν, καὶ ἀφάψετε αὐτὰ εἰς σημεῖον ἐπὶ τῆς χειρὸς ὑμῶν, καὶ ἔσται ἀσάλευτον πρὸ ὀφθαλμῶν ὑμῶν·
\vs{19}Καὶ διδάξετε αὐτὰ τὰ τέκνα ὑμῶν λαλεῖν ἐν αὐτοῖς καθημένου σου ἐν οἴκῳ, καὶ πορευομένου σου ἐν ὁδῷ, καὶ καθεύδοντός σου, καὶ διανισταμένου σου.
\vs{20}Καὶ γράψετε αὐτὰ ἐπὶ τὰς φλιὰς τῶν οἰκιῶν ὑμῶν, καὶ τῶν πυλῶν ὑμῶν,
\vs{21}ἵνα μακροημερεύσητε, καὶ αἱ ἡμέραι τῶν υἱῶν ὑμῶν ἐπὶ τῆς γῆς, ἧς ὤμοσε Κύριος τοῖς πατράσιν ὑμῶν δοῦναι αὐτοῖς, καθὼς αἱ ἡμέραι τοῦ οὐρανοῦ ἐπὶ τῆς γῆς.
\vs{22}Καὶ ἔσται ἐὰν ἀκοῇ ἀκούσητε πάσας τὰς ἐντολὰς ταύτας ἃς ἐγὼ ἐντέλλομαί σοι σήμερον ποιεῖν, ἀγαπᾷν Κύριον τὸν Θεὸν ἡμῶν, καὶ πορεύεσθαι ἐν πάσαις ταῖς ὁδοῖς αὐτοῦ, καὶ προσκολλᾶσθαι αὐτῷ,
\vs{23}καὶ ἐκβαλεῖ Κύριος πάντα τὰ ἔθνη ταῦτα ἀπὸ προσώπου ὑμῶν, καὶ κληρονομήσετε ἔθνη μεγάλα καὶ ἰσχυρὰ μᾶλλον ἢ ὑμεῖς.
\vs{24}Πάντα τὸν τόπον οὗ ἐὰν πατήσῃ τὸ ἴχνος τοῦ ποδὸς ὑμῶν, ὑμῖν ἔσται· ἀπὸ τῆς ἐρήμου καὶ Ἀντιλιβάνου, καὶ ἀπὸ τοῦ ποταμοῦ τοῦ μελάλου, ποταμοῦ Εὐφράτου, καὶ ἕως τῆς θαλάσσης τῆς ἐπὶ δυσμῶν ἔσται τὰ ὅριά σου.
\vs{25}Οὐκ ἀντιστήσεται οὐδεὶς κατὰ πρόσωπον ὑμῶν· καὶ τὸν φόβον ὑμῶν καὶ τὸν τρόμον ὑμῶν ἐπιθήσει Κύριος ὁ Θεὸς ὑμῶν ἐπὶ πρόσωπον πάσης τῆς γῆς, ἐφʼ ἧς ἂν ἐπιβῆτε ἐπʼ αὐτῆς, ὃν τρόπον ἐλάλησε πρὸς ὑμᾶς.

\vs{26}Ἰδοὺ ἐγὼ δίδωμι ἐνώπιον ὑμῶν σήμερον τὴν εὐλογίαν καὶ τὴν κατάραν·
\vs{27}Τὴν εὐλογίαν, ἐὰν ἀκούσητε τὰς ἐντολὰς Κυρίου τοῦ Θεοῦ ὑμῶν, ὅσας ἐγὼ ἐντέλλομαι ὑμῖν σήμερον·
\vs{28}καὶ τὴν κατάραν, ἐὰν μὴ ἀκούσητε τὰς ἐντολὰς Κυρίου τοῦ Θεοῦ ὑμῶν, ὅσα ἐγὼ ἐντέλλομαι ὑμῖν σήμερον, καὶ πλανηθῆτε ἀπὸ τῆς ὁδοῦ ἧς ἐνετειλάμην ὑμῖν, πορευθέντες λατρεύειν θεοῖς ἑτέροις, οὓς οὐκ οἴδατε.
\vs{29}Καὶ ἔσται ὅταν εἰσαγάγῃ σε Κύριος ὁ Θεός σου εἰς τὴν γῆν εἰς ἣν διαβαίνεις ἐκεῖ κληρονομῆσαι αὐτήν, καὶ δώσεις εὐλογίαν ἐπʼ ὄρος Γαριζὶν, καὶ τὴν κατάραν ἐπʼ ὄρος Γαιβάλ.
\vs{30}Οὐκ ἰδοὺ ταῦτα πέραν τοῦ Ἰορδάνου, ὀπίσω ὁδὸν δυσμῶν ἡλίου ἐν γῇ Χαναὰν, τὸ κατοικοῦν ἐπὶ δυσμῶν ἐχόμενον τοῦ Γολγὸλ πλησίον τῆς δρυὸς τῆς ὑψηλῆς;
\vs{31}Ὑμεῖς γὰρ διαβαίνετε τὸν Ἰορδάνην, εἰσελθόντες κληρονομῆσαι τὴν γῆν, ἣν Κύριος ὁ Θεὸς ἡμῶν δίδωσιν ὑμῖν ἐν κλήρῳ πάσας τὰς ἡμέρας, καὶ κατοικήσετε ἐν αὐτῇ.

\vs{32}Καὶ φυλάξεσθε τοῦ ποιεῖν πάντα τὰ προστάγματα αὐτοῦ, καὶ τὰς κρίσεις ταύτας, ὅσας ἐγὼ δίδωμι ἐνώπιον ὑμῶν σήμερον.

\ch{12}
Καὶ ταῦτα τὰ προστάγματα καὶ αἱ κρίσεις, ἃς φυλάξετε τοῦ ποιεῖν ἐν τῇ γῇ, ἣν Κύριος ὁ Θεὸς τῶν πατέρων ὑμῶν δίδωσιν ὑμῖν ἐν κλήρῳ, πάσας τὰς ἡμέρας, ἃς ὑμεῖς ζῆτε ἐπὶ τῆς γῆς.
\vs{2}Ἀπωλίᾳ ἀπολεῖτε πάντας τοὺς τόπους ἐν οἷς ἐλάτρευσαν ἐκεῖ τοῖς θεοῖς αὐτῶν, οὓς ὑμεῖς κληρονομεῖτε αὐτοὺς, ἐπὶ τῶν ὀρέων τῶν ὑψηλῶν, καὶ ἐπὶ τῶν θινῶν, καὶ ὑποκάτω δένδρου δασέος·
\vs{3}Καὶ κατασκάψετε τοὺς βωμοὺς αὐτῶν, καὶ συντρίψετε τὰς στήλας αὐτῶν, καὶ τὰ ἄλση αὐτῶν ἐκκόψετε, καὶ τὰ γλυπτὰ τῶν θεῶν αὐτῶν κατακαύσετε πυρί, καὶ ἀπολεῖτε τὸ ὄνομα αὐτῶν ἐκ τοῦ τόπου ἐκείνου.
\vs{4}Οὐ ποιήσετε οὕτω Κυρίῳ τῷ Θεῷ ὑμῶν.
\vs{5}Ἀλλʼ ἢ εἰς τὸν τόπον, ὃν ἂν ἐκλέξηται Κύριος ὁ Θεός σου ἐν μιᾷ τῶν πόλεων ὑμῶν ἐπονομάσαι τὸ ὄνομα αὐτοῦ ἐκεῖ καὶ ἐπικληθῆναι, καὶ ἐκζητήσετε καὶ ἐλεύσεσθε ἐκεῖ.
\vs{6}Καὶ οἴσετε ἐκεῖ τὰ ὁλοκαυτώματα ὑμων, καὶ τὰ θυσιάσματα ὑμῶν, καὶ τὰς ἀπαρχὰς ὑμῶν, καὶ τὰς εὐχὰς ὑμῶν, καὶ τὰ ἐκούσια ὑμῶν, καὶ τὰς ὁμολογίας ὑμῶν, τὰ πρωτότοκα τῶν βοῶν ὑμῶν, καὶ τῶν προβάτων ὑμῶν.
\vs{7}Καὶ φάγεσθε ἐκεῖ ἐναντίον Κυρίου τοῦ Θεοῦ ὑμῶν, καὶ εὐφρανθήσεσθε ἐπὶ πᾶσιν, οὗ ἐὰν ἐπιβάλητε τὴν χεῖρα ὑμεῖς, καὶ οἱ οἶκοι ὑμῶν, καθότι εὐλόγησέ σε Κύριος ὁ Θεός σου.

\vs{8}Οὐ ποιήσετε πάντα ὅσα ἡμεῖς ποιοῦμεν ὧδε σήμερον, ἕκαστος τὸ ἀρεστὸν ἐνώπιον αὐτοῦ·
\vs{9}Οὐ γὰρ ἥκατε ἕως τοῦ νῦν εἰς τὴν κατάπαυσιν, καὶ εἰς τὴν κληρονομίαν, ἣν Κύριος ὁ Θεὸς ἡμῶν δίδωσιν ὑμῖν.
\vs{10}Καὶ διαβήσεσθε τὸν Ἰορδάνην, καὶ κατοικήσετε ἐπὶ τῆς γῆς, ἧς Κύριος ὁ Θεὸς ἡμῶν κατακληρονομεῖ ὑμῖν, καὶ καταπαύσει ὑμᾶς ἀπὸ πάντων τῶν ἐχθρῶν ὑμῶν τῶν κύκλῳ, καὶ κατοικήσετε μετὰ ἀσφαλείας.
\vs{11}Καὶ ἔσται ὁ τόπος, ὃν ἂν ἐκλέξηται Κύριος ὁ Θεός σου ἐπικληθῆναι τὸ ὄνομα αὐτοῦ ἐκεῖ, ἐκεῖ οἴσετε πάντα ὅσα ἐγὼ ἐντέλλομαι ὑμῖν σήμερον· τὰ ὁλοκαυτώματα ὑμῶν, καὶ τὰ θυσιάσματα ὑμῶν, καὶ τὰ ἐπιδέκατα ὑμῶν, καὶ τὰς ἀπαρχὰς τῶν χειρῶν ὑμῶν, καὶ πᾶν ἐκλεκτὸν τῶν δώρων ὑμῶν, ὅσα ἂν εὔξησθε Κυρίῳ τῷ Θεῷ ὑμῶν.
\vs{12}Καὶ εὐφρανθήσεσθε ἐναντίον Κυρίου τοῦ Θεοῦ ὑμῶν, ὑμεῖς καὶ οἱ υἱοὶ ὑμῶν, καὶ αἱ θυγατέρες ὑμῶν, καὶ οἱ παῖδες ὑμῶν, καὶ αἱ παιδίσκαι ὑμῶν, καὶ ὁ Λευίτης ὁ ἐπὶ τῶν πυλῶν ὑμῶν· ὅτι οὐκ ἔστιν αὐτῷ μερὶς οὐδὲ κλῆρος μεθʼ ὑμῶν.
\vs{13}Πρόσεχε σεαυτῷ, μὴ ἀνενέγκῃς τὰ ὁλοκαυτώματά σου ἐν παντὶ τόπῳ οὗ ἐὰν ἴδῃς.
\vs{14}Ἀλλʼ ἢ εἰς τὸν τόπον, ὃν ἂν ἐκλέξεται Κύριος ὁ Θεός σου αὐτὸν, ἐν μιᾷ τῶν φυλῶν σου, ἐκεῖ ἀνοίσετε τὰ ὁλοκαυτώματα ὑμῶν, καὶ ἐκεῖ ποιήσεις πάντα ὅσα ἐγὼ ἐντέλλομαί σοι σήμερον.
\vs{15}Ἀλλʼ ἢ ἐν πάσῃ ἐπιθυμίᾳ σου θύσεις, καὶ φαγῇ κρέα κατὰ τὴν εὐλογίαν Κυρίου τοῦ Θεοῦ σου, ἣν ἔδωκέ σοι ἐν πάσῃ πόλει· ὁ ἀκάθαρτος ἐν σοὶ καὶ ὁ καθαρὸς ἐπὶ τὸ αὐτό φάγεται αὐτό, ὡς δορκάδα ἢ ἔλαφον.
\vs{16}Πλὴν τὸ αἷμα οὐ φάγεσθε· ἐπὶ τὴν γῆν ἐκχεεῖτε αὐτὸ, ὡς ὕδωρ.

\vs{17}Οὐ δυνήσῃ φαγεῖν ἐν ταῖς πόλεσί σου τὸ ἐπιδέκατον τοῦ σίτου σου, καὶ τοῦ οἴνου σου, καὶ τοῦ ἐλαίου σου, τὰ πρωτότοκα τῶν βοῶν σου, καὶ τῶν προβάτων σαυ, καὶ πάσας τὰς εὐχὰς, ὅσας ἂν εὔξησθε, καὶ τὰς ὁμολογίας ὑμῶν, καὶ τὰς ἀπαρχὰς τῶν χειρῶν σου.
\vs{18}Ἀλλʼ ἢ ἐναντίον Κυρίου τοῦ Θεοῦ σου φάγῃ αὐτὸ ἐν τῷ τόπῳ, ᾧ ἂν ἐκλέξηται Κύριος ὁ Θεός σου αὐτῷ, σὺ καὶ ὁ υἱός σου, καὶ ἡ θυγάτηρ σου, ὁ παῖς σου, καὶ ἡ παιδίσκη σου, καὶ ὁ προσήλυτος ὁ ἐν ταῖς πόλεσιν ὑμῶν· καὶ εὐφρανθήσῃ ἐναντίον Κυρίου τοῦ Θεοῦ σου ἐπὶ πάντα, οὗ ἐὰν ἐπιβάλῃς τὴν χεῖρά σου.

\vs{19}Πρόσεχε σεαυτῷ μὴ ἐγκαταλίπῃς τὸν Λευίτην πάντα τὸν χρόνον ὅσον ἂ ζῇς ἐπὶ τῆς γῆς.

\vs{20}Ἐὰν δὲ ἐμπλατύνῃ Κύριος ὁ Θεός σου τὰ ὅριά σου, καθάπερ ἐλάλησέ σοι, καὶ ἐρεῖς, φάγομαι κρέα, ἐὰν ἐπιθυμήσῃ ἡ ψυχή σου ὥστε φαγεῖν κρέα, ἐν πάσῃ ἐπιθυμίᾳ τῆς ψυχῆς σου φάγῃ κρέα.
\vs{21}Ἐὰν δὲ μακρὰν ἀπέχῃ σου ὁ τόπος, ὃν ἂν ἐκλέξηται Κύριος ὁ Θεός σου ἐκεῖ ἐπικληθῆναι τὸ ὄνομα αὐτοῦ ἐκεῖ, καὶ θύσεις ἀπὸ τῶν βοῶν σου, καὶ ἀπὸ τῶν προβάτων σου ὧν ἂν δῷ ὁ Θεός σοι, ὃν τρόπον ἐνετειλάμην σοι, καὶ φαγῃ ἐν ταῖς πόλεσί σου κατὰ τὴν ἐπιθυμίαν τῆς ψυχῆς σου.
\vs{22}Ὡς ἔσθεται ἡ δορκὰς καὶ ἡ ἔλαφος, οὕτω φαγῃ αὐτό· ὁ ἀκάθαρτος ἐν σοὶ καὶ ὁ καθαρὸς ὡσαύτως ἔδεται.
\vs{23}Πρόσεχε ἰσχυρῶς τοῦ μὴ φαγεῖν αἷμα, ὅτι αἷμα αὐτοῦ ψυχή· οὐ βρωθήσεται ψυχὴ μετὰ τῶν κρεῶν.
\vs{24}Οὐ φάγεσθε· ἐπὶ τὴν γῆν ἐκχεεῖτε αὐτὸ ὡς ὕδωρ.
\vs{25}Οὐ φαγῇ αὐτὸ, ἵνα εὖ σοι γένηται καὶ τοῖς υἱοῖς σου μετὰ σὲ, ἐὰν ποιήσῃς τὸ καλὸν καὶ τὸ ἀρεστὸν ἐναντίον Κυρίου τοῦ Θεοῦ σου.
\vs{26}Πλὴν τὰ ἅγιά σου ἐὰν γένηταί σοι, καὶ τὰς εὐχάς σου λαβὼν ἥξεις εἰς τὸν τόπον, ὃν ἂν ἐκλέξηται Κύριος ὁ Θεός σου ἐπικληθῆναι τὸ ὄνομα αὐτοῦ ἐκεῖ.
\vs{27}Καὶ ποιήσεις τὰ ὁλοκαυτώματά σου, τὰ κρέα ἀνοίσεις ἐπὶ τὸ θυσιαστήριον Κυρίου τοῦ Θεοῦ σου· τὸ δὲ αἷμα τῶν θυσιῶν σου προσχεεῖς πρὸς τὴν βάσιν τοῦ θυσιαστηρίου Κυρίου τοῦ Θεοῦ σου, τὰ δὲ κρέα φαγῇ.
\vs{28}Φυλάσσου καὶ ἄκουε καὶ ποιήσεις πάντας τοὺς λόγους οὓς ἐγὼ ἐντέλλομαί σοι, ἵνα εὖ σοι γένηται καὶ τοῖς υἱοῖς σου διʼ αἰῶνος, ἐὰν ποιήσῃς τὸ ἀρεστὸν καὶ τὸ καλὸν ἐναντίον Κυρίου τοῦ Θεοῦ σου.

\vs{29}Ἐὰν δὲ ἐξολοθρεύσῃ Κύριος ὁ Θεός σου τὰ ἔθνη, εἰς οὓς εἰσπορεύῃ ἐκεῖ κληρονομῆσαι τὴν γῆν αὐτῶν, ἀπὸ προσώπου σου, καὶ κατακληρονομήσῃς αὐτὴν, καὶ κατοικήσῃς ἐν τῇ γῇ αὐτῶν,
\vs{30}πρόσεχε σεαυτῷ μὴ ἐκζητήσῃς ἐπακολουθῆσαι αὐτοῖς μετὰ τὸ ἐξολοθρευθῆναι αὐτοὺς ἀπὸ προσώπου σου, λέγων, πῶς ποιοῦσι τὰ ἔθνη ταῦτα τοῖς θεοῖς αὐτῶν; ποιήσω κᾀγώ.
\vs{31}Οὐ ποιήσεις οὕτω τῷ Θεῷ σου· τὰ γὰρ βδελύματα Κυρίου ἃ ἐμίσησεν, ἐποίησαν ἐν τοῖς θεοῖς αὐτῶν, ὅτι τοὺς υἱοὺς αὐτῶν, καὶ τὰς θυγατέρας αὐτῶν κατακαίουσιν ἐν πυρὶ τοῖς θεοῖς αὐτῶν.

\ch{13}Πᾶν ῥῆμα ὃ ἐγὼ ἐντλλομαι ὑμῖν σήμερον, τοῦτο φυλάξῃ ποιεῖν· οὐ προσθήσεις ἐπʼ αὐτό, οὐδὲ ἀφελεῖς ἀπʼ αὐτοῦ.

\vs{2}Ἐὰν δὲ ἀναστῇ ἐν σοὶ προφήτης ἢ ἐνυπνιαζόμενος τὸ ἐνύπνιον, καὶ δῷ σοι σημεῖον ἢ τέρας,
\vs{3}καὶ ἔλθῃ τὸ σημεῖον ἢ τὸ τέρας ὃ ἐλάλησε πρὸς σὲ, λέγων, πορευθῶμεν καὶ λατρεύσωμεν θεοῖς ἑτέροις οὓς οὐκ οἴδατε,
\vs{4}οὐκ ἀκούσεσθε τῶν λόγων τοῦ προφήτου ἐκείνου ἢ τοῦ ἐνυπνιαζομένου τὸ ἐνύπνιον ἐκεῖνο· ὅτι πειράζει Κύριος ὁ Θεός σου ὑμᾶς, εἰδέναι εἰ ἀγαπᾶτε τὸν Θεὸν ὑμῶν ἐξ ὅλης τῆς καρδίας ὑμῶν, καὶ ἐξ ὅλης τῆς ψυχῆς ὑμῶν.
\vs{5}Ὀπίσω Κυρίου τοῦ Θεοῦ ὑμῶν πορεύεσθε, καὶ τοῦτον φοβηθήσεσθε, καὶ τῆς φωνῆς αὐτοῦ ἀκούσεσθε, καὶ αὐτῷ προστεθήσεσθε.
\vs{6}Καὶ ὁ προφήτης ἐκεῖνος ἢ ὁ τὸ ἐνύπνιον ἐνυπνιαζόμενος ἐκεῖνος, ἀποθανεῖται· ἐλάλησε γὰρ πλανῆσαί σε ἀπὸ Κυρίου τοῦ Θεοῦ σου ἐξαγαγόντος σε ἐκ γῆς Αἰγύπτου, τοῦ λυτρωσαμένου σε ἐκ τῆς δουλείας, ἐξῶσαί σε ἀπὸ τῆς ὁδοῦ, ἧς ἐνετείλατό σοι Κύριος ὁ Θεός σου πορεύεσθαι ἐν αὐτῇ· καὶ ἀφανιεῖς τὸν πονηρὸν ἐξ ὑμῶν αὐτῶν.

\vs{7}Ἐὰν δὲ παρακαλέσῃ σε ὁ ἀδελφός σου ἐκ πατρός σου ἢ ἐκ μητρός σου, ἢ ὁ υἱός σου, ἢ ἡ θυγάτηρ, ἢ ἡ γυνή σου ἡ ἐν κόλπῳ σου, ἢ φίλος ἴσος τῇ ψυχῇς σου λάθρα, λέγων, βαδίσωμεν καὶ λατρεύσωμεν θεοῖς ἑτέροις, οὓς οὐκ ᾔδεις σὺ καὶ οἱ πατέρες σου,
\vs{8}ἀπὸ τῶν θεῶν τῶν ἐθνῶν τῶν περὶ κύκλῳ ὑμῶν, τῶν ἐγγιζόντων σοι ἢ τῶν μακρὰν ἀπὸ σοῦ, ἀπʼ ἄκρου τῆς γῆς ἕως ἄκρου τῆς γῆς,
\vs{9}οὐ συνθελήσεις αὐτῷ, καὶ οὐκ εἰσακούσῃ αὐτοῦ, καὶ οὐ φείσεται ὁ ὀφθαλμός σου ἐπʼ αὐτῷ, οὐκ ἐπιποθήσεις ἐπʼ αὐτῷ, οὐδʼ οὐ μὴ σκεπάσῃς αὐτόν·
\vs{10}ἀναγγέλλων ἀναγγελεῖς περὶ αὐτοῦ, καὶ αἱ χεῖρές σου ἔσονται ἐπʼ αὐτὸν ἐν πρώτοις ἀποκτεῖναι αὐτὸν, καὶ αἱ χεῖρες παντὸς τοῦ λαοῦ ἐπʼ ἐσχάτῳ.
\vs{11}Καὶ λιθοβολήσουσιν αὐτὸν ἐν λίθοις, καὶ ἀποθανεῖται, ὅτι ἐζήτησεν ἀποστῆσαί σε ἀπὸ Κυρίου τοῦ Θεοῦ σου τοῦ ἐξαγαγόντος σε ἐκ γῆς Αἰγύπτου, ἐξ οἴκου δουλείας.
\vs{12}Καὶ πᾶς Ἰσραὴλ ἀκούσας φοβηθήσεται, καὶ οὐ προσθήσει ποιῆσαι ἔτι κατὰ τὸ ῥῆμα τὸ πονηρὸν τοῦτο ἐν ὑμῖν.

\vs{13}Ἐὰν δὲ ἀκούσῃς ἐν μιᾷ τῶν πόλεών σου, ὧν Κύριος ὁ Θεός σου δίδωσί σοι κατοικεῖν σε ἐκεῖ, λεγόντων,
\vs{14}ἐξήλθοσαν ἄνδρες παράνομοι ἐξ ὑμῶν, καὶ ἀπέστησαν πάντας τοὺς κατοικοῦντας τὴν γῆν αὐτῶν, λέγοντες, πορευθῶμεν καὶ λατρεύσωμεν θεοῖς ἑτέροις, οὓς οὐκ ᾔδειτε,
\vs{15}καὶ ἐτάσεις καὶ ἐρωτήσεις, καὶ ἐρευνήσεις σφόδρα, καὶ ἰδοὺ ἀληθὴς σαφῶς ὁ λόγος, γεγένηται τὸ βδέλυγμα τοῦτο ἐν ὑμῖν·
\vs{16}ἀναιρῶν ἀνελεῖς πάντας τοὺς κατοικοῦντας ἐν τῇ γῇ ἐκείνῃ ἐν φόνῳ μαχαίρας, ἀναθέματι ἀναθεματιεῖτε αὐτὴν, καὶ πάντα τὰ ἐν αὐτῇ.
\vs{17}Καὶ πάντα τὰ σκῦλα αὐτῆς συνάξεις εἰς τὰς διόδους αὐτῆς, καὶ ἐμπρήσεις τὴν πόλιν ἐν πυρὶ, καὶ πάντα τὰ σκῦλα αὐτῆς πανδημεὶ ἐναντίον Κυρίου τοῦ Θεοῦ σου· καὶ ἔσται ἀοίκητος εἰς τὸν αἰῶνα, οὐκ ἀνοικοδομηθήσεται ἔτι.
\vs{18}Καὶ οὐ προσκολληθήσεται οὐδὲν ἀπὸ τοῦ ἀναθέματος ἐν τῇ χειρί σου, ἵνα ἀποστραφῇ Κύριος ἀπὸ θυμοῦ τῆς ὀργῆς αὐτοῦ, καὶ δώσῃ σοι ἔλεος, καὶ ἐλεήσῇ σε, καὶ πληθύνῃ σε, ὃν τρόπον ὤμοσε τοῖς πατράσι σου,
\vs{19}ἐὰν ἀκούσῃς τῆς φωνῆς Κυρίου τοῦ Θεοῦ σου, φυλάσσειν τὰς ἐντολὰς αὐτοῦ, ὅσας ἐγὼ ἐντέλλομαί σοι σήμερον, ποιεῖν τὸ καλὸν καὶ τὸ ἀρεστὸν ἐναντίον Κυρίου τοῦ Θεοῦ σου.

\ch{14}
Υἱοί ἐστε Κυρίου τοῦ Θεοῦ ὑμῶν· οὐκ ἐπιθήσετε φαλάκρωμα ἀναμέσον τῶν ὀφθαλμῶν ὑμῶν ἐπὶ νεκρῷ.
\vs{2}Ὅτι λαὸς ἅγιος εἶ Κυρίῳ τῷ Θεῷ σου, καὶ σε ἐξελέξατο Κύριος ὁ Θεός σου γενέσθαι σε λαὸν αὐτῷ περιούσιον ἀπὸ πάντων τῶν ἐθνῶν τῶν ἐπὶ προσώπου τῆς γῆς.
\vs{3}Οὐ φάγεσθε πᾶν βδέλυγμα.
\vs{4}Ταῦτα κτήνη ἃ φάγεσθε· μόσχον ἐκ βοῶν, καὶ ἀμνὸν ἐκ προβάτων, καὶ χίμαρον ἐξ αἰγῶν·
\vs{5}ἔλαφον, καὶ δορκάδα, καὶ πύγαργον, ὄρυγα, καὶ καμηλοπάρδαλιν.
\vs{6}Πᾶν κτῆνος διχηλοῦν ὁπλὴν, καὶ ὀνυχιστῆρας ὀνυχίζον δύο χηλῶν, καὶ ἀνάγον μηρυκισμὸν ἐν τοῖς κτήνεσι, ταῦτα φάγεσθε.
\vs{7}Καὶ ταῦτα οὐ φάγεσθε ἀπὸ τῶν ἀναγόντων μηρυκισμὸν, καὶ ἀπὸ τῶν διχηλούντων τὰς ὁπλὰς, καὶ ὀνυχιζόντων ὀνυχιστῆρας· τὸν κάμηλον, καὶ δασύποδα, καὶ χοιρογρύλλιον· ὅτι ἀνάγουσι μηρυκισμὸν, καὶ ὁπλὴν οὐ διχηλοῦσιν, ἀκάθαρτα ταῦτα ὑμῖν ἐστι.
\vs{8}Καὶ τὸν ὗν, ὅτι διχηλεῖ ὁπλὴν τοῦτο, καὶ ὀνυχίζει ὀνυχιστῆρας ὁπλῆς, καὶ τοῦτο μηρυκισμὸν οὐ μηρυκᾶται ἀκάθαρτον τοῦτο ὑμῖν· ἀπὸ τῶν κρεῶν αὐτῶν οὐ φάγεσθε, τῶν θνησιμαίων αὐτῶν οὐχ ἅψεσθε.

\vs{9}Καὶ ταῦτα φάγεσθε ἀπὸ πάντων τῶν ἐν τῷ ὕδατι, πάντα ὅσα ἐστὶν ἐν αὐτοῖς πτερύγια καὶ λεπίδες, φάγεσθε.
\vs{10}Καὶ πάντα ὅσα οὐκ ἔστιν αὐτοῖς πτερύγια καὶ λεπίδες, οὐ φάγεσθε· ἀκάθαρτα ὑμῖν ἐστι.
\vs{11}Πᾶν ὄρνεον καθαρὸν φάγεσθε.
\vs{12}Καὶ ταῦτα οὐ φάγεσθε ἀπʼ αὐτῶν· τὸν ἀετὸν, καὶ τὸν γρύπα, καὶ τὸν ἁλιάετον,
\vs{13}καὶ τὸν γύπα, καὶ τὸν ἴκτινον, καὶ τὰ ὅμοια αὐτῷ,
\vs{15}καὶ στρουθὸν, καὶ γλαῦκα, καὶ λάρον,
\vs{16}καὶ ἐρωδιὸν, καὶ κύκνον, καὶ ἶβιν,
\vs{17}καὶ καταράκτην, καὶ ἱέρακα, καὶ τὰ ὅμοια αὐτῷ, καὶ ἔποπα, καὶ νυκτικόρακα,
\vs{18}καὶ πελακᾶνα, καὶ χαραδριὸν, καὶ τὰ ὅμοια αὐτῷ, καὶ πορφυρίωνα, καὶ νυκτερίδα.
\vs{19}Πάντα τὰ ἑρπετὰ τῶν πετεινῶν ἀκάθαρτά ἐστιν ὑμῖν· οὐ φάγεσθε ἀπʼ αὐτῶν.
\vs{20}Πᾶν πετεινὸν καθαρὸν φάγεσθε.
\vs{21}Πᾶν θνησιμαῖον οὐ φάγεσθε· τῷ παροίκῳ τῷ ἐν ταῖς πόλεσί σου δοθήσεται καὶ φάγεται, ἢ ἀποδώσῃ τῷ ἀλλοτρίῳ, ὅτι λαὸς ἅγιος εἶ Κυρίῳ τῷ Θεῷ σου. Οὐχ ἑψήσεις ἄρνα ἐν γάλακτι μητρὸς αὐτοῦ.

\vs{22}Δεκάτην ἀποδεκατώσεις παντὸς γεννήματος τοῦ σπέρματός σου, τὸ γέννημα τοῦ ἀγροῦ σου ἐνιαυτὸν κατʼ ἐνιαυτόν.
\vs{23}Καὶ φαγῇ αὐτὸ ἐν τῷ τόπῳ, ᾧ ἐὰν ἐκλέξηται Κύριος ὁ Θεός σου ἐπικληθῆναι τὸ ὄνομα αὐτοῦ ἐκεῖ· οἴσετε τὰ ἐπιδέκατα τοῦ σίτου σου, καὶ τοῦ οἴνου σου, καὶ τοῦ ἐλαίου σου, τὰ πρωτότοκα τῶν βοῶν σου, καὶ τῶν προβάτων σου, ἵνα μάθῃς φοβεῖσθαι Κύριον τὸν Θεόν σου πάσας τὰς ἡμέρας.
\vs{24}Ἐὰν δὲ μακρὰν γένηται ἡ ὁδὸς ἀπὸ σοῦ, καὶ μὴ δύνῃ ἀναφέρειν αὐτὰ, ὅτι μακρὰν ἀπὸ σοῦ ὁ τόπος, ὃν ἂν ἐκλέξηται Κύριος ὁ Θεός σου ἐπικληθῆναι τὸ ὄνομα αὐτοῦ ἐκεῖ, ὅτι εὐλογήσει σε Κύριος ὁ Θεός σοῦ,
\vs{25}καὶ ἀποδώσῃ αὐτὰ ἀργυρίου, καὶ λήψῃ τὸ ἀργύριον ἐν ταῖς χερσί σου, καὶ πορεύσῃ εἰς τὸν τόπον ὃν ἂν ἐκλέξηται Κύριος ὁ Θεός σου αὐτόν.
\vs{26}Καὶ δώσεις ἀργύριον ἐπὶ παντὸς οὗ ἂν ἐπιθυμεῇ ἡ ψυχή σου, ἐπὶ βουσὶν ἤ ἐπὶ προβάτοις, ἢ ἐτʼ οἴνῳ ἢ ἐπὶ σίκερα, ἢ ἐπὶ σίκερα, ἢ ἐπὶ παντὸς οὗ ἂν ἐπιθυμῇ ἡ φυχή σου, καὶ φαγῇ ἐκεῖ ἐναντίον Κυρίου τοῦ Θεοῦ σου, καὶ εὐφρανθήσῃ σὺ καὶ ὁ οἶκός σου,
\vs{27}καὶ ὁ Λευίτης ὁ ἐν ταῖς πόλεσί σου, ὅτι οὐκ ἔστιν αὐτῷ μερὶς οὐδὲ κλῆρος μετὰ σοῦ.

\vs{28}Μετὰ τρία ἔτη ἐξοίσεις πᾶν τὸ ἐπιδέκατον τῶν γεννημάτων σου, ἐν τῷ ἐνιαυτῷ ἐκείνῳ θήσεις αὐτὸ ἐν ταῖς πόλεσί σου.
\vs{29}Καὶ ἐλεύσεται ὁ Λευίτης, ὅτι οὐκ ἔστιν αὐτῷ μερὶς οὐδὲ κλῆρος μετὰ σοῦ, καὶ ὁ προσήλυτος καὶ ὁ ὀρφανὸς καὶ ἡ χήρα ἡ ἐν ταῖς πόλεσί σου, καὶ φάγονται καὶ ἐμπλησθήσονται, ἵνα εὐλογήσῃ σε Κύριος ὁ Θεός σου ἐν πᾶσι τοῖς ἔργοις οἷς ἐὰν ποιῇς.

\ch{15}
Διʼ ἑπτὰ ἐτῶν ποιήσεις ἄφεσιν.
\vs{2}Καὶ οὕτω τὸ πρόσταγμα τῆς ἀφέσεως· ἀφήσεις πᾶν χρέος ἴδιον, ὃ ὀφείλει σοι ὁ πλησίον, καὶ τὸν ἀδελφόν σου οὐκ ἀπαιτήσεις· ἐπικέκληται γὰρ ἄφεσις Κυρίῳ τῷ Θεῷ σου.
\vs{3}Τὸν ἀλλότριον ἀπαιτήσεις ὅσα ἐὰν ἠ· σοι παρʼ αὐτῷ, τῷ δὲ ἀδελφῷ σου ἄφεσιν ποιήσεις τοῦ χρέους σου.
\vs{4}Ὅτι οὐκ ἔσται ἐν σοὶ ἐνδεὴς, ὅτι εὐλογῶν εὐλογήσει σε Κύριος ὁ Θεός σου ἐν τῇ γῇ, ᾗ Κύριος ὁ Θεός σου δίδωσί σοι ἐν κλήρῳ κατακληρονομεῖν σε αὐτήν.

\vs{5}Ἐὰν δὲ ἀκοῇ εἰσακούσητε τῆς φωνῆς Κυρίου τοῦ Θεοῦ ὑμῶν φυλάσσειν καὶ ποιεῖν πάσας τὰς ἐντολὰς ταύτας, ὅσας ἐγὼ ἐντέλλομαί σοι σήμερον,
\vs{6}ὅτι Κύριος ὁ Θεός σου εὐλόγησέ σε ὃν τρόπον ἐλάλησέ σοι· καὶ δανειεῖς ἔθνεσι πολλοῖς, σὺ δὲ οὐ δανειῇ· καὶ ἄρξεις ἐθνῶν πολλῶν, σοῦ δὲ οὐκ ἄρξουσιν.

\vs{7}Ἐὰν δὲ γένηται ἐν σοὶ ἐνδεὴς ἐκ τῶν ἀδελφῶν σου ἐν μιᾷ τῶν πόλεών σου ἐν τῇ γῇ, ᾗ Κύριος ὁ Θεός σου δίδωσί σοι, οὐκ ἀποστέρξεις τὴν καρδίαν σου, οὐδʼ οὐ μὴ συσφίγξεις τὴν χεῖρά σου ἀπὸ τοῦ ἀδελφοῦ σου τοῦ ἐπιδεομένου.
\vs{8}Ἀνοίγων ἀνοίξεις τὰς χεῖράς σου αὐτῷ, καὶ δάνειον δανειεῖς αὐτῷ ὅσον ἐπιδέεται, καθότι ἐνδεεῖται.
\vs{9}Πρόσεχε σεαυτῷ μὴ γένηται ῥῆμα κρυπτὸν ἐν τῇ καρδίᾳ σου ἀνόμημα, λέγων, Ἐγγίζει τὸ ἔτος τὸ ἕβδομον, ἔτος τῆς ἀφέσεως, καὶ πονηρεύσηται ὁ ὀφθαλμός σου τῷ ἀδελφῷ σου τῷ ἐπιδεομένῳ, καὶ οὐ δώσεις αὐτῷ, καὶ καταβοήσεται κατὰ σοῦ πρὸς Κύριον, καὶ ἔσται ἐν σοὶ ἁμαρτία μεγάλη.
\vs{10}Διδοὺς δώσεις αὐτῷ, καὶ δάνειον δανειεῖς αὐτῷ ὅσον ἐπιδέεται, καθότι ἐνδεεῖται· καὶ οὐ λυπηθήσῃ τῇ καρδίᾳ σου διδόντος σου αὐτῷ, ὅτι διὰ τὸ ῥῆμα τοῦτο εὐλογήσει σε Κύριος ὁ Θεός σου ἐν πᾶσιν τοῖς ἔργοις καὶ ἐν πᾶσι οὗ ἂν ἐπιβάλῃς τὴν χεῖρά σου.
\vs{11}Οὐ γὰρ μὴ ἐκλίπῃ ἐνδεὴς ἀπὸ τῆς γῆς σου· διὰ τοῦτο ἐγώ σοι ἐντέλλομαι ποιεῖν τὸ ῥῆμα τοῦτο, λέγων, ἀνοίγων ἀνοίξεις τὰς χεῖράς σου τῷ ἀδελφῷ σου τῷ πένητι καὶ τῷ ἐπιδεομένῳ τῷ ἐπὶ τῆς γῆς σου.

\vs{12}Ἐὰν δὲ πραθῇ σοι ὁ ἀδελφός σου ὁ Ἑβραῖος ἢ Ἐβραία, δουλεύσει σοι ἓξ ἔτη, καὶ τῷ ἑβδόμῳ ἐξαποστελεῖς αὐτὸν ἐλεύθερον ἀπὸ σοῦ.
\vs{13}Ὅταν δὲ ἐξαποστέλλῃς αὐτὸν ἐλεύθερον ἀπὸ σοῦ, οὐκ ἐξαποστελεῖς αὐτὸν κενόν.
\vs{14}Ἐφόδιον ἐφοδιάσεις αὐτὸν ἀπὸ τῶν προβάτων σου, καὶ ἀπὸ τοῦ σίτου σου, καὶ ἀπὸ τοῦ οἴνου σου· καθὰ εὐλόγησέ σε Κύριος ὁ Θεός σου, δώσεις αὐτῷ.

\vs{15}Καὶ μνησθήσῃ ὅτι οἰκέτης ἦσθα ἐν γῇ Αἰγύπτου, καὶ ἐλυτρώσατό σε Κύριος ὁ Θεός σοῦ ἐκεῖθεν· διὰ τοῦτο ἐγώ σοι ἐντέλλομαι ποιεῖν τὸ ῥῆμα τοῦτο.
\vs{16}Ἐὰν δὲ λέγῃ πρὸς σὲ, οὐκ ἐξελεύσομαι ἀπὸ σοῦ, ὅτι ἠγάπηκέ σε καὶ τὴν οἰκίαν σου, ὅτι εὖ ἐστιν αὐτῷ παρὰ σοί.
\vs{17}Καὶ λήψῃ τὸ ὀπήτιον, καὶ τρυπήσεις τὸ ὠτίον αὐτοῦ πρὸς τὴν θύραν, καὶ ἔσται σοι οἰκέτης εἰς τὸν αἰῶνα· καὶ τὴν παιδίσκην σου ὡσαύτως ποιήσεις.
\vs{18}Οὐ σκληρὸν ἔσται ἐναντίον σου ἐξαποστελλομένων αὐτῶν ἐλευθέρων ἀπὸ σου, ὅτι ἐπέτειον μισθὸν τοῦ μισθωτοῦ ἐδούλευσέ σοι ἓξ ἔτη· καὶ εὐλογήσει σε Κύριος ὁ Θεός σου ἐν πᾶσιν οἷς ἐὰν ποιῇς.

\vs{19}Πᾶν πρωτότοκον ὃ ἐὰν τεχθῇ ἐν ταῖς βουσί σου, καὶ ἐν τοῖς προβάτοις σου, τὰ ἀρσενικὰ ἁγιάσεις Κυρίῳ τῷ Θεῷ σου· οὐκ ἐργᾷ ἐν τῷ πρωτοτόκῳ μόσχῳ σου, καὶ οὐ μὴ κείρῃς τὰ πρωτότοκα τῶν προβάτων σου.
\vs{20}Ἔναντι Κυρίου φαγῇ αὐτὸ ἐνιαυτὸν ἐξ ἐνιαυτοῦ ἐν τῷ τόπῳ ᾧ ἐὰν ἐκλέξηται Κύριος ὁ Θεός σου, σὺ καὶ ὁ οἶκός σου.
\vs{21}Ἐὰν δὲ ἠ· ἐν αὐτῷ μῶμος, χωλὸν ἢ τυφλον, μῶμον πονηρὸν, οὐ θύσεις αὐτὸ Κυρίῳ τῷ Θεῷ σου.

\vs{22}Ἐν ταῖς πόλεσί σου φαγῇ αὐτό· ὁ ἀκάθαρτος ἐν σοι, καὶ ὁ καθαρὸς ὡσαύτως ἔδεται ὡς δορκάδα ἢ ἔλαφον.
\vs{23}Πλὴν αἷμα οὐ φἀγεσθε· ἐπὶ τὴν γῆν ἐκχεεῖς αὐτὸ ὡς ὕδωρ.

\ch{16}
Φυλάξαι τὸν μῆνα τῶν νέων, καὶ ποιήσεις τὸ πάσχα Κυρίῳ τῷ Θεῷ σου, ὅτι ἐν τῷ μηνὶ τῶν νέων ἐξῆλθες ἐξ Αἰγύπτου νυκτός.
\vs{2}Καὶ θύσεις τὸ πάσχα Κυρίῳ τῷ Θεῷ σου πρόβατα καὶ βόας ἐν τῷ τόπῳ, ᾧ ἐὰν ἐκλέξηται Κύριος ὁ Θεός σου αὐτὸν, ἐπικληθῆναι τὸ ὄνομα αὐτοῦ ἐκεῖ.
\vs{3}Οὐ φαγῇ ἐπʼ αὐτοῦ ζύμην· ἐπτὰ ἡμέρας φαγῇ ἐπʼ αὐτοῦ ἄζυμα, ἄρτον κακώσεως, ὅτι ἐν σπουδῇ ἐξήλθετε ἐξ Αἰγύπτου, ἵνα μνησθῆτε τὴν ἡμέραν τῆς ἐξοδίας ὑμῶν ἐκ γῆς Αἰγύπτου πάσας τὰς ἡμέρας τῆς ζωῆς ὑμῶν.
\vs{4}Οὐκ ὀφθήσεται σοι ζύμη ἐν πᾶσι τοῖς ὁρίοις σου ἑπτὰ ἡμέρας, καὶ οὐ κοιμήθησεται ἀπὸ τῶν κρεῶν ὧν ἐὰν θύσῃς τὸ ἑσπέρας τῇ ἡμέρᾳ τῇ πρώτῃ εἰς τοπρωΐ.
\vs{5}Οὐ δυνήσῃ θῦσαι τὸ πάσχα ἐν οὐδεμιᾷ τῶν πόλεών σου, ὧν Κύριος ὁ Θεός σου δίδωσί σοι·
\vs{6}ἀλλʼ ἢ εἰς τὸν τόπον, ὃν ἂν ἐκλέξηται Κύριος ὁ Θεός σου, ἐπικληθῆναι τὸ ὄνομα αὐτοῦ ἐκεῖ, θύσεις τὸ πάσχα ἑσπέρας πρὸς δυσμὰς ἡλίου, ἐν τῷ καιρῷ ᾧ ἐξῆλθες ἐξ Αἰγύπτου.
\vs{7}Καὶ ἑψήσεις καὶ ὀπτήσεις καὶ φαγῇ ἐν τῷ τόπῳ, οὗ ἐὰν ἐκλέξηται Κύριος ὁ Θεός σου αὐτόν· καὶ ἀποστραφήσῃ τοπρωΐ, καὶ ἐλεύσῃ εἰς τοὺς οἴκους σου.
\vs{8}Ἓξ ἡμέρας φαγῇ ἄζυμα, καὶ τῇ ἡμέρᾳ τῇ ἑβδόμῃ ἐξόδιον ἑορτὴ Κυρίῳ τῷ Θεῷ σου· οὐ ποιήσεις ἐν αὐτῇ πᾶν ἔργον, πλὴν ὅσα ποιηθήσεται ψυχῇ.

\vs{9}Ἑπτὰ ἑβδομάδας ἐξαριθμήσεις σεαυτῷ· ἀρξαμένου σου δρέπανον ἐπʼ ἀμητὸν, ἄρξῃ ἐξαριθμῆσαι ἑπτὰ ἑβδομάδας.
\vs{10}Καὶ ποιήσεις ἑορτὴν ἑβδομάδων Κυρίῳ τῷ Θεῷ σου, καθὼς ἡ χείρ σου ἰσχύει, ὅσα ἂν δῷ Κύριος ὁ Θεός σου.

\vs{11}Καὶ εὐφρανθήσῃ ἐναντίον Κυρίου τοῦ Θεοῦ σου, σὺ καὶ ὁ υἱός σου, καὶ ἡ θυγάτηρ σου, ὁ παῖς σου, καὶ ἡ παιδίσκη σου, καὶ ὁ Λευίτης, καὶ ὁ προσήλυτος, καὶ ὁ ὀρφανὸς, καὶ ἡ χήρα ἡ οὖσα ἐν ὑμῖν, ἐν τῷ τόπῳ, ᾧ ἐὰν ἐκλέξηται Κύριος ὁ Θεός σου αὐτὸν, ἐπικληθῆναι τὸ ὄνομα αὐτοῦ ἐκεῖ.

\vs{12}Καὶ μνησθήσῃ ὅτι οἰκέτης ἐγένου ἐν γῇ Αἰγύπτῳ, καὶ φυλάξῃ καὶ ποιήσεις τὰς ἐντολὰς ταύτας.
\vs{13}Ἐορτὴν σκηνῶν ποιήσεις σεαυτῷ ἑπτὰ ἡμέρας ἐν τῷ συναγαγεῖν σε ἐκ τῆς ἅλωνός σου καὶ ἀπὸ τῆς ληνοῦ σου.
\vs{14}Καὶ εὐφρανθήσῃ ἐν τῇ ἑορτῇ σου, σὺ καὶ ὁ υἱός σου, καὶ ἡ θυγάτηρ σου, ὁ παῖς σου, καὶ ἡ παιδίσκη σου, καὶ ὁ Λευίτης, καὶ ὁ προσήλυτος, καὶ ὁ ὀρφανὸς, καὶ ἡ χήρα ἡ οὖσα ἐν ταῖς πόλεσί σου.
\vs{15}Ἑπτὰ ἡμέρας ἑορτάσεις Κυρἱω τῷ Θεῷ σου ἐν τῷ τόπῳ, ᾧ ἂν ἐκλέξηται Κύριος ὁ Θεός σου αὐτῷ· ἐὰν δὲ εὐλογήσῃ σε Κύριος ὁ Θεός σου ἐν πᾶσι γεννήμασί σου, καὶ ἐν πάντι ἔργῳ τῶν χειρῶν σου, καὶ ἔσῃ εὐφραινόμενος.

\vs{16}Τρεῖς καιροὺς τοῦ ἐνιαυτοῦ ὀφθήσεται πᾶν ἀρσενικόν σου ἐναντίον Κυρίου τοῦ Θεοῦ σου ἐν τῷ τόπῳ, ᾧ ἐὰν ἐκλέξηται αὐτὸν Κύριος· ἐν τῇ ἑορτῇ τῶν ἀζύμων, καὶ ἐν τῇ ἑορτῇ τῶν ἑβδομάδων, καὶ ἐν τῇ ἑορτῇ τῆς σκηνοπηγίας· οὐκ ὀφθήσῃ ἐνώπιον Κυρίου τοῦ Θεοῦ σου κενός.
\vs{17}Ἕκαστος κατὰ δύναμιν τῶν χειρῶν ὑμῶν, κατὰ τὴν εὐλογίαν Κυρίου τοῦ Θεοῦ σου ἣν ἔδωκέ σοι.

\vs{18}Κριτὰς καὶ γραμματοεισαγωγεῖς ποιήσεις σεαυτῷ ἐν ταῖς πόλεσί σου, αἷς Κύριος ὁ Θεός σου δίδωσί σοι κατὰ φυλάς· καὶ κρινοῦσι τὸν λαὸν κρίσιν δικαίαν.
\vs{19}Οὐκ ἐκκλινοῦσι κρίσιν, οὐδὲ λήψονται δῶρον· τὰ γὰρ δῶρα ἀποτυφλοῖ ὀφθαλμοὺς σοφῶν, καὶ ἐξαίρει λόγους δικαίων.
\vs{20}Δικαίως τὸ δίκαιον διώξῃ, ἵνα ζῆτε, καὶ εἰσελθόντες κληρονομήσητε τὴν γῆν ἣν Κύριος ὁ Θεός σου δίδωσί σοι.

\vs{21}Οὐ φυτεύσεις σεαυτῷ ἄλσος· πᾶν ξύλον παρὰ τὸ θυσιαστήριον τοῦ Θεοῦ σου οὐ ποιήσεις σεαυτῷ.
\vs{22}Οὐ στήσεις σεαυτῷ στήλην, ἃ ἐμίσησε Κύριος ὁ Θεός σου.

\ch{17}
Οὐ θύσεις Κυρίῳ τῷ Θεῷ σου μόσχον ἢ πρόβατον, ἐν ᾧ ἐστιν ἐν αὐτῷ μῶμος, πᾶν ῥῆμα πονηρόν· ὅτι βδέλυγμα Κυρίῳ τῷ Θεῷ σου ἐστίν.

\vs{2}Ἐὰν δὲ εὑρεθῇ ἐν μιᾷ τῶν πόλεών σου, ὧν Κύριος ὁ Θεός σου δίδωσί σοι, ἀνὴρ ἢ γυνὴ ὃς ποιήσει τὸ πονηρὸν ἐναντίον Κυρίου τοῦ Θεοῦ σου, παρελθεῖν τὴν διαθήκην αὐτοῦ,
\vs{3}καὶ ἐλθόντες λατρεύσωσι θεοῖς ἑτέροις, καὶ προσκυνήσωσιν αὐτοῖς, τῷ ἡλίῳ, ἢ τῇ σελήνῃ, ἢ παντὶ τῶν ἐκ τοῦ κόσμου τοῦ οὐρανοῦ, ἂ οὐ προσέταξέ σοι,
\vs{4}καὶ ἀναγγελῇ σοι καὶ ἐκζητήσῃς σφόδρα, καὶ ἰδοὺ ἀληθῶς γέγονε τὸ ῥῆμα, γεγένηται τὸ βδέλυγμα τοῦτο ἐν Ἰσραήλ·
\vs{5}Καὶ ἐξάξεις τὸν ἄνθρωπον ἐκεῖνον, ἢ τὴν γυναῖκα ἐκείνην, καὶ λιθοβολήσετε αὐτοὺς ἐν λίθοις, καὶ τελευτήσουσιν.
\vs{6}Ἐπὶ δυσὶ μάρτυσιν ἢ ἐπὶ τρισὶ μάρτυσιν ἀποθανεῖται· ὁ ἀποθνήσκων οὐκ ἀποθανεῖται ἐφʼ ἑνὶ μάρτυρι.
\vs{7}Καὶ ἡ χεὶρ τῶν μαρτύρων ἔσται ἐπʼ αὐτῷ ἐν πρώτοις θανατῶσαι αὐτὸν, καὶ ἡ χεὶρ τοῦ λαοῦ ἐπʼ ἐσχάτων· καὶ ἐξαρεῖς τὸν πονηρὸν ἐξ ὑμῶν αὐτῶν.

\vs{8}Ἐὰν δὲ ἀδυνατήσῃ ἀπὸ σοῦ ῥῆμα ἐν κρίσει ἀναμέσον αἷμα αἵματος, καὶ ἀναμέσον κρίσις κρίσεως, καὶ ἀναμέσον ἁφὴ ἁφῆς, καὶ ἀναμέσον ἀντιλογία ἀντιλογίας, ῥήματα κρίσεως ἐν ταῖς πόλεσιν ὑμῶν, καὶ ἀναστὰς ἀναβήσῃ εἰς τὸν τόπον ὃν ἂν ἐκλέξηται Κύριος ὁ Θεός σου ἐκεῖ,
\vs{9}καὶ ἐγεύσῃ πρὸς τὸνς ἱερεῖς τοὺς Λευίτας, καὶ πρὸς τὸν κριτὴν ὃς ἂν γένηται ἐν ταῖς ἡμέραις ἐκείναις, καὶ ἐκζητήσαντες ἀναγγελοῦσί σοι τὴν κρίσιν.
\vs{10}Καὶ ποιήσεις κατὰ τὸ πρᾶγμα ὃ ἂν ἀναγγείλωσί σοι ἐκ τοῦ τόπου, οὗ ἐὰν ἐκλέξηται Κύριος ὁ Θεός σου, καὶ φυλάξῃ ποιῆσαι πάντα ὅσα ἂν νομοθετηθῇ σοι.
\vs{11}Κατὰ τὸν νόμον καὶ κατὰ τὴν κρίσιν ἣν ἂν εἴπωσί σοι, ποιήσεις· οὐκ ἐκκλινεῖς ἀπὸ τοῦ ῥήματος οὗ ἐὰν ἀναγγείλωσί σοι δεξιὰ οὐδὲ ἀριστερά.

\vs{12}Καὶ ὁ ἄνθρωπος ὃς ἐὰν ποιήσῃ ἐν ὑπερηφανίᾳ, ὥστε μὴ ὑπακοῦσαι τοῦ ἱερέως τοῦ παρεστηκότος λειτουργεῖν ἐπὶ τῷ ὀνόματι Κυρίου τοῦ Θεοῦ σου, ἢ τοῦ κριτοῦ ὃς ἂν ἠ· ἐν ταῖς ἡμέραις ἐκείναις, καὶ ἀποθανεῖται ὁ ἄνθρωπος ἐκεῖνος, καὶ ἐξαρεῖς τὸν πονηρὸν ἐξ Ἰσραήλ.
\vs{13}Καὶ πᾶς ὁ λαὸς ἀκούσας φοβηθήσεται, καὶ οὐκ ἀσεβήσει ἔτι.

\vs{14}Ἐὰν δὲ εἰσέλθῃς εἰς τὴν γῆν ἣν Κύριος ὁ Θεός σου δίδωσί σοι, καὶ κληρονομήσῃς αὐτὴν, καὶ κατοικήσῃς ἐπʼ αὐτὴν, καὶ εἴπῃς, καταστήσω ἐπʼ ἐμαυτὸν ἄρχοντα, καθὰ καὶ τὰ λοιπὰ ἔθνη τὰ κύκλῳ μου·
\vs{15}καθιστῶν καταστήσεις ἐπὶ σεαυτὸν ἄρχοντα, ὃν ἂν ἐκλέξηται Κύριος ὁ Θεὸς αὐτόν· ἐκ τῶν ἀδελφῶν σου καταστήσεις ἐπὶ σεαυτὸν ἄρχοντα· οὐ δυνήσῃ καταστῆσαι ἐπὶ σεαυτὸν ἄνθρωπον ἀλλότριον, ὅτι οὐκ ἀδελφός σου ἐστί.
\vs{16}Διότι οὐ πληθυνεῖ ἑαυτῷ ἵππον, οὐδὲ μὴ ἀποστρέψῃ τὸν λαὸν εἰς Αἴγυπτον, ὅπως μὴ πληθύνῃ αὑτῷ ἵππον· ὁ δὲ Κύριος εἶπεν, οὐ προσθήσεσθε ἀποστρέψαι τῇ ὁδῷ ταύτῃ ἔτι.
\vs{17}Καὶ οὐ πληθυνεῖ ἑαυτῷ γυναῖκας, ἵνα μὴ μεταστῇ αὐτοῦ καρδία· καὶ ἀργύριον καὶ χρυσίον οὐ πληθυνεῖ ἑαυτῷ σφόδρα.

\vs{18}Καὶ ὅταν καθίσῃ ἐπὶ τῆς ἀρχῆς αὐτοῦ, καὶ γράψει αὑτῷ τὸ δευτερονόμιον τοῦτο εἰς βιβλίον παρὰ τῶν ἱερέων τῶν Λευιτῶν,
\vs{19}καὶ ἔσται μετʼ αὐτοῦ, καὶ ἀναγνώσεται ἐν αὐτῷ πάσας τὰς ἡμέρας τῆς ζωῆς αὐτοῦ, ἵνα μάθῃ φοβεῖσθαι Κύριον τὸν Θεόν σου, καὶ φυλάσσεσθαι πάσας τὰς ἐντολὰς ταύτας, καὶ τὰ δικαιώματα ταῦτα ποιεῖν·
\vs{20}ἵνα μὴ ὑψωθῇ ἡ καρδία αὐτοῦ ἀπὸ τῶν ἀδελφῶν αὐτοῦ, ἵνα μὴ παραβῇ ἀπὸ τῶν ἐντολῶν δεξιὰ ἢ ἀριστερὰ, ὅπως ἂν μακροχρονίσῃ ἐπὶ τῆς ἀρχῆς αὐτοῦ, αὐτὸς καὶ οἱ υἱοὶ αὐτοῦ ἐν τοῖς υἱοῖς Ἰσραήλ.

\ch{18}
Οὐκ ἔσται τοῖς ἱερεῦσι τοῖς Λευίταις ὅλῃ φυλῇ Λευὶ μερὶς οὐδὲ κλῆρος μετὰ Ἰσραήλ· καρπώματα Κυρίου ὁ κλῆρος αὐτῶν, φάγονται αὐτά.
\vs{2}Κλῆρος δὲ οὐκ ἔσται αὐτοῖς ἐν τοῖς ἀδελφοῖς αὐτῶν· Κύριος αὐτὸς κλῆρος αὐτοῦ, καθότι εἶπεν αὐτῷ.
\vs{3}Καὶ αὕτη ἡ κρίσις τῶν ἱερέων τὰ παρὰ τοῦ λαοῦ παρὰ τῶν θυόντων τὰ θύματα, ἐάν τε μόσχον, ἐάν τε πρόβατον· καὶ δώσεις τὸν βραχίονα τῷ ἱερεῖ, καὶ τὰ σιαγόνια, καὶ τὸ ἔνυστρον,
\vs{4}καὶ τὰς ἀπαρχὰς τοῦ σίτου σου, καὶ τοῦ οἴνου σου, καὶ τοῦ ἐλαίου σου· καὶ τὴν ἀπαρχὴν τῶν κουρῶν τῶν προβάτων σου δώσεις αὐτῷ.
\vs{5}Ὅτι αὐτὸν ἐξελέξατο Κύριος ἐκ πασῶν τῶν φυλῶν σου, παρεστάναι ἔναντι Κυρίου τοῦ Θεοῦ, λειτουργεῖν καὶ εὐλογεῖν ἐπὶ τῷ ὀνόματι αὐτοῦ, αὐτὸς καὶ οἱ υἱοὶ αὐτοῦ ἐν τοῖς υἱοῖς Ἰσραήλ.

\vs{6}Ἐὰν δὲ παραγένηται ὁ Λευίτης ἐκ μιᾶς τῶν πόλεων ἐκ πάντων τῶν υἱῶν Ἰσραὴλ, οὗ αὐτὸς παροικεῖ, καθʼ ὅτι ἐπιθυμεῖ ἡ ψυχὴ αὐτοῦ, εἰς τὸν τόπον ὃν ἂν ἐκλέξηται,
\vs{7}λειτουργήσει τῷ ὀνόματι Κυρίου τοῦ Θεοῦ αὐτοῦ, ὥσπερ πάντες οἱ ἀδελφοὶ αὐτοῦ οἱ Λευῖται οἱ παρεστηκότες ἐκεῖ ἐναντίον Κυρίου τοῦ Θεοῦ σου.
\vs{8}Μερίδα μεμερισμένην φάγεται, πλὴν τῆς πράσεως τῆς κατὰ πατριάν.
\vs{9}Ἐὰν δὲ εἰσέλθῃς εἰς τὴν γῆν ἣν Κύριος ὁ Θεός σου δίδωσί σοι, οὐ μαθήσῃ ποιεῖν κατὰ τὰ βδελύγματα τῶν ἐθνῶν ἐκείνων.

\vs{10}Οὐχ εὑρεθήσεται ἐν σοὶ περικαθαίρων τὸν υἱὸν αὐτοῦ καὶ τὴν θυγατέρα αὐτοῦ ἐν πυρὶ, μαντευόμενος μαντείαν, κληδονιζόμενος, καὶ οἰωνιζόμενος, φαρμακοῖς
\vs{11}ἐπαείδων ἐπαοιδὴν, ἐγγαστρίμυθος, καὶ τερατοσκόπος, ἐπερωτῶν τοὺς νεκρούς.
\vs{12}Ἔστι γὰρ βδέλυγμα Κυρίῳ τῷ Θεῷ σου πᾶς ποιῶν ταῦτα· ἕνεκεν γὰρ τῶν βδελυγμάτων τούτων Κύριος ἐξολοθρεύσει αὐτοὺς ἀπὸ προσώπου σου.
\vs{13}Τέλειος ἔσῃ ἐναντίον Κυρίου τοῦ Θεοῦ σου.
\vs{14}Τὰ γὰρ ἔθνη ταῦτα, οὓς σὺ κατακληρονομεῖς αὐτοὺς, οὗτοι κληδόνων καὶ μαντειῶν ἀκούσονται· καί σοι οὐχ οὕτως ἔδωκε Κύριος ὁ Θεός σου.

\vs{15}Προφήτην ἐκ τῶν ἀδελφῶν σου, ὡς ἐμὲ, ἀναστήσει σοι Κύριος ὁ Θεός σου· αὐτοῦ ἀκούσεσθε·
\vs{16}Κατὰ πάντα ὅσα ᾐτήσω παρὰ Κυρίου τοῦ Θεοῦ σου ἐν Χωρὴβ τῇ ἡμέρᾳ τῆς ἐκκλησίας, λέγοντες, οὐ προσθήσομεν ἀκοῦσαι τὴν φωνὴν Κυρίου τοῦ Θεοῦ σου, καὶ τὸ πῦρ τοῦτο τὸ μὲγα οὐκ ὀψόμεθα ἔτι, οὐδὲ μὴ ἀποθάνωμεν.
\vs{17}Καὶ εἶπε Κύριος πρὸς μὲ, ὀρθῶς πάντα ὅσα ἐλάλησαν πρὸς σέ.
\vs{18}Προφήτην ἀναστήσω αὐτοῖς ἐκ τῶν ἀδελφῶν αὐτῶν, ὥσπερ σέ· καὶ δώσω τὰ ῥήματα ἐν τῷ στόματι αὐτοῦ, καὶ λαλήσει αὐτοῖς καθʼ ὅτι ἂν ἐντείλωμαι αὐτῷ.
\vs{19}Καὶ ὁ ἄνθρωπος ὃς ἐὰν μὴ ἀκούσῃ ὅσα ἂν λαλήσῃ ὁ προφήτης ἐκεῖνος ἐπὶ τῷ ὀνόματί μου, ἐγὼ ἐκδικήσω ἐξ αὐτοῦ.
\vs{20}Πλὴν ὁ προφήτης ὃς ἂν ἀσεβήσῃ λαλῆσαι ἐπὶ τῷ ὀνόματί μου ῥῆμα ὃ οὐ προσέταξα λαλῆσαι, καὶ ὃς ἂν λαλήσῃ ἐν ὀνόματι θεῶν ἑτέρων, ἀποθανεῖται ὁ προφήτης ἐκεῖνος.
\vs{21}Ἐὰν δὲ εἴπῃς ἐν τῇ καρδίᾳ σου, πῶς γνωσόμεθα τὸ ῥῆμα ὃ οὐκ ἐλάλησε Κύριος;
\vs{22}Ὅσα ἐὰν λαλήσῃ ὁ προφήτης ἐκεῖνος τῷ ὀνόματι Κυρίου, καὶ μὴ γένηται, καὶ μὴ συμβῇ, τοῦτο τὸ ῥῆμα ὃ οὐκ ἐλάλησε Κύριος, ἐν ἀσεβείᾳ ἐλάλησεν ὁ προφήτης ἐκεῖνος· οὐκ ἀφέξεσθε αὐτοῦ.

\ch{19}
Ἐὰν δὲ ἀφανίσῃ Κύριος ὁ Θεός σου τὰ ἔθνη, ἃ ὁ Θεὸς δίδωσί σοι τὴν γῆν, καὶ κατακληρονομήσητε αὐτοὺς, καὶ κατοικήσητε ἐν ταῖς πόλεσιν αὐτῶν, καὶ ἐν τοῖς οἴκοις αὐτῶν,
\vs{2}τρεῖς πόλεις διαστελεῖς σεαυτῷ ἐν μέσῳ τῆς γῆς σου, ἧς Κύριος ὁ Θεός σου δίδωσί σοι.
\vs{3}Στόχασαί σοι τὴν ὁδὸν, καὶ τριμεριεῖς τὰ ὅρια τῆς γῆς σου, ἣν καταμερίζει σοι Κύριος ὁ Θεός σου, καὶ ἔσται καταφυγὴ παντὶ φονευτῇ.
\vs{4}Τοῦτο δὲ ἔσται τὸ πρόσταγμα τοῦ φονευτοῦ, ὃς ἂν φύγῃ ἐκεῖ, καὶ ζήσεται, ὃς ἂν πατάξῃ τὸν πλησίον αὐτοῦ οὐκ εἰδῶς, καὶ οὗτος οὐ μισῶν αὐτὸν πρὸ τῆς χθὲς καὶ τρίτης.
\vs{5}Καὶ ὅς ἐὰν εἰσέλθῃ μετὰ τοῦ πλησίον εἰς τὸν δρυμὸν συναγαγεῖν ξύλα, καὶ ἐκκρουσθῇ ἡ χεὶρ αὐτοῦ τῇ ἀξίνῃ κόπτοντος τὸ ξύλον, καὶ ἐκπεσὸν τὸ σιδήριον ἀπὸ τοῦ ξύλου τύχῃ τοῦ πλησίον, καὶ ἀποθάνῃ, οὗτος καταφεύξεται εἰς μίαν τῶν πόλεων τούτων, καὶ ζήσεται.
\vs{6}Ἵνα μὴ διώξας ὁ ἀγχιστεύων τοῦ αἵματος ὀπίσω τοῦ φονεύσαντος, ὅτι παρατεθέρμανται τῇ καρδίᾳ, καὶ καταλάβῃ αὐτὸν, ἐὰν μακροτέρα ἠ· ἡ ὁδὸς, καὶ πατάξῃ αὐτοῦ ψυχήν· καὶ τούτῳ οὐκ ἔστι κρίσις θανάτου, ὅτι οὐ μισῶν ἦν αὐτὸν πρὸ τῆς χθὲς, οὐδὲ πρὸ τῆς τρίτης.
\vs{7}Διὰ τοῦτο ἐγώ σοι ἐντέλλομαι τὸ ῥῆμα τοῦτο, λέγων, τρεῖς πόλεις διαστελεῖς σεαυτῷ.

\vs{8}Ἐὰν δὲ ἐμπλατύνῃ Κύριος ὁ Θεός σου τὰ ὅριά σου, ὃν τρόπον ὤμοσε τοῖς πατράσι σου, καὶ δῷ σοι Κύριος πᾶσαν τὴν γῆν, ἣν εἶπε δοῦναι τοῖς πατράσι σου,
\vs{9}ἐὰν ἀκούσῃς ποιεῖν πάσας τὰς ἐντολὰς ταύτας, ἃς ἐγὼ ἐντέλλομαί σοι σήμερον, ἀγαπᾷν Κύριον τὸν Θεόν σου, πορεύεσθαι ἐν πάσαις ταῖς ὁδοῖς αὐτοῦ πάσας τὰς ἡμέρας· προσθήσεις σεαυτῷ ἔτι τρεῖς πόλεις πρὸς τὰς τρεῖς ταύτας.
\vs{10}Καὶ οὐκ ἐκχυθήσεται αἷμα ἀναίτιον ἐν τῇ γῇ, ἡ· Κύριος ὁ Θεός σου δίδωσί σοι ἐν κλήρῳ, καὶ οὐκ ἔσται ἐν σοὶ αἵματι ἔνοχος.

\vs{11}Ἐὰν δὲ γένηται ἐν σοὶ ἄνθρωπος μισῶν τὸν πλησίον, καὶ ἐνεδρεύσῃ αὐτὸν, καὶ ἐπαναστῇ ἐπʼ αὐτὸν, καὶ πατάξῃ αὐτοῦ ψυχὴν, καὶ ἀποθάνῃ, καὶ φύγῃ εἰς μίαν τῶν πόλεων τούτων·
\vs{12}καὶ ἀποστελοῦσιν ἡ γερουσία τῆς πόλεως αὐτοῦ, καὶ λήμψονται αὐτὸν ἐκεῖθεν, καὶ παραδώσουσιν αὐτὸν εἰς χεῖρας τῶν ἀγχιστευόντων τοῦ αἵματος, καὶ ἀποθανεῖται.
\vs{13}Οὐ φείσεται ὁ ὀφθαλμός σου ἐπʼ αὐτῷ, καὶ καθαριεῖς τὸ αἷμα τὸ ἀναίτιον ἐξ Ἰσραὴλ, καὶ εὖ σοι ἔσται.

\vs{14}Οὐ μετακινήσεις ὅρια τοῦ πλησίον, ἃ ἔστησαν οἱ πατέρες σου ἐν τῇ κληρονομίᾳ, ᾗ κατεκληρονομήθης ἐν τῇ γῇ, ἣν Κύριος ὁ Θεός σου δίδωσί σοι ἐν κλήρῳ.
\vs{15}Οὐκ ἐμμενεῖ μάρτυς εἷς μαρτυρῆσαι κατὰ ἀνθρώπου κατὰ πᾶσαν ἀδικίαν, καὶ κατὰ πᾶν ἁμάρτημα, καὶ κατὰ πᾶσαν ἁμαρτίαν, ἣν ἐὰν ἁμάρτῃ· ἐπὶ στόματος δύο μαρτύρων, καὶ ἐπὶ στόματος τριῶν μαρτύρων, στήσεται πᾶν ῥῆμα.
\vs{16}Ἐὰν δὲ καταστῇ μάρτυς ἄδικος κατὰ ἀνθρώπου, καταλέγων αὐτοῦ ἀσέβειαν·
\vs{17}καὶ στήσονται οἱ δύο ἄνθρωποι οἷς ἐστιν αὐτοῖς ἡ ἀντιλογία, ἔναντι Κυρίου, καὶ ἔναντι τῶν ἱερέων, καὶ ἔναντι τῶν κριτῶν, οἳ ἂν ὦσιν ἐν ταῖς ἡμέραις ἐκείναις·
\vs{18}Καὶ ἐξετάσωσιν οἱ κριταὶ ἀκριβῶς, καὶ ἰδοὺ μάρτυς ἄδικος ἐμαρτύρησεν ἄδικα, ἀντέστη κατὰ τοῦ ἀδελφοῦ αὐτοῦ·
\vs{19}Καὶ ποιήσετε αὐτῷ ὃν τρόπον ἐπονηρεύσατο ποιῆσαι κατὰ τοῦ ἀδελφοῦ αὐτοῦ, καὶ ἐξαρεῖς τὸ πονηρὸν ἐξ ὑμῶν αὐτῶν.
\vs{20}Καὶ οἱ ἐπίλοιποι ἀκούσαντες φοβηθήσονται, καὶ οὐ προσθήσουσιν ἔτι ποιῆσαι κατὰ τὸ ῥῆμα τὸ πονηρὸν τοῦτο ἐν ὑμῖν.
\vs{21}Οὐ φείσεται ὁ ὀφθαλμός σου ἐπʼ αὐτῷ· ψυχὴν ἀντὶ ψυχῆς, ὀφθαλμὸν ἀντὶ ὀφθαλμοῦ, ὀδόντα ἀντὶ ὀδόντος, χεῖρα ἀντὶ χειρὸς, πόδα ἀντὶ ποδός.

\ch{20}
Ἐὰν δὲ ἐξέλθῃς εἰς πόλεμον ἐπὶ τοὺς ἐχθρούς σου, καὶ ἴδῃς ἵππον καὶ ἀναβάτην καὶ λαὸν πλείονά σου, οὐ φοβηθήσῃ ἀπʼ αὐτῶν, ὅτι Κύριος ὁ Θεός σου μετὰ σοῦ, ὁ ἀναβιβάσας σε ἐκ γῆς Αἰγύπτου.
\vs{2}Καὶ ἔσται ὅταν ἐγγίσῃς τῷ πολέμῳ, καὶ προσεγγίσας ὁ ἱερεὺς λαλήσει τῷ λαῷ,
\vs{3}καὶ ἐρεῖ πρὸς αὐτοὺς, ἄκουε Ἰσραήλ· ὑμεῖς πορεύεσθε σήμερον εἰς τὸν πόλεμον ἐπὶ τοὺς ἐχθροὺς ὑμῶν· μὴ ἐκλυέσθω ἡ καρδία ὑμῶν, μὴ φοβεῖσθε, μηδὲ θραύεσθε, μηδὲ ἐκκλίνετε ἀπὸ προσώπου αὐτῶν.
\vs{4}Ὅτι Κύριος ὁ Θεὸς ὑμῶν ὁ προπορευόμενος μεθʼ ὑμῶν, συνεκπολεμῆσαι ὑμῖν τοὺς ἐχθροὺς ὑμῶν διασῶσαι ὑμᾶς.

\vs{5}Καὶ λαλήσουσιν οἱ γραμματεῖς πρὸς τὸν λαὸν, λέγοντες, τίς ὁ ἄνθρωπος ὁ οἰκοδομήσας οἰκίαν καινὴν, καὶ οὐκ ἐνεκαίνισεν αὐτήν; πορευέσθω καὶ ἀποστραφήτω εἰς τὴν· οἰκίαν αὐτοῦ, μὴ ἀποθάνῃ ἐν τῷ πολέμῳ, καὶ ἄνθρωπος ἕτερος ἐγκαινιεῖ αὐτήν.
\vs{6}Καὶ τίς ὁ ἄνθρωπος ὅστις ἐφύτευσεν ἀμπελῶνα, καὶ οὐκ εὐφράνθη ἐξ αὐτοῦ; πορευέσθω καὶ ἀποστραφήτω εἰς τὴν οἰκίαν αὐτοῦ, μὴ ἀποθάνῃ ἐν τῷ πολέμῳ, καὶ ἄνθρωπος ἕτερος εὐφρανθήσεται ἐξ αὐτοῦ.
\vs{7}Καὶ τίς ὁ ἄνθρωπος ὅστις μεμνήστευται γυναῖκα, καὶ οὐκ ἔλαβεν αὐτήν; πορευέσθω καὶ ἀποστραφήτω εἰς τὴν οἰκίαν αὐτοῦ, μὴ ἀποθάνῃ ἐν τῷ πολέμῳ, καὶ ἄνθρωπος ἕτερος λήψεται αὐτήν.
\vs{8}Καὶ προσθήσουσιν οἱ γραμματεῖς λαλῆσαι πρὸς τὸν λαὸν, καὶ ἐροῦσι, τίς ὁ ἄνθρωπος ὁ φοβούμενος καὶ δειλὸς τῇ καρδίᾳ; πορευέσθω καὶ ἀποστραφήτω εἰς τὴν οἰκίαν αὐτοῦ, ἵνα μὴ δειλιάνῃ τὴν καρδίαν τοῦ ἀδελφοῦ αὐτοῦ, ὥσπερ ἡ αὐτοῦ.
\vs{9}Καὶ ἔσται ὅταν παύσωνται οἱ γραμματεῖς λαλοῦντες πρὸς τὸν λαὸν, καὶ καταστήσουσιν ἄρχοντας τῆς στρατιᾶς προηγουμένους τοῦ λαοῦ.

\vs{10}Ἐὰν δὲ προσέλθῃς πρὸς πόλιν ἐκπολεμῆσαι αὐτοὺς, καὶ ἐκκαλέσαι αὐτοὺς μετʼ εἰρήνης.
\vs{11}Ἐὰν μὲν εἰρηνικὰ ἀποκριθῶσί σοι, καὶ ἀνοίξωσί σοι, ἔσται πᾶς ὁ λαὸς οἱ εὑρεθέντες ἐν αὐτῇ ἔσονταί σοι φορολόγητοι καὶ ὑπήκοοί σου.
\vs{12}Ἐὰν δὲ μὴ ὑπακούσωσί σοι, καὶ ποιῶσι πρὸς σὲ πόλεμον, περικαθιεῖς αὐτὴν,
\vs{13}ἕως ἂν παραδῷ σοι αὐτὴν Κύριος ὁ Θεός σου εἰς τὰς χεῖράς σου, καὶ πατάξεις πᾶν ἀρσενικὸν αὐτῆς ἐν φόνῳ μαχαίρας,
\vs{14}πλὴν τῶν γυναικῶν καὶ τῆς ἀποσκευῆς· καὶ πάντα τὰ κτήνη, καὶ πάντα ὅσα ἂν ὑπάρχῃ ἐν τῇ πόλει, καὶ πᾶσαν τὴν ἀπαρτίαν προνομεύσεις σεαυτῷ, καὶ φαγῇ πᾶσαν τὴν προνομὴν τῶν ἐχθρῶν σου, ὧν Κύριος ὁ Θεός σου δίδωσί σοι.
\vs{15}Οὕτως ποιήσεις πάσας τὰς πόλεις τὰς μακρὰν οὔσας σου σφόδρα, οὐχὶ ἐκ τῶν πόλεων τῶν ἐθνῶν τούτων,
\vs{16}ὧν Κύριος ὁ Θεός σου δίδωσί σοι κληρονομεῖν τὴν γῆν αὐτῶν. Οὐ ζωγρήσετε πᾶν ἐμπνέον,
\vs{17}ἀλλʼ ἢ ἀναθέματι ἀναθεματιεῖτε αὐτοὺς, τὸν Χετταῖον, καὶ Ἀμοῤῥαῖον, καὶ Χαναναῖον, καὶ Φερεζαῖον, καὶ Εὐαῖον, καὶ Ἰεβουσαῖον, καὶ Γεργεσαῖον, ὃν τρόπον ἐνετείλατό σοι Κύριος ὁ Θεός σου,
\vs{18}ἵνα μὴ διδάξωσι ποιεῖν ὑμᾶς πάντα τὰ βδελύγματα αὐτῶν, ὅσα ἐποίησαν τοῖς θεοῖς αὐτῶν, καὶ ἁμαρτήσεσθε ἐναντίον Κυρίου τοῦ Θεοῦ ὑμῶν.

\vs{19}Ἐὰν δὲ περικαθίσῃς περὶ πόλιν μίαν ἡμέρας πλείους ἐκπολεμῆσαι αὐτὴν εἰς κατάληψιν αὐτῆς, οὐκ ἐξολεθρεύσεις τὰ δένδρα αὐτῆς, ἐπιβαλεῖν ἐπʼ αὐτὰ σίδηρον, ἀλλʼ ἢ ἀπʼ αὐτοῦ φαγῇ, αὐτὸ δὲ οὐκ ἐκκόψεις· μὴ ἄνθρωπος τὸ ξύλον τὸ ἐν τῷ ἀγρῷ, εἰσελθεῖν ἀπὸ προσώπου σου εἰς τὸν χάρακα;
\vs{20}Ἀλλὰ ξύλον ὃ ἐπίστασαι ὅτι οὐ καρπόβρωτόν ἐστι, τοῦτο ὀλοθρεύσεις καὶ ἐκκόψεις· καὶ οἰκοδομήσεις χαράκωσιν ἐπὶ τὴν πόλιν, ἥτις ποιεῖ πρὸς σὲ τὸν πόλεμον, ἕως ἂν παραδοθῇ.

\ch{21}
Ἐὰν δὲ εὑρεθῇ τραυματίας ἐν τῇ γῇ, ᾗ Κύριος ὁ Θεός σου δίδωσί σοι κληρονομῆσαι, πεπτωκὼς ἐν τῷ πεδίῳ, καὶ οὐκ οἴδασι τὸν πατάξαντα,
\vs{2}ἐξελεύσεται ἡ γερουσία σου καὶ οἱ κριταί σου, καὶ ἐκμετρήσουσιν ἐπὶ τὰς πόλεις τὰς κύκλῳ τοῦ τραυματίου·
\vs{3}Καὶ ἔσται ἡ πόλις ἡ ἐγγίζουσα τῷ τραυματίᾳ, καὶ λήψεται ἡ γερουσία τῆς πόλεως ἐκείνης δάμαλιν ἐκ βοῶν, ἥτις οὐκ εἴργασται, καὶ ἥτις οὐχ εἵλκυσε ζυγόν·
\vs{4}Καὶ καταβιβάσουσιν ἡ γερουσία τῆς πόλεως ἐκείνης δάμαλιν εἰς φάραγγα τραχεῖαν, ἥτις οὐκ εἴργασται οὐδὲ σπείρεται, καὶ νευροκοπήσουσι τὴν δάμαλιν ἐν τῇ φάραγγι.
\vs{5}Καὶ προσελεύσονται οἱ ἱερεῖς οἱ Λευῖται, ὅτι αὐτοὺς ἐπέλεξε Κύριος ὁ Θεὸς παρεστηκέναι αὐτῷ, καὶ εὐλογεῖν ἐπὶ τῷ ὀνόματι αὐτοῦ· καὶ ἐπὶ τῷ στόματι αὐτῶν ἔσται πᾶσα ἀντιλογία, καὶ πᾶσα ἁφή.
\vs{6}Καὶ πᾶσα ἡ γερουσία τῆς πόλεως ἐκείνης οἱ ἐγγίζοντες τῷ τραυματίᾳ νίψονται τὰς χεῖρας ἐπὶ τὴν κεφαλὴν τῆς δαμάλεως τῆς νενευροκοπημένης ἐν τῇ φάραγγι·
\vs{7}καὶ ἀποκριθέντες, ἐροῦσιν, αἱ χεῖρες ἡμῶν οὐκ ἐξέχεαν τὸ αἷμα τοῦτο, καὶ οἱ ὀφθαλμοὶ ἡμῶν οὐκ ἑωράκασιν.
\vs{8}Ἵλεως γένου τῷ λαῷ σου Ἰσραὴλ, οὓς ἐλυτρώσω Κύριε, ἵνα μὴ γένηται αἷμα ἀναίτιον ἐν τῷ λαῷ σου Ἰσραήλ· καὶ ἐξιλασθήσεται αὐτοῖς τὸ αἷμα.
\vs{9}Σὺ δὲ ἐξαρεῖς τὸ αἷμα τὸ ἀναίτιον ἐξ ὑμῶν αὐτῶν, ἐὰν ποιήσῃς τὸ καλὸν καὶ τὸ ἀρεστὸν ἔναντι Κυρίου τοῦ Θεοῦ σου.

\vs{10}Ἐὰν δὲ ἐξελθὼν εἰς πόλεμον ἐπὶ τοὺς ἐχθρούς σου, καὶ παραδῷ σοι Κύριος ὁ Θεός σου εἰς τὰς χεῖράς σου, καὶ προνομεύσῃς τὴν προνομὴν αὐτῶν,
\vs{11}καὶ ἴδῃς ἐν τῇ προνομῇ γυναῖκα καλὴν τῷ εἴδει, καὶ ἐνθυμηθῇς αὐτῆς, καὶ λάβῃς αὐτὴν σεαυτῷ γυναῖκα,
\vs{12}καὶ εἰσάξῃς αὐτὴν ἔνδον εἰς τὴν οἰκίαν σου, καὶ ξυρήσεις τὴν κεφαλὴν αὐτῆς, καὶ περιονυχιεῖς αὐτὴν,
\vs{13}καὶ περιελεῖς τὰ ἱμάτια τῆς αἰχμαλωσίας ἀπʼ αὐτῆς, καὶ καθιεῖται ἐν τῇ οἰκίᾳ σου, καὶ κλαύσεται τὸν πατέρα καὶ τὴν μητέρα μηνὸς ἡμέρας· καὶ μετὰ ταῦτα εἰσελεύσῃ πρὸς αὐτὴν καὶ συνοικισθήσῃ αὐτῇ, καὶ ἔσται σου γυνή.

\vs{14}Καὶ ἔσται ἐὰν μὴ θέλῃς αὐτὴν, ἐξαποστελεῖς αὐτὴν ἐλευθέραν, καὶ πράσει οὐ πραθήσεται ἀργυρίου· οὐκ ἀθετήσεις αὐτὴν, διότι ἐταπείνωσας αὐτήν.

\vs{15}Ἐὰν δὲ γένωνται ἀνθρώπῳ δύο γυναῖκες, μία αὐτῶν ἠγαπημένη, καὶ μία αὐτῶν μισουμένη, καὶ τέκωσιν αὐτῷ ἡ ἠγαπημένη καὶ ἡ μισουμένη, καὶ γένηται υἱὸς πρωτότοκος τῆς μισουμένης·
\vs{16}Καὶ ἔσται ἡ· ἂν ἡμέρᾳ κατακληρονομῇ τοῖς υἱοῖς αὐτοῦ τὰ ὑπάρχοντα αὐτοῦ, οὐ δυνήσεται πρωτοτοκεῦσαι τῷ υἱῷ τῆς ἠγαπημένης, ὑπεριδὼν τὸν υἱὸν τῆς μισουμένης τὸν πρωτότοκον·
\vs{17}Ἀλλὰ τὸν πρωτότοκον υἱὸν τῆς μισουμένης ἐπιγνώσεται δοῦναι αὐτῷ διπλᾶ ἀπὸ πάντων ὧν ἂν εὑρεθῇ αὐτῷ, ὅτι οὗτός ἐστιν ἀρχὴ τέκνων αὐτοῦ, καὶ τούτῳ καθήκει τὰ πρωτοτοκεῖα.
\vs{18}Ἐὰν δέ τινι ἠ· υἱὸς ἀπειθῆς καὶ ἐρεθιστὴς οὐχ ὑπακούων φωνὴν πατρὸς καὶ φωνὴν μητρὸς, καὶ παιδεύωσιν αὐτὸν, καὶ μὴ εἰσακούῃ αὐτῶν·
\vs{19}Καὶ συλλαβόντες αὐτὸν ὁ πατὴρ αὐτοῦ καὶ ἡ μήτηρ αὐτοῦ, καὶ ἐξάξουσιν αὐτὸν ἐπὶ τὴν γερουσίαν τῆς πόλεως αὐτοῦ, καὶ ἐπὶ τὴν πύλην τοῦ τόπου·
\vs{20}Καὶ ἐροῦσι τοῖς ἀνδράσι τῆς πόλεως αὐτῶν, ὁ υἱὸς ἡμῶν οὗτος ἀπειθεῖ καὶ ἐρεθίζει, οὐχ ὑπακούει τῆς φωνῆς ἡμῶν, συμβολοκοπῶν οἰνοφλυγεῖ.
\vs{21}Καὶ λιθοβολήσουσιν αὐτὸν οἱ ἄνδρες τῆς πόλεως αὐτοῦ ἐν λίθοις, καὶ ἀποθανεῖται· καὶ ἐξαρεῖς τὸν πονηρὸν ἐξ ὑμῶν αὐτῶν· καὶ οἱ ἐπίλοιποι ἀκούσαντες φοβηθήσονται.

\vs{22}Ἐὰν δὲ γένηται ἔν τινι ἁμαρτία, κρίμα θανάτου, καὶ ἀποθάνῃ, καὶ κρεμάσητε αὐτὸν ἐπὶ ξύλου·
\vs{23}οὐ κοιμηθήσεται τὸ σῶμα αὐτοῦ ἐπὶ τοῦ ξύλου, ἀλλὰ ταφῇ θάψετε αὐτὸ ἐν τῇ ἡμέρᾳ ἐκείνῃ, ὅτι κεκατηραμένος ὑπὸ Θεοῦ πᾶς κρεμάμενος ἐπὶ ξύλου· καὶ οὐ μὴ μιανεῖτε τὴν γῆν, ἣν Κύριος ὁ Θεός σου δίδωσί σοι ἐν κλήρῳ.

\ch{22}
Μὴ ἰδὼν τὸν μόσχον τοῦ ἀδελφοῦ σου, ἢ τὸ πρόβατον αὐτοῦ, πλανώμενα ἐν τῇ ὁδῷ, ὑπερίδῃς αὐτά· ἀποστροφῇ ἀποστρέψεις αὐτὰ τῷ ἀδελφῷ σου, καὶ ἀποδώσεις αὐτῷ.
\vs{2}Ἐὰν δὲ μὴ ἐγγίζῃ ὁ ἀδελφός σου πρὸς σὲ, μηδὲ ἐπίοτῃ αὐτὸν, συνάξεις αὐτὸν ἔνδον εἰς τὴν οἰκίαν σου, καὶ ἔσται μετὰ σοῦ ἕως ἂν ζητήσῃ αὐτὰ ὁ ἀδελφός σου, καὶ ἀποδώσεις αὐτῷ.
\vs{3}Οὕτω ποιήσεις τὸν ὄνον αὐτοῦ, καὶ οὕτω ποιήσεις τὸ ἱμάτιον αὐτοῦ, καὶ οὕτω ποιήσεις κατὰ πᾶσαν ἀπώλειαν τοῦ ἀδελφοῦ σου· ὅσα ἐὰν ἀπολῆται παρʼ αὐτοῦ, καὶ εὕρῃς, οὐ δυνήσῃ ὑπεριδεῖν.
\vs{4}Οὐκ ὄψῃ τὸν ὄνον τοῦ ἀδελφοῦ σου ἢ τὸν μόσχον αὐτοῦ πεπτωκότας ἐν τῇ ὁδῷ, μὴ ὑπερίδῃς αὐτοὺς, ἀνιστῶν ἀναστήσεις μετʼ αὐτοῦ.

\vs{5}Οὐκ ἔσται σκεύη ἀνδρὸς ἐπὶ γυναικὶ, οὐδὲ μὴ ἐνδύσηται ἀνὴρ στολὴν γυναικίαν, ὅτι βδέλυγμα Κυρίῳ τῷ Θεῷ σου ἐστὶ πᾶς ποιῶν ταῦτα.
\vs{6}Ἐὰν δὲ συναντήσῃς νοσσιᾷ ὀρνέων πρὸ προσώπου σου ἐν τῇ ὁδῷ ἢ ἐπὶ παντὶ δένδρῳ, ἢ ἐπὶ τῆς γῆς, νοσσοῖς ἢ ὠοῖς, καὶ ἡ μήτηρ θάλπῃ ἐπὶ τῶν νοσσῶν ἢ ἐπὶ τῶν ὠῶν, οὐ λήψῃ τὴν μητέρα μετὰ τῶν τέκνων.
\vs{7}Ἀποστολῇ ἀποστελεῖς τὴν μητέρα, τὰ δὲ παιδία λήψῃ σεαυτῷ, ἵνα εὖ σοι γένηται καὶ πολυήμερος γένῃ.

\vs{8}Ἐὰν οἰκοδομήσῃς οἰκίαν καινήν, καὶ ποιήσεις στεφάνην τῷ δώματί σου, καὶ οὐ ποιήσεις φόνον ἐν τῇ οἰκίᾳ σου, ἐὰν πέσῃ ὁ πεσὼν ἀπʼ αὐτοῦ.
\vs{9}Οὐ κατασπερεῖς τὸν ἀμπελῶνά σου διάφορον, ἵνα μὴ ἁγιασθῇ τὸ γέννημα, καὶ τὸ σπέρμα ὃ ἐὰν σπείρῃς μετὰ τοῦ γεννήματος τοῦ ἀμπελῶνός σου.
\vs{10}Οὐκ ἀροτριάσεις ἐν μόσχῳ καὶ ὄνῳ ἐπὶ τὸ αὐτό.
\vs{11}Οὐκ ἐνδύσῃ κίβδηλον, ἔρια καὶ λίνον ἐν τῷ αὐτῷ.
\vs{12}Στρεπτὰ ποιήσεις σεαυτῷ ἐπὶ τῶν τεσσάρων κρασπέδων τῶν περιβολαίων σου, ἃ ἐὰν περιβάλῃ ἐν αὐτοῖς.

\vs{13}Ἐὰν δέ τις λάβῃ γυναῖκα καὶ συνοικήσῃ αὐτῇ, καὶ μισήσῃ αὐτήν,
\vs{14}καὶ ἐπιθῇ αὐτῇ προφασιστικοὺς λόγους, καὶ κατενέγκῃ αὐτῆς ὄνομα πονηρὸν, καὶ λέγῃ, τὴν γυναῖκα ταύτην εἴληφα, καὶ προσελθὼν αὐτῇ οὐχ εὕρηκα αὐτῆς τὰ παρθένια·
\vs{15}Καὶ λαβὼν ὁ πατὴρ τῆς παιδὸς καὶ ἡ μήτηρ ἐξοίσουσι τὰ παρθένια τῆς παιδὸς πρὸς τὴν γερουσίαν ἐπὶ τὴν πύλην.
\vs{16}Καὶ ἐρεῖ ὁ πατὴρ τῆς παιδὸς τῇ γερουσίᾳ, τὴν θυγατέρα μου ταύτην δέδωκα τῷ ἀνθρώπῳ τούτῳ γυναῖκα, καὶ μισήσας αὐτὴν
\vs{17}νῦν οὗτος, ἐπιτίθησιν αὐτῇ προφασιστικοὺς λόγους, λέγων, οὐχ εὕρηκα τῇ θυγατρί σου παρθένια· καὶ ταῦτα τὰ παρθένια τῆς θυγατρός μου. Καὶ ἀναπτύξουσι τὸ ἱμάτιον ἐναντίον τῆς γερουσίας τῆς πόλεως.
\vs{18}Καὶ λήψεται ἡ γερουσία τῆς πόλεως ἐκείνης τὸν ἄνθρωπον ἐκεῖνον, καὶ παιδεύσουσιν αὐτόν,
\vs{19}καὶ ξημιώσουσιν αὐτὸν ἑκατὸν σίκλους, καὶ δώσουσι τῷ πατρὶ τῆς νεάνιδος, ὅτι ἐξήνεγκεν ὄνομα πονηρὸν ἐπὶ παρθένον Ἰσραηλίτιν, καὶ αὐτοῦ ἔσται γυνή· οὐ δυνήσεται ἐξαποστεῖλαι αὐτὴν τὸν ἅπαντα χρόνον.

\vs{20}Ἐὰν δὲ ἐπʼ ἀληθείας γένηται ὁ λόγος οὗτος, καὶ μὴ εὑρεθῇ παρθένια τῇ νεάνιδι,
\vs{21}καὶ ἐξάξουσι τὴν νεᾶνιν ἐπὶ τὰς θύρας τοῦ οἴκου τοῦ πατρὸς αὐτῆς, καὶ λιθοβολήσουσιν αὐτὴν ἐν λίθοις, καὶ ἀποθανεῖται, ὅτι ἐποίησεν ἀφροσύνην ἐν υἱοῖς Ἰσραήλ ἐκπορνεῦσαι τὸν οἶκον τοῦ πατρὸς αὐτῆς· καὶ ἐξαρεῖς τὸν πονηρὸν ἐξ ὑμῶν αὐτῶν.

\vs{22}Ἐὰν δὲ εὑρεθῇ ἄνθρωπος κοιμώμενος μετὰ γυναικὸς συνωκισμένης ἀνδρί, ἀποκτενεῖτε ἀμφοτέρους, τὸν ἄνδρα τὸν κοιμώμενον μετὰ τῆς γυναικὸς, καὶ τὴν γυναῖκα· καὶ ἐξαρεῖς τὸν πονηρὸν ἐξ Ἰσραήλ.
\vs{23}Ἐὰν δὲ γένηται παῖς παρθένος μεμνηστευμένη ἀνδρί, καὶ εὑρὼν αὐτὴν ἄνθρωπος ἐν πόλει κοιμηθῇ μετʼ αὐτῆς,
\vs{24}ἐξάξετε ἀμφοτέρους ἐπὶ τὴν πυλὴν τῆς πόλεως αὐτῶν, καὶ λιθοβοληθήσονται ἐν λίθοις, καὶ ἀποθανοῦνται· τὴν νεᾶνιν, ὅτι οὐκ ἐβόησεν ἐν τῇ πόλει· καὶ τὸν ἄνθρωπον, ὅτι ἐταπείνωσε τὴν γυναῖκα τοῦ πλησίον· καὶ ἐξαρεῖς τὸν πονηρὸν ἐξ ὑμῶν αὐτῶν.
\vs{25}Ἐὰν δὲ ἐν πεδίῳ εὕρῃ ἄνθρωπος τὴν παῖδα τὴν μεμνηστευμένῃν, καὶ βιασάμενος κοιμηθῇ μετʼ αὐτῆς, ἀποκτενεῖτε τὸν κοιμώμενον μετʼ αὐτῆς μόνον.
\vs{26}Καὶ τῇ νεάνιδι οὐκ ἔστιν ἁμάρτημα θανάτου· ὡς εἴ τις ἐπαναστῇ ἄνθρωπος ἐπὶ τὸν πλησίον, καὶ φονεύσῃ αὐτοῦ ψυχήν, οὕτω τὸ πρᾶγμα τοῦτο,
\vs{27}ὅτι ἐν τῷ ἀγρῷ εὗρεν αὐτήν· ἐβόησεν ἡ νεᾶνις ἡ μεμνηστευμένη, καὶ οὐκ ἦν ὁ βοηθήσων αὐτῇ.

\vs{28}Ἐὰν δέ τις εὕρῃ τὴν παῖδα τὴν παρθένον, ἥτις οὐ μεμνήστευται, καὶ βιασάμενος κοιμηθῇ μετʼ αὐτῆς, καὶ εὑρεθῇ,
\vs{29}δώσει ὁ ἄνθρωπος ὁ κοιμηθεὶς μετʼ αὐτῆς τῷ πατρὶ τῆς νεάνιδος πεντήκοντα δίδραχμα ἀργυρίου, καὶ αὐτοῦ ἔσται γυνή, ἀνθʼ ὧν ἐταπείνωσεν αὐτήν· οὐ δυνήσεται ἐξαποστεῖλαι αὐτὴν τὸν ἅπαντα χρόνον.

\ch{23}Οὐ λήψεται ἄνθρωπος τὴν γυναῖκα τοῦ πατρὸς αὐτοῦ, καὶ οὐκ ἀποκαλύψει συγκάλυμμα τοῦ πατρὸς αὐτοῦ.

\vs{2}Οὐκ εἰσελεύσεται θλαδίας, οὐδὲ ἀποκεκομμένος, εἰς ἐκκλησίαν Κυρίου.
\vs{3}Οὐκ εἰσελεύσεται ἐκ πόρνης εἰς ἐκκλησίαν Κύριου.

\vs{4}Οὐκ εἰσελεύσεται Ἀμμανίτης καὶ Μωαβίτης εἰς ἐκκλησίαν Κυρίου, καὶ ἕως δεκάτης γενεᾶς οὐκ εἰσελεύσεται εἰς ἐκκλησίαν Κυρίου, καὶ ἕως εἰς τὸν αἰῶνα·
\vs{5}παρὰ τὸ μὴ συναντῆσαι αὐτοὺς ὑμῖν μετὰ ἄρτων καὶ ὕδατος ἐν τῇ ὁδῷ, ἐκπορευομένων ὑμῶν ἐξ Αἰγύπτου, καὶ ὅτι ἐμισθώσαντο ἐπὶ σὲ τὸν Βαλαὰμ υἱὸν Βεὼρ ἐκ τῆς Μεσοποταμίας καταρᾶσθαί σε.
\vs{6}Καὶ οὐκ ἠθέλησε Κύριος ὁ Θεός σου εἰσακοῦσαι τοῦ Βαλαάμ· καὶ μετέστρεψε Κύριος ὁ Θεός σου τὰς κατάρας εἰς εὐλογίαν, ὅτι ἠγάπησέ σε Κύριος ὁ Θεός σου.
\vs{7}Οὐ προσαγορεύσεις εἰρηνικὰ αὐτοῖς καὶ συμφέροντα αὐτοῖς πάσας τὰς ἡμέρας σου εἰς τὸν αἰῶνα.
\vs{8}Οὐ βδελύξῃ Ἰδουμαῖον, ὅτι ἀδελφός σού ἐστίν· οὐ βδελύξῃ Αἰγύπτιον, ὅτι πάροικος ἐγένου ἐν τῇ γῇ αὐτοῦ.
\vs{9}Υἱοὶ ἐὰν γεννηθῶσιν αὐτοῖς, γενεᾷ τρίτῃ εἰσελεύσονται εἰς ἐκκλησίαν Κυρίου.

\vs{10}Ἐὰν δὲ ἐξέλθῃς παρεμβαλεῖν ἐπὶ τοὺς ἐχθρούς σου, καὶ φυλάξῃ ἀπὸ παντὸς ῥήματος πονηροῦ.
\vs{11}Ἐὰν ἠ· ἐν σοὶ ἄνθρωπος ὃς οὐκ ἔσται καθαρὸς ἐκ ῥύσεως αὐτοῦ νυκτὸς, καὶ ἐξελεύσεται ἔξω τῆν παρεμβολῆς, καὶ οὐκ εἰσελεύσεται εἰς τὴν παρεμβολήν.
\vs{12}Καὶ ἔσται τὸ πρὸς ἐσπέραν λούσεται τὸ σῶμα αὐτοῦ ὕδατι, καὶ δεδυκότος ἡλίου εἰσελεύσεται εἰς τὴν παρεμβολήν.
\vs{13}Καὶ τόπος ἔσται σοι ἔξω τῆς παρεμβολῆς, καὶ ἐξελεύσῃ ἐκεῖ ἔξω.
\vs{14}Καὶ πάσσαλος ἔσται σοι ἐπὶ τῆς ζώνης σου· καὶ ἔσται ὅταν διακαθιζάνῃς ἔξω, καὶ ὀρύξεις ἐν αὐτῷ, καὶ ἐπαγαγὼν καλύψεις τὴν ἀσχημοσύνην σου·
\vs{15}Ὅτι Κύριος ὁ Θεός σου ἐμπεριπατεῖ ἐν τῇ παρεμβολῇ σου ἐξελέσθαι σε καὶ παραδοῦναι τὸν ἐχθρόν σου πρὸ προσώπου σου· καὶ ἔσται ἡ παρεμβολή σου ἁγία, καὶ οὐκ ὀφθήσεται ἐν σοὶ ἀσχημοσύνη πράγματος, καὶ ἀποστρέψει ἀπὸ σοῦ.

\vs{16}Οὐ παραδώσεις παῖδα τῷ κυρίῳ αὐτοῦ, ὃς προστέθειταί σοι παρὰ τοῦ κυρίου αὐτοῦ.
\vs{17}Μετὰ σοῦ κατοικήσει, ἐν ὑμῖν κατοικήσει οὗ ἂν ἀρέσῃ αὐτῷ· οὐ θλίψεις αὐτόν.
\vs{18}Οὐκ ἔσται πόρνη ἀπὸ θυγατέρων Ἰσραὴλ, καὶ οὐκ ἔσται πορνεύων ἀπὸ υἱῶν Ἰσραήλ· οὐκ ἔσται τελεσφόρος ἀπὸ θυγατέρων Ἰσραήλ, καὶ οὐκ ἔσται τελισκόμενος ἀπὸ υἱῶν Ἰσραήλ.
\vs{19}Οὐ προσοίσεις μίσθωμα πόρνης, οὐδὲ ἄλλαγμα κυνὸς εἰς τὸν οἶκον Κυρίου τοῦ Θεοῦ σου πρὸς πᾶσαν εὐχὴν, ὅτι βδέλυγμα Κυρίῳ τῷ Θεῷ σου ἐστὶ καὶ ἀμφότερα.

\vs{20}Οὐκ ἐκτοκιεῖς τῷ ἀδελφῷ σου τόκον ἀργυρίου, καὶ τόκον βρωμάτων, καὶ τοκον παντὸς πράγματος, οὗ ἐὰν ἐκδανίσῃς.
\vs{21}Τῷ ἀλλοτρίῳ ἐκτοκιεῖς, τῷ δὲ ἀδελφῷ σου οὐκ ἐκτοκιεῖς, ἵνα εὐλογήσῃ σε Κύριος ὁ Θεός σου ἐν πᾶσι τοῖς ἔργοις σου ἐπὶ τῆς γῆς, εἰς ἣν εἰσπορεύῃ ἐκεῖ κληρονομῆσαι αὐτήν.

\vs{22}Ἐὰν δὲ εὔξῃ εὐχὴν Κυρίῳ τῷ Θεῷ σου, οὐ χρονιεῖς ἀποδοῦναι αὐτὴν, ὅτι ἐκζητῶν ἐκζητήσει Κύριος ὁ Θεός σου παρὰ σοῦ, καὶ ἔσται ἐν σοὶ ἁμαρτία.
\vs{23}Ἐὰν δὲ μὴ θέλῃς εὔξασθαι, οὐκ ἔστιν ἐν σοὶ ἁμαρτία.
\vs{24}Τὰ ἐκπορευόμενα διὰ τῶν χειλέων σου φυλάξῃ, καὶ ποιήσεις ὃν τρόπον ηὔξω Κυρίῳ τῷ Θεῷ δόμα, ὃ ἐλάλησας τῷ στόματί σου.

\vs{25}Ἐὰν δὲ εἰσέλθῃς εἰς ἀμητὸν τοῦ πλησίον σου, καὶ συλλέξῃς ἐν ταῖς χερσί σου στάχυς, καὶ δρέπανον οὐ μὴ ἐπιβάλῃς ἐπʼ ἀμητὸν τοῦ πλησίον σου.
\vs{26}Ἐὰν δὲ εἰσέλθῃς εἰς τὸν ἀμπελῶνα τοῦ πλησίον σου, φαγῇ σταφυλὴν, ὅσον ψυχήν σου ἐμπλησθῆναι, εἰς δὲ ἄγγος οὐκ ἐμβάλῃς.

\ch{24}Ἐὰν δέ τις λάβῃ γυναῖκα, καὶ συνοικήσῃ αὐτῇ, καὶ ἔσται ἐὰν μὴ εὕρῃ χάριν ἐναντίον αὐτοῦ, ὅτι εὗρεν ἐν αὐτῇ ἄσχημον πρᾶγμα, καὶ γράψει αὐτῇ βιβλίον ἀποστασίου, καὶ δώσει εἰς τὰς χεῖρας αὐτῆς, καὶ ἐξαποστελεῖ αὐτὴν ἐκ τῆς οἰκίας αὐτοῦ,
\vs{2}καὶ ἀπελθοῦσα γένηται ἀνδρὶ ἑτέρῳ,
\vs{3}καὶ μισήσῃ αὐτὴν ὁ ἀνὴρ ὁ ἔσχατος, καὶ γράψῃ αὐτῇ βιβλίον ἀποστασίου, καὶ δώσει εἰς τὰς χεῖρας αὐτῆς, καὶ ἐξαποστελεῖ αὐτὴν ἐκ τῆς οἰκίας αὐτοῦ, καὶ ἀποθάνῃ ὁ ἀνὴρ ὁ ἔσχατος, ὃς ἔλαβεν αὐτὴν ἐαυτῷ γυναῖκα,
\vs{4}οὐ δυνήσεται ὁ ἀνὴρ ὁ πρότερος ὁ ἐξαποστείλας αὐτὴν, ἐπαναστρέψας λαβεῖν αὐτὴν ἑαυτῷ γυναῖκα, μετὰ τὸ μιανθῆναι αὐτὴν, ὅτι βδέλυγμά ἐστιν ἐναντίον Κυρίου τοῦ Θεοῦ σου, καὶ οὐ μιανεῖτε τὴν γῆν, ἣν Κύριος ὁ Θεός σου δίδωσί σοι ἐν κλήρῳ.

\vs{5}Ἐὰν δέ τις λάβῃ γυναῖκα προσφάτως, οὐκ ἐξελεύσεται εἰς πόλεμον, καὶ οὐκ ἐπιβληθήσεται αὐτῷ οὐδὲν πρᾶγμα· ἀθῶος ἔσται ἐν τῇ οἰκίᾳ αὐτοῦ, ἐνιαυτὸν ἕνα εὐφρανεῖ τὴν γυναῖκα αὐτοῦ ἣν ἔλαβεν.

\vs{6}Οὐκ ἐνεχυράσεις μύλον, οὐδὲ ἐπιμύλιον, ὅτι ψυχὴν οὗτος ἐνεχυράζει.
\vs{7}Ἐὰν δὲ ἁλῷ ἄνθρωπος κλέπτων ψυχὴν ἐκ τῶν ἀδελφῶν αὐτοῦ τῶν υἱῶν Ἰσραὴλ, καὶ καταδυναστεύσας αὐτὸν ἀποδῶται, ἀποθανεῖται ὁ κλέπτης ἐκεῖνος· καὶ ἐξαρεῖς τὸν πονηρὸν ἐξ ὑμῶν αὐτῶν.
\vs{8}Πρόσεχε σεαυτῷ ἐν τῇ ἁφῇ τῆς λέπρας· φυλάξῃ σφόδρα ποιεῖν κατὰ πάντα τὸν νόμον, ὃν ἂν ἀναγγείλωσιν ὑμῖν οἱ ἱερεῖς οἱ Λευῖται· ὃν τρόπον ἐνετειλάμην ὑμῖν, φυλάξασθε ποιεῖν.
\vs{9}Μνήσθητι ὅσα ἐποίησε Κύριος ὁ Θεός σου τῇ Μαριὰμ ἐν τῇ ὁδῷ, ἐκπορευομένων ὑμῶν ἐξ Αἰγύπτου.

\vs{10}Ἐὰν ὀφείλημα ἠ· ἐν τῷ πλησίον σου, ὀφείλημα ὁτιοῦν, οὐκ εἰσελεύσῃ εἰς τὴν οἰκίαν αὐτοῦ ἐνεχυράσαι τὸ ἐνέχυρον αὐτοῦ.
\vs{11}Ἔξω στήσῃ, καὶ ὁ ἄνθρωπος οὗ τὸ δάνειόν σού ἐστιν ἐν αὐτῷ, ἐξοίσει σοι τὸ ἐνέχυρον ἔξω.
\vs{12}Ἐὰν δὲ ὁ ἄνθρωπος πένηται, οὐ κοιμηθήσῃ ἐν τῷ ἐνεχύρῳ αὐτοῦ.
\vs{13}Ἀποδόσει ἀποδώσεις τὸ ἐνέχυρον αὐτοῦ πρὸς δυσμαῖς ἡλίου, καὶ κοιμηθήσεται ἐν τῷ ἱματίῳ αὐτοῦ, καὶ εὐλογήσει σε, καὶ ἔσται σοι ἐλεημοσύνη ἐναντίον Κυρίου τοῦ Θεοῦ σου.
\vs{14}Οὐκ ἀπαδικήσεις μισθὸν πένητος καὶ ἐνδεοῦς ἐκ τῶν ἀδελφῶν σου, ἢ ἐκ τῶν προσηλύτων τῶν ἐν ταῖς πόλεσίν σου.
\vs{15}Αὐθημερὸν ἀποδώσεις τὸν μισθὸν αὐτοῦ, οὐκ ἐπιδύσεται ὁ ἥλιος ἐπʼ αὐτῷ, ὅτι πένης ἐστὶ, καὶ ἐν αὐτῷ ἔχει τὴν ἐλπίδα, καὶ καταβοήσεται κατὰ σοῦ πρὸς Κύριον, καὶ ἔσται ἐν σοὶ ἁμαρτία.
\vs{16}Οὐκ ἀποθανοῦνται πατέρες ὑπὲρ τέκνων, καὶ οἱ υἱοὶ οὐκ ἀποθανοῦνται ὑπὲρ πατέρων· ἕκαστος ἐν τῇ ἑαυτοῦ ἁματρίᾳ ἀποθανεῖται.
\vs{17}Οὐκ ἐκκλινεῖς κρίσιν προσηλύτου καὶ ὀρφανοῦ καὶ χήρας·
\vs{18}οὐκ ἐνεχυράσεις ἱμαίτιον χήρας, καὶ μνησθήσῃ ὅτι οἰκέτης ἦσθα ἐν γῇ Αἰγύπτῳ, καὶ ἐλυτρώσατό σε Κύριος ὁ Θεός σου ἐκεῖθεν· διὰ τοῦτο ἐγώ σοι ἐντέλλομαι ποιεῖν τὸ ῥῆμα τοῦτο.

\vs{19}Ἐὰν δὲ ἀμήσῃς ἀμητὸν ἐν τῷ ἀγρῷ σου, καὶ ἐπιλάθῃ δράγμα ἐν τῷ ἀγρῷ σου, οὐκ ἀναστραφήσῃ λαβεῖν αὐτό· τῷ προσηλύτῳ καὶ τῷ ὀρφανῷ καὶ τῇ χήρᾳ ἔσται, ἵνα εὐλογήσῃ σε Κύριος ὁ Θεός σου ἐν πᾶσι τοῖς ἔργοις τῶν χειρῶν σου.
\vs{20}Ἐὰν δὲ ἐλαιολογῇς, οὐκ ἐπαναστρέψεις καλαμήσασθαι τὰ ὀπίσω σου· τῷ προσηλύτῳ καὶ τῷ ὀρφανῷ καὶ τῇ χήρᾳ ἔσται· καὶ μνησθήσῃ ὅτι οἰκέτης ἦσθα ἐν γῇ Αἰγύπτῳ· διὰ τοῦτο ἐγώ σοι ἐντέλλομαι ποιεῖν τὸ ῥῆμα τοῦτο.
\vs{21}Ἐὰν δὲ τρυγήσῃς τὸν ἀμπελῶνά σου, οὐκ ἐπανατρυγήσεις αὐτὸν τὰ ὀπίσω σου· τῷ προσηλύτῳ καὶ τῷ ὀρφανῷ καὶ τῇ χήρᾳ ἔσται·
\vs{22}καὶ μνησθήσῃ ὅτι οἰκέτης ἦσθα ἐν γῇ Αἰγύπτῳ· διὰ τοῦτο ἐγώ σοι ἐντέλλομαι ποιεῖν τὸ ῥῆμα τοῦτο.

\ch{25}
Ἐὰν δὲ γένηται ἀντιλογία ἀναμέσον ἀνθρώπων, καὶ προσέλθωσιν εἰς κρίσιν, καὶ κρίνωσι, καὶ δικαιώσωσι τὸ δίκαιον, καὶ καταγνῶσι τοῦ ἀσεβοῦς·
\vs{2}Καὶ ἔσται, ἐὰν ἄξιος ἠ· πληγῶν ὁ ἀσεβῶν, καθιεῖς αὐτὸν ἔναντι τῶν κριτῶν, καὶ μαστιγώσουσιν αὐτὸν ἐναντίον αὐτῶν κατὰ τὴν ἀσέβειαν αὐτοῦ.
\vs{3}Καὶ ἀριθμῷ τεσσαράκοντα μαστιγώσουσιν αὐτόν οὐ προσθήσουσιν· ἐὰν δὲ προσθῇς μαστιγῶσαι ὑπὲρ ταύτας τὰς πληγὰς πλείους, ἀσχημονήσει ὁ ἀδελφός σου ἐναντίον σου.
\vs{4}Οὐ φιμώσεις βοῦν ἀλοῶντα.

\vs{5}Ἐὰν δὲ κατοικῶσιν ἀδελφοὶ ἐπὶ τὸ αὐτό, καὶ ἀποθάνῃ εἷς ἐξ αὐτῶν, σπέρμα δὲ μὴ ἠν αὐτῷ, οὐκ ἔσται ἡ γυνὴ τοῦ τεθνηκότος ἔξω ἀνδρὶ μὴ ἐγγίζοντι· ὁ ἀδελφὸς τοῦ ἀνδρὸς αὐτῆς εἰσελεύσεται πρὸς αὐτὴν, καὶ λήψεται αὐτὴν ἑαυτῷ γυναῖκα, καὶ συνοικήσει αὐτῇ.
\vs{6}Καὶ ἔσται τὸ παιδίον ὃ ἐὰν τέκῃ, κατασταθήσεται ἐκ τοῦ ὀνόματος τοῦ τετελευτηκότος, καὶ οὐκ ἐξαλειφθήσεται τὸ ὄνομα αὐτοῦ ἐξ Ἰσραήλ.

\vs{7}Ἐὰν δὲ μὴ βούληται ὁ ἄνθρωπος λαβεῖν τὴν γυναῖκα τοῦ ἀδελφοῦ αὐτοῦ, καὶ ἀναβήσεται ἡ γυνὴ ἐπὶ τὴν πύλην ἐπὶ τὴν γερουσίαν, καὶ ἐρεῖ, οὐ θέλει ὁ ἀδελφὸς τοῦ ἀνδρός μου ἀναστῆσαι τὸ ὄνομα τοῦ ἀδελφοῦ αὐτοῦ ἐν Ἰσραήλ, οὐκ ἠθέλησεν ὁ ἀδελφὸς τοῦ ἀνδρός μου.
\vs{8}Καὶ καλέσουσιν αὐτὸν ἡ γερουσία τῆς πόλεως αὐτοῦ, καὶ ἐροῦσιν αὐτῷ· καὶ στὰς εἴπῃ, οὐ βούλομαι λαβεῖν αὐτήν·
\vs{9}καὶ προσελθοῦσα ἡ γυνὴ τοῦ ἀδελφοῦ αὐτοῦ ἔναντι τῆς γερουσίας, καὶ ὑπολύσει τὸ ὑπόδημα αὐτοῦ τὸ ἓν ἀπὸ τοῦ ποδὸς αὐτοῦ, καὶ ἐμπτύσεται κατὰ πρόσωπον αὐτοῦ, καὶ ἀποκριθεῖσα ἐρεῖ, οὕτως ποιήσουσι τῷ ἀνθρώπῳ, ὃς οὐκ οἰκοδομήσει τὸν οἴκον τοῦ ἀδελφοῦ αὐτοῦ ἐν Ἰσραήλ.
\vs{10}Καὶ κληθήσεται τὸ ὄνομα αὐτοῦ ἐν Ἰσραὴλ, οἶκος τοῦ ὑπολυθέντος τὸ ὑπόδημα.

\vs{11}Ἐὰν δὲ μάχωνται ἄνθρωποι ἐπὶ τὸ αὐτό, ἄνθρωπος μετὰ τοῦ ἀδελφοῦ αὐτοῦ, καὶ προσέλθῃ ἡ γυνὴ ἑνὸς αὐτῶν ἐξελέσθαι τὸν ἄνδρα αὐτῆς ἐκ χειρὸς τοῦ τύπτοντος αὐτόν, καὶ ἐκτείνασα τὴν χεῖρα ἐπιλάβηται τῶν διδύμων αὐτοῦ,
\vs{12}ἀποκόψεις τὴν χεῖρα· οὐ φείσεται ὁ ὀφθαλμός σου ἐπʼ αὐτῇ.

\vs{13}Οὐκ ἔσται σοι ἐν τῷ μαρσίππῳ σου στάθμιον καὶ στάθμιον, μέγα ἢ μικρόν.
\vs{14}Οὐκ ἔσται ἐν τῇ οἰκίᾳ σου μέτρον καὶ μέτρον, μέγα ἢ μικρόν.
\vs{15}Στάθμιον ἀληθινὸν καὶ δίκαιον ἔσται σοι, καὶ μέτρον ἀληθινὸν καὶ δίκαιον ἔσται σοι, ἵνα πολυήμερος γένῃ ἐπὶ τῆς γῆς, ἧς Κύριος ὁ Θεός σου δίδωσί σοι ἐν κλήρῳ.
\vs{16}Ὅτι βδέλυγμα Κυρίῳ τῷ Θεῷ σου πας ποιῶν ταῦτα, πας ποιῶν ἄδικον.

\vs{17}Μνήσθητι ὅσα ἐποίησέ σοι Ἀμαλὴκ ἐν τῇ ὁδῷ, ἐκπορευομένου σου ἐκ γῆς Αἰγύπτου,
\vs{18}πῶς ἀντέστη σοι ἐν τῇ ὁδῷ, καὶ ἔκοψέ σου τὴν οὐραγίαν, τοὺς κοπιῶντας ὀπίσω σου, σὺ δὲ ἐπείνας καὶ ἐκοπίας· καὶ οὐκ ἐφοβήθη τὸν Θεόν.
\vs{19}Καὶ ἔσται ἡνίκα ἐὰν καταπαύσῃ σε Κύριος ὁ Θεός σου ἀπὸ πάντων τῶν ἐχθρῶν σου τῶν κύκλῳ σου ἐν τῇ γῇ, ᾗ Κύριος ὁ Θεός σου δίδωσί σοι κληρονομῆσαι, ἐξαλείψεις τὸ ὄνομα Ἀμαλὴκ ἐκ τῆς ὑπὸ τὸν οὐρανόν, καὶ οὐ μὴ ἐπιλάθῃ.

\ch{26}
Καὶ ἔσται ἐὰν εἰσέλθῆς εἰς τὴν γῆν, ἣν Κύριος ὁ Θεός σου δίδωσί σοι κληρονομῆσαι, καὶ κατακληρονομήσῃς αὐτὴν, καὶ κατοικήσῃς ἐπʼ αὐτὴν,
\vs{2}καὶ λήμψῃ ἀπὸ τῆς ἀπαρχῆς τῶν καρπῶν τῆς γῆς σου, ἧς Κύριος ὁ Θεός σου δίδωσί σοι, καὶ ἐμβαλεῖς εἰς κάρταλλον, καὶ πορεύσῃ εἰς τὸν τόπον, ὃν ἂν ἐκλέξηται Κύριος ὁ Θεός σου ἐπικληθῆναι τὸ ὄνομα αὐτοῦ ἐκεῖ.
\vs{3}Καὶ ἐλεύσῃ πρὸς τὸν ἱερέα ὃς ἔσται ἐν ταῖς ἡμέραις ἐκείναις, καὶ ἐρεῖς πρὸς αὐτὸν, ἀναγγέλλω σήμερον Κυρίῳ τῷ Θεῷ μου, ὅτι εἰσελήλυθα εἰς τὴν γῆν, ἣν ὤμοσε Κύριος τοῖς πατράσιν ἡμῶν δοῦναι ἡμῖν.
\vs{4}Καὶ λήψεται ὁ ἱερεὺς τὸν κάρταλλον ἐκ τῶν χειρῶν σου, καὶ θήσει αὐτὸν ἀπέναντι τοῦ θυσιαστηρίου Κυρίου τοῦ Θεοῦ σου.
\vs{5}Καὶ ἀποκριθεὶς ἐρεῖ ἔναντι Κυρίου τοῦ Θεοῦ σου, Συρίαν ἀπέβαλεν ὁ πατήρ μου, καὶ κατέβη εἰς Αἴγυπτον, καὶ παρῴκησεν ἐκεῖ ἐν ἀριθμῷ βραχεῖ, καὶ ἐγένετο ἐκεῖ εἰς ἔθνος μέγα καὶ πλῆθος πολύ.
\vs{6}Καὶ ἐκάκωσαν ἡμᾶς οἱ Αἰγύπτιοι, καὶ ἐταπείνωσαν ἡμᾶς, καὶ ἐπέθηκαν ἡμῖν ἔργα σκληρά·
\vs{7}Καὶ ἀνεβοήσαμεν πρὸς Κύριον τὸν Θεὸν ἡμῶν, καὶ εἰσήκουσε Κύριος τῆς φωνῆς ἡμῶν, καὶ εἴδε τὴν ταπείνωσιν ἡμῶν, καὶ τὸν μόχθον ἡμῶν, καὶ τὸν θλιμμὸν ἡμῶν.
\vs{8}Καὶ ἐξήγαγεν ἡμᾶς Κύριος ἐξ Αἰγύπτου αὐτὸς ἐν ἰσχύϊ αὐτοῦ τῇ μεγάλῃ, καὶ ἐν χειρὶ κραταιᾷ, καὶ βραχίονι ὑψηλῷ, καὶ ἐν ὁράμασι μεγάλοις, καὶ ἐν σημείοις, καὶ ἐν τέρασι.
\vs{9}Καὶ εἰσήγαγεν ἡμᾶς εἰς τὸν τόπον τοῦτον, καὶ ἔδωκεν ἡμῖν τὴν γῆν ταύτην, γῆν ῥέουσαν γάλα καὶ μέλι.
\vs{10}Καὶ νῦν ἰδοὺ ἐνήνοχα τὴν ἀπαρχὴν τῶν γεννμάτων τῆς γῆς, ἧς ἔδωκάς μοι Κύριε, γῆν ῥέουσαν γάλα καὶ μέλι· καὶ ἀφήσεις αὐτὸ ἀπέναντι Κυρίου τοῦ Θεοῦ σου, καὶ προσκυνήσεις ἔναντι Κυρίου τοῦ Θεοῦ σου,
\vs{11}καὶ εὐφρανθήσῃ ἐν πᾶσι τοῖς ἀγαθοῖς, οἷς ἔδωκέ σοι Κύριος ὁ Θεός σου, καὶ ἡ οἰκία σου, καὶ ὁ Λευίτης, καὶ ὁ προσήλυτος ὁ ἐν σοί.

\vs{12}Ἐὰν δὲ συντελέσῃς ἀποδεκατῶσαι πᾶν τὸ ἐπιδέκατον τῶν γενημάτων σου ἐν τῷ ἔτει τῷ τρίτῳ, τὸ δεύτερον ἐπιδέκατον δώσεις τῷ Λευίτῃ καὶ τῷ προσηλύτῳ καὶ τῷ ὀρφανῷ καὶ τῇ χήρᾳ, καὶ φάγονται ἐν ταῖς πόλεσί σου, καὶ εὐφρανθήσονται.

\vs{13}Καὶ ἐρεῖς ἔναντι Κυρίου τοῦ Θεοῦ σου, ἐξεκάθαρα τὰ ἅγια ἐκ τῆς οἰκίας μου, καὶ ἔδωκα αὐτὰ τῷ Λευίτῃ καὶ τῷ προσηλύτῳ καὶ τῷ ὀρφανῷ καὶ τῇ χήρᾳ, κατὰ πάσας τὰς ἐντολὰς ἃς ἐνετείλω μοι· οὐ παρῆλθον τὴν ἐντολήν σου, καὶ οὐκ ἐπελαθόμην.
\vs{14}Καὶ οὐκ ἔφαγον ἐν ὀδύνῃ μου ἀπʼ αὐτῶν, οὐκ ἐκάρπωσα ἀπʼ αὐτῶν εἰς ἀκάθαρτον, οὐκ ἔδωκα ἀπʼ αὐτῶν τῷ τεθνηκότι· ὑπήκουσα τῆς φωνῆς Κυρίου τοῦ Θεοῦ ἡμῶν, ἐποίησα καθὰ ἐνετείλω μοι.
\vs{15}Κάτιδε ἐκ τοῦ οἴκου τοῦ ἁγίου σου ἐκ τοῦ οὐρανοῦ, καὶ εὐλόγησον τὸν λαόν σου τὸν Ἰσραὴλ, καὶ τὴν γῆν ἣν ἔδωκας αὐτοῖς, καθὰ ὤμοσας τοῖς πατράσιν ἡμῶν, δοῦναι ἡμῖν γῆν ῥέουσαν γάλα καὶ μέλι.

\vs{16}Ἐν τῇ ἡμέρᾳ ταύτῃ Κύριος ὁ Θεός σου ἐνετείλατό σοι ποιῆσαι πάντα τὰ δικαιώματα καὶ τὰ κρίματα· καὶ φυλάξεσθε καὶ ποιήσετε αὐτὰ ἐξ ὅλης τῆς καρδίας ὑμῶν, καὶ ἐξ ὅλης τῆς ψυχῆς ὑμῶν.
\vs{17}Τὸν Θεὸν εἵλου σήμερον εἶναί σου Θεόν, καὶ πορεύεσθαι ἐν πάσαις ταῖς ὁδοῖς αὐτοῦ, καὶ φυλάσσεσθαι τὰ δικαιώματα καὶ τὰ κρίματα, καὶ ὑπακούειν τῆς φωνῆς αὐτοῦ.
\vs{18}Καὶ Κύριος εἵλατό σε σήμερον γενέσθαι σε αὐτῷ λαὸν περιούσιον, καθάπερ εἶπε, φυλάττειν τὰς ἐντολὰς αὐτοῦ,
\vs{19}καὶ εἶναί σε ὑπεράνω πάντων τῶν ἐθνῶν, ὡς ἐποίησέ σε ὀνομαστὸν καὶ καύχημα καὶ δοξαστόν, εἶναί σε λαὸν ἅγιον Κυρίῳ τῷ Θεῷ σου, καθὼς ἐλάλησε.

\ch{27}
Καὶ προσέταξε Μωυσῆς καὶ ἡ γερουσία Ἰσραὴλ, λέγων, φυλάσσεσθε πάσας τὰς ἐντολάς ταύτας, ὅσας ἐγὼ ἐντέλλομαι ὑμῖν σήμερον.
\vs{2}Καὶ ἔσται ᾗ ἂν ἡμέρᾳ διαβῆτε τὸν Ἰορδάνην εἰς τὴν γῆν, ἣν Κύριος ὁ Θεός σου δίδωσί σοι, καὶ στήσεις σεαυτῷ λίθους μεγάλους, καὶ κονιάσεις αὐτοὺς κονίᾳ.
\vs{3}Καὶ γράψεις ἐπὶ τῶν λίθων τούτων πάντας τούς λόγους τοῦ νόμου τούτου, ὠς ἂν διαβῆτε τὸν Ἰορδάνην, ἡνίκα ἂν εἰσέλθητε εἰς τὴν γῆν, ἣν Κύριος ὁ Θεὸς τῶν πατέρων σου δίδωσί σοι, γῆν ῥέουσαν γάλα καὶ μέλι, ὃν τρόπον εἶπε Κύριος ὁ Θεὸς τῶν πατέρων σού σοί.
\vs{4}Καὶ ἔσται ὡς ἂν διαβῆτε τὸν Ἰορδάνην, στήσετε τοὺς λίθους τούτους, οὓς ἐγὼ ἐντέλλομαί σοι σήμερον, ἐν ὄρει Γαιβάλ, καὶ κονιάσεις αὐτοὺς κονίᾳ.
\vs{5}Καὶ οἰκοδομήσεις ἐκεῖ θυσιαστήριον Κυρίῳ τῷ Θεῷ σου, θυσιαστήριον ἐκ λίθων· οὐκ ἐπιβαλεῖς ἐπʼ αὐτὸ σίδηρον·
\vs{6}λίθους ὁλοκλήρους οἰκοδομήσεις θυσιαστήριον Κυρίῳ τῷ Θεῷ σου, καὶ ἀνοίσεις ἐπʼ αὐτὸ ὁλοκαυτώματα Κυρίῳ τῷ Θεῷ σου.
\vs{7}Καὶ θύσεις ἐκεῖ θυσίαν σωτηρίου· καὶ φάγῃ, καὶ ἐμπλησθήσῃ, καὶ εὐφρανθήσῃ ἔναντι Κυρίου τοῦ Θεοῦ σου.
\vs{8}Καὶ γράψεις ἐπὶ τῶν λίθων πάντα τὸν νόμον τοῦτον σαφῶς σφόδρα.

\vs{9}Καὶ ἐλάλησε Μωυσῆς καὶ οἱ ἱερεῖς οἱ Λευῖται παντὶ Ἰσραὴλ, λέγοντες, σιώπα καὶ ἄκουε Ἰσραήλ· ἐν τῇ ἡμέρᾳ ταύτῃ γέγονας εἰς λαὸν Κυρίῳ τῷ Θεῷ σου,
\vs{10}καὶ εἰσακούσῃ τῆς φωνῆς Κυρίου τοῦ Θεοῦ σου, καὶ ποιήσεις πάσας τὰς ἐντολὰς αὐτοῦ, καὶ τὰ δικαιώματα αὐτοῦ, ὅσα ἐγὼ ἐντέλλομαί σοι σήμερον.

\vs{11}Καὶ ἐνετείλατο Μωυσῆς τῷ λαῷ ἐν τῇ ἡμέρᾳ ἐκείνῃ, λέγων,
\vs{12}οὗτοι στήσονται εὐλογεῖν τὸν λαὸν ἐν ὄρει Γαριζίν διαβάντες τὸν Ἰορδάνην, Συμεὼν, Λευὶ, Ἰουδας, Ἰσσάχαρ, Ἰωσὴφ, καὶ Βενιαμίν.
\vs{13}Καὶ οὗτοι στήσονται ἐπὶ τῆς κατάρας ἐν ὄρει Γαιβάλ, Ῥουβὴν, Γὰδ, καὶ Ἀσὴρ, Ζαβουλὼν, Δὰν, καὶ Νεφθαλί.

\vs{14}Καὶ ἀποκριθέντες ἐροῦσιν οἱ Λευῖται παντὶ Ἰσραὴλ φωνῇ μεγάλῃ,
\vs{15}ἐπικατάρατος ἄνθρωπος ὅστις ποιήσει γλυπτὸν καὶ χωνευτὸν, βδέλυγμα Κυρίῳ, ἔργον χειρῶν τεχνιτῶν, καὶ θήσει αὐτὸ ἐν ἀποκρύφῳ· καὶ ἀποκριθεὶς πᾶς ὁ λαὸς, ἐροῦσι, γένοιτο.
\vs{16}Ἐπικατάρατος ὁ ἀτιμάζων πατέρα αὐτοῦ ἢ μητέρα αὐτοῦ· καὶ ἐροῦσι πᾶς ὁ λαὸς, γένοτιο.
\vs{17}Ἐπικατάρατος ὁ μετατιθεὶς ὅρια τοῦ πλησὶον· καὶ ἐροῦσι πᾶς ὁ λαός, γένοιτο.
\vs{18}Ἐπικατάρατος ὁ πλανῶν τυφλὸν ἐν ὁδῷ· καὶ ἐροῦσι πᾶς ὁ λαός, γένοιτο.
\vs{19}Ἐπικατάρατος ὃς ἂν ἐκκλίνῃ κρίσιν προσηλύτου καὶ ὀρφανοῦ καὶ χήρας· καὶ ἐροῦσι πᾶς ὁ λαός, γένοιτο.
\vs{20}Ἐπικατάρατος ὁ κοιμώμενος μετὰ γυναικὸς τοῦ πατρὸς αὐτοῦ, ὅτι ἀπεκάλυψε συγκάλυμμα τοῦ πατρὸς αὐτοῦ· καὶ ἐροῦσι πᾶς ὁ λαὸς, γένοιτο.
\vs{21}Ἐπικατάρατος ὁ κοιμώμενος μετὰ παντὸς κτήνους· καὶ ἐροῦσι πᾶς ὁ λαὸς, γένοιτο.
\vs{22}Ἐπικατάρατος ὁ κοιμώμενος μετὰ ἀδελφῆς ἐκ πατρὸς ἢ μητρὸς αὐτοῦ. καὶ ἐροῦσι πᾶς ὁ λαός, γένοιτο.
\vs{23}Ἐπικατάρατος ὁ κοιμώμενος μετὰ νύμφης αὐτοῦ· καὶ ἐροῦσι πᾶς ὁ λαός, γένοιτο· ἐπικατάρατος ὁ κοιμώμενος μετὰ τῆς ἀδελφῆς τῆς γυναικὸς αὐτοῦ· καὶ ἐροῦσι πᾶς ὁ λαὸς, γένοιτο.
\vs{24}Ἐπικατάρατος ὁ τύπτων τὸν πλησίον δόλῳ· καὶ ἐροῦσι πᾶς ὁ λαός, γένοιτο.
\vs{25}Ἐπικατάρατος ὃς ἂν λάβῃ δῶρα πατάξαι ψυχὴν αἵματος ἀθῴου· καὶ ἐροῦσι πᾶς ὁ λαὸς, γένοιτο.
\vs{26}Ἐπικατάρατος πᾶς ἄνθρωπος ὃς οὐκ ἐμμένει ἐν πᾶσι τοῖς λόγοις τοῦ νόμου τούτου ποιῆσαι αὐτούς· καὶ ἐροῦσι πᾶς ὁ λαὸς, γένοιτο.

\ch{28}
Καὶ ἔσται ἐὰν ἀκοῇ ἀκούσῃς τῆς φωνῆς Κυρίου τοῦ Θεοῦ σου, φυλάσσειν καὶ ποιεῖν πάσας τὰς ἐντολὰς ταύτας, ἃς ἐγὼ ἐντέλλομαί σοι σήμερον, καὶ δώσει σε Κύριος ὁ Θεός σου ὑπεράνω ἐπὶ πάντα τὰ ἔθνη τῆς γῆς,
\vs{2}καὶ ἥξουσιν ἐπὶ σὲ πᾶσαι αἱ εὐλογίαι αὗται, καὶ εὑρήσουσί σε· ἐὰν ἀκοῇ ἀκούσῃς τῆς φωνῆς Κυρίου τοῦ Θεοῦ σου,
\vs{3}εὐλογημένος σὺ ἐν πόλει, καὶ εὐλογημένος σὺ ἐν ἀγρῶ.
\vs{4}Εὐλογημένα τὰ ἔκγονα τῆς κοιλίας σου, καὶ τὰ γεννήματα τῆς γης σου, καὶ τὰ βουκόλια τῶν βοῶν σου, καὶ τὰ ποίμνια τῶν προβάτων σου.
\vs{5}Εὐλογημέναι αἱ ἀποθῆκαί σου, καὶ τὰ ἐγκαταλείμματά σου.
\vs{6}Εὐλογημένος σὺ ἐν τῷ εἰσπορεύεσθαί σε, καὶ εὐλογημένος σὺ ἐν τῷ ἐκπορεύεσθαί σε.

\vs{7}Παραδῷ Κύριος ὁ Θεός σου τοὺς ἐχθρούς σου τοὺς ἀνθεστηκότας σοι συντετριμμένους πρὸ προσώπου σου· ὁδῷ μιᾷ ἐξελεύσονται πρὸς σέ, καὶ ἐν ἑπτὰ ὁδοῖς φεύξονται ἀπὸ προσώπου σου.
\vs{8}Ἀποστείλαι Κύριος ἐπὶ σὲ τὴν εὐλογίαν ἐν τοῖς ταμείοις σου, καὶ ἐπὶ πάντα οὗ ἂν ἐπιβάλῃς τὴν χεῖρά σου, ἐπὶ τῆς γῆς, ἧς Κύριος ὁ Θεός σου δίδωσί σοι.
\vs{9}Ἀναστήσαι σε Κύριος ἑαυτῷ λαὸν ἅγιον, ὃν τρόπον ὤμοσε τοῖς πατράσι σου· ἐὰν ἀκούσῃς τῆς φωνῆς Κυρίου τοῦ Θεοῦ σου, καὶ πορευθῇς ἐν πάσαις ταῖς ὁδοῖς αὐτοῦ,
\vs{10}καὶ ὄψονταί σε πάντα τὰ ἔθνη τῆς γῆς, ὅτι τὸ ὄνομα Κυρίου ἐπικέκληταί σοι, καὶ φοβηθήσονταί σε.
\vs{11}Καὶ πληθυνεῖ σε Κύριος ὁ Θεός σου εἰς ἀγαθὰ ἐν τοῖς ἐκγόνοις τῆς κοιλίας σου, καὶ ἐπὶ τοῖς ἐκγόνοις τῶν κτηνῶν σου, καὶ ἐπὶ τοῖς γεννήμασι τῆς γῆς σου, ἐπὶ τῆς γῆς σου ἧς ὤμοσε Κύριος τοῖς πατράσι σου δοῦναί σοι.

\vs{12}Ἀνοίξαι σοι Κύριος τὸν θησαυρὸν αὐτοῦ τὸν ἀγαθὸν, τὸν οὐρανὸν, δοῦναι τὸν ὑετὸν τῇ γῇ σου ἐπὶ καιροῦ· εὐλογῆσαι πάντα τὰ ἔργα τῶν χειρῶν σου· καὶ δανειεῖς ἔθνεσι πολλοῖς, σὺ δὲ οὐ δανειῇ. καὶ ἄρξεις σὺ ἐθνῶν πολλῶν, σοῦ δὲ οὐκ ἄρξουσι.
\vs{13}Καταστήσαι σε Κύριος ὁ Θεός σου εἰς κεφαλὴν καὶ μὴ εἰς οὐρὰν, καὶ ἔσῃ τότε ἐπάνω καὶ οὐκ ἔσῃ ὑποκάτω, ἐὰν ἀκούσῃς τῆς φωνῆς Κυρίου τοῦ Θεοῦ σου, ὅσα ἐγὼ ἐντέλλομαί σοι σήμερον φυλάσσειν.
\vs{14}Οὐ παραβήσῃ ἀπὸ πασῶν τῶν ἐντολῶν, ὧν ἐγὼ ἐντέλλομαί σοι σήμερον δεξιὰ οὐδὲ ἀριστερὰ, πορεύσεθαι ὀπίσω θεῶν ἑτέρων λατρεύειν αὐτοῖς.

\vs{15}Καὶ ἔσται ἐὰν μὴ εἰσακούσῃς τῆς φωνῆς Κυρίου τοῦ Θεοῦ σου, φυλάσσεσθαι πάσας τὰς ἐντολὰς αὐτοῦ, ὅσας ἐγὼ ἐντέλλομαί σοι σήμερον, καὶ ἐλεύσονται ἐπὶ σὲ πᾶσαι αἱ κατάραι αὗται, καὶ καταλήψονταί σε.
\vs{16}Ἐπικατάρατος σὺ ἐν πόλει, καὶ ἐπικατάρατος σὺ ἐν ἀγρῷ.
\vs{17}Ἐπικατάρατοι αἱ ἀποθῆκαί σου, καὶ τὰ ἐγκαταλείμματά σου.
\vs{18}Ἐπικατάρατα τὰ ἔκγονα τῆς κοιλίας σου, καὶ τὰ γεννήματα τῆς γῆς σου, τὰ βουκόλια τῶν βοῶν σου, καὶ τὰ ποίμνια τῶν προβάτων σου·
\vs{19}Ἐπικατάρατος σὺ ἐν τῷ εἰσπορεύεσθαί σε, καὶ ἐπικατάρατος σὺ ἐν τῷ ἐκπορεύεσθαί σε.

\vs{20}Ἀποστείλαι Κύριος ἐπὶ σὲ τὴν ἔνδειαν καὶ τὴν ἐκλιμίαν καὶ τὴν ἀνάλωσιν ἐπὶ πάντα οὗ ἐὰν ἐπιβάλῃς τὴν χεῖρά σου, ἕως ἂν ἐξολοθρεύσῃ σε, καὶ ἕως ἂν ἀπολέσῃ σε ἐν τάχει διὰ τὰ πονηρὰ ἐπιτηδεύματά σου, διότι ἐνκατέλιπές με.
\vs{21}Προσκολλήσαι Κύριος εἰς σὲ τὸν θάνατον, ἕως ἂν ἐξαναλώσῃ σε ἀπὸ τῆς γῆς, εἰς ἣν εἰσπορεύῃ ἐκεῖ κληρονομῆσαι αὐτήν.
\vs{22}Πατάξαι σε Κύριος ἐν ἀπορίᾳ, καὶ πυρετῷ, καὶ ῥίγει, καὶ ἐρεθισμῷ, καὶ ἀνεμοφθορίᾳ, καὶ τῇ ὤχρᾳ, καὶ καταδιώξονταί σε ἕως ἂν ἀπολέσωσί σε.
\vs{23}Καὶ ἔσται σοι ὁ οὐρανὸς ὁ ὑπὲρ κεφαλῆς σου χαλκοῦς, καὶ ἡ γῆ ἡ ὑποκάτω σου σιδηρᾶ.
\vs{24}Δῴη Κύριος ὁ Θεός σου τὸν ὑετὸν τῇς γῇς σου κονιορτὸν, καὶ χοῦς ἐκ τοῦ οὐρανοῦ καταβήσεται, ἕως ἂν ἐκτρίψῃ σε, καὶ ἕως ἂν ἀπολέσῃ σε ἐν τάχει.
\vs{25}Δῴη σε Κύριος ἐπὶ κοπὴν ἐναντίον τῶν ἐχθρῶν· ἐν ὁδῷ μιᾷ ἐξελεύσῃ πρὸς αὐτοὺς, καὶ ἐν ἑπτὰ ὁδοῖς φεύξῃ ἀπὸ προσώπου αὐτῶν· καὶ ἔσῃ διασπορὰ ἐν πάσαις βασιλείαις τῆς γῆς.
\vs{26}Καὶ ἔσονται οἱ νεκροὶ ὑμῶν κατάβρωμα τοῖς πετεινοῖς τοῦ οὐρανοῦ, καὶ τοῖς θηρίοις τῆς γῆς, καὶ οὐκ ἔσται ὁ ἐκφοβῶν.
\vs{27}Πατάξαι σε Κύριος ἕλκει Αἰγυπτίῳ εἰς τὴν ἕδραν, καὶ ψώρᾳ ἀγρίᾳ, καὶ κνήφῃ, ὥστε μὴ δύνασθαί σε ἰαθῆναι.
\vs{28}Πατάξαι σε Κύριος παραπληξίᾳ, καὶ ἀορασίᾳ, καὶ ἐκστάσει διανοίας.
\vs{29}Καὶ ἔσῃ ψηλαφῶν μεσημβρίας, ὡσεί τις ψηλαφήσαι τυφλὸς ἐν τῷ σκότει, καὶ οὐκ εὐοδώσει τὰς ὁδούς σου· καὶ ἔσῃ τότε ἀδικούμενος, καὶ διαρπαζόμενος πάσας τὰς ἡμέρας, καὶ οὐκ ἔσται ὁ βοηθῶν.

\vs{30}Γυναῖκα λήψῃ, καὶ ἀνὴρ ἕτερος ἕξει αὐτήν· οἰκίαν οἰκοδομήσεις, καὶ οὐκ οἰκήσεις ἐν αὐτῇ· ἀμπελῶνα φυτεύσεις, καὶ οὐ μὴ τρυγήσεις αὐτόν.
\vs{31}Ὁ μόσχος σου ἐσφαγμένος ἐναντίον σου, καὶ οὐ φάγῃ ἐξ αὐτοῦ· ὁ ὄνος σου ἡρπασμένος ἀπὸ σοῦ, καὶ οὐκ ἀποδοθήσεταί σοι· τὰ πρόβατά σου δεδομένα τοῖς ἐχθροῖς σου,
\vs{32}καὶ οὐκ ἔσται σοι ὁ βοηθῶν. Οἱ υἱοί σου καὶ αἱ θυγατέρες σου δεδομέναι ἔθνει ἑτέρῳ, καὶ οἱ ὀφθαλμοί σου βλέψονται σφακελίζοντες εἰς αὐτὰ· οὐκ ἰσχύσει ἡ χείρ σου.
\vs{33}Τὰ ἐκφόρια τῆς γῆς σου, καὶ πάντας τοὺς πόνους σου φάγεται ἔθνος, ὃ οὐκ ἐπίστασαι· καὶ ἔσῃ ἀδικούμενος καὶ τεθραυσμένος πάσας τὰς ἡμέρας.
\vs{34}Καὶ ἔσῃ παράπληκτος διὰ τὰ ὁράματα τῶν ὀφθαλμῶν σου, ἃ βλέψῃ.

\vs{35}Πατάξαι σε Κύριος ἐν ἕλκει πονηρῷ ἐπὶ τὰ γόνατα καὶ ἐπὶ τὰς κνήμας, ὥστε μὴ δύνασθαι ἰαθῆναί σε ἀπὸ ἴχνους τῶν ποδῶν σου ἕως τῆς κορυφῆς σου.

\vs{36}Ἀπαγάγοι Κύριός σε καὶ τοὺς ἄρχοντάς σου, οὓς ἂν καταστήσῃς ἐπὶ σεαυτὸν, ἐπʼ ἔθνος ὃ οὐκ ἐπίστασαι σὺ καὶ οἱ πατέρες σου, καὶ λατρεύσεις ἐκεῖ θεοῖς ἑτέροις ξύλοις καὶ λίθοις.
\vs{37}Καὶ ἔσῃ ἐκεῖ ἐν αἰνίγματι καὶ παραβολῇ καὶ διηγήματι ἐν πᾶσι τοῖς ἔθνεσιν, εἰς οὓς ἂν ἀπαγάγῃ σε Κύριος ἐκεῖ.

\vs{38}Σπέρμα πολὺ ἐξοίσεις εἰς τὸ πεδίον, καὶ ὀλίγα εἰσοίσεις, ὅτι κατέδεται αὐτὰ ἡ ἀκρίς·
\vs{39}Ἀμπελῶνα φυτεύσεις καὶ κατεργᾷ, καὶ οἶνον οὐ πίεσαι οὐδὲ εὐφρανθήσῃ ἐξ αὐτοῦ, ὅτι καταφάγεται αὐτὰ ὁ σκώληξ.
\vs{40}Ἐλαῖαι ἔσονταί σοι ἐν πᾶσι τοὶς ὁρίοις σου, καὶ ἔλαιον οὐ χρίσῃ, ὅτι ἐκρυήσεται ἡ ἐλαία σου.
\vs{41}Υἱοὺς καὶ θυγατέρας γεννήσεις καὶ οὐκ ἔσονται· ἀπελεύσονται γὰρ ἐν αἰχμαλωσίᾳ.
\vs{42}Πάντα τὰ ξύλινά σου, καὶ τὰ γεννήματα τῆς γῆς σου ἐξαναλώσει ἡ ἐρισύβη.
\vs{43}Ὁ προσήλυτος ὅς ἐστιν ἐν σοὶ, ἀναβήσεται ἄνω ἄνω, σὺ δὲ καταβήσῃ κάτω κάτω.
\vs{44}Οὗτος δανειεῖ σοι, σὺ δὲ τούτῳ οὐ δανειεῖς· οὗτος ἔσται κεφαλὴ, σὺ δὲ ἔσῃ οὐρά.

\vs{45}Καὶ ἐλεύσονται ἐπὶ σὲ πᾶσαι αἱ κατάραι αὗται, καὶ καταδιώξονταί σε, καὶ καταλήψονταί σε, ἔως ἂν ἐξολοθρεύσῃ σε, καὶ ἕως ἂν ἀπολέσῃ σε· ὅτι οὐκ εἰσήκουσας τῆς φωνῆς Κυρίου τοῦ Θεοῦ σου, φυλάξαι τὰς ἐντολὰς αὐτοῦ, καὶ τὰ δικαιώματα ὅσα ἐνετείλατό σοι.
\vs{46}Καὶ ἔσται ἐν σοὶ σημεῖα, καὶ τέρατα ἐν τῷ σπέρματί σου ἕως τοῦ αἰῶνος,
\vs{47}ἀνθʼ ὧν οὐκ ἐλάτρευσας Κυρίῳ τῷ Θεῷ σου ἐν εὐφροσύνῃ καὶ ἀγαθῇ διανοίᾳ διὰ τὸ πλῆθος πάντων.

\vs{48}Καὶ λατρεύσεις τοῖς ἐχθροῖς σου, οὓς ἐπαποστελεῖ Κύριος ἐπὶ σέ, ἐν λιμῷ, καὶ ἐν δίψει, καὶ ἐν γυμνότητι, καὶ ἐν ἐκλείψει πάντων· καὶ ἐπιθήσῃ κλοιὸν σιδηροῦν ἐπὶ τὸν τράχηλόν σου, ἕως ἂν ἐξολοθρεύσῃ σε.
\vs{49}Ἐπάξει ἐπὶ σὲ Κύριος ἔθνος μακρόθεν ἀπʼ ἐσχάτου τῆς γῆς ὡσεὶ ὅρμημα ἀετοῦ, ἔθνος ὃ οὐκ ἀκούσῃ τῆς φωνῆς αὐτοῦ,
\vs{50}ἔθνος ἀναιδὲς προσώπῳ, ὅστις οὐ θαυμάσει πρόσωπον πρεσβύτου, καὶ νέον οὐκ ἐλεήσει.
\vs{51}Καὶ κατέδεται τὰ ἔγκονα τῶν κτηνῶν σου, καὶ τὰ γεννήματα τῆς γῆς σου, ὥστε μὴ καταλιπεῖν σοι σῖτον, οἶνον, ἔλαιον, τὰ βουκόλια τῶν βοῶν σου, καὶ τὰ ποίμνια τῶν προβάτων σου, ἕως ἂν ἀπολέσῃ σε.
\vs{52}Καὶ ἐκτρίψῃ σε ἐν ταῖς πόλεσί σου, ἕως ἂν καθαιρεθῶσι τὰ τείχη τὰ ὑψηλὰ καὶ τὰ ὀχυρὰ, ἐφʼ οἷς σὺ πέποιθας ἐπʼ αὐτοῖς, ἐν πάσῃ τῇ γῇ σου· καὶ θλίψει σε ἐν ταῖς πόλεσί σου, αἷς ἔδωκέ σοι.
\vs{53}Καὶ φαγῇ τὰ ἔκγονα τῆς κοιλίας σου, κρέα υἱῶν σου καὶ θυγατέρων σου, ὅσα ἔδωκέ σοι, ἐν τῇ στενοχωρίᾳ σου καὶ ἐν τῇ θλίψει σου, ἡ· θλίψει σε ὁ ἐχθρός σου.

\vs{54}Ὁ ἁπαλὸς ὁ ἐν σοὶ καὶ ὁ τρυφερὸς σφόδρα, βασκανεῖ τῷ ὀφθαλμῷ αὐτοῦ τὸν ἀδελφὸν αὐτοῦ, καὶ τὴν γυναῖκα τὴν ἐν τῷ κόλπῳ αὐτοῦ, καὶ τὰ καταλελειμμένα τέκνα, ἃ ἂν καταλειφθῇ
\vs{55}αὐτῷ, ὥστε δοῦναι ἑνὶ αὐτῶν ἀπὸ τῶν σαρκῶν τῶν τέκνων αὐτοῦ, ὧν ἂν κατέσθῃ διὰ τὸ μὴ καταλειφθῆναι αὐτῷ οὐδὲν ἐν τῇ στενοχωρίᾳ σου, καὶ ἐν τῇ θλίψει σου, ἡ· ἂν θλίψωσί σε οἱ ἐχθροί σου ἐν πάσαις ταῖς πόλεσί σου.

\vs{56}Καὶ ἡ ἁπαλὴ ἐν ὑμῖν καὶ ἡ τρυφερά, ἧς οὐχὶ πεῖραν ἔλαβεν ὁ ποὺς αὐτῆς βαίνειν ἐπὶ τῆς γῆς διὰ τὴν τρυφερότητα καὶ διὰ τὴν ἁπαλότητα, βασκανεῖ τῷ ὀφθαλμῷ αὐτῆς τὸν ἄνδρα αὐτῆς τὸν ἐν κόλπῳ αὐτῆς, καὶ τὸν υἱὸν καὶ τὴν θυγατέρα αὐτῆς,
\vs{57}καὶ τὸ κόριον αὐτῆς τὸ ἐξελθὸν διὰ τῶν μηρῶν αὐτῆς, καὶ τὸ τέκνον αὐτῆς ὃ ἐὰν τέκῃ· καταφάγεται γὰρ αὐτὰ διὰ τὴν ἔνδειαν πάντων κρυφῇ ἐν τῇ στενοχωρίᾳ σου, καὶ ἐν τῇ θλίψει σου, ἡ· θλίψει σε ὁ ἐχθρός σου ἐν ταῖς πόλεσί σου,
\vs{58}ἐὰν μὴ εἰσακούσῃς ποιεῖν πάντα τὰ ῥήματα τοῦ νόμου τούτου, τὰ γεγραμμένα ἐν τῷ βιβλίῳ τούτῳ, φοβεῖσθαι τὸ ὄνομα τὸ ἔντιμον τὸ θαυμαστὸν τοῦτο, ΚΥΡΙΟΝ τὸν ΘΕΟΝ σου.
\vs{59}Καὶ παραδοξάσει Κύριος τὰς πληγάς σου, καὶ τὰς πληγὰς τοῦ σπέρματός σου, πληγὰς μεγάλας καὶ θαυμαστὰς, καὶ νόσους πονηρὰς καὶ πιστάς.
\vs{60}Καὶ ἐπιστρέψει πᾶσαν τὴν ὀδύνην Αἰγύπτου τὴν πονηρὰν, ἣν διευλαβοῦ ἀπὸ προσώπου αὐτῶν, καὶ κολληθήσονται ἐν σοί.
\vs{61}Καὶ πᾶσαν μαλακίαν, καὶ πᾶσαν πληγὴν τὴν μὴ γεγραμμένην, καὶ πᾶσαν τὴν γεγραμμένην ἐν τῷ βιβλίῳ τοῦ νόμου τούτου, ἐπάξει Κύριος ἐπὶ σέ, ἕως ἂν ἐξολοθρεύσῃ σε.
\vs{62}Καὶ καταλειφθήσεσθε ἐν ἀριθμῷ βραχεῖ, ἀνθʼ ὧν ὅτι ἦτε ὡσεὶ τὰ ἄστρα τοῦ οὐρανοῦ τῷ πλήθει, ὅτι οὐκ εἰσήκουσας τῆς φωνῆς Κυρίου τοῦ Θεοῦ σου.

\vs{63}Καὶ ἔσται ὃν τρόπον εὐφράνθη Κύριος ἐφʼ ὑμῖν εὖ ποιῆσαι ὑμᾶς, καὶ πληθῦναι ὑμᾶς, οὕτως εὐφρανθήσεται Κύριος ἐφʼ ὑμῖν ἐξολοθρεῦσαι ὑμᾶς· καὶ ἐξαρθήσεσθε ἐν τάχει ἀπὸ τῆς γῆς, εἰς ἣν εἰσπορεύῃ ἐκεῖ κληρονομῆσαι αὐτήν.
\vs{64}Καὶ διασπερεῖ σε Κύριος ὁ Θεός σου εἰς πάντα τὰ ἔθνη, ἀπʼ ἄκρου τῆς γῆς ἕως ἄκρου τῆς γῆς, καὶ δουλεύσεις ἐκεῖ θεοῖς ἑτέροις, ξύλοις καὶ λίθοις, οὓς οὐκ ἠπίστω σὺ καὶ οἱ πατέρες σου.
\vs{65}Ἀλλὰ καὶ ἐν τοῖς ἔθνεσιν ἐκείνοις οὐκ ἀναπαύσει σε, οὐδʼ οὐ μὴ γένηται στάσις τῷ ἴχνει τοῦ ποδός σου· καὶ δώσει σοι Κύριος ἐκεῖ καρδίαν ἑτέραν ἀπειθοῦσαν, καὶ ἐκλείποντας ὀφθαλμοὺς, καὶ τηκομένην ψυχήν.
\vs{66}Καὶ ἔσται ἡ ζωή σου κρεμαμένη ἀπέναντι τῶν ὀφθαλμῶν σου· καὶ φοβηθήσῃ ἡμέρας καὶ νυκτὸς, καὶ οὐ πιστεύσεις τῇ ζωῇ σου.
\vs{67}Τὸ πρωῒ ἐρεῖς, πῶς ἂν γένοιτο ἑσπέρα· καὶ τὸ ἑσπέρας ἐρεῖς, πῶς ἂν γένοιτο πρωΐ· ἀπὸ τοῦ φόβου τῆς καρδίας σου ἃ φοβηθήσῃ, καὶ ἀπὸ τῶν ὁραμάτων τῶν ὀφθαλμῶν σου ὧν ὄψῃ.
\vs{68}Καὶ ἀποστρέψει σε Κύριος εἰς Αἴγυπτον ἐν πλοίοις, ἐν τῇ ὁδῷ ἡ· εἶπα, οὐ προσθήσῃ ἔτι ἰδεῖν αὐτήν· καὶ πραθήσεσθε ἐκεῖ τοῖς ἐχθροῖς ὑμῶν εἰς παῖδας καὶ παιδίσκας, καὶ οὐκ ἔσται ὁ κτώμενος.

\vs{69}Οὗτοι οἱ λόγοι τῆς διαθήκης, οὓς ἐνετείλατο Κύριος Μωυσῇ στῆσαι τοῖς υἱοῖς Ἰσραὴλ ἐν γῇ Μωὰβ, πλὴν τῆς διαθήκης ἧς διέθετο αὐτοῖς ἐν Χωρήβ.

\ch{29}
Καὶ ἐκάλεσε Μωυσῆς πάντας τοὺς υἱοὺς Ἰσραὴλ, καὶ εἶπε πρὸς αὐτοὺς, ὑμεῖς ἑωράκατε πάντα ὅσα ἐποίησε Κύριος ἐν γῇ Αἰγύπτῳ ἐνώπιον ὑμῶν Φαραὼ καὶ τοῖς θεράπουσιν αὐτοῦ, καὶ πάσῃ τῇ γῇ αὐτοῦ,
\vs{2}τοὺς πειρασμοὺς τοὺς μεγάλους οὓς ἑωράκασιν οἱ ὀφθαλμοί σου, τὰ σημεῖα καὶ τὰ τέρατα τὰ μεγάλα ἐκεῖνα.
\vs{3}Καὶ οὐκ ἔδωκε Κύριος ὁ Θεὸς ὑμῖν καρδίαν εἰδέναι, καὶ ὀφθαλμοὺς βλέπειν, καὶ ὦτα ἀκούειν ἕως τῆς ἡμέρας ταύτης.
\vs{4}Καὶ ἤγαγεν ὑμᾶς τεσσαράκοντα ἔτη ἐν τῇ ἐρήμῳ· οὐκ ἐπαλαιώθη τὰ ἱμάτια ὑμῶν, καὶ τὰ ὑποδήματα ὑμῶν οὐ κατετρίβη ἀπὸ τῶν ποδῶν ὑμῶν.
\vs{5}Ἄρτον οὐκ ἐφάγετε, οἶνον καὶ σίκερα οὐκ ἐπίετε, ἵνα γνῶτε ὃτι Κύριος ὁ Θεὸς ὑμῶν ἐγώ.
\vs{6}Καὶ ἤλθετε ἕως τοῦ τόπου τούτου· καὶ ἑξῆλθε Σηὼν βασιλεὺς Ἐσεβὼν, καὶ Ὢγ βασιλεὺς Βασὰν εἰς συνάντησιν ἡμῖν ἐν πολέμῳ.
\vs{7}Καὶ ἐπατάξαμεν αὐτοὺς, καὶ ἐλάβομεν τὴν γῆν αὐτῶν, καὶ ἔδωκα αὐτὴν ἐν κλήρῳ τῷ Ῥουβὴν, καὶ τῷ Γαδδὶ, καὶ τῷ ἡμίσει φυλῆς Μανασσῆ.
\vs{8}Καὶ φυλάξεσθε ποιεῖν πάντας τοὺς λόγους τῆς διαθήκης ταύτης, ἵνα συνῆτε πάντα ὅσα ποιήσετε.

\vs{9}Ὑμεῖς ἑστήκατε πάντες σήμερον ἐναντίον Κυρίου τοῦ Θεοῦ ὑμῶν, οἱ ἀρχίφυλοι ὑμῶν, καὶ ἡ γερουσία ὑμῶν, καὶ οἱ κριταὶ ὑμῶν, καὶ οἱ γραμματοεισαγωγεῖς ὑμῶν, πᾶς ἀνὴρ Ἰσραήλ,
\vs{10}αἱ γυναῖκες ὑμῶν, καὶ τὰ ἔκγονα ὑμῶν καὶ ὁ προσήλυτος ὁ ἐν μέσῳ τῆς παρεμβολῆς ὑμῶν, ἀπὸ ξυλοκόπου ὑμῶν καὶ ἕως ὑδροφόρου ὑμῶν,
\vs{11}παρελθεῖν ἐν τῇ διαθήκῃ Κυρίου τοῦ Θεοῦ ὑμῶν, καὶ ἐν ταῖς ἀραῖς αὐτοῦ, ὅσα Κύριος ὁ Θεός σου διατίθεται πρὸς σὲ σήμερον·
\vs{12}ἵνα στήσῃ σε αὐτῷ εἰς λαὸν, καὶ αὐτὸς ἔσται σου Θεὸς, ὃν τρόπον εἶπέ σοι, καὶ ὃν τρόπον ὤμοσε τοῖς πατράσι σου Ἁβραὰμ καὶ Ἰσαὰκ καὶ Ἰακώβ.
\vs{13}Καὶ οὐχ ὑμῖν μόνοις ἐγὼ διατίθεμαι τὴν διαθήκην ταύτην καὶ τὴν ἀρὰν ταύτην,
\vs{14}ἀλλὰ καὶ τοῖς ὧδε οὖσι μεθʼ ὑμῶν σήμερον ἐναντίον Κυρίου τοῦ Θεοῦ ὑμῶν, καὶ τοῖς μὴ οὖσι μεθʼ ὑμῶν ὧδε σήμερον.

\vs{15}Ὅτι ὑμεῖς οἴδατε πῶς κατῳκήσαμεν ἐν γῇ Αἰγύπτῳ, ὡς παρήλθομεν ἐν μέσῳ τῶν ἐθνῶν οὓς παρήλθετε.
\vs{16}Καὶ ἴδετε τὰ βδελύγματα αὐτῶν, καὶ τὰ εἴδωλα αὐτῶν, ξύλον καὶ λίθον, ἀργύριον καὶ χρυσίον, ἅ ἐστι παρʼ αὐτοῖς.
\vs{17}Μή τις ἐστὶν ἐν ὑμῖν ἀνὴρ, ἢ γυνὴ, ἢ πατριὰ, ἢ φυλὴ, τινὸς ἡ διάνοια ἐξέκλινεν ἀπὸ Κυρίου τοῦ Θεου ὑμῶν, πορευθέντες λατρεύειν τοῖς θεοῖς τῶν ἐθνῶν ἐκείνων· μή τις ἐστὶν ἐν ὑμῖν ῥίζα ἄνω φύουσα ἐν χολῇ καὶ πικρίᾳ·
\vs{18}Καὶ ἔσται ἐὰν ἀκούσῃ τὰ ῥήματα τῆς ἀρᾶς ταύτης, καὶ ἐπιφημίσηται ἐν τῇ καρδίᾳ αὐτοῦ, λέγων, ὅσιά μοι γένοιτο, ὅτι ἐν τῇ ἀποπλανήσει τῆς καρδίας μου πορεύσομαι, ἵνα μὴ συναπολέσῃ ὁ ἁμαρτωλὸς τὸν ἀναμάρτητον·
\vs{19}Οὐ μὴ θελήσει ὁ Θεὸς εὐϊλατεῦσαι αὐτῷ, ἀλλʼ ἢ τότε ἐκκαυθήσεται ὀργὴ Κυρίου καὶ ὁ ζῆλος αὐτοῦ ἐν τῷ ἀνθρώπῳ ἐκείνῳ· καὶ κολληθήσονται ἐν αὐτῷ πᾶσαι αἱ ἀραὶ τῆς διαθήκης ταύτης, αἱ γεγραμμέναι ἐν τῷ βιβλίῳ τούτῳ· καὶ ἐξαλείψει Κύριος τὸ ὄνομα αὐτοῦ ἐκ τῆς ὑπὸ τὸν οὐρανόν.
\vs{20}Καὶ διαστελεῖ αὐτὸν Κύριος εἰς κακὰ ἐκ πάντων υἱῶν Ἰσραὴλ, κατὰ πάσας τὰς ἀρὰς τῆς διαθήκης τὰς γεγραμμένας ἐν τῷ βιβλίῳ τοῦ νόμου τούτου.

\vs{21}Καὶ ἐροῦσιν ἡ γενεὰ ἡ ἑτέρα οἱ υἱοὶ ὑμῶν, οἳ ἀναστήσονται μεθʼ ὑμᾶς, καὶ ὁ ἀλλότριος ὃς ἂν ἔλθῃ ἐκ γῆς μακρόθεν, καὶ ὄψονται τὰς πληγὰς τῆς γῆς ἐκείνης καὶ τὰς νόσους αὐτῆς, ἃς ἀπέστειλε Κύριος ἐπʼ αὐτὴν,
\vs{22}θεῖον καὶ ἅλα κατακεκαυμένον· πᾶσα ἡ γῆ αὐτῆς οὐ σπαρήσεται, οὐδὲ ἀνατελεῖ, οὐδὲ μὴ ἀναβῇ ἐπʼ αὐτὴν πᾶν χλωρόν. ὥσπερ κατεστράφη Σόδομα καὶ Γόμοῤῥα, Ἀδαμὰ καὶ Σεβωῒμ, ἃς κατέστρεψε Κύριος ἐν θυμῷ καὶ ὀργῇ·
\vs{23}Καὶ ἐροῦσι πάντα τὰ ἔθνη, διατί ἐποίησε Κύριος οὕτω τῇ γῇ ταύτῃ; τίς ὁ θυμὸς τῆς ὀργῆς ὁ μέγας οὗτος;
\vs{24}Καὶ ἐροῦσιν, ὅτι κατέλιπον τὴν διαθήκην Κυρίου τοῦ Θεοῦ τῶν πατέρων αὐτῶν, ἃ διέθετο τοῖς πατράσιν αὐτῶν, ὅτε ἐξήγαγεν αὐτοὺς ἐκ γῆς Αἰγύπτου,
\vs{25}καὶ πορεύθεντες ἐλάτρευσαν θεοῖς ἑτέροις, οὓς οὐκ ἠπίσταντο, οὐδὲ διένειμεν αὐτοῖς·
\vs{26}Καὶ ὠργίσθη θυμῷ Κύριος ἐπὶ τὴν γῆν ἐκείνην ἐπαγαγεῖν ἐπʼ αὐτὴν κατὰ πάσας τὰς κατάρας τὰς γεγραμμένας ἐν τῷ βιβλίῳ τοῦ νόμου τούτου.
\vs{27}Καὶ ἐξῇρεν αὐτοὺς Κύριος ἀπὸ τῆς γῆς αὐτῶν ἐν θυμῷ καὶ ὀργῇ καὶ παροξυσμῷ μεγάλῳ σφόδρα, καὶ ἐξέβαλεν αὐτοὺς εἰς γῆν ἑτέραν ὡσεὶ νῦν.

\vs{28}Τὰ κρυπτὰ Κυρίῳ τῷ Θεῷ ἡμῶν, τὰ δὲ φανερὰ ἡμῖν καὶ τοῖς τέκνοις ἡμῶν εἰς τὸν αἰῶνα, ποιεῖν πάντα τὰ ῥήματα τοῦ νόμου τούτου.

\ch{30}
Καὶ ἔσται ὡς ἂν ἔλθωσιν ἐπὶ σὲ πάντα τὰ ῥήματα ταῦτα, ἡ εὐλογία καὶ ἡ κατάρα, ἣν ἔδωκα πρὸ προσώπου σου; καὶ δέξῃ εἰς τὴν καρδίαν σου ἐν πᾶσι τοῖς ἔθνεσιν, οὗ ἐὰν σε διασκορπίσῃ σε Κύριος ἐκεῖ,
\vs{2}καὶ ἐπιστραφήσῃ ἐπὶ Κύριον τὸν Θεόν σου, καὶ εἰσακούσῃ τῆς φωνῆς αὐτοῦ κατὰ πάντα ὅσα ἐγὼ ἐντέλλομαί σοι σήμερον, ἐξ ὅλης τῆς καρδίας σου, καὶ ἐξ ὅλης τῆς ψυχῆς σου,
\vs{3}καὶ ἰάσεται Κύριος τὰς ἁμαρτίας σου, καὶ ἐλεήσει σε, καὶ πάλιν συνάξει σε ἐκ πάντων τῶν ἐθνῶν, εἰς οὓς διεσκόρπισέ σε Κύριος ἐκεῖ.
\vs{4}Ἐὰν ἠ· ἡ διασπορά σου ἀπʼ ἄκρου τοῦ οὐρανοῦ ἕως ἄκρου τοῦ οὐρανοῦ, ἐκεῖθεν συνάξει σε Κύριος ὁ Θεός σου, καὶ ἐκεῖθεν λήψεταί σε Κύριος ὁ Θεός σου.
\vs{5}Καὶ εἰσάξει σε ὁ Θεός σου ἐκεῖθεν εἰς τὴν γῆν ἣν ἐκληρονόμησαν οἱ πατέρες σου, καὶ κληρονομήσεις αὐτήν· καὶ εὖ σε ποιήσει, καὶ πλεοναστόν σε ποιήσει ὑπὲρ τοὺς πατέρας σου.
\vs{6}Καὶ περικαθαριεῖ Κύριος τὴν καρδίαν σου, καὶ τὴν καρδίαν τοῦ σπέρματός σου, ἀγαπᾷν Κύριον τὸν Θεόν σου ἐξ ὅλης τῆς καρδίας σου, καὶ ἐξ ὅλης τῆς ψυχῆς σου, ἵνα ζῇς σύ.

\vs{7}Καὶ δώσει Κύριος ὁ Θεός σου τὰς ἀρὰς ταύτας ἐπὶ τοὺς ἐχθρούς σου, καὶ ἐπὶ τοὺς μισοῦντάς σε, οἳ ἐδίωξάν σε.
\vs{8}Καὶ σὺ ἐπιστραφήσῃ καὶ εἰσακούσῃ τῆς φωνῆς Κυρίου τοῦ Θεοῦ σου, καὶ ποιήσεις τὰς ἐντολὰς αὐτοῦ, ὅσας ἐγὼ ἐντέλλομαί σοι σήμερον.
\vs{9}Καὶ εὐλογήσει σε Κύριος ὁ Θεός σου ἐν παντὶ ἔργῳ τῶν χειρῶν σου, ἐν τοῖς ἐκγόνοις τῆς κοιλίας σου, καὶ ἐν τοῖς ἐκγόνοις τῶν κτηνῶν σου, καὶ ἐν τοῖς γεννήμασι τῆς γῆς σου, ὅτι ἐπιστρέψει Κύριος ὁ Θεός σου εὐφρανθῆναὶ ἐπὶ σοὶ εἰς ἀγαθὰ, καθότι εὐφράνθη ἐπὶ τοῖς πατράσι σου·
\vs{10}Ἐὰν εἰσακούσῃς τῆς φωνῆς Κυρίου τοῦ Θεοῦ σου, φυλάσσεσθαι τὰς ἐντολὰς αὐτοῦ, καὶ τὰ δικαιώματα αὐτοῦ, καὶ τὰς κρίσεις αὐτοῦ τὰς γεγραμμένας ἐν τῷ βιβλίῳ τοῦ νόμου τούτου· ἐὰν ἐπιστραφῇς ἐπὶ Κύριον τὸν Θεόν σου ἐξ ὅλης τῆς καρδίας σου, καὶ ἐξ ὅλης τῆς ψυχῆς σου.
\vs{11}Ὅτι ἡ ἐντολὴ αὕτη ἣν ἐγὼ ἐντέλλομαί σοι σήμερον, οὐχ ὑπέρογκός ἐστιν, οὐδέ μακρὰν ἀπὸ σοῦ ἐστιν.
\vs{12}Οὐκ ἐν τῷ οὐρανῷ ἄνω ἐστὶ, λέγων, τίς ἀναβήσεται ἡμῖν εἰς τὸν οὐρανὸν, καὶ λήψεται ἡμῖν αὐτὴν, καὶ ἀκούσαντες αὐτὴν ποιήσομεν;
\vs{13}Οὐδὲ πέραν τῆς θαλάσσης ἐστί, λέγων, τίς διαπεράσει ἡμῖν εἰς τὸ πέραν τῆς θαλάσσης, καὶ λάβῃ ἡμῖν αὐτὴν, καὶ ἀκουστὴν ἡμῖν ποιήσῃ αὐτὴν, καὶ ποιήσομεν;
\vs{14}Ἐγγύς σου ἐστὶ τὸ ῥῆμα σφόδρα ἐν τῷ στόματί σου, καὶ ἐν τῇ καρδίᾳ σου, καὶ ἐν ταῖς χερσί σου ποιεῖν αὐτό.

\vs{15}Ἰδοὺ δέδωκα πρὸ προσώπου σου σήμερον τὴν ζωὴν καὶ τὸν θάνατον, τὸ ἀγαθὸν καὶ τὸ κακόν.
\vs{16}Ἐὰν εἰσακούσῃς τὰς ἐντολὰς Κυρίου τοῦ Θεοῦ σου, ἃς ἐγὼ ἐντέλλομαί σοι σήμερον, ἀγαπᾷν Κύριον τὸν Θεόν σου, πορεύεσθαι ἐν πάσαις ταῖς ὁδοῖς αὐτοῦ, καὶ φυλάσσεσθαι τὰ δικαιώματα αὐτοῦ, καὶ τὰς κρίσεις αὐτοῦ, καὶ ζήσεσθε, καὶ πολλοὶ ἔσεσθε, καὶ εὐλογήσει σε Κύριος ὁ Θεός σου ἐν πάσῃ τῇ γῇ, εἰς ἣν εἰσπορεύῃ ἐκεῖ κληρονομῆσαι αὐτήν.
\vs{17}Καὶ ἐὰν μεταστῇ ἡ καρδία σου, καὶ μὴ εἰσακούσῃς, καὶ πλανηθεὶς προσκυνήσῃς θεοῖς ἑτέροις καὶ λατρεύσῃς αὐτοῖς,
\vs{18}ἀναγγέλλω σοι σήμερον, ὅτι ἀπωλείᾳ ἀπολεῖσθε, καὶ οὐ μὴ πολυήμεροι γένησθε ἐπὶ τῆς γῆς, εἰς ἣν ὑμεῖς διαβαίνετε τὸν Ἰορδάνην ἐκεῖ κληρονομῆσαι αὐτήν.

\vs{19}Διαμαρτύρομαι ὑμῖν σήμερον τόν τε οὐρανὸν καὶ τὴν γῆν, τὴν ζωὴν καὶ τὸν θάνατον δέδωκα πρὸ προσώπου ὑμῶν, τὴν εὐλογίαν καὶ τὴν κατάραν· ἔκλεξαι τὴν ζωὴν σὺ, ἵνα ζήσῃς σὺ καὶ τὸ σπέρμα σου,
\vs{20}ἀγαπᾷν Κύριον τὸν Θεόν σου, εἰσακούειν τῆς φωνῆς αὐτοῦ, καὶ ἔχεσθαι αὐτοῦ· ὅτι τοῦτο ἡ ζωή σου καὶ ἡ μακρότης τῶν ἡμερῶν σου, τὸ κατοικεῖν ἐπὶ τῆς γῆς, ἧς ὤμοσε Κύριος τοῖς πατράσι σου Ἁβραὰμ καὶ Ἰσαὰκ καὶ Ἰακὼβ δοῦναι αὐτοῖς.

\ch{31}
Καὶ συνετέλεσε Μωυσῆς λαλῶν πάντας τοὺς λόγους τούτους πρὸς πάντας υἱοὺς Ἰσραὴλ,
\vs{2}καὶ εἶπε πρὸς αὐτοὺς, ἐκατὸν καὶ εἴκοσι ἐτῶν ἐγώ εἰμι σήμερον· οὐ δυνήσομαι ἔτι εἰσπορεύεσθαι καὶ ἐκπορεύεσθαι· Κύριος δὲ εἶπε πρὸς μὲ, οὐ διαβήσῃ τὸν Ιορδάνην τοῦτον.
\vs{3}Κύριος ὁ Θεός σου ὁ προπορευόμενος πρὸ προσώπου σου, οὗτος ἐξολοθρεύσει τὰ ἔθνη ταῦτα ἀπὸ προσώπου σου, καὶ κατακληρονομήσεις αὐτούς· καὶ Ἰησοῦς ὁ προπορευόμενος πρὸ προσώπου σου, καθὰ ἐλάλησε Κύριος.
\vs{4}Καὶ ποιήσει Κύριος ὁ Θεός σου αὐτοῖς καθὼς ἐποίησε Σηὼν καὶ Ὢγ δυσὶ βασιλεῦσι τῶν Ἀμοῤῥαίων, οἳ ἦσαν πέραν τοῦ Ἰορδάνου, καὶ τῇ γῇ αὐτῶν, καθότι ἐξωλόθρευσεν αὐτούς.
\vs{5}Καὶ παρέδωκεν αὐτοὺς Κύριος ὑμῖν· καὶ ποιήσετε αὐτοῖς, καθότι ἐνετειλάμην ὑμῖν.
\vs{6}Ἀνδρίζου καὶ ἴσχυε, μὴ φοβοῦ, μηδὲ δειλιάσῃς, μηδὲ πτοηθῇς ἀπὸ προσώπου αὐτῶν· ὅτι Κύριος ὁ Θεός σου ὁ προπορευόμενος μεθʼ ὑμῶν ἐν ὑμῖν, οὔτε μή σε ἀνῇ, οὔτε μή σε ἐγκαταλίπῃ.
\vs{7}Καὶ ἐκάλεσε Μωυσῆς Ἰησοῦν, καὶ εἶπεν αὐτῷ ἔναντι παντὸς Ἰσραὴλ, ἀνδρίζου καὶ ἴσχυε, σὺ γὰρ εἰσελεύσῃ πρὸ προσώπου τοῦ λαοῦ τούτου εἰς τὴν γῆν ἣν ὤμοσε Κύριος τοῖς πατράσιν ὑμῶν δοῦναι αὐτοῖς, καὶ σὺ κατακληρονομήσεις αὐτοῖς.
\vs{8}Καὶ Κύριος ὁ συμπορευόμενος μετὰ σοῦ, οὐκ ἀνήσει σε, οὐδὲ μὴ σε ἐγκαταλίπῃ· μή φοβοῦ, μηδὲ δειλία.

\vs{9}Καὶ ἔγραψε Μωυσῆς τὰ ῥήματα τοῦ νόμου τούτου εἰς βιβλίον, καὶ ἔδωκε τοῖς ἱερεῦσι τοῖς υἱοῖς Λευὶ τοῖς αἴρουσι τὴν κιβωτὸν τῆς διαθήκης Κυρίου, καὶ τοῖς πρεσβυτέροις τῶν υἱῶν Ἰσραήλ.

\vs{10}Καὶ ἐνετείλατο Μωυσῆς αὐτοῖς ἐν τῇ ἡμέρᾳ ἐκείνῃ, λέγων, μετὰ ἑπτὰ ἔτη ἐν καιρῷ ἐνιαυτοῦ ἀφέσεως ἐν ἑορτῇ σκηνοπηγίας,
\vs{11}ἐν τῷ συμπορεύεσθαι πάντα Ἰσραὴλ ὀφθῆναι ἐνώπιον Κύριου τοῦ Θεοῦ ὑμῶν, ἐν τῷ τόπῳ ᾧ ἂν ἐκλέξηται Κύριος, ἀναγνώσεσθε τὸν νόμον τοῦτον ἐναντίον παντὸς Ἰσραὴλ εἰς τὰ ὦτα αὐτῶν,
\vs{12}ἐκκλησιάσας τὸν λαὸν, τοὺς ἄνδρας καὶ τὰς γυναῖκας καὶ τὰ ἔκγονα καὶ τὸν προσήλυτον τὸν ἐν ταῖς πόλεσιν ὑμῶν, ἵνʼ ἀκούσωσι, καὶ ἵνα μάθωσι φοβεῖσθαι Κύριον τὸν Θεὸν ὑμῶν· καὶ ἀκούσονται ποιεῖν πάντας τοὺς λόγους τοῦ νόμου τούτου.
\vs{13}Καὶ οἱ υἱοὶ αὐτῶν οἳ οὐκ οἴδασιν, ἀκούσονται, καὶ μαθήσονται φοβεῖσθαι Κύριον τὸν Θεόν σου πάσας τὰς ἡμέρας ὅσας αὐτοὶ ζῶσιν ἐπὶ τῆς γῆς, εἰς ἣν ὑμεῖς διαβαίνετε τὸν Ἰορδάνην ἐκεῖ κληρονομῆσαι αὐτήν.

\vs{14}Καὶ εἶπε Κύριος πρὸς Μωυσῆν, ἰδοὺ ἐγγίκασιν αἱ ἡμέραι τοῦ θανάτου σου· κάλεσον Ἰησοῦν, καὶ στῆτε παρὰ τὰς θύρας τῆς σκηνῆς τοῦ μαρτυρίου, καὶ ἐντελοῦμαι αὐτῷ· καὶ ἐπορεύθη Μωυσῆς καὶ Ἰησοῦς εἰς τὴν σκηνὴν τοῦ μαρτυρίου, καὶ ἔστησαν παρὰ τὰς θύρας τῆς σκηνῆς τοῦ μαρτυρίου.
\vs{15}Καὶ κατέβη Κύριος ἐν νεφέλῃ, καὶ ἔστη παρὰ τὰς θύρας τῆς σκηνῆς τοῦ μαρτυρίου· καὶ ἔστη ὁ στύλος τῆς νεφέλης παρὰ τὰς θύρας τῆς σκηνῆς τοῦ μαρτυρίου.
\vs{16}Καὶ εἶπε Κύριος πρὸς Μωυσῆν, ἰδοὺ σὺ κοιμᾷ μετὰ τῶν πατέρων σου, καὶ ἀναστὰς οὗτος ὁ λαὸς ἐκπορνεύσει ὀπίσω θεῶν ἀλλοτρίων τῆς γῆς, εἰς ἣν οὗτος εἰσπορεύεται, καὶ καταλείψουσί με, καὶ διασκεδάσουσι τὴν διαθήκην μου, ἣν διεθέμην αὐτοῖς.
\vs{17}Καὶ ὀργισθήσομαι θυμῷ εἰς αὐτοὺς ἐν τῇ ἡμέρᾳ ἐκείνῃ, καὶ καταλείψω αὐτοὺς, καὶ ἀποστρέψω τὸ πρόσωπόν μου ἀπʼ αὐτῶν, καὶ ἔσται κατάβρωμα· καὶ εὑρήσουσιν αὐτὸν κακὰ πολλὰ καὶ θλίψεις· καὶ ἐρεῖ ἐν τῇ ἡμέρᾳ ἐκείνῃ, διότι οὐκ ἔστι Κύριος ὁ Θεός μου ἐν ἐμοὶ, εὕροσάν με τὰ κακὰ ταῦτα.
\vs{18}Ἐγὼ δὲ ἀποστροφῇ ἀποστρέψω τὸ πρόσωπόν μου ἀπʼ αὐτῶν ἐν τῇ ἡμέρᾳ ἐκείνῃ, διὰ πάσας τὰς κακίας ἃς ἐποίησαν, ὅτι ἀπέστρεψαν ἐπὶ θεοὺς ἀλλοτρίους.

\vs{19}Καὶ νῦν γράψατε τὰ ῥήματα τῆς ᾠδῆς ταύτης, καὶ διδάξατε αὐτὴν τοὺς υἱοὺς Ἰσραὴλ, καὶ ἐμβαλεῖτε αὐτὴν εἰς τὸ στόμα αὐτῶν, ἵνα γένηταί μοι ἡ ᾠδὴ αὕτη κατὰ πρόσωπον μαρτυροῦσα ἐν υἱοῖς Ἰσραήλ.
\vs{20}Εἰσάξω γὰρ αὐτοὺς εἰς τὴν γῆν τὴν ἀγαθὴν, ἣν ὤμοσα τοῖς πατράσιν αὐτῶν, δοῦναι αὐτοῖς γῆν ῥέουσαν γάλα καὶ μέλι, καὶ φάγονται, καὶ ἐμπλησθέντες κορήσουσι, καὶ ἐπιστραφήσονται ἐπὶ θεοὺς ἀλλοτρίους, καὶ λατρεύσουσιν αὐτοῖς, καὶ παροξυνοῦσί με, καὶ διασκεδάσουσι τὴν διαθήκην μου.
\vs{21}Καὶ ἀντικαταστήσεται ἡ ᾠδὴ αὕτη κατὰ πρόσωπον μαρτυροῦσα· οὐ γὰρ μὴ ἐπιλησθῇ ἀπὸ στόματος αὐτῶν, καὶ ἀπὸ στόματος τοῦ σπέρματος αὐτῶν· ἐγὼ γὰρ οἶδα τὴν πονηρίαν αὐτῶν, ὅσα ποιοῦσιν ὧδε σήμερον, πρὸ τοῦ εἰσαγαγεῖν με αὐτοὺς εἰς τὴν γῆν τὴν ἀγαθὴν, ἣν ὤμοσα τοῖς πατράσιν αὐτῶν.

\vs{22}Καὶ ἔγραψε Μωυσῆς τὴν ᾠδὴν ταύτην ἐν ἐκείνῃ τῇ ἡμέρᾳ, καὶ ἐδίδαξεν αὐτὴν τοὺς υἱοὺς Ἰσραήλ.
\vs{23}Καὶ ἐνετείλατο Ἰησοῖ, καὶ εἶπεν, ἀνδρίζου καὶ ἴσχυε, σὺ γὰρ εἰσάξεις τοὺς υἱοὺς Ἰσραὴλ εἰς τὴν γῆν, ἣν ὤμοσεν αὐτοῖς Κύριος, καὶ αὐτὸς ἔσται μετὰ σοῦ.

\vs{24}Ἡνίκα δὲ συνετέλεσε Μωυσῆς γράφων πάντας τοὺς λόγους τοῦ νόμου τούτου εἰς βιβλίον ἕως εἰς τέλος,
\vs{25}καὶ ἐνετείλατο τοῖς Λευίταις τοῖς αἴρουσι τὴν κιβωτὸν τῆς διαθήκης Κυρίου, λέγων,
\vs{26}λαβόντες τὸ βιβλίον τοῦ νόμου τούτου, θήσετε αὐτὸ ἐκ πλαγίων τῆς κιβωτοῦ τῆς διαθήκης Κυρίου τοῦ Θεοῦ ὑμῶν· καὶ ἔσται ἐκεῖ ἐν σοὶ εἰς μαρτύριον.
\vs{27}Ὅτι ἐγὼ ἐπίσταμαι τὸν ἐρεθισμόν σου, καὶ τὸν τράχηλόν σου τὸν σκληρόν· ἔτι γὰρ ἐμοῦ ζῶντος μεθʼ ὑμῶν σήμερον, παραπικραίνοντες ἦτε τὰ πρὸς τὸν Θεόν· πῶς οὐχὶ καὶ ἔσχατον τοῦ θανάτου μου;
\vs{28}Ἐκκλησιάσατε πρὸς μὲ τοὺς φυλάρχους ὑμῶν, καὶ τοὺς πρεσβυτέρους ὑμῶν, καὶ τοὺς κριτὰς ὑμῶν, καὶ τοὺς γραμματοεισαγωγεῖς ὑμῶν, ἵνα λαλήσω εἰς τὰ ὦτα αὐτῶν πάντας τοὺς λόγους τούτους· καὶ διαμαρτύρομαι αὐτοῖς τόν τε οὐρανὸν καὶ τὴν γῆν.
\vs{29}Οἶδα γὰρ ὅτι ἔσχατον τῆς τελευτῆς μου ἀνομίᾳ ἀνομήσετε, καὶ ἐκκλινεῖτε ἐκ τῆς ὁδοῦ ἧς ἐνετειλάμην ὑμῖν, καὶ συναντήσεται ὑμῖν τὰ κακὰ ἔσχατον τῶν ἡμερῶν, ὅτι ποιήσετε τὰ πονηρὰ ἐναντίον Κυρίου, παροργίσαι αὐτὸν ἐν τοῖς ἔργοις τῶν χειρῶν ὑμῶν.

\vs{30}Καὶ ἐλάλησε Μωυσῆς εἰς τὰ ὦτα πάσης ἐκκλησίας τὰ ῥήματα τῆς ᾠδῆς ταύτης ἕως εἰς τέλος.

\ch{32}
Πρόσεχε οὐρανὲ, καὶ λαλήσω, καὶ ἀκουέτω ἡ γῆ ῥήματα ἐκ στόματός μου.
\vs{2}Προσδοκάσθω ὡς ὑετὸς τὸ ἀπόφθεγμά μου, καὶ καταβήτω ὡς δρόσος τὰ ῥήματά μου, ὡσεὶ ὄμβρος ἐπʼ ἄγρωστιν, καὶ ὡσεὶ νιφετὸς ἐπὶ χόρτον.
\vs{3}Ὅτι τὸ ὄνομα Κυρίου ἐκάλεσα· δότε μεγαλωσύνην τῷ Θεῷ ἡμῶν.
\vs{4}Θεὸς, ἀληθινὰ τὰ ἔργα αὐτοῦ, καὶ πᾶσαι αἱ ὁδοὶ αὐτοῦ κρίσεις· Θεὸς πιστὸς, καὶ οὐκ ἔστιν ἀδικία· δίκαιος καὶ ὅσιος Κύριος.
\vs{5}Ἡμάρτοσαν οὐκ αὐτῷ τέκνα μωμητά· γενεὰ σκολιὰ καὶ διεστραμμένη.
\vs{6}Ταῦτα Κυρίῳ ἀνταποδίδοτε; οὕτω λαὸς μωρὸς καὶ οὐχὶ σοφός; οὐκ αὐτὸς οὗτός σου πατὴρ ἐκτήσατό σε καὶ ἐποίησέ σε καὶ ἔπλασέ σε;
\vs{7}Μνήσθητε ἡμέρας αἰῶνος, σύνετε ἔτη γενεῶν γενεαῖς. ἐπερώτησον τὸν πατέρα σου καὶ ἀναγγελεῖ σοι, τοὺς πρεσβυτέρους σου καὶ ἐροῦσί σοι.

\vs{8}Ὅτε διεμέριζεν ὁ ὕψιστος ἔθνη, ὡς διέσπειρεν υἱοὺς Αδὰμ, ἔστησεν ὅρια ἐθνῶν κατὰ ἀριθμὸν ἀγγέλων Θεοῦ.
\vs{9}Καὶ ἐγενήθη μερὶς Κυρίου λαὸς αὐτοῦ Ἰακώβ· σχοίνισμα κληρονομίας αὐτοῦ Ἰσραήλ.
\vs{10}Αὐτάρκησεν αὐτὸν ἐν τῇ ἐρήμῳ, ἐν δίψει καύματος ἐν γῇ ἀνύδρῳ· ἐκύκλωσεν αὐτὸν καὶ ἐπαίδευσεν αὐτὸν, καὶ διεφύλαξεν αὐτὸν, ὡς κόρην ὀφθαλμοῦ·
\vs{11}Ὡς ἀετὸς σκεπάσαι νοσσιὰν αὐτοῦ, καὶ ἐπὶ τοῖς νοσσοῖς αὐτοῦ ἐπεπόθησε, διεὶς τὰς πτέρυγας αὐτοῦ ἐδέξατο αὐτοὺς, καὶ ἀνέλαβεν αὐτοὺς ἐπὶ τῶν μεταφρένων αὐτοῦ.
\vs{12}Κύριος μόνος ἦγεν αὐτοὺς, οὐκ ἦν μετʼ αὐτῶν θεὸς ἀλλότριος.
\vs{13}Ἀνεβίβασεν αὐτοὺς ἐπὶ τὴν ἰσχὺν τῆς γῆς· ἐψώμισεν αὐτοὺς γεννήματα ἀγρῶν· ἐθήλασαν μέλι ἐκ πέτρας, καὶ ἔλαιον ἐκ στερεᾶς πέτρας.
\vs{14}Βούτυρον βοῶν, καὶ γάλα προβάτων, μετὰ στέατος ἀρνῶν καὶ κριῶν, υἱῶν ταύρων καὶ τράγων, μετὰ στέατος νεφρῶν πυροῦ, καὶ αἷμα σταφυλῆς ἔπιεν οἶνον.
\vs{15}Καὶ ἔφαγεν Ἰακὼβ καὶ ἐνεπλήσθη, καὶ ἀπελάκτισεν ὁ ἠγαπημένος· ἐλιπάνθη, ἐπαχύνθη, ἐπλατύνθη, καὶ ἐγκατέλιπε τὸν Θεὸν τὸν ποιήσαντα αὐτὸν, καὶ ἀπέστη ἀπὸ Θεοῦ σωτῆρος αὐτοῦ.

\vs{16}Παρώξυνάν με ἐπʼ ἀλλοτρίοις· ἐν βδελύγμασιν αὐτῶν παρεπίκρανάν με.
\vs{17}Εθυσαν δαιμονίοις, καὶ οὐ Θεῷ· θεοῖς οἷς οὐκ ᾔδεισαν· καινοὶ πρόσφατοι ἥκασιν, οὓς οὐκ ᾔδεισαν οἱ πατέρες αὐτῶν.
\vs{18}Θεὸν τὸν γεννήσαντά σε ἐγκατέλιπες, καὶ ἐπελάθου Θεοῦ τοῦ τρέφοντός σε.

\vs{19}Καὶ εἶδε Κύριος, καὶ ἐζήλωσε· καὶ παρωξύνθη διʼ ὀργὴν υἱῶν αὐτοῦ καὶ θυγατέρων,
\vs{20}καὶ εἶπεν, ἀποστρέψω τὸ πρόσωπόν μου ἀπʼ αὐτῶν, καὶ δείξω τί ἔσται αὐτοῖς ἐπʼ ἐσχάτων ἡμερῶν· ὅτι γενεὰ ἐξεστραμμένη ἐστίν, υἱοὶ οἷς οὐκ ἔστι πίστις ἐν αὐτοῖς.

\vs{21}Αὐτοὶ παρεζήλωσάν με ἐπʼ οὐ Θεῷ, παρώξυνάν με ἐν τοῖς εἰδώλοις αὐτῶν· κᾀγὼ παραζηλώσω αὐτοὺς ἐπʼ οὐκ ἔθνει, ἐπὶ ἔθνει ἀσυνέτῳ παροργιῶ αὐτούς.
\vs{22}Ὅτι πῦρ ἐκκέκαυται ἐκ τοῦ θυμοῦ μου, καυθήσεται ἕως ᾄδου κάτω· καταφάγεται γῆν καὶ τὰ γεννήματα αὐτῆς· φλέξει θεμέλια ὀρέων.
\vs{23}Συνάξω εἰς αὐτοὺς κακὰ, καὶ τὰ βέλη μου συμπολεμήσω εἰς αὐτούς.
\vs{24}Τηκόμενοι λιμῷ καὶ βρώσει ὀρνέων, καὶ ὀπισθότονος ἀνίατος· ὀδόντας θηρίων ἐπαποστελῶ εἰς αὐτοὺς, μετὰ θυμοῦ συρόντων ἐπὶ γῆν.
\vs{25}Ἔξωθεν ἀτεκνώσει αὐτοὺς μάχαιρα, καὶ ἐκ τῶν ταμιείων, φόβος· νεανίσκος σὺν παρθένῳ, θηλάζων μετὰ καθεστηκότος πρεσβύτου.
\vs{26}Εἶπα, διασπερῶ αὐτοὺς, παύσω δὲ ἐξ ἀνθρώπων τὸ μνημόσυνον αὐτῶν.
\vs{27}Εἰ μὴ διʼ ὀργὴν ἐχθρῶν, ἵνα μὴ μακροχρονίσωσι, ἵνα μὴ συνεπιθῶνται οἱ ὑπεναντίοι· μὴ εἴπωσιν, ἡ χεὶρ ἡμῶν ἡ ὑψηλὴ, καὶ οὐχὶ Κύριος, ἐποίησε ταῦτα πάντα.

\vs{28}Ἔθνος ἀπολωλεκὸς βουλήν ἐστι, καὶ οὐκ ἔστιν ἐν αὐτοῖς ἐπιστήμη.
\vs{29}Οὐκ ἐφρόνησαν συνιέναι· ταῦτα καταδεξάσθωσαν εἰς τὸν ἐπιόντα χρόνον.
\vs{30}Πῶς διώξεται εἷς χιλίους, καὶ δύο μετακινήσουσι μυριάδας, εἰ μὴ ὁ Θεὸς ἀπέδοτο αὐτοὺς, καὶ Κύριος παρέδωκεν αὐτούς;
\vs{31}Ὅτι οὐκ εἰσὶν ὡς ὁ Θεὸς ἡμῶν οἱ θεοὶ αὐτῶν· οἱ δὲ ἐχθροὶ ἡμῶν ἀνόητοι.
\vs{32}Ἐκ γὰρ ἀμπέλου Σοδόμων ἡ ἄμπελος αὐτῶν, καὶ ἡ κληματὶς αὐτῶν ἐκ Γομόῤῥας· σταφυλὴ αὐτῶν σταφυλὴ χολῆς, βότρυς πικρίας αὐτοῖς.
\vs{33}Θυμὸς δρακόντων ὁ οἶνος αὐτῶν, καὶ θυμὸς ἀσπίδων ἀνίατος.
\vs{34}Οὐκ ἰδοὺ ταῦτα συνῆκται παρʼ ἐμοὶ, καὶ ἐσφράγισται ἐν τοῖς θησαυροῖς μου;
\vs{35}Ἐν ἡμέρᾳ ἐκδικήσεως ἀνταποδώσω, ὅταν σφαλῇ ὁ ποὺς αὐτῶν· ὅτι ἐγγὺς ἡμέρα ἀπωλίας αὐτοῖς, καὶ πάρεστιν ἕτοιμα ὑμῖν.
\vs{36}Ὅτι κρινεῖ Κύριος τὸν λαὸν αὐτοῦ, καὶ ἐπὶ τοῖς δούλοις αὐτοῦ παρακληθήσεται· εἶδε γὰρ παραλελυμένους αὐτοὺς, καὶ ἐκλελοιπότας ἐν ἐπαγωγῇ, καὶ παρειμένους·
\vs{37}Καὶ εἶπε Κύριος, ποῦ εἰσιν οἱ θεοὶ αὐτῶν, ἐφʼ οἷς ἐπεποίθεισαν ἐπʼ αὐτοῖς,
\vs{38}ὧν τὸ στέαρ τῶν θυσιῶν αὐτῶν ἠσθίετε, καὶ ἐπίνετε τὸν οἶνον τῶν σπονδῶν αὐτῶν; ἀναστήτωσαν καὶ βοηθησάτωσαν ὑμῖν καὶ γενηθήτωσαν ὑμῖν σκεπασταί.
\vs{39}Ἴδετε ἴδετε ὅτι ἐγώ εἰμι, καὶ οὐκ ἔστι Θεὸς πλὴν ἐμοῦ· ἐγὼ ἀποκτείνω, καὶ ζῇν ποιήσω· πατάξω, κἀγὼ ἰάσομαι· καὶ οὐκ ἔστιν ὃς ἐξελεῖται ἐκ τῶν χειρῶν μου.
\vs{40}Ὅτι ἀρῶ εἰς τὸν οὐρανὸν τὴν χεῖρά μου, καὶ ὀμοῦμαι τὴν δεξιάν μου· καὶ ἐρῶ, ζῶ ἐγὼ εἰς τὸν αἰῶνα·
\vs{41}Ὅτι παροξυνῶ ὡς ἀστραπὴν τὴν μάχαιράν μου, καὶ ἀνθέξεται κρίματος ἡ χείρ μου, καὶ ἀποδώσω δίκην τοῖς ἐχθροῖς, καὶ τοῖς μισοῦσί με ἀνταποδώσω.
\vs{42}Μεθύσω τὰ βέλη μου ἀφʼ αἵματος, καὶ ἡ μάχαιρά μου φάγεται κρέα ἀφʼ αἵματος τραυματιῶν καὶ αἰχμαλωσίας ἀπὸ κεφαλῆς ἀρχόντων ἐχθρῶν.

\vs{43}Εὐφράνθητε οὐρανοὶ ἅμα αὐτῷ, καὶ προσκυνησάτωσαν αὐτῷ πάντες ἄγγελοι Θεοῦ· εὐφράνθητε ἔθνη μετὰ τοῦ λαοῦ αὐτοῦ, καὶ ἐνισχυσάτωσαν αὐτῷ πάντες υἱοὶ Θεοῦ, ὅτι τὸ αἷμα τῶν υἱῶν αὐτοῦ ἐκδικᾶται· καὶ ἐκδικήσει καὶ ἀνταποδώσει δίκην τοῖς ἐχθροῖς, καὶ τοῖς μισοῦσιν ἀνταποδώσει· καὶ ἐκκαθαριεῖ Κύριος τὴν γῆν τοῦ λαοῦ αὐτοῦ.

\vs{44}Καὶ ἔγραψε Μωυσῆς τὴν ᾠδὴν ταύτην ἐν τῇ ἡμέρᾳ ἐκείνῃ, καὶ ἐδίδαξεν αὐτὴν τοὺς υἱοὺς Ἰσραήλ· καὶ εἰσῆλθε Μωυσῆς, καὶ ἐλάλησε πάντας τοὺς λόγους τοῦ νόμου τούτου εἰς τὰ ὦτα τοῦ λαοῦ, αὐτὸς καὶ Ἰησοῦς ὁ τοῦ Ναυή.
\vs{45}Καὶ ἐξετέλεσε Μωυσῆς λαλῶν παντὶ Ἰσραήλ.
\vs{46}Καὶ εἶπε πρὸς αὐτοὺς, προσέχετε τῇ καρδίᾳ ἐπὶ πάντας τοὺς λόγους τούτους, οὓς ἐγὼ διαμαρτύρομαι ὑμῖν σήμερον, ἃ ἐντελεῖσθε τοῖς υἱοῖς ὑμῶν, φυλάσσειν καὶ ποιεῖν πάντας τοὺς λόγους τοῦ νόμου τούτου.
\vs{47}Ὅτι οὐχὶ λόγος κενὸς οὗτος ὑμῖν· ὅτι αὕτη ἡ ζωὴ ὑμῶν, καὶ ἕνεκεν τοῦ λόγου τούτου μακροημερεύσετε ἐπὶ τῆς γῆς, εἰς ἣν ὑμεῖς διαβαίνετε τὸν Ἰορδάνην ἐκεῖ κληρονομῆσαι.
\vs{48}Καὶ ἐλάλησε Κύριος πρὸς Μωυσῆν ἐν τῇ ἡμέρᾳ ταύτῃ, λέγων,
\vs{49}ἀνάβηθι εἰς τὸ ὄρος τὸ Ἀβαρίμ, τοῦτο ὄρος Ναβαῦ ὅ ἐστιν ἐν γῇ Μωὰβ κατὰ πρόσωπον Ἰεριχὼ, καὶ ἴδε τὴν γῆν Χαναάν, ἣν ἐγὼ δίδωμι τοῖς υἱοῖς Ἰσραὴλ,
\vs{50}καὶ τελεύτα ἐν τῷ ὄρει εἰς ὃ ἀναβαίνεις ἐκεῖ, καὶ προστέθητι πρὸς τὸν λαόν σου· ὃν τρόπον ἀπέθανεν Ἀαρὼν ὁ ἀδελφός σου ἐν Ὢρ τῷ ὄρει, καὶ προσετέθη πρὸς τὸν λαὸν αὐτοῦ.
\vs{51}Ὅτι ἠπειθήσατε τῷ ῥήματί μου ἐν τοῖς υἱοῖς Ἰσραὴλ ἐπὶ τοῦ ὕδατος ἀντιλογίας Κάδης ἐν τῇ ἐρήμῳ Σὶν, διότι οὐχ ἡγιάσατέ με ἐν τοῖς υἱοῖς Ἰσραήλ.
\vs{52}Ἀπέναντι ὄψει τὴν γῆν, καὶ ἐκεῖ οὐκ εἰσελεύσῃ.

\ch{33}
Καὶ αὕτη ἡ εὐλογία ἣν ηὐλόγησε Μωυσῆς ἄνθρωπος τοῦ Θεοῦ τοὺς υἱοὺς Ἰσραὴλ πρὸ τῆς τελευτῆς αὐτοῦ.
\vs{2}Καὶ εἶπε, Κύριος ἐκ Σινὰ ἥκει, καὶ ἐπέφανεν ἐκ Σηεὶρ ἡμῖν, καὶ κατέσπευσεν ἐξ ὄρους Φαρὰν, σὺν μυρίασι Κάδης, ἐκ δεξιῶν αὐτοῦ ἄγγελοι μετʼ αὐτοῦ.
\vs{3}Καὶ ἐφείσατο τοῦ λαοῦ αὐτοῦ, καὶ πάντες οἱ ἡγιασμένοι ὑπὸ τὰς χεῖράς σου· καὶ οὗτοι ὑπὸ σέ εἰσί· καὶ ἐδέξατο ἀπὸ τῶν λόγων αὐτοῦ
\vs{4}νόμον, ὃν ἐνετείλατο ἡμῖν Μωυσῆς, κληρονομίαν συναγωγαῖς Ἰακώβ.
\vs{5}Καὶ ἔσται ἐν τῷ ἠγαπημένῳ ἄρχων, συναχθέντων ἀρχόντων λαῶν ἅμα φυλαῖς Ἰσραήλ.
\vs{6}Ζήτω Ῥουβὴν, καὶ μὴ ἀποθανέτω, καὶ ἔστω πολὺς ἐν ἀριθμῷ.

\vs{7}Καὶ αὕτη Ἰούδα· εἰσάκουσον Κύριε φωνῆς Ἰούδα, καὶ εἰς τὸν λαὸν αὐτοῦ ἔλθοις ἄν· αἱ χεῖρες αὐτοῦ διακρινοῦσιν αὐτῷ, καὶ βοηθὸς ἐκ τῶν ἐχθρῶν ἔσῃ.

\vs{8}Καὶ τῷ Λευὶ εἶπε, δότε Λευὶ δήλους αὐτοῦ, καὶ ἀλήθειαν αὐτοῦ τῷ ἀνδρὶ τῷ ὁσίῳ, ὃν ἐπείρασαν αὐτὸν ἐν πείρᾳ· ἐλοιδόρησαν αὐτὸν ἐφʼ ὕδατος ἀντιλογίας·
\vs{9}Ὁ λέγων τῷ πατρὶ καὶ τῇ μητρὶ, οὐχ ἑώρακά σε, καὶ τοὺς ἀδελφοὺς αὐτοῦ οὐκ ἐπέγνω, καὶ τοὺς υἱοὺς αὐτοῦ ἀπέγνω· ἐφύλαξε τὰ λόγιά σου, καὶ τὴν διαθήκην σου διετήρησε.
\vs{10}Δηλώσουσι τὰ δικαιώματά σου τῷ Ἰακὼβ, καὶ τὸν νόμον σου τῷ Ἰσραήλ· ἐπιθήσουσι θυμίαμα ἐν ὀργῇ σου διαπαντὸς ἐπὶ τὸ θυσιαστήριόν σου.
\vs{11}Εὐλόγησον, Κύριε, τὴν ἰσχὺν αὐτοῦ, καὶ τὰ ἔργα τῶν χειρῶν αὐτοῦ δέξαι· κάταξον ὀσφῦν ἐχθρῶν ἐπανεστηκότων αὐτῷ, καὶ οἱ μισοῦντες αὐτὸν μὴ ἀναστήτωσαν.
\vs{12}Καὶ τῷ Βενιαμὶν εἶπεν, ἠγαπημένος ὑπὸ Κυρίου κατασκηνώσει πεποιθὼς, καὶ ὁ Θεὸς σκιάζει ἐπʼ αὐτῷ πάσας τὰς ἡμέρας, καὶ ἀναμέσον τῶν ὤμων αὐτοῦ κατέπαυσε.

\vs{13}Καὶ τῷ Ἰωσὴφ εἶπεν, ἀπʼ εὐλογίας Κύριου ἡ γῆ αὐτοῦ, ἀπὸ ὡρῶν οὐρανοῦ, καὶ δρόσου, καὶ ἀπὸ ἀβύσσων πηγῶν κάτωθεν,
\vs{14}καὶ καθʼ ὥραν γεννημάτων ἡλίου τροπῶν, καὶ ἀπὸ συνόδων μηνῶν,
\vs{15}ἀπὸ κορυφῆς ὀρέων ἀρχῆς, καὶ ἀπὸ κορυφῆς βουνῶν ἀενάων,
\vs{16}καὶ καθʼ ὥραν γῆς πληρώσεως· καὶ τὰ δεκτὰ τῷ ὀφθέντι ἐν τῇ βάτῳ ἔλθοισαν ἐπὶ κεφαλὴν Ἰωσὴφ, καὶ ἐπὶ κορυφῆς δοξασθεὶς ἐπʼ ἀδελφοῖς.
\vs{17}Πρωτότοκος ταύρου τὸ κάλλος αὐτοῦ, κέρατα μονοκέρωτος τὰ κέρατα αὐτοῦ· ἐν αὐτοῖς ἔθνη κερατιεῖ ἅμα, ἕως ἀπʼ ἄκρου γῆς· αὗται μυρίαδες Ἐφραὶμ, καὶ αὗται χιλιάδες Μανασσῆ.
\vs{18}Καὶ τῷ Ζαβουλὼν εἶπεν, εὐφράνθητι Ζαβουλὼν ἐν ἐξοδίᾳ σου, καὶ Ἰσσάχαρ ἐν τοῖς σκηνώμασιν αὐτοῦ.
\vs{19}Ἔθνη ἐξολοθρεύσουσι· καὶ ἐπικαλέσεσθε ἐκεῖ, καὶ θύσετε ἐκεῖ θυσίαν δικαιοσύνης· ὅτι πλοῦτος θαλάσσης θηλάσει σε, καὶ ἐμπόρια παράλιον κατοικούντων.

\vs{20}Καὶ τῷ Γὰδ εἶπεν, εὐλογημένος ἐμπλατύνων Γάδ· ὡς λέων ἐνεπαύσατο, συντρίψας βραχίονα καὶ ἄρχοντα.
\vs{21}Καὶ εἶδεν ἀπαρχὴν αὐτοῦ, ὅτι ἐκεῖ ἐμερίσθη γῆ ἀρχόντων συνηγμένων ἅμα ἀρχηγοῖς λαῶν· δικαιοσύνην Κύριος ἐποίησε, καὶ κρίσιν αὐτοῦ μετὰ Ἰσραήλ.

\vs{22}Καὶ τῷ Δὰν εἶπε, Δὰν σκύμνος λέοντος, καὶ ἐκπηδήσεται ἐκ τοῦ Βασάν.
\vs{23}Καὶ τῷ Νεφθαλὶ εἶπε, Νεφθαλὶ πλησμονὴ δεκτῶν· καὶ ἐμπλησθήτω εὐλογία παρὰ Κυρίου· θάλασσαν καὶ Λίβα κληρονομήσει.
\vs{24}Καὶ τῷ Ἀσὴρ εἶπεν, εὐλογημένος ἀπὸ τέκνων Ἀσήρ, καὶ ἔσται δεκτὸς τοῖς ἀδελφοῖς αὐτοῦ· βάψει ἐν ἐλαίῳ τὸν πόδα αὐτοῦ.
\vs{25}Σίδηρος καὶ χαλκὸς τὸ ὑπόδημα αὐτοῦ ἔσται· ὡς αἱ ἡμέραι σου, ἡ ἰσχύς σου.

\vs{26}Οὐκ ἔστιν ὥσπερ ὁ Θεὸς τοῦ ἠγαπημένου, ὁ ἐπιβαίνων ἐπὶ τὸν οὐρανὸν βοηθός σου, καὶ ὁ μεγαλοπρεπὴς τοῦ στερεώματος.
\vs{27}Καὶ σκεπάσει σε Θεοῦ ἀρχὴ, καὶ ὑπὸ ἰσχὺν βραχιόνων ἀενάων· καὶ ἐκβαλεῖ ἀπὸ προσώπου σου ἐχθρὸν, λέγων, ἀπόλοιο.
\vs{28}Καὶ κατασκηνώσει Ἰσραὴλ πεποιθὼς, μόνος ἐπὶ γῆς Ἰακὼβ, ἐπὶ σίτῳ καὶ οἴνῳ· καὶ ὁ οὐρανός σοι συννεφὴς δρόσῳ.
\vs{29}Μακάριος σὺ Ἰσραήλ· τίς ὅμοιός σοι λαὸς σωζόμενος ὑπὸ Κυρίου; ὑπερασπιεῖ ὁ βοηθός σου, καὶ ἡ μάχαιρα καύχημά σου· καὶ ψεύσονταί σε οἱ ἐχθροί σου· καὶ σὺ ἐπὶ τὸν τράχηλον αὐτῶν ἐπιβήσῃ.

\ch{34}
Καὶ ἀνεβη Μωυσῆς ἀπὸ Ἀραβὼθ Μωὰβ ἐπὶ τὸ ὄρος Ναβαῦ, ἐπὶ κορυφὴν Φασγὰ, ἥ ἐστιν ἐπὶ προσώπου Ἱεριχώ· καὶ ἔδειξεν αὐτῷ Κύριος πᾶσαν τὴν γῆν Γαλαὰδ ἕως Δὰν,
\vs{2}καὶ πᾶσαν τὴν γῆν Νεφθαλὶ, καὶ πᾶσαν τὴν γῆν Ἐφραὶμ, καὶ Μανασσῆ, καὶ πᾶσαν τὴν γῆν Ἰούδα ἕως τῆς θαλάσσης τῆς ἐσχάτης,
\vs{3}καὶ τὴν ἔρημον, καὶ τὰ περίχωρα Ἱεριχὼ, πόλιν φοινίκων ἕως Σηγώρ.
\vs{4}Καὶ εἶπε Κύριος πρὸς Μωυσῆν, αὕτη ἡ γῆ ἣν ὤμοσα τῷ Ἁβραὰμ καὶ Ἰσαὰκ καὶ Ἰακὼβ, λέγων, τῷ σπέρματι ὑμῶν δώσω αὐτήν· καὶ ἔδειξα τοῖς ὀφθαλμοῖς σου, καὶ ἐκεῖ οὐκ εἰσελεύσῃ.

\vs{5}Καὶ ἐτελεύτησε Μωυσῆς ὁ οἰκέτης Κυρίου ἐν γῇ Μωὰβ διὰ ῥήματος Κυρίου.
\vs{6}Καὶ ἔθαψαν αὐτὸν ἐν Γαὶ ἐγγὺς οἴκου Φογώρ· καὶ οὐκ εἶδεν οὐδεὶς τὴν ταφὴν αὐτοῦ ἕως τῆς ἡμέρας ταύτης.
\vs{7}Μωυσῆς δὲ ἦν ἑκατὸν καὶ εἴκοσι ἐτῶν ἐν τῷ τελευτᾷν αὐτόν· οὐκ ἠμαυρώθησαν οἱ ὀφθαλμοὶ αὐτοῦ, οὐδὲ ἐφθάρησαν τὰ χελώνια αὐτοῦ.

\vs{8}Καὶ ἔκλαυσαν οἱ υἱοὶ Ἰσραὴλ Μωυσῆν ἐν Ἀραβὼθ Μωὰβ ἐπὶ τοῦ Ἰορδάνου κατὰ Ἱεριχὼ τριάκοντα ἡμέρας, καὶ συνετελέσθησαν αἱ ἡμέραι πένθους κλαυθμοῦ Μωυσῆ.
\vs{9}Καὶ Ἰησοῦς υἱὸς Ναυῆ ἐνεπλήσθη πνεύματος συνέσεως, ἐπέθηκε γὰρ Μωυσῆς τὰς χεῖρας αὐτοὺ ἐπʼ αὐτόν· καὶ εἰσήκουσαν αὐτοῦ οἱ υἱοὶ Ἰσραήλ· καὶ ἐποίησαν καθότι ἐνετείλατο Κύριος τῷ Μωυσῇ.

\vs{10}Καὶ οὐκ ἀνέστη ἔτι προφήτης ἐν Ἰσραὴλ ὡς Μωσῆς· ὃν ἔγνω Κύριος αὐτὸν πρόσωπον κατὰ πρόσωπον
\vs{11}ἐν πᾶσι τοῖς σημείοις καὶ τέρασιν, ὃν ἀπέστειλεν αὐτὸν Κύριος ποιῆσαι αὐτὰ ἐν γῇ Αἰγύπτῳ Φαραὼ, καὶ τοῖς θεράπουσιν αὐτοῦ, καὶ πάσῃ τῇ γῇ αὐτοῦ·
\vs{12}τὰ θαυμάσια τὰ μεγάλα, καὶ τὴν χεῖρα τὴν κραταιὰν, ἃ ἐποίησε Μωυσῆς ἔναντι παντὸς Ἰσραήλ.


\def\book{ΙΗΣΟΥΣ ΝΑΥΗ}
\biblebook{ΙΗΣΟΥΣ ΝΑΥΗ}


\lettrine[lines=2, loversize=0.2, nindent=0em, findent=.25em]{\textcolor{bookheadingcolor}{Κ}}{ΑΙ} ἐγένετο μετὰ τὴν τελευτὴν Μωυσῆ, εἶπε Κύριος τῷ Ἰησοῖ υἱῷ Ναυῆ τῷ ὑπουργῷ Μωυσῆ, λέγων,
\vs{2}Μωυσῆς ὁ θεράπωνμου τετελεύτηκε. νῦν οὖν ἀναστὰς, διάβηθι τὸν Ἰορδάνην σὺ καὶ πᾶς ὁ λαὸς οὗτος εἰς τὴν γῆν, ἣν ἐγὼ δίδωμι αὐτοῖς.
\vs{3}πᾶς ὁ τόπος ἐφʼ ὃν ἂν ἐπιβῆτε τῷ ἴχνει τῶν ποδῶν ὑμῶν, ὑμῖν δώσω αὐτόν, ὃν τρόπον εἴρηκα τῷ Μωυσῇ·
\vs{4}Τὴν ἔρημον καὶ τὸν Ἀντιλίβανον, ἕως τοῦ ποταμοῦ τοῦ μεγάλου, ποταμοῦ Εὐφράτου, καὶ ἕως τῆς θαλάσσης τῆς ἐσχάτης· ἀφʼ ἡλίου δυσμῶν ἔσται τὰ ὅρια ὑμῶν.
\vs{5}οὐκ ἀντιστήσεται ἄνθρωπος κατενώπιον ὑμῶν πάσας τὰς ἡμέρας τῆς ζωῆς σου· καὶ ὥσπερ ἤμην μετὰ Μωυσῆ, οὕτως ἔσομαι καὶ μετὰ σοῦ, καὶ οὐκ ἐγκαταλείψω σε οὐδʼ ὑπερόψομαί σε.
\vs{6}Ἴσχυε καὶ ἀνδρίζου· σὺ γὰρ ἀποδιεῖς τῷ λαῷ τούτῳ τὴν γῆν, ἣν ὤμοσα τοῖς πατράσινν ὑμῶν δοῦναι αὐτοῖς.
\vs{7}Ἴσχυε οὖν καὶ ἀνδρίζου, φυλάσσεσθαι καὶ ποιεῖν καθότι ἐνετείλατό σοι Μωυσῆς ὁ παῖς μου· καὶ οὐκ ἐκκλινεῖς ἀπʼ αὐτῶν εἰς δεξιὰ οὐδὲ εἰς ἀριστερὰ, ἵνα συνῇς ἐν πᾶσιν οἷς ἐὰν πράσσῃς.
\vs{8}Καὶ οὐκ ἀποστήσεται ἡ βίβλος τοῦ νόμου τούτου ἐκ τοῦ στόματός σου, καὶ μελετήσεις ἐν αὐτῇ ἡμέρας καὶ νυκτὸς, ἵνα εἰδῇς ποιεῖν πάντα τὰ γεγραμμένα· τότε εὐοδωθήσῃ, καὶ εὐοδώσεις τὰς ὁδούς σου, καὶ τότε συνήσεις.
\vs{9}Ἰδοὺ ἐντέαλμαί σοι· ἴσχυε καὶ ἀνδρίζου, μὴ δειλιάσῃς, μηδὲ φοβηθῇ, ὅτι μετὰ σοῦ Κύριος ὁ Θεός σου εἰς πάντα οὗ ἐὰν πορεύῃ·
\vs{10}Καὶ ἐνετείλατο Ἰησοῦς τοῖς γραμματεῦσιν τοῦ λαοῦ, λέγων,
\vs{11}Εἰσέλθατε κατὰ μέσον τῆς παρεμβολῆς τοῦ λαοῦ, καὶ ἐντείλασθε τῷ λαῷ, λέγοντες, ἐτοιμάζεσθε ἐπισιτισμόν, ὅτι ἔτι τρεῖς ἡμέραι καὶ ὑμεῖς διαβαίνετε τὸν Ἰορδάνην τοῦτον, εἰσελθόντες κατασχεῖν τὴν γῆν, ἣν Κύριος ὁ Θεὸς τῶν πατέρων ὑμῶν δίδωσιν ὑμῖν.

\vs{12}Καὶ τῷ Ῥουβὴν, καὶ τῷ Γὰδ, καὶ τῷ ἡμίσει φυλῆς Μανασσῆ εἶπεν Ἰησοῦς,
\vs{13}μνήσθητε τὸ ῥῆμα ὃ ἐνετείλατο ὑμῖν Μωυσῆς ὁ παῖς Κυρίου, λέγων, Κύριος ὁ Θεὸς ὑμῶν κατέπαυσεν ὑμᾶς, καὶ ἔδωκεν ὑμῖν τὴν γῆν ταύτην.
\vs{14}Αἱ γυναῖκες ὑμῶν καὶ τὰ παιδία ὑμῶν καὶ τὰ κτήνη ὑμῶν κατοικείτωσαν ἐν τῇ γῇ, ᾗ ἔδωκεν ὑμῖν· ὑμεῖς δὲ διαβήσεσθε εὔζωνοι πρότεροι τῶν ἀδελφῶν ὑμῶν πᾶς ὁ ἰσχύων· καὶ συμμαχήσετε αὐτοῖς,
\vs{15}ἕως ἂν καταπαύσῃ Κύριος ὁ Θεὸς ἡμῶν τοὺς ἀδελφοὺς ὑμῶν, ὥσπερ καὶ ὑμᾶς, καὶ κληρονομήσωσι καὶ οὗτοι τὴν γῆν, ἣν Κύριος ὁ Θεὸς ἡμῶν δίδωσιν αὐτοῖς· καὶ ἀπελεύσεσθε ἕκαστος εἰς τὴν κληρονομίαν αὐτοῦ, ἣν ἔδωκεν ὑμῖν Μωυσῆς εἰς τὸ πέραν τοῦ Ἰορδάνου ἐπʼ ἀνατολῶν ἡλίου.
\vs{16}Καὶ ἀποκριθέντες τῷ Ἰησοῦ εἶπαν, πάντα ὅσα ἐὰν ἐντείλῃ ἡμῖν, ποιήσομεν, καὶ εἰς πάντα τόπον οὗ ἐὰν ἀποστείλῃς ἡμᾶς, πορευσόμεθα.
\vs{17}Κατὰ πάντα ὅσα ἠκούσαμεν Μωυσῆ, ἀκουσόμεθά σοῦ· πλὴν ἔστω Κύριος ὁ Θεὸς ἡμῶν μετὰ σοῦ, ὃν τρόπον ἦν μετὰ Μωυσῆ.
\vs{18}Ὁ δὲ ἄνθρωπος ὃς ἂν ἀπειθήσῃ σοι, καὶ ὅστις μὴ ἀκούσῃ τῶν ῥημάτων σου καθότι ἐὰν ἐντείλῃ αὐτῷ, ἀποθανέτω· ἀλλὰ ἴσχυε καὶ ἀνδρίζου.

\ch{2}
Καὶ ἀπέστειλεν Ἰησοῦς υἱὸς Ναυῆ ἐκ Σαττὶν δύο νεανίσκους κατασκοπεῦσαι, λέγων, ἀνάβητε καὶ ἴδετε τὴν γῆν καὶ τὴν Ἱεριχώ· καὶ πορευθέντες οἱ δύο νεανίσκοι εἰσήλθοσαν εἰς Ἱεριχώ· καὶ εἰσήλθοσαν εἰς οἰκίαν γυναικὸς πόρνης, ᾗ ὄνομα Ῥαάβ· Καὶ κατέλυσαν ἐκεῖ.

\vs{2}Καὶ ἀπηγγέλη τῷ βασιλεῖ Ἱεριχὼ, λέγοντες, εἰσπεπόρευνται ὧδε ἄνδρες τῶν υἱῶν Ἰσραὴλ κατασκοπεῦσαι τὴν γῆν.
\vs{3}Καὶ ἀπέστειλεν ὁ βασιλεὺς Ἱεριχὼ, καὶ εἶπε πρὸς Ῥαὰβ, λέγων, ἐξάγαγε τοὺς ἄνδρας τοὺς εἰσπεπορευμένους εἰς τὴν οἰκίαν σου τὴν νύκτα, κατασκοπεῦσαι γὰρ τὴν γῆν ἥκασι.
\vs{4}Καὶ λαβοῦσα ἡ γυνὴ τοὺς δύο ἄνδρας, ἔκρυψεν αὐτούς· καὶ εἶπεν αὐτοῖς, λέγουσα, εἰσεληλύθασι πρὸς μὲ οἱ ἄνδρες,
\vs{5}ὡς δὲ ἡ πύλη ἐκλείετο ἐν τῷ σκότει, καὶ οἱ ἄνδρες ἐξῆλθον· οὐκ ἐπίσταμαι ποῦ πεπόρευνται· καταδιώξατε ὀπίσω αὐτῶν, εἰ καταλήψεσθε αὐτούς.
\vs{6}Αὕτη δὲ ἀνεβίβασεν αὐτοὺς ἐπὶ τὸ δῶμα, καὶ ἔκρυψεν αὐτοὺς ἐν τῇ λινοκαλάμῃ τῇ ἐστοιβασμένῃ αὐτῇ ἐπὶ τοῦ δώματος.
\vs{7}Καὶ οἱ ἄνδρες κατεδίωξαν ὀπίσω αὐτῶν ὁδὸν τὴν ἐπὶ τοῦ Ἰορδάνου ἐπὶ τὰς διαβάσεις, καὶ ἡ πύλη ἐκλείσθη.

\vs{8}Καὶ ἐγένετο ὡς ἐξήλθοσαν οἱ διώκοντες ὀπίσω αὐτῶν, καὶ αὐτοὶ δὲ πρὶν ἢ κοιμηθῆναι αὐτοὺς, αὕτη δὲ ἀνέβη πρὸς αὐτοὺς ἐπὶ τὸ δῶμα,
\vs{9}καὶ εἶπε πρὸς αὐτοὺς, ἐπίσταμαι ὅτι ἔδωκεν ὑμῖν Κύριος τὴν γῆν· ἐπιπέπτωκε γὰρ ὁ φόβος ὑμῶν ἐφʼ ἡμᾶς.
\vs{10}Ἀκηκόαμεν γὰρ ὅτι κατεξήρανε Κύριος ὁ Θεὸς τὴν ἐρυθρὰν θάλασσαν ἀπὸ προσώπου ὑμῶν, ὅτε ἐξεπορεύεσθε ἐκ γῆς Αἰγυπτου, καὶ ὅσα ἐποίησε τοῖς δυσὶ βασιλεῦσι τῶν Ἀμοῤῥαίων, οἳ ἦσαν πέραν τοῦ Ἰορδάνου, τῷ Σηὼν καὶ Ὢγ, οὓς ἐξωλοθρεύσατε αὐτούς.
\vs{11}Καὶ ἀκούσαντες ἡμεῖς ἐξέστημεν τῇ καρδίᾳ ἡμῶν, καὶ οὐκ ἔστη ἔτι πνεῦμα ἐν οὐδενὶ ἡμῶν ἀπὸ προσώπου ὑμῶν· ὅτι Κύριος ὁ Θεὸς ὑμῶν, Θεὸς ἐν οὐρανῷ ἄνω καὶ ἐπὶ τῆς γῆς κάτω.
\vs{12}Καὶ νῦν ὀμόσατέ μοι Κύριον τὸν Θεὸν, ὅτι ποιῶ ὑμῖν ἔλεος, καὶ ποιήσατε Καὶ ὑμεῖς ἔλεος ἐν τῷ οἴκῳ τοῦ πατρός μου,
\vs{13}καὶ ζωγρήσατε τὸν οἶκον τοῦ πατρὸς μου, τὴν μητέρα μου, καὶ τοὺς ἀδελφούς μου, καὶ πάντα τὸν οἶκόν μου, καὶ πάντα ὅσα ἐστὶν αὐτοῖς, καὶ ἐξελεῖσθε τὴν ψυχήν μου ἐκ θανάτου.

\vs{14}Καὶ εἶπαν αὐτῇ οἱ ἄνδρες, ἡ ψυχὴ ἡμῶν ἀνθʼ ὑμῶν εἰς θάνατον· καὶ αὐτὴ εἶπεν, ὡς ἂν παραδῷ Κύριος ὑμῖν τὴν πόλιν, ποιήσετε εἰς ἐμὲ ἔλεος καὶ ἀλήθειαν.
\vs{15}Καὶ κατεχάλασεν αὐτοὺς διὰ τῆς θυρίδος,
\vs{16}καὶ εἶπεν αὐτοῖς, εἰς τὴν ὀρεινὴν ἀπέλθετε, μὴ συναντήσωσιν ὑμῖν οἱ καταδιώκοντες, καὶ κρυβήσεσθε ἐκεῖ τρεῖς ἡμέρας ἕως ἂν ἀποστρέψωσιν οἱ καταδιώκοντες ὀπίσω ὑμῶν, καὶ μετὰ ταῦτα ἀπελεύσεσθε εἰς τὴν ὁδὸν ὑμῶν.

\vs{17}Καὶ εἶπαν πρὸς αὐτὴν οἱ ἄνδρες, ἀθῶοι ἐσμὲν τῷ ὅρκῳ σου τούτῳ.
\vs{18}Ἰδοὺ ἡμεῖς εἰσπορευόμεθα εἰς μέρος τῆς πόλεως, καὶ θήσεις τὸ σημεῖον, τὸ σπαρτίον τὸ κόκκινον τοῦτο ἐκδήσεις εἰς τὴν θυρίδα διʼ ἧς κατεβίβασας ἡμᾶς διʼ αὐτῆς· τὸν δὲ πατέρα σου, καὶ τὴν μητέρα σου, καὶ τοὺς ἀδελφούς σου, καὶ πάντα τὸν οἶκον τοῦ πατρός σου συνάξεις πρὸς σεαυτὴν εἰς τὴν οἰκίαν σου.
\vs{19}Καὶ ἔσται πᾶς ὃς ἂν ἐξέλθῃ τὴν θύραν τῆς οἰκίας σου ἔξω, ἔνοχος ἑαυτῷ ἔσται, ἡμεῖς δὲ ἀθῶοι τῷ ὅρκῳ σου τούτῳ· καὶ ὅσοι ἐὰν γένωνται μετὰ σοῦ ἐν τῇ οἰκίᾳ σου, ἡμεῖς ἔνοχοι ἐσόμεθα.
\vs{20}Ἐὰν δέ τις ἡμᾶς ἀδικήσῃ ἢ καὶ ἀποκαλύψῃ τοὺς λόγους ἡμῶν τούτους, ἐσόμεθα ἀθῶοι τῷ ὅρκῳ σου τούτῳ.
\vs{21}Καὶ εἶπεν αὐτοῖς, κατὰ τὸ ῥῆμα ὑμῶν ἔστω· καὶ ἐξαπέστειλεν αὐτοὺς, καὶ ἐπορεύθησαν.
\vs{22}Καὶ ἤλθοσαν εἰς τὴν ὀρεινήν, καὶ κατέμειναν ἐκεῖ τρεῖς ἡμέρας· καὶ ἐξεζήτησαν οἱ καταδιώκοντες πάσας τὰς ὁδοὺς, καὶ οὐχ εὕροσαν.

\vs{23}Καὶ ὑπέστρεψαν οἱ δύο νεανίσκοι, καὶ κατέβησαν ἐκ τοῦ ὄρους· καὶ διέβησαν πρὸς Ἰησοῦν υἱὸν Ναυὴ, καὶ διηγήσαντο αὐτῷ πάντα τὰ συμβεβηκότα αὐτοῖς.
\vs{24}Καὶ εἶπαν πρὸς Ἰησοῦν, ὅτι παραδέδωκε Κύριος πᾶσαν τὴν γῆν ἐν χειρὶ ἡμῶν, καὶ κατέπτηχε πᾶς ὁ κατοικῶν τὴν γῆν ἐκείνην ἀφʼ ἡμῶν.

\ch{3}
Καὶ ὤρθρισεν Ἰησοῦς τοπρωῒ, καὶ ἀπῇρεν ἐκ Σαττὶν, καὶ ἤλθοσαν ἕως τοῦ Ἰορδάνου, καὶ κατέλυσαν ἐκεῖ πρὸ τοῦ διαβῆναι.
\vs{2}Καὶ ἐγένετο μετὰ τρεῖς ἡμέρας διῆλθον οἱ γραμματεῖς διὰ τῆς παρεμβολῆς,
\vs{3}καὶ ἐνετείλαντο τῷ λαῷ, λέγοντες, ὅταν ἴδητε τὴν κιβωτὸν τῆς διαθήκης Κυρίου τοῦ Θεοῦ ἡμῶν, καὶ τοὺς ἱερεῖς ἡμῶν καὶ τοὺς Λευίτας αἴροντας αὐτὴν, ἀπαρεῖτε ἀπὸ τῶν τόπων ὑμῶν, καὶ πορεύσεσθε ὀπίσω αὐτῆς.
\vs{4}Ἀλλὰ μακρὰν ἔστω ἀναμέσον ὑμῶν καὶ ἐκείνης, ὅσον δισχιλίους πήχεις στήσεσθε· μὴ προσεγγίσητε αὐτῇ, ἵνα ἐπίστησθε τὴν ὁδὸν, ἣν πορεύσεσθε αὐτήν· οὐ γὰρ πεπόρευσθε τὴν ὁδὸν ἀπʼ ἐχθὲς καὶ τρίτης ἡμέρας.

\vs{5}Καὶ εἶπεν Ἰησοῦς τῷ λαῷ, ἁγνίσασθε εἰς αὔριον, ὅτι αὔριον ποιήσει Κύριος ἐν ὑμῖν θαυμαστά.
\vs{6}Καὶ εἶπεν Ἰησοῦς τοῖς ἱερεῦσιν, ἄρατε τὴν κιβωτὸν τῆς διαθήκης Κυρίου, καὶ προπορεύεσθε τοῦ λαοῦ· καὶ ᾖραν οἱ ἱερεῖς τὴν κιβωτὸν τῆς διαθήκης Κυρίου, καὶ ἐπορεύοντο ἔμπροσθεν τοῦ λαοῦ.
\vs{7}Καὶ εἶπε Κύριος πρὸς Ἰησοῦν, ἐν τῇ ἡμέρᾳ ταύτῃ ἄρχομαι ὑψῶσαί σε κατενώπιον πάντων υἱῶν Ἰσραὴλ, ἵνα γνῶσιν ὅτι καθότι ἤμην μετὰ Μωυσῆ, οὕτως ἔσομαι καὶ μετὰ σοῦ.
\vs{8}Καὶ νῦν ἔντειλαι τοῖς ἱερεῦσι τοῖς αἴρουσι τὴν κιβωτὸν τῆς διαθήκης, λέγων, ὠς ἂν εἰσέλθητε ἐπὶ μέρους τοῦ ὕδατος τοῦ Ἰορδάνου, καὶ ἐν τῷ Ἰορδάνῃ στήσεσθε.

\vs{9}Καὶ εἶπεν Ἰησοῦς τοῖς υἱοῖς Ἰσραὴλ, προσαγάγετε ὧδε, καὶ ἀκούσατε τὸ ῥῆμα Κυρίου τοῦ Θεοῦ ἡμῶν.
\vs{10}Ἐν τούτῳ γνώσεσθε, ὅτι Θεὸς ζῶν ἐν ὑμῖν, καὶ ὀλοθρεύων ὀλοθρεύσει ἀπὸ προσώπου ἡμῶν τὸν Χαναναῖον, καὶ τὸν Χετταῖον, καὶ τὸν Φερεζαῖον, καὶ τὸν Εὐαῖον, καὶ τὸν Ἀμοῤῥαῖον, καὶ τὸν Γεργεσαῖον, καὶ τὸν Ἰεβουσαῖον.
\vs{11}Ἰδοὺ ἡ κιβωτὸς διαθήκης Κυρίου πάσης τῆς γῆς διαβαίνει τὸν Ἰορδάνην.
\vs{12}Προχειρίσασθε ὑμῖν δώδεκα ἄνδρας ἀπὸ τῶν υἱῶν Ἰσραὴλ, ἕνα ἀφʼ ἑκάστης φυλῆς.
\vs{13}Καὶ ἔσται, ὡς ἂν καταπαύσωσιν οἱ πόδες τῶν ἱερέων τῶν αἰρόντων τὴν κιβωτὸν τῆς διαθήκης Κυρίου πάσης τῆς γῆς ἐν τῷ ὕδατι τοῦ Ἰορδάνου, τὸ ὕδωρ τοῦ Ἰορδάνου ἐκλείψει, τὸ δὲ ὕδωρ τὸ καταβαῖνον στήσεται.

\vs{14}Καὶ ἀπῇρεν ὁ λαὸς ἐκ τῶν σκηνωμάτων αὐτῶν διαβῆναι τὸν Ἰορδάνην, οἱ δὲ ἱερεῖς ᾔροσαν τὴν κιβωτὸν τῆς διαθήκης Κυρίου πρότεροι τοῦ λαοῦ.
\vs{15}Ὡς δὲ εἰσεπορεύοντο οἱ ἱερεῖς οἱ αἴροντες τὴν κιβωτὸν τῆς διαθήκης ἐπὶ τὸν Ἰορδάνην, καὶ οἱ πόδες τῶν ἱερέων τῶν αἰρόντων τὴν κιβωτὸν τῆς διαθήκης Κυρίου ἐβάφησαν εἰς μέρος τοῦ ὕδατος τοῦ Ἰορδάνου· ὁ δὲ Ἰορδάνης ἐπληροῦτο καθʼ ὅλην τὴν κρηπίδα αὐτοῦ, ὡσεὶ ἡμέραι θερισμοῦ πυρῶν·
\vs{16}Καὶ ἔστη τὰ ὕδατα τὰ καταβαίνοντα ἄνωθεν, ἔστη πῆγμα ἓν ἀφεστηκὸς μακρὰν σφόδρα σφοδρῶς ἕως μέρους Καριαθιαρίμ· τὸ δὲ καταβαῖνον κατέβη εἰς τὴν θάλασσαν Ἄραβα θάλασσαν ἁλὸς, ἕως εἰς τὸ τέλος ἐξέλιπε· καὶ ὁ λαὸς εἱστήκει ἀπέναντι Ἱεριχώ.
\vs{17}Καὶ ἔστησαν οἱ ἱερεῖς οἱ αἴροντες τὴν κιβωτὸν τῆς διαθήκης Κυρίου ἐπὶ ξηρᾶς ἐν μέσῳ τοῦ Ἰορδάνου· καὶ πάντες οἱ υἱοὶ Ἰσραὴλ διέβαινον διὰ ξηρᾶς, ἕως συνετέλεσε πᾶς ὁ λαὸς διαβαίνων τὸν Ἰορδάνην.

\ch{4}
Καὶ ἐπεὶ συνετέλεσε πᾶς ὁ λαὸς διαβαίνων τὸν Ἰορδανην, καὶ εἶπε, Κύριος τῷ Ἰησοῖ, λέγων,
\vs{2}ταραλαβὼν ἄνδρας ἀπὸ τοῦ λαοῦ, ἕνα ἀφʼ ἑκάστης φυλῆς,
\vs{3}σύνταξον αὐτοῖς· καὶ ἀνέλεσθε ἐκ μέσου τοῦ Ἰορδάνου ἑτοίμους δώδεκα λίθους, καὶ τούτους διακομίσαντες ἅμα ὑμῖν αὐτοῖς, θέτε αὐτοὺς ἐν τῇ στρατοπεδίᾳ ὑμῶν, οὗ ἐὰν παρεμβάλητε ἐκεῖ τὴν νύκτα.

\vs{4}Καὶ ἀνακαλεσάμενος Ἰησοῦς δώδεκα ἄνδρας τῶν ἐνδόξων ἀπὸ τῶν υἱῶν Ἰσραὴλ, ἕνα ἀφʼ ἑκάστης φυλῆς,
\vs{5}εἶπεν αὐτοῖς, προσαγάγετε ἔμπροσθέν μου πρὸ προσώπου Κυρίου εἰς μέσον τοῦ Ἰροδάνου· καὶ ἀνελόμενος ἐκεῖθεν ἕκαστος λίθον, ἀράτω ἐπὶ τῶν ὤμων αὐτοῦ κατὰ τὸν ἀριθμὸν τῶν δώδεκα φυλῶν τοῦ Ἰσραὴλ,
\vs{6}ἵνα ὑπάρχωσιν ὑμῖν οὗτοι εἰς σημεῖον κείμενον διαπαντός· ἵνα ὅταν ἐρωτᾷ σε ὁ υἱός σου αὔριον λέγων, τί εἰσιν οἱ λιθοι οὗτοι ἡμῖν;
\vs{7}Καὶ σὺ δηλώσεις τῷ υἱῷ σου, λέγων, ὅτι ἐξέλιπεν ὁ Ἰορδάνης ποταμὸς ἀπὸ προσώπου κιβωτοῦ διαθήκης Κυρίου πάσης τῆς γῆς, ὡς διέβαινεν αὐτόν· καὶ ἔσονται οἱ λίθοι οὗτοι ὑμῖν μνημόσυνον τοῖς υἱοῖς Ἰσραὴλ ἕως τοῦ αἰῶνος.

\vs{8}Καὶ ἐποίησαν οὕτως οἱ υἱοὶ Ἰσραὴλ, καθότι ἐνετείλατο Κύριος τῷ Ἰησοῖ· καὶ ἀναλαβόντες δώδεκα λίθους ἐκ μέσου τοῦ Ἰορδάνου, καθάπερ συνέταξε Κύριος τῷ Ἰησοῖ ἐν τῇ συντελείᾳ τῆς διαβάσεως τῶν υἱῶν Ἰσραὴλ, καὶ διεκόμισαν ἅμα ἑαυτοῖς εἰς τὴν παρεμβολὴν, καὶ ἀπέθηκαν ἐκεῖ.
\vs{9}Ἔστησε δὲ Ἰησοῦς καὶ ἄλλους δώδεκα λίθους ἐν αὐτῷ τῷ Ἰορδάνῃ, ἐν τῷ γενομένῳ τόπῳ ὑπὸ τοὺς πόδας τῶν ἱερέων τῶν αἰρόντων τὴν κιβωτὸν τῆς διαθήκης Κυρίου· καὶ εἰσὶν ἐκεῖ ἕως τῆς σήμερον ἡμέρας.

\vs{10}Εἱστήκεισαν δὲ οἱ ἱερεῖς οἱ αἴροντες τὴν κιβωτὸν τῆς διαθήκης ἐν τῷ Ἰορδάνῃ, ἕως οὗ συνετέλεσεν Ἰησοῦς πάντα ἃ ἐνετείλατο Κύριος ἀναγγεῖλαι τῷ λαῷ· καὶ ἔσπευσεν ὁ λαὸς, καὶ διέβησαν.
\vs{11}Καὶ ἐγένετο ὡς συνετέλεσε πᾶς ὁ λαὸς διαβῆναι, καὶ διέβη ἡ κιβωτὸς τῆς διαθήκης Κυρίου, καὶ οἱ λίθοι ἔμπροσθεν αὐτῶν.
\vs{12}Καὶ διέβησαν οἱ υἱοὶ Ῥουβὴν, καὶ οἱ υἱοὶ Γὰδ, καὶ οἱ ἡμίσεις φυλῆς Μανασσῆ διεσκευασμένοι ἔμπροσθεν τῶν υἱῶν Ἰσραὴλ, καθάπερ ἐνετείλατο αὐτοῖς Μωυσῆς.
\vs{13}Τετρακισμύριοι εὔζωνοι εἰς μάχην διέβησαν ἐναντίον Κυρίου εἰς πόλεμον πρὸς τὴν Ἱεριχὼ πόλιν.
\vs{14}Ἐν ἐκείνῃ τῇ ἡμέρᾳ ηὔξησε Κύριος τὸν Ἰησοῦν ἐναντίον τοῦ παντὸς γένους Ἰσραὴλ· καὶ ἐφοβοῦντο αὐτὸν, ὥσπερ Μωυσῆν, ὅσον χρόνον ἔζη.

\vs{15}Καὶ εἶπε Κύριος τῷ Ἰησοῖ, λέγων,
\vs{16}ἔντειλαι τοῖς ἱερεῦσι τοῖς αἴρουσι τὴν κιβωτὸν τῆς διαθήκης τοῦ μαρτυρίου Κυρίου, ἐκβῆναι ἐκ τοῦ Ἰορδάνου.
\vs{17}Καὶ ἐνετείλατο Ἰησοῦς τοῖς ἱερεῦσι, λέγων, ἔκβητε ἐκ τοῦ Ἰορδάνου.
\vs{18}Καὶ ἐγένετο ὡς ἐξέβησαν οἱ ἱερεῖς οἱ αἴροντες τὴν κιβωτὸν τῆς διαθήκης Κυρίου ἐκ τοῦ Ἰορδάνου, καὶ ἔθηκαν τοὺς πόδας ἐπὶ τῆς γῆς, ὥρμησε τὸ ὕδωρ τοῦ Ἰορδάνου κατὰ χώραν, καὶ ἐπορεύετο καθὰ χθὲς καὶ τρίτην ἡμέραν διʼ ὅλης τῆς κρηπίδος.

\vs{19}Καὶ ὁ λαὸς ἀνέβη ἐκ τοῦ Ἰορδάνου δεκάτῃ τοῦ μηνὸς τοῦ πρώτου· καὶ κατεστρατοπέδευσαν οἱ υἱοὶ Ἰσραὴλ ἑν Γαλγάλοις κατὰ μέρος τὸ πρὸς ἡλίου ἀνατολὰς ἀπὸ τῆς Ἱεριχώ.
\vs{20}Καὶ τοὺς δώδεκα λίθους τούτους, οὓς ἔλαβεν ἐκ τοῦ Ἰορδάνου, ἔστησεν Ἰησοῦς ἐν Γαλγάλοις,
\vs{21}λέγων, ὅταν ἐρωτῶσιν ὑμᾶς οἱ υἱοι ὑμῶν λέγοντες, τί εἰσιν οἱ λίθοι οὗτοι;
\vs{22}Ἀναγγείλατε τοῖς υἱοῖς ὑμῶν, ὅτι ἐπὶ ξηρᾶς διέβη Ἰσραὴλ τὸν Ἰορδάνην τοῦτον,
\vs{23}ἀποξηράναντος Κυρίου τοῦ Θεοῦ ἡμῶν τὸ ὕδωρ τοῦ Ἰορδάνου ἐκ τῶν ἔμπροσθεν αὐτῶν, μέχρις οὗ διέβησαν· καθάπερ ἐποίησε Κύριος ὁ Θεὸς ἡμῶν τὴν ἐρυθρὰν θάλασσαν, ἣν ἀπεξήρανε Κύριος ὁ Θεὸς ἡμῶν ἔμπροσθεν ἡμῶν, ἕως παρήλθομεν·
\vs{24}Ὅπως γνῶσι πάντα τὰ ἔθνη τῆς γῆς, ὅτι ἡ δύναμις τοῦ κυρίου ἰσχυρά ἐστι, καὶ ἵνα ὑμεῖς σέβησθε Κύριον τὸν Θεὸν ἡμῶν ἐν παντὶ ἔργῳ.

\ch{5}
Καὶ ἐγένετο ὡς ἤκουσαν οἱ βασιλεῖς τῶν Ἀμοῤῥαίων οἳ ἦσαν πέραν τοῦ Ἰορδάνου, καὶ οἱ βασιλεῖς τῆς Φοινίκης οἳ παρὰ τὴν θάλασσαν, ὅτι ἀπεξήρανε Κύριος ὁ Θεὸς τὸν Ἰορδάνην ποταμὸν ἐκ τῶν ἔμπροσθεν τῶν νἱῶν Ἰσραὴλ ἐν τῷ διαβαίνειν αὐτοὺς καὶ ἐτάκησαν αὐτῶν αἱ διάνοιαι, καὶ κατεπλάγησαν, καὶ οὐκ ἦν ἐν αὐτοῖς φρόνησις οὐδεμία ἀπὸ προσώπου τῶν υἱῶν Ἰσραήλ.

\vs{2}Ὑπὸ δὲ τοῦτον τὸν καιρὸν εἶπε Κύριος τῷ Ἰησοῖ, ποίησον σεαυτῷ μαχαίρας πετρίνας ἐκ πέτρας ἀκροτόμου, καὶ καθίσας περίτεμε τοὺς υἱοὺς Ἰσραὴλ ἐκ δευτέρον.
\vs{3}Καὶ ἐποίησεν Ἰησοῦς μαχαίρας πετρίνας ἀκροτόμους, καὶ περιέτεμε τοὺς υἱοὺς Ἰσραὴλ ἐπὶ τοῦ καλουμένου τόπου, Βουνὸς τῶν ἀκροβυστιῶν.
\vs{4}Ὃν δὲ τρόπον περιεκάθαρεν Ἰησοῦς τοὺς υἱοὺς Ἰσραήλ· ὅσοι ποτὲ ἐγένοντο ἐν τῇ ὁδῷ, καὶ ὅσοι ποτὲ ἀπερίτμητοι ἦσαν τῶν ἐξεληλυθότων ἐξ Αἰγύπτου,
\vs{5}πάντας τούτους περιέτεμεν Ἰησοῦς· τεσσαράκοντα γὰρ καὶ δύο ἔτη ἀνέστραπται Ἰσραὴλ ἐν τῇ ἐρήμῳ τῇ Μαβδαρίτιδτ.
\vs{6}Διὸ ἀπερίτμητοι ἦσαν οἱ πλεῖστοι αὐτῶν τῶν μαχίμων τῶν ἐξεληλυθότων ἐκ γῆς Αἰγύπτου, οἱ ἀπειθήσαντες τῶν ἐντολῶν τοῦ Θεοῦ, οἷς καὶ διώρισε μὴ ἰδεῖν αὐτοὺς τὴν γῆν, ἣν ὤμοσε Κύριος τοῖς πατράσιν αὐτῶν δοῦναι γῆν ῥέουσαν γάλα καὶ μέλι.
\vs{7}Ἀντὶ δὲ τούτων ἀντικατέστησε τοὺς υἱοὺς αὐτῶν, οὓς Ἰησοῦς περιέτεμε, διὰ τὸ αὐτοὺς γεγεννῆσθαι κατὰ τὴν ὁδὸν ἀπεριτμήτους.
\vs{8}Περιτμηθέντες δὲ ἡσυχίαν εἶχον αὐτόθι καθήμενοι ἐν τῇ παρεμβολῇ ἕως ὑγιάσθησαν.
\vs{9}Καὶ εἶπε Κύριος τῷ Ἰησοῖ υἱῷ Ναυή, ἐν τῇ σήμερον ἡμέρᾳ ἀφεῖλον τὸν ὀνειδισμὸν Αἰγύπτου ἀφʼ ὑμῶν· καὶ ἐκάλεσε τὸ ὄνομα τοῦ τόπου ἐκείνου, Γάλγαλα.

\vs{10}Καὶ ἐποίησαν οἱ υἱοὶ Ἰσραὴλ τὸ πάσχα τῇ τεσσαρεσκαιδεκάτῃ ἡμέρᾳ τοῦ μηνὸς ἀφʼ ἑσπέρας ἐπὶ δυσμῶν Ἱεριχὼ ἐν τῷ πέραν τοῦ Ἰορδάνου ἐν τῷ πεδίῳ.
\vs{11}Καὶ ἐφάγοσαν ἀπὸ τοῦ σίτου τῆς γῆς ἄζυμα καὶ νέα.
\vs{12}Ἐν ταύτῃ τῇ ἡμέρᾳ ἐξέλιπε τὸ μάννα μετὰ τὸ βεβρωκέναι αὐτοὺς ἐκ τοῦ σίτου τῆς γῆς, καὶ οὐκέτι ὑπῆρχε τοῖς υἱοῖς Ἰσραὴλ μάννα· ἐκαρπίσαντο δὲ τὴν χώραν τῶν Φοινίκων ἐν τῷ ἐνιαυτῷ ἐκείνῳ.

\vs{13}Καὶ ἐγένετο ὡς ἦν Ἰησοῦς ἐν Ἱεριχὼ, καὶ ἀναβλέψας τοῖς ὀφθαλμοῖς εἶδεν ἄνθρωπον ἑστηκότα ἐναντίον αὐτοῦ, καὶ ἡ ῥομφαία ἐσπασμένη ἐν τῇ χειρὶ αὐτοῦ· καὶ προσελθὼν Ἰησοῦς, εἶπεν αὐτῷ, ἡμέτερος εἶ, ἢ τῶν ὑπεναντίων;
\vs{14}Ὁ δὲ εἶπεν αὐτῷ, ἐγὼ ἀρχιστράτηγος δυνάμεως Κυρίου, νυνὶ παραγέγονα. Καὶ Ἰησοῦς ἔπεσεν ἐπὶ πρόσωπον ἐπὶ τὴν γῆν, καὶ εἶπεν αὐτῷ, δέσποτα, τί προστάσσεις τῷ σῷ οἰκέτῃ;
\vs{15}Καὶ λέγει ὁ ἀρχιστράτηγος Κυρίου πρὸς Ἰησοῦν, Λῦσαι τὸ ὑπόδημα ἐκ τῶν ποδῶν σου, ὁ γὰρ τόπος ἐφʼ ᾧ νῦν ἔστηκας ἐπʼ αὐτοῦ, ἅγιός ἐστι.

\ch{6}
Καὶ Ἱεριχὼ συρκεκλεισμένη καὶ ὠχυρωμένη, καὶ οὐδεὶς ἐξεπορεύετο ἐξ αὐτῆς, οὐδὲ εἰσεπορεύετο.
\vs{2}Καὶ εἶπε Κύριος πρὸς Ἰησοῦν, ἰδον ἐγὼ παραδίδωμι ὑποχείριόν σοι τὴν Ἱεριχὼ, καὶ τὸν βασιλέα αὐτῆς τὸν ἐν αὐτῇ, δυνατοὺς ὄντας ἐν ἰσχύϊ.
\vs{3}Σὺ δὲ περίστησον αὐτῇ τοὺς μαχίμους κύκλῳ.
\vs{5}Καὶ ἔσται ὡς ἂν σαλπίσητε τῇ σάλπιγγι, ἀνακραγέτω πᾶς ὁ λαὸς ἅμα, καὶ ἀνακραγόντων αὐτῶν πεσεῖται αὐτόματα τὰ τείχη τῆς πόλεως, καὶ εἰσελεύσεται πᾶς ὁ λαὸς ὁρμήσας ἕκαστος κατὰ πρόσωπον εἰς τὴν πόλιν.

\vs{6}Καὶ εἰσῆλθεν Ἰησοῦς ὁ τοῦ Ναυῆ πρὸς τοὺς ἱερεῖς,
\vs{7}καὶ εἶπεν αὐτοῖς, λέγων, παραγγείλατε τῷ λαῷ περιελθεῖν, καὶ κυκλῶσαι τὴν πόλιν· καὶ οἱ μάχιμοι παραπορευέσθωσαν ἐνωπλισμένοι ἐναντίον Κυρίου.
\vs{8}Καὶ ἑπτὰ ἱερεῖς ἔχοντες ἑπτὰ σάλπιγγας ἱερὰς παρελθέτωσαν ὡσαύτως ἐναντίον τοῦ Κυρίου, καὶ σημαινέτωσαν εὐτόνως· καὶ ἡ κιβωτὸς τῆς διαθήκης Κυρίου ἐπακολουθείτω.
\vs{9}Οἱ δὲ μάχιμοι παραπορευέσθωσαν ἔμπροσθεν, καὶ οἱ ἱερεῖς οἱ οὐραγοῦντες ὀπίσω τῆς κιβωτοῦ τῆς διαθήκης Κυρίου σαλπίζοντες.
\vs{10}Τῷ δὲ λαῷ ἐνετείλατο Ἰησοῦς, λέγων, μὴ βοᾶτε, μηδὲ ἀκουσάτω μηδεὶς τὴν φωνὴν ὑμῶν, ἕως ἂν ἡμέραν διαγγείλῃ αὐτὸς ἀναβοῆσαι, καὶ τότε ἀναβοήσετε·
\vs{11}Καὶ περιελθοῦσα ἡ κιβωτὸς τῆς διαθήκης τοῦ Θεοῦ εὐθέως ἀπῆλθεν εἰς τὴν παρεμβολὴν, καὶ ἐκοιμήθη ἐκεῖ.

\vs{12}Καὶ τῇ ἡμέρᾳ τῇ δευτέρᾳ ἀνέστη Ἰησοῦς τοπρωῒ, καὶ ᾖραν οἱ ἱερεῖς τὴν κιβωτὸν τῆς διαθήκης Κυρίου.
\vs{13}Καὶ οἱ ἑπτὰ ἱερεῖς οἱ φέροντες τὰς σάλπιγγας τὰς ἑπτὰ προεπορεύοντο ἐναντίον Κυρίου· καὶ μετὰ ταῦτα εἰσεπορεύοντο οἱ μάχιμοι, καὶ ὁ λοιπὸς ὄχλος ὄπισθεν τῆς κιβωτοῦ τῆς διαθήκης Κυρίου·
\vs{14}καὶ οἱ ἱερεῖς ἐσάλπισαν ταῖς σάλπιγξι, καὶ ὁ λοιπὸς ὄχλος ἅπας περιεκύκλωσε τὴν πόλιν ἑξάκις ἐγγύθεν, καὶ ἀπῆλθε πάλιν εἰς τὴν παρεμβολήν· οὕτως ἐποίει ἐπὶ ἓξ ἡμέρας.

\vs{15}Καὶ τῇ ἡμέρᾳ τῇ ἑβδόμῃ ἀνέστησαν ὄρθρου, καὶ περιήλθοσαν τὴν πόλιν ἐν τῇ ἡμέρᾳ ἐκείνῃ ἑπτάκις.
\vs{16}Καὶ ἐγένετο τῇ περιόδῳ τῇ ἑβδόμῃ ἐσάλπισαν οἱ ἱερεῖς· καὶ εἶπεν Ἰησοῦς τοῖς υἱοῖς Ἰσραὴλ, κεκράξατε, παρέδωκε γὰρ Κύριος ὑμῖν τὴν πόλιν.
\vs{17}Καὶ ἔσται ἡ πόλις ἀνάθεμα, αὐτὴ καὶ πάντα ὅσα ἐστὶν ἐν αὐτῇ, Κυρὶῳ σαβαώθ· πλὴν Ῥαὰβ τὴν πόρνην περιποιήσασθε αὐτὴν, καὶ πάντα ὅσα ἐστὶν ἐν τῷ οἴκῳ αὐτῆς.
\vs{18}Ἀγγὰ ὑμεῖς φυλάξεσθε σφόδρα ἀπὸ τοῦ ἀναθέματος, μήποτε ἐνθυμηθέντες ὑμεῖς αὐτοὶ λάβητε ἀπὸ τοῦ ἀναθέματος, καὶ ποιήσητε τὴν παρεμβολὴν τῶν υἱῶν Ἰσραὴλ ἀνάθεμα, καὶ ἐκτρίψητε ἡμᾶς.
\vs{19}Καὶ πᾶν ἀργύριον ἢ χρυσίον, ἢ χαλκὸς ἢ σίδηρος, ἅγιον ἔσται τῷ Κυρὶῳ· εἰς θησαυρὸν Κυρίου εἰσενεχθήσεται.

\vs{20}Καὶ ἐσάλπισαν ταῖς σάλπιγξιν οἱ ἱερεῖς· ὡς δὲ ἤκουσεν ὁ λαὸς τῶν σαλπίγγων, ἠλάλαξε πᾶς ὁ λαὸς ἅμα ἀλαλαγμῷ μεγάλῳ καὶ ἰσχυρῷ· καὶ ἔπεσεν ἅπαν τὸ τεῖχος κύκλῳ· καὶ ἀνέβη πᾶς ὁ λαὸς εἰς τὴν πόλιν.
\vs{21}Καὶ ἀνεθεμάτισεν αὐτὴν Ἰησοῦς, καὶ ὅσα ἦν ἐν τῇ πόλει ἀπὸ ἀνδρὸς καὶ ἕως γυναικὸς, ἀπὸ νεανίσκου καὶ ἕως πρεσβύτου, καὶ ἕως μόσχου καὶ ὑποζυγίου, ἐν στόματι ῥομφαίας.

\vs{22}Καὶ τοῖς δυσὶ νεανίσκοις τοῖς κατασκοπεύσασιν εἶπεν Ἰησοῦς, εἰσέλθατε εἰς τὴν οἰκίαν τῆς γυναικὸς, καὶ ἐξαγάγετε αὐτὴν ἐκεῖθεν, καὶ ὅσα ἐστὶν αὐτῇ.
\vs{23}Καὶ εἰσῆλθον οἱ δύο νεανίσκοι οἱ κατασκοπεύσαντες τὴν πόλιν, εἰς τὴν οἰκίαν τῆς γυναικὸς, καὶ ἐξηγάγοσαν Ῥαὰβ τὴν πόρνην, καὶ τὸν πατέρα αὐτῆς, καὶ τὴν μητέρα αὐτῆς, καὶ τοὺς ἀδελφοὺς αὐτῆς, καὶ τὴν συγγένειαν αὐτῆς, καὶ πάντα ὅσα ἦν αὐτῆ· καὶ κατέστησαν αὐτὴν ἔξω τῆς παρεμβολῆς Ἰσραήλ.
\vs{24}Καὶ ἡ πόλις ἐνεπρήσθη ἐν πυρισμῷ σὺν πᾶσι τοῖς ἐν αὐτῇ· πλὴν ἀργυρίου καὶ χρυσίου καὶ χαλκοῦ καὶ σιδήρου ἔδωκαν εἰς θησαυρὸν Κυρίου εἰσενεχθῆναι.

\vs{25}Καὶ Ῥαὰβ τὴν πόρνην, καὶ πάντα τὸν οἶκον αὐτῆς τὸν πατρικὸν ἐζώγρησεν Ἰησοῦς· καὶ κατῴκισεν ἐν τῷ Ἰσραὴλ ἕως τῆς σήμερον ἡμέρας, διότι ἔκρυψε τοὺς κατασκοπεύαντας, οὓς ἀπέστειλεν Ἰησοῦς κατασκοπεῦσαι τὴν Ἱεριχώ.
\vs{26}Καὶ ὥρκισεν Ἰησοῦς ἐν τῇ ἡμέρᾳ ἐκείνῃ ἐναντίον Κυρίου, λέγων, ἐπικατάρατος ὁ ἄνθρωπος, ὃς οἰκοδομήσει τὴν πόλιν ἐκείνην· ἐν τῷ πρωτοτόκῳ αὐτοῦ θεμελιώσει αὐτὴν, καὶ ἐν τῷ ἐλαχίστῳ αὐτοῦ ἐπιστήσει τὰς πύλας αὐτῆς. Καὶ οὕτως ἐποίησεν Ὁζᾶν ὁ ἐκ Βαιθήλ· ἐν τῷ Ἀβιρὼν τῷ πρωτοτόκῳ ἐθεμελίωσεν αὐτὴν, καὶ ἐν τῷ ἐλαχίστῳ διασωθέντι ἐπέστησε τὰς πύλας αὐτῆς.

\vs{27}Καὶ ἦν Κύριος μετὰ Ἰησοῦ, καὶ ἦν τὸ ὄνομα αὐτοῦ κατὰ πᾶσαν τὴν γῆν.

\ch{7}
Καὶ ἐπλημμέλησαν οἱ υἱοὶ Ἰσραὴλ πλημμέλιαν μεγάλην, καὶ ἐνοσφίσαντο ἀπὸ τοῦ ἀναθέματος· καὶ ἔλαβεν Ἄχαρ υἱὸς Χαρμὶ υἱοῦ Ζαμβρὶ υἱοῦ Ζαρὰ ἐκ τῆς φυλῆς Ἰούδα ἀπὸ τοῦ ἀναθέματος· καὶ ἐθυμώθη Κύριος ὀργῇ τοῖς υἱοῖς Ἰσραήλ.

\vs{2}Καὶ ἀπέστειλεν Ἰησοῦς ἄνδρας εἰς Γαὶ, ἥ ἐστι κατὰ Βαιθὴλ, λέγων, κατασκέψασθε τὴν Γαί
\vs{3}καὶ ἀνέβησαν οἱ ἄνδρες καὶ κατεσκέψαντο τὴν Γαί· Καὶ ἀνέστρεψαν πρὸς Ἰησοῦν, καὶ εἶπαν πρὸς αὐτὸν, μὴ ἀναβήτω πᾶς ὁ λαὸς, ἀλλʼ ὡσεὶ δισχίλιοι ἢ τρισχίλιοι ἄνδρες ἀναβήτωσαν καὶ ἐκπολιορκησάτωσαν τὴν πόλιν· μὴ ἀναγάγῃς ἐκεῖ τὸν λαὸν ἅπαντα, ὀλίγοι γάρ εἰσι.
\vs{4}Καὶ ἀνέβησαν ὡσεὶ τρισχίλιοι ἄνδρες, καὶ ἔφυγον ἀπὸ προσώπου ἀνδρῶν Γαί.
\vs{5}Καὶ ἀπέκτειναν ἀπʼ αὐτῶν ἄνδρες Γαὶ εἰς τριακονταὲξ ἄνδρας, καὶ κατεδίωξαν αὐτοὺς ἀπὸ τῆς πύλης, καὶ συνέτριψαν αὐτοὺς ἀπὸ τοῦ καταφεροῦς· καὶ ἐπτοήθη ἡ καρδία τοῦ λαοῦ, καὶ ἐγένετο ὥσπερ ὕδωρ.

\vs{6}Καὶ διέῤῥηξεν Ἰησοῦς τὰ ἱμάτια αὐτοῦ· καὶ ἔπεσεν Ἰησοῦς ἐπὶ τὴν γῆν ἐπὶ πρόσωπον ἐναντίον Κυρίου ἕως ἑσπέρας, αὐτὸς καὶ οἱ πρεσβύτεροι Ἰσραήλ· καὶ ἐπεβάλοντο χοῦν ἐπὶ τὰς κεφαλὰς αὐτῶν.
\vs{7}Καὶ εἶπεν Ἰησοῦς, δέομαι Κύριε· ἱνατί διεβίβασεν ὁ παῖς σου τὸν λαὸν τοῦτον τὸν Ἰορδάνην παραδοῦναι αὐτὸν τῷ Ἀμοῤῥαίῳ, ἀπολέσαι ἡμᾶς; καὶ εἰ κατεμείναμεν καὶ κατῳκίσθημεν παρὰ τὸν Ἰορδάνην.
\vs{8}Καὶ τί ἐρῶ ἐπεὶ μετέβαλεν Ἰσραὴλ αὐχένα ἀπέναντι τοῦ ἐχθροῦ αὐτοῦ;
\vs{9}Καὶ ἀκούσας ὁ Χαναναῖος καὶ πάντες οἱ κατοικοῦντες τὴν γῆν, περικυκλώσουσιν ἡμᾶς, καὶ ἐκτρίψουσιν ἡμᾶς ἀπὸ τῆς γῆς· καὶ τί ποιήσεις τὸ ὄνομά σου τὸ μέγα;

\vs{10}Καὶ εἶπε Κύριος πρὸς Ἰησοῦν, ἀνάστηθι, ἱνατί τοῦτο σὺ πέπτωκας ἐπὶ πρόσωπόν σου;
\vs{11}Ἡμάρτηκεν ὁ λαὸς καὶ παρέβη τὴν διαθήκην, ἣν διεθέμην πρὸς αὐτοὺς, κλέψαντες ἀπὸ τοῦ ἀναθέματος ἐνέβαλον εἰς τὰ σκεύη αὐτῶν.
\vs{12}Καὶ οὐ μὴ δύνωνται οἱ υἱοὶ Ἰσραὴλ ὑποστῆναι κατὰ πρόσωπον τῶν ἐχθρῶν αὐτῶν· αὐχένα ὑποστρέψουσιν ἔναντι τῶν ἐχθρῶν αὐτῶν, ὅτι ἐγενήθησαν ἀνάθεμα· οὐ προσθήσω ἔτι εἶναι μεθʼ ὑμῶν, ἐὰν μὴ ἐξάρητε τὸ ἀνάθεμα ἐξ ὑμῶν αὐτῶν.
\vs{13}Ἀναστὰς ἁγίασον τὸν λαὸν, καὶ εἶπον ἁγιασθῆναι εἰσαύριον· τάδε λέγει Κύριος ὁ Θεὸς Ἰσραὴλ, τὸ ἀνάθεμά ἐστιν ἐν ὑμῖν· οὐ δυνήσεσθε ἀντιστῆναι ἀπέναντι τῶν ἐχθρῶν ὑμῶν, ἕως ἂν ἐξάρητε τὸ ἀνάθεμα ἐξ ὑμῶν.
\vs{14}Καὶ συναχθήσεσθε πάντες τοπρωῒ κατὰ φυλὰς, καὶ ἔσται ἡ φυλὴ ἣν ἂν δείξῃ Κύριος, προσάξετε κατὰ δήμους· καὶ τὸν δῆμον ὃν ἂν δείξῃ Κύριος, προσάξετε κατʼ οἶκον· καὶ τὸν οἶκον ὃν ἂν δέξῃ Κύριος, προσάξετε κατʼ ἄνδρα.
\vs{15}Καὶ ὃς ἂν ἐνδειχθῇ, κατακαυθήσεται ἐν πυρὶ, καὶ πάντα ὅσα ἐστὶν αὐτῷ· ὅτι παρέβη τὴν διαθήκην Κυρίου, καὶ ἐποίησεν ἀνόμημα ἐν Ἰσραήλ.

\vs{16}Καὶ ὤρθρισεν Ἰησοῦς, καὶ προσήγαγε τὸν λαὸν κατὰ φυλάς· καὶ ἐνεδείχθη ἡ φυλὴ Ἰούδα.
\vs{17}Καὶ προσήχθη κατὰ δήμους, καὶ ἐνεδείχθη δῆμος Ζαραΐ. Καὶ προσήχθη κατʼ ἄνδρα,
\vs{18}καὶ ἐνεδείχθη Ἄχαρ υἱὸς Ζαμβρὶ υἱοῦ Ζάρά.

\vs{19}Καὶ εἶπεν Ἰησοῦς τῷ Ἄχαρ, δὸς δόξαν σήμερον τῷ Κυρίῳ Θεῷ Ἰσραὴλ, καὶ δὸς τὴν ἐξομολόγησιν, καὶ ἀνάγγειλόν μοι τί ἐποίησας, καὶ μὴ κρύψῃς ἀπʼ ἐμοῦ.
\vs{20}Καὶ ἀπεκρίθη Ἄχαρ τῷ Ἰησοῖ, καὶ εἶπεν, ἀληθῶς ἥμαρτον ἐναντίον Κυρίου τοῦ Θεοῦ Ἰσραήλ· οὕτως καὶ οὕτως ἐποίησα.
\vs{21}Εἶδον ἐν τῇ προνομῇ ψιλὴν ποικίλην, καὶ διακόσια δίδραχμα ἀργυρίου, καὶ γλῶσσαν μίαν χρυσῆν πεντήκοντα διδράχμων, καὶ ἐνθυμηθεὶς αὐτῶν ἔλαβον· καὶ ἰδοὺ αὐτὰ ἐγκέκρυπται ἐν τῇ σκηνῇ μου, καὶ τὸ ἀργύριον κέκρυπται ὑποκάτω αὐτῶν.
\vs{22}Καὶ ἀπέστειλεν Ἰησοῦς ἀγγέλους, καὶ ἔδραμον εἰς τὴν σκηνὴν εἰς τὴν παρεμβολήν· καὶ ταῦτα ἦν κεκρυμμένα εἰς τὴν σκηνὴν αὐτοῦ, καὶ τὸ ἀργύριον ὑποκάτω αὐτῶν.
\vs{23}Καὶ ἐξήνεγκαν αὐτὰ ἐκ τῆς σκηνῆς, καὶ ἤνεγκαν πρὸς Ἰησοῦν καὶ τοὺς πρεσβυτέρους Ἰσραὴλ, καὶ ἔθηκαν αὐτὰ ἔναντι Κυρίου.

\vs{24}Καὶ ἔλαβεν Ἰησοῦς τὸν Ἄχαρ υἱὸν Ζαρὰ, καὶ ἀνήγαγεν αὐτὸν εἰς φάραγγα Ἀχὼρ, καὶ τοὺς υἱοὺς αὐτοῦ, καὶ τὰς θυγατέρας αὐτοῦ, καὶ τοὺς μόσχους αὐτοῦ, καὶ τὰ ὑποζύγια αὐτοῦ, καὶ πάντα τὰ πρόβατα αὐτοῦ, καὶ τὴν σκηνὴν αὐτοῦ, καὶ πάντα τὰ ὑπάρχοντα αὐτοῦ, καὶ πᾶς ὁ λαὸς μετʼ αὐτοῦ· καὶ ἀνήγαγεν αὐτοὺς εἰς Ἐμεκαχώρ.
\vs{25}Καὶ εἶπεν Ἰησοῦς τῷ Ἄχαρ, τί ὠλόθρευσας ἡμᾶς; ἐξολοθρεύσαι σε Κύριος καθὰ καὶ σήμερον.
\vs{26}Καὶ ἐλιθοβόλησαν αὐτὸν λίθοις πᾶς Ἰσραὴλ, καὶ ἐπέστησαν αὐτῷ σωρὸν λίθων μέγαν· καὶ ἐπαύσατο Κύριος τοῦ θυμοῦ τῆς ὀργῆς. Διὰ τοῦτο ἐπωνόμασεν αὐτὸ Ἐμεκαχὼρ ἕως τῆς ἡμέρας ταύτης.

\ch{8}
Καὶ εἶπε Κύριος πρὸς Ἰησοῦν, μὴ φοβηθῇς, μηδὲ δειλιάσῃς· λάβε μετὰ σοῦ πάντας τοὺς ἄνδρας τοὺς πολεμιστάς, καὶ ἀναστὰς ἀνάβηθι εἰς Γαί· ἰδοὺ δέδωκα εἰς τὰς χεῖράς σου τὸν βασιλέα Γαὶ, καὶ τὴν γῆν αὐτοῦ.
\vs{2}Καὶ ποιήσεις τὴν Γαὶ, ὃν τρόπον ἐποίησας τὴν Ἱεριχὼ, καὶ τὸν βασιλέα αὐτῆς· καὶ τὴν προνομὴν τῶν κτηνῶν προνομεύσεις σεαυτῷ· κατάστησον δὲ σεαυτῷ ἔνεδρα τῇ πόλει εἰς τὰ ὀπίσω.

\vs{3}Καὶ ἀνέστη Ἰησοῦς καὶ πᾶς ὁ λαὸς ὁ πολεμιστὴς ὥστε ἀναβῆναι εἰς Γαί. ἐπέλεξε δὲ Ἰησοῦς τριάκοντα χιλιάδας ἀνδρῶν δυνατοὺς ἐν ἰσχύϊ, καὶ ἀπέστειλεν αὐτοὺς νυκτός.
\vs{4}Καὶ ἐνετείλατο αὐτοῖς, λέγων, ὑμεῖς ἐνεδρεύσατε ὀπίσω τῆς πόλεως· μὴ μακρὰν γίνεσθε ἀπὸ τῆς πόλεως, καὶ ἔσεσθε πάντες ἕτοιμοι.
\vs{5}Καὶ ἐγὼ καὶ πάντες οἱ μετʼ ἐμοῦ προσάξομεν πρὸς τὴν πόλιν· καὶ ἔσται ὡς ἂν ἐξέλθωσιν οἱ κατοικοῦντες Γαὶ εἰς συνάντησιν ἡμῖν, καθάπερ καὶ πρώην, καὶ φευξόμεθα ἀπὸ προσώπου αὐτῶν.
\vs{6}Καὶ ὡς ἂν ἐξέλθωσιν ὀπίσω ἡμῶν, ἀποσπάσομεν αὐτοὺς ἀπὸ τῆς πόλεως· καὶ ἐροῦσι, φεύγουσιν οὗτοι ἀπὸ προσώπου ἡμῶν, ὃν τρόπον καὶ ἔμπροσθεν.
\vs{7}Ὑμεῖς δὲ ἐξαναστήσεσθε ἐκ τῆς ἐνέδρας, καὶ πορεύσεσθε εἰς τὴν πόλιν.
\vs{8}Κατὰ τὸ ῥῆμα τοῦτο ποιήσετε· ἰδοὺ ἐντέταλμαι ὑμῖν.
\vs{9}Καὶ ἀπέστειλεν αὐτοὺς Ἰησοῦς, καὶ ἐπορεύθησαν εἰς τὴν ἔνεδραν· καὶ ἐνεκάθισαν ἀναμέσον Βαιθὴλ καὶ ἀναμέσον Γαὶ, ἀπὸ θαλάσσης τῆς Γαί.

\vs{10}Καὶ ὀρθρίσας Ἰησοῦς τοπρωῒ, ἐπεσκέψατο τὸν λαόν· καὶ ἀνέβησαν αὐτὸς καὶ οἱ πρεσβύτεροι κατὰ πρόσωπον τοῦ λαοῦ ἐπὶ Γαί.
\vs{11}Καὶ πᾶς ὁ λαὸς ὁ πολεμιστὴς μετʼ αὐτοῦ ἀνέβησαν· καὶ πορευόμενοι ἦλθον ἐξεναντίας τῆς πόλεως ἀπὸ ἀνατολῶν.
\vs{12}Καὶ τὰ ἔνεδρα τῆς πόλεως ἀπὸ θαλάσσης·
\vs{14}Καὶ ἐγένετο ὡς εἶδε βασιλεὺς Γαὶ, ἔσπευσε καὶ ἐξῆλθεν εἰς συνάντησιν αὐτοῖς ἐπʼ εὐθείας εἰς τὸν πόλεμον, αὐτὸς καὶ πᾶς ὁ λαὸς ὁ μετʼ αὐτοῦ· καὶ αὐτὸς οὐκ ᾔδει ὅτι ἔνεδρα αὐτῷ ἐστιν ὀπίσω τῆς πόλεως.
\vs{15}Καὶ εἶδε, καὶ ἀνεχώρησεν Ἰησοῦς καὶ Ἰσραὴλ ἀπὸ προσώπου αὐτῶν.
\vs{16}Καὶ κατεδίωξαν ὀπίσω τῶν υἱῶν Ἰσραήλ· καὶ αὐτοὶ ἀπέστησαν ἀπὸ τῆς πόλεως.
\vs{17}Οὐ κατελείφθη οὐδεὶς ἐν τῇ Γαὶ, ὃς οὐ κατεδίωξεν ὀπίσω Ἰσραήλ· καὶ κατέλιπον τὴν πόλιν ἠνεῳγμένην, καὶ κατεδίωξαν ὀπίσω Ἰσραήλ·

\vs{18}Καὶ εἶπε Κύριος πρὸς Ἰησοῦν, ἔκτεινον τὴν χεῖρά σου ἐν τῷ γαισῷ τῷ ἐν τῇ χειρί σου ἐπὶ τὴν πόλιν, εἰς γὰρ τὰς χεῖράς σου παραδέδωκα αὐτήν· καὶ τὰ ἔνεδρα ἐξαναστήσονται ἐν τάχει ἐκ τοῦ τόπου αὐτῶν. Καὶ ἐξέτεινεν Ἰησοῦς τὴν χεῖρα αὐτοῦ τὸν γαισὸν ἐπὶ τὴν πόλιν·
\vs{19}καὶ τὰ ἔνεδρα ἐξανέστησαν ἐν τάχει ἐκ τοῦ τόπου αὐτῶν· καὶ ἐξήλθοσαν ὅτε ἐξέτεινε τὴν χεῖρα, καὶ εἰσήλθοσαν καὶτὴν πόλιν, καὶ κατελάβοντο αὐτήν· καὶ σπεύσαντες ἐνέπρησαν τὴν πόλιν ἐν πυρί.
\vs{20}Καὶ περιβλέψαντες οἱ κάτοικοι Γαὶ εἰς τὰ ὀπίσω αὐτῶν, καὶ ἐθεώρουν καπνὸν ἀναβαίνοντα ἐκ τῆς πόλεως εἰς τὸν οὐρανόν· καὶ οὐκ ἔτι εἶχον ποῦ φύγωσιν ὧδε ἢ ὧδε.
\vs{21}Καὶ Ἰησοῦς καὶ πᾶς Ἰσραὴλ εἶδον, ὅτι ἔλαβον τὰ ἔνεδρα τὴν πόλιν, καὶ ὅτι ἀνέβη ὁ καπνὸς τῆς πόλεως εἰς τὸν οὐρανόν· καὶ μεταβαλλόμενοι, ἐπάταξαν τοὺς ἄνδρας τῆς Γαί.
\vs{22}Καὶ οὗτοι ἐξήλθοσαν ἐκ τῆς πόλεως εἰς συνάντησιν· καὶ ἐκενήθησαν ἀναμέσον τῆς παρεμβολῆς, οὗτοι ἐντεῦθεν καὶ οὗτοι ἐντεῦθεν· καὶ ἐπάταξαν αὐτοὺς ἕως τοῦ μὴ καταλειφθῆναι αὐτῶν σεσωσμένον καὶ διαπεφευγότα.
\vs{23}Καὶ τὸν βασιλέα τῆς Γαὶ συνέλαβον ζῶντα, καὶ προσήγαγον αὐτὸν πρὸς Ἰησοῦν.

\vs{24}Καὶ ὡς ἐπαύσαντο οἱ υἱοὶ Ἰσραὴλ ἀποκτείνοντες πάντας τοὺς ἐν τῇ Γαὶ, καὶ τοὺς ἐν τοῖς πεδίοις, καὶ ἐν τῷ ὄρει ἐπὶ τῆς καταβάσεως, οὗ κατεδίωξαν αὐτοὺς ἀπʼ αὐτῆς εἰς τέλος, καὶ ἐπέστρεψεν Ἰησοῦς εἰς Γαὶ, καὶ ἐπάταξεν αὐτὴν ἐν στόματι ῥομφαίας.
\vs{25}Καὶ ἐγενήθησαν οἱ πεσόντες ἐν τῇ ἡμέρᾳ ἐκείνῃ ἀπὸ ἄνδρος καὶ ἕως γυναικὸς, δώδεκα χιλιάδες, πάντας τοὺς κατοικοῦντας Γαί.
\vs{27}Πλὴν τῶν σκύλων τῶν ἐν τῇ πόλει πάντα, ἃ ἐπρονόμευσαν ἑαυτοῖς οἱ υἱοὶ Ἰσραήλ κατὰ πρόσταγμα Κυρίου, ὃν τρόπον συνέταξε Κύριος τῷ Ἰησοῖ.

\vs{28}Καὶ ἐνεπύρισεν Ἰησοῦς τὴν πόλιν ἐν πυρί· χῶμα ἀοίκητον εἰς τὸν αἰῶνα ἔθηκεν αὐτὴν ἕως τῆς ἡμέρας ταύτης.
\vs{29}Καὶ τὸν βασιλέα τῆς Γαὶ ἐκρέμασεν ἐπὶ ξύλου διδύμου· καὶ ἦν ἐπὶ τοῦ ξύλου ἕως ἑσπέρας· καὶ ἐπιδύνοντος τοῦ ἡλίου συνέταξεν Ἰησοῦς, καὶ καθείλοσαν τὸ σῶμα αὐτοῦ ἀπὸ τοῦ ξύλου, καὶ ἔῤῥιψαν αὐτὸ εἰς τὸν βόθρον· καὶ ἐπέστησαν αὐτῷ σωρὸν λίθων, ἕως τῆς ἡμέρας ταύτης.

\ch{9}
Ὡς δὲ ἤκουσαν οἱ βασιλεῖς τῶν Ἀμοῤῥαίων οἱ ἐν τῷ πέραν τοῦ Ἰορδάνου, οἱ ἐν τῇ ὀρεινῇ, καὶ οἱ ἐν τῇ πεδινῇ, καὶ οἱ ἐν πάσῃ τῇ παραλίᾳ τῆς θαλάσσης τῆς μεγάλης, καὶ οἱ πρὸς τῷ Ἀντιλιβάνῳ, καὶ οἱ Χετταῖοι, καὶ οἱ Χαναναῖοι, καὶ οἱ Φερεζαῖοι, καὶ οἱ Εὐαῖοι, καὶ οἱ Ἀμοῤῥαῖοι, καὶ οἱ Γεργεσαῖοι, καὶ οἱ Ἰεβουσαῖοι,
\vs{2}συνήλθοσαν ἐπὶ τὸ αὐτὸ ἐκπολεμῆσαι Ἰησοῦν καὶ Ἰσραὴλ ἅμα πάντες.

\vs{2a}Τότε ᾠκοδόμησεν Ἰησοῦς θυσιαστήριον Κυρίῳ τῷ Θεῷ Ἰσραὴλ ἐν ὄρει Γαιβὰλ,
\vs{2b}καθότι ἐνετείλατο Μωυσῆς ὁ θεράπων Κυρίου τοῖς υἱοῖς Ἰσραῆλ, καθὰ γέγραπται ἐν τῷ νόμῳ Μωυσῆ, θυσιαστήριον λίθων ὁλοκλήρων, ἐφʼ οὓς οὐκ ἐπεβλήθη σίδηρος· καὶ ἀνεβίβασεν ἐκεῖ ὁλοκαυτώματα Κυρίῳ, καὶ θυσίαν σωτηρίου.
\vs{2c}Καὶ ἔγραψεν Ἰησοῦς ἐπὶ τῶν λίθων τὸ δευτερονόμιον, νόμον Μωυσῆ, ἐνώπιον τῶν υἱῶν Ἰσραήλ.
\vs{2d}Καὶ πᾶς Ἰσραὴλ, καὶ οἱ πρεσβύτεροι αὐτῶν, καὶ οἱ δικασταὶ, καὶ οἱ γραμματεῖς αὐτῶν, παρεπορεύοντο ἔνθεν καὶ ἔνθεν τῆς κιβωτοῦ ἀπέναντι· καὶ οἱ ἱερεῖς καὶ οἱ Λευῖται ᾖρν τὴν κιβωτὸν τῆς διαθήκης Κυρίου· καὶ ὁ προσήλυτος καὶ ὁ αὐτόχθων, οἳ ἦσαν ἥμισυ πλησίον ὄρους Γαριζὶν, καὶ οἳ ἦσαν ἥμισυ πλησίον ὄρους Γαιβὰλ, καθότι ἐνετείλατο Μωυσῆς ὁ θεράπων Κυρίου εὐλογῆσαι τὸν λαὸν ἐν πρώτοις.

\vs{2e}Καὶ μετὰ ταῦτα οὕτως ἀνέγνω Ἰησοῦς πάντα τὰ ῥήματα τοῦ νόμου τούτου, τὰς εὐλογίας καὶ τὰς κατάρας, κατὰ πάντα τὰ γεγραμμένα ἐν τῷ νόμῳ Μωυσῆ.
\vs{2f}Οὐκ ἦν ῥῆμα ἀπὸ πάντων ὧν ἐνετείλατο Μωυσῆς τῷ Ἰησοῖ, ὃ οὐκ ἀνέγνω Ἰησοῦς εἰς τὰ ὦτα πάσης ἐκκλησίας υἱῶν Ἰσραὴλ, τοῖς ἀνδράσι καὶ ταῖς γυναιξὶ καὶ τοῖς παιδίοις καὶ τοῖς προσηλύτοις τοῖς προσπορευομένοις τῷ Ἰσραήλ.

\vs{3}Καὶ οἱ κατοικοῦντες Γαβαὼν ἤκουσαν πάντα ὅσα ἐποίησε Κύριος τῇ Ἱεριχὼ καὶ τῇ Γαί.
\vs{4}Καὶ ἐποίησαν καί γε αὐτοὶ μετὰ πανουργίας· καὶ ἐλθόντες ἐπεσιτίσαντο καὶ ἡτοιμάσαντο· καὶ λαβόντες σάκκους παλαιοὺς ἐπὶ τῶν ὤμων αὐτῶν, καὶ ἀσκοὺς οἴνου παλαιοὺς καὶ κατεῤῥωγότας ἀποδεδεμένους,
\vs{5}καὶ τὰ κοῖλα τῶν ὑποδημάτων αὐτῶν, καὶ τὰ σανδάλια αὐτῶν παλαιὰ καὶ καταπεπελματωμένα ἐν τοῖς ποσὶν αὐτῶν, καὶ τὰ ἱμάτια αὐτῶν πεπαλαιωμένα ἐπάνω αὐτῶν, καὶ ὁ ἄρτος αὐτῶν τοῦ ἐπισιτισμοῦ ξηρὸς καὶ εὐρωτιῶν καὶ βεβρωμένος.

\vs{6}Καὶ ἤλθοσαν πρὸς Ἰησοῦν εἰς τὴν παρεμβολὴν Ἰσραὴλ εἰς Γάλγαλα, καὶ εἶπαν πρὸς Ἰησοῦν καὶ Ἰσραήλ, ἐκ γῆς μακρόθεν ἥκαμεν· καὶ νῦν διάθεσθε ἡμῖν διαθήκην.
\vs{7}Καὶ εἶπαν οἱ υἱοὶ Ἰσραὴλ πρὸς τὸν Χοῤῥαῖον, ὅρα μὴ ἐν ἐμοὶ κατοικεῖς· καὶ πῶς σοι διαθῶμαι διαθήκην;
\vs{8}Καὶ εἶπαν πρὸς Ἰησοῦν, οἰκεται σου ἐσμέν· καὶ εἶπε πρὸς αὐτοὺς Ἰησοῦς, πόθεν ἐστὲ, καὶ πόθεν παραγεγόνατε;
\vs{9}Καὶ εἶπαν, ἐκ γῆς μακρόθεν σφόδρα ἥκασιν οἱ παῖδές σου ἐν ὀνόματι Κυρίου τοῦ Θεοῦ σου· ἀκηκόαμεν γὰρ τὸ ὄνομα αὐτοῦ, καὶ ὅσα ἐποίησεν ἐν Αἰγύπτῳ,
\vs{10}καὶ ὅσα ἐποίησε τοῖς βασιλεῦσι τῶν Ἀμοῤῥαίων, οἳ ἦσαν πέραν τοῦ Ἰορδάνου, τῷ Σηὼν βασιλεῖ τῶν Ἀμοῤῥαίων, καὶ τῷ Ὢγ βασιλεῖ τῆς Βασὰν, ὃς κατῴκει ἐν Ἀσταρὼθ καὶ ἐν Ἐδραΐν.
\vs{11}Καὶ ἀκούσαντες εἶπαν πρὸς ἡμᾶς οἱ πρεσβύτεροι ἡμῶν καὶ πάντες οἱ κατοικοῦντες τὴν γῆν ἡμῶν, λέγοντες, λάβετε ἑαυτοῖς ἐπισιτισμὸν εἰς τὴν ὁδὸν, καὶ πορεύθητε εἰς συνάντησιν αὐτῶν, καὶ ἐρεῖτε πρὸς αὐτοὺς, οἰκέται σου ἐσμὲν, καὶ νῦν διάθεσθε ἡμῖν τὴν διαθήκην.
\vs{12}Οὗτοι οἱ ἄρτοι, θερμοὺς ἐφωδιάσθημεν αὐτοὺς ἐν τῇ ἡμέρᾳ ᾗ ἐξήλθομεν παραγενέσθαι πρὸς ὑμᾶς· νῦν δὲ ἐξηράνθησαν καὶ γεγόνασι βεβρωμένοι.
\vs{13}Καὶ οὗτοι οἱ ἀσκοὶ τοῦ οἴνου οὓς ἐπλήσαμεν καινοὺς, καὶ οὗτοι ἐῤῥώγασι· καὶ τὰ ἱμάτια ἡμῶν, καὶ τὰ ὑποδήματα ἡμῶν πεπαλαίωται ἀπὸ τῆς πολλῆς ὁδοῦ σφόδρα.

\vs{14}Καὶ ἔλαβον οἱ ἄρχοντες τοῦ ἐπισιτισμοῦ αὐτῶν, καὶ Κύριον οὐκ ἐπηρώτησαν.
\vs{15}Καὶ ἐποίησεν Ἰησοῦς πρὸς αὐτοὺς εἰρήνην, καὶ διέθεντο πρὸς αὐτοὺς διαθήκην τοῦ διασῶσαι αὐτούς· καὶ ὤμοσαν αὐτοῖς οἱ ἄρχοντες τῆς συναγωγῆς.

\vs{16}Καὶ ἐγένετο μετὰ τρεῖς ἡμέρας μετὰ τὸ διαθέσθαι πρὸς αὐτοὺς διαθήκην, ἤκουσαν ὅτι ἐγγύθεν αὐτῶν εἰσι, καὶ ὅτι ἐν αὐτοῖς κατοικοῦσι.
\vs{17}Καὶ ἀπῇραν οἱ υἱοὶ Ἰσραὴλ, καὶ ἦλθον εἰς τὰς πόλεις αὐτῶν· αἱ δὲ πόλεις αὐτῶν Γαβαὼν καὶ Κεφιρὰ καὶ Βηρὼτ, καὶ πόλεις Ἰαρίν.
\vs{18}Καὶ οὐκ ἐμαχέσαντο αὐτοῖς οἱ υἱοὶ Ἰσραὴλ, ὅτι ὤμοσαν αὐτοῖς πάντες οἱ ἄρχοντες Κύριον τὸν Θεὸν Ἰσραήλ· καὶ διεγόγγυσαν πᾶσα ἡ συναγωγὴ ἐπὶ τοῖς ἄρχουσι.

\vs{19}Καὶ εἶπαν οἱ ἄρχοντες πάσῃ τῇ συναγωγῇ, ἡμεῖς ὠμόσαμεν αὐτοῖς Κύριον τὸν Θεὸν Ἰσραὴλ, καὶ νῦν οὐ δυνησόμεθα ἅψασθαι αὐτῶν.
\vs{20}Τοῦτο ποιήσομεν, ζωγρῆσαι αὐτοὺς, καὶ περιποιησόμεθα αὐτούς· καὶ οὐκ ἔσται καθʼ ἡμῶν ὀργὴ διὰ τὸν ὅρκον, ὃν ὠμόσαμεν αὐτοῖς.
\vs{21}Ζήσονται, καὶ ἔσονται ξυλοκόποι καὶ ὑδροφόροι πάσῃ τῇ συναγωγῇ, καθάπερ εἶπαν αὐτοῖς οἱ ἄρχοντες.

\vs{22}Καὶ συνεκάλεσεν αὐτοὺς Ἰησοῦς, καὶ εἶπεν αὐτοῖς, διατί παρελογίσασθέ με, λέγοντες, μακρὰν ἀπὸ σοῦ ἐσμὲν σφόδρα· ὑμεῖς δὲ ἐγχώριοί ἐστε τῶν κατοικούντων ἐν ἡμῖν;
\vs{23}Καὶ νῦν ἐπικατάρατοί ἐστε· οὐ μὴ ἐκλείπῃ ἐξ ὑμῶν δοῦλος, οὐδὲ ξυλοκόπος, οὐδὲ ὑδροφόρος ἐμοὶ καὶ τῷ Θεῷ μου.
\vs{24}Καὶ ἀπεκρίθησαν τῷ Ἰησοῖ, λέγοντες, ἀνηγγέλη ἡμῖν ὅσα συνέταξε Κύριος ὁ Θεός σου Μωυσῇ τῷ παιδὶ αὐτοῦ, δοῦναι ὑμῖν τὴν γῆν ταύτην, καὶ ἐξολοθρεῦσαι ἡμᾶς καὶ πάντας τοὺς κατοικοῦντας ἐπʼ αὐτῆς ἀπὸ προσώπου ὑμῶν· καὶ ἐφοβήθημεν σφόδρα περὶ τῶν ψυχῶν ἡμῶν ἀπὸ προσώπου ὑμῶν, καὶ ἐποιήσαμεν τὸ πρᾶγμα τοῦτο.
\vs{25}Καὶ νῦν ἰδοὺ ἡμεῖς ὑποχείριοι ὑμῖν· ὡς ἀρέσκει ὑμῖν καὶ ὡς δοκεῖ ὑμῖν, ποιήσατε ἡμῖν.

\vs{26}Καὶ ἐποίησαν αὐτοῖς οὕτως· καὶ ἐξείλατο αὐτοὺς Ἰησοῦς ἐν τῇ ἡμέρᾳ ἐκείνῃ ἐκ χειρῶν υἱῶν Ἰσραὴλ, καὶ οὐκ ἀνεῖλον αὐτούς.
\vs{27}Καὶ κατέστησεν αὐτοὺς Ἰησοῦς ἐν τῇ ἡμέρᾳ ἐκείνῃ ξυλοκόπους καὶ ὑδροφόρους πάσῃ τῇ συναγωγῇ, καὶ τῷ θυσιαστηρίῳ τοῦ Θεοῦ· διὰ τοῦτο ἐγένοντο οἱ κατοικοῦντες Γαβαὼν ξυλοκόποι καὶ ὑδροφόροι τοῦ θυσιαστηρίου τοῦ θεοῦ ἕως τῆς σήμερον ἡμέρας, καὶ εἰς τὸν τόπον ὃν ἂν ἐκλέξηται Κύριος.

\ch{10}
Ὡς δὲ ἤκουσεν Ἀδωνιβεζέκ βασιλεὺς Ἱερουσαλὴμ ὅτι ἔλαβεν Ἰησοῦς τὴν Γαὶ, καὶ ἐξωλόθρευσεν αὐτὴν, ὃν τρόπον ἐποίησαν τὴν Ἱεριχὼ καὶ τὸν βασιλέα αὐτῆς, οὕτως ἐποίησαν καὶ τὴν Γαὶ καὶ τὸν βασιλέα αὐτῆς, καὶ ὅτι ηὐτομόλησαν οἱ κατοικοῦντες Γαβαὼν πρὸς Ἰησοῦν καὶ πρὸς Ἰσραὴλ,
\vs{2}καὶ ἐφοβήθησαν ἀπʼ αὐτῶν σφόδρα· ᾔδει γὰρ ὅτι πόλις μεγάλη Γαβαὼν, ὡσεὶ μία τῶν μητροπόλεων, καὶ πάντες οἱ ἄνδρες αὐτῆς ἰσχυροί.
\vs{3}Καὶ ἀπέστειλεν Ἀδωνιβεζὲκ βασιλεὺς Ἰερουσαλὴμ πρὸς Ἐλὰμ βασιλέα Χεβρὼν, καὶ πρὸς Φιδὼν βασιλέα Ἱερειμοὺθ, καὶ πρὸς Ἰεφθα βασιλέα Λαχὶς καὶ πρὸς Δαβεὶν βασιλέα Ὀδολλὰμ, λέγων,
\vs{4}δεῦτε, ἀνάβητε πρός με, καὶ βοηθήσατέ μοι, καὶ ἐκπολεμήσωμεν Γαβαών· ηὐτομόλησαν γὰρ πρὸς Ἰησοῦν καὶ πρὸς τοὺς υἱοὺς Ἰσραήλ.
\vs{5}Καὶ ἀνέβησαν οἱ πέντε βασιλεῖς τῶν Ἰεβουσαίων, βασιλεὺς Ἱερουσαλὴμ, καὶ βασιλεὺς Χεβρὼν, καὶ βασιλεὺς Ἱεριμοὺθ, καὶ βασιλεὺς Λαχὶς, καὶ βασιλεὺς Ὀδολλάμ, αὐτοὶ καὶ πᾶς ὁ λαὸς αὐτῶν. καὶ περιεκάθισαν τὴν Γαβαὼν, καὶ ἐξεπολιόρκουν αὐτήν.

\vs{6}Καὶ ἀπέστειλαν οἱ κατοικοῦντες Γαβαὼν πρὸς Ἰησοῦν εἰς τὴν παρεμβολὴν Ἰσραὴλ εἰς Γάλγαλα, λέγοντες, μὴ ἐκλύσῃς τὰς χεῖράς σου ἀπὸ τῶν παίδων σου· ἀνάβηθι πρὸς ἡμᾶς, τοτάχος, και βοήθησον ἡμῖν, καὶ ἐξελοῦ ἡμᾶς· ὅτι συνηγμένοι εἰσὶν ἐφʼ ἡμᾶς πάντες οἱ βασιλεῖς τῶν Ἀμοῤῥαίων, οἱ κατοικοῦντες τὴν ὀρεινὴν.
\vs{7}Καὶ ἀνέβη Ἰησοῦς ἐκ Γαλγάλων, αὐτὸς καὶ πᾶς ὁ λαὸς ὁ πολεμιστὴς μετʼ αὐτοῦ, πᾶς δυνατὸς ἐν ἰσχύϊ.

\vs{8}Καὶ εἶπε Κύριος πρὸς Ἰησοῦν, μὴ φοβηθῇς αὐτούς, εἰς γὰρ τὰς χεῖράς σου παραδέδωκα αὐτούς· οὐχ ὑπολειφθήσεται ἐξ αὐτῶν οὐδεὶς ἐνώπιον ὑμῶν.

\vs{9}Καὶ ἐπεὶ παρεγένετο Ἰησοῦς ἐπʼ αὐτοὺς ἄφνω, ὅλην τὴν νύκτα εἰσεπορεύθη ἐκ Γαλγάλων.
\vs{10}Καὶ ἐξέστησεν αὐτοὺς Κύριος ἀπὸ προσώπου τῶν υἱῶν Ἰσραήλ· καὶ συνέτριψεν αὐτοὺς Κύριος συντρίψει μεγάλῃ ἐν Γαβαών· καὶ κατεδίωξαν αὐτοὺς ὁδὸν ἀναβάσεως Ὠρωνείν, καὶ κατέκοπτον αὐτοὺς ἕως Ἀζηκὰ καὶ ἕως Μακηδά.
\vs{11}Ἐν δὲ τῷ φεύγειν αὐτοὺς ἀπὸ προσώπου τῶν υἱῶν Ἰσραὴλ ἐπὶ τὴς καταβάσεως Ὡρωνεὶν, καὶ Κύριος ἐπέῤῥιψεν αὐτοῖς λίθους χαλάζης ἐκ τοῦ οὐρανοῦ ἕως Ἀζηκά· καὶ ἐγένοντο πλείους οἱ ἀποθανόντες διὰ τοὺς λίθους τῆς χαλάζης, ἢ οὓς ἀπέκτειναν οἱ υἱοὶ Ἰσραὴλ μαχαίρᾳ ἐν τῷ πολέμῳ.

\vs{12}Τότε ἐλάλησεν Ἰησοῦς πρὸς Κύριον, ᾗ ἡμέρᾳ παρέδωκεν ὁ Θεὸς τὸν Ἀμοῤῥαῖον ὑποχείριον Ἰσραὴλ, ἡνίκα συνέτριψεν αὐτοὺς ἐν Γαβαὼν, καὶ συνετρίβησαν ἀπὸ προσώπου υἱῶν Ἰσραήλ· καὶ εἶπεν Ἰησοῦς, στήτω ὁ ἥλιος κατὰ Γαβαὼν, καὶ ἡ σελήνη κατὰ φάραγγα Αἰλών.
\vs{13}Καὶ ἔστη ὁ ἥλιος καὶ ἡ σελήνη ἐν στάσει, ἕως ἠμύνατο ὁ Θεὸς τοὺς ἐχθροὺς αὐτῶν· καὶ ἔστη ὁ ἥλιος κατὰ μέσον τοῦ οὐρανοῦ· οὐ προεπορεύετο εἰς δυσμὰς εἰς τέλος ἡμέρας μιᾶς.
\vs{14}Καὶ οὐκ ἐγένετο ἡμέρα τοιαύτη οὐδὲ τὸ πρότερον οὐδὲ τὸ ἔσχατον, ὥστε ἐπακοῦσαι Θεὸν ἀνθρώπου, ὅτι Κύριος συνεξεπολέμησε τῷ Ἰσραήλ.

\vs{16}Καὶ ἔφυγον οἱ πέντε βασιλεῖς οὗτοι, καὶ κατεκρύβησαν εἰς τὸ σπήλαιον τὸ ἐν Μακηδά.
\vs{17}Καὶ ἀπηγγέλη τῷ Ἰησοῦ, λέγοντες, εὕρηνται οἱ πέντε βασιλεῖς κεκρυμμένοι ἐν τῷ σπηλαίῳ τῷ ἐν Μακηδά.
\vs{18}Καὶ εἶπεν Ἰησοῦς, κυλίσατε λίθους ἐπὶ τὸ στόμα τοῦ σπηλαίου, καὶ καταστήσατε ἄνδρας φυλάσσειν ἐπʼ αὐτούς.
\vs{19}Ὑμεῖς δὲ μὴ ἑστήκατε, καταδιώκοντες ὀπίσω τῶν ἐχθρῶν ὑμῶν, καὶ καταλάβετε τὴν οὐραγίαν αὐτῶν, καὶ μὴ ἀφῆτε εἰσελθεῖν εἰς τὰς πόλεις αὐτῶν· παρέδωκε γὰρ αὐτοὺς Κύριος ὁ Θεὸς ἡμῶν εἰς τὰς χεῖρας ἡμῶν.
\vs{20}Καὶ ἐγένετο ὡς κατέπαυσεν Ἰησοῦς καὶ πᾶς υἱὸς Ἰσραὴλ κόπτοντες αὐτοὺς κοπὴν μεγάλην σφόδρα ἕως εἰς τέλος, καὶ οἱ διασωζόμενοι διεσώθησαν εἰς τὰς πόλεις τὰς ὀχυράς.

\vs{21}Καὶ ἀπεστράφη πᾶς ὁ λαὸς πρὸς Ἰησοῦν εἰς Μακηδὰ ὑγιεῖς· καὶ οὐκ ἔγρυξεν οὐδεὶς τῶν υἱῶν Ἰηραὴλ τῇ γλώσσῃ αὐτοῦ.

\vs{22}Καὶ εἶπεν Ἰησοῦς, ἀνοίξατε τὸ σπήλαιον, καὶ ἐξαγάγετε τοὺς πέντε βασιλεῖς τούτους ἐκ τοῦ σπηλαίου.
\vs{23}Καὶ ἐξηγάγοσαν τοὺς πέντε βασιλεῖς ἐκ τοῦ σπηλαίου, τὸν βασιλέα Ἱερουσαλὴμ, καὶ τὸν βασιλέα Χεβρὼν, καὶ τὸν βασιλέα Ἱεριμοὺθ, καὶ τὸν βασιλέα Λαχὶς, καὶ τὸν βασιλέα Ὀδολλάμ.
\vs{24}Καὶ ἐπεὶ ἐξήγαγον αὐτοὺς πρὸς Ἰησοῦν, καὶ συνεκάλεσεν Ἰησοῦς πάντα Ἰσραὴλ, καὶ τοὺς ἐναρχομένους τοῦ πολέμου τοὺς συμπορευομένους αὐτῷ, λέγων αὐτοῖς, προπορεύεσθε καὶ ἐπίθετε τοὺς πόδας ὑμῶν ἐπὶ τοὺς τραχήλους αὐτῶν· καὶ προσελθόντες ἐπέθηκαν τοὺς πόδας αὐτῶν ἐπὶ τοὺς τραχήλους αὐτῶν.
\vs{25}Καὶ εἶπεν Ἰησοῦς πρὸς αὐτοὺς, μὴ φοβηθῆτε αὐτοὺς, μηδὲ δειλιάσητε, ἀνδρίζεσθε καὶ ἰσχύετε, ὅτι οὕτω ποιήσει Κύριος πᾶσι τοῖς ἐχθροῖς ὑμῶν, οὓς ὑμεῖς καταπολεμεῖτε αὐτούς.
\vs{26}Καὶ ἀπέκτεινεν αὐτοὺς Ἰησοῦς, καὶ ἐκρέμασεν αὐτοὺς ἐπὶ πέντε ξύλων· καὶ ἦσαν κρεμάμενοι ἐπὶ τῶν ξύλων ἕως ἑσπέρας.
\vs{27}Καὶ ἐγενήθη πρὸς ἡλίου δυσμὰς, ἐνετείλατο Ἰησοῦς, καὶ καθεῖλον αὐτοὺς ἀπὸ τῶν ξύλων, καὶ ἔῤῥιψαν αὐτοὺς εἰς τὸ σπήλαιον, εἰς ὃ κατεφύγοσαν ἐκεῖ, καὶ ἐπεκύλισαν λίθους ἐπὶ τὸ σπήλαιον ἕως τῆς σήμερον ἡμέρας.

\vs{28}Καὶ τὴν Μακηδὰ ἐλάβοσαν ἐν τῇ ἡμέρᾳ ἐκείνῃ, καὶ ἐφόνευσαν αὐτὴν ἐν στόματι ξίφους, καὶ ἐξωλόθρευσαν πᾶν ἐμπνέον ὃ ἦν ἐν αὐτῇ· καὶ οὐ κατελείφθη οὐδεὶς ἐν αὐτῇ διασεσωσμένος καὶ διαπεφευγώς· καὶ ἐποίησαν τῷ βασιλεῖ Μακηδὰ, ὃν τρόπον ἐποίησαν τῷ βασιλεῖ Ἱεριχώ.

\vs{29}Καὶ ἀπῆλθεν Ἰησοῦς καὶ πᾶς Ἰσραὴλ μετʼ αὐτοῦ ἐκ Μακηδὰ εἰς Λεβνὰ, καὶ ἐπολιόρκει Λεβνά.
\vs{30}Καὶ παρέδωκεν αὐτὴν Κύριος εἰς χεῖρας Ἰσραήλ· καὶ ἔλαβον αὐτὴν, καὶ τὸν βασιλέα αὐτῆς, καὶ ἐφόνευσαν αὐτὴν ἐν στόματι ξίφους, καὶ πᾶν ἐμπνέον ἐν αὐτῇ· καὶ οὐ κατελείφθη ἐν αὐτῇ διασεσωσμένος καὶ διαπεφευγώς· καὶ ἐποίησαν τῷ βασιλεῖ αὐτῆς, ὃν τρόπον ἐποίησαν τῷ βασιλεῖ Ἱεριχώ.

\vs{31}Καὶ ἀπῆλθεν Ἰησοῦς καὶ πᾶς Ἰσραὴλ μετʼ αὐτοῦ ἐκ Λεβνὰ εἰς Λαχὶς, καὶ περιεκάθισεν αὐτὴν, καὶ ἐπολιόρκει αὐτήν.
\vs{32}Καὶ παρέδωκε Κύριος τὴν Λαχὶς εἰς τὰς χεῖρας Ἰσραήλ. καὶ ἔλαβεν αὐτὴν ἐν τῇ ἡμέρᾳ τῇ δευτέρᾳ, καὶ ἐφόνευσαν αὐτὴν ἐν στόματι ξίφους, καὶ ἐξωλόθρευσαν αὐτὴν, ὃν τρόπον ἐποίησαν τὴν Λεβνά.
\vs{33}Τότε ἀνέβη Ἐλὰμ βασιλεὺς Γαζὲρ βοηθήσων τῇ Λαχείς· καὶ ἐπάταξεν αὐτὸν Ἰησοῦς ἐν στόματι ξίφους, καὶ τὸν λαὸν αὐτοῦ, ἕως τοῦ μὴ καταλειφθῆναι αὐτῶν σεσωσμένον καὶ διαπεφευγότα.

\vs{34}Καὶ ἀπῆλθεν Ἰησοῦς καὶ πᾶς Ἰσραὴλ μετʼ αὐτοῦ ἐκ Λαχὶς εἰς Ὀδολλὰμ, καὶ περιεκάθισεν αὐτὴν καὶ ἐξεπολιόρκησεν αὐτήν.
\vs{35}Καὶ παρέδωκεν αὐτὴν Κύριος ἐν χειρὶ Ἰσραήλ· καὶ ἔλαβεν αὐτὴν ἐν τῇ ἡμέρᾳ ἐκείνῃ, καὶ ἐφόνευσεν αὐτὴν ἐν στόματι ξίφους, καὶ πᾶν ἐμπνέον ἐν ἀτῇ ἐφόνευσαν, ὃν τρόπον ἐποίησαν τῇ Λαχίς.

\vs{36}Καὶ ἀπῆλθεν Ἰησοῦς καὶ πᾶς Ἰσραὴλ μετʼ αὐτοῦ εἰς Χεβρὼν, καὶ περιεκάθισεν αὐτήν.
\vs{37}Καὶ ἐπάταξεν αὐτὴν ἐν στόματι ξίφους, καὶ πᾶν τὸ ἐμπνέον ὅσα ἦν ἐν αὐτῇ· οὐκ ἦν διασεσωσμένος· ὃν τρόπον ἐποίησαν τὴν Ὀδολλὰμ, ἐξωλόθρευσαν αὐτὴν, καὶ ὅσα ἦν ἐν αὐτῇ.

\vs{38}Καὶ ἀπέστρεψεν Ἰησοῦς καὶ πᾶς Ἰσραὴλ εἰς Δαβίρ· καὶ περικαθίσαντες αὐτὴν,
\vs{39}ἔλαβον αὐτὴν, καὶ τὸν βασιλέα αὐτῆς, καὶ τὰς κώμας αὐτῆς· καὶ ἐπάταξεν αὐτὴν ἐν στόματι ξίφους, καὶ ἐξωλόθρευσαν αὐτὴν, καὶ πᾶν ἐμπνέον ἐν αὐτῇ· καὶ οὐ κατέλιπον αὐτῇ οὐδένα διασεσωσμένον· ὃν τρόπον ἐποίησαν τῇ Χεβρὼν καὶ τῷ βασιλεῖ αὐτῆς, οὕτως ἐποίησαν τῇ Δαβὶρ καὶ τῷ βασιλεῖ αὐτῆς.

\vs{40}Καὶ ἐπάταξεν Ἰησοῦς πᾶσαν τὴν γῆν τῆς ὀρεινῆς, καὶ τὴν Ναγὲβ, καὶ τὴν πεδινὴν, καὶ τὴν Ἀσηδὼθ, καὶ τοὺς βασιλεῖς αὐτῆς· οὐ κατέλιπον αὐτῶν σεσωσμένον· καὶ πᾶν ἐμπνέον ζωῆς ἐξωλόθρευσεν, ὃν τρόπον ἐνετείλατο Κύριος ὁ Θεὸς Ἰσραὴλ,
\vs{41}ἀπὸ Κάδης Βαρνῆ ἕως Γάζης πᾶσαν τὴν Γοσὸμ ἕως τῆς Γαβαών.
\vs{42}Καὶ πάντας τοὺς βασιλεῖς αὐτῶν, καὶ τὴν γῆν αὐτῶν ἐπάταξεν Ἰησοῦς εἰσάπαξ· ὅτι Κύριος ὁ Θεὸς Ἰσραὴλ συνεπολέμει τῷ Ἰσραήλ.

\ch{11}
Ὡς δὲ ἤκουσεν Ἰαβὶς βασιλεὺς Ἀσὼρ, ἀπέστειλε πρὸς Ἰωβὰβ βασιλέα Μαρῶν, καὶ πρὸς βασιλέα Συμοὼν, καὶ πρὸς βασιλέα Ἀζὶφ,
\vs{2}καὶ πρὸς βασιλεῖς τοὺς κατὰ Σιδῶνα τὴν μεγάλην, εἰς τὴν ὀρεινὴν καὶ εἰς Ἄραβα ἀπέναντι Κενερὼθ, καὶ εἰς τὸ πεδίον, καὶ εἰς Φεναεδδὼρ,
\vs{3}καὶ εἰς τοὺς παραλίους Χαναναίους ἀπὸ ἀνατολῶν, καὶ εἰς τοὺς παραλίους Ἀμοῤῥαίους, καὶ τοὺς Χετταίους, καὶ Φερεζαίους, καὶ Ἰεβουσαίους τοὺς ἐν τῷ ὄρει, καὶ τοὺς Εὐαίους, καὶ τοὺς ὑπὸ τὴν Ἀερμὼν εἰς γῆν Μασσύμα.
\vs{4}Καὶ ἐξῆλθον αὐτοὶ καὶ οἱ βασιλεῖς αὐτῶν μετʼ αὐτῶν, ὥσπερ ἡ ἄμμος τῆς θαλάσσης τῷ πλήθει, καὶ ἵπποι καὶ ἅρματα πολλὰ σφόδρα.
\vs{5}Καὶ συνῆλθον πάντες οἱ βασιλεῖς αὐτοὶ καὶ παρεγένοντο ἐπὶ τὸ αὐτὸ, καὶ παρενέβαλον ἐπὶ τοῦ ὕδατος Μαρὼν πολεμῆσαι τὸν Ἰσραήλ.

\vs{6}Καὶ εἶπε Κύριος πρὸς Ἰησοῦν, μὴ φοβηθῇς ἀπὸ προσώπου αὐτῶν, ὅτι αὔριον ταύτην τὴν ὥραν ἐγὼ παραδίδωμι τετροπωμένους αὐτοὺς ἐναντίον τοῦ Ἰσραήλ· τοὺς ἵππους αὐτῶν νευροκοπήσεις, καὶ τὰ ἅρματα αὐτῶν κατακαύσεις ἐν πυρί.
\vs{7}Καὶ ἦλθεν Ἰησοῦς καὶ πᾶς ὁ λαὸς ὁ πολεμιστὴς ἐπʼ αὐτοὺς ἐπὶ τὸ ὕδωρ Μαρὼν ἐξάπινα· καὶ ἐπέπεσαν ἐπʼ αὐτοὺς ἐν τῇ ὀρεινῇ.
\vs{8}Καὶ παρέδωκεν αὐτοὺς Κύριος ὑποχειρίους Ἰσραήλ· καὶ κόπτοντες αὐτοὺς κατεδίωκον ἕως Σιδῶνος τῆς μεγάλης, καὶ ἕως Μασερὼν, καὶ ἕως τῶν πεδίων Μασσὼχ κατʼ ἀνατολάς· καὶ κατέκοψαν αὐτοὺς ἕως τοῦ μὴ καταλειφθῆναι αὐτῶν διασεσωσμένον.
\vs{9}Καὶ ἐποίησεν αὐτοῖς Ἰησοῦς, ὃν τρόπον ἐνετείλατο αὐτῷ Κύριος· τοὺς ἵππους αὐτῶν ἐνευροκόπησε, καὶ τὰ ἅρματα αὐτῶν ἐνέπρησε πυρί.

\vs{10}Καὶ ἐπεστράφη Ἰησοῦς ἐν τῷ καιρῷ ἐκείνῳ, καὶ κατελάβετο Ἀσὼρ, καὶ τὸν βασιλέα αὐτῆς· ἦν δὲ Ἀσὼρ τοπρότερον ἄρχουσα πασῶν τῶν βασιλειῶν τούτων.
\vs{11}Καὶ ἀπέκτειναν πᾶν ἐμπνέον ἐν αὐτῇ ἐν ξίφει, καὶ ἐξωλόθρευσαν πάντας, καὶ οὐ κατελείφθη ἐν αὐτῇ ἐμπνέον· καὶ τὴν Ἀσὼρ ἐνέπρησαν ἐν πυρί.
\vs{12}Καὶ πάσας τὰς πόλεις τῶν βασιλειῶν, καὶ τοὺς βασιλεῖς αὐτῶν ἔλαβεν Ἰησοῦς, καὶ ἀνεῖλεν αὐτοὺς ἐν στόματι ξίφους· καὶ ἐξωλόθρευσαν αὐτοὺς, ὃν τρόπον συνέταξε Μωυσῆς ὁ παῖς Κυρίου.
\vs{13}Ἀλλὰ πάσας τὰς πόλεις τὰς κεχωματισμένας οὐκ ἐνέπρησεν Ἰσραήλ· πλὴν Ἀσὼρ μόνην ἐνέπρησεν Ἰσραὴλ,
\vs{14}καὶ πάντα τὰ σκῦλα αὐτῆς ἐπρονόμευσαν ἑαυτοῖς οἱ υἱοὶ Ἰσραήλ· αὐτοὺς δὲ πάντας ἐξωλόθρευσαν ἐν στόματι ξίφους, ἕως ἀπώλεσεν αὐτοὺς· οὐ κατέλιπον ἐξ αὐτῶν οὐδὲν ἐμπνέον.
\vs{15}Ὃν τρόπον συνέταξε Κύριος τῷ Μωυσῇ τῷ παιδὶ αὐτοῦ, καὶ Μωυσῆς ὡσαύτως ἐνετείλατο τῷ Ἰησοῖ· καὶ οὕτως ἐποίησεν Ἰησοῦς, οὐ παρέβη οὐδὲν ἀπὸ πάντων ὧν συνέταξεν αὐτῷ Μωυσῆς.

\vs{16}Καὶ ἔλαβεν Ἰησοῦς πᾶσαν τὴν γῆν τὴν ὀρεινὴν, καὶ πᾶσαν τὴν γῆν Ναγὲβ, καὶ πᾶσαν τὴν γῆν Γοσὸμ, καὶ τὴν πεδινὴν, καὶ τὴν πρὸς δυσμαῖς, καὶ τὸ ὄρος Ἰσραὴλ, καὶ τὰ ταπεινὰ
\vs{17}τὰ πρὸς τῷ ὄρει ἀπὸ ὄρους Χελχὰ, καὶ ὃ προσαναβαίνει εἰς Σηεὶρ, καὶ ἕως Βαλαγὰδ, καὶ τὰ πεδία τοῦ Λιβάνου ὑπὸ τὸ ὄρος τὸ Ἀερμών· καὶ πάντας τοὺς βασιλεῖς αὐτῶν ἔλαβε, καὶ ἀνεῖλε, καὶ ἀπέκτεινε.
\vs{18}Καὶ ἡμέρας πλείους ἐποίησεν Ἰησοῦς πρὸς τοὺς βασιλεῖς τούτους τὸν πόλεμον.

\vs{19}Καὶ οὐκ ἦν πόλις, ἣν οὐκ ἔλαβεν Ἰσραήλ· πάντα ἐλάβοσαν ἐν πολέμῳ.
\vs{20}Ὅτι διὰ Κυρίου ἐγένετο κατισχύσαι αὐτῶν τὴν καρδίαν συναντᾷν εἰς πόλεμον πρὸς Ἰσραὴλ, ἵνα ἐξολοθρευθῶσιν, ὅπως μὴ δοθῇ αὐτοῖς ἔλεος, ἀλλʼ ἵνα ἐξολοθρευθῶσιν, ὃν τρόπον εἶπε Κύριος πρὸς Μωυσῆν.

\vs{21}Καὶ ἦλθεν Ἰησοῦς ἐν τῷ καιρῷ ἐκείνῳ, καὶ ἐξωλόθρευσε τοὺς Ἐνακὶμ ἐκ τῆς ὀρεινῆς, ἐκ Χεβρὼν, καὶ ἐκ Δαβὶρ, καὶ ἐξ Ἀναβὼθ, καὶ ἐκ παντὸς Ἰούδα σὺν ταῖς πόλεσιν αὐτῶν· καὶ ἐξωλόθρευσεν αὐτοὺς Ἰησοῦς.
\vs{22}Οὐ κατελείφθη τῶν Ἐνακὶμ ἀπὸ τῶν υἱῶν Ἰσραὴλ, ἀλλὰ πλὴν ἐν Γάζῃ, καὶ ἐν Γὲθ, καὶ ἐν Ἀσελδὼ κατελείφθη.

\vs{23}Καὶ ἔλαβεν Ἰησοῦς πᾶσαν τὴν γῆν, καθότι ἐνετείλατο Κύριος τῷ Μωυσῇ· καὶ ἔδωκεν αὐτοὺς Ἰησοῦς ἐν κληρονομίᾳ Ἰσραὴλ ἐν μερισμῷ κατὰ φυλὰς αὐτῶν· καὶ ἡ γῆ κατέπαυσε πολεμουμένη.

\ch{12}
Καὶ οὗτοι οἱ βασιλεῖς τῆς γῆς, οὓς ἀνεῖλον οἱ υἱοὶ Ἰσραὴλ, καὶ κατεκληρονόμησαν τὴν γῆν αὐτῶν πέραν τοῦ Ἰορδάνου ἀφʼ ἡλίου ἀνατολῶν ἀπὸ φάραγγος Ἀρνῶν ἕως τοῦ ὄρους Ἀερμὼν, καὶ πᾶσαν τὴν γῆν Ἄραβα ἀπʼ ἀνατολῶν.
\vs{2}Σηὼν τὸν βασιλέα τῶν Ἀμοῤῥαίων, ὃς κατῴκει ἐν Ἐσεβὼν, κυριεύων ἀπὸ Ἀρνῶν, ἥ ἐστιν ἐν τῇ φάραγγι κατὰ μέρος τῆς φάραγγος, καὶ τὸ ἥμισυ τῆς Γαλαὰδ ἕως Ἰαβὸκ, ὅρια υἱῶν Ἀμμών.
\vs{3}Καὶ Ἄραβ ἕως τῆς θαλάσσης Χενερὲθ κατʼ ἀνατολὰς, καὶ ἕως τῆς θαλάσσης Ἄραβα, θάλασσαν τῶν ἁλῶν ἀπὸ ἀνατολῶν ὁδὸν τὴν κατὰ Ἀσειμὼθ, ἀπὸ Θαιμὰν τὴν ὑπὸ Ἀσηδὼθ Φασγά.
\vs{4}Καὶ Ὢγ βασιλεὺς Βασὰν ὑπελείφθη ἐκ τῶν γιγάντων, ὁ κατοικῶν ἐν Ἀσταρὼθ καὶ ἐν Ἐδραῒν,
\vs{5}ἄρχων ἀπὸ ὄρους Ἀερμὼν καὶ ἀπὸ Σεκχαὶ, καὶ πᾶσαν τὴν Βασὰν ἕως ὁρίων Γεργεσὶ, καὶ τὴν Μαχὶ, καὶ τὸ ἥμισυ Γαλαδ ὁρίων Σηὼν βασιλέως Ἐσεβών.
\vs{6}Μωυσῆς ὁ παῖς Κυρίου καὶ οἱ υἱοὶ Ἰσραὴλ ἐπάταξαν αὐτούς· καὶ ἔδωκεν αὐτὴν Μωυσῆς ἐν κληρονομίᾳ Ῥουβὴν, καὶ Γὰδ, καὶ τῷ ἡμίσει φυλῆς Μανασσῆ.

\vs{7}Καὶ οὗτοι οἱ βασιλεῖς τῶν Ἀμοῤῥαίων, οὓς ἀνεῖλεν Ἰησοῦς καὶ υἱοὶ Ἰσραὴλ ἐν τῷ πέραν τοῦ Ἰορδάνου παρὰ θάλασσαν Βαλαγὰδ ἐν τῷ πεδίῳ τοῦ Λιβάνου, καὶ ἕως ὄρους τοῦ Χελχὰ ἀναβαινόντων εἰς Σηείρ· καὶ ἔδωκεν αὐτὴν Ἰησοῦς ταῖς φυλαῖς Ἰσραὴλ κληρονομεῖν κατὰ κλῆρον αὐτῶν,
\vs{8}ἐν τῷ ὄρει, καὶ ἐν τῷ πεδίῳ, καὶ ἐν Ἄραβ, καὶ ἐν Ἀσηδὼθ, καὶ ἐν τῇ ἐρήμῳ, καὶ Ναγὲβ· τὸν Χετταῖον, καὶ τὸν Ἀμοῤῥαῖον, καὶ τὸν Χαναναῖον, καὶ τὸν Φερεζαῖον, καὶ τὸν Εὐαῖον, καὶ τὸν Ἰεβουσαῖον.

\vs{9}Τὸν βασιλέα Ἱεριχὼ, καὶ τὸν βασιλέα τῆς Γαὶ, ἥ ἐστι πλησίον Βαιθὴλ,
\vs{10}βασιλέα Ἱερουσαλὴμ, βασιλέα Χεβρὼν,
\vs{11}βασιλέα Ἱεριμοὺθ, βασιλέα Λαχὶς,
\vs{12}βασιλέα Αἰλὰμ, βασιλέα Γαζὲρ,
\vs{13}βασιλέα Δαβὶρ, βασιλέα Γαδὲρ,
\vs{14}βασιλέα Ἑρμὰθ, βασιλέα Ἀδὲρ,
\vs{15}βασιλέα Λεβνὰ, βασιλέα Ὀδολλὰμ,
\vs{16}βασιλέα Ἠλὰθ,
\vs{17}βασιλέα Ταφοὺτ, βασιλέα Ὀφὲρ,
\vs{18}βασιλέα Ὀφὲκ τῆς Ἀρὼκ,
\vs{19}βασιλέα Ἀσὼμ,
\vs{20}βασιλέα Συμοὼν, βασιλέα Μαμβρὼθ, βασιλέα Ἀζὶφ,
\vs{21}βασιλέα Κάδης, βασιλέα Ζαχὰκ,
\vs{22}βασιλέα Μαρεδὼθ, βασιλέα Ἰεκὸμ τοῦ Χερμὲλ,
\vs{23}βασιλέα Ὀδολλὰμ τοῦ Φεννεαδὼρ, βασιλέα Γεῒ τῆς Γαλιλαίας,
\vs{24}βασιλέα Θερσά· πάντες οὗτοι βασιλεῖς εἰκοσιεννέα.

\ch{13}
Καὶ Ἰησοῦς πρεσβύτερος προβεβηκὼς τῶν ἡμερῶν· καὶ εἶπε Κύριος πρὸς Ἰησοῦν, σὺ προβέβηκας τῶν ἡμερῶν, καὶ ἡ γῆ ὑπολέλειπται πολλὴ εἰς κληρονομίαν.
\vs{2}Καὶ αὕτη ἡ γῆ καταλελειμμένη· ὅρια Φυλιστιεὶμ, ὁ Γεσιρὶ, καὶ ὁ Χαναναῖος,
\vs{3}ἀπὸ τῆς ἀοικήτου τῆς κατὰ πρόσωπον Αἰγύπτου ἕως τῶν ὁρίων Ἀκκαρὼν ἐξ εὐωνύμων τῶν Χαναναίων προσλογίζεται ταῖς πέντε σατραπείαις τῶν Φυλιστιείμ, τῷ Γαζαίῳ, καὶ τῷ Ἀζωτίῳ, καὶ τῷ Ἀσκαλωνίτῃ, καὶ τῷ Γετθαίῳ, καὶ τῷ Ἀκκαρωνίτῃ, καὶ τῷ Εὐαίῳ,
\vs{4}ἐκ Θαιμὰν καὶ πάσῃ γῇ Χαναὰν ἐναντίον Γάζης, καὶ οἱ Σιδώνιοι ἕως Ἀφὲκ, ἕως τῶν ὁρίων τῶν Ἀμοῤῥαίων,
\vs{5}καὶ πᾶσαν τὴν γῆν Γαλιὰθ Φυλιστιεὶμ, καὶ πάντα τὸν Λίβανον ἀπὸ ἀνατολῶν ἡλίου ἀπὸ Γαλγὰλ ὑπὸ τὸ ὄρος τὸ Ἀερμὼν ἕως τῆς εἰσόδου Ἐμὰθ,
\vs{6}πᾶς ὁ κατοικῶν τὴν ὀρεινὴν ἀπὸ τοῦ Λιβάνου ἕως τῆς Μασερὲθ Μεμφωμαὶμ. Πάντας τοὺς Σιδωνίους, ἐγὼ αὐτοὺς ἐξολοθρεύσω ἀπὸ προσώπου Ἰσραήλ· ἀλλὰ διάδος αὐτὴν ἐν κλήρῳ τῷ Ἰσραὴλ, ὃν τρόπον σοὶ ἐνετειλάμην.

\vs{7}Καὶ νῦν μέρισον τὴν γῆν ταύτην ἐν κληρονομίᾳ ταῖς ἐννέα φυλαῖς, καὶ τῷ ἡμίσει φυλῆς Μανασσή. Ἀπὸ τοῦ Ἰορδάνου ἕως τῆς θαλάσσης τῆς μεγάλης κατὰ δυσμὰς ἡλίου δώσεις αὐτήν· ἡ θάλασσα ἡ μεγάλη ὁριεῖ.
\vs{8}ταῖς δυσὶ φυλαῖς, καὶ τῷ ἡμίσει φυλῆς Μανασσή, τῷ Ῥουβὴν, καὶ τῷ Γὰδ ἔδωκε Μωυσῆς ἐν τῷ πέραν τοῦ Ἰορδάνου· κατʼ ἀνατολὰς ἡλίου δέδωκεν αὐτῷ Μωυσῆς ὁ παῖς Κυρίου,
\vs{9}ἀπὸ Ἀροὴρ, ἥ ἐστιν ἐπὶ τοῦ χείλους χειμάῤῥου Ἀρνών, καὶ τὴν πόλιν τὴν ἐν μέσῳ τῆς φάραγγος, καὶ πᾶσαν τὴν Μισὼρ ἀπὸ Μαιδαβάν·
\vs{10}Πάσας τὰς πόλεις Σηὼν βασιλέως Ἀμοῤῥαίων, ὃς ἐβασίλευσεν ἐν Ἐσεβὼν ἕως τῶν ὁρίων υἱῶν Ἀμμῶν·
\vs{11}Καὶ τὴν Γαλααδίτιδα, καὶ τὰ ὅρια Γεσιρὶ, καὶ τοὺς Μαχατὶ, πᾶν ὄρος Ἀερμὼν, καὶ πᾶσαν τὴν Βασανίτιν ἕως Ἀχά·
\vs{12}Πᾶσαν τὴν βασιλείαν Ὢγ ἐν τῇ Βασανίτιδι, ὃς ἐβασίλευσεν ἐν Ἀσταρὼθ καὶ ἐν Ἐδραΐν· οὗτος κατελείφθη ἀπὸ τῶν γιγάντων, καὶ ἐπάταξεν αὐτὸν Μωυσῆς, καὶ ἐξωλόθρευσε.
\vs{13}Καὶ οὐκ ἐξωλόθρευσαν οἱ υἱοὶ Ἰσραὴλ τὸν Γεσιρὶ, καὶ τὸν Μαχατὶ, καὶ τὸν Χαναναῖον· καὶ κατῷκει βασιλεὺς Γεσιρὶ καὶ ὁ Μαχατὶ ἐν τοῖς υἱοῖς Ἰσραὴλ ἕως τῆς σήμερον ἡμέρας.

\vs{14}Πλὴν τῆς φυλῆς Λευὶ οὐκ ἐδόθη κληρονομία· Κύριος ὁ Θεὸς Ἰσραὴλ, οὗτος κληρονομία αὐτῶν, καθὰ εἶπεν αὐτοῖς Κύριος· καὶ οὗτος ὁ καταμερισμὸς, ὃν κατεμέρισε Μωυσῆς τοῖς υἱοῖς Ἰσραὴλ ἐν Ἀραβὼθ Μωὰβ ἐν τῷ πέραν τοῦ Ἰορδάνου κατὰ Ἱεριχώ.

\vs{15}Καὶ ἔδωκε Μωυσῆς τῇ φυλῇ Ῥουβὴν κατὰ δήμους αὐτῶν.
\vs{16}Καὶ ἐγενήθη αὐτῶν τὰ ὅρια ἀπὸ Ἀροὴρ, ἥ ἐστι κατὰ πρόσωπον φάραγγος Ἀρνῶν, καὶ ἡ πόλις ἡ ἐν τῇ φάραγγι Ἀρνῶν· καὶ πᾶσαν τὴν Μισὼρ,
\vs{17}ἕως Ἐσεβὼν, καὶ πάσας τὰς πόλεις τὰς οὔσας ἐν τῇ Μισὼρ, καὶ Δαιβὼν, καὶ Βαιμὼν Βαὰλ, καὶ οἴκου Μεελβὼθ,
\vs{18}καὶ Βασὰν, καὶ Βακεδμὼθ, καὶ Μαιφαὰδ, καὶ, Καριθαὶμ,
\vs{19}καὶ Σεβαμὰ, καὶ Σεραδὰ, καὶ Σιὼν ἐν τῷ ὄρει Ἐνὰβ,
\vs{20}καὶ Βαιθφογὼρ, καὶ Ἀσηδὼθ Φασγὰ, καὶ Βαιτθασεινὼθ,
\vs{21}καὶ πάσας τὰς πόλεις τοῦ Μισὼρ, καὶ πᾶσαν τὴν βασιλείαν τοῦ Σηὼν βασιλέως τῶν Ἀμοῤῥαίων, ὃν ἐπάταξε Μωυσῆς αὐτὸν καὶ τοὺς ἡγουμένους Μαδιὰμ, καὶ τὸν Εὐὶ, καὶ τὸν Ῥοβὸκ, καὶ τὸν Σοὺρ, καὶ τὸν Οὒρ, καὶ τὸν Ῥοβὲ ἄρχοντα ἔναρα Σιὼν, καὶ τοὺς κατοικοῦντας Σιών.

\vs{22}Καὶ τὸν Βαλαὰμ τὸν τοῦ Βαιὼρ τὸν μάντιν ἀπέκτειναν ἐν τῇ ῥοπῇ.

\vs{23}Ἐγένετο δὲ τὰ ὅρια Ῥουβὴν, Ἰορδάνης ὅριον· αὑτη ἡ κληρονομία υἱῶν Ῥουβὴν κατὰ δήμους αὐτῶν, αἱ πόλεις αὐτῶν καὶ αἱ ἐπαύλεις αὐτῶν.

\vs{24}Ἔδωκε δὲ Μωυσῆς τοῖς υἱοῖς Γὰδ κατὰ δήμους αὐτῶν.
\vs{25}Καὶ ἐγένετο τὰ ὅρια αὐτῶν Ἰαζήρ· πᾶσαι πόλεις Γαλαὰδ, καὶ τὸ ἥμισυ γῆς υἱῶν Ἀμμῶν ἕως Ἄραβα, ἥ ἐστι κατὰ προσωπον Ἀράδ.
\vs{26}Καὶ ἀπὸ Ἐσεβὼν ἕως Ἀραβὼθ κατὰ τὴν Μασσηφὰ, καὶ Βοτανὶμ, καὶ Μαὰν ἕως τῶν ὁρίων Δαιβὼν,
\vs{27}καὶ Ἐναδὼμ καὶ Ὀθαργαῒ καὶ Βαινθαναβρὰ καὶ Σοκχωθὰ καὶ Σαφὰν καὶ τὴν λοιπὴν βασιλείαν Σηὼν βασιλέως Ἐσεβών· καὶ ὁ Ἰορδάνης ὁριεῖ ἕως μέρους τῆς θαλάσσης Χενερὲθ πέραν τοῦ Ἰορδάνου ἀπʼ ἀνατολῶν.
\vs{28}Αὕτη ἡ κληρονομία υἱῶν Γὰδ κατὰ δήμους αὐτῶν καὶ κατὰ πόλεις αὐτῶν· κατὰ δήμους αὐτῶν αὐχένα ἐπιστρέψουσιν ἐναντίον τῶν ἐχθρῶν αὐτῶν, ὅτι ἐγενήθη κατὰ δήμους αὐτῶν αἱ πόλεις αὐτῶν, καὶ αἱ ἐπαύλεις αὐτῶν.

\vs{29}Καὶ ἔδωκε Μωυσῆς τῷ ἡμίσει φυλῆς Μανασσῆ κατὰ δήμους αὐτῶν.
\vs{30}Καὶ ἐγένετο τὰ ὅρια αὐτῶν ἀπὸ Μαὰν, καὶ πᾶσα βασιλεία Βασὰν, καὶ πᾶσα βασιλεία Ὢγ βασιλέως τῆς Βασὰν, καὶ πάσας τὰς κώμας Ἰαῒρ, αἵ εἰσιν ἐν τῇ Βασανίτιδι, ἑξήκοντα πόλεις.
\vs{31}Καὶ τὸ ἥμισυ τῆς Γαλαάδ· καὶ ἐν Ἀσταρὼθ, καὶ ἐν Ἐδραῒν πόλεις βασιλείας Ὢγ ἐν τῇ Βασανίτιδι, τοῖς υἱοῖς Μαχὶρ υἱοῖς Μανασσῆ, καὶ τοῖς ἡμίσεσιν υἱοις Μαχὶρ υἱοῖς Μανασσῆ, κατὰ δήμους αὐτῶν.
\vs{32}Οὗτοι οὓς κατεκληρονόμησε Μωυσῆς πέραν τοῦ Ἰορδάνου ἐν Ἀραβὼθ Μωὰβ ἐν τῷ πέραν τοῦ Ἰορδάνου τοῦ κατὰ Ἱεριχὼ ἀπʼ ἀνατολῶν.

\ch{14}
Καὶ οὗτοι οἱ κατακληρονομήσαντες υἱῶν Ἰσραὴλ ἐν τῇ γῇ Χαναὰν, οἷς κατεκληρονόμησαν αὐτοῖς Ἐλεάζαρ ὁ ἱερεὺς, καὶ Ἰησοῦς ὁ τοῦ Ναυῆ, καὶ οἱ ἄρχοντες πατριῶν φυλῶν τῶν υἱῶν Ἰσραήλ.
\vs{2}Κατὰ κλήρους ἐκληρονόμησαν, ὃν τρόπον ἐνετείλατο Κύριος ἐν χειρὶ Ἰησοῦ ταῖς ἐννέα φυλαῖς, καὶ τῷ ἡμίσει φυλῆς
\vs{3}ἀπὸ τοῦ πέραν τοῦ Ἰορδάνου. Καὶ τοῖς Λευίταις οὐκ ἔδωκε κλῆρον ἐν αὐτοῖς,
\vs{4}ὅτι ἦσαν οἱ υἱοὶ Ἰωσὴφ δύο φυλαὶ Μανασσὴ καὶ Ἐφραΐμ· καὶ οὐκ ἐδόθη μερὶς ἐν τῇ γῇ τοῖς Λευίταις, ἀλλʼ ἤ πόλεις κατοικεῖν, καὶ τὰ ἀφωρισμένα αὐτῶν τοῖς κτήνεσι, καὶ τὰ κτήνη αὐτῶν.
\vs{5}Ὃν τρόπον ἐνετείλατο Κύριος τῷ Μωυσῇ, οὕτως ἐποίησαν οἱ υἱοὶ Ἰσραὴλ, καὶ ἐμέρισαν τὴν γῆν.

\vs{6}Καὶ προσήλθοσαν οἱ υἱοῖ Ἰούδα πρὸς Ἰησοῦν ἐν Γαλγάλ· καὶ εἶπε πρὸς αὐτὸν Χάλεβ ὁ τοῦ Ἰεφονὴ ὁ Κενεζαῖος, σὺ ἐπίστῃ τὸ ῥῆμα, ὃ ἐλάλησε Κύριος πρὸς Μωυσῆν ἄνθρωπον τοῦ Θεοῦ περὶ ἐμοῦ καὶ σοῦ ἐν Κάδης Βαρνῆ.
\vs{7}Τεσσαράκοντα γὰρ ἐτῶν ἤμην ὅτε ἀπέστειλέ με Μωυσῆς ὁ παῖς τοῦ Θεοῦ ἐκ Κάδης Βαρνῆ κατασκοπεῦσαι τὴν γῆν· καὶ ἀπεκρίθην αὐτῷ λόγον κατὰ τὸν νοῦν αὐτοῦ.
\vs{8}Οἱ ἀδελφοί μου οἱ ἀναβάντες μετʼ ἐμοῦ μετέστησαν τὴν καρδίαν τοῦ λαοῦ, ἐγὼ δὲ προσετέθην ἐπακολουθῆσαι Κυρίῳ τῷ Θεῷ μου.
\vs{9}Καὶ ὤμοσε Μωυσῆς ἐν ἐκείνῃ τῇ ἡμέρᾳ, λέγων, ἡ γῆ ἐφʼ ἣν ἐπέβης, σοὶ ἔσται ἐν κλήρῳ καὶ τοῖς τέκνοις σου εἰς τὸν αἰῶνα, ὅτι προσετέθης ἐπακολουθῆσαι ὀπίσω Κυρίου τοῦ Θεοῦ ἡμῶν.
\vs{10}Καὶ νῦν διέθρεψέ με Κύριος ὃν τρόπον εἶπε· τοῦτο τεσσαρακοστὸν καὶ πέμπτον ἔτος, ἀφʼ οὗ ἐλάλησε Κύριος τὸ ῥῆμα τοῦτο πρὸς Μωυσῆν· καὶ ἐπορεύθη Ἰσραὴλ ἐν τῇ ἐρήμῳ· καὶ νῦν ἰδοὺ ἐγὼ σήμερον ὀγδοήκοντα καὶ πέντε ἐτῶν·
\vs{11}Ἔτι εἰμὶ σήμερον ἰσχύων, ὡσεὶ ὅτε ἀπέστειλέ με Μωυσῆς, ὡσαύτως ἰσχύω νῦν ἐζελθεῖν καὶ εἰσελθεῖν εἰς τὸν πόλεμον.
\vs{12}Καὶ νῦν αἰτοῦμαί σε τὸ ὄρος τοῦτο, καθὰ εἰπε Κύριος τῇ ἡμέρᾳ ἐκείνῃ, ὅτι σὺ ἀκήκοας τὸ ῥῆμα τοῦτο ἐν τῇ ἡμέρᾳ ἐκείνῃ· νῦν δὲ οἱ Ἐνακὶμ ἐκεῖ εἰσι, πόλεις ὀχυραὶ καὶ μεγάλαι· ἐὰν οὖν Κύριος μετʼ ἐμοῦ ᾖ, ἐξολοθρεύσω αὐτοὺς, ὃν τρόπον εἶπέ μοι Κύριος.

\vs{13}Καὶ εὐλόγησεν αὐτὸν Ἰησοῦς, καὶ ἔδωκε Χεβρὼν τῷ Χάλεβ υἱῷ Ἰεφονῆ υἱῷ Κενὲζ ἐν κλήρῳ.
\vs{14}Διὰ τοῦτο ἐγενήθη ἡ Χεβρὼν τῷ Χάλεβ τῷ τοῦ Ἰεφονῆ τοῦ Κενεζαίου ἐν κλήρῳ ἕως τῆς ἡμέρας ταύτης, διὰ τὸ αὐτὸν ἐπακολουθῆσαι τῷ προστάγματι Κυρίου Θεοῦ Ἰσραήλ.
\vs{15}Τὸ δὲ ὄνομα τῆς Χεβρὼν ἦν τὸ πρότερον πόλις Ἀργὸβ, μητρόπολις τῶν Ἐνακὶμ αὕτη· καὶ ἡ γῆ ἐκόπασε τοῦ πολέμου.

\ch{15}
Καὶ ἐγένετο τὰ ὅρια φυλῆς Ἰούδα κατὰ δήμους αὐτῶν ἀπὸ τῶν ὁρίων τῆς Ἰδουμαίας ἀπὸ τῆς ἐρήμου Σὶν ἕως Κάδης πρὸς Λίβα.

\vs{2}Καὶ ἐγενήθη αὐτῶν τὰ ὅρια ἀπὸ Λιβὸς ἕως μέρους θαλάσσης τῆς ἁλυκῆς ἀπὸ τῆς λοφιᾶς τῆς φερούσης ἐπὶ Λίβα.
\vs{3}Καὶ διαπορεύεται ἀπέναντι τῆς προσαναβάσεως Ἀκραβίν· καὶ ἐκπεριπορεύεται Σενὰ, καὶ ἀναβαίνει ἀπὸ Λιβὸς ἐπὶ Κάδης Βαρνῆ· καὶ ἐκπορεύεται Ἀσωρὼν, καὶ προσαναβαίνει εἰς Σάραδα· καὶ ἐκπορεύεται τὴν κατὰ δυσμὰς Κάδης,
\vs{4}καὶ ἐκπορεύεται ἐπὶ Σελμωνὰν, καὶ διεκβάλλει ἕως φάραγγος Αἰγύπτου· καὶ ἔσται αὐτοῦ ἡ διέξοδος τῶν ὁρίων ἐπὶ τὴν θάλασσαν· τοῦτό ἐστιν αὐτῶν ὅρια ἀπὸ Λιβός.

\vs{5}Καὶ τὰ ὅρια ἀπὸ ἀνατολῶν πᾶσα ἡ θάλασσα ἡ ἁλυκὴ ἕως τοῦ Ἰορδάνου. Καὶ τὰ ὅρια αὐτῶν ἀπὸ Βοῤῥᾶ, καὶ ἀπὸ τῆς λοφιᾶς τῆς θαλάσσης καὶ ἀπὸ τοῦ μέρους τοῦ Ἰορδάνου.
\vs{6}Ἐπιβαίνει τὰ ὅρια ἐπὶ Βαιθαγλαάμ· καὶ παραπορεύεται ἀπὸ Βοῤῥᾶ ἐπὶ Βαιθάραβα, καὶ προσαναβαίνει τὰ ὅρια ἐπὶ λίθον Βαιὼν υἱοῦ Ῥουβήν·
\vs{7}Καὶ προσαναβαίνει τὰ ὅρια ἐπὶ τὸ τέταρτον τῆς φάραγγος Ἀχὼρ, καὶ καταβαίνει ἐπὶ Γαλγὰλ, ἥ ἐστιν ἀπέναντι τῆς προσβάσεως Ἀδαμμὶν, ἥ ἐστι κατὰ Λίβα τῇ φάραγγι, καὶ διεκβάλλεῖ ἐπὶ τὸ ὕδωρ τῆς πηγῆς τοῦ ἡλίου· καὶ ἔσται αὐτοῦ ἡ διέξοδος πηγὴ Ῥωγήλ·
\vs{8}Καὶ ἀναβαίνει τὰ ὅρια εἰς φάραγγα Ἐννὸμ, ἐπὶ νώτου τοῦ Ἰεβοῦις ἀπὸ Λιβός· αὕτη ἐστιν Ἱερουσαλήμ· καὶ διεκβάλλει τὰ ὅρια ἐπὶ κορυφὴν ὄρους, ἥ ἐστι κατὰ πρόσωπον φάραγγος Ἐννὸμ πρὸς θαλάσσης, ἥ ἐστιν ἐκ μέρους γῆς Ῥαφαῒν ἐπὶ βοῤῥᾶ·
\vs{9}Καὶ διεκβάλλει τὸ ὅριον ἀπὸ κορυφῆς τοῦ ὄρους ἐπὶ πηγὴν ὕδατος Ναφθὼ, καὶ διεκβάλλει εἰς τὸ ὄρος Ἐφρών· καὶ ἄξει τὸ ὅριον εἰς Βαάλ· αὕτη ἐστὶ πόλις Ἰαρίμ.
\vs{10}Καὶ περιελεύσεται ὅριον ἀπὸ Βαὰλ ἐπὶ θάλασσαν, καὶ παρελεύσεται εἰς ὄρος Ἀσσὰρ ἐπὶ νώτου πόλιν Ἰαρὶν ἀπὸ Βοῤῥᾶ· αὕτη ἐστὶ Χασλών· καὶ καταβήσεται ἐπὶ πόλιν ἡλίου, καὶ παρελεύσεται ἐπὶ Λίβα·
\vs{11}Καὶ διεκβάλλει τὸ ὅριον κατὰ νώτου Ἀκκαρὼν ἐπὶ βοῤῥᾶν, καὶ διεκβαλεῖ τὰ ὅρια εἰς Σοκχὼθ, καὶ παρελεύσεται ὅρια ἐπὶ Λίβα, καὶ διεκβαλεῖ ἐπὶ Λεβνὰ, καὶ ἔσται ἡ διέξοδος τῶν ὁρίων ἐπὶ θάλασσαν·
\vs{12}Καὶ τὰ ὅρια αὐτῶν ἀπὸ θαλάσσης, ἡ θάλασσα ἡ μεγάλη ὁριεῖ. Ταῦτα τὰ ὅρια υἱῶν Ἰούδα κύκλῳ κατὰ δήμους αὐτῶν.

\vs{13}Καὶ τῷ Χάλεβ υἱῷ Ἰεφονῆ ἔδωκε μερίδα ἐν μέσῳ υἱῶν Ἰούδα διὰ προστάγματος τοῦ Θεοῦ· καὶ ἔδωκεν αὐτῷ Ἰησοῦς τὴν πόλιν Ἀρβὸκ μητρόπολιν Ἐνάκ· αὕτη ἐστὶ Χεβρών.
\vs{14}Καὶ ἐξωλόθρευσεν ἐκεῖθεν Χάλεβ υἱὸς Ἰεφονὴ τοὺς τρεῖς υἱοὺς Ἐνὰκ, τὸν Σουσὶ καὶ Θολαμὶ καὶ τὸν Ἀχιμᾶ.
\vs{15}Καὶ ἀνέβη ἐκεῖθεν Χάλεβ ἐπὶ τοὺς κατοικοῦντας Δαβίρ· τὸ δὲ ὄνομα Δαβὶρ ἦν τὸ πρότερον πόλις Γραμμάτων.

\vs{16}Καὶ εἶπε Χάλεβ, ὃς ἂν λάβῃ καὶ ἐκκόψῃ τὴν πόλιν τῶν Γραμμάτων καὶ κυριεύσῃ αὐτῆς, δώσω αὐτῷ τὴν Ἀσχὰν θυγατέρα μου εἰς γυναῖκα.
\vs{17}Καὶ ἔλαβεν αὐτὴν Γοθονιὴλ υἱὸς Χενὲζ ἀδελφοῦ Χάλεβ· καὶ ἔδωκεν αὐτῷ τὴν Ἀσχὰν θυγατέρα αὐτοῦ γυναῖκα.
\vs{18}Καὶ ἐγένετο ἐν τῷ ἐκπορεύέσθαι αὐτὴν καὶ συνέβουλεύσατο αὐτῷ, λέγουσα, αἰτήσομαι τὸν πατέρα μου ἀγρόν· καὶ ἐβόησεν ἐκ τοῦ ὄνου· καὶ εἶπεν αὐτῇ Χάλεβ, τί ἐστί σοι;
\vs{19}Καὶ εἶπεν αὐτῷ, δός μοι εὐλογίαν, ὅτι εἰς γῆν Ναγὲβ δέδωκάς με· δός μοι τὴν Βοτθανίς· καὶ ἔδωκεν αὐτῇ τὴν Γοναιθλὰν τὴν ἄνω καὶ τὴν Γοναιθλὰν τὴν κάτω.

\vs{20}Αὕτη ἡ κληρονομία φυλῆς υἱῶν Ἰούδα.
\vs{21}Ἐγενήθησαν δὲ πόλεις αὐτῶν πόλεις πρὸς τῇ φυλῇ υἱῶν Ἰούδα ἐφʼ ὁρίων Ἐδὼμ ἐπὶ τῆς ἐρήμου, καὶ Βαισελεὴλ, καὶ Ἀρὰ, καὶ Ἀσὼρ,
\vs{22}καὶ Ἰκὰμ, καὶ Ῥεγμὰ, καὶ Ἀρουὴλ,
\vs{23}καὶ Κάδης, καὶ Ἀσοριωναὶν, καὶ Μαινὰμ,
\vs{24}καὶ Βαλμαινὰν, καὶ αἱ κῶμαι αὐτῶν,
\vs{25}καὶ αἱ πόλεις Ἀσερὼν, αὕτη Ἀσὼρ,
\vs{26}καὶ Σὴν, καὶ Σαλμαὰ, καὶ Μωλαδὰ,
\vs{27}καὶ Σερὶ, καὶ Βαιφαλὰθ,
\vs{28}καὶ Χολασεωλὰ, καὶ Βηρσαβεέ· καὶ αἱ κῶμαι αὐτῶν, καὶ αἱ ἐπαύλεις αὐτῶν,
\vs{29}Βαλὰ, καὶ Βακὼκ, καὶ Ἀσὸμ,
\vs{30}καὶ Ἐλβωϋδὰδ, καὶ Βαιθὴλ, καὶ Ἑρμὰ,
\vs{31}καὶ Σεκελὰκ, καὶ Μαχαρὶμ, καὶ Σεθεννὰκ,
\vs{32}καὶ Λαβὼς, καὶ Σαλὴ, καὶ Ἐρωμώθ· πόλεις εἰκοσιεννέα, καὶ αἱ κῶμαι αὐτῶν.

\vs{33}Ἐν τῇ πεδινῇ Ἀσταὼλ, καὶ Ῥάα, καὶ Ἄσσα.
\vs{34}Καὶ Ῥάμεν, καὶ Τανὼ, καὶ Ἰλουθὼθ, καὶ Μαιανὶ,
\vs{35}καὶ Ἰερμοὺθ, καὶ Ὀδολλὰμ, καὶ Μεμβρὰ, καὶ Σαωχὼ, καὶ Ἰαζηκὰ,
\vs{36}καὶ Σακαρὶμ, καὶ Γάδηρα, καὶ αἱ ἐπαύλεις αὐτῆς· πόλεις δεκατέσσαρες, καὶ αἱ κῶμαι αὐτῶν.
\vs{37}Σεννὰ, καὶ Ἀδασὰν, καὶ Μαγαδαλγὰδ,
\vs{38}καὶ Δαλὰδ, καὶ Μασφὰ, καὶ Ἰαχαρεὴλ,
\vs{39}καὶ Βασηδὼθ, καὶ Ἰδεαδαλέα, καὶ
\vs{40}Χαβρά, καὶ Μαχὲς, καὶ Μααχὼς,
\vs{41}καὶ Γεδδὼρ, καὶ Βαγαδιὴλ, καὶ Νωμὰν, καὶ Μαχηδάν· πόλεις ἑκκαίδεκα, καὶ αἱ κῶμαι αὐτῶν·
\vs{42}Λεβνὰ, καὶ Ἰθὰκ, καὶ Ἀνὼχ,
\vs{43}καὶ Ἰανὰ, καὶ Νασὶβ,
\vs{44}καὶ Κεϊλὰμ, καὶ Ἀκιεζὶ, καὶ Κεζὶβ, καὶ Βαθησὰρ, καὶ Αἰλώμ· πόλεις δέκα, καὶ αἱ κῶμαι αὐτῶν·
\vs{45}Ἀκκαρὼν, καὶ αἱ κῶμαι αὐτῆς, καὶ αἱ ἐπαύλεις αὐτῶν,
\vs{46}ἀπὸ Ἀκκαρὼν Γεμνά· καὶ πᾶσαι ὅσαι εἰσὶ πλησίον Ἀσηδώθ· καὶ αἱ κῶμαι αὐτῶν,
\vs{47}Ἀσιεδὼθ, καὶ αἱ κῶμαι αὐτῆς, καὶ αἱ ἐπαύλεις αὐτῆς· Γάζα, καὶ αἱ κῶμαι αὐτῆς, καὶ ἐπαύλεις αὐτῆς ἕως τοῦ χειμάῤῥου Αἰγύπτου, καὶ ἡ θάλασσα ἡ μεγάλη διορίζει.

\vs{48}Καὶ ἐν τῇ ὀρεινῇ Σαμὶρ, καὶ Ἰεθὲρ, καὶ Σωχὰ,
\vs{49}καὶ Ῥεννὰ, καὶ πόλις Γραμμάτων, αὕτη Δαβὶρ,
\vs{50}καὶ Ἀνὼν, καὶ Ἒς, καὶ Μὰν, καὶ Αἰσὰμ,
\vs{51}καὶ Γοσὸμ, καὶ Χαλοὺ, καὶ Χαννὰ, καὶ Γηλώμ· πόλεις ἕνδεκα, καὶ αἱ κῶμαι αὐτῶν·
\vs{52}Αἰρὲμ, καὶ Ῥεμνὰ, καὶ Σομὰ,
\vs{53}καὶ Ἰεμαῒν, καὶ Βαιθαχοὺ, καὶ Φακουὰ,
\vs{54}καὶ Εὐμὰ, καὶ πόλις Ἀρβὸκ, αὕτη ἐστὶ Χεβρὼν, καὶ Σωραίθ· πόλεις ἐννέα, καὶ αἱ ἐπαύλεις αὐτῶν·
\vs{55}Μαὼρ, καὶ Χερμὲλ, καὶ Ὀζὶβ, καὶ Ἰτὰν,
\vs{56}καὶ Ἰαριὴλ, καὶ Ἀρικὰμ, καὶ Ζακαναῒμ,
\vs{57}καὶ Γαβαὰ, καὶ Θαμναθά· πόλεις ἐννέα, καὶ αἱ κῶμαι αὐτῶν·
\vs{58}Αἰλουὰ, καὶ Βηθσοὺρ, καὶ Γεδδὼν,
\vs{59}καὶ Μαγαρὼθ, καὶ Βαιθανὰμ, καὶ Θεκούμ· πόλεις ἓξ, καὶ αἱ κῶμαι αὐτῶν·
\vs{59a}Θεκὼ, καὶ Ἐφραθὰ, αὕτη ἐστὶ Βαιθλεὲμ, καὶ Φαγὼρ, καὶ Αἰτὰν, καὶ Κουλὸν, καὶ Τατὰμ, καὶ Θωβὴς, καὶ Καρὲμ, καὶ Γαλὲμ καὶ Θεθὴρ, καὶ Μανοχώ· πόλεις ἕνδεκα, καὶ αἱ κῶμαι αὐτῶν·
\vs{60}Καριαθβαὰλ, αὕτη ἡ πόλις Ἰαρὶμ, καὶ Σωθηβᾶ· πόλεις δύο, καὶ αἱ ἐπαύλεις αὐτῶν·
\vs{61}καὶ Βαδδαργεὶς, καὶ Θαραβαὰμ, καὶ Αἰνὼν,
\vs{62}καὶ Αἰοχιοζὰ, καὶ Ναφλαζὼν, καὶ αἱ πόλεις Σαδῶν, καὶ Ἀρκάδης· πόλεις ἑπτὰ, καὶ αἱ κῶμαι αὐτῶν.

\vs{63}Καὶ ὁ Ἰεβουσαῖος κατῴκει ἐν Ἱερουσαλὴμ, καὶ οὐκ ἠδυνήθησαν οἱ υἱοὶ Ἰούδα ἀπολέσαι αὐτούς· καὶ κατῴκησαν οἱ Ἰεβουσαῖοι ἐν Ἱερουσαλὴμ ἕως τῆς ἡμέρας ταύτης.

\ch{16}
Καὶ ἐγένετο τὰ ὅρια υἱῶν Ἰωσὴφ ἀπὸ τοῦ Ἰορδάνου τοῦ κατὰ Ἱεριχὼ ἀπὸ ἀνατολῶν· καὶ ἀναβήσεται ἀπὸ Ἱεριχὼ εἰς τὴν ὀρεινὴν, τὴν ἔρημον, εἰς Βαιθὴλ Λουζά.
\vs{2}Καὶ ἐξελεύσεται εἰς Βαιθὴλ, καὶ παρελεύσεται ἐπὶ τὰ ὅρια τοῦ Ἀχαταρωθί.
\vs{3}Καὶ διελεύσεται ἐπὶ τὴν θάλασσαν ἐπὶ τὰ ὅρια Ἀπταλὶμ ἕως τῶν ὁρίων Βαιθωρὼν τὴν κάτω, καὶ ἔσται ἡ διέξοδος αὐτῶν ἐπὶ τὴν θάλασσαν.
\vs{4}Καὶ ἐκληρονόμησαν οἱ υἱοὶ Ἱωσὴφ, Ἐφραῒμ καὶ Μανασσή.

\vs{5}Καὶ ἐγενήθη ὅρια υἱῶν Ἐφραῒμ κατὰ δήμους αὐτῶν· καὶ ἐγενήθη τὰ ὅρια τῆς κληρονομίας αὐτῶν ἀπʼ ἀνατολῶν Ἀταρὼθ, καὶ Ἐρὼκ ἕως Βαιθωρὼν τὴν ἄνω, καὶ Γαζαρά.
\vs{6}Καὶ ἐλεύσεται τὰ ὅρια ἐπὶ τὴν θάλασσαν εἰς Ἰκασμὼν ἀπὸ Βοῤῥᾶ Θερμᾶ· περιελεύσεται ἐπʼ ἀνατολὰς εἰς Θηνασὰ, καὶ Σέλλης, καὶ παρελεύσεται ἀπʼ ἀνατολῶν εἰς Ἰανῶκὰ,
\vs{7}καὶ εἰς Μαχὼ, καὶ Ἀταρὼθ, καὶ αἱ κῶμαι αὐτῶν· καὶ ἐλεύσεται ἐπὶ Ἱεριχὼ, καὶ διεκβαλεῖ ἐπὶ τὸν Ἰορδάνην.
\vs{8}Καὶ ἀπὸ Τάφου πορεύσεται τὰ ὅρια ἐπὶ θάλασσαν ἐπὶ Χελκανα· καὶ ἔσται ἡ διέξοδος αὐτῶν ἐπὶ θάλασσαν· αὕτη ἡ κληρονομία φυλῆς Ἐφραῒμ κατὰ δήμους αὐτῶν.

\vs{9}Καὶ αἱ πόλεις αἱ ἀφορισθεῖσαι τοῖς υἱοῖς Ἐφραῒμ ἀναμέσον τῆς κληρονομίας νἱῶν Μανασσή, πᾶσαι αἱ πόλεις καὶ αἱ κῶμαι αὐτῶν.
\vs{10}Καὶ οὐκ ἀπώλεσεν Ἐφραῒμ τὸν Χαναναῖον τὸν κατοικοῦντα ἐν Γάζέρ· καὶ κατῴκει ὁ Χαναναῖος ἐν τῷ Ἐφραῒμ ἕως τῆς ἡμέρας ταυτης, ἕως ἀνέβη Φαραὼ βασιλεὺς Αἰγύπτου, καὶ ἔλαβεν αὐτὴν, καὶ ἐνέπρησεν αὐτὴν ἐν πυρί· καὶ τοὺς Χαναναίους, καὶ τοὺς Φερεζαίους, καὶ τοὺς κατοικοῦντας ἐν Γαζὲρ, ἐξεκέντησαν· καὶ ἔδωκεν αὐτὴν Φαραὼ ἐν φερνῇ τῇ θυγατρὶ αὐτοῦ.

\ch{17}
Καὶ ἐγένετο τὰ ὅρια φυλῆς υἱῶν Μανασσῆ, ὅτι οὗτος πρωτότοκος τῷ Ἰωσὴφ, τῷ Μαχὶρ πρωτοτόκῳ Μανασσῆ πατρὶ Γαλαὰδ, ἀνὴρ γὰρ πολεμιστὴς ἦν, ἐν τῇ Γαλααδίτιδι καὶ ἐν τῇ Βασανίτιδι.
\vs{2}Καὶ ἐγενήθη τοῖς υἱοῖς Μανασσῆ τοῖς λοιποῖς κατὰ δήμους αὐτῶν· τοῖς υἱοῖς Ἰεζὶ, καὶ τοῖς υἱοῖς Κελὲζ, καὶ τοῖς υἱοῖς Ἰεζιὴλ, καὶ τοῖς υἱοῖς Συχὲμ, καὶ τοῖς υἱοῖς Συμαρὶμ, καὶ τοῖς υἱοῖς Ὀφέρ· οὗτοι ἄρσενες κατὰ δήμους αὐτῶν.

\vs{3}Καὶ τῷ Σαλπαὰδ υἱῷ Ὀφὲρ οὐκ ἦσαν αὐτῷ υἱοὶ ἀλλʼ ἢ θυγατέρες· καὶ ταῦτα τὰ ὀνόματα τῶν θυγατέρων Σαλπαάδ· Μααλὰ, καὶ Νουὰ, καὶ Ἐγλὰ, καὶ Μελχὰ, καὶ Θερσά.
\vs{4}Καὶ ἔστησαν ἐναντίον Ἐλεάζαρ τοῦ ἱερέως, καὶ ἐναντίον Ἰησοῦ, καὶ ἐναντίον τῶν ἀρχόντων, λέγουσαι, ὁ Θεὸς ἐνετείλατο διὰ χειρὸς Μωυσῆ δοῦναι ἡμῖν κληρονομίαν ἐν μέσῳ τῶν ἀδελφῶν ἡμῶν· καὶ ἐδόθη αὐταῖς διὰ προστάγματος Κυρίου κλῆρος ἐν τοῖς ἀδελφοῖς τοῦ πατρὸς αὐτῶν.
\vs{5}Καὶ ἔπεσεν ὁ σχοινισμὸς αὐτῶν ἀπὸ Ἀνάσσα, καὶ πεδίον Λαβὲκ ἐκ τῆς γῆς Γαλαὰδ, ἥ ἐστι πέραν τοῦ Ἰορδάνου·
\vs{6}Ὅτι θυγατέρες υἱῶν Μανασσὴ ἐκληρονόμησαν κλῆρον ἐν μέσῳ τῶν ἀδελφῶν αὐτῶν· ἡ δὲ γῆ Γαλαὰδ ἐγενήθη τοῖς υἱοῖς Μανασσὴ τοῖς καταλελειμμένοις.

\vs{7}Καὶ ἐγενήθη ὅρια υἱῶν Μανασσῆ Δηλανὰθ, ἥ ἐστι κατὰ πρόσωπον υἱῶν Ἀνὰθ, καὶ πορεύεται ἐπὶ τὰ ὅρια ἐπὶ Ἰαμὶν καὶ Ἰασσὶβ ἐπὶ πηγὴν Θαφθώθ.
\vs{8}Τῷ Μανασσῇ ἔσται· καὶ Θαφὲθ ἐπὶ τῶν ὁρίων Μανασσῆ, τοῖς υἱοῖς Ἐφραΐμ.
\vs{9}Καὶ καταβήσεται τὰ ὅρια ἐπὶ φάραγγα Καρανὰ ἐπὶ Λίβα κατὰ φάραγγα Ἰαριὴλ· τερέμινθος τῷ Ἐφραῒμ ἀναμέσον πόλεως Μανασσῆ· καὶ ὅρια Μανασσῆ ἐπὶ τὸν βοῤῥᾶν εἰς τὸν χειμάῤῥουν· καὶ ἔσται αὐτοῦ ἡ διέξοδος θάλασσα
\vs{10}ἀπὸ Λιβὸς τῷ Ἐφραῒμ, καὶ ἐπὶ Βοῤῥᾶν Μανασσῇ· καὶ ἔσται ἡ θάλασσα ὅρια αὐτοῖς· καὶ ἐπὶ Ἀσὴβ συνάψουσιν ἐπὶ Βοῤῥᾶν, καὶ τῷ Ἰσσάχαρ ἀπὸ ἀνατολῶν.
\vs{11}Καὶ ἔσται Μανασσῇ ἐν Ἰσσάχαρ καὶ ἐν Ἀσὴρ Βαιθοὰν καὶ αἱ κῶμαι αὐτῶν, καὶ τοὺς κατοικοῦντας Δὼρ, καὶ τὰς κώμας αὐτῆς, καὶ τοὺς κατοικοῦντας Μαγεδδὼ, καὶ τὰς κώμας αὐτῆς, καὶ τὸ τρίτον τῆς Μαφετὰ, καὶ τὰς κώμας αὐτῆς.

\vs{12}Καὶ οὐκ ἠδυνάσθησαν οἱ υἱοὶ Μανασσῆ ἐξολοθρεῦσαι τὰς πόλεις ταύτας· καὶ ἤρχετο ὁ Χαναναῖος κατοικεῖν ἐν τῇ γῇ ταύτῃ.
\vs{13}Καὶ ἐγενήθη καὶ ἐπεὶ κατίσχυσαν οἱ υἱοὶ Ἰσραὴλ, καὶ ἐποίησαν τοὺς Χαναναίους ὑπηκόους, ἐξολοθρεῦσαι δὲ αὐτοὺς οὐκ ἐξωλόθρευσαν.

\vs{14}Ἀντεῖπαν δὲ οἱ νἱοὶ Ἰωσὴφ τῷ Ἰησοῦ, λέγοντες, διατί ἐκληρονόμησας ἡμᾶς κλῆρον ἕνα, καὶ σχοίνισμα ἕν; ἐγὼ δὲ λαὸς πολύς εἰμι, καὶ ὁ Θεὸς εὐλόγησέ με.
\vs{15}Καὶ εἶπεν αὐτοῖς Ἰησοῦς, εἰ λαὸς πολὺς εἶ, ἀνάβηθι εἰς τὸν δρυμὸν, καὶ ἐκκάθαρον σεαυτῷ εἰ στενοχωρεῖ σε τὸ ὄρος τὸ Ἐφραΐμ.
\vs{16}Καὶ εἶπαν, οὐκ ἀρέσκει ἡμῖν τὸ ὄρος τὸ Ἐφραΐμ· καὶ ἵππος ἐπίλεκτος, καὶ σίδηρος τῷ Χαναναίῳ τῷ κατοικοῦντι ἐν αὐτῷ ἐν Βαιθσὰν, καὶ ἐν ταῖς κώμαις αὐτῆς, ἐν τῇ κοιλάδι Ἰεζραέλ.
\vs{17}Καὶ εἶπεν Ἰησοῦς τοῖς υἱοῖς Ἰωσὴφ, εἰ λαὸς πολὺς εἶ καὶ ἰσχὺν μεγάλην ἔχεις, οὐκ ἔσται σοι κλῆρος εἷς·
\vs{18}Ὁ γὰρ δρυμὸς ἔσται σοι, ὅτι δρυμός ἐστι καὶ ἐκκαθαριεῖς αὐτὸν, καὶ ἔσται σοι· καὶ ὅταν ἐξολοθρεύ· σῃς τὸν Χαναναῖον, ὅτι ἵππος ἐπίλεκτος αὐτῷ ἐστι· σὺ γὰρ ὑπερισχύεις αὐτοῦ.

\ch{18}
Καὶ ἐξεκκλησιάσθη πᾶσα συναγωγὴ υἱῶν Ἰσραὴλ εἰς Σηλὼ, καὶ ἔπηξαν ἐκεῖ τὴν σκηνὴν τοῦ μαρτυρίου· καὶ ἡ γῆ ἐκρατήθη ὑπʼ αὐτῶν.

\vs{2}Καὶ κατελείφθησαν οἱ υἱοὶ Ἰσραὴλ, οἳ οὐκ ἐκληρονόμησαν, ἑπτὰ φυλαί.
\vs{3}Καὶ εἶπεν Ἰησοῦς τοῖς υἱοῖς Ἰσραὴλ, ἕως τίνος ἐκλυθήσεσθε κληρονομῆσαι τὴν γῆν, ἣν ἔδωκε Κύριος ὁ Θεὸς ἡμῶν;
\vs{4}Δότε ἐξ ὑμῶν ἄνδρας τρεῖς ἐκ φυλῆς, καὶ ἀναστάντες διελθέτωσαν τὴν γῆν, καὶ διαγραψάτωσαν αὐτὴν ἐναντίον μου, καθὰ δεήσει διελεῖν αὐτήν. Καὶ διέλθοσαν πρὸς αὐτόν·
\vs{5}καὶ διεῖλεν αὐτοῖς ἑπτὰ μερίδας· Ἰούδας στήσεται αὐτοῖς ὅριον ἀπὸ Λιβὸς, καὶ οἱ υἱοὶ Ἰωσὴφ στήσονται αὐτοῖς ἀπὸ Βοῤῥᾶ.
\vs{6}Ὑμεῖς δὲ μερίσατε τὴν γῆν ἑπτὰ μερίδας, καὶ ἐνέγκατε ὧδε πρὸς μὲ, καὶ ἐξοίσω ὑμῖν κλῆρον ἔναντι Κυρίου τοῦ Θεοῦ ἡμῶν.
\vs{7}Οὐ γάρ ἐστι μερὶς τοῖς υἱοῖς Λευὶ ἐν ὑμῖν· ἱερατεία γὰρ Κυρίου μερὶς αὐτοῦ· καὶ Γὰδ καὶ Ῥουβὴν καὶ τὸ ἥμισυ φυλῆς Μανασσῆ ἐλάβοσαν τὴν κληρονομίαν αὐτῶν πέραν τοῦ Ἰορδάνου ἐπʼ ἀνατολῆς, ἣν ἔδωκεν αὐτοῖς Μωυσῆς ὁ παῖς Κυρίου.

\vs{8}Καὶ ἀναστάντες οἱ ἄνδρες ἐπορεύθησαν· καὶ ἐνετείλατο Ἰησοῦς τοῖς ἀνδράσι τοῖς πορευομένοις χωροβατῆσαι τὴν γῆν, λέγων, πορεύεσθε καὶ χωροβατήσατε τὴν γῆν, καὶ παραγενήθητε πρὸς μὲ, καὶ ὧδε ἐξοίσω ὑμῖν κλῆρον ἔναντι Κυρίου ἐν Σηλώ.
\vs{9}Καὶ ἐπορεύθησαν, καὶ ἐχωροβάτησαν τὴν γῆν· καὶ εἴδοσαν αὐτὴν, καὶ ἔγραψαν αὐτὴν κατὰ πόλεις, ἑπτα μερίδας εἰς βιβλίον, καὶ ἤνεγκαν πρὸς Ἰησοῦν.
\vs{10}Καὶ ἐνέβαλεν αὐτοῖς Ἰησοῦς κλῆρον ἐν Σηλὼ ἔναντι Κυρίου.

\vs{11}Καὶ ἐξῆλθεν ὁ κλῆρος φυλῆς Βενιαμὶν πρῶτος κατὰ δήμους αὐτῶν· καὶ ἐξῆλθεν ὅρια τοῦ κλήρου αὐτῶν ἀναμέσον υἱῶν Ἰούδα καὶ ἀναμέσον τῶν υἱῶν Ἰωσήφ.

\vs{12}Καὶ ἐγενήθη αὐτῶν τὰ ὅρια ἀπὸ Βοῤῥᾶ· ἀπὸ τοῦ Ἱορδάνου προσαναβήσεται τὰ ὅρια κατὰ νὼτου Ἱεριχὼ ἀπὸ Βοῤῥᾶ, καὶ ἀναβήσεται ἐπὶ τὸ ὄρος ἐπὶ τὴν θάλασσαν, καὶ ἔσται αὐτοῦ ἡ διέξοδος ἡ Μαβδαρίτις Βαιθών.
\vs{13}Καὶ διελεύσεται ἐκεῖθεν τὰ ὅρια Λοῦζὰ ἐπὶ νώτου Λουζὰ ἀπὸ Λιβὸς αὐτῆς· αὕτη ἐστὶ Βαιθήλ· καὶ καταβήσεται τὰ ὅρια Μααταρὼβ Ὀρὲχ ἐπὶ τὴν ὀρεινὴν, ἥ ἐστι πρὸς Λίβα Βαιθωρὼν ἡ κάτω.

\vs{14}Καὶ διελεύσεται τὰ ὅρια καὶ παρελεύσεται ἐπὶ τὸ μέρος τὸ βλέπον παρὰ θάλασσαν ἀπὸ Λιβὸς ἀπὸ τοῦ ὄρους ἐπὶ πρόσωπον Βαιθωρὼν Λίβα· καὶ ἔσται αὐτοῦ ἡ διέξοδος εἰς Καριὰθ Βαάλ· αὕτη ἐστὶ Καριαθιαρὶν, πόλις υἱῶν Ἰούδα· τοῦτό ἐστι τὸ μέρος τὸ πρὸς θάλασσαν.

\vs{15}Καὶ μέρος τὸ πρὸς Λίβα ἀπὸ μέρους Καριὰθ Βαάλ· καὶ διελεύσεται ὅρια εἰς Γασὶν, ἐπὶ πηγὴν ὕδατος Ναφθώ.
\vs{16}Καὶ καταβήσεται τὰ ὅρια ἐπὶ μέρους, τοῦτό ἐστι κατὰ πρόσωπον νάπης Σοννὰμ, ὅ ἐστιν ἐκ μέρους Ἐμὲκ Ῥαφαῒν ἀπὸ Βοῤῥᾶ, καὶ καταβήσεται Γαίεννα ἐπὶ νῶτον Ἰεβουσαὶ ἀπὸ Λιβός· καταβήσεται ἐπὶ πηγὴν Ῥωγήλ·
\vs{17}Καὶ διελεύσεται ἐπὶ πηγὴν Βαιθσαμύς· καὶ παρελεύσεται ἐπὶ Γαλιλὼθ, ἥ ἐστιν ἀπέναντι πρὸς ἀνάβασις Αἰθαμίν· καὶ καταβήσεται ἐπὶ λίθον Βαιὼν υἱῶν Ῥουβήν·
\vs{18}καὶ διελεύσεται κατὰ νώτου Βαιθάραβα ἀπὸ Βοῤῥᾶ, καὶ καταβήσεται ἐπὶ τὰ ὅρια ἐπὶ νῶτον θάλασσαν ἀπὸ Βοῤῥᾶ.
\vs{19}Καὶ ἔσται ἡ διέξοδος τῶν ὁρίων ἐπὶ λοφιὰν τῆς θαλάσσης τῶν ἁλῶν ἐπὶ Βοῤῥᾶν εἰς μέρος τοῦ Ἰορδάνου ἀπὸ Λιβός· ταῦτα τὰ ὅριά ἐστιν ἀπὸ Λιβός.

\vs{20}Καὶ ὁ Ἰορδάνης ὁριεῖ ἀπὸ μέρους ἀνατολῶν· αὕτη ἡ κληρονομία υἱῶν Βενιαμὶν, τὰ ὅρια αὐτῆς κύκλῳ κατὰ δήμους.

\vs{21}Καὶ ἐγενήθησαν αἱ πόλεις τῶν υιῶν Βενιαμὶν κατὰ δήμους αὐτῶν Ἱερειχὼ, καὶ Βεθεγαιὼ, καὶ Ἀμεκασὶς,
\vs{22}καὶ Βαιθαβαρὰ, καὶ Σαρὰ, καὶ Βησανὰ,
\vs{23}καὶ Αἰεὶν, καὶ Φαρὰ, καὶ Ἐφραθὰ,
\vs{24}καὶ Καραφὰ, καὶ Κεφιρὰ, καὶ Μονὶ, καὶ Γαβαὰ, πόλεις δώδεκα· καὶ αἱ κῶμαι αὐτῶν,
\vs{25}Γαβαὼν, καὶ Ῥαμὰ, καὶ Βεηρωθὰ,
\vs{26}καὶ Μασσημὰ, καὶ Μιρὼν, καὶ Ἀμωκὴ,
\vs{27}καὶ Φιρὰ, καὶ Καφὰν, καὶ Νακὰν, καὶ Ζεληκὰν, καὶ Θαρεηλὰ,
\vs{28}καὶ Ἰηβοῦς· αὕτη ἐστὶν Ἱερουσαλήμ· καὶ Γαβαὼθ, Ἰαρὶμ, πόλεις δεκατρεῖς, καὶ αἱ κῶμαι αὐτῶν· αὕτη ἡ κληρονομία υἱῶν Βενιαμὶν κατὰ δήμους αὐτῶν.

\ch{19}
Καὶ ἐξῆλθεν ὁ δεύτερος κλῆρος τῶν υἱῶν Συμεών· καὶ ἐγενήθη ἡ κληρονομία αὐτῶν ἀναμέσον κλήρων υἱῶν Ἰούδα.
\vs{2}Καὶ ἐγενήθη ὁ κλῆρος αὐτῶν Βηρσαβεὲ, καὶ Σαμαὰ, καὶ Καλαδὰμ,
\vs{3}καὶ Ἀρσωλὰ, καὶ Βωλὰ, καὶ Ἰασὸν,
\vs{4}καὶ Ἐρθουλὰ, καὶ Βουλὰ, καὶ Ἑρμὰ,
\vs{5}καὶ Σικελὰκ, καὶ Βαιθμαχερὲβ, καὶ Σαρσουσὶν,
\vs{6}καὶ Βαθαρὼθ, καὶ οἱ ἀγροὶ αὐτῶν· πόλεις δεκατρεῖς, καὶ αἱ κῶμαι αὐτῶν.
\vs{7}Ἐρεμμὼν, καὶ Θαλχὰ, καὶ Ἰεθὲρ, καὶ Ἀσάν· πόλεις τέσσαρες καὶ αἱ κῶμαι αὐτῶν,
\vs{8}κύκλῳ τῶν πόλεων αὐτῶν ἕως Βαλὲκ πορευομένων Βαμὲθ κατὰ Λίβα· αὕτη ἡ κληρονομία φυλῆς υἱῶν Συμεὼν κατὰ δήμους αὐτῶν.
\vs{9}Ἀπὸ τοῦ κλήρου τοῦ Ἰούδα ἡ κληρονομία φυλῆς υἱῶν Συμεὼν, ὅτι ἐγενήθη ἡ μερὶς υἱῶν Ἰούδα μείζων τῆς αὐτῶν· καὶ ἐκληρονόμησαν οἱ υἱοὶ Συμεὼν ἐν μέσῳ τοῦ κλήρου αὐτῶν.

\vs{10}Καὶ ἐξῆλθεν ὁ κλῆρος ὁ τρίτος τῷ Ζαβουλὼν κατὰ δήμους αὐτῶν· ἔσται τὰ ὅρια τῆς κληρονομίας αὐτῶν, Ἐσεδεκγωλὰ ὅρια αὐτῶν,
\vs{11}ἡ θάλασσα καὶ Μαγελδὰ, καὶ συνάψει ἐπὶ Βαιθάραβὰ εἰς τὴν φάραγγα, ἥ ἐστι κατὰ πρόσωπον Ἰεκμάν.
\vs{12}Καὶ ἀνέστρεψεν ἀπὸ Σεδδοὺκ ἐξ ἐναντίας ἀπὸ ἀνατολῶν Βαιθσαμὺς ἐπὶ τὰ ὅρια Χασελωθαὶθ, καὶ διελεύσεται ἐπὶ Δαβιρὼθ, καὶ προσαναβήσεται ἐπὶ Φαγγαί.
\vs{13}Καὶ ἐκεῖθεν περιελεύσεται ἐξ ἐναντίας ἐπʼ ἀνατολὰς ἐπὶ Γεβερὲ ἐπὶ πόλιν Κατασὲμ, καὶ διελεύσεται ἐπὶ Ῥεμμωναὰ Μαθαραοζά.
\vs{14}Καὶ περιελεύσεται ὅρια ἐπὶ Βοῤῥᾶν ἐπὶ Ἀμὼθ, καὶ ἔσται ἡ διέξοδος αὐτῶν ἐπὶ Γαιφαὴλ,
\vs{15}καὶ Κατανὰθ, καὶ Ναβαὰλ, καὶ Συμοὼν, καὶ Ἱεριχὼ, καὶ Βαιθμάν.
\vs{16}Αὕτη ἡ κληρονομία τῆς φυλῆς υἱῶν Ζαβουλὼν κατὰ δήμους αὐτῶν, πόλεις καὶ αἱ κῶμαι αὐτῶν.

\vs{17}Καὶ τῷ Ἰσσὰχαρ ἐξῆλθεν ὁ κλῆρος ὁ τέταρτος.
\vs{18}Καὶ ἐγενήθη τὰ ὅρια αὐτῶν Ἰαζὴλ, καὶ Χασαλὼθ, καὶ Σουνὰμ,
\vs{19}καὶ Ἀγὶν, καὶ Σιωνὰ, καὶ Ῥεηρὼθ, καὶ Ἀναχερὲθ,
\vs{20}καὶ Δαβιρὼν, καὶ Κισὼν, καὶ Ῥεβές,
\vs{21}καὶ Ῥεμμὰς, καὶ Ἰεὼν, καὶ Τομμὰν, καὶ Αἰμαρὲκ, καὶ Βηρσαφής.
\vs{22}Καὶ συνάψει τὰ ὅρια ἐπὶ Γαιθβὼρ, καὶ ἐπὶ Σαλὶμ κατὰ θάλασσαν, καὶ Βαιθσαμύς· καὶ ἔσται αὐτοῦ ἡ διέξοδος τῶν ὁρίων ὁ Ἰορδάνης.
\vs{23}Αὕτη ἡ κληρονομία φυλῆς υἱῶν Ἰσσάχαρ κατὰ δήμους αὐτῶν, αἱ πόλεις καὶ αἱ κῶμαι αὐτῶν.

\vs{24}Καὶ ἐξῆλθεν ὁ κλῆρος ὁ πέμπτος Ἀσὴρ κατὰ δήμους αὐτῶν.
\vs{25}Καὶ ἐγενήθη τὰ ὅρια αὐτῶν Ἐξελεκὲθ, καὶ Ἀλὲφ, καὶ Βαιθὸκ, καὶ Κεὰφ,
\vs{26}καὶ Ἐλιμελὲχ, καὶ Ἀμιὴλ, καὶ Μαασά· καὶ συνάψει τῷ Καρμήλῳ κατὰ θάλασσαν, καὶ τῷ Σιὼν, καὶ Λαβανάθ.
\vs{27}Καὶ ἐπιστρέψει ἀπὸ ἀνατολῶν ἡλίου καὶ Βαιθεγενὲθ, καὶ συνάψει τῷ Ζαβουλὼν καὶ Ἐκγαῖ, καὶ Φθαιὴλ κατὰ Βοῤῥᾶν, καὶ εἰσελεύσεται ὅρια Σαφθαιβαιθμὲ, καὶ Ἰναὴλ, καὶ διελεύσεται εἰς Χωβαμασομὲλ,
\vs{28}καὶ Ἐλβὼν, καὶ Ῥαὰβ, καὶ Ἐμεμαὼν, καὶ Κανθὰν ἕως Σιδῶνος τῆς μεγάλης.
\vs{29}Καὶ ἀναστρέψει τὰ ὅρια εἰς Ῥαμὰ, καὶ ἕως πηγῆς Μασφασσὰτ, καὶ τῶν Τυρίων· καὶ ἀναστρέψει τὰ ὅρια ἐπὶ Ἰασὶφ, καὶ ἔσται ἡ διέξοδος αὐτοῦ ἡ θάλασσα, καὶ Ἀπολὲβ, καὶ Ἐχοζὸβ,
\vs{30}καὶ Ἀρχὸβ, καὶ Ἀφὲκ, καὶ Ῥααύ.
\vs{31}Αὕτη ἡ κληρονομία φυλῆς υἱῶν Ἀσὴρ κατὰ δήμους αὐτῶν, πόλεις καὶ αἱ κῶμαι αὐτῶν.

\vs{32}Καὶ τῷ Νεφθαλὶ ἐξῆλθεν ὁ κλῆρος ὁ ἕκτος.
\vs{33}Καὶ ἐγενήθη τὰ ὅρια αὐτῶν Μοολὰμ, καὶ Μωλὰ, καὶ Βεσεμιῒν, καὶ Ἀρμὲ, καὶ Ναβὸκ, καὶ Ἰεφθαμαὶ, ἕως Δωδάμ· καὶ ἐγενήθησαν αἱ διέξοδοι αὐτοῦ Ἰορδάνης.
\vs{34}Καὶ ἐπιστρέψει τὰ ὅρια ἐπὶ θάλασσαν ἐν Ἀθθαβὼρ, καὶ διελεύσεται ἐκεῖθεν Ἰακανὰ, καὶ συνάψει τῷ Ζαβουλὼν ἀπὸ Νότου, καὶ Ἀσὴρ συνάψει κατὰ θάλασσαν, καὶ ὁ Ἰορδάνης ἀπὸ ἀνατολῶν ἡλίου.

\vs{35}Καὶ αἱ πόλεις τειχήρεις τῶν Τυρίων, Τυρὸς, καὶ Ὠμαθαδακὲθ, καὶ Κενερέθ,
\vs{36}καὶ Ἀρμαὶθ, καὶ Ἀραὴλ, καὶ Ἀσὼρ,
\vs{37}καὶ Κάδες, καὶ Ἀσσαρὶ, καὶ πηγὴ Ἀσὸρ,
\vs{38}καὶ Κερωὲ, καὶ Μεγαλααρὶμ, καὶ Βαιθθαμὲ, καὶ Θεσσαμύς.
\vs{39}Αὕτη ἡ κληρονομία φυλῆς υἱῶν Νεφθαλί.

\vs{40}Καὶ τῷ Δὰν ἐξῆλθεν ὁ κλῆρος ὁ ἕβδομος·
\vs{41}Καὶ ἐγενήθη τὰ ὅρια αὐτῶν Σαρὰθ, καὶ Ἀσὰ, καὶ πόλεις Σαμμαὺς,
\vs{42}καὶ Σαλαμὶν, καὶ Ἀμμὸν, καὶ Σιλαθὰ,
\vs{43}καὶ Ἐλὼν, καὶ Θαμναθὰ, καὶ Ἀκκαρὼν,
\vs{44}καὶ Ἀλκαθὰ, καὶ Βεγεθὼν, καὶ Γεβεελὰν,
\vs{45}καὶ Ἀζὼρ, καὶ Βαναιβακὰτ, καὶ Γεθρεμμὼν,
\vs{46}καὶ ἀπὸ θαλάσσης Ἱερὰκων ὅριον πλησίον Ἰόππης.
\vs{47}Αὕτη ἡ κληρονομία φυλῆς υἱῶν Δὰν κατὰ δήμους αὐτῶν, αἱ πόλεις αὐτῶν καὶ αἱ κῶμαι αὐτῶν·
\vs{47a}καὶ οὐκ ἐξέθλιψαν οἱ υἱοὶ Δὰν τὸν Ἀμοῤῥαῖον τὸν θλίβοντα αὐτοὺς ἐν τῷ ὄρει· καὶ οὐκ εἴων αὐτοὺς οἱ Ἀμοῥῥαῖοι καταβῆναι εἰς τὴν κοιλάδα, καὶ ἔθλιψαν ἀπʼ αὐτῶν τὸ ὅριον τῆς μερίδος αὐτῶν.

\vs{48}Καὶ ἐπορεύθησαν οἱ υἱοὶ Δὰν καὶ ἐπολέμησαν τὴν Λαχὶς, καὶ κατελάβοντο αὐτὴν, καὶ ἐπάταξαν αὐτὴν ἐν στόματι μαχαίρας· καὶ κατῴκησαν αὐτὴν καὶ ἐκάλεσαν τὸ ὄνομα αὐτῆς Λασενδάν·
\vs{48a}καὶ ὁ Ἀμοῤῥαῖος ὑπέμεινε τοῦ κατοικεῖν ἐν Ἐλὼμ καὶ ἐν Σαλαμίν· καὶ ἐβαρύνθη ἡ χεὶρ τοῦ Ἐφραὶμ ἐπʼ αὐτοὺς, καὶ ἐγένοντο αὐτοῖς εἰς φόρον.

\vs{49}Καὶ ἐπορεύθησαν ἐμβατεύσαι τὴν γῆν κατὰ τὸ ὅριον αὐτῶν· καὶ ἔδωκαν οἱ υἱοὶ Ἰσραὴλ κλῆρον τῷ Ἰησοῖ τῷ υἱῷ Ναυῆ ἐν αὐτοῖς
\vs{50}διὰ προστάγματος τοῦ Θεοῦ, καὶ ἔδωκαν αὐτῷ τὴν πόλιν, ἣν ᾐτήσατο, Θαμνασαρὰχ, ἥ ἐστιν ἐν τῷ ὄρει Ἐφραίμ· καὶ ᾠκοδόμησε τὴν πόλιν, καὶ κατῴκει ἐν αὐτῇ.

\vs{51}Αὗται αἱ διαιρέσεις ἃς κατεκληρονόμησεν Ἐλεάζαρ ὁ ἱερεὺς, καὶ Ἰησοῦς ὁ τοῦ Ναυῆ, καὶ οἱ ἄρχοντες τῶν πατριῶν ἐν ταῖς φυλαῖς Ἰσραὴλ κατὰ κλήρους ἐν Σηλὼ ἔναντι Κυρίου, παρὰ τὰς θύρας τῆς σκηνῆς τοῦ μαρτυρίου· καὶ ἐπορεύθησαν ἐμβατεῦσαι τὴν γῆν.

\ch{20}
Καὶ ἐλάλησε Κύριος τῷ Ἰησοῖ, λέγων,
\vs{2}λάλησον τοῖς υἱοῖς Ἰσραὴλ, λέγων, δότε τὰς πόλεις τῶν φυγαδευτηρίων, ἃς εἶπα πρὸς ὑμᾶς διὰ Μωυσῆ.
\vs{3}Φυγαδευτήριον τῷ φονευτῇ τῷ πατάξαντι ψυχὴν ἀκουσίως· καὶ ἔσονται ὑμῖν αἱ πόλεις φυγαδευτήριον, καὶ οὐκ ἀποθανεῖται ὁ φονευτὴς ὑπὸ τοῦ ἀγχιστεύοντος τὸ αἷμα, ἕως ἂν καταστῇ· ἐναντίον τῆς συναγωγῆς εἰς κρίσιν.

\vs{7}Καὶ διέστειλε τὴν Κάδης ἐν τῇ Γαλιλαίᾳ ἐν τῷ ὄρει τῷ Νεφθαλὶ, καὶ Συχὲμ ἐν τῷ ὄρει τῷ Ἐφραὶμ, καὶ τὴν πόλιν Ἀρβὸκ, αὕτη ἐστὶ Χεβρὼν, ἐν τῷ ὄρει τῷ Ἰούδα.

\vs{8}Καὶ ἐν τῷ πέραν τοῦ Ἰορδάνου ἔδωκε Βοσὸρ ἐν τῇ ἐρήμῳ ἐν τῷ πεδίῳ ἀπὸ τῆς φυλῆς Ῥουβὴν, καὶ Ἀρημὼθ ἐν τῇ Γαλαὰδ ἐκ τῆς φυλῆς Γὰδ, καὶ τὴν Γαυλὼν ἐν τῇ Βασανίτιδι ἐκ τῆς φυλῆς Μανασσῆ.

\vs{9}Αὗται αἱ πόλεις αἱ ἐπίκλητοι τοῖς υἱοῖς Ἰσραὴλ καὶ τῷ προσηλύτῳ τῷ προσκειμένῳ ἐν αὐτοῖς, καταφυγεῖν ἐκεῖ παντὶ παίοντι ψυχὴν ἀκουσίως, ἵνα μὴ ἀποθάνῃ ἐν χειρὶ τοῦ ἀγχιστεύοντος τὸ αἷμα, ἕως ἄν καταστῇ ἔναντι τῆς συναγωγῆς εἰς κρίσιν.

\ch{21}
Καὶ προσήλθοσαν οἱ ἀρχιπατριῶται τῶν υἱῶν Λευὶ πρὸς Ἐλεάζαρ τὸν ἱερέα, καὶ Ἰησοῦν τὸν τοῦ Ναυῆ, καὶ πρὸς τοὺς ἀρχιφύλους πατριῶν ἐκ τῶν φυλῶν Ἰσραήλ·
\vs{2}Καὶ εἶπον πρὸς αὐτοὺς ἐν Σηλὼ ἐν γῇ Χαναὰν, λέγοντες, ἐνετείλατο Κύριος ἐν χειρὶ Μωυσῆ δοῦναι ἡμῖν πόλεις κατοικεῖν, καὶ τὰ περισπόρια τοῖς κτήνεσιν ἡμῶν.
\vs{3}Καὶ ἔδωκαν οἱ υἱοὶ Ἰσραὴλ τοῖς Λευίταις ἐν τῷ κατακληρονομεῖν διὰ προστάγματος Κυρίου τὰς πόλεις καὶ τὰ περισπόρια αὐτῶν.

\vs{4}Καὶ ἐξῆλθεν ὁ κλῆρος τῷ δήμῳ Καάθ· καὶ ἐγένετο τοῖς υἱοῖς Ἀαρὼν τοῖς ἱερεῦσι τοῖς Λευίταις ἀπὸ φυλῆς Ἰούδα καὶ ἀπὸ φυλῆς Συμεὼν καὶ ἀπὸ φυλῆς Βενιαμὶν κληρωτὶ, πόλεις δεκατρεῖς.

\vs{5}Καὶ τοῖς υἱοῖς Καὰθ τοῖς καταλελειμμένοις ἐκ τῆς φυλῆς Ἐφραὶμ καὶ ἐκ τῆς φυλῆς Δὰν καὶ ἀπὸ τοῦ ἡμίσους φυλῆς Μανασσῆ κληρωτὶ, πόλεις δέκα.

\vs{6}Καὶ τοῖς υἱοῖς Γεδσὼν ἀπὸ τῆς φυλῆς Ἰσσάχαρ καὶ ἀπὸ τῆς φυλῆς Ἀσὴρ καὶ ἀπὸ τῆς φυλῆς Νεφθαλὶ καὶ ἀπὸ τοῦ ἡμίσους φυλῆς Μανασσῆ ἐν τῇ Βασὰν, πόλεις δεκατρεῖς.

\vs{7}Καὶ τοῖς υἱοῖς Μεραρὶ κατὰ δήμους αὐτῶν ἀπὸ φυλῆς Ῥουβὴν καὶ ἀπὸ φυλῆς Γὰδ καὶ ἀπὸ φυλῆς Ζαβουλὼν κληρωτὶ, πόλεις δώδεκα.

\vs{8}Καὶ ἔδωκαν οἱ υἱοὶ Ἰσραὴλ τοῖς Λευίταις τὰς πόλεις καὶ τὰ περισπόρια αὐτῶν, ὃν τρόπον ἐνετείλατο Κύριος τῷ Μωυσῇ, κληρωτί.

\vs{9}Καὶ ἔδωκεν ἡ φυλὴ υἱῶν Ἰούδα καὶ ἡ φυλὴ υἱῶν Συμεὼν καὶ ἀπὸ τῆς φυλῆς υἱῶν Βενιαμὶν τὰς πόλεις ταύτας· καὶ ἐπεκλήθησαν
\vs{10}τοῖς υἱοῖς Ἀαρὼν ἀπὸ τοῦ δήμου τοῦ Καὰθ τῶν υἱῶν Λευὶ, ὅτι τούτοις ἐγενήθη ὁ κλῆρος.
\vs{11}Καὶ ἔδωκεν αὐτοῖς τὴν Καριαθαρβὸκ μητρόπολιν τῶν Ἐνάκ· αὕτη ἐστὶ Χεβρὼν ἐν τῷ ὄρει Ἰούδα· τὰ δὲ περισπόρια κύκλῳ αὐτῆς,
\vs{12}καὶ τοὺς ἀγροὺς τῆς πόλεως, καὶ τὰς κώμας αὐτῆς ἔδωκεν Ἰησοῦς τοῖς υἱοῖς Χάλεβ υἱοῦ Ἰεφοννὴ ἐν κατασχέσει.

\vs{13}Καὶ τοῖς υἱοῖς Ἀαρὼν ἔδωκε τὴν πόλιν φυγαδευτήριον τῷ φονεύσαντι, τὴν Χεβρὼν, καὶ τὰ ἀφωρισμένα τὰ σὺν αὐτῇ· καὶ τὴν Λεμνὰ, καὶ τὰ ἀφωρισμένα τὰ πρὸς αὐτῇ·
\vs{14}Καὶ τὴν Αἰλὼμ, καὶ τὰ ἀφωρισμένα αὐτῇ· καὶ τὴν Τεμὰ, καὶ τὰ ἀφωρισμένα αὐτῇ·
\vs{15}Καὶ τὴν Γελλὰ, καὶ τὰ ἀφωρισμένα αὐτῇ· καὶ τὴν Δαβὶρ, καὶ τὰ ἀφωρισμένα αὐτῇ·
\vs{16}Καὶ Ἀσὰ, καὶ τὰ ἀφωρισμένα αὐτῇ· καὶ Τανὺ, καὶ τὰ ἀφωρισμένα αὐτῇ· καὶ Βαιθσαμὺς, καὶ τὰ ἀφωρισμένα αὐτῇ· πόλεις ἐννέα παρὰ τῶν δύο φυλῶν τούτων.
\vs{17}Καὶ παρὰ τῆς φυλῆς Βενιαμὶν, τὴν Γαβαὼν, καὶ τὰ ἀφωρισμένα αὐτῇ· καὶ Γαθὲθ, καὶ τὰ ἀφωρισμένα αὐτῇ·
\vs{18}Καὶ Ἀναθὼθ, καὶ τὰ ἀφωρισμένα αὐτῇ· καὶ Γάμαλα, καὶ τὰ ἀφωρισμένα αὐτῇ· πόλεις τέσσαρες.
\vs{19}Πᾶσαι αἱ πόλεις υἱῶν Ἀαρὼν τῶν ἱερέων, δεκατρεῖς.

\vs{20}Καὶ τοῖς δήμοις υἱοῖς Καὰθ τοῖς Λευίταις τοῖς καταλελειμμένοις ἀπὸ τῶν υἱῶν Καὰθ, καὶ ἐγενήθη ἡ πόλις τῶν ἱερέων αὐτῶν ἀπὸ φυλῆς Ἐφραίμ·
\vs{21}καὶ ἔδωκαν αὐτοῖς τὴν πόλιν τοῦ φυγαδευτηρίου τὴν τοῦ φονεύσαντος, τὴν Συχὲμ, καὶ τὰ ἀφωρισμένα αὐτῇ· καὶ Γάζαρα καὶ τὰ πρὸς αὐτὴν, καὶ τὰ ἀφωρισμένα αὐτῇ·
\vs{22}Καὶ Βαιθωρὼν, καὶ τὰ ἀφωρισμένα τὰ αὐτῇ· πόλεις τέσσαρες.
\vs{23}Καὶ ἐκ τῆς φυλῆς Δὰν, τὴν Ἑλκωθαὶμ, καὶ τὰ ἀφωρισμένα αὐτῇ· καὶ τὴν Γεθεδὰν, καὶ τὰ ἀφωρισμένα αὐτῇ·
\vs{24}Καὶ Αἰλὼν, καὶ τὰ ἀφωρισμένα αὐτῇ· καὶ τὴν Γεθερεμμὼν, καὶ τὰ ἀφωρισμένα αὐτῇ· πόλεις τέσσαρες.
\vs{25}Καὶ ἀπὸ τοῦ ἡμίσους φυλῆς Μανασσῆ, τὴν Τανὰχ, καὶ τὰ ἀφωρισμένα αὐτῇ· καὶ τὴν Ἰεβαθὰ, καὶ τὰ ἀφωρισμένα αὐτῇ· πόλεις δύο.
\vs{26}Πᾶσαι πόλεις δέκα, καὶ τὰ ἀφωρισμένα αὐτῇ τὰ πρὸς αὐταῖς, τοῖς δήμοις υἱῶν Καὰθ τοῖς ὑπολελειμμένοις.

\vs{27}Καὶ τοῖς υἱοῖς Γεδσὼν τοῖς Λευίταις ἐκ τοῦ ἡμίσους φυλῆς Μανασσῆ τὰς πόλεις τὰς ἀφωρισμένας τοῖς φονεύσασι, τὴν Γαυλὼν ἐν τῇ Βασανίτιδι, καὶ τὰ ἀφωρισμένα αὐτῇ· καὶ τὴν Βοσορὰν, καὶ τὰ ἀφωρισμένα αὐτῇ· πόλεις δύο.
\vs{28}Καὶ ἐκ τῆς φυλῆς Ἰσσάχαρ, τὴν Κισὼν, καὶ τὰ ἀφωρισμένα αὐτῇ· καὶ τὴν Δεββὰ, καὶ τὰ ἀφωρισμένα αὐτῇ·
\vs{29}Καὶ τὴν Ῥεμμὰθ, καὶ τὰ ἀφωρισμένα αὐτῇ· καὶ Πηγὴν γραμμάτων, καὶ τὰ ἀφωρισμένα αὐτῇ· πόλεις τέσσαρες.
\vs{30}Καὶ ἐκ τῆς φυλῆς Ἀσὴρ τὴν Βασελλὰν, καὶ τὰ ἀφωρισμένα αὐτῇ· καὶ τὴν Δαββὼν, καὶ τὰ ἀφωρισμένα αὐτῇ·
\vs{31}Καὶ Χελκὰτ, καὶ τὰ ἀφωρισμένα αὐτῇ· καὶ τὴν Ῥαὰβ, καὶ τὰ ἀφωρισμένα αὐτῇ· πόλεις τέσσαρες.
\vs{32}Καὶ ἐκ τῆς φυλῆς Νεφθαλὶ, τὴν πόλιν τὴν ἀφωρισμένην τῷ φονεύσαντι, τὴν Κάδης ἐν τῇ Γαλιλαίᾳ, καὶ τὰ ἀφωρισμένα αὐτῇ· καὶ τὴν Νεμμὰθ, καὶ τὰ ἀφωρισμένα αὐτῇ· καὶ Θεμμὼν, καὶ τὰ ἀφωρισμένα αὐτῇ· πόλεις τρεῖς.
\vs{33}Πᾶσαι αἱ πόλεις τοῦ Γεδσὼν κατὰ δήμους αὐτῶν, πόλεις δεκατρεῖς.

\vs{34}Καὶ τῷ δήμῳ υἱῶν Μεραρὶ τοῖς Λευίταις τοῖς λοιποῖς ἐκ τῆς φυλῆς Ζαβουλὼν, τὴν Μαὰν, καὶ τὰ περισπόρια αὐτῆς· καὶ τὴν Κάδης, καὶ τὰ περισπόρια αὐτῆς·
\vs{35}Καὶ Σελλὰ, καὶ τὰ περισπόρια αὐτῆς· πόλεις τρεῖς.
\vs{36}Καὶ πέραν τοῦ Ἰορδάνου τοῦ κατὰ Ἰεριχὼ ἐκ τῆς φυλῆς Ῥουβὴν, τὴν πόλιν τὸ φυγαδευτήριον τοῦ φονεύσαντος, τὴν Βοσὸρ ἐν τῇ ἐρήμῳ· τὴν Μισὼ, καὶ τὰ περισπόρια αὐτῆς· καὶ τὴν Ἰαζὴρ, καὶ τὰ περισπόρια αὐτῆς·
\vs{37}καὶ τὴν Δεκμὼν, καὶ τὰ περισπόρια αὐτῆς· καὶ τὴν Μαφὰ, καὶ τὰ περισπόρια αὐτῆς· πόλεις τέσσαρες.
\vs{38}Καὶ ἀπὸ τῆς φυλῆς Γὰδ, τὴν πόλιν τὸ φυγαδευτήριον τοῦ φονεύσαντος, καὶ τὴν Ῥαμὼθ ἐν τῇ Γαλαὰδ, καὶ τὰ περισπόρια αὐτῆς· τὴν Καμὶν, καὶ τὰ περισπόρια αὐτῆς·
\vs{39}καὶ τὴν Ἐσβὼν, καὶ τὰ περισπόρια αὐτῆς· καὶ τὴν Ἰαζὴρ, καὶ τὰ περισπόρια αὐτῆς· πᾶσαι αἱ πόλεις τέσσαρες.
\vs{40}Πᾶσαι αἱ πόλεις τοῖς υἱοῖς Μεραρὶ κατὰ δήμους αὐτῶν τῶν καταλελειμμένων ἀπὸ τῆς φυλῆς τῆς Λευί· καὶ ἐγενήθη τὰ ὅρια αἱ πόλεις δεκαδύο.

\vs{41}Πᾶσαι πόλεις τῶν Λευιτῶν ἐν μέσῳ κατασχέσεως υἱῶν Ἰσραὴλ, τεσσαρακονταοκτὼ πόλεις, καὶ τὰ περισπόρια αὐτῶν
\vs{42}κύκλῳ τῶν πόλεων τούτων· πόλις καὶ τὰ περισπόρια κύκλῳ τῆς πόλεως πάσαις ταῖς πόλεσι ταύταις·
\vs{42a}καὶ συνετέλεσεν Ἰησοῦς διαμερίσας τὴν γῆν ἐν τοῖς ὁρίοις αὐτῶν·
\vs{42b}καὶ ἔδωκαν οἱ υἱοὶ Ἰσραὴλ μερίδα τῷ Ἰησοῖ διὰ πρόσταγμα Κυρίου· ἔδωκαν αὐτῷ τὴν πόλιν, ἣν ᾐτήσατο· τὴν Θαμνασαχὰρ ἔδωκαν αὐτῷ ἐν τῷ ὄρει Ἐφραίμ·
\vs{42c}καὶ ᾠκοδόμησεν Ἰησοῦς τὴν πόλιν, καὶ ᾤκησεν ἐν αὐτῇ·
\vs{42d}καὶ ἔλαβεν Ἰησοῦς τὰς μαχαίρας τὰς πετρίνας, ἐν αἷς περιέτεμε τοὺς υἱοὺς Ἰσραὴλ τοὺς γενομένους ἐν τῇ ὁδῷ ἐν τῇ ἐρήμῳ, καὶ ἔθηκεν αὐτὰς ἐν Θαμνασαχάρ.

\vs{43}Καὶ ἔδωκε Κύριος τῷ Ἰσραὴλ πᾶσαν τὴν γῆν, ἣν ὤμοσε δοῦναι τοῖς πατράσιν αὐτῶν· καὶ κατεκληρονόμησαν αὐτὴν, καὶ κατῴκησαν ἐν αὐτῇ.
\vs{44}Καὶ κατέπαυσεν αὐτοὺς Κύριος κυκλόθεν, καθότι ὤμοσε τοῖς πατράσιν αὐτῶν· οὐκ ἀνέστη οὐθεὶς κατενώπιον αὐτῶν ἀπὸ πάντων τῶν ἐχθρῶν αὐτῶν· πάντας τοὺς ἐχθροὺς αὐτῶν παρέδωκε Κύριος εἰς τὰς χεῖρας αὐτῶν.
\vs{45}Οὐ διέπεσεν ἀπὸ πάντων τῶν ῥημάτων τῶν καλῶν, ὧν ἐλάλησε Κύριος τοῖς υἱοῖς Ἰσραὴλ· πάντα παρεγένετο.

\ch{22}
Τότε συνεκάλεσεν Ἰησοῦς τοὺς υἱοὺς Ῥουβὴν, καὶ τοὺς υἱοὺς Γὰδ, καὶ τὸ ἥμισυ φυλῆς Μανασσῆ,
\vs{2}καὶ εἶπεν αὐτοῖς, ὑμεῖς ἀκηκόατε πάντα ὅσα ἐνετείλατο ὑμῖν Μωυσῆς ὁ παῖς Κυρίου, καὶ ὑπηκούσατε τῆς φωνῆς μου κατὰ πάντα ὅσα ἐνετείλατο ὑμῖν.
\vs{3}Οὐκ ἐγκαταλελοίπατε τοὺς ἀδελφοὺς ὑμῶν ταύτας τὰς ἡμέρας πλείους· ἕως τῆς σήμερον ἡυμέρας ἐφυλάξατε τὴν ἐντολὴν Κυρίου τοῦ Θεοῦ ὑμῶν.
\vs{4}Νῦν δὲ κατέπαυσε Κύριος ὁ Θεὸς ἡμῶν τοὺς ἀδελφοὺς ἡμῶν, ὃν τρόπον εἶπεν αὐτοῖς· νῦν οὖν ἀποστραφέντες, ἀπέλθατε εἰς τοὺς οἴκους ὑμῶν, καὶ εἰς τὴν γῆν τῆς κατασχέσεως ὑμῶν, ἣν ἔδωκεν ὑμῖν Μωυσῆς ἐν τῷ πέραν τοῦ Ἰορδάνου.
\vs{5}Ἀλλὰ φυλάξασθε σφόδρα ποιεῖν τὰς ἐντολὰς καὶ τὸν νόμον, ὃν ἐνετείλατο ἡμῖν ποιεῖν Μωυσῆς ὁ παῖς Κυρίου· ἀγαπᾷν Κύριον τὸν Θεὸν ἡμῶν, πορεύεσθαι πάσαις ταῖς ὁδοῖς αὐτοῦ, φυλάξασθαι τὰς ἐντολὰς αὐτοῦ, καὶ προσκεῖσθαι αὐτῷ, καὶ λατρεύειν αὐτῷ ἐξ ὅλης τῆς διανοίας ὑμῶν, καὶ ἐξ ὅλης τῆς ψυχῆς ὑμῶν.
\vs{6}Καὶ εὐλόγησεν αὐτοὺς Ἰησοῦς, καὶ ἐξαπέστειλεν αὐτούς· καὶ ἐπορεύθησαν εἰς τοὺς οἴκους αὐτῶν.

\vs{7}Καὶ τῷ ἡμίσει φυλῆς Μανασσῆ ἔδωκε Μωυσῆς ἐν τῇ Βασανίτιδι, καὶ τῷ ἡμίσει ἔδωκεν Ἰησοῦς μετὰ τῶν ἀδελφῶν αὐτοῦ ἐν τῷ πέραν τοῦ Ἰορδάνου παρὰ θάλασσαν· καὶ ἡνίκα ἐξαπέστειλεν αὐτοὺς Ἰησοῦς εἰς τοὺς οἴκους αὐτῶν καὶ εὐλόγησεν αὐτούς.
\vs{8}Καὶ ἐν χρήμασι πολλοῖς ἀπήλθοσαν εἰς τοὺς οἴκους αὐτῶν· καὶ κτήνη πολλὰ σφόδρα, καὶ ἀργύριον, καὶ χρυσίον, καὶ σίδηρον, καὶ ἱματισμὸν πολὺν, διείλαντο τὴν προνομὴν τῶν ἐχθρῶν μετὰ τῶν ἀδελφῶν αὐτῶν.

\vs{9}Καὶ ἐπορεύθησαν οἱ υἱοὶ Ῥουβὴν, καὶ οἱ υἱοὶ Γὰδ, καὶ τὸ ἥμισυ φυλῆς Μανασσῆ ἀπὸ τῶν υἱῶν Ἰσραὴλ ἐν Σηλὼ ἐν γῇ Χαναὰν ἀπελθεῖν εἰς τὴν Γαλαὰδ εἰς γῆν κατασχέσεως αὐτῶν, ἣν ἐκληρονόμησαν αὐτὴν διὰ προστάγματος Κυρίου ἐν χειρὶ Μωυσῆ.

\vs{10}Καὶ ἦλθον εἰς Γαλαὰδ τοῦ Ἰορδάνου, ἥ ἐστιν ἐν γῇ Χαναάν· καὶ ᾠκοδόμησαν οἱ υἱοὶ Ῥουβὴν, καὶ οἱ υἱοὶ Γὰδ, καὶ τὸ ἥμισυ φυλῆς Μανασσῆ ἐκεῖ βωμὸν ἐπὶ τοῦ Ἰορδάνου, βωμὸν μέγαν τοῦ ἰδεῖν.

\vs{11}Καὶ ἤκουσαν οἱ υἱοὶ Ἰσραὴλ λεγόντων, Ἰδοῦ ᾠκοδομήκασιν οἱ υἱοὶ Ῥουβὴν, καὶ οἱ υἱοὶ Γὰδ, καὶ τὸ ἥμισυ φυλῆς Μανασσῆ βωμὸν ἐφʼ ὁρίων γῆς Χαναὰν ἐπὶ τοῦ Γαλαὰδ τοῦ Ἰορδάνου ἐν τῷ πέραν υἱῶν Ἰσραήλ.
\vs{12}Καὶ συνηθροίσθησαν πάντες οἱ υἱοὶ Ἰσραὴλ εἰς Σηλὼ, ὥστε ἀναβάντες ἐκπολεμῆσαι αὐτούς.

\vs{13}Καὶ ἀπέστειλαν οἱ υἱοὶ Ἰσραὴλ πρὸς τοὺς υἱοὺς Ῥουβὴν καὶ πρὸς τοὺς υἱοὺς Γὰδ καὶ πρὸς τοὺς υἱοὺς ἥμισυ φυλῆς Μανασσῆ εἰς γῆν Γαλαὰδ, τόν τε Φινεὲς υἱὸν Ἐλεάζαρ υἱοῦ Ἀαρὼν τοῦ ἀρχιερέως,
\vs{14}καὶ δέκα τῶν ἀρχόντων μετʼ αὐτοῦ· ἄρχων εἷς ἀπὸ οἴκου πατριᾶς ἀπὸ πασῶν φυλῶν Ἰσραήλ· ἄρχοντες οἴκων πατριῶν εἰσι χιλίαρχοι Ἰσραήλ.
\vs{15}Καὶ παρεγένοντο πρὸς τοὺς υἱοὺς Ῥουβὴν, καὶ πρὸς τοὺς υἱοὺς Γὰδ, καὶ πρὸς τοὺς ἡμίσεις φυλῆς Μανασσῆ εἰς γῆν Γαλαάδ· καὶ ἐλάλησαν πρὸς αὐτοὺς, λέγοντες,
\vs{16}τάδε λέγει πᾶσα ἡ συναγωγὴ Κυρίου, τίς ἡ πλημμέλεια αὕτη, ἣν ἐπλημμελήσατε ἐναντίον τοῦ Θεοῦ Ἰσραὴλ, ἀποστραφῆναι σήμερον ἀπὸ Κυρίου, οἰκοδομήσαντες ὑμῖν ἑαυτοῖς βωμὸν, ἀποστάτας ὑμᾶς γενέσθαι ἀπὸ τοῦ Κυρίου;
\vs{17}Μὴ μικρὸν ἡμῖν τὸ ἁμάρτημα Φογὼρ, ὅτι οὐκ ἐκαθαρίσθημεν ἀπʼ αὐτοῦ ἕως τῆς ἡμέρας ταύτης; καὶ ἐγενήθη πληγὴ ἐν τῇ συναγωγῇ Κυρίου.
\vs{18}Καὶ ὑμεῖς ἀπεστράφητε σήμερον ἀπὸ Κυρίου· καὶ ἔσται ἐὰν ἀποστῆτε σήμερον ἀπὸ Κυρίου, καὶ αὔριον ἐπὶ πάντα Ἰσραὴλ ἔσται ἡ ὀργή.
\vs{19}Καὶ νῦν εἰ μικρὰ ἡ γῆ ὑμῶν τῆς κατασχέσεως ὑμῶν, διάβητε εἰς τὴν γῆν τῆς Κυρίου κατασχέσεως, οὗ κατασκηνοῖ ἐκεῖ ἡ σκηνὴ Κυρίου, καὶ κατακληρονομήσετε ἐν ἡμῖν· καὶ μὴ ἀπὸ Θεοῦ ἀποστάται γενήθητε, καὶ ὑμεῖς μηδʼ ἀπόστητε ἀπὸ Κυρίου, διὰ τὸ οἰκοδομῆσαι ὑμᾶς βωμὸν ἔξω τοῦ θυσιαστηρίου Κυρίου τοῦ Θεοῦ ἡμῶν.
\vs{20}Οὐκ ἰδοὺ Ἄχαρ ὁ τοῦ Ζαρᾶ πλημμελείᾳ ἐπλημμέλησεν ἀπὸ τοῦ ἀναθέματος, καὶ ἐπὶ πᾶσαν συναγωγὴν Ἰσραὴλ ἐγενήθη ὀργή; καὶ οὗτος εἷς αὐτὸς ἀπέθανε τῇ ἑαυτοῦ ἁμαρτίᾳ.

\vs{21}Καὶ ἀπεκρίθησαν οἱ υἱοὶ Ῥουβὴν, καὶ οἱ υἱοὶ Γὰδ, καὶ τὸ ἥμισυ φυλῆς Μανασσῆ, καὶ ἐλάλησαν τοῖς χιλιάρχοις Ἰσραὴλ, λέγοντες,
\vs{22}ὁ Θεὸς Θεὸς Κύριός ἐστι, καὶ ὁ Θεὸς Θεὸς αὐτὸς οἶδε, καὶ Ἰσραὴλ αὐτὸς γνώσεται· εἰ ἐν ἀποστασίᾳ ἐπλημμελήσαμεν ἔναντι τοῦ Κυρίου, μὴ ῥύσαιτο ἡμᾶς ἐν τῇ ἡμέρᾳ ταύτῃ.
\vs{23}Καὶ εἰ ᾠκοδομήσαμεν ἑαυτοῖς βωμὸν, ὥστε ἀποστῆναι ἀπὸ Κυρίου τοῦ Θεοῦ ἡμῶν, ὥστε ἀναβιβάσαι ἐπʼ αὐτὸν θυσίαν ὁλοκαυτωμάτων, ὥστε ποιῆσαι ἐπʼ αὐτοῦ θυσίαν σωτηρίου, Κύριος ἐκζητήσει.

\vs{24}Ἀλλʼ ἕνεκεν εὐλαβείας ῥήματος ἐποιήσαμεν τοῦτο, λέγοντες, ἵνα μὴ εἴπωσιν αὔριον τὰ τέκνα ὑμῶν τοῖς τέκνοις ἡμῶν, τί ὑμῖν Κυρίῳ τῷ Θεῷ Ἰσραήλ;
\vs{25}Καὶ ὅρια ἔθηκε Κύριος ἀναμέσον ἡμῶν καὶ ὑμῶν τὸν Ἰορδάνην, καὶ οὐκ ἔστιν ὑμῖν μερὶς Κυρίου· καὶ ἀπαλλοτριώσουσιν οἱ υἱοὶ ὑμῶν τοὺς υἱῶν ἡμῶν, ἵνα μὴ σέβωνται Κύριον.
\vs{26}Καὶ εἴπαμεν ποιῆσαι οὕτω, τοῦ οἰκοδομῆσαι τὸν βωμὸν τοῦτον οὐχ ἕνεκεν καρπωμάτων οὐδὲ ἕνεκεν θυσιῶν,
\vs{27}ἀλλʼ ἵνα ᾖ τοῦτο μαρτύριον ἀναμέσον ἡμῶν καὶ ὑμῶν, καὶ ἀναμέσον τῶν γενεῶν ἡμῶν μεθʼ ἡμᾶς, τοῦ λατρεύειν λατρείαν Κυρίου ἐναντίον αὐτοῦ, ἐν τοῖς καρπώμασιν ἡμῶν καὶ ἐν ταῖς θυσίαις ἡμῶν καὶ ἐν ταῖς θυσίαις τῶν σωτηρίων ἡμῶν· καὶ οὐκ ἐροῦσι τὰ τέκνα ὑμῶν τοῖς τέκνοις ἡμῶν αὔριον, οὐκ ἔστιν ὑμῖν μερὶς Κυρίου.
\vs{28}Καὶ εἴπαμεν, ἐὰν γένηταί ποτε καὶ λαλήσωσι πρὸς ἡμᾶς, ἢ ταῖς γενεαῖς ἡμῶν αὔριον, καὶ ἐροῦσιν, ἴδετε ὁμοίωμα τοῦ θυσιαστηρίου Κυρίου, ὃ ἐποίησαν οἱ πατέρες ἡμῶν οὐχ ἕνεκεν καρπωμάτων οὐδὲ ἕνεκεν θυσιῶν, ἀλλὰ μαρτύριόν ἐστιν ἀναμέσον ὑμῶν καὶ ἀναμέσον ἡμῶν, καὶ ἀναμέσον τῶν υἱῶν ἡμῶν.
\vs{29}Μὴ γένοιτο οὖν ἡμᾶς ἀποστραφῆναι ἀπὸ Κυρίου ἐν τῇ σήμερον ἡμέρᾳ ἀποστῆναι ἀπὸ Κυρίου, ὥστε οἰκοδομῆσαι ἡμᾶς θυσιαστήριον τοῖς καρπώμασι, καὶ ταῖς θυσίαις Σαλαμὶν, καὶ τῇ θυσίᾳ τοῦ σωτηρίου, πλὴν τοῦ θυσιαστηρίου Κυρίου ὅ ἐστιν ἐναντίον τῆς σκηνῆς αὐτοῦ.

\vs{30}Καὶ ἀκούσας Φινεὲς ὁ ἱερεὺς καὶ πάντες οἱ ἄρχοντες τῆς συναγωγῆς Ἰσραὴλ οἳ ἦσαν μετʼ αὐτοῦ, τοὺς λόγους οὓς ἐλάλησαν οἱ υἱοὶ Ῥουβὴν καὶ οἱ υἱοὶ Γὰδ καὶ τὸ ἥμισυ φυλῆς Μανασσῆ, καὶ ἤρεσεν αὐτοῖς.
\vs{31}Καὶ εἶπε Φινεὲς ὁ ἱερεὺς τοῖς υἱοῖς Ῥουβὴν καὶ τοῖς υἱοῖς Γὰδ καὶ τῷ ἡμίσει φυλῆς Μανασσῆ, σήμερον ἐγνώκαμεν ὅτι μεθʼ ἡμῶν Κύριος, διότι οὐκ ἐπλημμελήσατε ἐναντίον Κυρίου πλημμέλειαν, καὶ ὅτι ἐῤῥύσασθε τοὺς υἱοὺς Ἰσραὴλ ἐκ χειρὸς Κυρίου.
\vs{32}Καὶ ἀπέστρεψε Φινεὲς ὁ ἱερεὺς καὶ οἱ ἄρχοντες ἀπὸ τῶν υἱῶν Ῥουβὴν καὶ ἀπὸ τῶν υἱῶν Γὰδ καὶ ἀπὸ τοῦ ἡμίσους φυλῆς Μανασσῆ ἐκ τῆς Γαλαὰδ εἰς γῆν Χαναὰν πρὸς τοὺς υἱοὺς Ἰσραήλ· καὶ ἀπεκρίθησαν αὐτοῖς τοὺς λόγους.
\vs{33}Καὶ ἤρεσε τοῖς υἱοῖς Ἰσραήλ· καὶ ἐλάλησαν πρὸς τοὺς υἱοὺς Ἰσραὴλ, καὶ εὐλόγησαν τὸν Θεὸν υἱῶν Ἰσραήλ· καὶ εἶπαν μηκέτι ἀναβῆναι πρὸς αὐτοὺς εἰς πόλεμον ἐξολοθρεῦσαι τὴν γῆν τῶν υἱῶν Ῥουβὴν καὶ τῶν υἱων Γὰδ καὶ τοῦ ἡμίσους φυλῆς Μανασσῆ· καὶ κατῴκησαν ἐπʼ αὐτῆς.

\vs{34}Καὶ ἐπωνόμασεν Ἰησοῦς τὸν βωμὸν τῶν Ῥουβὴν καὶ τῶν Γὰδ καὶ τοῦ ἡμίσους φυλῆς Μανασσῆ, καὶ εἶπεν, ὅτι μαρτύριόν ἐστιν ἀναμέσον αὐτῶν, ὅτι Κύριος ὁ Θεὸς αὐτῶν ἐστι.

\ch{23}
Καὶ ἐγένετο μεθʼ ἡμέρας πλείους μετὰ τὸ καταπαῦσαι Κύριον τὸν Ἰσραὴλ ἀπὸ πάντων τῶν ἐχθρῶν αὐτοῦ κυκλόθεν, καὶ Ἰησοῦς πρεσβύτερος προβεβηκὼς ταῖς ἡμέραις.
\vs{2}Καὶ συνεκάλεσεν Ἰησοῦς πάντας τοὺς υἱοὺς Ἰσραὴλ καὶ τὴν γερουσίαν αὐτῶν καὶ τοὺς ἄρχοντας αὐτῶν καὶ τοὺς δικαστὰς αὐτῶν καὶ τοὺς γραμματεῖς αὐτῶν, καὶ εἶπε πρὸς αὐτοὺς, ἐγὼ γεγήρακα καὶ προβέβηκα ταῖς ἡμέραις·
\vs{3}Ὑμεῖς δὲ ἑωράκατε ὅσα ἐποίησε Κύριος ὁ Θεὸς ἡμῶν πᾶσι τοῖς ἔθνεσι τούτοις ἀπὸ προσώπου ἡμῶν, ὅτι Κύριος ὁ Θεὸς ὑμῶν ὁ ἐκπολεμήσας ὑμῖν.
\vs{4}Ἴδετε ὅτι ἐπέῤῥιφα ὑμῖν τὰ ἔθνη τὰ καταλελειμμένα ὑμῖν ταῦτα ἐν τοῖς κλήροις εἰς τὰς φυλὰς ὑμῶν, ἀπὸ τοῦ Ἰορδάνου πάντα τὰ ἔθνη, καὶ ἐξωλόθρευσα, καὶ ἀπὸ τῆς θαλάσσης τῆς μεγάλης ὁριεῖ ἐπὶ δυσμὰς ἡλίου.

\vs{5}Κύριος δὲ ὁ Θεὸς ἡμῶν οὗτος ἐξολοθρεύσει αὐτοὺς ἀπὸ προσώπου ἡμῶν, ἕως ἂν ἀπόλωνται· καὶ ἀποστελεῖ αὐτοῖς τὰ θηρία τὰ ἄγρια, ἕως ἂν ἐξολοθρεύσῃ αὐτοὺς καὶ τοὺς βασιλεῖς αὐτῶν ἀπὸ προσώπου ὑμῶν, καὶ κατακληρονομήσετε τὴν γῆν αὐτῶν, καθὰ ἐλάλησε Κύριος ὁ Θεὸς ἡμῶν ὑμῖν.
\vs{6}Κατισχύσατε οὖν σφόδρα φυλάσσειν καὶ ποιεῖν πάντα τὰ γεγραμμένα ἐν τῷ βιβλίῳ τοῦ νόμου Μωυσῆ, ἵνα μὴ ἐκκλίνητε εἰς δεξιὰ ἢ εὐώνυμα,
\vs{7}ὅπως μὴ εἰσέλθητε εἰς τὰ ἔθνη τὰ καταλελειμμένα ταῦτα· καὶ τὰ ὀνόματα τῶν θεῶν αὐτῶν οὐκ ὀνομασθήσεται ἐν ὑμῖν, οὐδὲ μὴ λατρεύσητε, οὐδὲ μὴ προσκυνήσητε αὐτοῖς,
\vs{8}ἀλλὰ Κυρίῳ τῷ Θεῷ ἡμῶν προσκολληθήσεσθε, καθάπερ ἐποιήσατε ἕως τῆς ἡμέρας ταύτης.
\vs{9}Καὶ ἐξολοθρεύσει αὐτοὺς Κύριος ἀπὸ προσώπου ὑμῶν ἔθνη μεγάλα καὶ ἰσχυρά· καὶ οὐδεὶς ἀντέστη κατενώπιον ἡμῶν ἕως τῆς ἡμέρας ταύτης.
\vs{10}Εἷς ὑμῶν ἐδίωξε χιλίους, ὅτι Κύριος ὁ Θεὸς ἡμῶν οὗτος ἐξεπολέμει ὑμῖν, καθάπερ εἶπεν ἡμῖν.

\vs{11}Καὶ φυλάξασθε σφόδρα τοῦ ἀγαπᾷν Κύριον τὸν Θεὸν ἡμῶν.
\vs{12}Ἐὰν γὰρ ἀποστραφῆτε καὶ προσθῆσθε τοῖς ὑπολειφθεῖσιν ἔθνεσι τούτοις τοῖς μεθʼ ὑμῶν, καὶ ἐπιγαμίας ποιήσητε πρὸς αὐτοὺς, καὶ συγκαταμιγῆτε αὐτοῖς καὶ αὐτοὶ ὑμῖν,
\vs{13}γινώσκετε ὅτι οὐ μὴ προσθῇ Κύριος τοῦ ἐξολοθρεῦσαι τὰ ἔθνη ταῦτα ἀπὸ προσώπου ὑμῶν· καὶ ἔσονται ὑμῖν εἰς παγίδας, καὶ εἰς σκάνδαλα, καὶ εἰς ἥλους ἐν ταῖς πτέρναις ὑμῶν, καὶ εἰς βολίδας ἐν τοῖς ὀφθαλμοῖς ὑμῶν, ἕως ἂν ἀπόλησθε ἀπὸ τῆς γῆς τῆς ἀγαθῆς ταύτης, ἣν ἔδωκεν ὑμῖν Κύριος ὁ Θεὸς ὑμῶν.

\vs{14}Ἐγὼ δὲ ἀποτρέχω τὴν ὁδὸν, καθὰ καὶ πάντες οἱ ἐπὶ τῆς γῆς· καὶ γνώσεσθε τῇ καρδίᾳ ὑμῶν καὶ τῇ ψυχῇ ὑμῶν, διότι οὐκ ἔπεσεν εἷς λόγος ἀπὸ πάντων τῶν λόγων, ὧν εἶπε Κύριος ὁ Θεὸς ἡμῶν πρὸς πάντα τὰ ἀνήκοντα ἡμῖν, οὐ διεφώνησεν ἐξ αὐτῶν.
\vs{15}Καὶ ἔσται ὃν τρόπον ἥκει πρὸς ἡμᾶς πάντα τὰ ῥήματα τὰ καλὰ, ἃ ἐλάλησε Κύριος ἐφʼ ὑμᾶς· οὕτως ἐπάξει Κύριος ὁ Θεὸς ἐφʼ ὑμᾶς πάντα τὰ ῥήματα τὰ πονηρὰ ἕως ἂν ἐξολοθρεύσῃ ὑμᾶς ἀπὸ τῆς γῆς τῆς ἀγαθῆς ταύτης, ἧς ἔδωκε Κύριος ὑμῖν,
\vs{16}ἐν τῷ παραβῆναι ὑμᾶς τὴν διαθήκην Κυρίου τοῦ Θεοῦ ἡμῶν, ἣν ἐνετείλατο ἡμῖν, καὶ πορευθέντες λατρεύσητε θεοῖς ἑτέροις καὶ προσκυνήσητε αὐτοῖς.

\ch{24}
Καὶ συνήγαγεν Ἰησοῦς πάσας φυλὰς Ἰσραὴλ εἰς Σηλὼ, καὶ συνεκάλεσε τοὺς πρεσβυτέρους αὐτῶν καὶ τοὺς γραμματεῖς αὐτῶν καὶ τοὺς δικαστὰς αὐτῶν, καὶ ἔστησεν αὐτοὺς ἀπέναντι τοῦ Θεοῦ.

\vs{2}Καὶ εἶπεν Ἰησοῦς πρὸς πάντα τὸν λαὸν, τάδε λέγει Κύριος ὁ Θεὸς Ἰσραὴλ, πέραν τοῦ ποταμοῦ παρῴκησαν οἱ πατέρες ὑμῶν τὸ ἀπαρχῆς, Θάρα ὁ πατὴρ Ἁβραὰμ, καὶ ὁ πατὴρ Ναχὼρ, καὶ ἐλάτρευσαν θεοῖς ἑτέροις.
\vs{3}Καὶ ἔλαβον τὸν πατέρα ὑμῶν τὸν Ἁβραὰμ ἐκ τοῦ πέραν τοῦ ποταμοῦ, καὶ ὡδήγησα αὐτὸν ἐν πάσῃ τῇ γῇ· καὶ ἐπλήθυνα αὐτοῦ σπέρμα, καὶ ἔδωκα αὐτῷ τὸν Ἰσαὰκ,
\vs{4}καὶ τῷ Ἰσαὰκ τὸν Ἰακὼβ καὶ τὸν Ἡσαῦ· καὶ ἔδωκα τῷ Ἡσαῦ τὸ ὄρος τὸ Σηεὶρ κληρονομῆσαι αὐτῷ· καὶ Ἰακὼβ καὶ οἱ υἱοὶ αὐτοῦ κατέβησαν εἰς Αἴγυπτον, καὶ ἐγένοντο ἐκεῖ εἰς ἔθνος μέγα καὶ πολὺ καὶ κραταιόν· καὶ ἐκάκωσαν αὐτοὺς οἱ Αἰγύπτιοι.
\vs{5}Καὶ ἐπάταξα τὴν Αἴγυπτον ἐν σημείοις οἷς ἐποίησα ἐν αὐτοῖς.
\vs{6}Καὶ μετὰ ταῦτα ἐξήγαγε τοὺς πατέρας ἡμῶν ἐξ Αἰγύπτου, καὶ εἰσήλθατε εἰς τὴν θάλασσαν τὴν ἐρυθράν· καὶ κατεδίωξαν οἱ Αἰγύπτιοι ὀπίσω τῶν πατέρων ἡμῶν ἐν ἅρμασι καὶ ἐν ἵπποις εἰς τὴν θάλασσαν τὴν ἐρυθράν.
\vs{7}Καὶ ἀνεβοήσαμεν πρὸς Κύριον· καὶ ἔδωκε νεφέλην καὶ γνόφον ἀναμέσον ἡμῶν καὶ ἀναμέσον τῶν Αἰγυπτίων, καὶ ἐπήγαγεν ἐπʼ αὐτοὺς τὴν θάλασσαν, καὶ ἐκάλυψεν αὐτούς· καὶ εἴδοσαν οἱ ὀφθαλμοὶ ὑμῶν ὅσα ἐποίησε Κύριος ἐν γῇ· Αἰγύπτῳ· καὶ ἦτε ἐν τῇ ἐρήμῳ ἡμέρας πλείους.

\vs{8}Καὶ ἤγαγεν ἡμᾶς εἰς γῆν Ἀμοῤῥαίων τῶν κατοικούντων πέραν τοῦ Ἰορδάνου, καὶ παρέδωκεν αὐτοὺς Κύριος εἰς τὰς χεῖρας ἡμῶν· καὶ κατεκληρονομήσατε τὴν γῆν αὐτῶν, καὶ ἐξωλοθρεύσατε αὐτοὺς ἀπὸ προσώπου ὑμῶν.

\vs{9}Καὶ ἀνέστη Βαλὰκ ὁ τοῦ Σεπφὼρ βασιλεὺς Μωὰβ, καὶ παρετάξατο τῷ Ἰσραὴλ, καὶ ἀποστείλας ἐκάλεσε τὸν Βαλαὰμ ἀράσασθαι ἡμῖν.
\vs{10}Καὶ οὐκ ἠθέλησε Κύριος ὁ Θεός σου ἀπολέσαι σε· καὶ εὐλογίαις εὐλόγησεν ἡμᾶς, καὶ ἐξείλατο ἡμᾶς ἐκ χειρῶν αὐτῶν, καὶ παρέδωκεν αὐτούς.
\vs{11}Καὶ διέβητε τὸν Ἰορδάνην, καὶ παρεγενήθητε εἰς Ἱεριχώ· καὶ ἐπολέμησαν πρὸς ἡμᾶς οἱ κατοικοῦντες Ἱεριχὼ ὁ Ἀμοῤῥαῖος, καὶ ὁ Χαναναῖος, καὶ ὁ Φερεζαῖος, καὶ ὁ Εὑαῖος, καὶ ὁ Ἰεβουσαῖος, καὶ ὁ Χετταῖος, καὶ ὁ Γεργεσαῖος· καὶ παρέδωκεν αὐτοὺς Κύριος εἰς τὰς χεῖρας ἡμῶν.
\vs{12}Καὶ ἐξαπέστειλε προτέραν ὑμῶν τὴν σφηκίαν· καὶ ἐξαπέστειλεν αὐτοὺς ἀπὸ προσώπου ἡμῶν δώδεκα βασιλεῖς τῶν Ἀμοῤῥαίων, οὐκ ἐν τῇ ῥομφαίᾳ σου οὐδὲ ἐν τῷ τόξῳ σου.

\vs{13}Καὶ ἔδωκεν ὑμῖν γῆν ἐφʼ ἣν οὐκ ἐκοπιάσατε ἐπʼ αὐτῆς, καὶ πόλεις ἃς οὐκ ᾠκοδομήκατε, καὶ κατῳκίσθητε ἐν αὐταῖς, καὶ ἀμπελῶνας καὶ ἐλαιῶνας οὓς οὐκ ἐφυτεύσατε ὑμεῖς, ἔδεσθε.

\vs{14}Καὶ νῦν φοβήθητε Κύριον, καὶ λατρεύσατε αὐτῷ ἐν εὐθύτητι καὶ ἐν δικαιοσύνῃ, καὶ περιέλεσθε τοὺς θεοὺς τοὺς ἀλλοτρίους, οἷς ἐλάτρευσαν οἱ πατέρες ἡμῶν ἐν τῷ πέραν τοῦ ποταμοῦ καὶ ἐν Αἰγύπτῳ, καὶ λατρεύσατε Κυρίῳ.
\vs{15}Εἰ δὲ μὴ ἀρέσκει ὑμῖν λατρεύειν Κυρίῳ, ἐκλέξασθε ὑμῖν αὐτοῖς σήμερον τίνι λατρεύσητε, εἴτε τοῖς θεοῖς τῶν πατέρων ὑμῶν, τοῖς ἐν τῷ πέραν τοῦ ποταμοῦ, εἴτε τοῖς θεοῖς τῶν Ἀμοῤραίων, ἐν οἷς ὑμεῖς κατοικεῖτε ἐπὶ τῆς γῆς αὐτῶν· ἐγὼ δὲ καὶ ἡ οἰκία μου λατρεύσομεν Κυρίῳ, ὅτι ἅγιός ἐστι.

\vs{16}Καὶ ἀποκριθεὶς ὁ λαὸς εἶπε, μὴ γένοιτο ἡμῖν καταλιπεῖν Κύριον, ὥστε λατρεύειν θεοῖς ἑτέροις.
\vs{17}Κύριος ὁ Θεὸς ἡμῶν, αὐτὸς Θεός ἐστιν· αὐτὸς ἀνήγαγεν ἡμᾶς καὶ τοὺς πατέρας ἡμῶν ἐξ Αἰγύπτου, καἰ διεφύλαξεν ἡμᾶς ἐν πάσῃ τῇ ὁδῷ ᾗ ἐπορεύθημεν ἐν αὐτῇ, καὶ ἐν πᾶσι τοῖς ἔθνεσιν οὓς παρήλθομεν διʼ αὐτῶν.
\vs{18}Καὶ ἐξέβαλε Κύριος τὸν Ἀμοῤῥαῖον καὶ πάντα τὰ ἔθνη τὰ κατοικοῦντα τὴν γῆν ἀπὸ προσώπου ἡμῶν· ἀλλὰ καὶ ἡμεῖς λατρεύσομεν Κυρίῳ, οὗτος γὰρ Θεὸς ἡμῶν ἐστι.

\vs{19}Καὶ εἶπεν Ἰησοῦς πρὸς τὸν λαὸν, οὐ μὴ δύνησθε λατρεύειν Κυρίῳ, ὅτι Θεὸς ἅγιός ἐστι· καὶ ζηλώσας οὗτος οὐκ ἀνήσει τὰ ἁμαρτήματα ὑμῶν καὶ τὰ ἀνομήματα ὑμῶν,
\vs{20}ἡνίκα ἂν ἐγκαταλίπητε Κύριον καὶ λατρεύσητε θεοῖς ἑτέροις· καὶ ἐπελθὼν κακώσει ὑμᾶς καὶ ἐξαναλώσει ὑμᾶς ἀνθʼ ὧν εὖ ἐποίησεν ὑμᾶς.
\vs{21}Καὶ εἶπεν ὁ λαὸς πρὸς Ἰησοῦν, οὐχὶ, ἀλλὰ Κυρίῳ λατρεύσομεν.

\vs{22}Καὶ εἶπεν Ἰησοῦς πρὸς τὸν λαὸν, μάρτυρες ὑμεῖς καθʼ ὑμῶν, ὅτι ὑμεῖς ἐξελέξασθε Κυρίῳ λατρεύειν αὐτῷ.
\vs{23}Καὶ νῦν περιέλεσθε τοὺς θεοὺς τοὺς ἀλλοτρίους τοὺς ἐν ὑμῖν, καὶ εὐθύνατε τὴν καρδίαν ὑμῶν πρὸς Κύριον Θεὸν Ἰσραήλ.
\vs{24}Καὶ εἶπεν ὁ λαὸς πρὸς Ἰησοῦν, Κυρίῳ λατρεύσομεν καὶ τῆς φωνῆς αὐτοῦ ἀκουσόμεθα.

\vs{25}Καὶ διέθετο Ἰησοῦς διαθήκην πρὸς τὸν λαὸν ἐν τῇ ἡμέρᾳ ἐκείνῃ, καὶ ἔδωκεν αὐτῷ νόμον καὶ κρίσιν ἐν Σηλὼ ἐνώπιον τῆς σκηνῆς τοῦ Θεοῦ Ἰσραήλ.
\vs{26}Καὶ ἔγραψε τὰ ῥήματα ταῦτα εἰς βιβλίον νόμων τοῦ Θεοῦ· καὶ ἔλαβε λίθον μέγαν, καὶ ἔστησεν αὐτὸν Ἰησοῦς ὑπὸ τὴν τέρμινθον ἀπέναντι Κυρίου.
\vs{27}Καὶ εἶπεν Ἰησοῦς πρὸς τὸν λαὸν, ἰδοὺ ὁ λίθος οὗτος ἔσται ἐν ὑμῖν εἰς μαρτύριον, ὅτι αὐτὸς ἀκήκοε πάντα τὰ λεχθέντα αὐτῷ ὑπὸ Κυρίου· ὅτι ἐλάλησε πρὸς ὑμᾶς σήμερον, καὶ οὗτος ἔσται ἐν ὑμῖν εἰς μαρτύριον ἐπʼ ἐσχάτων τῶν ἡμερῶν, ἡνίκα ἂν ψεύσησθε Κυρίῳ τῷ Θεῷ μου.
\vs{28}Καὶ ἀπέστειλεν Ἰησοῦς τὸν λαὸν, καὶ ἐπορεύθησαν ἕκαστος εἰς τὸν τόπον αὐτοῦ.
\vs{29}Καὶ ἐλάτρευσεν Ἰσραὴλ τῷ Κυρίῳ πάσας τὰς ἡμέρας Ἰησοῦ, καὶ πάσας τὰς ἡμέρας τῶν πρεσβυτέρων ὅσοι ἐφείλκυσαν τὸν χρόνον μετὰ Ἰησοῦ, καὶ ὅσοι εἴδοσαν πάντα τὰ ἔργα Κυρίου ὅσα ἐποίησε τῷ Ἰσραήλ.

\vs{30}Καὶ ἐγένετο μετʼ ἐκεῖνα καὶ ἀπέθανεν Ἰησοῦς υἱὸς Ναυῆ δοῦλος Κυρίου ἐκατὸν δέκα ἐτῶν.
\vs{31}Καὶ ἔθαψαν αὐτὸν πρὸς τοῖς ὁρίοις τοῦ κλήρου αὐτοῦ ἐν Θαμνασαρὰχ ἐν τῷ ὄρει τῷ Ἐφραὶμ ἀπὸ βοῤῥᾶ τοῦ ὄρους τοῦ Γαλαάδ·
\vs{31a}ἐκεῖ ἔθηκαν μετʼ αὐτοῦ εἰς τὸ μνῆμα εἰς ὃ ἔθαψαν αὐτὸν ἐκεῖ τὰς μαχαίρας τὰς πετρίνας, ἐν αἷς περιέτεμε τοὺς υἱοὺς Ἰσραὴλ ἐν Γαλγάλοις, ὅτε ἐξήγαγεν αὐτοὺς ἐξ Αἰγύπτου, καθὰ συνέταξεν αὐτοῖς Κύριος· καὶ ἐκεῖ εἰσιν ἕως τῆς σήμερον ἡμέρας.

\vs{32}Καὶ τὰ ὀστᾶ Ἰωσὴφ ἀνήγαγον οἱ υἱοὶ Ἰσραὴλ ἐξ Αἰγύπτου, καὶ κατώρυξαν ἐν Σικίμοις, ἐν τῇ μερίδι τοῦ ἀγροῦ οὗ ἐκτήσατο Ἰακὼβ παρὰ τῶν Ἀμοῤῥαίων τῶν κατοικούντων ἐν Σικίμοις ἀμνάδων ἑκατὸν, καὶ ἔδωκεν αὐτὴν Ἰωσὴφ ἐν μερίδι.

\vs{33}Καὶ ἐγένετο μετὰ ταῦτα καὶ Ἐλεάζαρ υἱὸς Ἀαρὼν ὁ ἀρχιερεὺς ἐτελεύτησε, καὶ ἐτάφη ἐν Γαβαὰρ Φινεὲς τοῦ υἱοῦ αὐτοῦ, ἣν ἔδωκεν αὐτῷ ἐν τῷ ὄρει τῷ Ἐφραίμ.

\vs{33a}Ἐν ἐκείνῃ τῇ ἡμέρᾳ λαβόντες οἱ υἱοὶ Ἰσραὴλ τὴν κιβωτὸν τοῦ Θεοῦ, περιεφέροσαν ἐν ἑαυτοῖς· καὶ Φινεὲς ἱεράτευσεν ἀντὶ Ἐλεάζαρ τοῦ πατρὸς αὐτοῦ ἕως ἀπέθανε, καὶ κατωρύγη ἐν Γαβαὰρ τῇ ἑαυτοῦ·
\vs{33b}οἱ δὲ υἱοὶ Ἰσραὴλ ἀπήλθοσαν ἕκαστος εἰς τὸν τόπον αὐτῶν, καὶ εἰς τὴν ἑαυτῶν πόλιν· καὶ ἐσέβοντο οἱ υἱοὶ Ἰσραὴλ τὴν Ἀστάρτην, καὶ Ἀσταρὼθ, καὶ τοὺς θεοὺς τῶν ἐθνῶν τῶν κύκλῳ αὐτῶν· καὶ παρέδωκεν αὐτοὺς Κύριος εἰς χεῖρας Ἐγλὼμ τῷ βασιλεῖ Μωὰβ, καὶ ἐκυρίευσεν αὐτῶν ἔτη δεκαοκτώ.


\def\book{ΚΡΙΤΑΙ}
\biblebook{ΚΡΙΤΑΙ}


\lettrine[lines=2, loversize=0.2, nindent=0em, findent=.25em]{\textcolor{bookheadingcolor}{Κ}}{ΑΙ} ἐγένετο μετὰ τὴν τελευτὴν Ἰησοῦ, καὶ ἐπηρώτων οἱ υἱοὶ Ἰσραὴλ διὰ τοῦ Κυρίου, λέγοντες, τίς ἀναβήσεται ἡμῖν πρὸς τοὺς Χαναναίους ἀφηγούμενος τοῦ πολεμῆσαι πρὸς αὐτούς;
\vs{2}Καὶ εἶπε Κύριος, Ἰούδας ἀναβήσεται· ἰδοὺ δέδωκα τὴν γῆν ἐν χειρὶ αὐτοῦ.
\vs{3}Καὶ εἶπεν Ἰούδας τῷ Συμεὼν ἀδελφῷ αὐτοῦ, ἀνάβηθι μετʼ ἐμοῦ ἐν τῷ κλήρῳ μου, καὶ παραταξώμεθα πρὸς τοὺς Χαναναίους, καὶ πορεύσομαι κᾀγὼ μετὰ σοῦ ἐν τῷ κλήρῳ σου· καὶ ἐπορεύθη μετʼ αὐτοῦ Συμεών.
\vs{4}Καὶ ἀνέβη Ἰούδας· καὶ παρέδωκε Κύριος τὸν Χαναναῖον καὶ τὸν Φερεζαῖον εἰς τὰς χεῖρας αὐτῶν· καὶ ἔκοψαν αὐτοὺς ἐν Βεζὲκ εἰς δέκα χιλιάδας ἀνδρῶν·
\vs{5}Καὶ κατέλαβον τὸν Ἀδωνιβεζὲκ ἐν τῇ Βεζὲκ, καὶ παρετάξαντο πρὸς αὐτόν· καὶ ἔκοψαν τὸν Χαναναῖον καὶ Φερεζαῖον.
\vs{6}Καὶ ἔφυγεν Ἀδωνιβεζέκ· καὶ κατέδραμον ὀπίσω αὐτοῦ, καὶ ἐλάβοσαν αὐτὸν, καὶ ἀπέκοψαν τὰ ἄκρα τῶν χειρῶν αὐτοῦ καὶ τὰ ἄκρα τῶν ποδῶν αὐτοῦ.
\vs{7}Καὶ εἶπεν Ἀδωνιβεζὲκ, ἑβδομήκοντα βασιλεῖς, τὰ ἄκρα τῶν χειρῶν αὐτῶν καὶ τὰ ἄκρα τῶν ποδῶν αὐτῶν ἀποκεκομμένοι, ἦσαν συλλέγοντες τὰ ὑποκάτω τῆς τραπέζης μου· καθὼς οὖν ἐποίησα, οὕτως ἀνταπέδωκέ μοι ὁ Θεός. καὶ ἄγουσιν αὐτὸν εἰς Ἱερουσαλήμ, καὶ ἀπέθανεν ἐκεῖ.

\vs{8}Καὶ ἐπολέμουν υἱοὶ Ἰούδα τὴν Ἱερουσαλὴμ, καὶ κατελάβοντο αὐτὴν, καὶ ἐπάταξαν αὐτὴν ἐν στόματι ῥομφαίας, καὶ τὴν πόλιν ἐνέπρησαν ἐν πυρί.
\vs{9}Καὶ μετὰ ταῦτα κατέβησαν οἱ υἱοὶ Ἰούδα πολεμῆσαι πρὸς τὸν Χαναναῖον τὸν κατοικοῦντα τὴν ὀρεινὴν καὶ τὸν Νότον καὶ τὴν πεδινήν.
\vs{10}Καὶ ἐπορεύθη Ἰούδας πρὸς τὸν Χαναναῖον τὸν κατοικοῦντα ἐν Χεβρών· καὶ ἐξῆλθε Χεβρὼν ἐξ ἐναντίας· καὶ τὸ ὄνομα ἦν Χεβρὼν τὸ πρότερον Καριαθαρβοκσεφέρ· καὶ ἐπάτάξαν τὸν Σεσσὶ καὶ Ἀχιμὰν καὶ Θολμὶ γεννήματα τοῦ Ἐνάκ.
\vs{11}Καὶ ἀνέβησαν ἐκεῖθεν πρὸς τοὺς κατοικοῦντας Δαβίρ· τὸ δὲ ὄνομα τῆς Δαβὶρ ἦν ἔμπροσθεν Καριαθσεφὲρ, πόλις Γραμμάτων.

\vs{12}Καὶ εἶπε Χάλεβ, ὃς ἂν πατάξῃ τὴν πόλιν τῶν Γραμμάτων καὶ προκαταλάβηται αὐτὴν, δώσω αὐτῷ τὴν Ἀσχὰ θυγατέρα μου εἰς γυναῖκα.
\vs{13}Καὶ προκατελάβετο αὐτὴν Γοθονιὴλ υἱὸς Κενὲζ ἀδελφοῦ Χάλεβ ὁ νεώτερος· καὶ ἔδωκεν αὐτῷ Χάλεβ τὴν Ἀσχὰ θυγατέρα αὐτοῦ εἰς γυναῖκα.
\vs{14}Καὶ ἐγένετο ἐν τῇ εἰσόδῳ αὐτῆς, καὶ ἐπέσεισεν αὐτὴν Γοθονιὴλ τοῦ αἰτῆσαι παρὰ τοῦ πατρὸς αὐτῆς ἀγρόν· καὶ ἐγόγγυζε καὶ ἔκραζεν ἀπὸ τοῦ ὑποζυγίου, εἰς γῆν Νότου ἐκδέδοσαί με· καὶ εἶπεν αὐτῇ Χάλεβ, τί ἐστί σοι;
\vs{15}Καὶ εἶπεν αὐτῷ Ἀσχὰ, δὸς δή μοι εὐλογίαν, ὅτι εἰς γῆν Νότου ἐκδέδοσαί με, καὶ δώσεις μοι λύτρωσιν ὕδατος· καὶ ἔδωκεν αὐτῇ Χάλεβ κατὰ τὴν καρδίαν αὐτῆς λύτρωσιν μετεώρων καὶ λύτρωσιν ταπεινῶν.

\vs{16}Καὶ οἱ υἱοὶ Ἰοθὸρ τοῦ Κιναίου τοῦ γαμβροῦ Μωυσῆ ἀνέβησαν ἐκ πόλεως τῶν φοινίκων μετὰ τῶν υἱῶν Ἰούδα εἰς τὴν ἔρημον τὴν οὖσαν ἐν τῷ Νότῳ Ἰούδα, ἥ ἐστιν ἐπὶ καταβάσεως Ἀρὰδ, καὶ κατῴκησαν μετὰ τοῦ λαοῦ.

\vs{17}Καὶ ἐπορεύθη Ἰούδας μετὰ Συμεὼν τοῦ ἀδελφοῦ αὐτοῦ, καὶ ἔκοψε τὸν Χαναναῖον τὸν κατοικοῦντα Σεφὲθ, καὶ ἐξωλόθρευσαν αὐτους· καὶ ἐκάλεσε τὸ ὄνομα τῆς πόλεως, Ἀνάθεμα.
\vs{18}Καὶ οὐκ ἐκληρονόμησεν Ἰούδας τὴν Γάζαν οὐδὲ τὰ ὅρια αὐτῆς, οὐδὲ τὴν Ἀσκάλωνα οὐδὲ τὰ ὅρια αὐτῆς, καὶ τὴν Ἀκκαρὼν οὐδὲ τὰ ὅρια αὐτῆς, τὴν Ἄζωτον οὐδὲ τὰ περισπόρια αὐτῆς.
\vs{19}Καὶ ἦν Κύριος μετὰ Ἰούδα· καὶ ἐκληρονόμησε τὸ ὄρος, ὅτι οὐκ ἠδυνάσθησαν ἐξολοθρεῦσαι τοὺς κατοικοῦντας τὴν κοιλάδα, ὅτι Ῥηχὰβ διεστείλατο αὐτοῖς.
\vs{20}Καὶ ἔδωκαν τῷ Χάλεβ τὴν Χεβρὼν, καθὼς ἐλάλησε Μωυσῆς· καὶ ἐκληρονόμησεν ἐκεῖθεν τὰς τρεῖς πόλεις τῶν υἱῶν Ἐνάκ.

\vs{21}Καὶ τὸν Ἰεβουσαῖον τὸν κατοικοῦντα ἐν Ἱερουσαλὴμ οὐκ ἐκληρονόμησαν οἱ υἱοὶ Βενιαμίν· καὶ κατῴκησεν ὁ Ἰεβουσαῖος μετὰ τῶν υἱῶν Βενιαμὶν ἐν Ἱερουσαλὴμ ἕως τῆς ἡμέρας ταύτης.

\vs{22}Καὶ ἀνέβησαν οἱ υἱοὶ Ἰωσὴφ καί γε αὐτοὶ εἰς Βαιθήλ· καὶ Κύριος ἦν μετʼ αὐτῶν.
\vs{23}Καὶ παρενέβαλον, καὶ κατεσκέψαντο Βαιθήλ· τὸ δὲ ὄνομα τῆς πόλεως ἦν ἔμπροσθεν Λουζά.

\vs{24}Καὶ εἶδον οἱ φυλάσσοντες, καὶ ἰδοὺ ἀνὴρ ἐξεπορεύετο ἐκ τῆς πόλεως, καὶ ἔλαβον αὐτόν· καὶ εἶπον αὐτῷ, δεῖξον ἡμῖν τῆς πόλεως τὴν εἴσοδον, καὶ ποιήσομεν μετὰ σου ἔλεος.
\vs{25}Καὶ ἔδειξεν αὐτοῖς τὴν εἴσοδον τῆς πόλεως· καὶ ἐπάταξαν τὴν πόλιν ἐν στόματι ῥομφαίας· τὸν δὲ ἄνδρα καὶ τὴν συγγένειαν αὐτοῦ ἐξαπέστειλαν.
\vs{26}Καὶ ἐπορεύθη ὁ ἀνὴρ εἰς γῆν Χεττίν· καὶ ᾠκοδόμησεν ἐκεῖ πόλιν, καὶ ἐκάλεσε τὸ ὄνομα αὐτῆς Λουζά· τοῦτο ὄνομα αὐτῆς ἕως τῆς ἡμέρας ταύτης.

\vs{27}Καὶ οὐκ ἐξῇρε Μανασσῆ τὴν Βαιθσὰν, ἥ ἐστι Σκυθῶν πόλις, οὐδὲ τὰς θυγατέρας αὐτῆς οὐδὲ τὰ περίοικα αὐτῆς, οὐδὲ τὴν Θανὰκ οὐδὲ τὰς θυγατέρας αὐτῆς, οὐδὲ τοὺς κατοικοῦντας Δὼρ οὐδὲ τὰς θυγατέρας αὐτῆς, οὐδὲ τὸν κατοικοῦντα Βαλὰκ οὐδὲ τὰ περίοικα αὐτῆς οὐδὲ τὰς θυγατέρας αὐτῆς, οὐδὲ τοὺς κατοικοῦντας Μαγεδὼ οὐδὲ τὰ περίοικα αὐτῆς καὶ τὰς θυγατέρας αὐτῆς, οὐδὲ τοὺς κατοικοῦντας Ἰεβλαὰμ οὐδὲ τὰ περίοικα αὐτῆς, οὐδὲ τὰς θυγατέρας αὐτῆς· καὶ ἤρξατο ὁ Χαναναῖος κατοικεῖν ἐν τῇ γῇ ταύτῃ.
\vs{28}Καὶ ἐγένετο ὅτε ἐνίσχυσεν Ἰσραὴλ, καὶ ἐποίησε τὸν Χαναναῖον εἰς φόρον, καὶ ἐξαίρων οὐκ ἐξῇρεν αὐτόν.
\vs{29}Καὶ Ἐφραὶμ οὐκ ἐξῇρε τὸν Χαναναῖον τὸν κατοικοῦντα ἐν Γαζέρ· καὶ κατῴκησεν ὁ Χαναναῖος ἐν μέσῳ αὐτοῦ ἐν Γαζὲρ, καὶ ἐγένετο εἰς φόρον.
\vs{30}Καὶ Ζαβουλὼν οὐκ ἐξῇρε τοὺς κατοικοῦντας Κέδρων, οὐδὲ τοὺς κατοικοῦντας Δωμανά· καὶ κατῴκησεν ὁ Χαναναῖος ἐν μέσῳ αὐτῶν, καὶ ἐγένετο αὐτῷ εἰς φόρον. Καὶ Ἀσὴρ οὐκ ἐξῇρε τοὺς κατοικοῦντας Δωμανά· καὶ κατῴκησεν ὁ Χαναναῖος ἐν μέσῳ αὐτῶν, καὶ ἐγένετο αὐτῷ εἰς φόρον.
\vs{31}Καὶ Ἀσὴρ οὐκ ἐξῇρε τοὺς κατοικοῦντας Ἀκχὼ, καὶ ἐγένετο αὐτῷ εἰς φόρον, καὶ τοὺς κατοικοῦντας Δὼρ, καὶ τοὺς κατοικοῦντας Σιδῶνα, καὶ τοὺς κατοικοῦντας Δαλὰφ, τὸν Ἀσχαζὶ, καὶ τὸν Χεβδὰ, καὶ τὸν Ναῒ, καὶ τὸν Ἐρεώ.
\vs{32}Καὶ κατῴκησεν ὁ Ἀσὴρ ἐν μέσῳ τοῦ Χαναναίου τοῦ κατοικοῦντος τὴν γῆν, ὅτι οὐκ ἠδυνήθη ἐξάραι αὐτόν.
\vs{33}Καὶ Νεφθαλὶ οὐκ ἐξῇρε τοὺς κατοικοῦντας Βαιθσαμῦς, καὶ τοὺς κατοικοῦντας Βαιθανάχ· καὶ κατῴκησε Νεφθαλὶ ἐν μέσῳ τοῦ Χαναναίου τοῦ κατοικοῦντος τὴν γῆν· οἱ δὲ κατοικοῦντες Βαιθσαμὺς καὶ τὴν Βαιθενὲθ, ἐγένοντο αὐτοῖς εἰς φόρον.

\vs{34}Καὶ ἐξέθλιψεν ὁ Ἀμοῤῥαῖος τοὺς υἱοὺς Δὰν εἰς τὸ ὄρος, ὅτι οὐκ ἀφῆκαν αὐτὸν καταβῆναι εἰς τὴν κοιλάδα.
\vs{35}Καὶ ἤρξατο ὁ Ἀμοῤῥαῖος κατοικεῖν ἐν τῷ ὄρει τῷ ὀστρακώδει, ἐν ᾧ αἱ ἅρκτοι καὶ ἐν ᾧ αἱ ἀλώπεκες, ἐν τῷ Μυρσινῶνι, καὶ ἐν Θαλαβίν, καὶ ἐβαρύνθη ἡ χεὶρ οἴκου Ἰωσὴφ ἐπὶ τὸν Ἀμοῤῥαῖον, καὶ ἐγενήθη αὐτοῖς εἰς φόρον.
\vs{36}Καὶ τὸ ὅριον τοῦ Ἀμοῤῥαίου ἀπὸ τῆς ἀναβάσεως Ἀκραβὶν ἀπὸ τῆς πέτρας καὶ ἐπάνω.

\ch{2}
Καὶ ἀνέβη ἄγγελος Κυρίου ἀπὸ Γαλγὰλ ἐπὶ τὸν κλαυθμῶνα καὶ ἐπὶ Βαιθὴλ καὶ ἐπὶ τὸν οἶκον Ἰσραὴλ, καὶ εἶπε πρὸς αὐτοὺς, τάδε λέγει Κύριος, ἀνεβίβασα ὑμᾶς ἐξ Αἰγύπτου, καὶ εἰσήγαγον ὑμᾶς εἰς τὴν γῆν ἣν ὤμοσα τοῖς πατράσιν ὑμῶν· καὶ εἶπα, οὐ διασκεδάσω τὴν διαθήκην μου τὴν μεθʼ ὑμῶν εἰς τὸν αἰῶνα.
\vs{2}Καὶ ὑμεῖς οὐ διαθήσεσθε διαθήκην τοῖς ἐγκαθημένοις εἰς τὴν γῆν ταύτην, οὐδὲ τοῖς θεοῖς αὐτῶν προσκυνήσετε, ἀλλὰ τὰ γλυπτὰ αὐτῶν συντρίψετε, τὰ θυσιαστήρια αὐτῶν καθελεῖτε· καὶ οὐκ εἰσηκούσατε τῆς φωνῆς μου, ὅτι ταῦτα ἐποιήσατε.
\vs{3}Κᾀγὼ εἶπον, οὐ μὴ ἐξάρω αὐτοὺς ἐκ προσώπου ὑμῶν, καὶ ἔσονται ὑμῖν εἰς συνοχὰς, καὶ οἱ θεοὶ αὐτῶν ἔσονται ὑμῖν εἰς σκάνδαλον.
\vs{4}Καὶ ἐγένετο ὡς ἐλάλησεν ὁ ἄγγελος Κυρίου τοὺς λόγους τούτους πρὸς πάντας υἱοὺς Ἰσραὴλ, καὶ ἐπῇραν ὁ λαὸς τὴν φωνὴν αὐτῶν καὶ ἔκλαυσαν.
\vs{5}Καὶ ἐπωνόμασαν τὸ ὄνομα τοῦ τόπου ἐκείνου, Κλαυθμῶνες· καὶ ἐθυσίασαν ἐκεῖ τῷ Κυρίῳ.

\vs{6}Καὶ ἐξαπέστειλεν Ἰησοῦς τὸν λαὸν, καὶ ἦλθεν ἀνὴρ εἰς τὴν κληρονομίαν αὐτοῦ κατακληρονομῆσαι τὴν γῆν.
\vs{7}Καὶ ἐδούλευσεν ὁ λαὸς τῷ Κυρίῳ πάσας τὰς ἡμέρας Ἰησοῦ καὶ πάσας τὰς ἡμέρας τῶν πρεσβυτέρων, ὅσοι ἐμακροημέρευσαν μετὰ Ἰησοῦ, ὅσοι ἔγνωσαν πᾶν τὸ ἔργον Κυρίου τὸ μέγα ὅσα ἐποίησεν ἐν τῷ Ἰσραήλ.

\vs{8}Καὶ ἐτελεύτησεν Ἰησοῦς υἱὸς Ναυῆ δοῦλος Κυρίου, υἱὸς ἑκατὸν δέκα ἐτῶν.
\vs{9}Καὶ ἔθαψαν αὐτὸν ἐν ὁρίῳ τῆς κληρονομίας αὐτοῦ ἐν Θαμναθαρὲς, ἐν ὄρει Ἐφραὶμ ἀπὸ Βοῤῥᾶ τοῦ ὄρους Γαάς.
\vs{10}Καὶ πᾶσα ἡ γενεὰ ἐκείνη προσετέθησαν πρὸς τοὺς πατέρας αὐτῶν· καὶ ἀνέστη γενεὰ ἑτέρα μετʼ αὐτοὺς, οἳ οὐκ ἔγνωσαν τὸν Κύριον, καί γε τὸ ἔργον ὃ ἐποίησεν ἐν τῷ Ἰσραήλ.
\vs{11}Καὶ ἐποίησαν οἱ υἱοὶ Ἰσραὴλ τὸ πονηρὸν ἐνώπιον Κυρίου, καὶ ἐλάτρευσαν τοῖς Βααλίμ.
\vs{12}Καὶ ἐγκατέλιπον τὸν Κύριον τὸν Θεὸν τῶν πατέρων αὐτῶν, τὸν ἐξαγαγόντα αὐτοὺς ἐκ γῆς Αἰγύπτου, καὶ ἐπορεύθησαν ὀπίσω θεῶν ἑτέρων ἀπὸ τῶν θεῶν τῶν ἐθνῶν τῶν περικύκλῳ αὐτῶν, καὶ προσεκύνησαν αὐτοῖς· καὶ παρώργισαν τὸν Κύριον,
\vs{13}καὶ ἐγκατέλιπον αὐτὸν, καὶ ἐλάτρευσαν τῷ Βάαλ καὶ ταῖς Ἀστάρταις.

\vs{14}Καὶ ὠργίσθη θυμῷ Κύριος ἐν τῷ Ἰσραήλ· καὶ παρέδωκεν αὐτοὺς εἰς χεῖρας προνομευόντων, καὶ κατεπρονόμευσαν αὐτούς· καὶ ἀπέδοτο αὐτοὺς ἐν χερσὶ τῶν ἐχθρῶν αὐτῶν κυκλόθεν, καὶ οὐκ ἠδυνήθησαν ἔτι ἀντιστῆναι κατὰ πρόσωπον τῶν ἐχθρῶν αὐτῶν ἐν πᾶσιν οἷς ἐπορεύοντο·
\vs{15}καὶ χεὶρ Κυρίου ἦν ἐπʼ αὐτοὺς εἰς κακὰ, καθὼς ἐλάλησε Κύριος καὶ καθὼς ὤμοσε Κύριος αὐτοῖς, καὶ ἐξέθλιψεν αὐτοὺς σφόδρα.

\vs{16}Καὶ ἤγειρε Κύριος κριτὰς, καὶ ἔσωσεν αὐτοὺς Κύριος ἐκ χειρὸς τῶν προνομευόντων αὐτούς· καί γε τῶν κριτῶν οὐχ ὑπήκουσαν,
\vs{17}ὅτι ἐξεπόρνευσαν ὀπίσω θεῶν ἑτέρων, καὶ προσεκύνησαν αὐτοῖς· καὶ ἐξέκλιναν ταχὺ ἐκ τῆς ὁδοῦ, ἧς ἐπορεύθησαν οἱ πατέρες αὐτῶν τοῦ εἰσακούειν τῶν λόγων Κυρίου· οὐκ ἐποίησαν οὕτω.
\vs{18}Καὶ ὅτι ἤγειρε Κύριος αὐτοῖς κριτὰς, καὶ ἦν Κύριος μετὰ τοῦ κριτοῦ, καὶ ἔσωσεν αὐτοὺς ἐκ χειρὸς ἐχθρῶν αὐτῶν πάσας τὰς ἡμέρας τοῦ κριτοῦ, ὅτι παρεκλήθη Κύριος ἀπὸ τοῦ στεναγμοῦ αὐτῶν ἀπὸ προσώπου τῶν πολιορκούντων αὐτοὺς καὶ ἐκθλιβόντων αὐτούς.
\vs{19}Καὶ ἐγένετο ὡς ἀπέθνησκεν ὁ κριτὴς, καὶ ἀπέστρεψαν καὶ πάλιν διέφθειραν ὑπὲρ τοὺς πατέρας αὐτῶν πορεύεσθαι ὀπίσω θεῶν ἑτέρων, λατρεύειν αὐτοῖς καὶ προσκυνεῖν αὐτοῖς· οὐκ ἀπέῤῥιψαν τὰ ἐπιτηδεύματα αὐτῶν, καὶ τὰς ὁδοὺς αὐτῶν τὰς σκληράς.

\vs{20}Καὶ ὠργίσθη θυμῷ Κύριος ἐν τῷ Ἰσραήλ· καὶ εἶπεν, ἀνθʼ ὧν ὅσα ἐγκατέλιπον τὸ ἔθνος τοῦτο τὴν διαθήκην μου ἣν ἐνετειλάμην τοῖς πατράσιν αὐτῶν, καὶ οὐκ εἰσήκουσαν τῆς φωνῆς μου,
\vs{21}καὶ ἐγὼ οὐ προσθήσω τοῦ ἐξᾶραι ἄνδρα ἐκ προσώπου αὐτῶν ἀπὸ τῶν ἐθνῶν, ὧν κατέλιπεν Ἰησοῦς υἱὸς Ναυῆ ἐν τῇ γῇ· καὶ ἀφῆκε
\vs{22}τοῦ πειρᾶσαι ἐν αὐτοῖς τὸν Ἰσραὴλ, εἰ φυλάσσονται τὴν ὁδὸν Κυρίου πορεύεσθαι ἐν αὐτῇ, ὃν τρόπον ἐφύλαξαν οἱ πατέρες αὐτῶν, ἢ οὔ.
\vs{23}Καὶ ἀφήσει Κύριος τὰ ἔθνη ταῦτα τοῦ μὴ ἐξᾶραι αὐτὰ τὸ τάχος, καὶ οὐ παρέδωκεν αὐτὰ ἐν χειρὶ Ἰησοῦ.

\ch{3}
Καὶ ταῦτα τὰ ἔθνη ἃ ἀφῆκε Κύριος αὐτὰ ὥστε πειρᾶσαι ἐν αὐτοῖς τὸν Ἰσραὴλ, πάντας τοὺς μὴ ἐγνωκότας τοὺς πολέμους Χαναάν.
\vs{2}Πλὴν διὰ τὰς γενεὰς υἱῶν Ἰσραὴλ τοῦ διδάξαι αὐτοὺς πόλεμον, πλὴν οἱ ἔμπροσθεν αὐτῶν οὐκ ἔγνωσαν αὐτά.
\vs{3}Τὰς πέντε σατραπείας τῶν ἀλλοφύλων, καὶ πάντα τὸν Χαναναῖον, καὶ τὸν Σιδώνιον, καὶ τὸν Εὐαῖον τὸν κατοικοῦντα τὸν Λίβανον ἀπὸ τοῦ ὄρους τοῦ Ἀερμὼν ἕως Λαβωεμάθ.
\vs{4}Καὶ ἐγένετο ὥστε πειρᾶσαι ἐν αὐτοῖς τὸν Ἰσραὴλ, γνῶναι εἰ ἀκούσονται τὰς ἐντολὰς Κυρίου, ἃς ἐνετείλατο τοῖς πατράσιν αὐτῶν ἐν χειρὶ Μωυσῆ.

\vs{5}Καὶ οἱ υἱοὶ Ἰσραὴλ κατῴκησαν ἐν μέσῳ τοῦ Χαναναίου, καὶ τοῦ Χετταίου, καὶ τοῦ Ἀμοῤῥαίου, καὶ τοῦ Φερεζαίου, καὶ τοῦ Εὐαίου, καὶ τοῦ Ἰεβουσαίου.
\vs{6}Καὶ ἔλαβον τὰς θυγατέρας αὐτῶν ἑαυτοῖς εἰς γυναῖκας, καὶ τὰς θυγατέρας αὐτῶν ἔδωκαν τοῖς υἱοῖς αὐτῶν, καὶ ἐλάτρευσαν τοῖς θεοῖς αὐτῶν.
\vs{7}Καὶ ἐποίησαν οἱ υἱοὶ Ἰσραὴλ τὸ πονηρὸν ἐναντίον Κυρίου· καὶ ἐπελάθοντο Κυρίου τοῦ Θεοῦ αὐτῶν, καὶ ἐλάτρευσαν τοῖς Βααλὶμ καὶ τοῖς ἄλσεσι.
\vs{8}Καὶ ὠργίσθη θυμῷ Κύριος ἐν τῷ Ἰσραὴλ, καὶ ἀπέδοτο αὐτοὺς ἐν χειρὶ Χουσαρσαθαὶμ βασιλέως Συρίας ποταμῶν· καὶ ἐδούλευσαν οἱ υἱοὶ Ἰσραὴλ τῷ Χουσαρσαθαὶμ ἔτη ὀκτώ.

\vs{9}Καὶ ἐκέκραξαν οἱ υἱοὶ Ἰσραὴλ πρὸς Κύριον· καὶ ἤγειρε Κύριος σωτῆρα τῷ Ἰσραὴλ, καὶ ἔσωσεν αὐτοὺς, τὸν Γοθονιὴλ υἱὸν Κενὲζ ἀδελφοῦ Χάλεβ τὸν νεώτερον ὑπὲρ αὐτόν.
\vs{10}Καὶ ἐγένετο ἐπʼ αὐτὸν πνεῦμα Κυρίου, καὶ ἔκρινε τὸν Ἰσραήλ· καὶ ἐξῆλθεν εἰς πόλεμον πρὸς Χουσαρσαθαίμ· καὶ παρέδωκε Κύριος ἐν χειρὶ αὐτοῦ τὸν Χουσαρσαθαὶμ βασιλέα Συρίας ποταμῶν· καὶ ἐκραταιώθη χεὶρ αὐτοῦ ἐπὶ τὸν Χουσαρσαθαίμ.
\vs{11}Καὶ ἡσύχασεν ἡ γῆ ἔτη τεσσαράκοντα· καὶ ἀπέθανε Γοθονιὴλ υἱὸς Κενέζ.

\vs{12}Καὶ προσέθεντο οἱ υἱοὶ Ἰσραὴλ ποιῆσαι τὸ πονηρὸν ἐνώπιον Κυρίου· καὶ ἐνίσχυσε Κύριος τὸν Ἐγλὼμ βασιλέα Μωὰβ ἐπὶ τὸν Ἰσραὴλ, διὰ τὸ πεποιηκέναι αὐτοὺς τὸ πονηρὸν ἔναντι Κυρίου.
\vs{13}Καὶ συνήγαγε πρὸς ἑαυτὸν πάντας τοὺς υἱοὺς Ἀμμὼν καὶ Ἀμαλὴκ, καὶ ἐπορεύθη καὶ ἐπάταξε τὸν Ἰσραὴλ, καὶ ἐκληρονόμησε τὴν πόλιν τῶν φοινίκων.
\vs{14}Καὶ ἐδούλευσαν οἱ υἱοὶ Ἰσραὴλ τῷ Ἐγλὼμ βασιλεῖ Μωὰβ ἔτη δεκαοκτώ.

\vs{15}Καὶ ἐκέκραξαν οἱ υἱοὶ Ἰσραὴλ πρὸς Κύριον· καὶ ἤγειρεν αὐτοῖς σωτῆρα, τὸν Ἀὼδ υἱὸν Γηρὰ υἱὸν τοῦ Ἰεμενὶ ἄνδρα ἀμφοτεροδέξιον· καὶ ἐξαπέστειλαν οἱ υἱοὶ Ἰσραὴλ δῶρα ἐν χειρὶ αὐτοῦ τῷ Ἐγλὼμ βασιλεῖ Μωάβ.
\vs{16}Καὶ ἐποίησεν ἑαυτῷ Ἀὼδ μάχαιραν δίστομον, σπιθαμῆς τὸ μῆκος αὐτῆς· καὶ περιεζώσατο αὐτὴν ὑπὸ τὸν μανδύαν ἐπὶ τὸν μηρὸν αὐτοῦ τὸν δεξιόν.
\vs{17}Καὶ ἐπορεύθη, καὶ προσήνεγκε τὰ δῶρα τῷ Ἐγλὼμ βασιλεῖ Μωάβ· καὶ Ἐγλὼμ ἀνὴρ ἀστεῖος σφόδρα.

\vs{18}Καὶ ἐγένετο ἡνίκα συνετέλεσεν Ἀὼδ προσφέρων τὰ δῶρα, καὶ ἐξαπέστειλε τοὺς φέροντας τὰ δῶρα,
\vs{19}καὶ αὐτὸς ὑπέστρεψεν ἀπὸ τῶν γλυπτῶν τῶν μετὰ τῆς Γαλγάλ· καὶ εἶπεν Ἀὼδ, λόγος μοι κρύφιος πρὸς σὲ, βασιλεῦ· καὶ εἶπεν Ἐγλὼμ πρὸς αὐτὸν, σιώπα· καὶ ἐξαπέστειλεν ἀφʼ ἑαυτοῦ πάντας τοὺς ἐφεστῶτας ἐπʼ αὐτόν,
\vs{20}καὶ Ἀὼδ εἰσῆλθε πρὸς αὐτόν· καὶ αὐτὸς ἐκάθητο ἐν τῷ ὑπερῴῳ τῷ θερινῷ τῷ ἑαυτοῦ μονώτατος· καὶ εἶπεν Ἀὼδ, λόγος Θεοῦ μοι πρὸς σὲ, βασιλεῦ· καὶ ἐξανέστη ἀπὸ τοῦ θρόνου Ἐγλὼμ ἐγγὺς αὐτοῦ.
\vs{21}Καὶ ἐγένετο ἅμα τῷ ἀναστῆναι αὐτὸν, καὶ ἐξέτεινεν Ἀὼδ τὴν χεῖρα τὴν ἀριστερὰν αὐτοῦ, καὶ ἔλαβε τὴν μάχαιραν ἐπάνωθεν τοῦ μηροῦ αὐτοῦ τοῦ δεξιοῦ, καὶ ἐνέπηξεν αὐτὴν ἐν τῇ κοιλίᾳ αὐτοῦ,
\vs{22}καὶ ἐπεισήνεγκε καί γε τὴν λαβὴν ὀπίσω τῆς φλογὸς, καὶ ἀπέκλεισε τὸ στέαρ κατὰ τῆς φλογὸς, ὅτι οὐκ ἐξέσπασε τὴν μάχαιραν ἐκ τῆς κοιλίας αὐτοῦ.

\vs{23}Καὶ ἐξῆλθεν Ἀὼδ τὴν προστάδα· καὶ ἐξῆλθε τοὺς διατεταγμένους, καὶ ἀπέκλεισε τὰς θύρας τοῦ ὑπερῴου κατʼ αὐτοῦ, καὶ ἐσφήνωσε.
\vs{24}Καὶ αὐτὸς ἐξῆλθε· καὶ οἱ παῖδες αὐτοῦ ἐπῆλθον καὶ εἶδον, καὶ ἰδοὺ αἱ θύραι τοῦ ὑπερῴου ἐσφηνωμέναι· καὶ εἶπαν, μήποτε ἀποκενοῖ τοὺς πόδας αὐτοῦ ἐν τῷ ταμείῳ τῷ θερινῷ;
\vs{25}Καὶ ὑπέμειναν ἕως ᾑσχύνοντο· καὶ ἰδοὺ οὐκ ἔστιν ὁ ἀνοίγων τὰς θύρας τοῦ ὑπερῴου· καὶ ἔλαβον τὴν κλεῖδα, καὶ ἤνοιξαν· καὶ ἰδοὺ ὁ κύριος αὐτῶν πεπτωκὼς ἐπὶ τὴν γῆν τεθνηκώς.

\vs{26}Καὶ Ἀὼδ διεσώθη ἕως ἐθορυβοῦντο, καὶ οὐκ ἦν ὁ προσνοῶν αὐτῷ· καὶ αὐτὸς παρῆλθε τὰ γλυπτὰ, καὶ διεσώθη εἰς Σετειρωθά.

\vs{27}Καὶ ἐγένετο ἡνίκα ἦλθεν Ἀὼδ εἰς γῆν Ἰσραὴλ, καὶ ἐσάλπισεν ἐν κερατίνῃ ἐν τῷ ὄρει Ἐφραὶμ, καὶ κατέβησαν σὺν αὐτῷ οἱ υἱοὶ Ἰσραὴλ ἀπὸ τοῦ ὄρους, καὶ αὐτὸς ἔμπροσθεν αὐτῶν.
\vs{28}Καὶ εἶπε πρὸς αὐτοὺς, κατάβητε ὀπίσω μου, ὅτι παρέδωκε Κύριος ὁ Θεὸς τοὺς ἐχθροὺς ἡμῶν τὴν Μωὰβ ἐν χειρὶ ἡμῶν· καὶ κατέβησαν ὀπίσω αὐτοῦ, καὶ προκατελάβοντο τὰς διαβάσεις τοῦ Ἰορδάνου τῆς Μωὰβ, καὶ οὐκ ἀφῆκεν ἄνδρα διαβῆναι.
\vs{29}Καὶ ἐπάταξαν τὴν Μωὰβ τῇ ἡμέρᾳ ἐκείνῃ ὡσεὶ δέκα χιλιάδας ἀνδρῶν, πᾶν λιπαρὸν καὶ πάντα ἄνδρα δυνάμεως, καὶ οὐ διεσώθη ὁ ἀνήρ.
\vs{30}Καὶ ἐνετράπη Μωὰβ ἐν τῇ ἡμέρᾳ ἐκείνῃ ὑπὸ χεῖρα Ἰσραὴλ, καὶ ἡσύχασεν ἡ γῆ ὀγδοήκοντα ἔτη· καὶ ἔκρινεν αὐτοὺς Ἀὼδ ἕως οὗ ἀπέθανε.

\vs{31}Καὶ μετʼ αὐτὸν ἀνέστη Σαμεγὰρ υἱὸς Δινὰχ, καὶ ἐπάταξε τοὺς ἀλλοφύλους εἰς ἑξακοσίους ἄνδρας ἐν τῷ ἀροτρόποδι τῶν βοῶν· καὶ ἔσωσε καί γε αὐτὸς τὸν Ἰσραήλ.

\ch{4}
Καὶ προσέθεντο οἱ υἱοὶ Ἰσραὴλ ποιῆσαι τὸ πονηρὸν ἐνώπιον Κυρίου· καὶ Ἀὼδ ἀπέθανε.
\vs{2}Καὶ ἀπέδοτο τοὺς υἱοὺς Ἰσραὴλ Κύριος ἐν χειρὶ Ἰαβὶν βασιλέως Χαναὰν, ὃς ἐβασίλευσεν ἐν Ἀσώρ· καὶ ὁ ἂρχων τῆς δυνάμεως αὐτοῦ Σισάρα, καὶ αὐτὸς κατῴκει ἐν Ἀρισὼθ τῶν ἐθνῶν.
\vs{3}Καὶ ἐκέκραξαν οἱ υἱοὶ Ἰσραὴλ πρὸς Κύριον, ὅτι ἐννακόσια ἅρματα σιδηρᾶ ἦν αὐτῷ· καὶ αὐτὸς ἔθλιψε τὸν Ἰσραὴλ κατακράτος εἴκοσι ἔτη.

\vs{4}Καὶ Δεββῶρα γυνὴ προφῆτις, γυνὴ Λαφιδὼθ, αὕτη ἔκρινε τὸν Ἰσραὴλ ἐν τῷ καιρῷ ἐκείνῳ.

\vs{5}Καὶ αὐτὴ ἐκάθητο ὑπὸ φοίνικα Δεββῶρα ἀναμέσον τῆς Ῥαμὰ καὶ ἀναμέσον τῆς Βαιθὴλ ἐν τῷ ὄρει Ἐφραίμ· καὶ ἀνέβαινον πρὸς αὐτὴν οἱ υἱοὶ Ἰσραὴλ εἰς κρίσιν.

\vs{6}Καὶ ἀπέστειλε Δεββῶρα καὶ ἐκάλεσε τὸν Βαρὰκ υἱὸν Ἀβινεὲμ ἐκ Κάδης Νεφθαλὶ, καὶ εἶπε πρὸς αὐτὸν, οὐχὶ ἐνετείλατο Κύριος ὁ Θεὸς Ἰσραήλ σοι; καὶ ἀπελεύσῃ εἰς ὄρος Θαβὼρ, καὶ λήψῃ μετὰ σεαυτοῦ δέκα χιλιάδας ἀνδρῶν ἐκ τῶν υἱῶν Νεφθαλὶ, καὶ ἐκ τῶν υἱῶν Ζαβουλὼν.

\vs{7}Καὶ ἐπάξω πρὸς σὲ εἰς τὸν χειμάῤῥουν Κισῶν ἐπὶ τὸν Σισάρα ἄρχοντα τῆς δυνάμεως Ἰαβὶν, καὶ τὰ ἅρματα αὐτοῦ καὶ τὸ πλῆθος αὐτοῦ, καὶ παραδώσω αὐτὸν εἰς χεῖράς σου.

\vs{8}Καὶ εἶπε πρὸς αὐτὴν Βαρὰκ, ἐὰν πορευθῇς μετʼ ἐμοῦ, πορεύσομαι, καὶ ἐὰν μὴ πορευθῇς, οὐ πορεύσομαι· ὅτι οὐκ οἶδα τὴν ἡμέραν ἐν ᾗ εὐοδοῖ Κύριος τὸν ἄγγελον μετʼ ἐμοῦ.
\vs{9}Καὶ εἶπε, πορευομένη πορεύσομαι μετὰ σοῦ· πλὴν γίνωσκε ὅτι οὐκ ἔσται τὸ προτέρημά σου ἐπὶ τὴν ὁδὸν ἣν σὺ πορεύῃ, ὅτι ἐν χειρὶ γυναικὸς ἀποδώσεται Κύριος τὸν Σισάρα· καὶ ἀνέστη Δεββῶρα, καὶ ἐπορεύθη μετὰ τοῦ Βαρὰκ ἐκ Κάδης.
\vs{10}Καὶ ἐβόησε Βαρὰκ τὸν Ζαβουλὼν καὶ τὸν Νεφθαλὶ ἐκ Κάδης, καὶ ἀνέβησαν κατὰ πόδας αὐτοῦ δέκα χιλιάδες ἀνδρῶν, καὶ ἀνέβη Δεββῶρα μετʼ αὐτοῦ.

\vs{11}Καὶ Χαβὲρ ὁ Κιναῖος ἐχωρίσθη ἀπὸ Καινᾶ ἀπὸ τῶν υἱῶν Ἰωβὰβ γαμβροῦ Μωυσῆ· καὶ ἔπηξε τὴν σκηνὴν αὐτοῦ ἕως δρυὸς πλεονεκτούντων, ἥ ἐστιν ἐχόμενα Κεδές.

\vs{12}Καὶ ἀνηγγέλη Σισάρᾳ, ὅτι ἀνέβη Βαρὰκ υἱὸς Ἀβινεὲμ εἰς ὄρος Θαβώρ.
\vs{13}Καὶ ἐκάλεσε Σισάρα πάντα τὰ ἅρματα αὐτοὺ ἐννακόσια ἅρματα σιδηρᾶ, καὶ πάντα τὸν λαὸν τὸν μετʼ αὐτοῦ ἀπὸ Ἀρισὼθ τῶν ἐθνῶν εἰς τὸν χειμάῤῥουν Κισῶν.

\vs{14}Καὶ εἶπε Δεββῶρα πρὸς Βαρὰκ, ἀνάστηθι, ὅτι αὕτη ἡ ἡμέρα ἐν ᾗ παρέδωκε Κύριος τὸν Σισάρα ἐν τῇ χειρί σου, ὅτι Κύριος ἐξελεύσεται ἔμπροσθέν σου· καὶ κατέβη Βαρὰκ κατὰ τοῦ ὄρους Θαβὼρ, καὶ δέκα χιλιάδες ἀνδρῶν ὀπίσω αὐτοῦ.
\vs{15}Καὶ ἐξέστησε Κύριος τὸν Σισάρα καὶ πάντα τὰ ἅρματα αὐτοῦ καὶ πᾶσαν τὴν παρεμβολὴν αὐτοῦ ἐν στόματι ῥομφαίας ἐνώπιον Βαράκ· καὶ κατέβη Σισάρα ἐπάνωθεν τοὺ ἄρματος αὐτοῦ, καὶ ἔφυγε τοῖς ποσὶν αὐτοῦ.
\vs{16}Καὶ Βαρὰκ διώκων ὀπίσω τῶν ἁρμάτων καὶ ὀπίσω τῆς παρεμβολῆς ἕως Ἀρισὼθ τῶν ἐθνῶν· καὶ ἔπεσε πᾶσα παρεμβολὴ Σισάρα ἐν στόματι ῥομφαίας· οὐ κατελείφθη ἕως ἑνός.
\vs{17}Καὶ Σισάρα ἔφυγε τοῖς ποσὶν αὐτοῦ εἰς σκηνὴν Ἰαὴλ γυναικὸς Χαβὲρ ἑταίρου τοῦ Κιναίου· ὅτι εἰρήνη ἦν ἀναμέσον Ἰαβὶν βασιλέως Ἀσὼρ καὶ ἀναμέσον τοῦ οἴκου Χαβὲρ τοῦ Κιναίου.
\vs{18}Καὶ ἐξῆλθεν Ἰαὴλ εἰς συνάντησιν Σισάρα, καὶ εἶπεν αὐτῷ, ἔκκλινον, κύριέ μου, ἔκκλινον πρὸς μὲ, μὴ φοβοῦ· καὶ ἐξέκλινε πρὸς αὐτῆν εἰς τὴν σκηνήν· καὶ περιέβαλεν αὐτὸν ἐπιβολαίῳ.

\vs{19}Καὶ εἶπε Σισάρα πρὸς αὐτὴν, πότισόν με δὴ μικρὸν ὑδωρ, ὅτι ἐδίψησα· καὶ ἤνοιξε τὸν ἀσκὸν τοῦ γάλακτος, καὶ ἐπότισεν αὐτὸν, καὶ περιέβαλεν αὐτόν.
\vs{20}Καὶ εἶπε πρὸς αὐτὴν Σισάρα, στῆθι δὴ ἐπὶ τὴν θύραν τῆς σκηνῆς, καὶ ἔσται ἐὰν ἀνὴρ ἔλθῃ πρὸς σὲ, καὶ ἐρωτήσῃ σε, καὶ εἴπῃ, εἰ ἔστιν ὧδε ἀνήρ; καὶ ἐρεῖς, οὐκ ἔστι.
\vs{21}Καὶ ἔλαβεν Ἰαὴλ γυνὴ Χαβὲρ τὸν πάσσαλον τῆς σκηνῆς, καὶ ἔθηκε τὴν σφύραν ἐν τῇ χειρὶ αὐτῆς, καὶ εἰσῆλθε πρὸς αὐτὸν ἐν κρύφῇ, καὶ ἔπηξε τὸν πάσσαλον ἐν τῷ κροτάφῳ αὐτοῦ, καὶ διεξῆλθεν ἐν τῇ γῇ· καὶ αὐτὸς ἐξεστῶς ἐσκοτώθη, καὶ ἀπέθανε.
\vs{22}Καὶ ἰδοὺ Βαρὰκ διώκων τὸν Σισάρα· καὶ ἐξῆλθεν Ἰαὴλ εἰς συνάντησιν αὐτῷ, καὶ εἶπεν αὐτῷ, δεῦρο καὶ δείξω σοι τὸν ἄνδρα ὃν σὺ ζητεῖς· καὶ εἰσῆλθε πρὸς αὐτήν· καὶ ἰδοὺ Σισάρα ἐῤῥιμμένος νεκρὸς, καὶ ὁ πάσσαλος ἐν τῷ κροτάφῳ αὐτοῦ.
\vs{23}Καὶ ἐτρόπωσεν ὁ Θεὸς τὸν Ἰαβὶν βασιλέα Χαναὰν ἐν τῇ ἡμέρᾳ ἐκείνῃ ἔμπροσθεν υἱῶν Ἰσραήλ.

\vs{24}Καὶ ἐπορεύετο χεὶρ τῶν υἱῶν Ἰσραὴλ πορευομένη καὶ σκληρυνομένη ἐπὶ Ἰαβὶν βασιλέα Χαναὰν, ἕως οὗ ἐξωλόθρευσαν τὸν Ἰαβὶν βασιλέα Χαναάν.

\ch{5}
Καὶ ἦσαν Δεββῶρα καὶ Βαρὰκ υἱὸς Ἀβινεὲμ ἐν τῇ ἡμέρᾳ ἐκείνῃ, λέγοντες,

\vs{2}Ἀπεκαλύφθη ἀποκάλυμμα ἐν Ἰσραὴλ ἐν τῷ ἑκουσιασθῆναι λαὸν, εὐλογεῖτε Κύριον.
\vs{3}Ἀκούσατε βασιλεῖς, καὶ ἐνωτίσασθε σατράπαι· ᾄσομαι ἐγώ εἰμι τῷ Κυρίῳ ἐγώ εἰμι, ψαλῶ τῷ Κυρίῳ τῷ Θεῷ Ἰσραήλ.
\vs{4}Κύριε, ἐν τῇ ἐξόδῳ σου ἐν Σηεὶρ, ἐν τῷ ἀπαίρειν σε ἐξ ἀγροῦ Ἐδὼμ, γῆ ἐσείσθη, καὶ ὁ οὐρανὸς ἔσταξε δρόσους, καὶ αἱ νεφέλαι ἔσταξαν ὕδωρ.
\vs{5}Ὄρη ἐσαλεύθησαν ἀπὸ προσώπου Κυρίου Ἐλωῒ, τοῦτο Σινὰ ἀπὸ προσώπου Κυρίου Θεοῦ Ἰσραήλ.
\vs{6}Ἐν ἡμέραις Σαμεγὰρ υἱοῦ Ἀνὰθ, ἐν ἡμέραις Ἰαὴλ, ἐξέλιπον ὁδοὺς, καὶ ἐπορεύθησαν ἀτραποὺς, ἐπορεύθησαν ὁδοὺς διεστραμμένας.
\vs{7}Ἐξέλιπον δυνατοὶ ἐν Ἰσραὴλ, ἐξέλιπον ἕως οὗ ἀνέστη Δεββῶρα, ἕως οὗ ἀνέστη μήτηρ ἐν Ἰσραήλ.
\vs{8}Ἐξελέξαντο θεοὺς καινοὺς, τότε ἐπολέμησαν πόλεις ἀρχόντων· θυρεὸς ἐὰν ὀφθῇ καὶ λόγχη ἐν τεσσαράκοντα χιλιάσιν ἐν Ἰσραήλ.

\vs{9}Ἡ καρδία μου εἰς τὰ διατεταγμένα τῷ Ἰσραήλ· οἱ ἑκούσιαζόμενοι ἐν λαῷ εὐλογεῖτε Κύριον.
\vs{10}Ἐπιβεβηκότες ἐπὶ ὄνου θηλείας μεσημβρίας, καθήμενοι ἐπὶ κριτηρίου, καὶ πορευόμενοι ἐπὶ ὁδοὺς συνέδρων ἐφʼ ὁδῷ,
\vs{11}διηγεῖσθε, ἀπὸ φωνῆς ἀνακρουομένων ἀναμέσον ὑδρευομένων· ἐκεῖ δώσουσι δικαιοσύνας· Κύριε δικαιοσύνας αὔξησον ἐν Ἰσραήλ· τότε κατέβη εἰς τὰς πόλεις λαὸς Κυρίου.
\vs{12}Ἐξεγείρου, ἐξεγείρου, Δεββῶρα· ἐξεγείρου, ἐξεγείρου, λάλησον ᾠδήν· ἀνάστα Βαρὰκ, καὶ αἰχμαλώτισον αἰχμάλωσίαν σου υἱὸς Ἀβινεέμ.
\vs{13}Τότε κατέβη κατάλειμμα τοῖς ἰσχυροῖς· λαὸς Κυρίου κατέβη αὐτῷ ἐν τοῖς κραταιοῖς ἐξ ἐμοῦ.

\vs{14}Ἐφραῒμ ἐξεῤῥίζωσεν αὐτοὺς ἐν τῷ Ἀμαλὴκ, ὀπίσω σου Βενιαμὶν ἐν τοῖς λαοῖς σου· ἐν ἐμοὶ Μαχὶρ κατέβησαν ἐξερευνῶντες· καὶ ἀπὸ Ζαβουλὼν ἕλκοντες ἐν ῥάβδῳ διηγήσεως γραμματέως.
\vs{15}Καὶ ἀρχηγοι ἐν Ἰσσάχαρ μετὰ Δεββώρας καὶ Βαράκ· οὕτω Βαρὰκ ἐν κοιλάσιν ἀπέστειλεν ἐν ποσὶν αὐτοῦ, εἰς τὰς μερίδας Ῥουβὴν, μεγάλοι ἐξικνούμενοι καρδίαν.
\vs{16}Εἰς τί ἐκάθισαν ἀναμέσον τῆς διγομίας τοῦ ἀκοῦσαι συρισμοῦ ἀγελῶν εἰς διαιρέσεις Ῥουβήν; μεγάλοι ἐξετασμοὶ καρδίας.
\vs{17}Γαλαάδ ἐν τῷ πέραν τοῦ Ἰορδάνου οὗ ἐσκήνωσε· καὶ Δὰν εἰς τί παροικεῖ πλοίοις; Ἀσὴρ ἐκάθισε παραλίαν θαλασσῶν, καὶ ἐπὶ διεξόδοις αὐτοῦ σκηνώσει.
\vs{18}Ζαβουλὼν λαὸς ὠνείδισε ψυχὴν αὐτοῦ εἰς θάνατον, καὶ Νεφθαλὶ ἐπὶ ὕψη ἀγροῦ

\vs{19}ἦλθον αὐτῶν. Βασιλεῖς παρετάξαντο, τότε ἐπολέμησαν βασιλεῖς Χαναὰν ἐν Θαναὰχ ἐπὶ ὕδατι Μαγεδδὼ, δῶρον ἀργυρίου οὐκ ἔλαβον.
\vs{20}Ἐξ οὐρανοῦ παρετάξαντο οἱ ἀστέρες, ἐκ τρίβων αὐτῶν παρετάξαντο μετὰ Σισάρα.
\vs{21}Χειμάῤῥους Κισῶν ἐξέσυρεν αὐτοὺς, χειμάῤῥους ἀρχαίων, χειμάῤῥους Κισῶν· καταπατήσει αὐτὸν ψυχή μου δυνατή.
\vs{22}Ὅτε ἐνεποδίσθησαν πτέρναι ἵππου, σπουδῇ ἔσπευσαν ἰσχυροὶ αὐτοῦ
\vs{23}καταρᾶσθαι Μηρὼζ, εἶπεν ἄγγελος Κυρίου, καταρᾶσθε· ἐπικατάρατος πᾶς ὁ κατοικῶν αὐτὴν, ὅτι οὐκ ἤλθοσαν εἰς βοήθειαν Κυρίου, εἰς βοήθειαν ἐν δυνατοῖς.

\vs{24}Εὐλογηθείη ἐν γυναιξὶν Ἰαὴλ γυνὴ Χαβὲρ τοῦ Κιναίου, ἀπὸ γυναικῶν ἐν σκηναῖς εὐλογηθείη.
\vs{25}Ὕδωρ ᾔτησε, γάλα ἔδωκεν ἐν λεκάνῃ· ὑπερεχόντων προσήνεγκε βούτυρον.
\vs{26}Χεῖρα αὐτῆς ἀριστερὰν εἰς πάσσαλον ἐξέτεινε, καὶ δεξιὰν αὐτῆς εἰς σφύραν κοπίωντων, καὶ ἐσφυροκόπησε Σισάρα, διήλωσε κεφαλὴν αὐτοῦ καὶ ἐπάταξε, διήλωσε κρόταφον αὐτοῦ.
\vs{27}Ἀναμέσον τῶν ποδῶν αὐτῆς κατεκυλίσθη· ἔπεσε καὶ ἐκοιμήθη ἀναμέσον τῶν ποδῶν αὐτῆς, κατακλιθεὶς ἔπεσε· καθὼς κατεκλίθη ἐκεῖ ἔπεσεν ἐξοδευθείς.

\vs{28}Διὰ τῆς θυρίδος παρέκυψε μήτηρ Σισάρα ἐκτὸς τοῦ τοξικοῦ, διότι ᾐσχύνθη ἅρμα αὐτοῦ; διότι ἐχρόνισαν πόδες ἁρμάτων αὐτοῦ;
\vs{29}Αἱ σοφαὶ ἄρχουσαι αὐτῆς ἀπεκρίθησαν πρὸς αὐτὴν, καὶ αὐτὴ ἀπέστρεψε λόγους αὐτῆς ἑαυτῇ,
\vs{30}οὐχ εὑρήσουσιν αὐτὸν διαμερίζοντα σκῦλα; οἰκτίρμων οἰκτειρήσει εἰς κεφαλὴν ἀνδρός· σκῦλα βαμμάτων τῷ Σισάρᾳ, σκῦλα βαμμάτων ποικιλίας, βάμματα ποικιλτῶν αὐτὰ τῷ τραχήλῳ αὐτοῦ σκῦλα.
\vs{31}Οὕτως ἀπόλοιντο πάντες οἱ ἐχθροί σου, Κύριε· καὶ οἱ ἀγαπῶντες αὐτὸν, ὡς ἔξοδος ἡλίου ἐν δυνάμει αὐτοῦ.

Καὶ ἡσύχασεν ἡ γῆ τεσσαράκοντα ἔτη.

\ch{6}
Καὶ ἐποίησαν οἱ υἱοὶ Ἰσραὴλ τὸ πονηρὸν ἐνώπιον Κυρίου, καὶ ἔδωκεν αὐτοὺς Κύριος ἐν χειρὶ Μαδιὰμ ἑπτὰ ἔτη.
\vs{2}Καὶ ἴσχυσε χεὶρ Μαδιὰμ ἐπὶ Ἰσραήλ· καὶ ἐποίησαν ἑαυτοῖς οἱ υἱοὶ Ἰσραὴλ ἀπὸ προσώπου Μαδιὰμ τὰς τρυμαλιὰς τὰς ἐν τοῖς ὄρεσι, καὶ τὰ σπήλαια, καὶ τὰ κρεμαστά.
\vs{3}Καὶ ἐγένετο ἐὰν ἔσπειραν οἱ υἱοὶ Ἰσραὴλ, καὶ ἀνέβαινον Μαδιὰμ καὶ Ἀμαλὴκ, καὶ οἱ υἱοὶ ἀνατολῶν συνανέβαινον αὐτοῖς, καὶ παρενέβαλον εἰς αὐτοὺς,
\vs{4}καὶ διέφθειρον τοὺς καρποὺς αὐτῶν ἕως ἐλθεῖν εἰς Γάζαν· καὶ οὐ κατελείποντο ὑπόστασιν ζωῆς ἐν τῇ γῇ Ἰσραὴλ, οὐδὲ ἐν τοῖς ποιμνίοις ταῦρον καὶ ὄνον.
\vs{5}Ὅτι αὐτοὶ καὶ αἱ κτήσεις αὐτῶν ἀνέβαινον, καὶ αἱ σκηναὶ αὐτῶν παρεγίνοντο, καθὼς ἀκρὶς εἰς πλῆθος, καὶ αὐτοῖς καὶ ταῖς καμήλοις αὐτῶν οὐκ ἦν ἀριθμός· καὶ ἤρχοντο εἰς τὴν γῆν Ἰσραὴλ, καὶ διέφθειρον αὐτήν.
\vs{6}Καὶ ἐπτώχευσεν Ἰσραὴλ σφόδρα ἀπὸ προσώπου Μαδιάμ.
\vs{7}Καὶ ἐβόησαν οἱ υἱοὶ Ἰσραὴλ πρὸς Κύριον ἀπὸ προσώπου Μαδιάμ.

\vs{8}Καὶ ἐξαπέστειλε Κύριος ἄνδρα προφήτην πρὸς τοὺς νἱοὺς Ἰσραήλ· καὶ εἶπεν αὐτοῖς, τάδε λέγει Κύριος ὁ Θεὸς Ἰσραὴλ, ἐγώ εἰμι ὃς ἀνήγαγον ὑμᾶς ἐκ γῆς Αἰγύπτου, καὶ ἐξήγαγον ὑμᾶς ἐξ οἴκου δουλείας ὑμῶν·
\vs{9}καὶ ἐῤῥυσάμην ὑμᾶς ἐκ χειρὸς Αἰγύπτου καὶ ἐκ χειρὸς πάντων τῶν θλιβόντων ὑμᾶς, καὶ ἐξέβαλον αὐτοὺς ἐκ προσώπου ὑμῶν· καὶ ἔδωκα ὑμῖν τὴν γῆν αὐτῶν.
\vs{10}Καὶ εἶπα ὑμῖν, ἐγὼ Κύριος ὁ Θεὸς ὑμῶν· οὐ φοβηθήσεσθε τοὺς θεοὺς τοῦ Ἀμοῤῥαίου, ἐν οἷς ὑμεῖς κάθησθε ἐν τῇ γῇ αὐτῶν· καὶ οὐκ εἰσηκούσατε τῆς φωνῆς μου.

\vs{11}Καὶ ἦλθεν ἄγγελος Κυρίου, καὶ ἐκάθισεν ὑπὸ τὴν τερέμινθον τὴν ἐν Ἐφραθὰ ἐν γῇ Ἰωὰς πατρὸς τοῦ Ἐσδρί· καὶ Γεδεὼν ὁ υἱὸς αὐτοῦ ῥαβδίζων σῖτον ἐν ληνῷ εἰς ἐκφυγεῖν ἀπὸ προσώπου τοῦ Μαδίαμ.
\vs{12}Καὶ ὤφθη αὐτῷ ὁ ἄγγελος Κυρίου, καὶ εἶπε πρὸς αὐτὸν, Κύριος μετὰ σοῦ, ἰσχυρὸς τῶν δυνάμεων.
\vs{13}Καὶ εἶπε πρὸς αὐτὸν Γεδεὼν, ἐν ἐμοὶ, Κύριέ μου· καὶ εἰ ἔστι Κύριος μεθʼ ἡμῶν, εἰς τί εὗρεν ἡμᾶς τὰ κακὰ ταῦτα; καὶ ποῦ ἐστι πάντα τὰ θαυμάσια αὐτοῦ, ἃ διηγήσαντο ἡμῖν οἱ πατέρες ἡμῶν, λέγοντες, μὴ οὐχὶ ἐξ Αἰγύπτου ἀνήγαγεν ἡμᾶς Κύριος; καὶ νῦν ἐξέῤῥιψεν ἡμᾶς καὶ ἔδωκεν ἡμᾶς ἐν χειρὶ Μαδίαμ.
\vs{14}Καὶ ἐπέστρεψε πρὸς αὐτὸν ὁ ἄγγελος Κυρίου, καὶ εἶπε, πορεύου ἐν τῇ ἰσχύϊ σου ταύτῃ, καὶ σώσεις τὸν Ἰσραὴλ ἐκ χειρὸς Μαδιάμ· ἰδοὺ ἐξαπέστειλά σε.
\vs{15}Καὶ εἶπε πρὸς αὐτὸν Γεδεὼν, ἐν ἐμοὶ, Κύριέ μου, ἐν τίνι σώσω τὸν Ἰσραήλ; ἰδοὺ ἡ χιλιάς μου ἠσθένησεν ἐν Μανασσῇ, καὶ ἐγώ εἰμι μικρότερος ἐν οἴκῳ τοῦ πατρός μου.
\vs{16}Καὶ εἶπε πρὸς αὐτὸν ὁ ἄγγελος Κυρίου, Κύριος ἔσται μετὰ σοῦ, καὶ πατάξεις τὴν Μαδιὰμ ὡσεὶ ἄνδρα ἕνα.
\vs{17}Καὶ εἶπε πρὸς αὐτὸν Γεδεὼν, εἰ δὴ εὗρον ἔλεος ἐν ὀφθαλμοῖς σου, καὶ ποιήσεις μοι σήμερον πᾶν ὅτι, ἐλάλησας μετʼ ἐμοῦ,
\vs{18}μὴ χωρισθῇς ἐντεῦθεν ἕως τοῦ ἐλθεῖν με πρὸς σὲ, καὶ ἐξοίσω τὴν θυσίαν καὶ θύσω ἐνώπιόν σου· καὶ εἶπεν, ἐγώ εἰμι καθίσομαι ἕως τοῦ ἐπιστρέψαι σε.

\vs{19}Καὶ Γεδεὼν εἰσῆλθε, καὶ ἐποίησεν ἔριφον αἰγῶν καὶ οἰφὶ ἀλεύρου ἄζυμα, καὶ τὰ κρέα ἔθηκεν ἐν τῷ κοφίνῳ, καὶ τὸν ζωμὸν ἔβαλεν ἐν τῇ χύτρᾳ, καὶ ἐξήνεγκεν αὐτὰ πρὸς αὐτὸν ὑπὸ τὴν τερέμινθον, καὶ προσήγγισε.
\vs{20}Καὶ εἶπε πρὸς αὐτὸν ὁ ἄγγελος τοῦ Θεοῦ, λάβε τὰ κρέα καὶ τὰ ἄζυμα, καὶ θὲς πρὸς τὴν πέτραν ἐκείνην, καὶ τὸν ζωμὸν ἐχόμενα ἔκχεε· καὶ ἐποιήσεν οὕτως.
\vs{21}Καὶ ἐξέτεινεν ὁ ἄγγελος Κυρίου τὸ ἄκρον τῆς ῥάβδου τῆς ἐν τῇ χειρὶ αὐτοῦ, καὶ ἥψατο τῶν κρεῶν καὶ τὼν ἀζύμων· καὶ ἀνέβη πῦρ ἐκ τῆς πέτρας, καὶ κατέφαγε τὰ κρέα καὶ τοὺς ἀζύμους· καὶ ὁ ἄγγελος Κυρίου ἐπορεύθη ἀπʼ ὀφθαλμῶν αὐτοῦ.

\vs{22}Καὶ εἶδε Γεδεὼν, ὅτι ἄγγελος Κυρίου οὗτός ἐστι· καὶ εἶπε Γεδεὼν, ἆ ἆ, Κύριέ μου Κύριε, ὅτι εἶδον τὸν ἄγγελον Κυρίου πρόσωπον πρὸς πρόσωπον.
\vs{23}Καὶ εἶπεν αὐτῷ Κύριος, εἰρήνη σοι, μὴ φοβοῦ, οὐ μὴ ἀποθάνῃς.

\vs{24}Καὶ ᾠκοδόμησεν ἐκεῖ Γεδεὼν θυσιαστήριον τῷ Κυρίῳ, καὶ ἐπεκάλεσεν αὐτῷ, εἰρήνη Κυρίου, ἕως τῆς ἡμέρας ταύτης, ἔτι αὐτοῦ ὄντος ἐν Ἐφραθὰ πατρὸς τοῦ Ἐσδρί.
\vs{25}Καὶ ἐγένετο ἐν τῇ νυκτὶ ἐκείνῃ, καὶ εἶπεν αὐτῷ Κύριος, λάβε τὸν μόσχον τὸν ταῦρον ὅς ἐστι τῷ πατρί σου, καὶ μόσχον δεύτερον ἑπταετῇ, καὶ καθελεῖς τὸ θυσιαστήριον τοῦ Βάαλ ὅ ἐστι τῷ πατρί σου, καὶ τὸ ἄλσος τὸ ἐπʼ αὐτὸ ὀλοθρεύσεις.
\vs{26}Καὶ οἰκοδομήσεις θυσιαστήριον τῷ Κυρίῳ τῷ Θεῷ σου ἐπὶ κορυφὴν Μαωζὶ τούτου ἐν τῇ παρατάξει καὶ λήψῃ τὸν μόσχον τὸν δεύτερον, καὶ ἀνοίσεις ὁλοκαυτώματα ἐν τοῖς ξύλοις τοῦ ἄλσοῦς, οὗ ἐξολοθρεύσεις.
\vs{27}Καὶ ἔλαβε Γεδεὼν δέκα ἄνδρας ἀπὸ τῶν δούλων ἑαυτοῦ, καὶ ἐποίησεν ὃν τρόπον ἐλάλησε πρὸς αὐτὸν Κύριος· καὶ ἐγενήθη ὡς ἐφοβήθη τὸν οἶκον τοῦ πατρὸς αὐτοῦ καὶ τοὺς ἄνδρας τῆς πόλεως τοῦ ποιῆσαι ἡμέρας, καὶ ἐποίησε νυκτός.

\vs{28}Καὶ ὤρθρισαν οἱ ἄνδρες τῆς πόλεως τοπρωΐ· καὶ ἰδοὺ καθῄρητο τὸ θυσιαστήριον τοῦ Βάαλ, καὶ τὸ ἄλσος τὸ ἐπʼ αὐτῷ ὠλόθρευτο· καὶ εἶδον τὸν μόσχον τὸν δεύτερον, ὃν ἀνήνεγκεν ἐπὶ τὸ θυσιαστήριον τὸ ᾠκοδομημένον.
\vs{29}Καὶ εἶπεν ἀνὴρ πρὸς τὸν πλησίον αὐτοῦ, τίς ἐποίησε τὸ ῥῆμα τοῦτο; καὶ ἐπεζήτησαν καὶ ἠρεύνησαν, καὶ ἔγνωσαν ὅτι Γεδεὼν υἱὸς Ἰωὰς ἐποίησε τὸ ῥῆμα τοῦτο.
\vs{30}Καὶ εἶπαν οἱ ἄνδρες τῆς πόλεως πρὸς Ἰωὰς, ἐξένεγκε τὸν υἱόν σου, καὶ ἀποθανέτω, ὅτι καθεῖλε τὸ θυσιαστήριον τοῦ Βάαλ, καὶ ὅτι ὠλόθρευσε τὸ ἄλσος τὸ ἐπʼ αὐτῷ.
\vs{31}Καὶ εἶπε Γεδεὼν υἱὸς Ἰωὰς τοῖς ἀνδράσι πᾶσιν, οἳ ἐπανέστησαν αὐτῷ, μὴ ὑμεῖς νῦν δικάζεσθε ὑπὲρ τοῦ Βάαλ; ἢ ὑμεῖς σώσετε αὐτόν; ὃς ἐὰν δικάσηται αὐτῷ, θανατωθήτω ἕως πρωΐ· εἰ θεός ἐστι, δικαζέσθω αὐτῷ, ὅτι καθεῖλε τὸ θυσιαστήριον αὐτοῦ.
\vs{32}Καὶ ἐκάλεσεν αὐτὸ ἐν τῇ ἡμέρᾳ ἐκείνῃ Ἱεροβάαλ, λέγων, δικαζέσθω ἐν αὐτῷ ὁ Βάαλ, ὅτι καθῃρέθη τὸ θυσιαστήριον αὐτοῦ.

\vs{33}Καὶ πᾶσα Μαδιὰμ, καὶ Ἀμαλὴκ, καὶ υἱοὶ ἀνατολῶν συνήχθησαν ἐπὶ τοαυτὸ, καὶ παρενέβαλον ἐν τῇ κοιλάδι Ἰεζραέλ.
\vs{34}Καὶ πνεῦμα Κυρίου ἐνέδυσε τὸν Γεδεὼν, καὶ ἐσάλπισεν ἐν κερατίνῃ, καὶ ἐβόησεν Ἀβιέζερ ὀπίσω αὐτοῦ.
\vs{35}Καὶ ἀγγέλους ἐξαπέστειλεν εἰς πάντα Μανασσῆ, καὶ ἐν Ἀσὴρ, καὶ ἐν Ζαβουλὼν, καὶ ἐν Νεφθαλί· καὶ ἀνέβη εἰς συνάντησιν αὐτῶν.

\vs{36}Καὶ εἶπε Γεδεὼν πρὸς τὸν Θεὸν, εἰ σὺ σώζεις ἐν χειρί μου τὸν Ἰσραὴλ, καθὼς ἐλάλησας,
\vs{37}ἰδοὺ ἐγὼ τίθημι τὸν πόκον τοῦ ἐρίου ἐν τῇ ἅλωνι· ἐὰν δρόσος γένηται ἐπὶ τὸν πόκον μόνον, καὶ ἐπὶ πᾶσαν τὴν γῆν ξηρασία, γνώσομαι ὅτι σώσεις ἐν χειρί μου τὸν Ἰσραὴλ, καθὼς ἐλάλησας.
\vs{38}Καὶ ἐγένετο οὕτως· καὶ ὤρθρισε τῇ ἐπαύριον, καὶ ἐξεπίασε τὸν πόκον, καὶ ἔσταξε δρόσος ἀπὸ τοῦ πόκου πλήρης λεκάνη ὕδατος.
\vs{39}Καὶ εἶπε Γεδεὼν πρὸς τὸν Θεὸν, μὴ δὴ ὀργισθήτω ὁ θυμός σου ἐν ἐμοὶ, καὶ λαλήσω ἔτι ἅπαξ· πειράσω δὴ καί γε ἔτι ἅπαξ ἐν τῷ πόκῳ· καὶ γενέσθω ἡ ξηρασία ἐπὶ τὸν πόκον μόνον, καὶ ἐπὶ πᾶσαν τὴν γῆν γενηθήτω δρόσος.
\vs{40}Καὶ ἐποίησεν ὁ Θεὸς οὕτως ἐν τῇ νυκτὶ ἐκείνῃ· καὶ ἐγένετο ξηρασία ἐπὶ τὸν πόκον μόνον, καὶ ἐπὶ πᾶσαν τὴν γῆν ἐγενήθη δρόσος.

\ch{7}
Καὶ ὤρθρισεν Ἱεροβάαλ, αὐτός ἐστι Γεδεὼν, καὶ πᾶς ὁ λαὸς μετʼ αὐτοῦ, καὶ παρενέβαλον ἐπὶ πηγὴν Ἀράδ· καὶ παρεμβολὴ Μαδιὰμ ἦν αὐτῷ ἀπὸ Βοῤῥᾶ ἀπὸ Γαβααθαμωραὶ ἐν κοιλάδι.

\vs{2}Καὶ εἶπε Κύριος πρὸς Γεδεὼν, πολὺς ὁ λαὸς ὁ μετὰ σοῦ, ὥστε μὴ παραδοῦναί με τὴν Μαδιὰμ ἐν χειρὶ αὐτῶν, μὴ ποτε καυχήσηται Ἰσραὴλ ἐπʼ ἐμὲ, λέγων, ἡ χείρ μου ἔσωσέ με.
\vs{3}Καὶ νῦν λάλησον δὴ ἐν ὠσὶ τοῦ λαοῦ, λέγων, τίς ὁ φοβούμενος καὶ δειλός; ἐπιστραφέτω καὶ ἐκχωρείτω ἀπὸ ὄρους Γαλαάδ· καὶ ἐπέστρεψεν ἀπὸ τοῦ λαοῦ εἴκοσι καὶ δύο χιλιάδες, καὶ δέκα χιλιάδες ὑπελείφθησαν.
\vs{4}Καὶ εἶπε Κύριος πρὸς Γεδεὼν, ἔτι ὁ λαὸς πολύς ἐστι· κατένεγκον αὐτοὺς πρὸς τὸ ὕδωρ, καὶ ἐκκαθαρῶ σοι αὐτὸν ἐκεῖ· καὶ ἔσται ὃν ἐὰν εἴπω πρὸς σὲ, οὗτος πορεύσεται σὺν σοὶ, αὐτὸς πορεύσεται σὺν σοί· καὶ πᾶς ὃν ἂν εἴπω πρὸς σὲ, οὗτος οὐ πορεύσεται μετὰ σοῦ, αὐτὸς οὐ πορεύσεται μετὰ σοῦ.
\vs{5}Καὶ κατήνεγκε τὸν λαὸν πρὸς τὸ ὕδωρ· καὶ εἶπε Κύριος πρὸς Γεδεὼν, πᾶς ὃς ἂν λάψῃ τῇ γλώσσῃ αὐτοῦ ἀπὸ τοῦ ὕδατος ὡς ἐὰν λάψῃ ὁ κύων, στήσεις αὐτὸν κατὰ μόνας, καὶ πᾶς ὃς ἐὰν κλίνῃ ἐπὶ τὰ γόνατα αὐτοῦ πιεῖν.
\vs{6}Καὶ ἐγένετο ὁ ἀριθμὸς τῶν λαψάντων ἐν χειρὶ αὐτῶν πρὸς τὸ στόμα αὐτῶν, τριακόσιοι ἄνδρες· καὶ πᾶν τὸ κατάλοιπον τοῦ λαοῦ ἔκλιναν ἐπὶ τὰ γόνατα αὐτῶν πιεῖν ὕδωρ.
\vs{7}Καὶ εἶπε Κύριος πρὸς Γεδεὼν, ἐν τοῖς τριακοσίοις ἀνδράσι τοῖς λάψασι σώσω ὑμᾶς, καὶ δώσω τὴν Μαδιὰμ ἐν χειρί σου, καὶ πᾶς ὁ λαὸς πορεύσονται ἀνὴρ εἰς τὸν τόπον αὐτοῦ.
\vs{8}Καὶ ἔλαβον τὸν ἐπισιτισμὸν τοῦ λαοῦ ἐν χειρὶ αὐτῶν, καὶ τὰς κερατίνας αὐτῶν· καὶ τὸν πάντα ἄνδρα Ἰσραὴλ ἐξαπέστειλεν ἄνδρα εἰς σκηνὴν αὐτοῦ· καὶ τοὺς τριακοσίους ἄνδρας κατίσχυσε· καὶ ἡ παρεμβολὴ Μαδιὰμ ἦσαν αὐτοῦ ὑποκάτω ἐν τῇ κοιλάδι.

\vs{9}Καὶ ἐγενήθη ἐν τῇ νυκτὶ ἐκείνῃ, καὶ εἶπε πρὸς αὐτὸν Κύριος, ἀνάστα, κατάβηθι ἐν τῇ παρεμβολῇ, ὅτι παρέδωκα αὐτὴν ἐν τῇ χειρί σου.
\vs{10}Καὶ εἰ φοβῇ σὺ καταβῆναι, κατάβηθι σὺ καὶ Φαρὰ τὸ παιδάριόν σου εἰς τὴν παρεμβολὴν,
\vs{11}καὶ ἀκούσῃ τί λαλήσουσι, καὶ μετὰ τοῦτο ἰσχύσουσιν αἱ χεῖρές σου καὶ καταβήσῃ ἐν τῇ παρεμβολῇ· καὶ κατέβη αὐτὸς καὶ Φαρὰ τὸ παιδάριον αὐτοῦ πρὸς ἀρχὴν τῶν πεντήκοντα, οἳ ἦσαν ἐν τῇ παρεμβολῇ.
\vs{12}Καὶ Μαδιὰμ καὶ Ἀμαλὴκ καὶ πάντες οἱ υἱοὶ ἀνατολῶν βεβλημένοι ἐν τῇ κοιλάδι ὡς ἀκρὶς εἰς πλῆθος, καὶ ταῖς καμήλοις αὐτῶν οὐκ ἦν ἀριθμὸς, ἀλλʼ ἦσαν ὡς ἡ ἄμμος ἡ ἐπὶ χείλους τῆς θαλάσσης εἰς πλῆθος.

\vs{13}Καὶ ἦλθε Γεδεὼν, καὶ ἰδοὺ ἀνὴρ ἐξηγούμενος τῷ πλησίον αὐτοῦ ἐνύπνιον, καὶ εἶπεν, ἰδοὺ ἐνυπνιασάμην ἐνύπνιον, καὶ ἰδοὺ μαγὶς ἄρτου κριθίνου στρεφομένη ἐν τῇ παρεμβολῇ Μαδιὰμ, καὶ ἦλθεν ἕως τῆς σκηνῆς, καὶ ἐπάταξεν αὐτὴν, καὶ ἔπεσε, καὶ ἀνέστρεψεν αὐτὴν ἄνω, καὶ ἔπεσεν ἡ σκηνή.
\vs{14}Καὶ ἀπεκρίθη ὁ πλησίον αὐτοῦ, καὶ εἶπεν, οὐκ ἔστιν αὕτη εἰ μὴ ῥομφαία Γεδεὼν υἱοῦ Ἰωὰς ἀνδρὸς Ἰσραήλ· παρέδωκεν ὁ Θεὸς ἐν χειρὶ αὐτοῦ τὴν Μαδιὰμ καὶ πᾶσαν τὴν παρεμβολήν.

\vs{15}Καὶ ἐγένετο ὡς ἤκουσε Γεδεὼν τὴν ἐξήγησιν τοῦ ἐνυπνίου καὶ τὴν σύγκρισιν αὐτοῦ, καὶ προσεκύνησε Κυρίῳ, καὶ ὑπέστρεψεν εἰς τὴν παρεμβολὴν Ἰσραὴλ, καὶ εἶπεν, ἀνάστητε, ὅτι παρέδωκε Κύριος ἐν χειρὶ ἡμῶν τὴν παρεμβολὴν Μαδιάμ.
\vs{16}Καὶ διεῖλε τοὺς τριακοσίους ἄνδρας εἰς τρεῖς ἀρχὰς, καὶ ἔδωκε κερατίνας ἐν χειρὶ πάντων, καὶ ὑδρίας κενὰς, καὶ λαμπάδας ἐν ταῖς ὑδρίαις,
\vs{17}καὶ εἶπε πρὸς αὐτοὺς, ἀπʼ ἐμοῦ ὄψεσθε, καὶ οὕτω ποιήσετε· καὶ ἰδοὺ ἐγὼ εἰσπορεύομαι ἐν ἀρχῇ τῆς παρεμβολῆς, καὶ ἔσται καθὼς ἂν ποιήσω, οὕτω ποιήσετε.
\vs{18}Καὶ σαλπιῶ ἐν τῇ κερατίνῃ ἐγὼ, καὶ πάντες μετʼ ἐμοῦ σαλπιεῖτε ἐν ταῖς κερατίναις κύκλῳ ὅλης τῆς παρεμβολῆς, καὶ ἐρεῖτε, τῷ Κυρίῳ καὶ τῷ Γεδεών.

\vs{19}Καὶ εἰσῆλθε Γεδεὼν καὶ οἱ ἐκατὸν ἄνδρες οἱ μετʼ αὐτοῦ ἐν ἀρχῇ τῆς παρεμβολῆς ἐν ἀρχῇ τῆς φυλακῆς μέσης· καὶ ἐγείροντες ἤγειραν τοὺς φυλάσσοντας, καὶ ἐσάλπισαν ἐν ταῖς κερατίναις, καὶ ἐξετίναξαν τὰς ὑδρίας τὰς ἐν ταῖς χερσὶν αὐτῶν.
\vs{20}Καὶ ἐσάλπισαν αἱ τρεῖς ἀρχαὶ ἐν ταῖς κερατίναις, καὶ συνέτριψαν τὰς ὑδρίας, καὶ ἐκράτησαν ἐν χερσὶν ἀριστεραῖς αὐτῶν τὰς λαμπάδας, καὶ ἐν χερσὶ δεξιαῖς αὐτῶν τὰς κερατίνας τοῦ σαλπίζειν· καὶ ἀνέκραξαν, ῥομφαία τῷ Κυρίῳ καὶ τῷ Γεδεών.
\vs{21}Καὶ ἔστησεν ἀνὴρ ἐφʼ ἑαυτῷ κύκλῳ τῆς παρεμβολῆς· καὶ ἔδραμε πᾶσα ἡ παρεμβολὴ, καὶ ἐσήμαναν, καὶ ἔφυγον.
\vs{22}Καὶ ἐσάλπισαν ἐν ταῖς τριακοσίαις κερατίναις· καὶ ἔθηκε Κύριος τὴν ῥομφαίαν ἀνδρὸς ἐν τῷ πλησίον αὐτοῦ ἐν πάσῃ τῇ παρεμβολῇ.
\vs{23}Καὶ ἔφυγεν ἡ παρεμβολὴ ἕως Βηθσεὲδ Ταγαραγαθὰ Ἀβελμεουλὰ ἐπὶ Ταβάθ· καὶ ἐβόησαν ἀνὴρ Ἰσραὴλ ἀπὸ Νεφθαλὶ καὶ ἀπὸ Ἀσὴρ, καὶ ἀπὸ παντὸς Μανασσῆ, καὶ ἐδίωξαν ὀπίσω Μαδιάμ.

\vs{24}Καὶ ἀγγέλους ἀπέστειλε Γεδεὼν ἐν παντὶ ὄρει Ἐφραὶμ, λέγων, κατάβητε εἰς συνάντησιν Μαδιὰμ, καὶ καταλάβετε ἑαυτοῖς τὸ ὕδωρ ἕως Βαιθηρὰ καὶ τὸν Ἰορδάνην· καὶ ἐβόησε πᾶς ἀνὴρ Ἐφραὶμ, καὶ προκατελάβοντο τὸ ὕδωρ ἕως Βαιθηρὰ καὶ τὸν Ἰορδάνην.
\vs{25}Καὶ συνελάβοντο τοὺς ἄρχοντας Μαδιὰμ, καὶ τὸν Ὠρὴβ καὶ τὸν Ζήβ· καὶ ἀπέκτειναν τὸν Ὠρὴβ ἐν Σοὺρ Ὠρὴβ, καὶ τὸν Ζὴβ ἀπέκτειναν ἐν Ἱακεφζήφ· καὶ κατεδίωξαν τὴν Μαδιάμ· καὶ τὴν κεφαλὴν Ὠρὴβ καὶ Ζὴβ ἤνεγκαν πρὸς Γεδεὼν ἀπὸ πέραν τοῦ Ἰορδάνου.

\ch{8}
Καὶ εἶπαν πρὸς Γεδεὼν ἀνὴρ Ἐφραὶμ, τί τὸ ῥῆμα τοῦτο ἐποίησας ἡμῖν, τοῦ μὴ καλέσαι ἡμᾶς ὅτε ἐπορεύθης παρατάξασθαι ἐν Μαδιάμ; καὶ διελέξαντο πρὸς αὐτὸν ἰσχυρῶς.
\vs{2}Καὶ εἶπε πρὸς αὐτοὺς, τί ἐποίησα νῦν καθὼς ὑμεῖς; ἢ οὐχὶ κρείττῶν ἐπιφυλλὶς Ἐφραὶμ ἢ τρυγητὸς Ἀβιέζερ;
\vs{3}Ἐν χειρὶ ὑμῶν παρέδωκε Κύριος τοὺς ἄρχοντας Μαδιὰμ, τὸν Ὠρὴβ καὶ τὸν Ζήβ· καὶ τί ἠδυνήθην ποιῆσαι ὡς ὑμεῖς; τότε ἀνέθη τὸ πνεῦμα αὐτῶν ἀπʼ αὐτοῦ ἐν τῷ λαλῆσαι αὐτὸν τὸν λόγον τοῦτον.

\vs{4}Καὶ ἦλθε Γεδεὼν ἐπὶ τὸν Ἰορδανὴν, καὶ διέβη αὐτὸς καὶ οἱ τριακόσιοι ἄνδρες οἱ μετʼ αὐτοῦ πεινῶντες καὶ διώκοντες.
\vs{5}Καὶ εἶπε τοῖς ἀνδράσι Σοκχὼθ, δότε δὴ ἄρτους εἰς τροφὴν τῷ λαῷ τούτῳ τῷ ἐν ποσί μου, ὅτι ἐκλείπουσι, καὶ ἰδοὺ ἐγώ εἰμι διώκων ὀπίσω τοῦ Ζεβεὲ καὶ Σαλμανὰ βασιλέων Μαδιάμ.
\vs{6}Καὶ εἶπον οἱ ἄρχοντες Σοκχὼθ, μὴ χεὶρ Ζεβεὲ καὶ Σαλμανὰ νῦν ἐν χειρί σου, ὅτι δώσομεν τῇ δυνάμει σου ἄρτους;
\vs{7}Καὶ εἶπε Γεδεὼν, διὰ τοῦτο ἐν τῷ δοῦναι Κύριον τὸν Ζεβεὲ καὶ τὸν Σαλμανὰ ἐν χειρί μου, καὶ ἐγὼ ἀλοήσω τὰς σάρκας ὑμῶν ἐν ταῖς ἀκάνθαις τῆς ἐρήμου, καὶ ἐν ταῖς Βαρκηνίμ.
\vs{8}Καὶ ἀνέβη ἐκεῖθεν εἰς Φανουὴλ, καὶ ἐλάλησε πρὸς αὐτοὺς ὡσαύτως· καὶ ἀπεκρίθησαν αὐτῷ οἱ ἄνδρες Φανουὴλ ὃν τρόπον ἀπεκρίθησαν ἄνδρες Σοκχώθ.
\vs{9}Καὶ εἶπε Γεδεὼν πρὸς ἄνδρας Φανουὴλ, ἐν ἐπιστροφῇ μου μετʼ εἰρήνης, κατασκάψω τὸν πύργον τοῦτον.

\vs{10}Καὶ Ζεβεὲ καὶ Σαλμανὰ ἐν Καρκὰρ, καὶ ἡ παρεμβολὴ αὐτῶν μετʼ αὐτῶν ὡσεὶ δεκαπέντε χιλιάδες, πάντες οἱ καταλελειμμένοι ἀπὸ πάσης παρεμβολῆς ἀλλοφύλων· καὶ οἱ πεπτωκότες, ἑκατὸν εἴκοσι χιλιάδες ἀνδρῶν σπωμένων ῥομφαίαν.
\vs{11}Καὶ ἀνέβη Γεδεὼν ὁδὸν τῶν σκηνούντων ἐν σκηναῖς ἀπὸ ἀνατολῶν τῆς Ναβαὶ καὶ Ἰεγεβάλ· καὶ ἐπάταξε τὴν παρεμβολὴν, καὶ ἡ παρεμβολὴ ἦν πεποιθυῖα.
\vs{12}Καὶ ἔφυγον Ζεβεὲ καὶ Σαλμανά· καὶ ἐδίωξεν ὀπίσω αὐτῶν, καὶ ἐκράτησε τοὺς δύο βασιλεῖς Μαδιὰμ τὸν Ζεβεὲ καὶ τὸν Σαλμανὰ, καὶ πᾶσαν τὴν παρεμβολὴν ἐξέστησε.

\vs{13}Καὶ ἐπέστρεψε Γεδεὼν υἱὸς Ἰωὰς ἀπὸ τῆς παρατάξεως ἀπὸ ἐπάνωθεν τῆς παρατάξεως Ἀρές.
\vs{14}Καὶ συνέλαβε παιδάριον ἀπὸ τῶν ἀνδρῶν Σοκχὼθ, καὶ ἐπηρώτησεν αὐτόν· καὶ ἔγραψε πρὸς αὐτὸν ὀνόματα τῶν ἀρχόντων Σοκχὼθ καὶ τῶν πρεσβυτέρων αὐτῶν, ἑβδομήκοντα καὶ ἑπτὰ ἄνδρας.
\vs{15}Καὶ παρεγένετο Γεδεὼν πρὸς τοὺς ἄρχοντας Σοκχὼθ, καὶ εἶπεν, ἰδοὺ Ζεβεὲ καὶ Σαλμανὰ, ἐν οἷς ὠνειδίσατέ με, λέγοντες, μὴ χεὶρ Ζεβεὲ καὶ Σαλμανὰ νῦν ἐν χειρί σου, ὅτι δώσομεν τοῖς ἀνδράσι σου τοῖς ἐκλείπουσιν ἄρτους;
\vs{16}Καὶ ἔλαβε τοὺς πρεσβυτέρους τῆς πόλεως ἐν ταῖς ἀκάνθαις τῆς ἐρήμου καὶ ταῖς Βαρκηνὶμ, καὶ ἠλόησεν ἐν αὐτοῖς τοὺς ἄνδρας τῆς πόλεως·
\vs{17}Καὶ τὸν πύργον Φανουὴλ κατέστρεψε, καὶ ἀπέκτεινε τοὺς ἄνδρας τῆς πόλεως.

\vs{18}Καὶ εἶπε πρὸς Ζεβεὲ καὶ Σαλμανὰ, ποῦ οἱ ἄνδρες, οὓς ἀπεκτείνατε ἐν Θαβώρ; καὶ εἶπαν, ὡς σὺ, ὡς αὐτοὶ, εἰς ὁμοίωμα υἱοῦ βασιλέως.
\vs{19}Καὶ εἶπε Γεδεὼν, ἀδελφοί μου καὶ υἱοὶ τῆς μητρός μου ἦσαν· ζῇ Κύριος· εἰ ἐζωογονήκειτε αὐτοὺς, οὐκ ἂν ἀπέκτεινα ὑμᾶς.
\vs{20}Καὶ εἶπεν Ἰεθὲρ τῷ πρωτοτόκῳ αὐτοῦ, ἀναστὰς, ἀπόκτεινον αὐτούς· καὶ οὐκ ἔσπασε τὸ παιδάριον τὴν ῥομφαίαν αὐτοῦ, ὅτι ἐφοβήθη, ὅτι ἔτι νεώτερος ἦν.
\vs{21}Καὶ εἶπε Ζεβεὲ καὶ Σαλμανὰ, ἀνάστα σὺ, καὶ συνάντησον ἡμῖν, ὅτι ὡς ἀνδρὸς ἡ δύναμίς σου· καὶ ἀνέστη Γεδεὼν, καὶ ἀπέκτεινε τὸν Ζεβεὲ καὶ τὸν Σαλμανά· καὶ ἔλαβε τοὺς μηνίσκους τοὺς ἐν τοῖς τραχήλοις τῶν καμήλων αὐτῶν.

\vs{22}Καὶ εἶπον ἀνὴρ Ἰσραὴλ πρὸς Γεδεὼν, κύριε, ἄρξον ἡμῶν καὶ σὺ, καὶ ὁ υἱός σου, καὶ ὁ υἱὸς τοῦ υἱοῦ σου, ὅτι σὺ ἔσωσας ἡμᾶς ἐκ χειρὸς Μαδιάμ.
\vs{23}Καὶ εἶπε πρὸς αὐτοὺς Γεδεὼν, οὐκ ἄρξω ἐγὼ, καὶ οὐκ ἄρξει ὁ υἱός μου ἐν ὑμῖν· Κύριος ἄρξει ὑμῶν.
\vs{24}Καὶ εἶπε πρὸς αὐτοὺς Γεδεὼν, αἰτήσομαι παρʼ ὑμῶν αἴτημα, καὶ δότε μοι ἀνὴρ ἐνώτιον ἐκ σκύλων αὐτοῦ· ὅτι ἐνώτια χρυσᾶ αὐτοῖς, ὅτι ἦσαν Ἰσμαηλῖται.
\vs{25}Καὶ εἶπαν, διδόντες δώσομεν· καὶ ἀνέπτυξε τὸ ἱμάτιον αὐτοῦ, καὶ ἔβαλεν ἐκεῖ ἀνὴρ ἐνώτιον σκύλων αὐτοῦ.
\vs{26}Καὶ ἐγένετο ὁ σταθμὸς τῶν ἐνωτίων τῶν χρυσῶν ὧν ᾔτησε, χίλιοι καὶ ἑπτακόσιοι χρυσοὶ, πάρεξ τῶν μηνίσκων καὶ τῶν στραγγαλίδων καὶ τῶν ἱματίων καὶ πορφυρίδων τῶν ἐπὶ βασιλεῦσι Μαδιὰμ, καὶ ἐκτὸς τῶν περιθεμάτων ἃ ἦν ἐν τοῖς τραχήλοις τῶν καμήλων αὐτῶν.
\vs{27}Καὶ ἐποίησεν αὐτὸ Γεδεὼν εἰς ἐφὼδ, καὶ ἔστησεν αὐτὸ ἐν πόλει αὐτοῦ ἐν Ἐφραθά καὶ ἐξεπόρνευσε πᾶς Ἰσραὴλ ὀπίσω αὐτοῦ ἐκεῖ· καὶ ἐγένετο τῷ Γεδεὼν καὶ τῷ οἴκῳ αὐτοῦ εἰς σκῶλον.

\vs{28}Καὶ συνεστάλη Μαδιὰμ ἐνώπιον υἱῶν Ἰσραὴλ, καὶ οὐ προσέθηκαν ἆραι κεφαλὴν αὐτῶν· καὶ ἡσύχασεν ἡ γῆ τεσσαράκοντα ἔτη ἐν ἡμέραις Γεδεών.
\vs{29}Καὶ ἐπορεύθη Ἱεροβάαλ υἱὸς Ἰωὰς, καὶ ἐκάθισεν ἐν οἴκῳ αὐτοῦ.
\vs{30}Καὶ τῷ Γεδεὼν ἦσαν υἱοὶ ἑβδομήκοντα ἐκπορευόμενοι ἐκ μηρῶν αὐτοῦ, ὅτι γυναῖκες πολλαὶ ἦσαν αὐτῷ.
\vs{31}Καὶ παλλακὴ αὐτοῦ ἦν ἐν Συχὲμ, καὶ ἔτεκεν αὐτῷ καί γε αὐτὴ υἱὸν, καὶ ἔθηκε τὸ ὄνομα Ἀβιμέλεχ.
\vs{32}Καὶ ἀπέθανε Γεδεὼν υἱὸς Ἰωὰς ἐν πόλει αὐτοῦ, καὶ ἐτάφη ἐν τῷ τάφῳ Ἰωὰς τοῦ πατρὸς αὐτοῦ ἐν Ἐφραθὰ Ἀβὶ Ἐσδρί.

\vs{33}Καὶ ἐγενήθη ὡς ἀπέθανε Γεδεὼν, καὶ ἐπέστρεψαν οἱ υἱοὶ Ἰσραὴλ, καὶ ἐξεπόρνευσαν ὀπίσω τῶν Βααλὶμ, καὶ ἔθηκαν ἑαυτοῖς τῷ Βάαλ διαθήκην τοῦ εἶναι αὐτοῖς αὐτὸν εἰς θεόν.
\vs{34}Καὶ οὐκ ἐμνήσθησαν οἱ υἱοὶ Ἰσραὴλ Κυρίου τοῦ Θεοῦ τοῦ ῥυσαμένου αὐτοὺς ἐκ χειρὸς πάντων τῶν θλιβόντων αὐτοὺς κυκλόθεν.
\vs{35}Καὶ οὐκ ἐποίησαν ἔλεος μετὰ τοῦ οἴκου Ἱεροβάαλ, αὐτός ἐστι Γεδεὼν, κατὰ πάντα τὰ ἀγαθὰ ἃ ἐποίησε μετὰ Ἰσραήλ.

\ch{9}
Καὶ ἐπορεύθη Ἀβιμέλεχ υἱὸς Ἱεροβάαλ εἰς Συχὲμ πρὸς ἀδελφοὺς μητρὸς αὐτοῦ· καὶ ἐλάλησε πρὸς αὐτοὺς καὶ πρὸς πᾶσαν συγγένειαν οἴκου πατρὸς μητρὸς αὐτοῦ, λέγων,
\vs{2}λαλήσατε δὴ ἐν τοῖς ὠσὶ πάντων τῶν ἀνδρῶν Συχὲμ, τί τὸ ἀγαθὸν ὑμῖν, κυριεῦσαι ὑμῶν ἑβδομήκοντα ἄνδρας πάντας υἱοὺς Ἱεροβάαλ, ἢ κυριεύειν ὑμῶν ἄνδρα ἕνα; καὶ μνήσθητε ὅτι ὀστοῦν ὑμῶν καὶ σὰρξ ὑμῶν εἰμι.
\vs{3}Καὶ ἐλάλησαν περὶ αὐτοῦ οἱ ἀδελφοὶ τῆς μητρὸς αὐτοῦ ἐν τοῖς ὠσὶ πάντων τῶν ἀνδρῶν Συχὲμ πάντας τοὺς λόγους τούτους· καὶ ἔκλινεν ἡ καρδία αὐτῶν ὀπίσω Ἀβιμέλεχ, ὅτι εἶπαν, ἀδελφός ἡμῶν ἐστι.
\vs{4}Καὶ ἔδωκαν αὐτῷ ἑβδομήκοντα ἀργυρίου ἐξ οἴκου Βααλβερίθ· καὶ ἐμισθώσατο ἑαυτῷ Ἀβιμέλεχ ἄνδρας κενοὺς καὶ δειλοὺς, καὶ ἐπορεύθησαν ὀπίσω αὐτοῦ.
\vs{5}Καὶ εἰσῆλθεν εἰς τὸν οἶκον τοῦ πατρὸς αὐτοῦ εἰς Ἐφραθὰ, καὶ ἀπέκτεινε τοὺς ἀδελφοὺς αὐτοῦ υἱοὺς Ἱεροβάαλ, ἑβδομήκοντα ἄνδρας ἐπὶ λίθον ἕνα· καὶ κατελείφθη Ἰωάθαμ υἱὸς Ἱεροβάαλ ὁ νεώτερος, ὅτι ἐκρύβη.

\vs{6}Καὶ συνήχθησαν πάντες ἄνδρες Σικίμων, καὶ πᾶς οἶκος Βηθμααλὼ, καὶ ἐπορεύθησαν, καὶ ἐβασίλευσαν τὸν Ἀβιμέλεχ πρὸς τῇ βαλάνῳ τῇ εὑρετῇ τῆς στάσεως τῆς ἐν Σικίμοις.

\vs{7}Καὶ ἀνηγγέλη τῷ Ἰωάθαμ, καὶ ἐπορεύθη, καὶ ἔστη ἐπὶ κορυφὴν ὄρους Γαριζὶν, καὶ ἐπῇρε τὴν φωνὴν αὐτοῦ, καὶ ἔκλαυσε, καὶ εἶπεν αὐτοῖς, ἀκούσατέ μου ἄνδρες Σικίμων, καὶ ἀκούσεται ὑμῶν ὁ Θεός.

\vs{8}Πορευόμενα ἐπορεύθη τὰ ξύλα τοῦ χρίσαι ἐφʼ ἑαυτὰ βασιλέα, καὶ εἶπον τῇ ἐλαίᾳ, βασίλευσον ἐφʼ ἡμῶν.
\vs{9}Καὶ εἶπεν αὐτοῖς ἡ ἐλαία, μὴ ἀπολείψασα τὴν πιότητά μου, ἐν ᾗ δοξάσουσι τὸν Θεὸν ἄνδρες, πορεύσομαι κινεῖσθαι ἐπὶ τῶν ξύλων;
\vs{10}Καὶ εἶπον τὰ ξύλα τῇ συκῇ, δεῦρο, βασίλευσον ἐφʼ ἡμῶν.
\vs{11}Καὶ εἶπεν αὐτοῖς ἡ συκῆ, μὴ ἀπολείψασα ἐγὼ τὴν γλυκύτητά μου καὶ τὰ γεννήματά μου τὰ ἀγαθὰ, πορεύσομαι κινεῖσθαι ἐπὶ τῶν ξύλων;
\vs{12}Καὶ εἶπαν τὰ ξύλα πρὸς τὴν ἄμπελον, δεῦρο, βασίλευσον ἐφʼ ἡμῶν.
\vs{13}Καὶ εἶπεν αὐτοῖς ἡ ἄμπελος, μὴ ἀπολείψασα τὸν οἶνόν μου τὸν εὐφραίνοντα Θεὸν καὶ ἀνθρώπους, πορεύσομαι κινεῖσθαι ἐπὶ τῶν ξύλων;
\vs{14}Καὶ εἶπαν πάντα τὰ ξύλα τῇ ῥάμνῳ, δεῦρο σὺ, βασίλευσον ἐφʼ ἡμῶν.
\vs{15}Καὶ εἶπεν ἡ ῥάμνος πρὸς τὰ ξύλα, εἰ ἐν ἀληθείᾳ χρίετέ με ὑμεῖς τοῦ βασιλεύειν ἐφʼ ὑμᾶς, δεῦτε, ὑπόστητε ἐν τῇ σκιᾷ μου· καὶ εἰ μὴ, ἐξέλθοι πῦρ ἀπʼ ἐμοῦ καὶ καταφάγοι τὰς κέδρους τοῦ Λιβάνου.

\vs{16}Καὶ νῦν εἰ ἐν ἀληθείᾳ καὶ τελειότητι ἐποιήσατε, καὶ ἐβασιλεύσατε τὸν Ἀβιμέλεχ, καὶ εἰ ἀγαθωσύνην ἐποιήσατε μετὰ Ἱεροβάαλ, καὶ μετὰ τοῦ οἴκου αὐτοῦ, καὶ εἰ ὡς ἀνταπόδοσις χειρὸς αὐτοῦ ἐποιήσατε αὐτῷ,
\vs{17}ὡς παρετάξατο ὁ πατήρ μου ὑπὲρ ὑμῶν, καὶ ἐξέῤῥιψε τὴν ψυχὴν αὐτοῦ ἐξεναντίας, καὶ ἐῤῥύσατο ὑμᾶς ἐκ χειρὸς Μαδιὰμ,
\vs{18}καὶ ὑμεῖς ἐπανέστητε ἐπὶ τὸν οἶκον τοῦ πατρός μου σήμερον, καὶ ἀπεκτείνατε τοὺς υἱοὺς αὐτοῦ ἑβδομήκοντα ἄνδρας ἐπὶ λίθον ἕνα, καὶ ἐβασίλευσατε τὸν Ἀβιμέλεχ υἱὸν παιδίσκης αὐτοῦ ἐπὶ τοὺς ἄνδρας Σικίμων, ὅτι ἀδελφὸς ὑμῶν ἐστι·
\vs{19}Καὶ εἰ ἐν ἀληθείᾳ καὶ τελειότητι ἐποιήσατε μετὰ Ἱεροβάαλ, καὶ μετὰ τοῦ οἴκου αὐτοῦ ἐν τῇ ἡμέρᾳ ταύτῃ, εὐφρανθείητε ἐν Ἀβιμέλεχ, καὶ εὐφρανθείη καί γε αὐτὸς ἐφʼ ὑμῖν·
\vs{20}Εἰ δὲ οὐ, ἐξέλθοι πῦρ ἀπὸ Ἀβιμέλεχ, καὶ καταφάγοι τοὺς ἄνδρας Σικίμων καὶ τὸν οἴκον Βηθμααλώ· καὶ ἐξέλθοι πῦρ ἀπὸ ἀνδρῶν Σικίμων, καὶ ἐκ τοῦ οἴκου Βηθμααλὼ, καὶ καταφάγοι τὸν Ἀβιμέλεχ.

\vs{21}Καὶ ἔφυγεν Ἰωάθαμ καὶ ἀπέδρα, καὶ ἐπορεύθη ἕως Βαιὴρ, καὶ ᾤκησεν ἐκεῖ ἀπὸ προσώπου Ἀβιμέλεχ ἀδελφοῦ αὐτοῦ.

\vs{22}Καὶ ἦρξεν Ἀβιμέλεχ ἐπὶ Ἰσραὴλ τρία ἔτη.
\vs{23}Καὶ ἐξαπέστειλεν ὁ Θεὸς πνεῦμα πονηρὸν ἀναμέσον Ἀβιμέλεχ καὶ ἀναμέσον τῶν ἀνδρῶν Σικίμων· καὶ ἠθέτισαν ἄνδρες Σικίμων ἐν τῷ οἴκῳ Ἀβιμέλεχ,
\vs{24}τοῦ ἐπαγαγεῖν τὴν ἀδικίαν τῶν ἑβδομήκοντα υἱῶν Ἱεροβάαλ, καὶ τὰ αἵματα αὐτῶν τοῦ θεῖναι ἐπὶ Ἀβιμέλεχ τὸν ἀδελφὸν αὐτῶν, ὃς ἀπέκτεινεν αὐτοὺς, καὶ ἐπὶ ἄνδρας Σικίμων, ὅτι ἐνίσχυσαν τὰς χεῖρας αὐτοῦ ἀποκτεῖναι τοὺς ἀδελφοὺς αὐτοῦ.
\vs{25}Καὶ ἔθηκαν αὐτῷ οἱ ἄνδρες Σικίμων ἐνεδρεύοντας ἐπὶ τὰς κεφαλὰς τῶν ὀρέων, καὶ διήρπαζον πάντα ὃς παρεπορεύετο ἐπʼ αὐτοὺς ἐν τῇ ὁδῷ· καὶ ἀπηγγέλη τῷ βασιλεῖ Ἀβιμέλεχ.

\vs{26}Καὶ ἦλθε Γαὰλ υἱὸς Ἰωβὴλ, καὶ οἱ ἀδελφοὶ αὐτοῦ, καὶ παρῆλθον ἐν Σικίμοις, καὶ ἤλπισαν ἐν αὐτῷ οἱ ἄνδρες Σικίμων.
\vs{27}Καὶ ἐξῆλθον εἰς ἀγρὸν, καὶ ἐτρύγησαν τοὺς ἀμπελῶνας αὐτῶν, καὶ ἐπάτησαν, καὶ ἐποίησαν Ἐλλουλίμ· καὶ εἰσήνεγκαν εἰς οἶκον θεοῦ αὐτῶν, καὶ ἔφαγον καὶ ἔπιον, καὶ κατηράσαντο τὸν Ἀβιμέλεχ.
\vs{28}Καὶ εἶπε Γαὰλ υἱὸς Ἰωβὴλ, τίς ἐστιν Ἀβιμέλεχ, καὶ τίς ἐστιν υἱὸς Συχὲμ, ὅτι δουλεύσομεν αὐτῷ; οὐχ υἱὸς Ἱεροβάαλ, καὶ Ζεβοὺλ ἐπίσκοπος αὐτοῦ, δοῦλος αὐτοῦ σὺν τοῖς ἀνδράσιν Ἐμμὼρ πατρὸς Συχέμ; καὶ τί ὅτι δουλεύσομεν αὐτῷ ἡμεῖς;
\vs{29}Καὶ τίς δῴη τὸν λαὸν τοῦτον ἐν χειρί μου; καὶ μεταστήσω τὸν Ἀβιμέλεχ, καὶ ἐρῶ πρὸς αὐτὸν, πλήθυνον τὴν δύναμίν σου καὶ ἔξελθε.

\vs{30}Καὶ ἤκουσε Ζεβοὺλ ἄρχων τῆς πόλεως τοὺς λόγους Γαὰλ υἱοῦ Ἰωβὴλ, καὶ ὠργίσθη θυμῷ αὐτός.
\vs{31}Καὶ ἀπέστειλεν ἀγγέλους πρὸς Ἀβιμέλεχ ἐν κρυφῇ, λέγων, ἰδοὺ Γαὰλ υἱὸς Ἰωβὴλ καὶ οἱ ἀδελφοὶ αὐτοῦ ἔρχονται εἰς Συχὲμ, καὶ ἰδοὺ αὐτοὶ περικάθηνται τὴν πόλιν ἐπὶ σέ.
\vs{32}Καὶ νῦν ἀνάστηθι νυκτὸς, καὶ ὁ λαὸς ὁ μετὰ σου, καὶ ἐνέδρευσον ἐν τῷ ἀγρῷ.
\vs{33}Καὶ ἔσται τοπρωῒ ἅμα τῷ ἀνατεῖλαι τὸν ἥλιον, ὀρθριεῖς καὶ ἐκτενεῖς ἐπὶ τὴν πόλιν· καὶ ἰδοὺ αὐτὸς καὶ ὁ λαὸς ὁ μετʼ αὐτοῦ ἐκπορεύονται πρὸς σὲ, καὶ ποιήσεις αὐτῷ ὅσα ἂν εὕρῃ ἡ χείρ σου.

\vs{34}Καὶ ἀνέστη Ἀβιμέλεχ καὶ πᾶς ὁ λαὸς μετʼ αὐτοῦ νυκτὸς, καὶ ἐνήδρευσαν ἐπὶ Συχὲμ τέτρασιν ἀρχαῖς.
\vs{35}Καὶ ἐξῆλθε Γαὰλ υἱὸς Ἰωβὴλ, καὶ ἔστη πρὸς τῇ θύρᾳ τῆς πύλης τῆς πόλεως· καὶ ἀνέστη Ἀβιμέλεχ καὶ ὁ λαὸς ὁ μετʼ αὐτοῦ ἀπὸ τοῦ ἐνέδρου.
\vs{36}Καὶ εἶδε Γαὰλ υἱὸς Ἰωβὴλ τὸν λαὸν, καὶ εἶπε πρὸς Ζεβοὺλ, ἰδοὺ λαὸς καταβαίνει ἀπὸ τῶν κεφαλῶν τῶν ὀρέων· καὶ εἶπε πρὸς αὐτὸν Ζεβοὺλ, τὴν σκιὰν τῶν ὀρέων σὺ βλέπεις ὡς ἄνδρας.
\vs{37}Καὶ προσέθετο ἔτι Γαὰλ τοῦ λαλῆσαι, καὶ εἶπεν, ἰδοὺ λαὸς καταβαίνων κατὰ θάλασσαν ἀπὸ τοῦ ἐχόμενα ὀμφαλοῦ τῆς γῆς, καὶ ἀρχὴ ἑτέρα ἔρχεται διʼ ὁδοῦ Ἥλων Μαωνενίμ.
\vs{38}Καὶ εἶπε πρὸς αὐτὸν Ζεβοὺλ, καὶ ποῦ ἐστι τὸ στόμα σου ὡς ἐλάλησας, τίς ἐστιν Ἀβιμέλεχ, ὅτι δουλεύσομεν αὐτῷ; μὴ οὐχὶ οὗτος ὁ λαὸς ὃν ἐξουδένωσας; ἔξελθε δὴ νῦν καὶ παράταξαι αὐτῷ.
\vs{39}Καὶ ἐξῆλθε Γαὰλ ἐνώπιον ἀνδρῶν Συχὲμ, καὶ παρετάξατο πρὸς Ἀβιμέλεχ.
\vs{40}Καὶ ἐδίωξεν αὐτὸν Ἀβιμέλεχ, καὶ ἔφυγεν ἀπὸ προσώπου αὐτοῦ· καὶ ἔπεσον τραυματίαι πολλοὶ ἕως τῆς θύρας τῆς πύλης.

\vs{41}Καὶ εἰσῆλθεν Ἀβιμέλεχ ἐν Ἀρημά· καὶ ἐξέβαλε Ζεβοὺλ τὸν Γαὰλ καὶ τοὺς ἀδελφοὺς αὐτοῦ, μὴ οἰκεῖν ἐν Συχέμ.

\vs{42}Καὶ ἐγένετο τῇ ἐπαύριον καὶ ἐξῆλθεν ὁ λαὸς εἰς τὸν ἀγρὸν, καὶ ἀνήγγειλε τῷ Ἀβιμέλεχ.
\vs{43}Καὶ ἔλαβε τὸν λαὸν, καὶ διεῖλεν αὐτοὺς εἰς τρεῖς ἀρχὰς, καὶ ἐνήδρευσεν ἐν ἀγρῷ· καὶ εἶδε, καὶ ἰδοὺ λαὸς ἐξῆλθεν ἐκ τῆς πόλεως, καὶ ἀνέστη ἐπʼ αὐτοὺς, καὶ ἐπάταξεν αὐτούς.
\vs{44}Καὶ Ἀβιμέλεχ καὶ οἱ ἀρχηγοὶ οἱ μετʼ αὐτοῦ ἐξέτειναν, καὶ ἔστησαν παρὰ τὴν θύραν τῆς πύλης τῆς πόλεως· καὶ αἱ δύο ἀρχαὶ ἐξέτειναν ἐπὶ πάντας τοὺς ἐν τῷ ἀγρῷ, καὶ ἐπάταξαν αὐτούς.
\vs{45}Καὶ Ἀβιμέλεχ παρετάσσετο ἐν τῇ πόλει ὅλην τὴν ἡμέραν ἐκείνην, καὶ κατελάβετο τὴν πόλιν, καὶ τὸν λαὸν τὸν ἐν αὐτῇ ἀπέκτεινε, καὶ τὴν πόλιν καθεῖλε, καὶ ἔσπειρεν αὐτὴν ἅλας.

\vs{46}Καὶ ἤκουσαν πάντες οἱ ἄνδρες πύργου Συχὲμ, καὶ ἦλθον εἰς συνέλευσιν Βαιθηλβερίθ.
\vs{47}Καὶ ἀνηγγέλη τῷ Ἀβιμέλεχ, ὅτι συνήχθησαν πάντες οἱ ἄνδρες πύργου Συχέμ.
\vs{48}Καὶ ἀνέβη Ἀβιμέλεχ εἰς ὄρος Σελμὼν, καὶ πᾶς ὁ λαὸς ὁ μετʼ αὐτοῦ· καὶ ἔλαβεν Ἀβιμέλεχ τὰς ἀξίνας ἐν τῇ χειρὶ αὐτοῦ, καὶ ἔκοψε κλάδον ξύλου, καὶ ᾖρε, καὶ ἔθηκεν ἐπὶ ὤμων αὐτοῦ· καὶ εἶπε τῷ λαῷ τῷ μετʼ αὐτοῦ, ὃ εἴδετέ με ποιοῦντα, ταχέως ποιήσατε ὡς ἐγώ.
\vs{49}Καὶ ἔκοψαν καί γε ἀνὴρ κλάδον πᾶς ἀνὴρ, καὶ ἐπορεύθησαν ὀπίσω Ἀβιμέλεχ, καὶ ἐπέθηκαν ἐπὶ τὴν συνέλευσιν, καὶ ἐνεπύρισαν ἐπʼ αὐτοὺς τὴν συνέλευσιν ἐν πυρί· καὶ ἀπέθανον καί γε πάντες οἱ ἄνδρες πύργου Σικίμων, ὡσεὶ χίλιοι ἄνδρες καὶ γυναῖκες.

\vs{50}Καὶ ἐπορεύθη Ἀβιμέλεχ ἐκ Βαιθηγβερὶθ, καὶ παρενέβαλεν ἐν Θήβης, καὶ κατέλαβεν αὐτήν.
\vs{51}Καὶ πύργος ἰσχυρὸς ἦν ἐν μέσῳ τῆς πόλεως· καὶ ἔφυγον ἐκεῖ πάντες οἱ ἄνδρες καὶ αἱ γυναῖκες τῆς πόλεως, καὶ ἔκλεισαν ἔξωθεν αὐτῶν, καὶ ἀνέβησαν ἐπὶ τὸ δῶμα τοῦ πύργου.
\vs{52}Καὶ ἦλθεν Ἀβιμέλεχ ἕως τοῦ πύργου, καὶ παρετάξαντο αὐτῷ· καὶ ἤγγισεν Ἀβιμέλεχ ἕως τῆς θύρας τοῦ πύργου τοῦ ἐμπρῆσαι αὐτὸν ἐν πυρί.
\vs{53}Καὶ ἔῤῥιψε γυνὴ μία κλάσμα ἐπιμύλιον ἐπὶ κεφαλὴν Ἀβιμέλεχ, καὶ ἔκλασε τὸ κρανίον αὐτοῦ.
\vs{54}Καὶ ἐβόησε ταχὺ πρὸς τὸ παιδάριον τὸ αἶρον τὰ σκεύη αὐτοῦ, καὶ εἶπεν αὐτῷ, σπάσον τὴν ῥομφαίαν μου καὶ θανάτωσόν με, μή ποτε εἴπωσι, γυνὴ ἀπέκτεινεν αὐτόν· καὶ ἐξεκέντησεν αὐτὸν τὸ παιδάριον αὐτοῦ, καὶ ἀπέθανε.
\vs{55}Καὶ εἶδεν ἀνὴρ Ἰσραὴλ ὅτι ἀπέθανεν Ἀβιμέλεχ· καὶ ἐπορεύθησαν ἀνὴρ εἰς τὸν τόπον αὐτοῦ.

\vs{56}Καὶ ἐπέστρεψεν ὁ Θεὸς τὴν πονηρίαν Ἀβιμέλεχ, ἣν ἐποίησε τῷ πατρὶ αὐτοῦ, ἀποκτεῖναι τοὺς ἑβδομήκοντα ἀδελφοὺς αὐτοῦ.
\vs{57}Καὶ τὴν πᾶσαν πονηρίαν ἀνδρῶν Συχὲμ ἐπέστρεψεν ὁ Θεὸς εἰς κεφαλὴν αὐτῶν· καὶ ἐπῆλθεν ἐπʼ αὐτοὺς ἡ κατάρα Ἰωάθαμ υἱοῦ Ἰεροβάαλ.

\ch{10}
Καὶ ἀνέστη μετὰ Ἀβιμέλεχ τοῦ σῶσαι τὸν Ἰσραὴλ Θωλὰ υἱὸς Φουὰ, υἱὸς πατραδέλφου αὐτοῦ, ἀνὴρ Ἰσσάχαρ· καὶ αὐτὸς ᾤκει ἐν Σαμὶρ ἐν ὄρει Ἐφραίμ.
\vs{2}Καὶ ἔκρινε τὸν Ἰσραὴλ εἴκοσι τρία ἔτη, καὶ ἀπέθανε, καὶ ἐτάφη ἐν Σαμίρ.

\vs{3}Καὶ ἀνέστη μετʼ αὐτὸν Ἰαῒρ ὁ Γαλαὰδ, καὶ ἔκρινε τὸν Ἰσραὴλ εἴκοσι δύο ἔτη.
\vs{4}Καὶ ἦσαν αὐτῷ τριάκοντα καὶ δύο υἱοὶ ἐπιβαίνοντες ἐπὶ τριάκοντα δύο πώλους· καὶ τριάκοντα δύο πόλεις αὐτοῖς· καὶ ἐκάλουν αὐτὰς ἐπαύλεις Ἰαῒρ ἕως τῆς ἡμέρας ταύτης ἐν γῇ Γαλαάδ.
\vs{5}Καὶ ἀπέθανεν Ἰαῒρ, καὶ ἐτάφη ἐν Ῥαμνών.

\vs{6}Καὶ προσέθεντο οἱ υἱοὶ Ἰσραὴλ τοῦ ποιῆσαι τὸ πονηρὸν ἐνώπιον Κυρίου, καὶ ἐδούλευσαν τοῖς Βααλὶμ, καὶ ταῖς Ἀσταρὼθ, καὶ τοῖς θεοῖς Ἀρὰμ, καὶ τοῖς θεοῖς Σιδῶνος, καὶ τοῖς θεοῖς Μωὰβ, καὶ τοῖς θεοῖς υἱῶν Ἀμμὼν, καὶ τοῖς θεοῖς Φυλιστιῒμ, καὶ ἐγκατέλιπον τὸν Κύριον, καὶ οὐκ ἐδούλευσαν αὐτῷ.
\vs{7}Καὶ ὠργίσθη θυμῷ Κύριος ἐν Ἰσραὴλ, καὶ ἀπέδοτο αὐτοὺς ἐν χειρὶ Φυλιστιῒμ, καὶ ἐν χειρὶ υἱῶν Ἀμμών.
\vs{8}Καὶ ἔθλιψαν καὶ ἔθλασαν τοὺς υἱοὺς Ἰσραὴλ ἐν τῷ καιρῷ ἐκείνῳ ὀκτωκαίδεκα ἔτη, τοὺς πάντας υἱοὺς Ἰσραὴλ τοὺς ἐν τῷ πέραν τοῦ Ἰορδάνου ἐν γῇ τοῦ Ἀμοῤῥὶ τοῦ ἐν Γαλαάδ.
\vs{9}Καὶ διέβησαν οἱ υἱοὶ Ἀμμὼν τὸν Ἰορδάνην παρατάξασθαι πρὸς Ἰούδαν, καὶ Βενιαμὶν, καὶ πρὸς Ἐφραίμ· καὶ ἐθλίβησαν οἱ υἱοὶ Ἰσραὴλ σφόδρα.

\vs{10}Καὶ ἐβόησαν οἱ υἱοὶ Ἰσραὴλ πρὸς Κύριον, λέγοντες, ἡμάρτομέν σοι, ὅτι ἐγκατελίπομεν τὸν Θεὸν, καὶ ἐδουλεύσαμεν τῷ Βααλίμ.
\vs{11}Καὶ εἶπε Κύριος πρὸς τοὺς υἱοὺς Ἰσραὴλ, μὴ οὐχὶ ἐξ Αἰγύπτου, καὶ ἀπὸ τοῦ Ἀμοῤῥαίου, καὶ ἀπὸ υἱῶν Ἀμμὼν, καὶ ἀπὸ Φυλιστιῒμ,
\vs{12}καὶ Σιδωνίων, καὶ Ἀμαλὲκ, καὶ Μαδιὰμ, οἳ ἔθλιψαν ὑμᾶς; καὶ ἐβοήσατε πρὸς μὲ, καὶ ἔσωσα ὑμᾶς ἐκ χειρὸς αὐτῶν;
\vs{13}Καὶ ὑμεῖς ἐγκατελίπετέ με, καὶ ἐδουλεύσατε θεοῖς ἑτέροις· διὰ τοῦτο οὐ προσθήσω τοῦ σῶσαι ὑμᾶς.
\vs{14}Πορεύεσθε, καὶ βοήσατε πρὸς τοὺς θεοὺς οὓς ἐξελέξασθε ἑαυτοῖς, καὶ αὐτοὶ σωσάτωσαν ὑμᾶς ἐν καιρῷ θλίψεως ὑμῶν.
\vs{15}Καὶ εἶπαν οἱ υἱοὶ Ἰσραὴλ πρὸς Κύριον, ἡμάρτομεν, ποίησον σὺ ἡμῖν κατὰ πὰν τὸ ἀγαθὸν ἐν ὀφθαλμοῖς σου, πλὴν ἐξελοῦ ἡμᾶς ἐν τῇ ἡμέρᾳ ταύτῃ.
\vs{16}Καὶ ἐξέκλιναν τοὺς θεοὺς τοὺ ἀλλοτρίους ἐκ μέσου αὐτῶν, καὶ ἐδούλευσαν τῷ Κυρίῳ μόνῳ· καὶ ὠλιγώθη ἡ ψυχὴ αὐτοῦ ἐν κόπῳ Ἰσραήλ.

\vs{17}Καὶ ἀνέβησαν οἱ υἱοὶ Ἀμμὼν, καὶ παρενέβαλον ἐν Γαλαάδ· καὶ συνήχθησαν οἱ υἱοὶ Ἰσραὴλ, καὶ παρενέβαλον ἐν τῇ σκοπίᾳ.
\vs{18}Καὶ εἶπον ὁ λαὸς οἱ ἄρχοντες Γαλαὰδ, ἀνὴρ πρὸς τὸν πλησίον αὐτοῦ, τίς ὁ ἀνὴρ ὅστις ἂν ἄρξεται παρατάξασθαι πρὸς υἱοῖς Ἀμμὼν, καὶ ἔσται εἰς ἄρχοντα πᾶσι τοῖς κατοικοῦσι Γαλαάδ;

\ch{11}
Καὶ Ἰεφθάε ὁ Γαλααδίτης ἐπῃρμένος δυνάμει, καὶ αὐτὸς υἱοῖς γυναικὸς πόρνης, ἣ ἐγέννησε τῷ Γαλαὰδ τὸν Ἰεφθάε.
\vs{2}Καὶ ἔτεκεν ἡ γυνὴ Γαλαὰδ αὐτῷ υἱούς· καὶ ἡδρύνθησαν οἱ υἱοὶ τῆς γυναικὸς, καὶ ἐξέβαλον τὸν Ἰεφθάε, καὶ εἶπαν αὐτῷ, οὐ κληρονομήσεις ἐν τῷ οἴκῳ τοῦ πατρὸς ἡμῶν, ὅτι υἱὸς γυναικὸς ἑταίρας σύ.

\vs{3}Καὶ ἔφυγεν Ἰεφθάε ἀπὸ προσώπου τῶν ἀδελφῶν αὐτοῦ, καὶ ᾤκησεν ἐν γῇ Τώβ· καὶ συνεστράφησαν πρὸς Ἰεφθάε ἄνδρες κενοὶ, καὶ ἐξῆλθον μετʼ αὐτοῦ.

\vs{4}Καὶ ἐγένετο ἡνίκα παρετάξαντο οἱ υἱοὶ Ἀμμῶν μετὰ Ἰσραὴλ,
\vs{5}καὶ ἐπορεύθησαν οἱ πρεσβύτεροι Γαλαὰδ λαβεῖν τὸν Ἰεφθάε ἀπὸ τῆς γῆς Τὼβ,
\vs{6}καὶ εἶπαν τῷ Ἰεφθάε, δεῦρο καὶ ἔσῃ ἡμῖν εἰς ἀρχηγὸν, καὶ παραταξόμεθα πρὸς υἱοὺς Ἀμμών.
\vs{7}Καὶ εἶπεν Ἰεφθάε τοῖς πρεσβυτέροις Γαλαὰδ, οὐχὶ ὑμεῖς ἐμισήσατέ με, καὶ ἐξεβάλετέ με ἐκ τοῦ οἴκου τοῦ πατρός μου, καὶ ἐξαπεστείλατέ με ἀφʼ ὑμῶν; καὶ διατί ἤλθατε πρὸς μὲ νῦν ἡνίκα χρῄζετε;
\vs{8}Καὶ εἶπαν οἱ προσβύτεροι Γαλαὰδ πρὸς Ἰεφθάε, διὰ τοῦτο νῦν ἐπεστρέψαμεν πρὸς σὲ, καὶ πορεύσῃ μεθʼ ἡμῶν, καὶ παρατάξῃ πρὸς υἱοὺς Ἀμμὼν, καὶ ἔσῃ ἡμῖν εἰς ἄρχοντα πᾶσι τοῖς κατοικοῦσι Γαλαάδ.
\vs{9}Καὶ εἶπεν Ἰεφθάε πρὸς τοὺς πρεσβυτέρους Γαλαὰδ, εἰ ἐπιστρέφετέ με ὑμεῖς παρατάξασθαι ἐν υἱοῖς Ἀμμὼν, καὶ παραδῷ αὐτοὺς Κύριος ἐνώπιον ἐμοῦ, καὶ ἐγὼ ὑμῖν ἔσομαι εἰς ἄρχοντα.
\vs{10}Καὶ εἶπαν οἱ πρεσβύτεροι Γαλαὰδ πρὸς Ἰεφθάε, Κύριος ἔστω ἀκούων ἀναμέσον ἡμῶν, εἰ μὴ κατὰ τὸ ῥῆμά σου οὕτω ποιήσομεν.

\vs{11}Καὶ ἐπορεύθη Ἰεφθάε μετὰ τῶν πρεσβυτέρων Γαλαὰδ, καὶ ἔθηκαν αὐτὸν ὁ λαὸς ἐπʼ αὐτοὺς εἰς κεφαλὴν καὶ εἰς ἀρχηγόν· καὶ ἐλάλησεν Ἰεφθαέ πάντας τοὺς λόγους αὐτοῦ ἐνώπιον Κυρίου ἐν Μασσηφά.

\vs{12}Καὶ ἀπέστειλεν Ἰεφθάε ἀγγέλους πρὸς βασιλέα υἱῶν Ἀμμὼν, λέγων, τί ἐμοὶ καὶ σοὶ, ὅτι ἦλθες πρὸς μὲ τοῦ παρατάξασθαι ἐν τῇ γῇ μου;
\vs{13}Καὶ εἶπε βασιλεὺς υἱῶν Ἀμμὼν πρὸς τοὺς ἀγγέλους Ἰεφθάε, ὅτι ἔλαβεν Ἰσραὴλ τὴν γῆν μου ἐν τῷ ἀναβαίνειν αὐτὸν ἐξ Αἰγύπτου ἀπὸ Ἀρνὼν ἕως Ἰαβὸκ καὶ ἕως τοῦ Ἰορδάνου· καὶ νῦν ἐπίστρεψον αὐτὰς ἐν εἰρήνῃ, καὶ πορεύσομαι.

\vs{14}Καὶ προσέθηκεν ἔτι Ἰεφθάε, καὶ ἀπέστειλεν ἀγγέλους πρὸς βασιλέα υἱῶν Ἀμμών.
\vs{15}Καὶ εἶπεν αὐτῷ, οὕτω λέγει Ἰεφθάε, οὐκ ἔλαβεν Ἰσραὴλ τὴν γῆν Μωὰβ, καὶ τὴν γῆν υἱῶν Ἀμμών,
\vs{16}ὅτι ἐν τῷ ἀναβαίνειν αὐτοὺς ἐξ Αἰγύπτου ἐπορεύθη Ἰσραὴλ ἐν τῇ ἐρήμῳ ἕως θαλάσσης Σὶφ, καὶ ἦλθεν εἰς Κάδης.
\vs{17}Καὶ ἀπέστειλεν Ἰσραὴλ ἀγγέλους πρὸς βασιλέα Ἐδὼμ, λέγων, παρελεύσομαι δὴ ἐν τῇ γῇ σου· καὶ οὐκ ἤκουσε βασιλεὺς Ἐδώμ· καί γε πρὸς βασιλέα Μωὰβ ἀπέστειλε, καὶ οὐκ εὐδόκησε· καὶ ἐκάθισεν Ἰσραὴλ ἐν Κάδης,
\vs{18}καὶ ἐπορεύθη ἐν τῇ ἐρήμῳ, καὶ ἐκύκλωσε τὴν γῆν Ἐδὼμ καὶ τὴν γῆν Μωάβ· καὶ ἦλθεν ἀπὸ ἀνατολῶν ἡλίου τῇ γῇ Μωὰβ, καὶ παρενέβαλεν ἐν πέραν Ἀρνὼν, καὶ οὐκ εἰσῆλθεν ἐν ὁρίοις Μωὰβ, ὅτι Ἀρνὼν ὅριον Μωάβ.
\vs{19}Καὶ ἀπέστειλεν Ἰσραὴλ ἀγγέλους πρὸς Σηὼν βασιλέα τοῦ Ἀμοῤῥαίου βασιλέα Ἐσεβῶν, καὶ εἶπεν αὐτῷ Ἰσραήλ, παρέλθωμεν δὴ ἐν τῇ γῇ σου ἕως τοῦ τόπου ἡμῶν.
\vs{20}Καὶ οὐκ ἐνεπίστευσε Σηὼν τῷ Ἰσραὴλ παρελθεῖν ἐν τῷ ὁρίῳ αὐτοῦ· καὶ συνῆξε Σηὼν πάντα τὸν λαὸν αὐτοῦ, καὶ παρενέβαλον εἰς Ἰασὰ, καὶ παρετάξατο πρὸς Ἰσραήλ.
\vs{21}Καὶ παρέδωκε Κύριος ὁ Θεὸς Ἰσραὴλ τὸν Σηὼν καὶ πάντα τὸν λαὸν αὐτοῦ ἐν χειρὶ Ἰσραὴλ, καὶ ἐπάταξεν αὐτόν· καὶ ἐκληρονόμησεν Ἰσραὴλ πᾶσαν τὴν γῆν τοῦ Ἀμοῤῥαίου τοῦ κατοικοῦντος τὴν γῆν ἐκείνην
\vs{22}ἀπὸ Ἀρνὼν καὶ ἕως τοῦ Ἰαβὸκ, καὶ ἀπὸ τοῦ ἐρήμου ἕως τοῦ Ἰορδάνου.
\vs{23}Καὶ νῦν Κύριος ὁ Θεὸς Ἰσραὴλ ἐξῇρε τὸν Ἀμοῤῥαῖον ἀπὸ προσώπου λαοῦ αὐτοῦ Ἰσραήλ, καὶ σὺ κληρονομήσεις αὐτόν;
\vs{24}Οὐχὶ ἃ ἐὰν κληρονομήσει σε Χαμὼς ὁ θεὸς σοῦ, αὐτὰ κληρονομήσεις, καὶ τοὺς πάντας οὓς ἐξῇρε Κύριος ὁ Θεὸς ἡμῶν ἀπὸ προσώπου ἡμῶν, αὐτοὺς κληρονομήσομεν;
\vs{25}Καὶ νῦν μὴ ἐν ἀγαθῷ ἀγαθώτερος σὺ ὑπὲρ Βαλὰκ υἱὸν Σεπφὼρ βασιλέως Μωάβ; μὴ μαχόμενος ἐμαχέσατο μετὰ Ἰσραὴλ, ἢ πολεμῶν ἐπολέμησεν αὐτὸν,
\vs{26}ἐν τῷ οἰκῆσαι ἐν Ἐσεβὼν καὶ ἐν τοῖς ὁρίοις αὐτῆς, καὶ ἐν γῇ Ἀροὴρ καὶ ἐν τοῖς ὁρίοις αὐτῆς, καὶ ἐν πάσαις ταῖς πόλεσι ταῖς παρὰ τὸν Ἰορδάνην, τριακόσια ἔτη; καὶ διατί οὐκ ἐῤῥύσω αὐτοὺς ἐν τῷ καιρῷ ἐκείνῳ;
\vs{27}Καὶ νῦν ἐγώ εἰμι οὐχ ἥμαρτόν σοι, καὶ σὺ ποιεῖς μετʼ ἐμοῦ πονηρίαν τοῦ παρατάξασθαι ἐν ἐμοί· κρίναι Κύριος ὁ κρίνων σήμερον ἀναμέσον υἱῶν Ἰσραὴλ καὶ ἀναμέσον υἱῶν Ἀμμών.

\vs{28}Καὶ οὐκ ἤκουσε βασιλεὺς υἱῶν Ἀμμὼν τῶν λόγων Ἰεφθάε, ὧν ἀπέστειλε πρὸς αὐτόν.
\vs{29}Καὶ ἐγένετο ἐπὶ Ἰεφθάε πνεῦμα Κυρίου, καὶ παρῆλθε τὸν Γαλαὰδ, καὶ τὸν Μανασσῆ, καὶ παρῆλθε τὴν σκοπιὰν Γαλαὰδ εἰς τὸ πέραν υἱῶν Ἀμμών.

\vs{30}Καὶ ηὔξατο Ἰεφθάε εὐχὴν τῷ Κυρίῳ, καὶ εἶπεν, ἐὰν διδοὺς δῷς μοι τοὺς υἱοὺς Ἁμμὼν ἐν τῆ χειρί μου,
\vs{31}καὶ ἔσται ὁ ἐκπορευόμενος ὃς ἂν ἐξέλθῃ ἀπὸ τῆς θύρας τοῦ οἴκου μου εἰς συνάντησίν μου ἐν τῷ ἐπιστρέφειν με ἐν εἰρήνῃ ἀπὸ υἱῶν Ἀμμὼν, καὶ ἔσται τῷ Κυρίῳ, ἀνοίσω αὐτὸν ὁλοκαύτωμα.

\vs{32}Καὶ παρῆλθεν Ἰεφθάε πρὸς υἱοὺς Ἀμμὼν παρατάξασθαι πρὸς αὐτούς· καὶ παρέδωκεν αὐτοὺς Κύριος ἐν χειρὶ αὐτοῦ.
\vs{33}Καὶ ἐπάταξεν αὐτοὺς ἀπὸ Ἀροὴρ ἕως ἐλθεῖν ἄχρις Ἀρνὼν ἐν ἀριθμῷ εἴκοσι πόλεις, καὶ ἕως Ἐβελχαρμὶμ, πληγὴν μεγάλην σφόδρα· καὶ συνεστάλησαν οἱ υἱοὶ Ἀμμὼν ἀπὸ προσώπου υἱῶν Ἰσραήλ.

\vs{34}Καὶ ἦλθεν Ἰεφθάε εἰς Μασσηφὰ εἰς τὸν οἶκον αὐτοῦ· καὶ ἰδοὺ ἡ θυγάτηρ αὐτοῦ ἐξεπορεύετο εἰς ὑπάντησιν ἐν τυμπάνοις καὶ χοροῖς· καὶ αὕτη ἦν μονογενὴς αὐτῷ· οὐκ ἦν αὐτῷ ἕτερος υἱὸς ἢ θυγάτηρ.
\vs{35}Καὶ ἐγένετο ὡς εἶδεν αὐτὴν αὐτός, διέῤῥηξε τὰ ἱμάτια αὐτοῦ, καὶ εἶπεν, ἆ ἆ, θυγάτηρ μου, ταραχῇ ἐτάραξάς με, καὶ σὺ ἦς ἐν τῷ ταράχῳ μου, καὶ ἐγώ εἰμι ἤνοιξα κατὰ σοῦ τὸ στόμα μου πρὸς Κύριον, καὶ οὐ δυνήσομαι ἀποστρέψαι.
\vs{36}Ἡ δὲ εἶπε πρὸς αὐτὸν, πάτερ, ἤνοιξας τὸ στόμα σου πρὸς Κύριον; ποίησόν μοι ὃν τρόπον ἐξῆλθεν ἐκ στόματός σου, ἐν τῷ ποιῆσαί σοι Κύριον ἐκδίκησιν τῶν ἐχθρῶν σου ἀπὸ τῶν υἱῶν Ἀμμών.
\vs{37}Καὶ ἥδε εἶπε πρὸς τὸν πατέρα αὐτῆς, ποιησάτω δὴ ὁ πατήρ μου τὸν λόγον τοῦτον· ἔασόν με δύο μῆνας, καὶ πορεύσομαι καὶ καταβήσομαι ἐπὶ τὰ ὄρη, καὶ κλαύσομαι ἐπὶ τὰ παρθένιά μου ἐγώ εἰμι καὶ αἱ συνεταιρίδες μου.
\vs{38}Καὶ εἶπε, πορεύου· καὶ ἀπέστειλεν αὐτὴν δύο μῆνας· καὶ ἐπορεύθη αὐτὴ καὶ αἱ συνεταιρίδες αὐτῆς, καὶ ἔκλαυσεν ἐπὶ τὰ παρθένια αὐτῆς ἐπὶ τὰ ὄρη.

\vs{39}Καὶ ἐγένετο ἐν τέλει τῶν δύο μηνῶν, καὶ ἐπέστρεψε πρὸς τὸν πατέρα αὐτῆς· καὶ ἐποίησεν ἐν αὐτῇ τὴν εὐχὴν αὐτοῦ ἣν ηὔξατο· καὶ αὕτη οὐκ ἔγνω ἄνδρα· καὶ ἐγένετο εἰς πρόσταγμα ἐν Ἰσραήλ·
\vs{40}Ἀπὸ ἡμερῶν εἰς ἡμέρας ἐπορεύοντο θυγατέρες Ἰσραὴλ θρηνεῖν τὴν θυγατέρα Ἰεφθάε τοῦ Γαλααδίτου ἐπὶ τέσσαρας ἡμέρας ἐν τῷ ἐνιαυτῷ.

\ch{12}
Καὶ ἐβόησεν ἀνὴρ Ἐφραὶμ, καὶ παρῆλθαν εἰς Βοῤῥᾶν, καὶ εἶπαν πρὸς Ἰεφθάε, διατί παρῆλθες παρατάξασθαι ἐν υἱοῖς Ἀμμὼν, καὶ ἡμᾶς οὐ κέκληκας πορευθῆναι μετὰ σοῦ; τὸν οἶκόν σου ἐμπρήσομεν ἐπὶ σὲ ἐν πυρί.
\vs{2}Καὶ εἶπε πρὸς αὐτοὺς Ἰεφθὰε, ἀνὴρ μαχητὴς ἤμην ἐγὼ καὶ ὁ λαός μου, καὶ οἱ υἱοὶ Ἀμμὼν σφόδρα· καὶ ἐβόησα ὑμᾶς, καὶ οὐκ ἐσώσατέ με ἐκ χειρὸς αὐτῶν.
\vs{3}Καὶ εἶδον ὅτι οὐκ εἶ σωτὴρ, καὶ ἔθηκα τὴν ψυχήν μου ἐν χειρί μου, καὶ παρῆλθον πρὸς υἱοὺς Ἀμμὼν, καὶ ἔδωκεν αὐτοὺς Κύριος ἐν χειρί μου· καὶ εἰς τί ἀνέβητε ἐπʼ ἐμὲ ἐν τῇ ἡμέρᾳ ταύτῃ παρατάξασθαι ἐν ἐμοί;

\vs{4}Καὶ συνέστρεψεν Ἰεφθάε πάντας τοὺς ἄνδρας Γαλαὰδ, καὶ παρετάξατο τῷ Ἐφραὶμ, καὶ ἐπάταξαν ἄνδρες Γαλαὰδ τὸν Ἐφραὶμ, ὅτι εἶπαν οἱ διασωζόμενοι τοῦ Ἐφραὶμ, ὑμεῖς Γαλαὰδ ἐν μέσῳ τοῦ Ἐφραὶμ καὶ ἐν μέσῳ τοῦ Μανασσῆ.
\vs{5}Καὶ προκατελάβετο Γαλαὰδ τὰς διαβάσεις τοῦ Ἰορδάνου τοῦ Ἐφραίμ· καὶ εἶπαν αὐτοῖς οἱ διασωζόμενοι Ἐφραὶμ, διαβῶμεν· καὶ εἶπαν αὐτοῖς οἱ ἄνδρες Γαλαὰδ, μὴ Ἐφραθίτης εἶ; καὶ εἶπεν, οὔ.
\vs{6}Καὶ εἶπαν αὐτῷ, εἶπον δὴ στάχυς· καὶ οὐ κατεύθυνε τοῦ λαλῆσαι οὕτως· καὶ ἐπελάβοντο αὐτοῦ, καὶ ἔθυσαν αὐτὸν πρὸς τὰς διαβάσεις τοῦ Ἰορδάνου· καὶ ἔπεσαν ἐν τῷ καιρῷ ἐκείνῳ ἀπὸ Ἐφραὶμ δύο καὶ τεσσαράκοντα χιλιάδες.

\vs{7}Καὶ ἔκρινεν Ἰεφθάε τὸν Ἰσραὴλ ἓξ ἔτη· καὶ ἀπέθανεν Ἰεφθάε ὁ Γαλααδίτης, καὶ ἐτάφη ἐν πόλει αὐτοῦ Γαλαάδ.

\vs{8}Καὶ ἔκρινε μετʼ αὐτὸν τὸν Ἰσραὴλ Ἀβαισσὰν ἀπὸ Βηθλεέμ.
\vs{9}Καὶ ἦσαν αὐτῷ τριάκοντα υἱοὶ, καὶ τριάκοντα θυγατέρες, ἃς ἐξαπέστειλεν ἔξω, καὶ τριάκοντα θυγατέρας εἰσήνεγκε τοῖς υἱοῖς αὐτοῦ ἔξωθεν· καὶ ἔκρινε τὸν Ἰσραὴλ ἑπτὰ ἔτη.
\vs{10}Καὶ ἀπέθανεν Ἀβαισσὰν, καὶ ἐτάφη ἐν Βηθλεέμ.

\vs{11}Καὶ ἔκρινε μετʼ αὐτὸν τὸν Ἰσραὴλ Αἰλὼμ ὁ Ζαβουλωνίτης δέκα ἔτη.
\vs{12}Καὶ ἀπέθανεν Αἰλὼμ ὁ Ζαβουλωνίτης, καὶ ἐτάφη ἐν Αἰλὼμ ἐν γῇ Ζαβουλών.

\vs{13}Καὶ ἔκρινε μετʼ αὐτὸν τὸν Ἰσραὴλ Ἀβδὼν υἱὸς Ἑλλὴλ ὁ Φαραθωνίτης.
\vs{14}Καὶ ἦσαν αὐτῷ τεσσαράκοντα υἱοὶ, καὶ τριάκοντα υἱῶν υἱοὶ ἐπιβαίνοντες ἐπὶ ἑβδομήκοντα πώλους· καὶ ἔκρινε τὸν Ἰσραὴλ ὀκτὼ ἔτη.
\vs{15}Καὶ ἀπέθανεν Ἀβδὼν υἱὸς Ἐλλὴλ ὁ Φαραθωνίτης, καὶ ἐτάφη ἐν Φαραθὼν ἐν γῇ Ἐφραὶμ ἐν ὄρει τοῦ Ἀμαλήκ.

\ch{13}
Καὶ προσέθηκαν ἔτι οἱ υἱοὶ Ἰσραὴλ ποιῆσαι τὸ πονηρὸν ἐνώπιον Κυρίου· καὶ παρέδωκεν αὐτοὺς Κύριος ἐν χειρὶ Φυλιστιῒμ τεσσαράκοντα ἔτη.

\vs{2}Καὶ ἦν ἀνὴρ εἷς ἀπὸ Σαραὰ ἀπὸ δήμου συγγενείας τοῦ Δανὶ, καὶ ὄνομα αὐτῷ Μανωὲ, καὶ γυνὴ αὐτοῦ στεῖρα καὶ οὐκ ἔτεκε.
\vs{3}Καὶ ὤφθη ἄγγελος Κυρίου πρὸς τὴν γυναῖκα, καὶ εἶπε πρὸς αὐτὴν, ἰδοὺ σὺ στεῖρα καὶ οὐ τέτοκας, καὶ συλλήψῃ υἱόν.
\vs{4}Καὶ νῦν φύλαξαι δὴ, καὶ μὴ πίῃς οἶνον καὶ μέθυσμα, καὶ μὴ φάγῃς πᾶν ἀκάθαρτον,
\vs{5}ὅτι ἰδοὺ σὺ ἐν γαστρὶ ἔχεις καὶ τέξῃ υἱόν· καὶ σίδηρος ἐπὶ τὴν κεφαλὴν αὐτοῦ οὐκ ἀναβήσεται, ὅτι Ναζὶρ Θεοῦ ἔσται τὸ παιδάριον ἀπὸ τῆς κοιλίας· καὶ αὐτὸς ᾄρξεται σῶσαι τὸν Ἰσραὴλ ἐκ χειρὸς Φυλιστιΐμ.

\vs{6}Καὶ εἰσῆλθεν ἡ γυνὴ, καὶ εἶπε τῷ ἀνδρὶ αὐτῆς, λέγουσα, ἄνθρωπος Θεοῦ ἦλθε πρὸς μὲ, καὶ εἶδος αὐτοῦ ὡς εἶδος ἀγγέλου Θεοῦ, φοβερὸν σφόδρα· καὶ οὐκ ἠρώτησα αὐτὸν πόθεν ἐστὶ, καὶ τὸ ὄνομα αὐτοῦ οὐκ ἀπήγγειλέ μοι.
\vs{7}Καὶ εἶπέ μοι, ἰδοὺ σὺ ἐν γαστρὶ ἔχεις καὶ τέξῃ υἱόν· καὶ νῦν μὴ πίῃς οἶνον καὶ μέθυσμα, καὶ μὴ φάγῃς πᾶν ἀκάθαρτον, ὅτι Θεοῦ ἅγιον ἔσται τὸ παιδάριον ἀπὸ γαστρὸς ἕως ἡμέρας θανάτου αὐτοῦ.

\vs{8}Καὶ προσηύξατο Μανωὲ πρὸς Κύριον, καὶ εἶπεν, ἐν ἐμοὶ Κύριε ἀδωναϊὲ τὸν ἄνθρωπον τοῦ Θεοῦ ὃν ἀπέστειλας· ἐλθέτω δὴ ἔτι πρὸς ἡμᾶς, καὶ συμβιβασάτω ἡμᾶς τί ποιήσωμεν τῷ παιδίῳ τῷ τικτομένῳ.

\vs{9}Καὶ εἰσήκουσεν ὁ Θεὸς τῆς φωνῆς Μανωὲ, καὶ ἦλθεν ὁ ἄγγελος τοῦ Θεοῦ ἔτι πρὸς τὴν γυναῖκα· καὶ αὕτη ἐκάθητο ἐν ἀγρῷ, καὶ Μανωὲ ὁ ἀνὴρ αὐτῆς οὐκ ἦν μετʼ αὐτῆς.
\vs{10}Καὶ ἐτάχυνεν ἡ γυνὴ καὶ ἔδραμε καὶ ἀνήγγειλε τῷ ἀνδρὶ αὐτῆς, καὶ εἶπε πρὸς αὐτὸν, ἰδοὺ ὦπται πρὸς μὲ ὁ ἀνὴρ ὃς ἦλθεν ἐν ἡμέρᾳ πρὸς μέ.

\vs{11}Καὶ ἀνέστη καὶ ἐπορεύθη Μανωὲ ὀπίσω τῆς γυναικὸς αὐτοῦ, καὶ ἦλθε πρὸς τὸν ἄνδρα, καὶ εἶπεν αὐτῷ, εἰ σὺ εἶ ὁ ἀνὴρ, ὁ λαλήσας πρὸς τὴν γυναῖκα; καὶ εἶπεν ὁ ἄγγελος, ἐγώ.
\vs{12}Καὶ εἶπε Μανωὲ, νῦν ἐλεύσεται ὁ λόγος· τίς ἔσται κρίσις τοῦ παιδίου καὶ τὰ ποιήματα αὐτοῦ;
\vs{13}Καὶ εἶπεν ὁ ἄγγελος Κυρίου πρὸς Μανωὲ, ἀπὸ πάντων ὧν εἴρηκα πρὸς τὴν γυναῖκα, φυλάξεται·
\vs{14}Ἀπὸ παντὸς ὃ ἐκπορεύεται ἐξ ἀμπέλου τοῦ οἴνου, οὐ φάγεται, καὶ οἶνον καὶ μέθυσμα μὴ πιέτω, καὶ πᾶν ἀκάθαρτον μὴ φαγέτω· πάντα ὅσα ἐνετειλάμην αὐτῇ, φυλάξεται.

\vs{15}Καὶ εἶπε Μανωὲ πρὸς τὸν ἄγγελον Κυρίου, κατάσχωμεν ὧδέ σε, καὶ ποιήσωμεν ἐνώπιόν σου ἔριφον αἰγῶν.
\vs{16}Καὶ εἶπεν ὁ ἄγγελος Κυρίου πρὸς Μανωὲ, ἐὰν κατάσχῃς, οὐ φάγομαι ἀπὸ τῶν ἄρτων σου· καὶ ἐὰν ποιήσῃς ὁλοκαύτωμα, τῷ Κυρίῳ ἀνοίσεις αὐτό· ὅτι οὐκ ἔγνω Μανωὲ, ὅτι ἄγγελος Κυρίου αὐτός.
\vs{17}Καὶ εἶπε Μανωὲ πρὸς τὸν ἄγγελον Κυρίου, τί τὸ ὄνομά σοι, ὅτι ἔλθοι τὸ ῥῆμά σου, καὶ δοξάσομέν σε;
\vs{18}Καὶ εἶπεν αὐτῷ ὁ ἄγγελος Κυρίου, εἰς τί τοῦτο ἐρωτᾷς τὸ ὄνομά μου; καὶ αὐτό ἐστι θαυμαστόν.
\vs{19}Καὶ ἔλαβε Μανωὲ τὸν ἔριφον τῶν αἰγῶν καὶ τὴν θυσίαν, καὶ ἀνήνεγκεν ἐπὶ τὴν πέτραν τῷ Κυρίῳ· καὶ διεχώρισε ποιῆσαι, καὶ Μανωὲ καὶ ἡ γυνὴ αὐτοῦ βλέποντες.
\vs{20}Καὶ ἐγένετο ἐν τῷ ἀναβῆναι τὴν φλόγα ἐπάνω τοῦ θυσιαστηρίου ἕως τοῦ οὐρανοῦ, καὶ ἀνέβη ὁ ἄγγελος Κυρίου ἐν τῇ φλογί. καὶ Μανωὲ καὶ ἡ γυνὴ αὐτοῦ βλέποντες, καὶ ἔπεσον ἐπὶ πρόσωπον αὐτῶν ἐπὶ τὴν γῆν.
\vs{21}Καὶ οὐ προσέθηκεν ἔτι ὁ ἄγγελος Κυρίου ὀφθῆναι πρὸς Μανωὲ καὶ πρὸς τὴν γυναῖκα αὐτοῦ· τότε ἔγνω Μανωὲ, ὅτι ἄγγελος Κυρίου οὗτος.
\vs{22}Καὶ εἶπε Μανωὲ πρὸς τὴν γυναῖκα αὐτοῦ, θανάτῳ ἀποθανούμεθα ὅτι Θεὸν εἴδομεν.
\vs{23}Καὶ εἶπεν αὐτῷ ἡ γυνὴ αὐτοῦ, εἰ ἤθελεν ὁ Κύριος θανατῶσαι ἡμᾶς, οὐκ ἂν ἔλαβεν ἐκ χειρὸς ἡμῶν ὁλοκαύτωμα καὶ θυσίαν, καὶ οὐκ ἂν ἔδειξεν ἡμῖν ταῦτα πάντα, καὶ καθὼς καιρός οὐκ ἂν ἠκούτισεν ἡμᾶς ταῦτα.

\vs{24}Καὶ ἔτεκεν ἡ γυνὴ υἱὸν, καὶ ἐκάλεσε τὸ ὄνομα αὐτοῦ, Σαμψών· καὶ ἡδρύνθη τὸ παιδάριον, καὶ εὐλόγησεν αὐτὸ Κύριος.
\vs{25}Καὶ ἤρξατο πνεῦμα Κυρίου συνεκπορεύεσθαι αὐτῷ ἐν παρεμβολῇ Δὰν, καὶ ἀναμέσον Σαραὰ καὶ ἀναμέσον Ἐσθαόλ.

\ch{14}
Καὶ κατέβη Σαμψὼν εἰς Θαμναθὰ, καὶ εἶδε γυναῖκα ἐν Θαμναθὰ ἀπὸ τῶν θυγατέρων τῶν ἀλλοφύλων.
\vs{2}Καὶ ἀνέβη καὶ ἀπήγγειλε τῷ πατρὶ αὐτοῦ καὶ τῇ μητρὶ αὐτοῦ, καὶ εἶπε, γυναῖκα ἑώρακα ἐν Θαμναθὰ ἀπὸ τῶν θυγατέρων Φυλιστιῒμ, καὶ νῦν λάβετε αὐτήν μοι εἰς γυναῖκα.
\vs{3}Καὶ εἶπεν αὐτῷ ὁ πατὴρ αὐτοῦ, καὶ ἡ μήτηρ αὐτοῦ, μὴ οὐκ εἰσὶ θυγατέρες τῶν ἀδελφῶν σου, καὶ ἐκ παντὸς τοῦ λαοῦ μου γυνὴ, ὅτι συ πορεύῃ λαβεῖν γυναῖκα ἀπὸ τῶν ἀλλοφύλων τῶν ἀπεριτμήτων;

Καὶ εἶπε Σαμψὼν πρὸς τὸν πατέρα αὐτοῦ, ταύτην λάβε μοι, ὅτι αὕτη εὐθεῖα ἐν ὀφθαλμοῖς μου.
\vs{4}Καὶ ὁ πατὴρ αὐτοῦ καὶ ἡ μήτηρ αὐτοῦ οὐκ ἔγνωσαν ὅτι παρὰ Κυρίου ἐστὶν, ὅτι ἐκδίκησιν αὐτὸς ζητεῖ ἐκ τῶν ἀλλοφύλων· καὶ ἐν τῷ καιρῷ ἐκείνῳ οἱ ἀλλόφυλοι κυριεύοντες ἐν Ἰσραήλ.
\vs{5}Καὶ κατέβη Σαμψὼν καὶ ὁ πατὴρ αὐτοῦ καὶ ἡ μήτηρ αὐτοῦ εἰς Θαμναθά· καὶ ἦλθεν ἕως τοῦ ἀμπελῶνος Θαμναθὰ, καὶ ἰδοὺ σκύμνος λέοντος ὠρυόμενος εἰς συνάντησιν αὐτοῦ.
\vs{6}Καὶ ἥλατο ἐπʼ αὐτὸν πνεῦμα Κυρίου, καὶ συνέτριψεν αὐτὸν ὡσεὶ συντρίψει ἔριφον αἰγῶν, καὶ οὐδὲν ἦν ἐν ταῖς χερσὶν αὐτοῦ· καὶ οὐκ ἀπήγγειλε τῷ πατρὶ αὐτοῦ καὶ τῇ μητρὶ αὐτοῦ ὃ ἐποίησε.
\vs{7}Καὶ κατέβησαν καὶ ἐλάλησαν τῇ γυναικὶ, καὶ ηὐθύνθη ἐν ὀφθαλμοῖς Σαμψών.

\vs{8}Καὶ ὑπέστρεψε μεθʼ ἡμέρας λαβεῖν αὐτὴν, καὶ ἐξέκλινεν ἰδεῖν τὸ πτῶμα τοῦ λέοντος, καὶ ἰδοὺ συναγωγὴ μελισσῶν ἐν τῷ στόματι τοῦ λέοντος καὶ μέλι.
\vs{9}Καὶ ἐξεῖλεν αὐτὸ εἰς χεῖρας αὐτοῦ, καὶ ἐπορεύετο πορευόμενος καὶ ἐσθίων· καὶ ἐπορεύθη πρὸς τὸν πατέρα αὐτοῦ καὶ πρὸς τὴν μητέρα αὐτοῦ, καὶ ἔδωκεν αὐτοῖς καὶ ἔφαγον, καὶ οὐκ ἀπήγγειλεν αὐτοῖς ὅτι ἀπὸ στόματος τοῦ λέοντος ἐξεῖλε τὸ μέλι.

\vs{10}Καὶ κατέβη ὁ πατὴρ αὐτοῦ πρὸς τὴν γυναῖκα, καὶ ἐποίησεν ἐκεῖ Σαμψὼν πότον ἡμέρας ἑπτὰ, ὅτι οὕτως ποιοῦσιν οἱ νεανίσκοι.
\vs{11}Καὶ ἐγένετο ὅτε εἶδον αὐτὸν, καὶ ἔλαβον τριάκοντα κλητοὺς, καὶ ἦσαν μετʼ αὐτοῦ.

\vs{12}Καὶ εἶπεν αὐτοῖς Σαμψὼν, πρόβλημα ὑμῖν προβάλλομαι, ἐὰν ἀπαγγέλλοντες ἀπαγγείλητε αὐτὸ ἐν ταῖς ἑπτὰ ἡμέραις τοῦ πότου καὶ εὕρητε, δώσω ὑμῖν τριάκοντα σινδόνας καὶ τριάκοντα στολὰς ἱματίων.
\vs{13}Καὶ ἐὰν μὴ δύνησθε ἀπαγγεῖλαί μοι, δώσετε ὑμεῖς ἐμοὶ τριάκοντα ὀθόνια καὶ τριάκοντα ἀλλασσομένας στολὰς ἱματίων· καὶ εἶπαν αὐτῷ, προβάλου τὸ πρόβλημά σου, καὶ ἀκουσόμεθα αὐτό.
\vs{14}Καὶ εἶπεν αὐτοῖς, τὶ βρωτὸν ἐξῆλθεν ἐκ βιβρώσκοντος, καὶ ἀπὸ ἰσχυροῦ γλυκύ· καὶ οὐκ ἠδύναντο ἀπαγγεῖλαι τὸ πρόβλημα ἐπὶ τρεῖς ἡμέρας.

\vs{15}Καὶ ἐγένετο ἐν τῇ ἡμέρᾳ τῇ τετάρτῃ, καὶ εἶπαν τῇ γυναικὶ Σαμψὼν, ἀπάτησον δὴ τὸν ἄνδρα σου, καὶ ἀπαγγειλάτω σοι τὸ πρόβλημα, μή ποτε κατακαύσωμέν σε καὶ τὸν οἶκον τοῦ πατρός σου ἐν πυρί· ἢ ἐκβιᾶσαι ἡμᾶς κεκλήκατε;
\vs{16}Καὶ ἔκλαυσεν ἡ γυνὴ Σαμψὼν πρὸς αὐτὸν, καὶ εἶπε, πλὴν μεμίσηκάς με καὶ οὐκ ἠγάπησάς με, ὅτι τὸ πρόβλημα ὃ προεβάλου τοῖς υἱοῖς τοῦ λαοῦ μου, οὐκ ἀπήγγειλάς μοι αὐτό· καὶ εἶπεν αὐτῇ Σαμψὼν, εἰ τῷ πατρί μου καὶ τῇ μητρί μου οὐκ ἀπήγγελκα, σοὶ ἀπαγγείλω;
\vs{17}Καὶ ἔκλαυσε πρὸς αὐτὸν ἐπὶ τὰς ἑπτὰ ἡμέρας, ἃς ἦν αὐτοῖς ὁ πότος· καὶ ἐγένετο ἐν τῇ ἡμέρᾳ τῇ ἑβδόμῃ, καὶ ἀπήγγειλεν αὐτῇ, ὅτι παρηνόχλησεν αὐτῷ· καὶ αὐτὴ ἀπήγγειλε τοῖς υἱοῖς τοῦ λαοῦ αὐτῆς.
\vs{18}Καὶ εἶπαν αὐτῷ οἱ ἄνδρες τῆς πόλεως ἐν τῇ ἡμέρᾳ τῇ ἑβδόμῃ πρὸ τοῦ ἀνατεῖλαι τὸν ἥλιον, τί γλυκύτερον μέλιτος, καὶ τί ἰσχυρότερον λέοντος; καὶ εἶπεν αὐτοῖς Σαμψὼν, εἰ μὴ ἠροτριάσατε ἐν τῇ δαμάλει μου, οὐκ ἂν ἔγνωτε τὸ πρόβλημά μου.
\vs{19}Καὶ ἥλατο ἐπʼ αὐτὸν πνεῦμα Κυρίου, καὶ κατέβη εἰς Ἀσκάλωνα, καὶ ἐπάταξεν ἐξ αὐτῶν τριάκοντα ἄνδρας, καὶ ἔλαβε τὰ ἱμάτια αὐτῶν, καὶ ἔδωκε τὰς στολὰς τοῖς ἀπαγγείλασι τὸ πρόβλημα· Καὶ ὠργίσθη θυμῷ Σαμψὼν, καὶ ἀνέβη εἰς τὸν οἶκον τοῦ πατρὸς αὐτοῦ.
\vs{20}Καὶ ἐγένετο ἡ γυνὴ Σαμψὼν ἑνὶ τῶν φίλων αὐτοῦ, ὧν ἐφιλίασε.

\ch{15}
Καὶ ἐγένετο μεθʼ ἡμέρας ἐν ἡμέραις θερισμοῦ πυρῶν, καὶ ἐπεσκέψατο Σαμψὼν τὴν γυναῖκα αὐτοῦ ἐν ἐρίφῳ αἰγῶν, καὶ εἶπεν, εἰσελεύσομαι πρὸς τὴν γυναῖκά μου καὶ εἰς τὸ ταμεῖον· καὶ οὐκ ἔδωκεν αὐτὸν ὁ πατὴρ αὐτῆς εἰσελθεῖν.
\vs{2}Καὶ εἶπεν ὁ πατὴρ αὐτὴς, λέγων, εἶπα ὅτι μισῶν ἐμίσησας αὐτὴν, καὶ ἔδωκα αὐτὴν ἑνὶ τῶν ἐκ τῶν φίλων σου· μὴ οὐχὶ ἡ ἀδελφὴ αὐτῆς ἡ νεωτέρα ἀγαθωτέρα ὑπὲρ αὐτήν; ἔστω δή σοι ἀντὶ αὐτῆς.

\vs{3}Καὶ εἶπεν αὐτοῖς Σαμψὼν, ἠθὼωμαι καὶ τὸ ἅπαξ ἀπὸ ἀλλοφύλων, ὅτι ποιῶ ἐγὼ μετʼ αὐτῶν πονηρίαν·
\vs{4}Καὶ ἐπορεύθη Σαμψὼν, καὶ συνέλαβε τριακοσίας ἀλώπεκας, καὶ ἔλαβε λαμπάδας, καὶ ἐπέστρεψε κέρκον πρὸς κέρκον, καὶ ἔθηκε λαμπάδα μίαν ἀναμέσον τῶν δύο κέρκων καὶ ἔδησε,
\vs{5}καὶ ἐξέκαυσε πῦρ ἐν ταῖς λαμπάσι, καὶ ἐξαπέστειλεν ἐν τοῖς στάχυσι τῶν ἀλλοφύλων· καὶ ἐκάησαν ἀπὸ ἅλωνος καὶ ἕως σταχύων ὀρθῶν, καὶ ἕως ἀμπελῶνος καὶ ἐλαίας.
\vs{6}Καὶ εἶπαν οἱ ἀλλόφυλοι, τίς ἐποίησε ταῦτα; καὶ εἶπαν, Σαμψὼν ὁ νυμφίος τοῦ Θαμνὶ, ὅτι ἔλαβε τὴν γυναῖκα αὐτοῦ, καὶ ἔδωκεν αὐτὴν τῷ ἐκ τῶν φίλων αὐτοῦ· καὶ ἀνέβησαν οἱ ἀλλόφυλοι, καὶ ἐνέπρησαν αὐτὴν καὶ τὸν οἶκον τοῦ πατρὸς αὐτῆς ἐν πυρί.

\vs{7}Καὶ εἶπεν αὐτοῖς Σαμψὼν, ἐὰν ποιήσητε οὕτως ταύτην, ὅτι ἦ μὴν ἐκδικήσω ἐν ὑμῖν, καὶ ἔσχατον κοπάσω.
\vs{8}Καὶ ἐπάταξεν αὐτοὺς κνήμην ἐπὶ μηρὸν πληγὴν μεγάλην· καὶ κατέβη καὶ ἐκάθισεν ἐν τρυμαλιᾷ τῆς πέτρας Ἠτάμ.

\vs{9}Καὶ ἀνέβησαν οἱ ἀλλόφυλοι, καὶ παρενέβαλον ἐν Ἰούδα, καὶ ἐξεῤῥίφησαν ἐν Λεχί.
\vs{10}Καὶ εἶπαν ἀνὴρ Ἰούδα, εἰς τί ἀνέβητε ἐφʼ ἡμᾶς; καὶ εἶπον οἱ ἀλλόφυλοι, δῆσαι τὸν Σαμψὼν ἀνέβημεν, καὶ ποιῆσαι αὐτῷ ὃν τρόπον ἐποίησεν ἡμῖν.
\vs{11}Καὶ κατέβησαν τρισχίλιοι ἀπὸ Ἰούδα ἄνδρες εἰς τρυμαλιὰν πέτρας Ἠτὰμ, καὶ εἶπαν πρὸς Σαμψὼν, οὐκ οἶδας ὅτι κυριεύουσιν οἱ ἀλλόφυλοι ἡμῶν; καὶ τί τοῦτο ἐποίησας ἡμῖν; καὶ εἶπεν αὐτοῖς Σαμψὼν, ὃν τρόπον ἐποίησάν μοι, οὕτως ἐποίησα αὐτοῖς·
\vs{12}Καὶ εἶπαν αὐτῷ, δῆσαί σε κατέβημεν τοῦ δοῦναί σε ἐν χειρὶ ἀλλοφύλων· καὶ εἶπεν αὐτοῖς Σαμψὼν, ὀμόσατέ μοι μή ποτε συναντήσητε ἐν ἐμοὶ ὑμεῖς.
\vs{13}Καὶ εἶπον αὐτῷ, λέγοντες, οὐχὶ, ὅτι ἀλλʼ ἢ δεσμῷ δήσομέν σε, καὶ παραδώσωμέν σε ἐν χειρὶ αὐτῶν, καὶ θανάτῳ οὐ θανατώσωμέν σε· καὶ ἔδησαν αὐτὸν ἐν δυσὶ καλωδίοις καινοῖς, καὶ ἀνήνεγκαν αὐτὸν ἀπὸ τῆς πέτρας ἐκείνης.

\vs{14}Καὶ ἦλθον ἕως σιαγόνος· καὶ οἱ ἀλλόφυλοι ἠλάλαξαν, καὶ ἔδραμον εἰς συνάντησιν αὐτοῦ· καὶ ἥλατο ἐπʼ αὐτὸν πνεῦμα Κυρίου· καὶ ἐγενήθη τὰ καλώδια τὰ ἐπὶ βραχίοσιν αὐτοῦ ὡσεὶ στυππίον ὃ ἐξεκαύθη ἐν πυρί· καὶ ἐτάκησαν δεσμοὶ αὐτοῦ ἀπὸ χειρῶν αὐτοῦ.
\vs{15}Καὶ εὗρε σιαγόνα ὄνου ἐξεῤῥιμμένην, καὶ ἐξέτεινε τὴν χεῖρα αὐτοῦ καὶ ἔλαβεν αὐτὴν, καὶ ἐπάταξεν ἐν αὐτῇ χιλίους ἄνδρας.
\vs{16}Καὶ εἶπε Σαμψὼν, ἐν σιαγόνι ὄνου ἐξαλείφων ἐξήλειψα αὐτοὺς, ὅτι ἐν τῇ σιαγόνι τοῦ ὄνου ἐπάταξα χιλίους ἄνδρας.
\vs{17}Καὶ ἐγένετο ὡς ἐπαύσατο λαλῶν, καὶ ἔῤῥιψε τὴν σιαγόνα ἐκ τῆς χειρὸς αὐτοῦ· καὶ ἐκάλεσε τὸν τόπον ἐκεῖνον, ἀναίρεσις σιαγόνος.

\vs{18}Καὶ ἐδίψησε σφόδρα, καὶ ἔκλαυσε πρὸς Κύριον, καὶ εἶπε, σὺ εὐδόκησας ἐν χειρὶ δούλου σου τὴν σωτηρίαν τὴν μεγάλην ταύτην, καὶ νῦν ἀποθανοῦμαι τῷ δίψει, καὶ ἐμπεσοῦμαι ἐν χειρὶ τῶν ἀπεριτμήτων;
\vs{19}Καὶ ἔῤῥηξεν ὁ Θεὸς τὸν λάκκον τὸν ἐν τῇ σιαγόνι, καὶ ἐξῆλθεν ἐκ αὐτοῦ ὕδωρ, καὶ ἔπιε· καὶ ἐπέστρεψε τὸ πνεῦμα αὐτοῦ καὶ ἔζησε. διὰ τοῦτο ἐκλήθη τὸ ὄνομα αὐτῆς, Πηγὴ τοῦ ἐπικαλουμένου, ἥ ἐστιν ἐν σιαγόνι, ἕως τῆς ἡμέρας ταύτης.

\vs{20}Καὶ ἔκρινε τὸν Ἰσραὴλ ἐν ἡμέραις ἀλλοφύλων εἴκοσι ἔτη.

\ch{16}
Καὶ ἐπορεύθη Σαμψὼν εἰς Γάζαν, καὶ εἶδεν ἐκεῖ γυναῖκα πόρνην, καὶ εἰσῆλθε πρὸς αὐτήν.
\vs{2}Καὶ ἀνηγγέλη τοῖς Γαζαίοις, λέγοντες, ἥκει Σαμψὼν ὧδε. καὶ ἐκύκλωσαν, καὶ ἐνήδρευσαν ἐπʼ αὐτὸν ὅλην τὴν νύκτα ἐν τῇ πύλῃ τῆς πόλεως· καὶ ἐκώφευσαν ὅλην τὴν νύκτα, λέγοντες, ἕως διαφαύσῃ ὁ ὄρθρος, καὶ φονεύσωμεν αὐτόν.
\vs{3}Καὶ ἐκοιμήθη Σαμψὼν ἕως μεσονυκτίου, καὶ ἀνέστη ἐν ἡμίσει τῆς νυκτὸς, καὶ ἐπελάβετο τῶν θυρῶν τῆς πύλης τῆς πόλεως σὺν τοῖς δυσὶ σταθμοῖς, καὶ ἀνεβάσταζεν αὐτὰς σὺν τῷ μοχλῷ, καὶ ἔθηκεν ἐπὶ ὤμων αὐτοῦ· καὶ ἀνέβη ἐπὶ τὴν κορυφὴν τοῦ ὄρους τοῦ ἐπὶ προσώπου τοῦ Χεβρῶν, καὶ ἔθηκεν αὐτὰ ἐκεῖ.

\vs{4}Καὶ ἐγένετο μετὰ τοῦτο, καὶ ἠγάπησε γυναῖκα ἐν Ἀλσωρήχ· καὶ ὄνομα αὐτῇ Δαλιδά.
\vs{5}Καὶ ἀνέβησαν πρὸς αὐτὴν οἱ ἄρχοντες τῶν ἀλλοφύλων, καὶ εἶπαν αὐτῇ, ἀπάτησον αὐτὸν, καὶ ἴδε ἐν τίνι ἡ ἰσχὺς αὐτοῦ ἡ μεγάλη, καὶ ἐν τίνι δυνησόμεθα αὐτῷ, καὶ δήσομεν αὐτὸν τοῦ ταπεινῶσαι αὐτόν· καὶ ἡμεῖς δώσομέν σοι ἀνὴρ χιλίους καὶ ἑκατὸν ἀργυρίου.

\vs{6}Καὶ εἶπε Δαλιδὰ πρὸς Σαμψὼν, ἀπάγγειλον δή μοι ἐν τίνι ἡ ἰσχύς σου ἡ μεγάλη, καὶ ἐν τίνι δεθήσῃ τοῦ ταπεινωθῆναί σε.
\vs{7}Καὶ εἶπε πρὸς αὐτὴν Σαμψὼν, ἐὰν δήσωσί με ἐν ἑπτὰ νευραῖς ὑγραῖς μὴ διεφθαρμέναις, καὶ ἀσθενήσω καὶ ἔσομαι ὡς εἷς τῶν ἀνθρώπων.
\vs{8}Καὶ ἀνήνεγκαν αὐτῇ οἱ ἄρχοντες τῶν ἀλλοφύλων ἑπτὰ νευρὰς ὑγρὰς μὴ διεφθαρμένας, καὶ ἔδησεν αὐτὸν ἐν αὐταῖς.
\vs{9}Καὶ τὸ ἔνεδρον αὐτῇ ἐκάθητο ἐν τῷ ταμείῳ· καὶ εἶπεν αὐτῷ, ἀλλόφυλοι ἐπὶ σὲ Σαμψών· καὶ διέσπασε τὰς νευρὰς ὡς εἴ τις ἀποσπάσοι στρέμμα στυππίου ἐν τῷ ὀσφρανθῆναι αὐτὸ πυρὸς, καὶ οὐκ ἐγνώσθη ἡ ἰσχὺς αὐτοῦ.

\vs{10}Καὶ εἶπε Δαλιδὰ πρὸς Σαμψὼν, ἰδοὺ ἐπλάνησάς με, καὶ ἐλάλησας πρὸς μὲ ψευδῆ· νῦν οὖν ἀνάγγειλόν μοι ἐν τίνι δεθήσῃ.
\vs{11}Καὶ εἶπε πρὸς αὐτὴν, ἐὰν δεσμεύοντες δήσωσί με ἐν καλωδίοις καινοῖς οἷς οὐκ ἐγένετο ἐν αὐτοῖς ἔργον, καὶ ἀσθενήσω καὶ ἔσομαι ὡς εἶς τῶν ἀνθρώπων.
\vs{12}Καὶ ἔλαβε Δαλιδὰ καλώδια καινὰ, καὶ ἔδησεν αὐτὸν ἐν αὐτοῖς, καὶ τὰ ἔνεδρα ἐξῆλθεν ἐκ τοῦ ταμείου· καὶ εἶπεν, ἀλλόφυλοι ἐπὶ σὲ Σαμψών· καὶ διέσπασεν αὐτὰ ἀπὸ βραχιόνων αὐτοῦ ὡσεὶ σπαρτίον.

\vs{13}Καὶ εἶπε Δαλιδὰ πρὸς Σαμψὼν, ἰδοὺ ἐπλάνησάς με, καὶ ἐλάλησας πρὸς μὲ ψευδῆ· ἀνάγγειλον δή μοι ἐν τίνι δεθήσῃ· καὶ εἶπε πρὸς αὐτὴν, ἐὰν ὑφάνῃς τὰς ἑπτὰ σειρὰς τῆς κεφαλῆς μου σὺν τῷ διάσματι, καὶ ἐγκρούσῃς τῷ πασσάλῳ εἰς τὸν τοῖχον, καὶ ἔσομαι ὡς εἷς τῶν ἀνθρώπων ἀσθενής.
\vs{14}Καὶ ἐγένετο ἐν τῷ κοιμᾶσθαι αὐτὸν, καὶ ἔλαβε Δαλιδὰ τὰς ἑπτὰ σειρὰς τῆς κεφαλῆς αὐτοῦ, καὶ ὕφανεν ἐν τῷ διάσματι, καὶ ἔπηξεν τῷ πασσάῳ εἰς τὸν τοῖχον, καὶ εἶπεν, ἀλλόφυλοι ἐπὶ σὲ Σαμψών· καὶ ἐξυπνίσθη ἀπὸ τοῦ ὕπνου αὐτοῦ, καὶ ἐξῇρε τὸν πάσσαλον τοῦ ὑφάσματος ἐκ τοῦ τοίχου.

\vs{15}Καὶ εἶπε πρὸς Σαμψὼν Δαλιδὰ, πῶς λέγεις, ἠγάπηκά σε, καὶ ἡ καρδία σου οὐκ ἔστι μετʼ ἐμοῦ; τοῦτο τρίτον ἐπλάνησάς με καὶ οὐκ ἀπήγγειλάς μοι ἐν τίνι ἡ ἰσχύς σου ἡ μεγάλη.
\vs{16}Καὶ ἐγένετο ὅτε ἐξέθλιψεν αὐτὸν ἐν λόγοις αὐτῆς πάσας τὰς ἡμέρας, καὶ ἐστενοχώρησεν αὐτὸν, καὶ ὠλιγοψύχησεν ἕως τοῦ ἀποθανεῖν.
\vs{17}Καὶ ἀνήγγειλεν αὐτῇ πᾶσαν τὴν καρδίαν αὐτοῦ, καὶ εἶπεν αὐτῇ, σίδηρος οὐκ ἀνέβη ἐπὶ τὴν κεφαλήν μου, ὅτι ἅγιος Θεοῦ ἐγώ εἰμι ἀπὸ κοιλίας μητρός μου· ἐὰν οὖν ξυρήσωμαι, ἀποστήσεται ἀπʼ ἐμοῦ ἡ ἰσχύς μου καὶ ἀσθενήσω, καὶ ἔσομαι ὡς πάντες οἱ ἄνθρωποι.

\vs{18}Καὶ εἶδε Δαλιδὰ, ὅτι ἀπήγγειλεν αὐτῇ πᾶσαν τὴν καρδίαν αὐτοῦ· καὶ ἀπέστειλε καὶ ἐκάλεσε τοὺς ἄρχοντας τῶν ἀλλοφύλων, λέγουσα, ἀνάβητε ἔτι τὸ ἅπαξ τοῦτο, ὅτι ἀπήγγειλέ μοι πᾶσαν τὴν καρδίαν αὐτοῦ· καὶ ἀνέβησαν πρὸς αὐτὴν οἱ ἄρχοντες τῶν ἀλλοφύλων, καὶ ἀνήνεγκαν τὸ ἀργύριον ἐν χερσὶν αὐτῶν.
\vs{19}Καὶ ἐκοίμισε Δαλιδὰ τὸν Σαμψὼν ἐπὶ τὰ γόνατα αὐτῆς· καὶ ἐκάλεσεν ἄνδρα, καὶ ἐξύρησε τὰς ἑπτὰ σειρὰς τῆς κεφαλῆς αὐτοῦ, καὶ ἤρξατο ταπεινῶσαι αὐτὸν, καὶ ἀπέστη ἡ ἰσχὺς αὐτοῦ ἀπʼ αὐτοῦ.
\vs{20}Καὶ εἶπε Δαλιδά, ἀλλόφυλοι ἐπὶ σὲ Σαμψών. καὶ ἐξυπνίσθη ἐκ τοῦ ὕπνου αὐτοῦ, καὶ εἶπεν, ἐξελεύσομαι ὡς ἅπαξ καὶ ἅπαξ, καὶ ἐκτιναχθήσομαι· καὶ αὐτὸς οὐκ ἔγνω ὅτι ὁ Κύριος ἀπέστη ἀπάνωθεν αὐτοῦ.
\vs{21}Καὶ ἐκράτησαν αὐτὸν οἱ ἀλλόφυλοι, καὶ ἐξέκοψαν τοὺς ὀφθαλμοὺς αὐτοῦ, καὶ κατήνεγκαν αὐτὸν εἰς Γάζαν, καὶ ἐπέδησαν αὐτὸν ἐν πέδαις χαλκείαις· καὶ ἦν ἀλήθων ἐν οἴκῳ τοῦ δεσμωτηρίου.
\vs{22}Καὶ ἤρξατο θρὶξ τῆς κεφαλῆς αὐτοῦ βλαστάνειν καθὼς ἐξυρήσατο.

\vs{23}Καὶ οἱ ἄρχοντες τῶν ἀλλοφύλων συνήχθησαν θυσαιάσαι θυσίασμα μέγα τῷ Δαγὼν θεῷ αὐτῶν, καὶ εὐφρανθῆναι, καὶ εἶπαν, ἔδωκεν ὁ θεὸς ἐν χειρὶ ἡμῶν τὸν Σαμψὼν τὸν ἐχθρὸν ἡμῶν.
\vs{24}Καὶ εἶδον αὐτὸν ὁ λαὸς, καὶ ὕμνησαν τὸν θεὸν αὐτῶν, ὅτι παρέδωκεν ὁ θεὸς ἡμῶν τὸν ἐχθρὸν ἡμῶν ἐν χειρὶ ἡμῶν, τὸν ἐρημοῦντα τὴν γῆν ἡμῶν, καὶ ὃς ἐπλήθυνε τοὺς τραυματίας ἡμῶν.
\vs{25}Καὶ ὅτε ἠγαθύνθη ἡ καρδία αὐτῶν, καὶ εἶπαν, καλέσατε τὸν Σαμψὼν ἐξ οἴκου φυλακῆς, καὶ παιξάτω ἐνώπιον ἡμῶν· καὶ ἐκάλεσαν τὸν Σαμψὼν ἐξ οἴκου δεσμωτηρίου, καὶ ἔπαιζεν ἐνώπιον αὐτῶν· καὶ ἐῤῥάπιζον αὐτὸν, καὶ ἔστησαν αὐτὸν ἀναμέσον τῶν κιόνων.
\vs{26}Καὶ εἶπε Σαμψὼν πρὸς τὸν νεανίαν τὸν κρατοῦντα τὴν χεῖρα αὐτοῦ, ἄφες με, καὶ ψηλαφήσω τοὺς κίονας ἐφʼ οἷς ὁ οἶκος ἐπʼ αὐτοὺς, καὶ ἐπιστηριχθήσομαι ἐπʼ αὐτούς.
\vs{27}Καὶ ὁ οἶκος πλήρης τῶν ἀνδρῶν καὶ τῶν γυναικῶν, καὶ ἐκεῖ πάντες οἱ ἄρχοντες τῶν ἀλλοφύλων, καὶ ἐπὶ τὸ δῶμα ὡσεὶ τρισχίλιοι ἄνδρες καὶ γυναῖκες οἱ θεωροῦντες ἐν παιγνίαις Σαμψών.

\vs{28}Καὶ ἔκλαυσε Σαμψὼν πρὸς Κύριον, καὶ εἶπεν, ἀδωναϊὲ Κύριε μνήσθητι δή μου, καὶ ἐνίσχυσόν με ἔτι τὸ ἅπαξ τοῦτο Θεὲ, καὶ ἀνταποδώσω ἀνταπόδοσιν μίαν περὶ τῶν δύο ὀφθαλμῶν μου τοῖς ἀλλοφύλοις.
\vs{29}Καὶ περιέλαβε Σαμψὼν τοὺς δύο κίονας τοῦ οἴκου ἐφʼ οὓς ὁ οἶκος εἱστήκει, καὶ ἐπεστηρίχθη ἐπʼ αὐτοὺς, καὶ ἐκράτησεν ἕνα τῇ δεξιᾷ αὐτοῦ, καὶ ἕνα τῇ ἀριστερᾷ αὐτοῦ.
\vs{30}Καὶ εἶπε Σαμψὼν, ἀποθανέτω ψυχή μου μετὰ τῶν ἀλλοφύλων· καὶ ἐβάσταξεν ἐν ἰσχύϊ· καὶ ἔπεσεν ὁ οἶκος ἐπὶ τοὺς ἄρχοντας, καὶ ἐπὶ πάντα τὸν λαὸν τὸν ἐν αὐτῷ· καὶ ἦσαν οἱ τεθνηκότες οὓς ἐθανάτωσε Σαμψὼν ἐν τῷ θανάτῳ αὐτοῦ, πλείους ἢ οὓς ἐθανάτωσεν ἐν τῇ ζωῇ αὐτοῦ.

\vs{31}Καὶ κατέβησαν οἱ ἀδελφοὶ αὐτοῦ, καὶ ὁ οἶκος τοῦ πατρὸς αὐτοῦ, καὶ ἔλαβον αὐτόν· καὶ ἀνέβησαν καὶ ἔθαψαν αὐτὸν ἀναμέσον Σαραὰ καὶ ἀναμέσον Ἐσθαὸλ ἐν τῷ τάφῳ Μανωὲ τοῦ πατρὸς αὐτοῦ· καὶ αὐτὸς ἔκρινε τὸν Ἰσραὴλ εἴκοσι ἔτη.

\ch{17}
Καὶ ἐγένετο ἀνὴρ ἀπὸ ὄρους Ἐφραὶμ, καὶ ὄνομα αὐτῷ Μιχαίας.
\vs{2}Καὶ εἶπε τῇ μητρὶ αὐτοῦ, οἱ χίλιοι καὶ ἑκατὸν οὗς ἔλαβες ἀργυρίου σεαυτῇ, καί με ἠράσω, καὶ προσεῖπας ἐν ὠσί μου, ἰδοὺ τὸ ἀργύριον παρʼ ἐμοὶ, ἐγὼ ἔλαβον αὐτό. καὶ εἶπεν ἡ μήτηρ αὐτοῦ, εὐλογητὸς ὁ υἱός μου τῷ Κυρίῳ.
\vs{3}Καὶ ἀπέδωκε τοὺς χιλίους καὶ ἑκατὸν τοῦ ἀργυρίου τῇ μητρὶ αὐτοῦ· καὶ εἶπεν ἡ μήτηρ αὐτοῦ, ἁγιάζουσα ἡγίασα τὸ ἀργύριον τῷ Κυρίῳ ἐκ τῆς χειρός μου τῷ υἱῷ μου τοῦ ποιῆσαι γλυπτὸν καὶ χωνευτὸν, καὶ νῦν ἀποδώσω αὐτό σοι.
\vs{4}Καὶ ἀπέδωκε τὸ ἀργύριον τῇ μητρὶ αὐτοῦ· καὶ ἔλαβεν ἡ μήτηρ αὐτοῦ διακοσίους ἀργυρίου, καὶ ἔδωκεν αὐτὸ ἀργυροκόπῳ, καὶ ἐποίησεν αὐτὸ γλυπτὸν καὶ χωνευτόν· καὶ ἐγενήθη ἐν οἴκῳ Μιχαία.
\vs{5}Καὶ ὁ οἶκος Μιχαία αὐτῷ οἶκος Θεοῦ· καὶ ἐποίησεν ἐφὼδ καὶ θαραφίν· καὶ ἐπλήρωσε τὴν χεῖρα ἀπὸ ἑνὸς υἱῶν αὐτοῦ, καὶ ἐγένετο αὐτῷ εἰς ἱερέα.

\vs{6}Ἐν δὲ ταῖς ἡμέραις ἐκείναις οὐκ ἦν βασιλεὺς ἐν Ἰσραήλ· ἀνὴρ τὸ εὐθὲς ἐν ὀφθαλμοῖς αὐτοῦ ἐποίει.

\vs{7}Καὶ ἐγενήθη νεανίας ἐκ Βηθλεέμ δήμου Ἰούδα, καὶ αὐτὸς Λευίτης, καὶ οὗτος παρῴκει ἐκεῖ.
\vs{8}Καὶ ἐπορεύθη ὁ ἀνὴρ ἀπὸ Βηθλεὲμ τῆς πόλεως Ἰούδα παροικῆσαι ἐν ᾧ ἐὰν εὕρῃ τόπῳ· καὶ ἦλθεν ἕως ὄρους Ἐφραὶμ, καὶ ἕως οἴκου Μιχαία τοῦ ποιῆσαι ὁδὸν αὐτοῦ.
\vs{9}Καὶ εἶπεν αὐτῷ Μιχαίας, πόθεν ἔρχῃ; καὶ εἶπε πρὸς αὐτὸν, Λευίτης εἰμὶ ἐκ Βηθλεὲμ Ἰούδα, καὶ ἐγὼ πορεύομαι παροικῆσαι ἐν ᾧ ἐὰν εὕρω τόπῳ.
\vs{10}Καὶ εἶπεν αὐτῷ Μιχαίας, κάθου μετʼ ἐμοῦ, καὶ γίνου μοι εἰς πατέρα καὶ εἰς ἱερέα, καὶ ἐγὼ δώσω σοι δέκα ἀργυρίου εἰς ἡμέραν, καὶ στολὴν ἱματίων, καὶ τὰ πρὸς ζωήν σου.
\vs{11}Καὶ ἐπορεύθη ὁ Λευίτης, καὶ ἤρξατο παροικεῖν παρὰ τῷ ἀνδρί· καὶ ἐγενήθη ὁ νεανίας αὐτῷ ὡς εἷς ἀπὸ υἱῶν αὐτοῦ.
\vs{12}Καὶ ἐπλήρωσε Μιχαίας τὴν χεῖρα τοῦ Λευίτου, καὶ ἐγένετο αὐτῷ εἰς ἱερέα, καὶ ἐγένετο ἐν τῷ οἴκῳ Μιχαία.
\vs{13}Καὶ εἶπε Μιχαίας, νῦν ἔγνων ὅτι ἀγαθυνεῖ μοι Κύριος, ὅτι ἐγένετό μοι ὁ Λευίτης εἰς ἱερέα.

\ch{18}
Ἐν ταῖς ἡμέραις ἐκείναις οὐκ ἦν βασιλεὺς ἐν Ἰσραήλ· καὶ ἐν ταῖς ἡμέραις ἐκείναις ἡ φυλὴ Δὰν ἐζήτει ἑαυτῇ κληρονομίαν κατοικῆσαι, ὅτι οὐκ ἐνέπεσεν αὐτῇ ἕως τῆς ἡμέρας ἐκείνης ἐν μέσῳ φυλῶν υἱῶν Ἰσραὴλ κληρονομία.
\vs{2}Καὶ ἀπέστειλαν οἱ υἱοὶ Δὰν ἀπὸ δήμων αὐτῶν πέντε ἄνδρας υἱοὺς δυνάμεως, ἀπὸ Σαραὰ καὶ ἀπὸ Ἐσθαὸλ τοῦ κατασκέψασθαι τὴν γῆν καὶ ἐξιχνιάσαι αὐτήν· καὶ εἶπαν πρὸς αὐτοὺς, πορεύεσθε καὶ ἐξιχνιάσατε τὴν γῆν· καὶ ἦλθον ἕως ὄρους Ἐφραὶμ ἕως οἴκου Μιχαία· καὶ ηὐλίσθησαν αὐτοὶ ἐκεῖ
\vs{3}ἐν οἴκῳ Μιχαία, καὶ αὐτοὶ ἐπέγνωσαν τὴν φωνὴν τοῦ νεανίσκου τοῦ Λευίτου, καὶ ἐξέκλιναν ἐκεῖ· καὶ εἶπαν αὐτῷ, τίς ἤνεγκέ σε ὧδε; καὶ σὺ τί ποιεῖς ἐν τῷ τόπῳ τούτῳ; καὶ τί σοι ὧδε;
\vs{4}Καὶ εἶπε πρὸς αὐτοὺς, οὕτω καὶ οὕτως ἐποίησέ μοι Μιχαίας, καὶ ἐμισθώσατό με, καὶ ἐγενόμην αὐτῷ εἰς ἱερέα.
\vs{5}Καὶ εἶπαν αὐτῷ, ἐπερώτησον δὴ ἐν τῷ Θεῷ, καὶ γνωσόμεθα εἰ εὐοδωθήσεται ἡ ὁδὸς ἡμῶν, ἐν ᾗ ἡμεῖς πορευόμεθα ἐν αὐτῇ.
\vs{6}Καὶ εἶπεν αὐτοῖς ὁ ἱερεὺς, πορεύεσθε ἐν εἰρήνῃ· ἐνώπιον Κυρίου ἡ ὁδὸς ὑμῶν, ἐν ᾗ πορεύεσθε ἐν αὐτῇ.

\vs{7}Καὶ ἐπορεύθησαν οἱ πέντε ἄνδρες, καὶ ἦλθον εἰς Λαισά· καὶ εἶδαν τὸν λαὸν τὸν ἐν μέσῳ αὐτῆς καθήμενον ἐπʼ ἐλπίδι, ὡς κρίσις Σιδωνίων ἡσυχάζουσα, καὶ οὐκ ἔστι διατρέπων ἢ καταισχύνων λόγον ἐν τῇ γῇ, κληρονόμος ἐκπιέζων θησαυροὺς, καὶ μακράν εἰσι Σιδωνίων, καὶ λόγον οὐκ ἔχουσι πρὸς ἄνθρωπον.
\vs{8}Καὶ ἦλθον οἱ πέντε ἄνδρες πρὸς τοὺς ἀδελφοὺς αὐτῶν εἰς Σαραὰ καὶ Ἐσθαὸλ, καὶ εἶπον τοῖς ἀδελφοῖς αὐτῶν, τί ὑμεῖς κάθησθε;
\vs{9}Καὶ εἶπαν, ἀνάστητε, καὶ ἀναβῶμεν ἐπʼ αὐτοὺς, ὅτι εἴδομεν τὴν γῆν, καὶ ἰδοὺ ἀγαθὴ σφόδρα, καὶ ὑμεῖς ἡσυχάζετε, μὴ ὀκνήσητε τοῦ πορευθῆναι, καὶ εἰσελθεῖν τοῦ κληρονομῆσαι τὴν γῆν.
\vs{10}Καὶ ἡνίκα ἐὰν ἔλθητε, εἰσελεύσεσθε πρὸς λαὸν ἐπʼ ἐλπίδι, καὶ ἡ γῆ πλατεῖα, ὅτι ἔδωκεν αὐτὴν ὁ Θεὸς ἐν χειρὶ ὑμῶν· τόπος ὅπου οὐκ ἔστιν ἐκεῖ ὑστέρημα παντὸς ῥήματος τῶν ἐν τῇ γῇ.

\vs{11}Καὶ ἀπῇραν ἐκεῖθεν ἀπὸ δήμων τοῦ Δὰν ἀπὸ Σαραὰ καὶ ἀπὸ Ἐσθαὸλ, ἑξακόσιοι ἄνδρες ἐζωσμένοι σκεύη παρατάξεως.
\vs{12}Καὶ ἀνέβησαν καὶ παρενέβαλον ἐν Καριαθιαρὶμ ἐν Ἰούδα. διὰ τοῦτο ἐκλήθη ἐν ἐκείνῳ τῷ τόπῳ, παρεμβολὴ Δὰν, ἕως τῆς ἡμέρας ταύτης· ἰδοὺ ὀπίσω Καριαθιαρίμ.

\vs{13}Καὶ παρῆλθον ἐκεῖθεν ὄρος Ἐφραὶμ, καὶ ἦλθον ἕως οἴκου Μιχαία.
\vs{14}Καὶ ἀπεκρίθησαν οἱ πέντε ἄνδρες οἱ πορευόμενοι κατασκέψασθαι τὴν γῆν Λαισὰ, καὶ εἶπαν πρὸς τοὺς ἀδελφοὺς, ἔγνωτε ὅτι ἐστὶν ἐν τῷ οἴκῳ τούτῳ ἐφὼδ καὶ θεραφὶν καὶ γλυπτὸν καὶ χωνευτόν· καὶ νῦν γνῶτε ὅ, τι ποιήσετε.
\vs{15}Καὶ ἐξέκλιναν ἐκεῖ, καὶ εἰσῆλθον εἰς τὸν οἶκον τοῦ νεανίσκου τοῦ Λευίτου, εἰς τὸν οἶκον Μιχαία, καὶ ἠρώτησαν αὐτὸν εἰς εἰρήνην.
\vs{16}Καὶ οἱ ἑξακόσιοι ἄνδρες οἱ ἀνεζωσμένοι τὰ σκεύη τῆς παρατάξεως αὐτῶν ἑστῶτες παρὰ θύρας τῆς πύλης, οἱ ἐκ τῶν υἱῶν Δάν.
\vs{17}Καὶ ἀνέβησαν οἱ πέντε ἄνδρες οἱ πορευθέντες κατασκέψασθαι τὴν γῆν,
\vs{18}καὶ εἰσῆλθον ἐκεῖ εἰς οἶκον Μιχαία, καὶ ὁ ἱερεὺς ἑστώς. Καὶ ἔλαβον τὸ γλυπτὸν καὶ τὸ ἐφὼδ καὶ τὸ θεραφὶν καὶ τὸ χωνευτόν· καὶ εἶπε πρὸς αὐτοὺς ὁ ἱερεὺς, τί ὑμεῖς ποιεῖτε;
\vs{19}Καὶ εἶπαν αὐτῷ, κώφευσον, ἐπίθες τὴν χεῖρά σου ἐπὶ τὸ στόμα σου, καὶ δεῦρο μεθʼ ἡμῶν, καὶ γένου ἡμῖν εἰς πατέρα καὶ εἰς ἱερέα· μὴ ἀγαθὸν εἶναί σε ἱερέα οἴκου ἀνδρὸς ἑνὸς, ἢ γενέσθαι σε ἱερέα φυλῆς καὶ οἴκου εἰς δῆμον Ἰσραήλ;
\vs{20}Καὶ ἠγαθύνθη ἡ καρδία τοῦ ἱερέως, καὶ ἔλαβε τὸ ἐφὼδ καὶ τὸ θεραφὶν καὶ τὸ γλυπτὸν καὶ τὸ χωνευτὸν, καὶ ἦλθεν ἐν μέσῳ τοῦ λαοῦ.

\vs{21}Καὶ ἐπέστρεψαν καὶ ἀπῆλθον, καὶ ἔθηκαν τὰ τέκνα καὶ τὴν κτῆσιν καὶ τὸ βάρος ἔμπροσθεν αὐτῶν.

\vs{22}Αὐτοὶ ἐμάκρυναν ἀπὸ οἴκου Μιχαία, καὶ ἰδοὺ Μιχαίας καὶ οἱ ἄνδρες οἱ ἐν ταῖς οἰκίαις ταῖς μετὰ οἴκου Μιχαία ἐβόησαν, καὶ κατελάβοντο τοὺς υἱοὺς Δάν.
\vs{23}Καὶ ἐπέστρεψαν οἱ υἱοὶ Δὰν τὸ πρόσωπον αὐτῶν, καὶ εἶπαν τῷ Μιχαίᾳ, τί ἐστί σοι, ὅτι ἐβόησας;
\vs{24}Καὶ εἶπε Μιχαίας, ὅτι τὸ γλυπτόν μου, ὃ ἐποίησα, ἐλάβετε, καὶ τὸν ἱερέα, καὶ ἐπορεύθητε· καὶ τί μοι ἔτι; καὶ τί τοῦτο λέγετε πρὸς μὲ, τί κράζεις;
\vs{25}Καὶ εἶπαν πρὸς αὐτὸν οἱ υἱοὶ Δὰν, μὴ ἀκουσθήτω δὴ φωνή σου μεθʼ ἡμῶν, μή ποτε συναντήσωσιν ὑμῖν ἄνδρες πικροὶ ψυχῇ, καὶ προσθήσουσι ψυχήν σου, καὶ τὴν ψυχὴν τοῦ οἴκου σου.
\vs{26}Καὶ ἐπορεύθησαν οἱ υἱοὶ Δὰν εἰς ὁδὸν αὐτῶν· καὶ εἶδε Μιχαίας, ὅτι δυνατώτεροί εἰσιν ὑπὲρ αὐτόν· καὶ ἐπέστρεψεν εἰς τὸν οἶκον αὐτοῦ.

\vs{27}Καὶ οἱ υἱοὶ Δὰν ἔλαβον ὃ ἐποιησε Μιχαίας, καὶ τὸν ἱερέα ὃς ἦν αὐτῷ, καὶ ἦλθον ἐπὶ Λαισὰ, ἐπὶ λαὸν ἡσυχάζοντα καὶ πεποιθότα ἐπʼ ἐλπίδι· καὶ ἐπάταξαν αὐτοὺς ἐν στόματι ῥομφαίας, καὶ τὴν πόλιν ἐνέπρησαν ἐν πυρί.
\vs{28}Καὶ οὐκ ἦν ὁ ῥυόμενος, ὅτι μακράν ἐστιν ἀπὸ Σιδωνίων, καὶ λόγος οὐκ ἔστιν αὐτοῖς μετὰ ἀνθρώπου· καὶ αὐτὴ ἐν τῇ κοιλάδι τοῦ οἴκου Ῥαάβ· καὶ ᾠκοδόμησαν τὴν πόλιν, καὶ κατεσκήνωσαν ἐν αὐτῇ,
\vs{29}καὶ ἐκάλεσαν τὸ ὄνομα τῆς πόλεως Δὰν, ἐν ὀνόματι Δὰν πατρὸς αὐτῶν, ὃς ἐτέχθη τῷ Ἰσραήλ· καὶ ἦν Οὐλαμαῒς ὄνομα τῆς πόλεως τοπρότερον.

\vs{30}Καὶ ἔστησαν ἑαυτοῖς οἱ υἱοὶ Δὰν τὸ γλυπτόν· καὶ Ἰωνάθαν υἱὸς Γηρσὼν υἱὸς Μανασσῆ αὐτὸς καὶ οἱ υἱοὶ αὐτοῦ ἦσαν ἱερεῖς τῇ φυλῇ Δὰν ἕως ἡμέρας τῆς ἀποικίας τῆς γῆς.
\vs{31}Καὶ ἔθηκαν ἑαυτοῖς τὸ γλυπτὸν ὁ ἐποίησε Μιχαίας, πάσας τὰς ἡμέρας ἃς ἦν ὁ οἶκος τοῦ Θεοῦ ἐν Σηλώμ· καὶ ἐγένετο ἐν ταῖς ἡμέραις ἐκείναις οὐκ ἦν βασιλεὺς ἐν Ἰσραήλ.

\ch{19}
Καὶ ἐγένετο ἀνὴρ Λευίτης παροικῶν ἐν μηροῖς ὄρους Ἐφραὶμ, καὶ ἔλαβεν αὐτῷ γυναῖκα παλλακὴν ἀπὸ Βηθλεὲμ Ἰούδα.
\vs{2}Καὶ ἐπορεύθη ἀπʼ αὐτοῦ ἡ παλλακὴ αὐτοῦ, καὶ ἀπῆλθε παρʼ αὐτοῦ εἰς οἶκον πατρὸς αὐτῆς εἰς Βηθλεὲμ Ἰούδα, καὶ ἦν ἐκεῖ ἡμέρας μηνῶν τεσσάρων.

\vs{3}Καὶ ἀνέστη ὁ ἀνὴρ αὐτῆς, καὶ ἐπορεύθη ὀπίσω αὐτῆς τοῦ λαλῆσαι ἐπὶ καρδίαν αὐτῆς, τοῦ ἐπιστρέψαι αὐτὴν αὐτῷ· καὶ νεανίας αὐτοῦ μετʼ αὐτοῦ, καὶ ζεῦγος ὄνων· ἡ δὲ εἰσήνεγκεν αὐτὸν εἰς οἶκον πατρὸς αὐτῆς· καὶ εἶδεν αὐτὸν ὁ πατὴρ τῆς νεάνιδος, καὶ ηὐφράνθη εἰς συνάντησιν αὐτοῦ.
\vs{4}Καὶ κατέσχεν αὐτὸν ὁ γαμβρὸς αὐτοῦ ὁ πατὴρ τῆς νεάνιδος, καὶ ἐκάθισε μετʼ αὐτοῦ ἐπὶ τρεῖς ἡμέρας, καὶ ἔφαγον καὶ ἔπιον, καὶ ηὐλίσθησαν ἐκεῖ.
\vs{5}Καὶ ἐγένετο τῇ ἡμέρᾳ τῇ τετάρτῃ, καὶ ὤρθρισαν τοπρωῒ καὶ ἀνέστη τοῦ πορευθῆναι, καὶ εἶπεν ὁ πατὴρ τῆς νεάνιδος πρὸς τὸν νυμφίον αὐτοῦ, στήρισον τὴν καρδιάν σου ψωμῷ ἄρτου, καὶ μετὰ τοῦτο πορεύσεσθε.
\vs{6}Καὶ ἐκάθισαν καὶ ἔφαγον οἱ δύο ἐπὶ τὸ αὐτὸ καὶ ἔπιον· καὶ εἶπεν ὁ πατὴρ τῆς νεάνιδος πρὸς τὸν ἄνδρα, ἄγε δὴ αὐλίσθητι, καὶ ἀγαθυνθήσεται ἡ καρδία σου.
\vs{7}Καὶ ἀνέστη ὁ ἀνὴρ τοῦ πορεύεσθαι αὐτός· καὶ ἐβιάσατο αὐτὸν ὁ γαμβρὸς αὐτοῦ, καὶ ἐκάθισε καὶ ηὐλίσθη ἐκεῖ.

\vs{8}Καὶ ὤρθρισε τοπρωῒ τῇ ἡμέρᾳ τῇ πέμπτῃ τοῦ πορευθῆναι· καὶ εἶπεν ὁ πατὴρ τῆς νεάνιδος, στήρισον δὴ τὴν καρδίαν σου, καὶ στράτευσον ἕως κλῖναι τὴν ἡμέραν· καὶ ἔφαγον οἱ δύο.
\vs{9}Καὶ ἀνέστη ὁ ἀνὴρ τοῦ πορευθῆναι αὐτὸς, καὶ ἡ παλλακὴ αὐτοῦ, καὶ ὁ νεανίας αὐτοῦ· καὶ εἶπεν αὐτῷ ὁ γάμβρὸς αὐτοῦ ὁ πατὴρ τῆς νεάνιδος, ἰδοὺ δὴ ἠσθένησεν ἡμέρα εἰς τὴν ἑσπέραν· αὐλίσθητι ὧδε, καὶ ἀγαθυνθήσεται ἡ καρδία σου, καὶ ὀρθριεῖτε αὔριον εἰς ὁδὸν ὑμῶν, καὶ πορεύσῃ εἰς τὸ σκήνωμά σου.
\vs{10}Καὶ οὐκ εὐδόκησεν ὁ ἀνὴρ αὐλισθῆναι, καὶ ἀνέστη καὶ ἀπῆλθε, καὶ ἦλθεν ἕως ἀπέναντι Ἰεβοὺς, αὕτη ἐστὶν Ἱερουσαλὴμ, καὶ μετʼ αὐτοῦ ζεῦγος ὄνων ἐπισεσαγμένων, καὶ ἡ παλλακὴ αὐτοῦ μετʼ αὐτοῦ.

\vs{11}Καὶ ἦλθοσαν ἕως Ἰεβούς· καὶ ἡ ἡμέρα προβεβήκει σφόδρα, καὶ εἴπεν ὁ νεανίας πρὸς τὸν κύριον αὐτοῦ, δεῦρο δὴ καὶ ἐκκλίνωμεν εἰς πόλιν τοῦ Ἰεβουσὶ ταύτην, καὶ αὐλισθῶμεν ἐν αὐτῇ.
\vs{12}Καὶ εἶπε πρὸς αὐτὸν ὁ κύριος αὐτοῦ, οὐκ ἐκκλινοῦμεν εἰς πόλιν ἀλλοτρίαν, ἐν ᾗ οὐκ ἔστιν ἀπὸ υἱῶν Ἰσραὴλ ὧδε, καὶ παρελευσόμεθα ἕως Γαβαά.
\vs{13}Καὶ εἶπε τῷ νεανίᾳ αὐτοῦ, δεῦρο καὶ ἐγγίσωμεν ἑνὶ τῶν τόπων, καὶ αὐλισθησόμεθα ἐν Γαβαᾷ ἢ ἐν Ῥαμᾷ.
\vs{14}Καὶ παρῆλθον καὶ ἐπορεύθησαν, καὶ ἔδυ αὐτοῖς ὁ ἥλιος ἐχόμενα τῆς Γαβαὰ, ἥ ἐστιν ἐν τῷ Βενιαμίν.
\vs{15}Καὶ ἐξέκλιναν ἐκεῖ τοῦ εἰσελθεὶν αὐλισθῆναι ἐν Γαβαᾷ· καὶ εἰσῆλθον, καὶ ἐκάθισαν ἐν τῇ πλατείᾳ τῆς πόλεως, καὶ οὐκ ἦν ἀνὴρ συνάγων αὐτοὺς εἰς οἰκίαν αὐλισθῆναι.

\vs{16}Καὶ ἰδοὺ ἀνὴρ πρεσβύτης ἤρχετο ἐξ ἔργων αὐτοῦ ἐξ ἀγροῦ ἐν ἑσπέρᾳ, καὶ ὁ ἀνὴρ ἦν ἐξ ὄρους Ἐφραὶμ, καὶ αὐτὸς παρῴκει ἐν Γαβαᾷ, καὶ οἱ ἄνδρες τοῦ τόπου υἱοὶ Βενιαμίν.
\vs{17}Καὶ ᾖρε τοὺς ὀφθαλμοὺς αὐτοῦ, καὶ εἶδε τὸν ὁδοιπόρον ἄνδρα ἐν τῇ πλατείᾳ τῆς πόλεως· καὶ εἶπεν ὁ ἀνὴρ ὁ πρεσβύτης, ποῦ πορεύῃ, καὶ πόθεν ἔρχῃ;
\vs{18}Καὶ εἶπε πρὸς αὐτὸν, παραπορευόμεθα ἡμεῖς ἀπὸ Βηθλεὲμ Ἰούδα ἕως μηρῶν ὄρους Ἐφραίμ· ἐκεῖθεν ἐγώ εἰμι, καὶ ἐπορεύθην ἕως Βηθλεὲμ Ἰούδα, καὶ εἰς τὸν οἶκόν μου ἐγὼ πορεύομαι, καὶ οὐκ ἔστιν ἀνὴρ συνάγων με εἰς τὴν οἰκίαν.
\vs{19}Καί γε ἄχυρα καὶ χορτάσματά ἐστι τοῖς ὄνοις ἡμῶν, καὶ ἄρτος καὶ οἶνός ἐστιν ἐμοὶ καὶ τῇ παιδίσκῃ καὶ τῷ νεανίσκῳ μετὰ τῶν παίδων σου· οὐκ ἔστιν ὑστέρημα παντὸς πράγματος.
\vs{20}Καὶ εἶπεν ὁ ἀνὴρ πρεσβύτης, εἰρήνη σοι· πλὴν πᾶν τὸ ὑστέρημά σου ἐπʼ ἐμὲ, πλὴν ἐν τῇ πλατείᾳ οὐ μὴ αὐλισθήσῃ.
\vs{21}Καὶ εἰσήνεγκεν αὐτὸν εἰς τὸν οἶκον αὐτοῦ, καὶ τόπον ἐποίησε τοῖς ὄνοις, καὶ αὐτοὶ ἐνίψαντο τοὺς πόδας αὐτῶν, καὶ ἔφαγον καὶ ἔπιον.

\vs{22}Αὐτοὶ δὲ ἀγαθύνοντες καρδίαν αὐτῶν· καὶ ἰδοὺ ἄνδρες τῆς πόλεως υἱοὶ παρανόμων ἐκύκλωσαν τὴν οἰκίαν, κρούοντες ἐπὶ τὴν θύραν· καὶ εἶπον πρὸς τὸν ἄνδρα τὸν κύριον τοῦ οἴκου τὸν πρεσβύτην, λέγοντες, ἐξένεγκε τὸν ἄνδρα ὅς εἰσῆλθεν εἰς τὴν οἰκίαν σου, ἵνα γνῶμεν αὐτόν.
\vs{23}Καὶ ἐξῆλθε πρὸς αὐτοὺς ὁ ἀνὴρ ὁ κύριος τοῦ οἴκου, καὶ εἶπε, μὴ ἀδελφοί, μὴ κακοποιήσητε δὴ μετὰ τὸ εἰσελθεῖν τὸν ἄνδρα τοῦτον εἰς τὴν οἰκίαν μου, μὴ ποιήσητε τὴν ἀφροσύνην ταύτην.
\vs{24}Ἴδε ἡ θυγάτηρ μου ἡ παρθένος, καὶ ἡ παλλακὴ αὐτοῦ· ἐξάξω αὐτὰς, καὶ ταπεινώσατε αὐτὰς, καὶ ποιήσατε αὐταῖς τὸ ἀγαθὸν ἐν ὀφθαλμοῖς ὑμῶν, καὶ τῷ ἀνδρὶ τούτῳ μή ποιήσητε τὸ ῥῆμα τῆς ἀφροσύνης ταύτης.
\vs{25}Καὶ οὐκ εὐδόκησαν οἱ ἄνδρες τοῦ εἰσακοῦσαι αὐτοῦ· καὶ ἐπελάβετο ὁ ἀνηρ τῆς παλλακῆς αὐτοῦ, καὶ ἐξήγαγεν αὐτὴν πρὸς αὐτοὺς ἔξω· καὶ ἔγνωσαν αὐτὴν, καὶ ἐνέπαιζον ἐν αὐτῇ ὅλην τὴν νύκτα ἕως τοπρωῒ, καὶ ἐξαπέστειλαν αὐτὴν ὡς ἀνέβη τοπρωΐ.

\vs{26}Καὶ ἦλθεν ἡ γυνὴ πρὸς τὸν ὄρθρον, καὶ ἔπεσε παρὰ τὴν θύραν τοῦ οἴκου οὗ ἦν αὐτῆς ἐκεῖ ὁ ἀνὴρ, ἕως οὗ διέφαυσε.
\vs{27}Καὶ ἀνέστη ὁ ἀνὴρ αὐτῆς τοπρωῒ, καὶ ἤνοιξε τὰς θύρας τοῦ οἴκου, καὶ ἐξῆλθε τοῦ πορευθῆναι τὴν ὁδὸν αὐτοῦ· καὶ ἰδοὺ ἡ γυνὴ αὐτοῦ ἡ παλλακὴ πεπτωκυῖα παρὰ τὰς θύρας τοῦ οἴκου, καὶ αἱ χεῖρες αὐτῆς ἐπὶ τὸ πρόθυρον.
\vs{28}Καὶ εἶπε πρὸς αὐτὴν, ἀνάστα καὶ ἀπέλθωμεν· καὶ οὐκ ἀπεκρίθη, ὅτι ἦν νεκρά· καὶ ἔλαβεν αὐτὴν ἐπὶ τὸν ὄνον, καὶ ἐπορεύθη εἰς τὸν τόπον αὐτοῦ.

\vs{29}Καὶ ἔλαβε τὴν ῥομφαίαν, καὶ ἐκράτησε τὴν παλλακὴν αὐτοῦ· καὶ ἐμέλισεν αὐτὴν εἰς δώδεκα μέλη, καὶ ἀπέστειλεν αὐτὰ ἐν παντὶ ὁρίῳ Ἰσραήλ.
\vs{30}Καὶ ἐγένετο πᾶς ὁ βλέπων ἔλεγεν, οὐκ ἐγένετο καὶ οὐχ ἑὼραται ἀπὸ ἡμέρας ἀναβάσεως υἱῶν Ἰσραὴλ ἐκ γῆς Αἰγύπτου ἕως τῆς ἡμέρας ταύτης ὠς αὐτή· θέσθε ὑμῖν αὐτοῖς βουλὴν ἐπʼ αὐτὴν, καὶ λαλήσατε.

\ch{20}
Καὶ ἐξῆλθον πάντες οἱ υἱοὶ Ἰσραὴλ, καὶ ἐξεκκλησιάσθη ἡ συναγωγὴ ὡς ἀνὴρ εἷς ἀπὸ Δὰν καὶ ἕως Βηρσαβεὲ, καὶ γῇ τοῦ Γαλαὰδ, πρὸς Κύριον εἰς Μασσηφά.
\vs{2}Καὶ ἐστάθησαν κατὰ πρόσωπον Κυρίου πᾶσαι αἱ φυλαὶ τοῦ Ἰσραὴλ ἐν ἐκκλησίᾳ τοῦ λαοῦ τοῦ Θεοῦ, τετρακόσιαι χιλιάδες ἀνδρῶν πεζῶν ἕλκοντες ῥομφαίαν.
\vs{3}Καὶ ἤκουσαν οἱ υἱοὶ Βενιαμεὶν, ὅτι ἀνέβησαν οἱ υἱοὶ Ἰσραὴλ εἰς Μασσηφά· Καὶ ἐλθόντες εἶπαν οἱ υἱοὶ Ἰσραὴλ, λαλήσατε, ποῦ ἐγένετο ἡ πονηρία αὕτη;
\vs{4}Καὶ ἀπεκριθη ὁ ἀνὴρ ὁ Λευίτης, ὁ ἀνὴρ τῆς γυναικὸς τῆς φονευθείσης, καὶ εἶπεν, εἰς Γαβαὰ τῆς Βενιαμὶν ἦλθον ἐγὼ καὶ ἡ παλλακή μου τοῦ αὐλισθῆναι,
\vs{5}καὶ ἀνέστησαν ἐπʼ ἐμὲ οἱ ἄνδρες τῆς Γαβαὰ, καὶ ἐκύκλωσαν ἐπʼ ἐμὲ ἐπὶ τὴν οἰκίαν νυκτός· ἐμὲ ἠθέλησαν φονεῦσαι, καὶ τὴν παλλακήν μου ἐταπείνωσαν, καὶ ἀπέθανε.
\vs{6}Καὶ ἐκράτησα τὴν παλλακήν μου, καὶ ἐμέλισα αὐτὴν, καὶ ἐξαπέστειλα ἐν παντὶ ὁρίῳ κληρονομίας υἱῶν Ἰσραήλ· ὅτι ἐποίησαν ζέμα καὶ ἀπόπτωμα ἐν Ἰσραήλ.
\vs{7}Ἰδοὺ πάντες ὑμεῖς υἱοὶ Ἰσραὴλ, δότε ἑαυτοῖς λόγον καὶ βουλὴν ἐκεῖ.

\vs{8}Καὶ ἀνέστη πᾶς ὁ λαὸς ὡς ἀνὴρ εἷς, λέγοντες, οὐκ ἀπελευσόμεθα ἀνὴρ εἰς σκήνωμα αὐτοῦ, καὶ οὐκ ἐπιστρέψομεν ἀνὴρ εἰς τὸν οἶκον αὐτοῦ.
\vs{9}Καὶ νῦν τοῦτο τὸ ῥῆμα, ὃ ποιηθήσεται τῇ Γαβαᾷ· ἀναβησόμεθα ἐπʼ αὐτὴν ἐν κλήρῳ.
\vs{10}Πλὴν ληψόμεθα δέκα ἄνδρας τοῖς ἑκατὸν εἰς πάσας φυλὰς Ἰσραὴλ, καὶ ἑκατὸν τοῖς χιλίοις, καὶ χιλίους τοῖς μυρίοις, λαβεῖν ἐπισιτισμὸν τοῦ ποιῆσαι ἐλθεῖν αὐτοὺς εἰς Γαβαὰ Βενιαμὶν, ποιῆσαι αὐτῇ κατὰ πᾶν τὸ ἀπόπτωμα, ὃ ἐποίησεν ἐν Ἰσραήλ.
\vs{11}Καὶ συνήχθη πᾶς ἀνὴρ Ἰσραὴλ εἰς τὴν πόλιν ὡς ἀνὴρ εἷς.

\vs{12}Καὶ ἀπέστειλαν αἱ φυλαὶ Ἰσραὴλ ἄνδρας ἐν πάσῃ φυλῇ Βενιαμεὶν, λέγοντες, τίς ἡ πονηρία αὕτη ἡ γενομένη ἐν ὑμῖν;
\vs{13}Καὶ νῦν δότε τοὺς ἄνδρας υἱοὺς παρανόμων τοὺς ἐν Γαβαὰ, καὶ θανατώσομεν αὐτοὺς, καὶ ἐκκαθαριοῦμεν πονηρίαν ἀπὸ Ἰσραήλ· καὶ οὐκ εὐδόκησαν οἱ υἱοὶ Βενιαμὶν ἀκοῦσαι τῆς φωνῆς τῶν ἀδελφῶν αὐτῶν υἱῶν Ἰσραήλ.
\vs{14}Καὶ συνήχθησαν οἱ υἱοὶ Βενιαμὶν ἀπὸ τῶν πόλεων αὐτῶν εἰς Γαβαὰ ἐξελθεῖν εἰς παράταξιν πρὸς υἱοὺς Ἰσραήλ.
\vs{15}Καὶ ἐπεσκέπησαν οἱ υἱοὶ Βενιαμὶν ἐν τῇ ἡμέρᾳ ἐκείνῃ ἀπὸ τῶν πόλεων εἰκοσιτρεῖς χιλιάδες ἀνὴρ ἕλκων ῥομφαίαν, ἐκτὸς τῶν οἰκούντων τὴν Γαβαά, οἱ ἐπεσκέπησαν ἑπτακόσιοι ἄνδρες
\vs{16}ἐκλεκτοὶ ἐκ παντὸς λαοῦ ἀμφοτεροδέξιοι· πάντες οὗτοι σφενδονῆται ἐν λίθοις πρὸς τρίχα, καὶ οὐκ ἐξαμαρτάνοντες.
\vs{17}Καὶ ἀνὴρ Ἰσραὴλ ἐπεσκέπησαν ἐκτὸς τοῦ Βενιαμὶν τετρακόσιαι χιλιάδες ἀνδρῶν ἑλκόντων ῥομφαίαν· πάντες οὗτοι ἄνδρες παρατάξεως.

\vs{18}Καὶ ἀνέστησαν καὶ ἀνέβησαν εἰς Βαιθὴλ, καὶ ἠρώτησαν ἐν τῷ Θεῷ· καὶ εἶπαν οἱ υἱοὶ Ἰσραὴλ, τίς ἀναβήσεται ἡμῖν ἐν ἀρχῇ εἰς παράταξιν πρὸς υἱοὺς Βενιαμίν; καὶ εἶπε Κύριος, Ἰούδας ἐν ἀρχῇ ἀναβήσεται ἀφηγούμενος.
\vs{19}Καὶ ἀνέστησαν οἱ υἱοὶ Ἰσραὴλ τοπρωῒ, καὶ παρενέβαλον ἐπὶ Γαβαά.

\vs{20}Καὶ ἐξῆλθον πᾶς ἀνὴρ Ἰσραὴλ εἰς παράταξιν πρὸς Βενιαμὶν, καὶ συνῆψαν αὐτοῖς ἐπὶ Γαβαά.
\vs{21}Καὶ ἐξῆλθον οἱ υἱοὶ Βενιαμὶν ἀπὸ τῆς Γαβαὰ, καὶ διέφθειραν ἐν Ἰσραὴλ ἐν τῇ ἡμέρᾳ ἐκείνῃ δύο καὶ εἴκοσι χιλιάδας ἀνδρῶν ἐπὶ τὴν γῆν.

\vs{22}Καὶ ἐνίσχυσαν ἀνῆρ Ἰσραὴλ, καὶ προσέθηκαν συνάψαι παράταξιν ἐν τῷ τόπῳ ὅπου συνῆψαν ἐν τῇ ἡμέρᾳ τῇ πρώτῃ.
\vs{23}Καὶ ἀνέβησαν οἱ υἱοὶ Ἰσραὴλ, καὶ ἔκλαυσαν ἐνώπιον Κυρίου ἕως ἑσπέρας, καὶ ἠρώτησαν ἐν Κυρίῳ, λέγοντες, εἰ προσθῶμεν ἐγγίσαι εἰς παράταξιν πρὸς υἱοὺς Βενιαμὶν ἀδελφοὺς ἡμῶν; καὶ εἶπε Κύριος, ἀνάβητε πρὸς αὐτούς.
\vs{24}Καὶ προσῆλθον οἱ υἱοὶ Ἰσραὴλ πρὸς υἱοὺς Βενιαμὶν ἐν τῇ ἡμέρᾳ τῇ δευτέρᾳ.
\vs{25}Καὶ ἐξῆλθον οἱ υἱοὶ Βενιαμὶν εἰς συνάντησιν αὐτοὶς ἀπὸ τῆς Γαβαὰ ἐν τῇ ἡμέρᾳ τῇ δευτέρᾳ, καὶ διέφθειραν ἀπὸ υἱῶν Ἰσραὴλ ἔτι ὀκτωκαίδεκα χιλιάδας ἀνδρῶν ἐπὶ τὴν γῆν· πάντες οὗτοι ἕλκοντες ῥομφαίαν.

\vs{26}Καὶ ἀνέβησαν πάντες οἱ υἱοὶ Ἰσραὴλ καὶ πᾶς ὁ λαὸς, καὶ ἦλθον εἰς Βαιθήλ· καὶ ἔκλαυσαν, καὶ ἐκάθισαν ἐκεῖ ἐνώπιον Κυρίου· καὶ ἐνήστευσαν ἐν τῇ ἡμέρᾳ ἐκείνῃ ἕως ἑσπέρας, καὶ ἀνήνεγκαν ὁλοκαυτώσεις καὶ τελείας ἐνώπιον Κυρίου,
\vs{27}ὅτι ἐκεῖ κιβωτὸς διαθήκης Κυρίου τοῦ Θεοῦ ἐν ταῖς ἡμέραις ἐκείναις,
\vs{28}καὶ Φινεὲς υἱὸς Ἐλεάζαρ υἱοῦ Ἀαρὼν παρεστηκὼς ἐνώπιον αὐτῆς ἐν ταῖς ἡμέραις ἐκείναις· καὶ ἐπηρώτησαν οἱ υἱοὶ Ἰσραὴλ ἐν Κυρίῳ, λέγοντες, εἰ προσθῶμεν ἔτι ἐξελθεῖν εἰς παράταξιν πρὸς υἱοὺς Βενιαμὶν ἀδελφοὺς ἡμῶν; καὶ εἶπε Κύριος, ἀνάβητε, αὔριον δώσω αὐτοὺς εἰς χεῖρας ὑμῶν.
\vs{29}Καὶ ἔθηκαν οἱ υἱοῖ Ἰσραὴλ ἔνεδρα τῇ Γαβαὰ κύκλῳ.

\vs{30}Καὶ ἀνέβησαν οἱ υἱοὶ Ἰσραὴλ πρὸς υἱοὺς Βενιαμὶν ἐν τῇ ἡμέρᾳ τῇ τρίτῃ, καὶ συνῆψαν πρὸς τὴν Γαβαὰ ὡς ἅπαξ καὶ ἅπαξ.
\vs{31}Καὶ ἐξῆλθον οἱ υἱοὶ Βενιαμὶν εἰς συνάντησιν τοῦ λαοῦ, καὶ ἐξεκενώθησαν ἐκ τῆς πόλεως, καὶ ἤρξαντο πατάσσειν ἀπὸ τοῦ λαοῦ τραυματίας ὡς ἅπαξ καὶ ἅπαξ ἐν ταῖς ὁδοῖς, ἥ ἐστι μία ἀναβαίνουσα εἰς Βαιθὴλ, καὶ μία εἰς Γαβαὰ ἐν ἀγρῷ, ὡς τριάκοντα ἄνδρας ἐν Ἰσραήλ.
\vs{32}Καὶ εἶπαν οἱ υἱοὶ Βενιαμὶν, πίπτουσιν ἐνώπιον ἡμῶν ὡς τὸ πρῶτον. καὶ οἱ υἱοὶ Ἰσραὴλ εἶπαν, Φύγωμεν, καὶ ἐκκενώσωμεν αὐτοὺς ἀπὸ τῆς πόλεως εἰς τὰς ὁδούς· καὶ ἐποίησαν οὕτω.

\vs{33}Καὶ πᾶς ἀνὴρ ἀνέστη ἐκ τοῦ τόπου αὐτῶν, καὶ συνῆψαν ἐν Βάαλ Θαμάρ· καὶ τὸ ἔνεδρον Ἰσραὴλ ἐπήρχετο ἐκ τοῦ τόπου αὐτοῦ ἀπὸ Μαρααγαβέ.
\vs{34}Καὶ ἦλθον ἐξεναντίας Γαβαὰ δέκα χιλιάδες ἀνδρῶν ἐκλεκτῶν ἐκ παντὸς Ἰσραήλ· καὶ παράταξις βαρεῖα· καὶ αὐτοὶ οὐκ ἔγνωσαν, ὅτι φθάνει ἀπʼ αὐτοὺς ἡ κακία.
\vs{35}Καὶ ἐπάταξε Κύριος τὸν Βενιαμὶν ἐνώπιον υἱῶν Ἰσραήλ· καὶ διέφθειραν οἱ υἱοὶ Ἰσραὴλ ἐκ τοῦ Βενιαμὶν ἐν τῇ ἡμέρᾳ ἐκείνῃ εἴκοσι καὶ πέντε χιλιάδας καὶ ἑκατὸν ἄνδρας· πάντες οὗτοι εἷλκον ῥομφαίαν.
\vs{36}Καὶ εἶδον οἱ υἱοὶ Βενιαμὶν ὅτι ἐπλήγησαν· καὶ ἔδωκεν ἀνὴρ Ἰσραὴλ τῷ Βενιαμὶν τόπον, ὅτι ἤλπισαν πρὸς τὸ ἔνεδρον ὃ ἔθηκαν ἐπὶ τῇ Γαβαᾷ.

\vs{37}Καὶ ἐν τῷ αὐτοὺς ὑποχωρῆσαι, καὶ τὸ ἔνεδρον ἐκινήθη· καὶ ἐξέτειναν ἐπὶ τὴν Γαβαὰ, καὶ ἐξεχύθη τὸ ἔνεδρον, καὶ ἐπάταξαν τὴν πόλιν ἐν στόματι ῥομφαίας.

\vs{38}Καὶ σημεῖον ἦν τοῖς υἱοῖς Ἰσραὴλ μετὰ τοῦ ἐνέδρου τῆς μάχης ἀνενέγκαι αὐτοὺς σύσσημον καπνοῦ ἀπὸ τῆς πόλεως.
\vs{39}Καὶ εἶδον οἱ υἱοὶ Ἰσραὴλ, ὅτι προκατελάβετο τὸ ἔνεδρον τὴν Γαβαὰ, καὶ ἔστησαν ἐν τῇ παρατάξει· καὶ Βενιαμὶν ἤρξατο πατάσσειν τραυματίας ἐν ἀνδράσιν Ἰσραὴλ ὡς τριάκοντα ἄνδρας· ὅτι εἶπαν, πάλιν πτώσει πίπτουσιν ἐνώπιον ἡμῶν ὡς ἡ παράταξις ἡ πρώτη.

\vs{40}Καὶ τὸ σύσσημον ἀνέβη ἐπὶ πλεῖον ἐπὶ τῆς πόλεως ὡς στύλος καπνοῦ· καὶ ἐπέβλεψε Βενιαμὶν ὀπίσω αὐτοῦ, καὶ ἰδοὺ ἀνέβη ἡ συντέλεια τῆς πόλεως ἕως οὐρανοῦ.

\vs{41}Καὶ ἀνὴρ Ἰσραὴλ ἐπέστρεψε· καὶ ἔσπευσαν ἄνδρες Βενιαμὶν, ὅτι εἶδον ὅτι συνήντησεν ἐπʼ αὐτοὺς ἡ πονηρία.
\vs{42}Καὶ ἐπέβλεψαν ἐνώπιον υἱῶν Ἰσραὴλ εἰς τὴς ὁδὸν τῆς ἐρήμου, καὶ ἔφυγον· καὶ ἡ παράταξις ἔφθασεν ἐπʼ αὐτοὺς, καὶ οἱ ἀπὸ τῶν πόλεων διέφθειρον αὐτοὺς ἐν μέσῳ αὐτῶν.

\vs{43}Καὶ κατέκοπτον τὸν Βενιαμὶν, καὶ ἐδίωξαν αὐτὸν ἀπὸ Νουὰ κατὰ πόδα αὐτοῦ ἕως ἀπέναντι Γαβαὰ πρὸς ἀνατολὰς ἡλίου.
\vs{44}Καὶ ἔπεσον ἀπὸ Βενιαμὶν ὀκτωκαίδεκα χιλιάδες ἀνδρῶν· οἱ πάντες οὗτοι ἄνδρες δυνάμεως.

\vs{45}Καὶ ἐπέβλεψαν οἱ λοιποὶ, καὶ ἔφευγον εἰς τὴν ἔρημον πρὸς τὴν πέτραν τοῦ Ῥέμμών· καὶ ἐκαλαμήσαντο ἐξ αὐτῶν οἱ υἱοὶ Ἰσραὴλ πεντακισχιλίους ἄνδρας· καὶ κατέβησαν ὀπίσω αὐτῶν οἱ υἱοὶ Ἰσραὴλ ἕως Γεδᾶν, καὶ ἐπάταξαν ἐξ αὐτῶν δισχιλίους ἄνδρας.
\vs{46}Καὶ ἐγένοντο πάντες οἱ πεπτωκότες ἀπὸ Βενιαμὶν, εἰκοσιπέντε χιλιάδες ἀνδρῶν ἑλκόντων ῥομφαίαν ἐν τῇ ἡμέρᾳ ἐκείνῃ· οἱ πάντες οὗτοι ἄνδρες δυνάμεως.
\vs{47}Καὶ ἐπέβλεψαν οἱ λοιποὶ, καὶ ἔφυγον εἰς τὴν ἔρημον πρὸς τὴν πέτραν τοῦ Ῥεμμὼν ἑξακόσιοι ἄνδρες, καὶ ἐκάθισαν ἐν πέτρᾳ Ῥεμμὼν τέσσαρας μῆνας.

\vs{48}Καὶ οἱ υἱοὶ Ἰσραὴλ ἐπέστρεψαν πρὸς υἱοὺς Βενιαμὶν, καὶ ἐπάταξαν αὐτοὺς ἐν στόματι ῥομφαίας ἀπὸ πόλεως Μεθλὰ καὶ ἕως κτήνους, καὶ ἕως παντὸς τοῦ εὑρισκομένου εἰς πάσας τὰς πόλεις· καὶ τὰς πόλεις τὰς εὑρεθείσας ἐνέπρησαν ἐν πυρί.

\ch{21}
Καὶ οἱ υἱοὶ Ἰσραὴλ ὤμοσαν ἐν Μασσηφὰθ, λέγοντες, ἀνὴρ ἐξ ἡμῶν οὐ δώσει θυγατέρα αὐτοῦ τῷ Βενιαμὶν εἰς γυναῖκα.
\vs{2}Καὶ ἦλθεν ὁ λαὸς εἰς Βαιθὴλ, καὶ ἐκάθισαν ἐκεῖ ἕως ἑσπέρας ἐνώπιον τοῦ Θεοῦ· καὶ ᾖραν φωνὴν αὐτῶν, καὶ ἔκλαυσαν κλαυθμὸν μέγαν,
\vs{3}καὶ εἶπαν, εἰς τί Κύριε Θεὲ Ισραὴλ ἐγενήθη αὕτη, τοῦ ἐπισκεπῆναι σήμερον ἀπὸ Ἰσραὴλ φυλὴν μίαν;
\vs{4}Καὶ ἐγένετο τῇ ἐπαύριον, καὶ ὤρθρισεν ὁ λαὸς, καὶ ᾠκοδόμησαν ἐκεῖ θυσιαστήριον, καὶ ἀνήνεγκαν ὁλοκαυτώσεις καὶ τελείας.

\vs{5}Καὶ εἶπαν οἱ υἱοὶ Ἰσραὴλ, τίς οὐκ ἀνέβη ἐν τῇ ἐκκλησίᾳ ἀπὸ πασῶν φυλῶν Ἰσραὴλ πρὸς Κύριον; ὅτι ὁ ὅρκος μέγας ἦν τοῖς οὐκ ἀναβεβηκόσι πρὸς Κύριον εἰς Μασσηφὰθ, λέγοντες, θανάτῳ θανατωθήσεται.

\vs{6}Καὶ παρεκλήθησαν οἱ υἱοὶ Ἰσραὴλ πρὸς Βενιαμὶν ἀδελφὸν αὐτῶν, καὶ εἶπαν, ἐξεκόπη σήμερον φυλὴ μία ἀπὸ Ἰσραήλ.
\vs{7}Τί ποιήσωμεν αὐτοῖς τοῖς περισσοῖς τοῖς ὑπολειφθεῖσιν εἰς γυναῖκας; καὶ ἡμεῖς ὠμόσαμεν ἐν Κυρίῳ τοῦ μὴ δοῦναι αὐτοῖς ἀπὸ τῶν θυγατέρων ἡμῶν εἰς γυναῖκας.
\vs{8}Καὶ εἶπαν, τίς εἷς ἀπὸ φυλῶν Ἰσραὴλ, ὃς οὐκ ἀνέβη πρὸς Κύριον εἰς Μασσηφάθ; Καὶ ἰδοὺ οὐκ ἦλθεν ἀνὴρ εἰς τὴν παρεμβολὴν ἀπὸ Ἰαβεῖς Γαλαὰδ εἰς τὴν ἐκκλησίαν.
\vs{9}Καὶ ἐπεσκέπη ὁ λαὸς, καὶ οὐκ ἦν ἐκεῖ ἀνὴρ ἀπὸ οἰκούντων Ἰαβὶς Γαλαάδ.

\vs{10}Καὶ ἀπέστειλεν ἐκεῖ ἡ συναγωγὴ δώδεκα χιλιάδας ἀνδρῶν ἀπὸ υἱῶν τῆς δυνάμεως, καὶ ἐνετείλαντο αὐτοῖς λέγοντες, πορεύεσθε καὶ πατάξατε τοὺς οἰκοῦντας Ἰαβεῖς Γαλαὰδ ἐν στόματι ῥομφαίας.
\vs{11}Καὶ τοῦτο ποιήσετε· πᾶν ἄρσεν καὶ πᾶσαν γυναῖκα εἰδυῖαν κοίτην ἄρσενος, ἀναθεματιεῖτε· τὰς δὲ παρθένους, περιποιήσεσθε· καὶ ἐποίησαν οὕτως.

\vs{12}Καὶ εὗρον ἀπὸ οἰκούντων Ἰαβεῖς Γαλαὰδ, τετρακοσίας νεάνιδας παρθένους, αἵτινες οὐκ ἔγνωσαν ἄνδρα εἰς κοίτην ἄρσενος, καὶ ἤνεγκαν αὐτὰς εἰς τὴν παρεμβολὴν εἰς Σηλὼμ τὴν ἐν γῇ Χαναάν.

\vs{13}Καὶ ἀπέστειλαν πᾶσα ἡ συναγωγὴ, καὶ ἐλάλησαν πρὸς τοὺς υἱοὺς Βενιαμεὶν ἐν τῇ πέτρᾳ Ῥεμμὼν, καὶ ἐκάλεσαν αὐτοὺς εἰς εἰρήνην.
\vs{14}Καὶ ἐπέστρεψε Βενιαμὶν πρὸς τοὺς υἱοὺς Ἰσραὴλ ἐν τῷ καιρῷ ἐκείνῳ, καὶ ἔδωκαν αὐτοῖς οἱ υἱοὶ Ἰσραὴλ τὰς γυναῖκας ἃς ἐζωοποίησαν ἀπὸ τῶν θυγατέρων Ἰαβὶς Γαλαάδ· καὶ ἤρεσεν αὐτοῖς οὕτω.

\vs{15}Καὶ ὁ λαὸς παρεκλήθη ἐπὶ τῷ Βενιαμὶν, ὅτι ἐποίησε Κύριος διακοπὴν ἐν ταῖς φυλαῖς Ἰσραήλ.

\vs{16}Καὶ εἶπον οἱ πρεσβύτεροι τῆς συναγωγῆς, τί ποιήσωμεν τοῖς περισσοῖς εἰς γυναῖκας; ὅτι ἠφανίσθη ἀπὸ Βενιαμὶν γυνή.
\vs{17}Καὶ εἶπαν, κληρονομία διασωζομένων τῶν Βενιαμίν· καὶ οὐκ ἐξαλειφθήσεται φυλὴ ἀπὸ Ἰσραὴλ,
\vs{18}ὅτι ἡμεῖς οὐ δυνησόμεθα δοῦναι αὐτοῖς γυναῖκας ἀπὸ τῶν θυγατέρων ἡμῶν, ὅτι ὠμόσαμεν ἐν υἱοῖς Ἰσραὴλ, λέγοντες, ἐπικατάρατος ὁ διδοὺς γυναῖκα τῷ Βενιαμίν.

\vs{19}Καὶ εἶπαν, ἰδοὺ δὴ ἑορτὴ Κυρίου ἐν Σηλὼμ ἀφʼ ἡμερῶν εἰς ἡμέρας, ἥ ἐστιν ἀπὸ Βοῤῥᾶ τῆς Βαιθὴλ, κατʼ ἀνατολὰς ἡλίου ἐπὶ τῆς ὁδοῦ τῆς ἀναβαινούσης ἀπὸ Βαιθὴλ εἰς Συχὲμ, καὶ ἀπὸ Νότου τῆς Λεβωνᾶ.
\vs{20}Καὶ ἐνετείλαντο τοῖς υἱοῖς Βενιαμεὶν, λέγοντες, πορεύεσθε καὶ ἐνεδρεύσατε ἐν τοῖς ἀμπελῶσι,
\vs{21}καὶ ὄψεσθε, καὶ ἰδοὺ, ἐὰν ἐξέλθωσιν αἱ θυγατέρες τῶν οἰκούντων Σηλὼ χορεύειν ἐν τοῖς χοροῖς, καὶ ἐξελεύσεσθε ἐκ τῶν ἀμπελώνων, καὶ ἁρπάσατε αὑτοῖς ἀνὴρ γυναῖκα ἀπὸ τῶν θυγατέρων Σηλὼμ, καὶ πορεύεσθε εἰς γῆν Βενιαμίν.
\vs{22}Καὶ ἔσται ὅταν ἔλθωσιν οἱ πατέρες αὐτῶν ἢ οἱ ἀδελφοὶ αὐτῶν κρίνεσθαι πρὸς ἡμᾶς, καὶ ἐροῦμεν αὐτοῖς, ἔλεος ποιήσατε ἡμῖν αὐτὰς, ὅτι οὐκ ἐλάβομεν ἀνὴρ γυναῖκα αὐτοῦ ἐν τῇ παρατάξει, ὅτι οὐχ ὑμεῖς ἐδώκατε αὐτοῖς, ὡς κλῆρος πλημμελήσατε.

\vs{23}Καὶ ἐποίησαν οὕτως οἱ υἱοὶ Βενιαμίν· καὶ ἔλαβον γυναῖκας εἰς ἀριθμὸν αὐτῶν ἀπὸ τῶν χορευουσῶν ὧν ἥρπασαν· καὶ ἐπορεύθησαν, καὶ ὑπέστρεψαν εἰς τὴν κληρονομίαν αὐτῶν· καὶ ᾠκοδόμησαν τὰς πόλεις, καὶ ἐκάθισαν ἐν αὐταῖς.
\vs{24}Καὶ περιεπάτησαν ἐκεῖθεν οἱ υἱοὶ Ἰσραὴλ ἐν τῷ καιρῷ ἐκείνῳ ἀνὴρ εἰς φυλὴν αὐτοῦ καὶ εἰς συγγένειαν αὐτοῦ· καὶ ἐξῆλθον ἐκεῖθεν ἀνὴρ εἰς τὴν κληρονομίαν αὐτοῦ.
\vs{25}Ἐν δὲ ταῖς ἡμέραις ἐκείναις οὐκ ἦν βασιλεὺς ἐν Ἰσραήλ· ἀνὴρ τὸ εὐθὲς ἐνώπιον αὐτοῦ ἐποίει.


\def\book{ΡΟΥΘ}
\biblebook{ΡΟΥΘ}


\lettrine[lines=2, loversize=0.2, nindent=0em, findent=.25em]{\textcolor{bookheadingcolor}{Κ}}{ΑΙ} ἐγένετο ἐν τῷ κρίνειν τοὺς κριτὰς, καὶ ἐγένετο λιμὸς ἐν τῇ γῇ· καὶ ἐπορεύθη ἀνὴρ ἀπὸ Βηθλεὲμ Ἰούδα τοῦ παροικῆσαι ἐν ἀγρῷ Μωὰβ, αὐτὸς καὶ ἡ γυνὴ αὐτοῦ, καὶ οἱ δύο υἱοὶ αὐτοῦ.
\vs{2}Καὶ ὄνομα τῷ ἀνδρὶ Ἐλιμέλεχ, καὶ ὄνομα τῇ γυναικὶ αὐτοῦ Νωεμὶν, καὶ ὄνομα τοῖς δυσὶν υἱοῖς αὐτοῦ Μααλὼν, καὶ Χελαιὼν, Ἐφραθαῖοι ἐκ Βηθλεὲμ τῆς Ἰούδα· καὶ ἤλθοσαν εἰς ἀγρὸν Μωὰβ, καὶ ἦσαν ἐκεῖ.

\vs{3}Καὶ ἀπέθανεν Ἐλιμέλεχ ὁ ἀνὴρ τῆς Νωεμὶν, καὶ κατελείφθη αὕτη καὶ οἱ δύο υἱοὶ αὐτῆς.
\vs{4}Καὶ ἐλάβοσαν ἑαυτοῖς γυναῖκας Μωαβίτιδας· ὄνομα τῇ μιᾷ, Ὀρφά· καὶ ὄνομα τῇ δευτέρᾳ, Ῥούθ· καὶ κατῴκησαν ἐκεῖ ὡς δέκα ἔτη.
\vs{5}Καὶ ἀπέθανον καί γε ἀμφότεροι Μααλὼν καὶ Χελαιών· καὶ κατελείφθη ἡ γυνὴ ἀπὸ τοῦ ἀνδρὸς αὐτῆς, καὶ ἀπὸ τῶν δύο υἱῶν αὐτῆς.

\vs{6}Καὶ ἀνεστη αὕτη καὶ αἱ δύο νύμφαι αὐτῆς, καὶ ἀπέστρεψαν ἐξ ἀγροῦ Μωὰβ, ὅτι ἤκουσεν ἐν ἀγρῷ Μωὰβ ὅτι ἐπέσκεπται Κύριος τὸν λαὸν αὐτοῦ, δοῦναι αὐτοῖς ἄρτους.
\vs{7}Καὶ ἐξῆλθεν ἐκ τοῦ τόπου οὗ ἦν ἐκεῖ, καὶ αἱ δύο νύμφαι αὐτῆς μετʼ αὐτῆς· καὶ ἐπορεύοντο ἐν τῇ ὁδῷ τοῦ ἐπιστρέψαι εἰς τὴν γῆν Ἰούδα.

\vs{8}Καὶ εἶπε Νωεμὶν, ταῖς δυσὶ νύμφαις αὐτῆς, πορεύεσθε δὴ, ἀποστράφητε ἑκάστη εἰς οἶκον μητρὸς αὐτῆς· ποιήσαι Κύριος μεθʼ ὑμῶν ἔλεος, καθὼς ἐποιήσατε μετὰ τῶν τεθνηκότων καὶ μετʼ ἐμοῦ·
\vs{9}Δῴη Κύριος ὑμῖν καὶ εὕρητε ἀνάπαυσιν ἑκάστη ἐν οἴκῳ ἀνδρὸς αὐτῆς· καὶ κατεφίλησεν αὐτάς· καὶ ἐπῇραν τὴν φωνὴν αὐτῶν, καὶ ἔλαυσαν.
\vs{10}Καὶ εἶπαν αὐτῇ, μετὰ σου ἐπιστρέφομεν εἰς τὸν λαόν σου.

\vs{11}Καὶ εἶπε Νωεμὶν, ἐπιστράφητε δὴ θυγατέρες μου· καὶ ἱνατί πορεύεσθε μετʼ ἐμοῦ; μὴ ἔτι μοι υἱοὶ ἐν τῇ κοιλίᾳ μου, καὶ ἔσονται ὑμῖν εἰς ἄνδρας;
\vs{12}Ἐπιστράφητε δὴ θυγατέρες μου, διότι γεγήρακα τοῦ μὴ εἶναι ἀνδρί· ὅτι εἶπα, ὅτι ἐστί μοι ὑπόστασις τοῦ γενηθῆναί με ἀνδρὶ, καὶ τέξομαι υἱούς·
\vs{13}Μὴ αὐτοὺς προσδέξεσθε ἕως οὗ ἁδρυνθώσιν; ἢ αὐτοῖς κατασχεθήσεσθε τοῦ μὴ γενέσθαι ἀνδρί; μὴ δὴ θυγατέρες μου, ὅτι ἐπικράνθη μοι ὑπὲρ ὑμᾶς, ὅτι ἐξῆλθεν ἐν ἐμοὶ χεὶρ Κυρίου.

\vs{14}Καὶ ἐπῇραν τὴν φωνὴν αὐτῶν, καὶ ἔκλαυσαν ἔτι· καὶ κατεφίλησεν Ὀρφὰ τὴν πενθερὰν αὐτῆς, καὶ ἐπέστρεψεν εἰς τὸν λαὸν αὐτῆς· Ῥοὺθ δὲ ἠκολούθησεν αὐτῇ.

\vs{15}Καὶ εἶπε Νωεμῖν πρὸς Ῥοὺθ, ἰδοὺ ἀνέστρεψε σύννυμφός σου πρὸς λαὸν αὐτῆς καὶ πρὸς τοὺς θεοὺς αὐτῆς· ἐπιστράφηθι δὴ καὶ σὺ ὀπίσω τῆς συννύμφου σου.
\vs{16}Εἶπε δὴ Ῥοῦθ, μὴ ἀπάντησαί μοι τοῦ καταλιπεῖν σε, ἢ ἀποστρέψαι ὄπισθέν σου, ὅτι σὺ ὅπου ἐὰν πορευθῇς, πορεύσομαι, καὶ οὗ ἐὰν αὐλισθῇς, αὐλισθήσομαι· ὁ λαός σου, λαός μου, καὶ ὁ Θεός σου, Θεός μου·
\vs{17}Καὶ οὗ ἐὰν ἀποθάνῃς, ἀποθανοῦμαι, κᾀκεῖ ταφήσομαι· τάδε ποιήσαι μοι Κύριος, καὶ τάδε προσθείη, ὅτι θάνατος διαστελεῖ ἀναμέσον ἐμοῦ καὶ σοῦ.
\vs{18}Ἰδοῦσα δὲ Νωεμὶν ὅτι κραταιοῦται αὐτὴ τοῦ πορεύεσθαι μετʼ αὐτῆς, ἐκόπασε τοῦ λαλῆσαι πρὸς αὐτὴν ἔτι.

\vs{19}Ἐπορεύθησαν δὲ ἀμφότεραι, ἕως τοῦ παραγενέσθαι αὐτὰς εἰς Βηθλεέμ· καὶ ἐγένετο ἐν τῷ ἐλθεῖν αὐτὰς εἰς Βηθλεὲμ, καὶ ἤχησε πᾶσα ἡ πόλις ἐπʼ αὐταῖς, καὶ εἶπον, εἰ αὕτη ἐστὶ Νωεμίν;
\vs{20}Καὶ εἶπε πρὸς αὐτὰς, μὴ δὴ καλεῖτέ με Νωεμίν· καλέσατέ με πικρὰν, ὅτι ἐπικράνθη ἐν ἐμοὶ ὁ ἱκανὸς σφόδρα.
\vs{21}Ἐγὼ πλήρης ἐπορεύθην, καὶ κενὴν ἀπέστρεψέ με ὁ Κύριος· καὶ ἱνατί καλεῖτέ με Νωεμὶν, καὶ Κύριος ἐταπείνωσέ με, καὶ ὁ ἱκανὸς ἐκάκωσέ με;

\vs{22}Καὶ ἐπέστρεψε Νωεμὶν καὶ Ῥοὺθ ἡ Μωαβίτις ἡ νύμφη αὐτῆς ἐπιστρέφουσαι ἐξ ἀγροῦ Μωάβ· αὗται δὲ παρεγενήθησαν εἰς Βηθλεὲμ ἐν ἀρχῇ θερισμοῦ κριθῶν.

\ch{2}
Καὶ τῇ Νωεμὶν ἀνὴρ γνώριμος τῷ ἀνδρὶ αὐτῆς, ὁ δὲ ἀνὴρ δυνατὸς ἰσχύϊ ἐκ τῆς συγγενείας Ἐλιμέλεχ, καὶ ὄνομα αὐτῷ Βοόζ.
\vs{2}Καὶ εἶπε Ῥοὺθ ἡ Μωαβίτις πρὸς Νωεμὶν, πορεῦθω δὴ εἰς ἀγρὸν, καὶ συνάξω ἐν τοῖς στάχυσι κατόπισθεν οὗ ἐὰν εὕρω χάριν ἐν ὀφθαλμοῖς αὐτοῦ. εἶπε δὲ αὐτῇ, πορεύου, θύγατερ.
\vs{3}Καὶ ἐπορεύθη· καὶ ἐλθοῦσα συνέλεξεν ἐν τῷ ἀγρῷ κατόπισθε τῶν θεριζόντων· καὶ περιέπεσε περιπτώματι τῇ μερίδι τοῦ ἀγροῦ Βοὸζ, τοῦ ἐκ τῆς συγγενείας Ἐλιμέλεχ.

\vs{4}Καὶ ἰδοὺ Βοὸζ ἦλθεν ἐκ Βηθλεὲμ, καὶ εἶπε τοῖς θερίζουσι, Κύριος μεθʼ ὑμῶν· καὶ εἶπον αὐτῷ, εὐλογήσαι σε Κύριος.
\vs{5}Καὶ εἶπε Βοὸζ τῷ παιδαρίῳ αὐτοῦ τῷ ἐφεστῶτι ἐπὶ τοὺς θερίζοντας, τίνος ἡ νεᾶνις αὕτη;
\vs{6}Καὶ ἀπεκρίθη τὸ παιδάριον τὸ ἐφεστὸς ἐπὶ τοὺς θερίζοντας, καὶ εἶπεν, ἡ παῖς ἡ Μωαβίτις ἐστὶν ἡ ἀποστραφεῖσα μετὰ Νωεμὶν ἐξ ἀγροῦ Μωάβ·
\vs{7}Καὶ εἶπε, συλλέξω δὴ καὶ συνάξω ἐν τοῖς δράγμασιν ὄπισθεν τῶν θεριζόντων· καὶ ἦλθε καὶ ἔστη ἀπὸ πρωΐθεν καὶ ἕως ἑσπέρας, οὐ κατέπαυσεν ἐν τῷ ἀγρῷ μικρόν.

\vs{8}Καὶ εἶπε Βοὸζ πρὸς Ῥοὺθ, οὐκ ἤκουσας θύγατερ; μὴ πορευθῇς ἐν ἀγρῷ συλλέξαι ἑτέρῳ· καὶ σὺ οὐ πορεύσῃ ἐντεῦθεν, ὧδε κολλήθητι μετὰ τῶν κορασίων μου.
\vs{9}Οἱ ὀφθαλμοί σου εἰς τὸν ἀγρὸν οὗ ἐὰν θερίζωσι, καὶ πορεύσῃ κατόπισθεν αὐτῶν· ἰδοὺ ἐνετειλάμην τοῖς παιδαρίοις τοῦ μὴ ἅψασθαί σου· καὶ ὅτε διψήσεις καὶ πορευθήσῃ εἰς τὰ σκεύη, καὶ πίεσαι ὅθεν ἐὰν ὑδρεύωνται τὰ παιδάρια.
\vs{10}Καὶ ἔπεσεν ἐπὶ πρόσωπον αὐτῆς, καὶ προσεκύνησεν ἐπὶ τὴν γῆν, καὶ εἶπε πρὸς αὐτὸν, τί ὅτι εὗρον χάριν ἐν ὀφθαλμοῖς σου τοῦ ἐπιγνῶναί με, καὶ ἐγώ εἰμι ξένη;

\vs{11}Καὶ ἀπεκρίθη Βοὸζ, καὶ εἶπεν αὐτῇ, ἀπαγγελίᾳ ἀπηγγέλη μοι ὅσα πεποίηκας μετὰ τῆς πενθερᾶς σου μετὰ τὸ ἀποθανεῖν τὸν ἄνδρα σου· καὶ πῶς κατέλιπες τὸν πατέρα σου καὶ τὴν μητέρα σου, καὶ τὴν γῆν γενέσεώς σου, καὶ ἐπορεύθης πρὸς λαὸν ὃν οὐκ ᾔδεις ἐχθὲς καὶ τρίτης.
\vs{12}Ἀποτίσαι Κύριος τὴν ἐργασίαν σου· γένοιτο ὁ μισθός σου πλήρης παρὰ Κυρίου Θεοῦ Ἰσραὴλ, πρὸς ὃν ἦλθες πεποιθέναι ὑπὸ τὰς πτέρυγας αὐτοῦ.
\vs{13}Ἡ δὲ εἶπεν, εὕροιμι χάριν ἐν ὀφθαλμοῖς σου κύριε, ὅτι παρεκάλεσάς με, καὶ ὅτι ἐλάλησας ἐπὶ καρδίαν τῆς δούλης σου, καὶ ἰδοὺ ἐγὼ ἔσομαι ὡς μία τῶν παιδισκῶν σου.

\vs{14}Καὶ εἶπεν αὐτῇ Βοὸζ, ἤδη ὥρα τοῦ φαγεῖν, πρόσελθε ὧδε καὶ φάγεσαι τῶν ἄρτων, καὶ βάψεις τὸν ψωμόν σου ἐν τῷ ὄξει· καὶ ἐκάθισε Ῥοὺθ ἐκ πλαγίων τῶν θεριζόντων· καὶ ἐβούνισεν αὐτῇ Βοὸζ ἄλφιτον, καὶ ἔφαγε καὶ ἐνεπλήσθη καὶ κατέλιπε,

\vs{15}Καὶ ἀνέστη τοῦ συλλέγειν· καὶ ἐνετείλατο Βοὸζ τοῖς παιδαρίοις αὐτοῦ, λέγων, καί γε ἀναμέσον τῶν δραγμάτων συλλεγέτω, καὶ μὴ καταισχύνητε αὐτήν·
\vs{16}Καὶ βαστάζοντες βαστάσατε αὐτῇ, καί γε παραβάλλοντες παραβαλεῖτε αὐτῇ ἐκ τῶν βεβουνισμένων, καὶ φάγεται, καὶ συλλέξει, καὶ οὐκ ἐπιτιμήσετε αὐτῇ.
\vs{17}Καὶ συνέλεξεν ἐν τῷ ἀγρῷ ἕως ἑσπέρας, καὶ ἐῤῥάβδισεν ἃ συνέλεξε, καὶ ἐγενήθη ὡς οἰφὶ κριθῶν.

\vs{18}Καὶ ᾖρε καὶ εἰσῆλθεν εἰς τὴν πόλιν· καὶ εἶδεν ἡ πενθερὰ αὐτῆς ἃ συνέλεξε· καὶ ἐξενέγκασα Ῥοὺθ ἔδωκεν αὐτῇ ἃ κατέλιπεν ἐξ ὧν ἐνεπλήσθη.
\vs{19}Καὶ εἶπεν αὐτῇ ἡ πενθερὰ αὐτῆς, ποῦ συνέλεξας σήμερον καὶ ποῦ ἐποίησας; εἴη ὁ ἐπιγνούς σε εὐλογημένος· καὶ ἀνήγγειλε Ῥοὺθ τῇ πενθερᾷ αὐτῆς ποῦ ἐποίησε, καὶ εἶπε, τὸ ὄνομα τοῦ ἀνδρὸς μεθʼ οὗ ἐποίησα σήμερον Βοόζ.
\vs{20}Εἶπε δὲ Νωεμὶν τῇ νύμφῃ αὐτῆς, εὐλογητός ἐστι τῷ Κυρίῳ, ὅτι οὐκ ἐγκατέλιπε τὸ ἔλεος αὐτοῦ μετὰ τῶν ζώντων καὶ μετὰ τῶν τεθνηκότων· καὶ εἶπεν αὐτῇ Νωεμὶν, ἐγγίζει ἡμῖν ὁ ἀνὴρ, ἐκ τῶν ἀγχιστευόντων ἡμῖν ἐστι.
\vs{21}Καὶ εἶπε Ῥοὺθ πρὸς τὴν πενθερὰν αὐτῆς, καί γε ὅτι εἶπε πρὸς μέ, μετὰ τῶν κορασίων τῶν ἐμῶν προσκολλήθητι, ἕως ἂν τελέσωσιν ὅλον τὸν ἀμητὸν ὃς ὑπάρχει μοι.

\vs{22}Καὶ εἶπε Νωεμὶν πρὸς Ῥοὺθ τὴν νύμφην αὐτῆς, ἀγαθὸν θύγατερ, ὅτι ἐξῆλθες μετὰ τῶν κορασίων αὐτου, καὶ οὐκ ἀπαντήσονταί σοι ἐν ἀγρῷ ἑτέρῳ.
\vs{23}Καὶ προσεκολλήθη Ῥοὺθ τοῖς κορασίοις τοῦ Βοὸζ τοῦ συλλέγειν, ἕως τοῦ συντελέσαι τὸν θερισμὸν τῶν κριθῶν καὶ τῶν πυρῶν.

\ch{3}
Καὶ ἐκάθισε μετὰ τῆς πενθερᾶς αὐτῆς· εἶπε δὲ αὐτῇ Νωεμὶν ἡ πενθερὰ αὐτῆς, θύγατερ, οὐ μὴ ζητήσω σοι ἀνάπαυσιν, ἵνα εὖ γένηταί σοι;
\vs{2}Καὶ νῦν οὐχὶ Βοὸζ γνώριμος ἡμῶν, οὗ ἦς μετὰ τῶν κορασίων αὐτοῦ; ἰδοὺ αὐτὸς λικμᾷ τὸν ἅλωνα τῶν κριθῶν ταύτῃ τῇ νυκτί.

\vs{3}Σὺ δὲ λούσῃ, καὶ ἀλείψῃ, καὶ περιθήσεις τὸν ἱματισμόν σου ἐπὶ σὲ, καὶ ἀναβήσῃ ἐπὶ τὸν ἅλω· μὴ γνωρισθῇς τῷ ἀνδρὶ ἕως τοῦ συντελέσαι αὐτὸν τοῦ φαγεῖν καὶ πιεῖν.
\vs{4}Καὶ ἔσται ἐν τῷ κοιμηθῆναι αὐτὸν, καὶ γνώσῃ τὸν τόπον ὅπου κοιμᾶται ἐκεῖ, καὶ ἐλεύσῃ καὶ ἀποκαλύψεις τὰ πρὸς ποδῶν αὐτοῦ, καὶ κοιμηθήσῃ, καὶ αὐτὸς ἀπαγγελεῖ σοι ἃ ποιήσεις.
\vs{5}Εἶπε δὲ Ῥοὺθ πρὸς αὐτὴν, πάντα ὅσα ἂν εἴπῃς, ποιήσω.

\vs{6}Καὶ κατέβη εἰς τὸν ἅλω, καὶ ἐποίησε κατὰ πάντα, ὅσα ἐνετείλατο αὐτῇ ἡ πενθερὰ αὐτῆς.
\vs{7}Καὶ ἔφαγε Βοὸζ καὶ ἔπιε, καὶ ἠγαθύνθη ἡ καρδία αὐτοῦ, καὶ ἦλθε κοιμηθῆναι ἐν μερίδι τῆς στοιβῆς· ἡ δὲ ἦλθεν ἐν κρυφῇ, καὶ ἀπεκάλυψε τὰ πρὸς ποδῶν αὐτοῦ.
\vs{8}Ἐγένετο δὲ ἐν τῷ μεσονυκτίῳ, καὶ ἐξέστη ὁ ἀνὴρ, καὶ ἐταράχθη, καὶ ἰδοὺ γυνὴ κοιμᾶται πρὸς ποδῶν αὐτοῦ.
\vs{9}Εἶπε δὲ, τίς εἶ σύ; ἡ δὲ εἶπεν, ἐγὼ εἰμι Ῥοὺθ ἡ δούλη σου, καὶ περιβαλεῖς τὸ πτερύγιόν σου ἐπὶ τὴν δούλην σου, ὅτι ἀγχιστεὺς εἶ σύ.
\vs{10}Καὶ εἶπε Βοὸζ, εὐλογημένη σὺ τῷ Κυρίῳ Θεῷ, θύγατερ, ὅτι ἠγάθυνας τὸ ἔλεός σου τὸ ἔσχατον ὑπὲρ τὸ πρῶτον, μὴ πορευθῆναί σε ὀπίσω νεανιῶν, εἴτοι πτωχὸς εἴτοι πλούσιος.
\vs{11}Καὶ νῦν θύγατερ μὴ φοβοῦ, πάντα ὅσα ἐὰν εἴπῃς ποιήσω σοι· οἶδε γὰρ πᾶσα φυλὴ λαοῦ μου ὅτι γυνὴ δυνάμεως εἶ σύ.
\vs{12}Καὶ νῦν ὁ ἀληθῶς ἀγχιστεὺς ἐγώ εἰμι· καί γε ἐστὶν ἀγχιστεὺς ἐγγίων ὑπὲρ ἐμέ.
\vs{13}Αὐλίσθητι τὴν νύκτα, καὶ ἔσται τοπρωῒ ἐὰν ἀγχιστεύσῃ σε, ἀγαθόν· ἀγχιστευέτω· ἐὰν δὲ μὴ βούληται ἀγχιστεῦσαί σε, ἀγχιστεύσω σε ἐγώ· ζῇ Κύριος· κοιμήθητι ἕως τοπρωΐ.

\vs{14}Καὶ ἐκοιμήθη πρὸς ποδῶν αὐτοῦ ἕως πρωΐ· ἡ δὲ ἀνέστη πρὸ τοῦ ἐπιγνῶναι ἄνδρα τὸν πλησίον αὐτοῦ· καὶ εἶπε Βοὸζ, μὴ γνωσθήτω, ὅτι ἦλθε γυνὴ εἰς τὸν ἅλω.

\vs{15}Καὶ εἶπεν αὐτῇ, Φέρε τὸ περίζωμα τὸ ἐπάνω σου· καὶ ἐκράτησεν αὐτὸ, καὶ ἐμέτρησεν ἓξ κριθῶν, καὶ ἐπέθηκεν ἐπʼ αὐτὴν, καὶ εἰσῆλθεν εἰς τὴν πόλιν.

\vs{16}Καὶ Ῥοὺθ εἰσῆλθε πρὸς τὴν πενθερὰ αὐτῆς· ἡ δὲ εἶπεν αὐτῇ, θύγατερ· καὶ εἶπεν αὐτῇ πάντα ὅσα ἐποίησεν αὐτῇ ὁ ἀνήρ.
\vs{17}Καὶ εἶπεν αὐτῇ, τὰ ἕξ τῶν κριθῶν ταῦτα ἔδωκέ μοι, ὅτι εἶπε πρὸς μὲ, μὴ εἰσέλθῃς κενὴ πρὸς τὴν πενθεράν σου.
\vs{18}Ἡ δὲ εἶπε, κάθου θύγατερ, ἕως τοῦ ἐπιγνῶναί σε πῶς οὐ πεσεῖται ῥῆμα· οὐ γὰρ μὴ ἡσυχάσῃ ὁ ἀνὴρ ἕως ἂν τελεσθῇ τὸ ῥῆμα σήμερον.

\ch{4}
Καὶ Βοὸζ ἀνέβη ἐπὶ τὴν πύλην, και ἐκαθισεν ἐκεῖ, καὶ ἰδοὺ ὁ ἀγχιστεὺς παρεπορεύετο, ὃν ἐλάλησε Βοόζ· καὶ εἶπε πρὸς αὐτὸν Βοὸζ, ἐκκλίνας κάθισον ὧδε κρύφιε· καὶ ἐξέκλινε καὶ ἐκάθισε.
\vs{2}Καὶ ἔλαβε Βοὸζ δέκα ἄνδρας ἀπὸ τῶν πρεσβυτέρων τῆς πόλεως, καὶ εἶπε, καθίσατε ὧδε· καὶ ἐκάθισαν.

\vs{3}Καὶ εἶπε Βοὸζ τῷ ἀγχιστεῖ, τὴν μερίδα τοῦ ἀγροῦ ἥ ἐστι τοῦ ἀδελφοῦ ἡμῶν τοῦ Ἐλιμέλεχ, ἣ δέδοται Νωεμὶν τῇ ἐπιστρεφούσῃ ἐξ ἀγροῦ Μωὰβ,
\vs{4}κᾀγὼ εἶπα, ἀποκαλύψω τὸ οὖς σου λέγων, κτῆσαι ἐναντίον τῶν καθημένων, καὶ ἐναντίον τῶν πρεσβυτέρων τοῦ λαοῦ μου. εἰ ἀγχιστεύεις, ἀγχίστευε· εἰ δὲ μὴ ἀγχιστεύεις, ἀνάγγειλόν μοι, καὶ γνώσομαι, ὅτι οὐκ ἔστι πάρεξ σοῦ τοῦ ἀγχιστεῦσαι, κᾀγώ εἰμι μετὰ σέ· ὁ δὲ εἶπεν, ἐγώ εἰμι, ἀγχιστεῦσαι, κἀγώ εἰμι, ἀγχιστεύσω.
\vs{5}Καὶ εἶπε Βοὸζ, ἐν ἡμέρᾳ τοῦ κτήσασθαί σε τὸν ἀγρὸν ἐκ χειρὸς Νωεμὶν καὶ παρὰ Ῥοὺθ τῆς Μωαβίτιδος γυναικὸς τοῦ τεθνηκότος, καὶ αὐτὴν κτήσασθαί σε δεῖ, ὥστε ἀναστῆσαι τὸ ὄνομα τοῦ τεθνηκότος ἐπὶ τῆς κληρονομίας αὐτοῦ.
\vs{6}Καὶ εἶπεν ὁ ἀγχιστεὺς, οὐ δυνήσομαι ἀγχιστεῦσαι ἐμαυτῷ, μή ποτε διαφθείρω τὴν κληρονομίαν μου· ἀγχίστευσον σεαυτῷ τὴν ἀγχιστείαν μου, ὅτι οὐ δυνήσομαι ἀγχιστεῦσαι.

\vs{7}Καὶ τοῦτο τὸ δικαίωμα ἔμπροσθεν ἐν τῷ Ἰσραὴλ ἐπὶ τὴν ἀγχιστείαν, καὶ ἐπὶ τὸ ἀντάλλαγμα τοῦ στῆσαι πάντα λόγον· καὶ ὑπελύετο ἀνὴρ τὸ ὑπόδημα αὐτοῦ, καὶ ἐδίδου τῷ πλησίον αὐτοῦ τῷ ἀγχιστεύοντι τὴν ἀγχιστείαν αὐτοῦ· καὶ τοῦτο ἦν μαρτύριον ἐν Ἰσραήλ.
\vs{8}Καὶ εἶπεν ὁ ἀγχιστεὺς τῷ Βοὸζ, κτῆσαι σεαυτῷ τὴν ἀγχιστείαν μου· καὶ ὑπελύσατο τὸ ὑπόδημα αὐτοῦ, καὶ ἔδωκεν αὐτῷ.

\vs{9}Καὶ εἶπε Βοὸζ τοῖς πρεσβυτέροις καὶ παντὶ τῷ λαῷ, μάρτυρες ὑμεῖς σήμερον, ὅτι κέκτημαι πάντα τὰ τοῦ Ἐλιμέλεχ, καὶ πάντα ὅσα ὑπάρχει τῷ Χελαιὼν καὶ τῷ Μααλὼν ἐκ χειρὸς Νωεμίν.
\vs{10}Καί γε Ῥοὺθ τὴν Μωαβίτιν τὴν γυναῖκα Μααλὼν κέκτημαι ἐμαυτῷ εἰς γυναῖκα, τοῦ ἀναστῆσαι τὸ ὄνομα τοῦ τεθνηκότος ἐπὶ τῆς κληρονομίας αὐτοῦ, καὶ οὐκ ἐξολοθρευθήσεται τὸ ὄνομα τοῦ τεθνηκότος ἐκ τῶν ἀδελφῶν αὐτοῦ, καὶ ἐκ τῆς φυλῆς λαοῦ αὐτοῦ· μάρτυρες ὑμεῖς σήμερον.

\vs{11}Καὶ εἴποσαν πᾶς ὁ λαὸς οἱ ἐν τῇ πύλῃ, μάρτυρες· καὶ οἱ πρεσβύτεροι εἴποσαν, δῴη Κύριος τὴν γυναῖκά σου, τὴν εἰσπορευομένην εἰς τὸν οἶκόν σου, ὡς Ῥαχὴλ καὶ ὡς Λίαν, αἳ ᾠκοδόμησαν ἀμφότεραι τὸν οἶκον τοῦ Ἰσραὴλ, καὶ ἐποίησαν δύναμιν ἐν Ἐφραθᾷ, καὶ ἔσται ὄνομα ἐν Βηθλεέμ.
\vs{12}Καὶ γένοιτο οἶκός σου, ὡς οἶκος Φαρὲς, ὃν ἔτεκε Θάμαρ τῷ Ἰούδᾳ, ἐκ τοῦ σπέρματος οὗ δώσει Κύριός σοι ἐκ τῆς παιδίσκης ταύτης.

\vs{13}Καὶ ἔλαβε Βοὸζ τὴν Ῥοὺθ, καὶ ἐγενήθη αὐτῷ εἰς γυναῖκα, καὶ εἰσῆλθε πρὸς αὐτήν. καὶ ἔδωκεν αὐτῇ Κύριος κύησιν, καὶ ἔτεκεν υἱόν.
\vs{14}Καὶ εἶπαν αἱ γυναῖκες πρὸς Νωεμὶν, εὐλογητὸς Κύριος, ὃς οὐ κατέλυσέ σοι σήμερον τὸν ἀγχιστέα, καὶ καλέσαι τὸ ὄνομά σου ἑν Ἰσραήλ.
\vs{15}Καὶ ἔσται σοι εἰς ἐπιστρέφοντα ψυχὴν, καὶ τοῦ διαθρέψαι τὴν πολιάν σου, ὅτι ἡ νύμφη ἡ ἀγαπήσασά σε, ἔτεκεν αὐτὸν, ἥ ἐστιν ἀγαθή σοι ὑπὲρ ἑπτὰ υἱούς.
\vs{16}Καὶ ἔλαβε Νωεμὶν τὸ παιδίον, καὶ ἔθηκεν εἰς τὸν κόλπον αὐτῆς, καὶ ἐγενήθη αὐτῷ εἰς τιθηνόν.

\vs{17}Καὶ ἐκάλεσαν αὐτοῦ αἱ γείτονες ὄνομα, λέγουσαι, ἐτέχθη υἱὸς τῇ Νωεμίν. καὶ ἐκάλεσαν τὸ ὄνομα αὐτοῦ, Ὠβήδ· οὗτος πατὴρ Ἰεσσαὶ πατρὸς Δαυίδ.
\vs{18}Καὶ αὗται αἱ γενέσεις Φαρές. Φαρές ἐγέννησε τὸν Ἐσρώμ·
\vs{19}Ἐσρὼμ ἐγέννησε τὸν Ἀράμ· καὶ Ἀρὰμ ἐγέννησε τὸν Ἀμιναδάβ·
\vs{20}Καὶ Ἀμιναδὰβ ἐγέννησε τὸν Ναασσών· καὶ Ναασσὼν ἐγέννησε τὸν Σαλμών·
\vs{21}Καὶ Σαλμὼν ἐγέννησε τὸν Βοόζ· καὶ Βοὸζ ἐγέννησε τὸν Ὠβήδ·
\vs{22}Καὶ Ὡβὴδ ἐγέννησε τὸν Ἰεσσαί· καὶ Ἰεσσαὶ ἐγέννησε τὸν Δαυίδ.


\end{spacing}
\end{document}